\newcommand{\xchapter}[2]{
  \setcounter{chapter}{#1}
  \setcounter{section}{0}
  \chapter*{#2}
  \chaptermark{#2}
  \addcontentsline{toc}{chapter}{#1. #2}
}

\newcommand{\xchapterEnd}[1]{
  \begin{center}
    \chapter*{#1}
  \end{center}
  \vspace{7mm}
}

\newcommand{\xsectionEnd}[1]{
  \begin{center}
    \section*{#1}
  \end{center}
  \vspace{7mm}
}

\newcommand{\xsubsectionEnd}[1]{
  \begin{center}
    \subsection*{#1}
  \end{center}
  \vspace{7mm}
}

\newcommand{\xsubsubsectionEnd}[1]{
  \begin{center}
    \subsubsection{#1}
  \end{center}
  \vspace{7mm}
}

\newcommand{\xxsubsubsectionEnd}[1]{
  \begin{center}
    \subsubsection[x]{#1}
  \end{center}
  \vspace{7mm}
}

%%%%%%%%%%%%%%%%%%%%%%%%%%%%%%%%%%%%%%%%%%%%%%%%%%%%%%%

\documentclass[b5paper,11pt,twoside,draft]{book}
\usepackage{fontspec}
\usepackage[a5paper]{geometry}
\usepackage{polyglossia}
\setdefaultlanguage{pali}

%\usepackage{fancyhdr}
\usepackage{hyphenat} %hyphenation at - : \hyp{}
%Hyperlinks:
\usepackage{hyperref}
\hypersetup{pdftex,colorlinks=true,allcolors=blue,pdfpagemode=UseOutlines}
\usepackage{hypcap}

\overfullrule=1mm

%\pretolerance=1500
%\setlength{\emergencystretch}{2mm}
%\setlength{\parskip}{0pt}
%%%%%%%%%%%%%%%

\title{Tipiṭaka \\ \vspace{2 mm} \large Suttapiṭaka}
\begin{document}
\maketitle

\thispagestyle{empty}
\cleardoublepage


\vspace*{\fill}
\textit{Namo tassa bhagavato arahato sammāsambuddhassa}

\textit{Namo tassa bhagavato arahato sammāsambuddhassa}

\textit{Namo tassa bhagavato arahato sammāsambuddhassa}
\vspace*{\fill}

\tableofcontents

%Remove text number from Part pages:
\makeatletter
\renewcommand\part{
  \if@openright
  \cleardoublepage
  \else
  \clearpage
  \fi
  \thispagestyle{empty}
  \if@twocolumn
  \onecolumn
  \@tempswatrue
  \else
  \@tempswafalse
  \fi
  \null\vfil
  \secdef\@part\@spart}
\makeatother

\part*{Dīghanikāyo}

%\pagestyle{myheadings}

\xchapter{1}{Sīlakkhandhavaggapāḷi}

\section{Brahmajālasuttaṃ}

\subsubsection{Paribbājakakathā}

\paragraph{1.}
Evaṃ me sutaṃ – ekaṃ samayaṃ bhagavā antarā ca rājagahaṃ antarā ca nāḷandaṃ addhānamaggappaṭipanno hoti mahatā bhikkhusaṅghena saddhiṃ pañcamattehi bhikkhusatehi. Suppiyopi kho paribbājako antarā ca rājagahaṃ antarā ca nāḷandaṃ addhānamaggappaṭipanno hoti saddhiṃ antevāsinā brahmadattena māṇavena. Tatra sudaṃ suppiyo paribbājako anekapariyāyena buddhassa avaṇṇaṃ bhāsati, dhammassa avaṇṇaṃ bhāsati, saṅghassa avaṇṇaṃ bhāsati; suppiyassa pana paribbājakassa antevāsī brahmadatto māṇavo anekapariyāyena buddhassa vaṇṇaṃ bhāsati, dhammassa vaṇṇaṃ bhāsati, saṅghassa vaṇṇaṃ bhāsati. Itiha te ubho ācariyantevāsī aññamaññassa ujuvipaccanīkavādā bhagavantaṃ piṭṭhito piṭṭhito anubandhā\footnote{anubaddhā (ka. sī. pī.)} honti bhikkhusaṅghañca.

\paragraph{2.}
Atha kho bhagavā ambalaṭṭhikāyaṃ rājāgārake ekarattivāsaṃ upagacchi\footnote{upagañchi (sī. syā. kaṃ. pī.)} saddhiṃ bhikkhusaṅghena. Suppiyopi kho paribbājako ambalaṭṭhikāyaṃ rājāgārake ekarattivāsaṃ upagacchi\footnote{upagañchi (sī. syā. kaṃ. pī.)} antevāsinā brahmadattena māṇavena. Tatrapi sudaṃ suppiyo paribbājako anekapariyāyena buddhassa avaṇṇaṃ bhāsati, dhammassa avaṇṇaṃ bhāsati, saṅghassa avaṇṇaṃ bhāsati; suppiyassa pana paribbājakassa antevāsī brahmadatto māṇavo anekapariyāyena buddhassa vaṇṇaṃ bhāsati, dhammassa vaṇṇaṃ bhāsati, saṅghassa vaṇṇaṃ bhāsati. Itiha te ubho ācariyantevāsī aññamaññassa ujuvipaccanīkavādā viharanti.

\paragraph{3.}
Atha kho sambahulānaṃ bhikkhūnaṃ rattiyā paccūsasamayaṃ paccuṭṭhitānaṃ maṇḍalamāḷe sannisinnānaṃ sannipatitānaṃ ayaṃ saṅkhiyadhammo udapādi – ``acchariyaṃ, āvuso, abbhutaṃ, āvuso, yāvañcidaṃ tena bhagavatā jānatā passatā arahatā sammāsambuddhena sattānaṃ nānādhimuttikatā suppaṭividitā. Ayañhi suppiyo paribbājako anekapariyāyena buddhassa avaṇṇaṃ bhāsati, dhammassa avaṇṇaṃ bhāsati, saṅghassa avaṇṇaṃ bhāsati; suppiyassa pana paribbājakassa antevāsī brahmadatto māṇavo anekapariyāyena buddhassa vaṇṇaṃ bhāsati, dhammassa vaṇṇaṃ bhāsati, saṅghassa vaṇṇaṃ bhāsati. Itihame ubho ācariyantevāsī aññamaññassa ujuvipaccanīkavādā bhagavantaṃ piṭṭhito piṭṭhito anubandhā honti bhikkhusaṅghañcā'' ti.

\paragraph{4.}
Atha kho bhagavā tesaṃ bhikkhūnaṃ imaṃ saṅkhiyadhammaṃ viditvā yena maṇḍalamāḷo tenupasaṅkami; upasaṅkamitvā paññatte āsane nisīdi. Nisajja kho bhagavā bhikkhū āmantesi – ‘‘kāyanuttha, bhikkhave, etarahi kathāya sannisinnā sannipatitā, kā ca pana vo antarākathā vippakatā’’ti? Evaṃ vutte te bhikkhū bhagavantaṃ etadavocuṃ – ‘‘idha, bhante, amhākaṃ rattiyā paccūsasamayaṃ paccuṭṭhitānaṃ maṇḍalamāḷe sannisinnānaṃ sannipatitānaṃ ayaṃ saṅkhiyadhammo udapādi – ‘acchariyaṃ, āvuso, abbhutaṃ, āvuso, yāvañcidaṃ tena bhagavatā jānatā passatā arahatā sammāsambuddhena sattānaṃ nānādhimuttikatā suppaṭividitā. Ayañhi suppiyo paribbājako anekapariyāyena buddhassa avaṇṇaṃ bhāsati, dhammassa avaṇṇaṃ bhāsati, saṅghassa avaṇṇaṃ bhāsati; suppiyassa pana paribbājakassa antevāsī brahmadatto māṇavo anekapariyāyena buddhassa vaṇṇaṃ bhāsati, dhammassa vaṇṇaṃ bhāsati, saṅghassa vaṇṇaṃ bhāsati. Itihame ubho ācariyantevāsī aññamaññassa ujuvipaccanīkavādā bhagavantaṃ piṭṭhito piṭṭhito anubandhā honti bhikkhusaṅghañcā’ti. Ayaṃ kho no, bhante, antarākathā vippakatā, atha bhagavā anuppatto’’ti.

\paragraph{5.}
‘‘Mamaṃ vā, bhikkhave, pare avaṇṇaṃ bhāseyyuṃ, dhammassa vā avaṇṇaṃ bhāseyyuṃ, saṅghassa vā avaṇṇaṃ bhāseyyuṃ, tatra tumhehi na āghāto na appaccayo na cetaso anabhiraddhi karaṇīyā. Mamaṃ vā, bhikkhave, pare avaṇṇaṃ bhāseyyuṃ, dhammassa vā avaṇṇaṃ bhāseyyuṃ, saṅghassa vā avaṇṇaṃ bhāseyyuṃ, tatra ce tumhe assatha kupitā vā anattamanā vā, tumhaṃ yevassa tena antarāyo. Mamaṃ vā, bhikkhave, pare avaṇṇaṃ bhāseyyuṃ, dhammassa vā avaṇṇaṃ bhāseyyuṃ, saṅghassa vā avaṇṇaṃ bhāseyyuṃ, tatra ce tumhe assatha kupitā vā anattamanā vā, api nu tumhe paresaṃ subhāsitaṃ dubbhāsitaṃ ājāneyyāthā’’ti? ‘‘No hetaṃ, bhante’’. ‘‘Mamaṃ vā, bhikkhave, pare avaṇṇaṃ bhāseyyuṃ, dhammassa vā avaṇṇaṃ bhāseyyuṃ, saṅghassa vā avaṇṇaṃ bhāseyyuṃ, tatra tumhehi abhūtaṃ abhūtato nibbeṭhetabbaṃ – ‘itipetaṃ abhūtaṃ, itipetaṃ atacchaṃ, natthi cetaṃ amhesu, na ca panetaṃ amhesu saṃvijjatī’ti.

\paragraph{6.}
‘‘Mamaṃ vā, bhikkhave, pare vaṇṇaṃ bhāseyyuṃ, dhammassa vā vaṇṇaṃ bhāseyyuṃ, saṅghassa vā vaṇṇaṃ bhāseyyuṃ, tatra tumhehi na ānando na somanassaṃ na cetaso uppilāvitattaṃ karaṇīyaṃ. Mamaṃ vā, bhikkhave, pare vaṇṇaṃ bhāseyyuṃ, dhammassa vā vaṇṇaṃ bhāseyyuṃ, saṅghassa vā vaṇṇaṃ bhāseyyuṃ, tatra ce tumhe assatha ānandino sumanā uppilāvitā tumhaṃ yevassa tena antarāyo. Mamaṃ vā, bhikkhave, pare vaṇṇaṃ bhāseyyuṃ, dhammassa vā vaṇṇaṃ bhāseyyuṃ, saṅghassa vā vaṇṇaṃ bhāseyyuṃ, tatra tumhehi bhūtaṃ bhūtato paṭijānitabbaṃ – ‘itipetaṃ bhūtaṃ, itipetaṃ tacchaṃ, atthi cetaṃ amhesu, saṃvijjati ca panetaṃ amhesū’ti.

\subsubsection{Cūḷasīlaṃ}

\paragraph{7.}
‘‘Appamattakaṃ kho panetaṃ, bhikkhave, oramattakaṃ sīlamattakaṃ, yena puthujjano tathāgatassa vaṇṇaṃ vadamāno vadeyya. Katamañca taṃ, bhikkhave, appamattakaṃ oramattakaṃ sīlamattakaṃ, yena puthujjano tathāgatassa vaṇṇaṃ vadamāno vadeyya?

\paragraph{8.}
‘‘‘Pāṇātipātaṃ pahāya pāṇātipātā paṭivirato samaṇo gotamo nihitadaṇḍo, nihitasattho, lajjī, dayāpanno, sabbapāṇabhūtahitānukampī viharatī’ti – iti vā hi, bhikkhave, puthujjano tathāgatassa vaṇṇaṃ vadamāno vadeyya. ‘‘‘Adinnādānaṃ pahāya adinnādānā paṭivirato samaṇo gotamo dinnādāyī dinnapāṭikaṅkhī, athenena sucibhūtena attanā viharatī’ti – iti vā hi, bhikkhave, puthujjano tathāgatassa vaṇṇaṃ vadamāno vadeyya. ‘‘‘Abrahmacariyaṃ pahāya brahmacārī samaṇo gotamo ārācārī\footnote{anācārī (ka.)} virato\footnote{paṭivirato (katthaci)} methunā gāmadhammā’ti – iti vā hi, bhikkhave, puthujjano tathāgatassa vaṇṇaṃ vadamāno vadeyya.

\paragraph{9.}
‘‘‘Musāvādaṃ pahāya musāvādā paṭivirato samaṇo gotamo saccavādī saccasandho theto\footnote{ṭheto (syā. kaṃ.)} paccayiko avisaṃvādako lokassā’ti – iti vā hi, bhikkhave, puthujjano tathāgatassa vaṇṇaṃ vadamāno vadeyya. ‘‘‘Pisuṇaṃ vācaṃ pahāya pisuṇāya vācāya paṭivirato samaṇo gotamo, ito sutvā na amutra akkhātā imesaṃ bhedāya, amutra vā sutvā na imesaṃ akkhātā amūsaṃ bhedāya. Iti bhinnānaṃ vā sandhātā, sahitānaṃ vā anuppadātā samaggārāmo samaggarato samagganandī samaggakaraṇiṃ vācaṃ bhāsitā’ti – iti vā hi, bhikkhave, puthujjano tathāgatassa vaṇṇaṃ vadamāno vadeyya. ‘‘‘Pharusaṃ vācaṃ pahāya pharusāya vācāya paṭivirato samaṇo gotamo, yā sā vācā nelā kaṇṇasukhā pemanīyā hadayaṅgamā porī bahujanakantā bahujanamanāpā tathārūpiṃ vācaṃ bhāsitā’ti – iti vā hi, bhikkhave, puthujjano tathāgatassa vaṇṇaṃ vadamāno vadeyya. ‘‘‘Samphappalāpaṃ pahāya samphappalāpā paṭivirato samaṇo gotamo kālavādī bhūtavādī atthavādī dhammavādī vinayavādī, nidhānavatiṃ vācaṃ bhāsitā kālena sāpadesaṃ pariyantavatiṃ atthasaṃhita’nti – iti vā hi, bhikkhave, puthujjano tathāgatassa vaṇṇaṃ vadamāno vadeyya.

\paragraph{10.}
‘Bījagāmabhūtagāmasamārambhā\footnote{samārabbhā (sī. ka.)} paṭivirato samaṇo gotamo’ti – iti vā hi, bhikkhave …pe…. ‘‘‘Ekabhattiko samaṇo gotamo rattūparato virato\footnote{paṭivirato (katthaci)} vikālabhojanā…. Naccagītavāditavisūkadassanā\footnote{naccagītavāditavisukadassanā (ka.)} paṭivirato samaṇo gotamo…. Mālāgandhavilepanadhāraṇamaṇḍanavibhūsanaṭṭhānā paṭivirato samaṇo gotamo…. Uccāsayanamahāsayanā paṭivirato samaṇo gotamo…. Jātarūparajatapaṭiggahaṇā paṭivirato samaṇo gotamo…. Āmakadhaññapaṭiggahaṇā paṭivirato samaṇo gotamo…. Āmakamaṃsapaṭiggahaṇā paṭivirato samaṇo gotamo…. Itthikumārikapaṭiggahaṇā paṭivirato samaṇo gotamo…. Dāsidāsapaṭiggahaṇā paṭivirato samaṇo gotamo…. Ajeḷakapaṭiggahaṇā paṭivirato samaṇo gotamo…. Kukkuṭasūkarapaṭiggahaṇā paṭivirato samaṇo gotamo…. Hatthigavassavaḷavapaṭiggahaṇā paṭivirato samaṇo gotamo…. Khettavatthupaṭiggahaṇā paṭivirato samaṇo gotamo…. Dūteyyapahiṇagamanānuyogā paṭivirato samaṇo gotamo…. Kayavikkayā paṭivirato samaṇo gotamo…. Tulākūṭakaṃsakūṭamānakūṭā paṭivirato samaṇo gotamo…. Ukkoṭanavañcananikatisāciyogā\footnote{sāviyogā (syā. kaṃ. ka.)} paṭivirato samaṇo gotamo…. Chedanavadhabandhanaviparāmosaālopasahasākārā paṭivirato samaṇo gotamo’ti – iti vā hi, bhikkhave, puthujjano tathāgatassa vaṇṇaṃ vadamāno vadeyya.

\xsubsubsectionEnd{Cūḷasīlaṃ niṭṭhitaṃ.}

\subsubsection{Majjhimasīlaṃ}

\paragraph{11.}
‘‘‘Yathā vā paneke bhonto samaṇabrāhmaṇā saddhādeyyāni bhojanāni bhuñjitvā te evarūpaṃ bījagāmabhūtagāmasamārambhaṃ anuyuttā viharanti, seyyathidaṃ\footnote{seyyathīdaṃ (sī. syā.)} – mūlabījaṃ khandhabījaṃ phaḷubījaṃ aggabījaṃ bījabījameva pañcamaṃ\footnote{pañcamaṃ iti vā (sī. syā. ka.)}; iti evarūpā bījagāmabhūtagāmasamārambhā paṭivirato samaṇo gotamo’ti – iti vā hi, bhikkhave, puthujjano tathāgatassa vaṇṇaṃ vadamāno vadeyya.

\paragraph{12.}
‘‘‘Yathā vā paneke bhonto samaṇabrāhmaṇā saddhādeyyāni bhojanāni bhuñjitvā te evarūpaṃ sannidhikāraparibhogaṃ anuyuttā viharanti, seyyathidaṃ – annasannidhiṃ pānasannidhiṃ vatthasannidhiṃ yānasannidhiṃ sayanasannidhiṃ gandhasannidhiṃ āmisasannidhiṃ iti vā iti evarūpā sannidhikāraparibhogā paṭivirato samaṇo gotamo’ti – iti vā hi, bhikkhave, puthujjano tathāgatassa vaṇṇaṃ vadamāno vadeyya.

\paragraph{13.}
 ‘‘‘Yathā vā paneke bhonto samaṇabrāhmaṇā saddhādeyyāni bhojanāni bhuñjitvā te evarūpaṃ visūkadassanaṃ anuyuttā viharanti, seyyathidaṃ – naccaṃ gītaṃ vāditaṃ pekkhaṃ akkhānaṃ pāṇissaraṃ vetāḷaṃ kumbhathūṇaṃ\footnote{kumbhathūnaṃ (syā. ka.), kumbhathūṇaṃ (sī.)} sobhanakaṃ\footnote{sobhanagharakaṃ (sī.), sobhanagarakaṃ (syā. kaṃ. pī.)} caṇḍālaṃ vaṃsaṃ dhovanaṃ hatthiyuddhaṃ assayuddhaṃ mahiṃsayuddhaṃ\footnote{mahisayuddhaṃ (sī. syā. kaṃ. pī.)} usabhayuddhaṃ ajayuddhaṃ meṇḍayuddhaṃ kukkuṭayuddhaṃ vaṭṭakayuddhaṃ daṇḍayuddhaṃ muṭṭhiyuddhaṃ nibbuddhaṃ uyyodhikaṃ balaggaṃ senābyūhaṃ anīkadassanaṃ iti vā iti evarūpā visūkadassanā paṭivirato samaṇo gotamo’ti – iti vā hi, bhikkhave, puthujjano tathāgatassa vaṇṇaṃ vadamāno vadeyya.

\paragraph{14.}
‘‘‘Yathā vā paneke bhonto samaṇabrāhmaṇā saddhādeyyāni bhojanāni bhuñjitvā te evarūpaṃ jūtappamādaṭṭhānānuyogaṃ anuyuttā viharanti, seyyathidaṃ – aṭṭhapadaṃ dasapadaṃ ākāsaṃ parihārapathaṃ santikaṃ khalikaṃ ghaṭikaṃ salākahatthaṃ akkhaṃ paṅgacīraṃ vaṅkakaṃ mokkhacikaṃ ciṅgulikaṃ\footnote{ciṅgulakaṃ (ka. sī.)} pattāḷhakaṃ rathakaṃ dhanukaṃ akkharikaṃ manesikaṃ yathāvajjaṃ iti vā iti evarūpā jūtappamādaṭṭhānānuyogā paṭivirato samaṇo gotamo’ti – iti vā hi, bhikkhave, puthujjano tathāgatassa vaṇṇaṃ vadamāno vadeyya.

\paragraph{15.}
‘‘‘Yathā vā paneke bhonto samaṇabrāhmaṇā saddhādeyyāni bhojanāni bhuñjitvā te evarūpaṃ uccāsayanamahāsayanaṃ anuyuttā viharanti, seyyathidaṃ – āsandiṃ pallaṅkaṃ gonakaṃ cittakaṃ paṭikaṃ paṭalikaṃ tūlikaṃ vikatikaṃ uddalomiṃ ekantalomiṃ kaṭṭissaṃ koseyyaṃ kuttakaṃ hatthattharaṃ assattharaṃ rathattharaṃ\footnote{hatthattharaṇaṃ assattharaṇaṃ rathattharaṇaṃ (sī. ka. pī.)} ajinappaveṇiṃ kadalimigapavarapaccattharaṇaṃ sauttaracchadaṃ ubhatolohitakūpadhānaṃ iti vā iti evarūpā uccāsayanamahāsayanā paṭivirato samaṇo gotamo’ti – iti vā hi, bhikkhave, puthujjano tathāgatassa vaṇṇaṃ vadamāno vadeyya.

\paragraph{16.}
 ‘‘‘Yathā vā paneke bhonto samaṇabrāhmaṇā saddhādeyyāni bhojanāni bhuñjitvā te evarūpaṃ maṇḍanavibhūsanaṭṭhānānuyogaṃ anuyuttā viharanti, seyyathidaṃ – ucchādanaṃ parimaddanaṃ nhāpanaṃ sambāhanaṃ ādāsaṃ añjanaṃ mālāgandhavilepanaṃ\footnote{mālāvilepanaṃ (sī. syā. kaṃ. pī.)} mukhacuṇṇaṃ mukhalepanaṃ hatthabandhaṃ sikhābandhaṃ daṇḍaṃ nāḷikaṃ asiṃ\footnote{khaggaṃ (sī. pī.), asiṃ khaggaṃ (syā. kaṃ.)} chattaṃ citrupāhanaṃ uṇhīsaṃ maṇiṃ vālabījaniṃ odātāni vatthāni dīghadasāni iti vā iti evarūpā maṇḍanavibhūsanaṭṭhānānuyogā paṭivirato samaṇo gotamo’ti – iti vā hi, bhikkhave, puthujjano tathāgatassa vaṇṇaṃ vadamāno vadeyya.

\paragraph{17.}
‘‘‘Yathā vā paneke bhonto samaṇabrāhmaṇā saddhādeyyāni bhojanāni bhuñjitvā te evarūpaṃ tiracchānakathaṃ anuyuttā viharanti, seyyathidaṃ – rājakathaṃ corakathaṃ mahāmattakathaṃ senākathaṃ bhayakathaṃ yuddhakathaṃ annakathaṃ pānakathaṃ vatthakathaṃ sayanakathaṃ mālākathaṃ gandhakathaṃ ñātikathaṃ yānakathaṃ gāmakathaṃ nigamakathaṃ nagarakathaṃ janapadakathaṃ itthikathaṃ\footnote{itthikathaṃ purisakathaṃ (syā. kaṃ. ka.)} sūrakathaṃ visikhākathaṃ kumbhaṭṭhānakathaṃ pubbapetakathaṃ nānattakathaṃ lokakkhāyikaṃ samuddakkhāyikaṃ itibhavābhavakathaṃ iti vā iti evarūpāya tiracchānakathāya paṭivirato samaṇo gotamo’ti – iti vā hi, bhikkhave, puthujjano tathāgatassa vaṇṇaṃ vadamāno vadeyya.

\paragraph{18.}
‘‘‘Yathā vā paneke bhonto samaṇabrāhmaṇā saddhādeyyāni bhojanāni bhuñjitvā te evarūpaṃ viggāhikakathaṃ anuyuttā viharanti, seyyathidaṃ – na tvaṃ imaṃ dhammavinayaṃ ājānāsi, ahaṃ imaṃ dhammavinayaṃ ājānāmi, kiṃ tvaṃ imaṃ dhammavinayaṃ ājānissasi, micchā paṭipanno tvamasi, ahamasmi sammā paṭipanno, sahitaṃ me, asahitaṃ te, purevacanīyaṃ pacchā avaca, pacchāvacanīyaṃ pure avaca, adhiciṇṇaṃ te viparāvattaṃ, āropito te vādo, niggahito tvamasi, cara vādappamokkhāya, nibbeṭhehi vā sace pahosīti iti vā iti evarūpāya viggāhikakathāya paṭivirato samaṇo gotamo’ti – iti vā hi, bhikkhave, puthujjano tathāgatassa vaṇṇaṃ vadamāno vadeyya.
\paragraph{19.}
‘‘‘Yathā vā paneke bhonto samaṇabrāhmaṇā saddhādeyyāni bhojanāni bhuñjitvā te evarūpaṃ dūteyyapahiṇagamanānuyogaṃ anuyuttā viharanti, seyyathidaṃ – raññaṃ, rājamahāmattānaṃ, khattiyānaṃ, brāhmaṇānaṃ, gahapatikānaṃ, kumārānaṃ ‘‘idha gaccha, amutrāgaccha, idaṃ hara, amutra idaṃ āharā’’ti iti vā iti evarūpā dūteyyapahiṇagamanānuyogā paṭivirato samaṇo gotamo’ti – iti vā hi, bhikkhave, puthujjano tathāgatassa vaṇṇaṃ vadamāno vadeyya.

\paragraph{20.}
‘‘‘Yathā vā paneke bhonto samaṇabrāhmaṇā saddhādeyyāni bhojanāni bhuñjitvā te kuhakā ca honti, lapakā ca nemittikā ca nippesikā ca, lābhena lābhaṃ nijigīṃsitāro ca\footnote{lābhena lābhaṃ nijigiṃ bhitāro (sī. syā.), lābhena ca lābhaṃ nijigīsitāro (pī.)} iti\footnote{iti vā, iti (syā. kaṃ. ka.)} evarūpā kuhanalapanā paṭivirato samaṇo gotamo’ti – iti vā hi, bhikkhave, puthujjano tathāgatassa vaṇṇaṃ vadamāno vadeyya.

\xsubsubsectionEnd{Majjhimasīlaṃ niṭṭhitaṃ.}

\subsubsection{Mahāsīlaṃ}

\paragraph{21.}
‘‘‘Yathā vā paneke bhonto samaṇabrāhmaṇā saddhādeyyāni bhojanāni bhuñjitvā te evarūpāya tiracchānavijjāya micchājīvena jīvitaṃ kappenti, seyyathidaṃ – aṅgaṃ nimittaṃ uppātaṃ supinaṃ lakkhaṇaṃ mūsikacchinnaṃ aggihomaṃ dabbihomaṃ thusahomaṃ kaṇahomaṃ taṇḍulahomaṃ sappihomaṃ telahomaṃ mukhahomaṃ lohitahomaṃ aṅgavijjā vatthuvijjā khattavijjā\footnote{khettavijjā (bahūsu)} sivavijjā bhūtavijjā bhūrivijjā ahivijjā visavijjā vicchikavijjā mūsikavijjā sakuṇavijjā vāyasavijjā pakkajjhānaṃ saraparittāṇaṃ migacakkaṃ iti vā iti evarūpāya tiracchānavijjāya micchājīvā paṭivirato samaṇo gotamo’ti – iti vā hi, bhikkhave, puthujjano tathāgatassa vaṇṇaṃ vadamāno vadeyya.

\paragraph{22.}
‘‘‘Yathā vā paneke bhonto samaṇabrāhmaṇā saddhādeyyāni bhojanāni bhuñjitvā te evarūpāya tiracchānavijjāya micchājīvena jīvitaṃ kappenti, seyyathidaṃ – maṇilakkhaṇaṃ vatthalakkhaṇaṃ daṇḍalakkhaṇaṃ satthalakkhaṇaṃ asilakkhaṇaṃ usulakkhaṇaṃ dhanulakkhaṇaṃ āvudhalakkhaṇaṃ itthilakkhaṇaṃ purisalakkhaṇaṃ kumāralakkhaṇaṃ kumārilakkhaṇaṃ dāsalakkhaṇaṃ dāsilakkhaṇaṃ hatthilakkhaṇaṃ assalakkhaṇaṃ mahiṃsalakkhaṇaṃ\footnote{mahisalakkhaṇaṃ (sī. syā. kaṃ. pī.)} usabhalakkhaṇaṃ golakkhaṇaṃ ajalakkhaṇaṃ meṇḍalakkhaṇaṃ kukkuṭalakkhaṇaṃ vaṭṭakalakkhaṇaṃ godhālakkhaṇaṃ kaṇṇikālakkhaṇaṃ kacchapalakkhaṇaṃ migalakkhaṇaṃ iti vā iti evarūpāya tiracchānavijjāya micchājīvā paṭivirato samaṇo gotamo’ti – iti vā hi, bhikkhave, puthujjano tathāgatassa vaṇṇaṃ vadamāno vadeyya.
\paragraph{23.}
‘‘‘Yathā vā paneke bhonto samaṇabrāhmaṇā saddhādeyyāni bhojanāni bhuñjitvā te evarūpāya tiracchānavijjāya micchājīvena jīvitaṃ kappenti, seyyathidaṃ – raññaṃ niyyānaṃ bhavissati, raññaṃ aniyyānaṃ bhavissati, abbhantarānaṃ raññaṃ upayānaṃ bhavissati, bāhirānaṃ raññaṃ apayānaṃ bhavissati, bāhirānaṃ raññaṃ upayānaṃ bhavissati, abbhantarānaṃ raññaṃ apayānaṃ bhavissati, abbhantarānaṃ raññaṃ jayo bhavissati, bāhirānaṃ raññaṃ parājayo bhavissati, bāhirānaṃ raññaṃ jayo bhavissati, abbhantarānaṃ raññaṃ parājayo bhavissati, iti imassa jayo bhavissati, imassa parājayo bhavissati iti vā iti evarūpāya tiracchānavijjāya micchājīvā paṭivirato samaṇo gotamo’ti – iti vā hi, bhikkhave, puthujjano tathāgatassa vaṇṇaṃ vadamāno vadeyya.

\paragraph{24.}
‘‘‘Yathā vā paneke bhonto samaṇabrāhmaṇā saddhādeyyāni bhojanāni bhuñjitvā te evarūpāya tiracchānavijjāya micchājīvena jīvitaṃ kappenti, seyyathidaṃ – candaggāho bhavissati, sūriyaggāho\footnote{suriyaggāho (sī. syā. kaṃ. pī.)} bhavissati, nakkhattaggāho bhavissati, candimasūriyānaṃ pathagamanaṃ bhavissati, candimasūriyānaṃ uppathagamanaṃ bhavissati, nakkhattānaṃ pathagamanaṃ bhavissati, nakkhattānaṃ uppathagamanaṃ bhavissati, ukkāpāto bhavissati, disāḍāho bhavissati, bhūmicālo bhavissati, devadudrabhi\footnote{devadundubhi (syā. kaṃ. pī.)} bhavissati, candimasūriyanakkhattānaṃ uggamanaṃ ogamanaṃ saṃkilesaṃ vodānaṃ bhavissati, evaṃvipāko candaggāho bhavissati, evaṃvipāko sūriyaggāho bhavissati, evaṃvipāko nakkhattaggāho bhavissati, evaṃvipākaṃ candimasūriyānaṃ pathagamanaṃ bhavissati, evaṃvipākaṃ candimasūriyānaṃ uppathagamanaṃ bhavissati, evaṃvipākaṃ nakkhattānaṃ pathagamanaṃ bhavissati, evaṃvipākaṃ nakkhattānaṃ uppathagamanaṃ bhavissati, evaṃvipāko ukkāpāto bhavissati, evaṃvipāko disāḍāho bhavissati, evaṃvipāko bhūmicālo bhavissati, evaṃvipāko devadudrabhi bhavissati, evaṃvipākaṃ candimasūriyanakkhattānaṃ uggamanaṃ ogamanaṃ saṃkilesaṃ vodānaṃ bhavissati iti vā iti evarūpāya tiracchānavijjāya micchājīvā paṭivirato samaṇo gotamo’ti – iti vā hi, bhikkhave, puthujjano tathāgatassa vaṇṇaṃ vadamāno vadeyya.

\paragraph{25.}
‘‘‘Yathā vā paneke bhonto samaṇabrāhmaṇā saddhādeyyāni bhojanāni bhuñjitvā te evarūpāya tiracchānavijjāya micchājīvena jīvitaṃ kappenti, seyyathidaṃ – suvuṭṭhikā bhavissati, dubbuṭṭhikā bhavissati, subhikkhaṃ bhavissati, dubbhikkhaṃ bhavissati, khemaṃ bhavissati, bhayaṃ bhavissati, rogo bhavissati, ārogyaṃ bhavissati, muddā, gaṇanā, saṅkhānaṃ, kāveyyaṃ, lokāyataṃ iti vā iti evarūpāya tiracchānavijjāya micchājīvā paṭivirato samaṇo gotamo’ti – iti vā hi, bhikkhave, puthujjano tathāgatassa vaṇṇaṃ vadamāno vadeyya.

\paragraph{26.}
‘‘‘Yathā vā paneke bhonto samaṇabrāhmaṇā saddhādeyyāni bhojanāni bhuñjitvā te evarūpāya tiracchānavijjāya micchājīvena jīvitaṃ kappenti, seyyathidaṃ – āvāhanaṃ vivāhanaṃ saṃvaraṇaṃ vivaraṇaṃ saṃkiraṇaṃ vikiraṇaṃ subhagakaraṇaṃ dubbhagakaraṇaṃ viruddhagabbhakaraṇaṃ jivhānibandhanaṃ hanusaṃhananaṃ hatthābhijappanaṃ hanujappanaṃ kaṇṇajappanaṃ ādāsapañhaṃ kumārikapañhaṃ devapañhaṃ ādiccupaṭṭhānaṃ mahatupaṭṭhānaṃ abbhujjalanaṃ sirivhāyanaṃ iti vā iti evarūpāya tiracchānavijjāya micchājīvā paṭivirato samaṇo gotamo’ti – iti vā hi, bhikkhave, puthujjano tathāgatassa vaṇṇaṃ vadamāno vadeyya.

\paragraph{27.}
‘‘‘Yathā vā paneke bhonto samaṇabrāhmaṇā saddhādeyyāni bhojanāni bhuñjitvā te evarūpāya tiracchānavijjāya micchājīvena jīvitaṃ kappenti, seyyathidaṃ – santikammaṃ paṇidhikammaṃ bhūtakammaṃ bhūrikammaṃ vassakammaṃ vossakammaṃ vatthukammaṃ vatthuparikammaṃ ācamanaṃ nhāpanaṃ juhanaṃ vamanaṃ virecanaṃ uddhaṃvirecanaṃ adhovirecanaṃ sīsavirecanaṃ kaṇṇatelaṃ nettatappanaṃ natthukammaṃ añjanaṃ paccañjanaṃ sālākiyaṃ sallakattiyaṃ dārakatikicchā mūlabhesajjānaṃ anuppadānaṃ osadhīnaṃ paṭimokkho iti vā iti evarūpāya tiracchānavijjāya micchājīvā paṭivirato samaṇo gotamo’ti – iti vā hi, bhikkhave, puthujjano tathāgatassa vaṇṇaṃ vadamāno vadeyya. ‘‘Idaṃ kho, bhikkhave, appamattakaṃ oramattakaṃ sīlamattakaṃ, yena puthujjano tathāgatassa vaṇṇaṃ vadamāno vadeyya.

\xsubsubsectionEnd{Mahāsīlaṃ niṭṭhitaṃ.}

\subsubsection{Pubbantakappikā}

\paragraph{28.}
‘‘Atthi, bhikkhave, aññeva dhammā gambhīrā duddasā duranubodhā santā paṇītā atakkāvacarā nipuṇā paṇḍitavedanīyā, ye tathāgato sayaṃ abhiññā sacchikatvā pavedeti, yehi tathāgatassa yathābhuccaṃ vaṇṇaṃ sammā vadamānā vadeyyuṃ. Katame ca te, bhikkhave, dhammā gambhīrā duddasā duranubodhā santā paṇītā atakkāvacarā nipuṇā paṇḍitavedanīyā, ye tathāgato sayaṃ abhiññā sacchikatvā pavedeti, yehi tathāgatassa yathābhuccaṃ vaṇṇaṃ sammā vadamānā vadeyyuṃ?

\paragraph{29.}
‘‘Santi, bhikkhave, eke samaṇabrāhmaṇā pubbantakappikā pubbantānudiṭṭhino, pubbantaṃ ārabbha anekavihitāni adhimuttipadāni\footnote{adhivuttipadāni (sī. pī.)} abhivadanti aṭṭhārasahi vatthūhi. Te ca bhonto samaṇabrāhmaṇā kimāgamma kimārabbha pubbantakappikā pubbantānudiṭṭhino pubbantaṃ ārabbha anekavihitāni adhimuttipadāni abhivadanti aṭṭhārasahi vatthūhi?

\subsubsection{Sassatavādo}

\paragraph{30.}
‘‘Santi, bhikkhave, eke samaṇabrāhmaṇā sassatavādā, sassataṃ attānañca lokañca paññapenti catūhi vatthūhi. Te ca bhonto samaṇabrāhmaṇā kimāgamma kimārabbha sassatavādā sassataṃ attānañca lokañca paññapenti catūhi vatthūhi?

\paragraph{31.}
‘‘Idha, bhikkhave, ekacco samaṇo vā brāhmaṇo vā ātappamanvāya padhānamanvāya anuyogamanvāya appamādamanvāya sammāmanasikāramanvāya tathārūpaṃ cetosamādhiṃ phusati, yathāsamāhite citte ( )\footnote{(parisuddhe pariyodāte anaṅgaṇe vigatūpattilese) (syā. ka.)} anekavihitaṃ pubbenivāsaṃ anussarati. Seyyathidaṃ – ekampi jātiṃ dvepi jātiyo tissopi jātiyo catassopi jātiyo pañcapi jātiyo dasapi jātiyo vīsampi jātiyo tiṃsampi jātiyo cattālīsampi jātiyo paññāsampi jātiyo jātisatampi jātisahassampi jātisatasahassampi anekānipi jātisatāni anekānipi jātisahassāni anekānipi jātisatasahassāni – ‘amutrāsiṃ evaṃnāmo evaṃgotto evaṃvaṇṇo evamāhāro evaṃsukhadukkhappaṭisaṃvedī evamāyupariyanto, so tato cuto amutra udapādiṃ; tatrāpāsiṃ evaṃnāmo evaṃgotto evaṃvaṇṇo evamāhāro evaṃsukhadukkhappaṭisaṃvedī evamāyupariyanto, so tato cuto idhūpapanno’ti. Iti sākāraṃ sauddesaṃ anekavihitaṃ pubbenivāsaṃ anussarati. ‘‘So evamāha – ‘sassato attā ca loko ca vañjho kūṭaṭṭho esikaṭṭhāyiṭṭhito; te ca sattā sandhāvanti saṃsaranti cavanti upapajjanti, atthitveva sassatisamaṃ. Taṃ kissa hetu? Ahañhi ātappamanvāya padhānamanvāya anuyogamanvāya appamādamanvāya sammāmanasikāramanvāya tathārūpaṃ cetosamādhiṃ phusāmi, yathāsamāhite citte anekavihitaṃ pubbenivāsaṃ anussarāmi seyyathidaṃ – ekampi jātiṃ dvepi jātiyo tissopi jātiyo catassopi jātiyo pañcapi jātiyo dasapi jātiyo vīsampi jātiyo tiṃsampi jātiyo cattālīsampi jātiyo paññāsampi jātiyo jātisatampi jātisahassampi jātisatasahassampi anekānipi jātisatāni anekānipi jātisahassāni anekānipi jātisatasahassāni – amutrāsiṃ evaṃnāmo evaṃgotto evaṃvaṇṇo evamāhāro evaṃsukhadukkhappaṭisaṃvedī evamāyupariyanto, so tato cuto amutra udapādiṃ; tatrāpāsiṃ evaṃnāmo evaṃgotto evaṃvaṇṇo evamāhāro evaṃsukhadukkhappaṭisaṃvedī evamāyupariyanto, so tato cuto idhūpapannoti. Iti sākāraṃ sauddesaṃ anekavihitaṃ pubbenivāsaṃ anussarāmi. Imināmahaṃ etaṃ jānāmi ‘‘yathā sassato attā ca loko ca vañjho kūṭaṭṭho esikaṭṭhāyiṭṭhito; te ca sattā sandhāvanti saṃsaranti cavanti upapajjanti, atthitveva sassatisama’’nti. Idaṃ, bhikkhave, paṭhamaṃ ṭhānaṃ, yaṃ āgamma yaṃ ārabbha eke samaṇabrāhmaṇā sassatavādā sassataṃ attānañca lokañca paññapenti.

\paragraph{32.}
‘‘Dutiye ca bhonto samaṇabrāhmaṇā kimāgamma kimārabbha sassatavādā sassataṃ attānañca lokañca paññapenti? Idha, bhikkhave, ekacco samaṇo vā brāhmaṇo vā ātappamanvāya padhānamanvāya anuyogamanvāya appamādamanvāya sammāmanasikāramanvāya tathārūpaṃ cetosamādhiṃ phusati, yathāsamāhite citte anekavihitaṃ pubbenivāsaṃ anussarati. Seyyathidaṃ – ekampi saṃvaṭṭavivaṭṭaṃ dvepi saṃvaṭṭavivaṭṭāni tīṇipi saṃvaṭṭavivaṭṭāni cattāripi saṃvaṭṭavivaṭṭāni pañcapi saṃvaṭṭavivaṭṭāni dasapi saṃvaṭṭavivaṭṭāni – ‘amutrāsiṃ evaṃnāmo evaṃgotto evaṃvaṇṇo evamāhāro evaṃsukhadukkhappaṭisaṃvedī evamāyupariyanto, so tato cuto amutra udapādiṃ; tatrāpāsiṃ evaṃnāmo evaṃgotto evaṃvaṇṇo evamāhāro evaṃsukhadukkhappaṭisaṃvedī evamāyupariyanto, so tato cuto idhūpapanno’ti. Iti sākāraṃ sauddesaṃ anekavihitaṃ pubbenivāsaṃ anussarati. ‘‘So evamāha – ‘sassato attā ca loko ca vañjho kūṭaṭṭho esikaṭṭhāyiṭṭhito; te ca sattā sandhāvanti saṃsaranti cavanti upapajjanti, atthitveva sassatisamaṃ. Taṃ kissa hetu? Ahañhi ātappamanvāya padhānamanvāya anuyogamanvāya appamādamanvāya sammāmanasikāramanvāya tathārūpaṃ cetosamādhiṃ phusāmi yathāsamāhite citte anekavihitaṃ pubbenivāsaṃ anussarāmi. Seyyathidaṃ – ekampi saṃvaṭṭavivaṭṭaṃ dvepi saṃvaṭṭavivaṭṭāni tīṇipi saṃvaṭṭavivaṭṭāni cattāripi saṃvaṭṭavivaṭṭāni pañcapi saṃvaṭṭavivaṭṭāni dasapi saṃvaṭṭavivaṭṭāni. Amutrāsiṃ evaṃnāmo evaṃgotto evaṃvaṇṇo evamāhāro evaṃsukhadukkhappaṭisaṃvedī evamāyupariyanto, so tato cuto amutra udapādiṃ; tatrāpāsiṃ evaṃnāmo evaṃgotto evaṃvaṇṇo evamāhāro evaṃsukhadukkhappaṭisaṃvedī evamāyupariyanto, so tato cuto idhūpapannoti. Iti sākāraṃ sauddesaṃ anekavihitaṃ pubbenivāsaṃ anussarāmi. Imināmahaṃ etaṃ jānāmi ‘‘yathā sassato attā ca loko ca vañjho kūṭaṭṭho esikaṭṭhāyiṭṭhito, te ca sattā sandhāvanti saṃsaranti cavanti upapajjanti, atthitveva sassatisama’’nti. Idaṃ, bhikkhave, dutiyaṃ ṭhānaṃ, yaṃ āgamma yaṃ ārabbha eke samaṇabrāhmaṇā sassatavādā sassataṃ attānañca lokañca paññapenti.

\paragraph{33.}
‘‘Tatiye ca bhonto samaṇabrāhmaṇā kimāgamma kimārabbha sassatavādā sassataṃ attānañca lokañca paññapenti? Idha, bhikkhave, ekacco samaṇo vā brāhmaṇo vā ātappamanvāya padhānamanvāya anuyogamanvāya appamādamanvāya sammāmanasikāramanvāya tathārūpaṃ cetosamādhiṃ phusati, yathāsamāhite citte anekavihitaṃ pubbenivāsaṃ anussarati. Seyyathidaṃ – dasapi saṃvaṭṭavivaṭṭāni vīsampi saṃvaṭṭavivaṭṭāni tiṃsampi saṃvaṭṭavivaṭṭāni cattālīsampi saṃvaṭṭavivaṭṭāni – ‘amutrāsiṃ evaṃnāmo evaṃgotto evaṃvaṇṇo evamāhāro evaṃsukhadukkhappaṭisaṃvedī evamāyupariyanto, so tato cuto amutra udapādiṃ; tatrāpāsiṃ evaṃnāmo evaṃgotto evaṃvaṇṇo evamāhāro evaṃsukhadukkhappaṭisaṃvedī evamāyupariyanto, so tato cuto idhūpapanno’ti. Iti sākāraṃ sauddesaṃ anekavihitaṃ pubbenivāsaṃ anussarati. ‘‘So evamāha – ‘sassato attā ca loko ca vañjho kūṭaṭṭho esikaṭṭhāyiṭṭhito; te ca sattā sandhāvanti saṃsaranti cavanti upapajjanti, atthitveva sassatisamaṃ. Taṃ kissa hetu? Ahañhi ātappamanvāya padhānamanvāya anuyogamanvāya appamādamanvāya sammāmanasikāramanvāya tathārūpaṃ cetosamādhiṃ phusāmi, yathāsamāhite citte anekavihitaṃ pubbenivāsaṃ anussarāmi. Seyyathidaṃ – dasapi saṃvaṭṭavivaṭṭāni vīsampi saṃvaṭṭavivaṭṭāni tiṃsampi saṃvaṭṭavivaṭṭāni cattālīsampi saṃvaṭṭavivaṭṭāni – ‘amutrāsiṃ evaṃnāmo evaṃgotto evaṃvaṇṇo evamāhāro evaṃsukhadukkhappaṭisaṃvedī evamāyupariyanto, so tato cuto amutra udapādiṃ; tatrāpāsiṃ evaṃnāmo evaṃgotto evaṃvaṇṇo evamāhāro evaṃsukhadukkhappaṭisaṃvedī evamāyupariyanto, so tato cuto idhūpapannoti. Iti sākāraṃ sauddesaṃ anekavihitaṃ pubbenivāsaṃ anussarāmi. Imināmahaṃ etaṃ jānāmi ‘‘yathā sassato attā ca loko ca vañjho kūṭaṭṭho esikaṭṭhāyiṭṭhito, te ca sattā sandhāvanti saṃsaranti cavanti upapajjanti, atthitveva sassatisama’’nti. Idaṃ, bhikkhave, tatiyaṃ ṭhānaṃ, yaṃ āgamma yaṃ ārabbha eke samaṇabrāhmaṇā sassatavādā sassataṃ attānañca lokañca paññapenti.
\paragraph{34.}
‘‘Catutthe ca bhonto samaṇabrāhmaṇā kimāgamma kimārabbha sassatavādā sassataṃ attānañca lokañca paññapenti? Idha, bhikkhave, ekacco samaṇo vā brāhmaṇo vā takkī hoti vīmaṃsī, so takkapariyāhataṃ vīmaṃsānucaritaṃ sayaṃ paṭibhānaṃ evamāha – ‘sassato attā ca loko ca vañjho kūṭaṭṭho esikaṭṭhāyiṭṭhito; te ca sattā sandhāvanti saṃsaranti cavanti upapajjanti, atthitveva sassatisama’nti. Idaṃ, bhikkhave, catutthaṃ ṭhānaṃ, yaṃ āgamma yaṃ ārabbha eke samaṇabrāhmaṇā sassatavādā sassataṃ attānañca lokañca paññapenti.

\paragraph{35.}
‘‘Imehi kho te, bhikkhave, samaṇabrāhmaṇā sassatavādā sassataṃ attānañca lokañca paññapenti catūhi vatthūhi. Ye hi keci, bhikkhave, samaṇā vā brāhmaṇā vā sassatavādā sassataṃ attānañca lokañca paññapenti, sabbe te imeheva catūhi vatthūhi, etesaṃ vā aññatarena; natthi ito bahiddhā.

\paragraph{36.}
‘‘Tayidaṃ, bhikkhave, tathāgato pajānāti – ‘ime diṭṭhiṭṭhānā evaṃgahitā evaṃparāmaṭṭhā evaṃgatikā bhavanti evaṃabhisamparāyā’ti, tañca tathāgato pajānāti, tato ca uttaritaraṃ pajānāti; tañca pajānanaṃ\footnote{pajānaṃ (?) dī. ni. 3.36 pāḷiaṭṭhakathā passitabbaṃ} na parāmasati, aparāmasato cassa paccattaññeva nibbuti viditā. Vedanānaṃ samudayañca atthaṅgamañca assādañca ādīnavañca nissaraṇañca yathābhūtaṃ viditvā anupādāvimutto, bhikkhave, tathāgato.

\paragraph{37.}
‘‘Ime kho te, bhikkhave, dhammā gambhīrā duddasā duranubodhā santā paṇītā atakkāvacarā nipuṇā paṇḍitavedanīyā, ye tathāgato sayaṃ abhiññā sacchikatvā pavedeti, yehi tathāgatassa yathābhuccaṃ vaṇṇaṃ sammā vadamānā vadeyyuṃ. Paṭhamabhāṇavāro.

\subsubsection{Ekaccasassatavādo}

\paragraph{38.}
‘‘Santi, bhikkhave, eke samaṇabrāhmaṇā ekaccasassatikā ekaccaasassatikā ekaccaṃ sassataṃ ekaccaṃ asassataṃ attānañca lokañca paññapenti catūhi vatthūhi. Te ca bhonto samaṇabrāhmaṇā kimāgamma kimārabbha ekaccasassatikā ekaccaasassatikā ekaccaṃ sassataṃ ekaccaṃ asassataṃ attānañca lokañca paññapenti catūhi vatthūhi?

\paragraph{39.}
‘‘Hoti kho so, bhikkhave, samayo, yaṃ kadāci karahaci dīghassa addhuno accayena ayaṃ loko saṃvaṭṭati. Saṃvaṭṭamāne loke yebhuyyena sattā ābhassarasaṃvattanikā honti. Te tattha honti manomayā pītibhakkhā sayaṃpabhā antalikkhacarā subhaṭṭhāyino, ciraṃ dīghamaddhānaṃ tiṭṭhanti.

\paragraph{40.}
‘‘Hoti kho so, bhikkhave, samayo, yaṃ kadāci karahaci dīghassa addhuno accayena ayaṃ loko vivaṭṭati. Vivaṭṭamāne loke suññaṃ brahmavimānaṃ pātubhavati. Atha kho aññataro satto āyukkhayā vā puññakkhayā vā ābhassarakāyā cavitvā suññaṃ brahmavimānaṃ upapajjati. So tattha hoti manomayo pītibhakkho sayaṃpabho antalikkhacaro subhaṭṭhāyī, ciraṃ dīghamaddhānaṃ tiṭṭhati.

\paragraph{41.}
‘‘Tassa tattha ekakassa dīgharattaṃ nivusitattā anabhirati paritassanā upapajjati – ‘aho vata aññepi sattā itthattaṃ āgaccheyyu’nti. Atha aññepi sattā āyukkhayā vā puññakkhayā vā ābhassarakāyā cavitvā brahmavimānaṃ upapajjanti tassa sattassa sahabyataṃ. Tepi tattha honti manomayā pītibhakkhā sayaṃpabhā antalikkhacarā subhaṭṭhāyino, ciraṃ dīghamaddhānaṃ tiṭṭhanti.

\paragraph{42.}
‘‘Tatra, bhikkhave, yo so satto paṭhamaṃ upapanno tassa evaṃ hoti – ‘ahamasmi brahmā mahābrahmā abhibhū anabhibhūto aññadatthudaso vasavattī issaro kattā nimmātā seṭṭho sajitā\footnote{sajjitā (syā. kaṃ.)} vasī pitā bhūtabhabyānaṃ. Mayā ime sattā nimmitā. Taṃ kissa hetu? Mamañhi pubbe etadahosi – ‘‘aho vata aññepi sattā itthattaṃ āgaccheyyu’’nti. Iti mama ca manopaṇidhi, ime ca sattā itthattaṃ āgatā’ti. ‘‘Yepi te sattā pacchā upapannā, tesampi evaṃ hoti – ‘ayaṃ kho bhavaṃ brahmā mahābrahmā abhibhū anabhibhūto aññadatthudaso vasavattī issaro kattā nimmātā seṭṭho sajitā vasī pitā bhūtabhabyānaṃ. Iminā mayaṃ bhotā brahmunā nimmitā. Taṃ kissa hetu? Imañhi mayaṃ addasāma idha paṭhamaṃ upapannaṃ, mayaṃ panamha pacchā upapannā’ti.

\paragraph{43.}
‘‘Tatra, bhikkhave, yo so satto paṭhamaṃ upapanno, so dīghāyukataro ca hoti vaṇṇavantataro ca mahesakkhataro ca. Ye pana te sattā pacchā upapannā, te appāyukatarā ca honti dubbaṇṇatarā ca appesakkhatarā ca.

\paragraph{44.}
‘‘Ṭhānaṃ kho panetaṃ, bhikkhave, vijjati, yaṃ aññataro satto tamhā kāyā cavitvā itthattaṃ āgacchati. Itthattaṃ āgato samāno agārasmā anagāriyaṃ pabbajati. Agārasmā anagāriyaṃ pabbajito samāno ātappamanvāya padhānamanvāya anuyogamanvāya appamādamanvāya sammāmanasikāramanvāya tathārūpaṃ cetosamādhiṃ phusati, yathāsamāhite citte taṃ pubbenivāsaṃ anussarati, tato paraṃ nānussarati. ‘‘So evamāha – ‘yo kho so bhavaṃ brahmā mahābrahmā abhibhū anabhibhūto aññadatthudaso vasavattī issaro kattā nimmātā seṭṭho sajitā vasī pitā bhūtabhabyānaṃ, yena mayaṃ bhotā brahmunā nimmitā, so nicco dhuvo sassato avipariṇāmadhammo sassatisamaṃ tatheva ṭhassati. Ye pana mayaṃ ahumhā tena bhotā brahmunā nimmitā, te mayaṃ aniccā addhuvā appāyukā cavanadhammā itthattaṃ āgatā’ti. Idaṃ kho, bhikkhave, paṭhamaṃ ṭhānaṃ, yaṃ āgamma yaṃ ārabbha eke samaṇabrāhmaṇā ekaccasassatikā ekaccaasassatikā ekaccaṃ sassataṃ ekaccaṃ asassataṃ attānañca lokañca paññapenti.

\paragraph{45.}
‘‘Dutiye ca bhonto samaṇabrāhmaṇā kimāgamma kimārabbha ekaccasassatikā ekaccaasassatikā ekaccaṃ sassataṃ ekaccaṃ asassataṃ attānañca lokañca paññapenti? Santi, bhikkhave, khiḍḍāpadosikā nāma devā, te ativelaṃ hassakhiḍḍāratidhammasamāpannā\footnote{hasakhiḍḍāratidhammasamāpannā (ka.)} viharanti. Tesaṃ ativelaṃ hassakhiḍḍāratidhammasamāpannānaṃ viharataṃ sati sammussati.\footnote{pamussati (sī. syā.)} Satiyā sammosā te devā tamhā kāyā cavanti.

\paragraph{46.}
‘‘Ṭhānaṃ kho panetaṃ, bhikkhave, vijjati yaṃ aññataro satto tamhā kāyā cavitvā itthattaṃ āgacchati. Itthattaṃ āgato samāno agārasmā anagāriyaṃ pabbajati. Agārasmā anagāriyaṃ pabbajito samāno ātappamanvāya padhānamanvāya anuyogamanvāya appamādamanvāya sammāmanasikāramanvāya tathārūpaṃ cetosamādhiṃ phusati, yathāsamāhite citte taṃ pubbenivāsaṃ anussarati, tato paraṃ nānussarati. ‘‘So evamāha – ‘ye kho te bhonto devā na khiḍḍāpadosikā, te na ativelaṃ hassakhiḍḍāratidhammasamāpannā viharanti. Tesaṃ na ativelaṃ hassakhiḍḍāratidhammasamāpannānaṃ viharataṃ sati na sammussati. Satiyā asammosā te devā tamhā kāyā na cavanti; niccā dhuvā sassatā avipariṇāmadhammā sassatisamaṃ tatheva ṭhassanti. Ye pana mayaṃ ahumhā khiḍḍāpadosikā, te mayaṃ ativelaṃ hassakhiḍḍāratidhammasamāpannā viharimhā. Tesaṃ no ativelaṃ hassakhiḍḍāratidhammasamāpannānaṃ viharataṃ sati sammussati. Satiyā sammosā evaṃ mayaṃ tamhā kāyā cutā aniccā addhuvā appāyukā cavanadhammā itthattaṃ āgatā’ti. Idaṃ, bhikkhave, dutiyaṃ ṭhānaṃ, yaṃ āgamma yaṃ ārabbha eke samaṇabrāhmaṇā ekaccasassatikā ekaccaasassatikā ekaccaṃ sassataṃ ekaccaṃ asassataṃ attānañca lokañca paññapenti.

\paragraph{47.}
‘‘Tatiye ca bhonto samaṇabrāhmaṇā kimāgamma kimārabbha ekaccasassatikā ekaccaasassatikā ekaccaṃ sassataṃ ekaccaṃ asassataṃ attānañca lokañca paññapenti? Santi, bhikkhave, manopadosikā nāma devā, te ativelaṃ aññamaññaṃ upanijjhāyanti. Te ativelaṃ aññamaññaṃ upanijjhāyantā aññamaññamhi cittāni padūsenti. Te aññamaññaṃ paduṭṭhacittā kilantakāyā kilantacittā. Te devā tamhā kāyā cavanti.

\paragraph{48.}
‘‘Ṭhānaṃ kho panetaṃ, bhikkhave, vijjati yaṃ aññataro satto tamhā kāyā cavitvā itthattaṃ āgacchati. Itthattaṃ āgato samāno agārasmā anagāriyaṃ pabbajati. Agārasmā anagāriyaṃ pabbajito samāno ātappamanvāya padhānamanvāya anuyogamanvāya appamādamanvāya sammāmanasikāramanvāya tathārūpaṃ cetosamādhiṃ phusati, yathāsamāhite citte taṃ pubbenivāsaṃ anussarati, tato paraṃ nānussarati. ‘‘So evamāha – ‘ye kho te bhonto devā na manopadosikā, te nātivelaṃ aññamaññaṃ upanijjhāyanti. Te nātivelaṃ aññamaññaṃ upanijjhāyantā aññamaññamhi cittāni nappadūsenti. Te aññamaññaṃ appaduṭṭhacittā akilantakāyā akilantacittā. Te devā tamhā kāyā na cavanti, niccā dhuvā sassatā avipariṇāmadhammā sassatisamaṃ tatheva ṭhassanti. Ye pana mayaṃ ahumhā manopadosikā, te mayaṃ ativelaṃ aññamaññaṃ upanijjhāyimhā. Te mayaṃ ativelaṃ aññamaññaṃ upanijjhāyantā aññamaññamhi cittāni padūsimhā, te mayaṃ aññamaññaṃ paduṭṭhacittā kilantakāyā kilantacittā. Evaṃ mayaṃ tamhā kāyā cutā aniccā addhuvā appāyukā cavanadhammā itthattaṃ āgatā’ti. Idaṃ, bhikkhave, tatiyaṃ ṭhānaṃ, yaṃ āgamma yaṃ ārabbha eke samaṇabrāhmaṇā ekaccasassatikā ekaccaasassatikā ekaccaṃ sassataṃ ekaccaṃ asassataṃ attānañca lokañca paññapenti.

\paragraph{49.}
‘‘Catutthe ca bhonto samaṇabrāhmaṇā kimāgamma kimārabbha ekaccasassatikā ekaccaasassatikā ekaccaṃ sassataṃ ekaccaṃ asassataṃ attānañca lokañca paññapenti? Idha, bhikkhave, ekacco samaṇo vā brāhmaṇo vā takkī hoti vīmaṃsī. So takkapariyāhataṃ vīmaṃsānucaritaṃ sayaṃpaṭibhānaṃ evamāha – ‘yaṃ kho idaṃ vuccati cakkhuṃ itipi sotaṃ itipi ghānaṃ itipi jivhā itipi kāyo itipi, ayaṃ attā anicco addhuvo asassato vipariṇāmadhammo. Yañca kho idaṃ vuccati cittanti vā manoti vā viññāṇanti vā ayaṃ attā nicco dhuvo sassato avipariṇāmadhammo sassatisamaṃ tatheva ṭhassatī’ti. Idaṃ, bhikkhave, catutthaṃ ṭhānaṃ, yaṃ āgamma yaṃ ārabbha eke samaṇabrāhmaṇā ekaccasassatikā ekaccaasassatikā ekaccaṃ sassataṃ ekaccaṃ asassataṃ attānañca lokañca paññapenti.

\paragraph{50.}
‘‘Imehi kho te, bhikkhave, samaṇabrāhmaṇā ekaccasassatikā ekaccaasassatikā ekaccaṃ sassataṃ ekaccaṃ asassataṃ attānañca lokañca paññapenti catūhi vatthūhi. Ye hi keci, bhikkhave, samaṇā vā brāhmaṇā vā ekaccasassatikā ekaccaasassatikā ekaccaṃ sassataṃ ekaccaṃ asassataṃ attānañca lokañca paññapenti, sabbe te imeheva catūhi vatthūhi, etesaṃ vā aññatarena; natthi ito bahiddhā.

\paragraph{51.}
‘‘Tayidaṃ, bhikkhave, tathāgato pajānāti – ‘ime diṭṭhiṭṭhānā evaṃgahitā evaṃparāmaṭṭhā evaṃgatikā bhavanti evaṃabhisamparāyā’ti. Tañca tathāgato pajānāti, tato ca uttaritaraṃ pajānāti, tañca pajānanaṃ na parāmasati, aparāmasato cassa paccattaññeva nibbuti viditā. Vedanānaṃ samudayañca atthaṅgamañca assādañca ādīnavañca nissaraṇañca yathābhūtaṃ viditvā anupādāvimutto, bhikkhave, tathāgato.
\paragraph{52.}
‘‘Ime kho te, bhikkhave, dhammā gambhīrā duddasā duranubodhā santā paṇītā atakkāvacarā nipuṇā paṇḍitavedanīyā, ye tathāgato sayaṃ abhiññā sacchikatvā pavedeti, yehi tathāgatassa yathābhuccaṃ vaṇṇaṃ sammā vadamānā vadeyyuṃ.

\subsubsection{Antānantavādo}

\paragraph{53.}
‘‘Santi, bhikkhave, eke samaṇabrāhmaṇā antānantikā antānantaṃ lokassa paññapenti catūhi vatthūhi. Te ca bhonto samaṇabrāhmaṇā kimāgamma kimārabbha antānantikā antānantaṃ lokassa paññapenti catūhi vatthūhi?

\paragraph{54.}
‘‘Idha, bhikkhave, ekacco samaṇo vā brāhmaṇo vā ātappamanvāya padhānamanvāya anuyogamanvāya appamādamanvāya sammāmanasikāramanvāya tathārūpaṃ cetosamādhiṃ phusati, yathāsamāhite citte antasaññī lokasmiṃ viharati. ‘‘So evamāha – ‘antavā ayaṃ loko parivaṭumo. Taṃ kissa hetu? Ahañhi ātappamanvāya padhānamanvāya anuyogamanvāya appamādamanvāya sammāmanasikāramanvāya tathārūpaṃ cetosamādhiṃ phusāmi, yathāsamāhite citte antasaññī lokasmiṃ viharāmi. Imināmahaṃ etaṃ jānāmi – yathā antavā ayaṃ loko parivaṭumo’ti. Idaṃ, bhikkhave, paṭhamaṃ ṭhānaṃ, yaṃ āgamma yaṃ ārabbha eke samaṇabrāhmaṇā antānantikā antānantaṃ lokassa paññapenti.

\paragraph{55.}
‘‘Dutiye ca bhonto samaṇabrāhmaṇā kimāgamma kimārabbha antānantikā antānantaṃ lokassa paññapenti? Idha, bhikkhave, ekacco samaṇo vā brāhmaṇo vā ātappamanvāya padhānamanvāya anuyogamanvāya appamādamanvāya sammāmanasikāramanvāya tathārūpaṃ cetosamādhiṃ phusati, yathāsamāhite citte anantasaññī lokasmiṃ viharati. ‘‘So evamāha – ‘ananto ayaṃ loko apariyanto. Ye te samaṇabrāhmaṇā evamāhaṃsu – ‘‘antavā ayaṃ loko parivaṭumo’’ti, tesaṃ musā. Ananto ayaṃ loko apariyanto. Taṃ kissa hetu? Ahañhi ātappamanvāya padhānamanvāya anuyogamanvāya appamādamanvāya sammāmanasikāramanvāya tathārūpaṃ cetosamādhiṃ phusāmi, yathāsamāhite citte anantasaññī lokasmiṃ viharāmi. Imināmahaṃ etaṃ jānāmi – yathā ananto ayaṃ loko apariyanto’ti. Idaṃ, bhikkhave, dutiyaṃ ṭhānaṃ, yaṃ āgamma yaṃ ārabbha eke samaṇabrāhmaṇā antānantikā antānantaṃ lokassa paññapenti.

\paragraph{56.}
‘‘Tatiye ca bhonto samaṇabrāhmaṇā kimāgamma kimārabbha antānantikā antānantaṃ lokassa paññapenti? Idha, bhikkhave, ekacco samaṇo vā brāhmaṇo vā ātappamanvāya padhānamanvāya anuyogamanvāya appamādamanvāya sammāmanasikāramanvāya tathārūpaṃ cetosamādhiṃ phusati, yathāsamāhite citte uddhamadho antasaññī lokasmiṃ viharati, tiriyaṃ anantasaññī. ‘‘So evamāha – ‘antavā ca ayaṃ loko ananto ca. Ye te samaṇabrāhmaṇā evamāhaṃsu – ‘‘antavā ayaṃ loko parivaṭumo’’ti, tesaṃ musā. Yepi te samaṇabrāhmaṇā evamāhaṃsu – ‘‘ananto ayaṃ loko apariyanto’’ti, tesampi musā. Antavā ca ayaṃ loko ananto ca. Taṃ kissa hetu? Ahañhi ātappamanvāya padhānamanvāya anuyogamanvāya appamādamanvāya sammāmanasikāramanvāya tathārūpaṃ cetosamādhiṃ phusāmi, yathāsamāhite citte uddhamadho antasaññī lokasmiṃ viharāmi, tiriyaṃ anantasaññī. Imināmahaṃ etaṃ jānāmi – yathā antavā ca ayaṃ loko ananto cā’ti. Idaṃ, bhikkhave, tatiyaṃ ṭhānaṃ, yaṃ āgamma yaṃ ārabbha eke samaṇabrāhmaṇā antānantikā antānantaṃ lokassa paññapenti.

\paragraph{57.}
‘‘Catutthe ca bhonto samaṇabrāhmaṇā kimāgamma kimārabbha antānantikā antānantaṃ lokassa paññapenti? Idha, bhikkhave, ekacco samaṇo vā brāhmaṇo vā takkī hoti vīmaṃsī. So takkapariyāhataṃ vīmaṃsānucaritaṃ sayaṃpaṭibhānaṃ evamāha – ‘nevāyaṃ loko antavā, na panānanto. Ye te samaṇabrāhmaṇā evamāhaṃsu – ‘‘antavā ayaṃ loko parivaṭumo’’ti, tesaṃ musā. Yepi te samaṇabrāhmaṇā evamāhaṃsu – ‘‘ananto ayaṃ loko apariyanto’’ti, tesampi musā. Yepi te samaṇabrāhmaṇā evamāhaṃsu – ‘‘antavā ca ayaṃ loko ananto cā’’ti, tesampi musā. Nevāyaṃ loko antavā, na panānanto’ti. Idaṃ, bhikkhave, catutthaṃ ṭhānaṃ, yaṃ āgamma yaṃ ārabbha eke samaṇabrāhmaṇā antānantikā antānantaṃ lokassa paññapenti.

\paragraph{58.}
‘‘Imehi kho te, bhikkhave, samaṇabrāhmaṇā antānantikā antānantaṃ lokassa paññapenti catūhi vatthūhi. Ye hi keci, bhikkhave, samaṇā vā brāhmaṇā vā antānantikā antānantaṃ lokassa paññapenti, sabbe te imeheva catūhi vatthūhi, etesaṃ vā aññatarena; natthi ito bahiddhā.

\paragraph{59.}
‘‘Tayidaṃ, bhikkhave, tathāgato pajānāti – ‘ime diṭṭhiṭṭhānā evaṃgahitā evaṃparāmaṭṭhā evaṃgatikā bhavanti evaṃabhisamparāyā’ti. Tañca tathāgato pajānāti, tato ca uttaritaraṃ pajānāti, tañca pajānanaṃ na parāmasati, aparāmasato cassa paccattaññeva nibbuti viditā. Vedanānaṃ samudayañca atthaṅgamañca assādañca ādīnavañca nissaraṇañca yathābhūtaṃ viditvā anupādāvimutto, bhikkhave, tathāgato.

\paragraph{60.}
‘‘Ime kho te, bhikkhave, dhammā gambhīrā duddasā duranubodhā santā paṇītā atakkāvacarā nipuṇā paṇḍitavedanīyā, ye tathāgato sayaṃ abhiññā sacchikatvā pavedeti, yehi tathāgatassa yathābhuccaṃ vaṇṇaṃ sammā vadamānā vadeyyuṃ.

\subsubsection{Amarāvikkhepavādo}

\paragraph{61.}
‘‘Santi, bhikkhave, eke samaṇabrāhmaṇā amarāvikkhepikā, tattha tattha pañhaṃ puṭṭhā samānā vācāvikkhepaṃ āpajjanti amarāvikkhepaṃ catūhi vatthūhi. Te ca bhonto samaṇabrāhmaṇā kimāgamma kimārabbha amarāvikkhepikā tattha tattha pañhaṃ puṭṭhā samānā vācāvikkhepaṃ āpajjanti amarāvikkhepaṃ catūhi vatthūhi?

\paragraph{62.}
‘‘Idha, bhikkhave, ekacco samaṇo vā brāhmaṇo vā ‘idaṃ kusala’nti yathābhūtaṃ nappajānāti, ‘idaṃ akusala’nti yathābhūtaṃ nappajānāti. Tassa evaṃ hoti – ‘ahaṃ kho ‘‘idaṃ kusala’’nti yathābhūtaṃ nappajānāmi, ‘‘idaṃ akusala’’nti yathābhūtaṃ nappajānāmi. Ahañce kho pana ‘‘idaṃ kusala’’nti yathābhūtaṃ appajānanto, ‘‘idaṃ akusala’’nti yathābhūtaṃ appajānanto, ‘idaṃ kusala’nti vā byākareyyaṃ, ‘idaṃ akusala’nti vā byākareyyaṃ, taṃ mamassa musā. Yaṃ mamassa musā, so mamassa vighāto. Yo mamassa vighāto so mamassa antarāyo’ti. Iti so musāvādabhayā musāvādaparijegucchā nevidaṃ kusalanti byākaroti, na panidaṃ akusalanti byākaroti, tattha tattha pañhaṃ puṭṭho samāno vācāvikkhepaṃ āpajjati amarāvikkhepaṃ – ‘evantipi me no; tathātipi me no; aññathātipi me no; notipi me no; no notipi me no’ti. Idaṃ, bhikkhave, paṭhamaṃ ṭhānaṃ, yaṃ āgamma yaṃ ārabbha eke samaṇabrāhmaṇā amarāvikkhepikā tattha tattha pañhaṃ puṭṭhā samānā vācāvikkhepaṃ āpajjanti amarāvikkhepaṃ.

\paragraph{63.}
‘‘Dutiye ca bhonto samaṇabrāhmaṇā kimāgamma kimārabbha amarāvikkhepikā tattha tattha pañhaṃ puṭṭhā samānā vācāvikkhepaṃ āpajjanti amarāvikkhepaṃ? Idha, bhikkhave, ekacco samaṇo vā brāhmaṇo vā ‘idaṃ kusala’nti yathābhūtaṃ nappajānāti, ‘idaṃ akusala’nti yathābhūtaṃ nappajānāti. Tassa evaṃ hoti – ‘ahaṃ kho ‘‘idaṃ kusala’’nti yathābhūtaṃ nappajānāmi, ‘‘idaṃ akusala’’nti yathābhūtaṃ nappajānāmi. Ahañce kho pana ‘‘idaṃ kusala’’nti yathābhūtaṃ appajānanto, ‘‘idaṃ akusala’’nti yathābhūtaṃ appajānanto, ‘‘idaṃ kusala’’nti vā byākareyyaṃ, ‘‘idaṃ akusala’nti vā byākareyyaṃ, tattha me assa chando vā rāgo vā doso vā paṭigho vā. Yattha\footnote{yo (?)} me assa chando vā rāgo vā doso vā paṭigho vā, taṃ mamassa upādānaṃ. Yaṃ mamassa upādānaṃ, so mamassa vighāto. Yo mamassa vighāto, so mamassa antarāyo’ti. Iti so upādānabhayā upādānaparijegucchā nevidaṃ kusalanti byākaroti, na panidaṃ akusalanti byākaroti, tattha tattha pañhaṃ puṭṭho samāno vācāvikkhepaṃ āpajjati amarāvikkhepaṃ – ‘evantipi me no; tathātipi me no; aññathātipi me no; notipi me no; no notipi me no’ti. Idaṃ, bhikkhave, dutiyaṃ ṭhānaṃ, yaṃ āgamma yaṃ ārabbha eke samaṇabrāhmaṇā amarāvikkhepikā tattha tattha pañhaṃ puṭṭhā samānā vācāvikkhepaṃ āpajjanti amarāvikkhepaṃ.

\paragraph{64.}
‘‘Tatiye ca bhonto samaṇabrāhmaṇā kimāgamma kimārabbha amarāvikkhepikā tattha tattha pañhaṃ puṭṭhā samānā vācāvikkhepaṃ āpajjanti amarāvikkhepaṃ? Idha, bhikkhave, ekacco samaṇo vā brāhmaṇo vā ‘idaṃ kusala’nti yathābhūtaṃ nappajānāti, ‘idaṃ akusala’nti yathābhūtaṃ nappajānāti. Tassa evaṃ hoti – ‘ahaṃ kho ‘‘idaṃ kusala’’nti yathābhūtaṃ nappajānāmi, ‘‘idaṃ akusala’nti yathābhūtaṃ nappajānāmi. Ahañce kho pana ‘‘idaṃ kusala’’nti yathābhūtaṃ appajānanto ‘‘idaṃ akusala’’nti yathābhūtaṃ appajānanto ‘‘idaṃ kusala’’nti vā byākareyyaṃ, ‘‘idaṃ akusala’’nti vā byākareyyaṃ. Santi hi kho samaṇabrāhmaṇā paṇḍitā nipuṇā kataparappavādā vālavedhirūpā, te bhindantā\footnote{vobhindantā (sī. pī.)} maññe caranti paññāgatena diṭṭhigatāni, te maṃ tattha samanuyuñjeyyuṃ samanugāheyyuṃ samanubhāseyyuṃ. Ye maṃ tattha samanuyuñjeyyuṃ samanugāheyyuṃ samanubhāseyyuṃ, tesāhaṃ na sampāyeyyaṃ. Yesāhaṃ na sampāyeyyaṃ, so mamassa vighāto. Yo mamassa vighāto, so mamassa antarāyo’ti. Iti so anuyogabhayā anuyogaparijegucchā nevidaṃ kusalanti byākaroti, na panidaṃ akusalanti byākaroti, tattha tattha pañhaṃ puṭṭho samāno vācāvikkhepaṃ āpajjati amarāvikkhepaṃ – ‘evantipi me no; tathātipi me no; aññathātipi me no; notipi me no; no notipi me no’ti. Idaṃ, bhikkhave, tatiyaṃ ṭhānaṃ, yaṃ āgamma yaṃ ārabbha eke samaṇabrāhmaṇā amarāvikkhepikā tattha tattha pañhaṃ puṭṭhā samānā vācāvikkhepaṃ āpajjanti amarāvikkhepaṃ.

\paragraph{65.}
‘‘Catutthe ca bhonto samaṇabrāhmaṇā kimāgamma kimārabbha amarāvikkhepikā tattha tattha pañhaṃ puṭṭhā samānā vācāvikkhepaṃ āpajjanti amarāvikkhepaṃ? Idha, bhikkhave, ekacco samaṇo vā brāhmaṇo vā mando hoti momūho. So mandattā momūhattā tattha tattha pañhaṃ puṭṭho samāno vācāvikkhepaṃ āpajjati amarāvikkhepaṃ – ‘atthi paro loko’ti iti ce maṃ pucchasi, ‘atthi paro loko’ti iti ce me assa, ‘atthi paro loko’ti iti te naṃ byākareyyaṃ, ‘evantipi me no, tathātipi me no, aññathātipi me no, notipi me no, no notipi me no’ti. ‘Natthi paro loko …pe… ‘atthi ca natthi ca paro loko …pe… ‘nevatthi na natthi paro loko …pe… ‘atthi sattā opapātikā …pe… ‘natthi sattā opapātikā …pe… ‘atthi ca natthi ca sattā opapātikā …pe… ‘nevatthi na natthi sattā opapātikā …pe… ‘atthi sukatadukkaṭānaṃ \footnote{sukaṭadukkaṭānaṃ (sī. syā. kaṃ.)} kammānaṃ phalaṃ vipāko …pe… ‘natthi sukatadukkaṭānaṃ kammānaṃ phalaṃ vipāko …pe… ‘atthi ca natthi ca sukatadukkaṭānaṃ kammānaṃ phalaṃ vipāko …pe… ‘nevatthi na natthi sukatadukkaṭānaṃ kammānaṃ phalaṃ vipāko …pe… ‘hoti tathāgato paraṃ maraṇā …pe… ‘na hoti tathāgato paraṃ maraṇā …pe… ‘hoti ca na ca hoti\footnote{na hoti ca (sī. ka.)} tathāgato paraṃ maraṇā …pe… ‘neva hoti na na hoti tathāgato paraṃ maraṇāti iti ce maṃ pucchasi, ‘neva hoti na na hoti tathāgato paraṃ maraṇā’ti iti ce me assa, ‘neva hoti na na hoti tathāgato paraṃ maraṇā’ti iti te naṃ byākareyyaṃ, ‘evantipi me no, tathātipi me no, aññathātipi me no, notipi me no, no notipi me no’ti. Idaṃ, bhikkhave, catutthaṃ ṭhānaṃ, yaṃ āgamma yaṃ ārabbha eke samaṇabrāhmaṇā amarāvikkhepikā tattha tattha pañhaṃ puṭṭhā samānā vācāvikkhepaṃ āpajjanti amarāvikkhepaṃ.

\paragraph{66.}
‘‘Imehi kho te, bhikkhave, samaṇabrāhmaṇā amarāvikkhepikā tattha tattha pañhaṃ puṭṭhā samānā vācāvikkhepaṃ āpajjanti amarāvikkhepaṃ catūhi vatthūhi. Ye hi keci, bhikkhave, samaṇā vā brāhmaṇā vā amarāvikkhepikā tattha tattha pañhaṃ puṭṭhā samānā vācāvikkhepaṃ āpajjanti amarāvikkhepaṃ, sabbe te imeheva catūhi vatthūhi, etesaṃ vā aññatarena, natthi ito bahiddhā …pe… yehi tathāgatassa yathābhuccaṃ vaṇṇaṃ sammā vadamānā vadeyyuṃ.

\subsubsection{Adhiccasamuppannavādo}

\paragraph{67.}
‘‘Santi, bhikkhave, eke samaṇabrāhmaṇā adhiccasamuppannikā adhiccasamuppannaṃ attānañca lokañca paññapenti dvīhi vatthūhi. Te ca bhonto samaṇabrāhmaṇā kimāgamma kimārabbha adhiccasamuppannikā adhiccasamuppannaṃ attānañca lokañca paññapenti dvīhi vatthūhi?

\paragraph{68.}
‘‘Santi, bhikkhave, asaññasattā nāma devā. Saññuppādā ca pana te devā tamhā kāyā cavanti. Ṭhānaṃ kho panetaṃ, bhikkhave, vijjati, yaṃ aññataro satto tamhā kāyā cavitvā itthattaṃ āgacchati. Itthattaṃ āgato samāno agārasmā anagāriyaṃ pabbajati. Agārasmā anagāriyaṃ pabbajito samāno ātappamanvāya padhānamanvāya anuyogamanvāya appamādamanvāya sammāmanasikāramanvāya tathārūpaṃ cetosamādhiṃ phusati, yathāsamāhite citte saññuppādaṃ anussarati, tato paraṃ nānussarati. So evamāha – ‘adhiccasamuppanno attā ca loko ca. Taṃ kissa hetu? Ahañhi pubbe nāhosiṃ, somhi etarahi ahutvā santatāya pariṇato’ti. Idaṃ, bhikkhave, paṭhamaṃ ṭhānaṃ, yaṃ āgamma yaṃ ārabbha eke samaṇabrāhmaṇā nnadhiccasamuppannikā adhiccasamuppannaṃ attānañca lokañca paññapenti.

\paragraph{69.}
‘‘Dutiye ca bhonto samaṇabrāhmaṇā kimāgamma kimārabbha adhiccasamuppannikā adhiccasamuppannaṃ attānañca lokañca paññapenti? Idha, bhikkhave, ekacco samaṇo vā brāhmaṇo vā takkī hoti vīmaṃsī. So takkapariyāhataṃ vīmaṃsānucaritaṃ sayaṃpaṭibhānaṃ evamāha – ‘adhiccasamuppanno attā ca loko cā’ti. Idaṃ, bhikkhave, dutiyaṃ ṭhānaṃ, yaṃ āgamma yaṃ ārabbha eke samaṇabrāhmaṇā adhiccasamuppannikā adhiccasamuppannaṃ attānañca lokañca paññapenti.

\paragraph{70.}
‘‘Imehi kho te, bhikkhave, samaṇabrāhmaṇā adhiccasamuppannikā adhiccasamuppannaṃ attānañca lokañca paññapenti dvīhi vatthūhi. Ye hi keci, bhikkhave, samaṇā vā brāhmaṇā vā adhiccasamuppannikā adhiccasamuppannaṃ attānañca lokañca paññapenti, sabbe te imeheva dvīhi vatthūhi, etesaṃ vā aññatarena, natthi ito bahiddhā… pe… yehi tathāgatassa yathābhuccaṃ vaṇṇaṃ sammā vadamānā vadeyyuṃ.

\paragraph{71.}
‘‘Imehi kho te, bhikkhave, samaṇabrāhmaṇā pubbantakappikā pubbantānudiṭṭhino pubbantaṃ ārabbha anekavihitāni adhimuttipadāni abhivadanti aṭṭhārasahi vatthūhi. Ye hi keci, bhikkhave, samaṇā vā brāhmaṇā vā pubbantakappikā pubbantānudiṭṭhino pubbantamārabbha anekavihitāni adhimuttipadāni abhivadanti, sabbe te imeheva aṭṭhārasahi vatthūhi, etesaṃ vā aññatarena, natthi ito bahiddhā.

\paragraph{72.}
‘‘Tayidaṃ, bhikkhave, tathāgato pajānāti – ‘ime diṭṭhiṭṭhānā evaṃgahitā evaṃparāmaṭṭhā evaṃgatikā bhavanti evaṃabhisamparāyā’ti. Tañca tathāgato pajānāti, tato ca uttaritaraṃ pajānāti, tañca pajānanaṃ na parāmasati, aparāmasato cassa paccattaññeva nibbuti viditā. Vedanānaṃ samudayañca atthaṅgamañca assādañca ādīnavañca nissaraṇañca yathābhūtaṃ viditvā anupādāvimutto, bhikkhave, tathāgato.

\paragraph{73.}
‘‘Ime kho te, bhikkhave, dhammā gambhīrā duddasā duranubodhā santā paṇītā atakkāvacarā nipuṇā paṇḍitavedanīyā, ye tathāgato sayaṃ abhiññā sacchikatvā pavedeti, yehi tathāgatassa yathābhuccaṃ vaṇṇaṃ sammā vadamānā vadeyyuṃ. Dutiyabhāṇavāro.

\subsubsection{Aparantakappikā}

\paragraph{74.}
‘‘Santi, bhikkhave, eke samaṇabrāhmaṇā aparantakappikā aparantānudiṭṭhino, aparantaṃ ārabbha anekavihitāni adhimuttipadāni abhivadanti catucattārīsāya \footnote{catucattālīsāya (syā. kaṃ.)} vatthūhi. Te ca bhonto samaṇabrāhmaṇā kimāgamma kimārabbha aparantakappikā aparantānudiṭṭhino aparantaṃ ārabbha anekavihitāni adhimuttipadāni abhivadanti catucattārīsāya vatthūhi?

\subsubsection{Saññīvādo}

\paragraph{75.}
‘‘Santi, bhikkhave, eke samaṇabrāhmaṇā uddhamāghātanikā saññīvādā uddhamāghātanaṃ saññiṃ attānaṃ paññapenti soḷasahi vatthūhi. Te ca bhonto samaṇabrāhmaṇā kimāgamma kimārabbha uddhamāghātanikā saññīvādā uddhamāghātanaṃ saññiṃ attānaṃ paññapenti soḷasahi vatthūhi?

\paragraph{76.}
‘‘‘Rūpī attā hoti arogo paraṃ maraṇā saññī’ti naṃ paññapenti. ‘Arūpī attā hoti arogo paraṃ maraṇā saññī’ti naṃ paññapenti. ‘Rūpī ca arūpī ca attā hoti …pe… nevarūpī nārūpī attā hoti… antavā attā hoti… anantavā attā hoti… antavā ca anantavā ca attā hoti… nevantavā nānantavā attā hoti… ekattasaññī attā hoti… nānattasaññī attā hoti… parittasaññī attā hoti… appamāṇasaññī attā hoti… ekantasukhī attā hoti… ekantadukkhī attā hoti. Sukhadukkhī attā hoti. Adukkhamasukhī attā hoti arogo paraṃ maraṇā saññī’ti naṃ paññapenti.

\paragraph{77.}
‘‘Imehi kho te, bhikkhave, samaṇabrāhmaṇā uddhamāghātanikā saññīvādā uddhamāghātanaṃ saññiṃ attānaṃ paññapenti soḷasahi vatthūhi. Ye hi keci, bhikkhave, samaṇā vā brāhmaṇā vā uddhamāghātanikā saññīvādā uddhamāghātanaṃ saññiṃ attānaṃ paññapenti, sabbe te imeheva soḷasahi vatthūhi, etesaṃ vā aññatarena, natthi ito bahiddhā …pe… yehi tathāgatassa yathābhuccaṃ vaṇṇaṃ sammā vadamānā vadeyyuṃ.

\subsubsection{Asaññīvādo}

\paragraph{78.}
‘‘Santi, bhikkhave, eke samaṇabrāhmaṇā uddhamāghātanikā asaññīvādā uddhamāghātanaṃ asaññiṃ attānaṃ paññapenti aṭṭhahi vatthūhi. Te ca bhonto samaṇabrāhmaṇā kimāgamma kimārabbha uddhamāghātanikā asaññīvādā uddhamāghātanaṃ asaññiṃ attānaṃ paññapenti aṭṭhahi vatthūhi?

\paragraph{79.}
‘‘‘Rūpī attā hoti arogo paraṃ maraṇā asaññī’ti naṃ paññapenti. ‘Arūpī attā hoti arogo paraṃ maraṇā asaññī’ti naṃ paññapenti. ‘Rūpī ca arūpī ca attā hoti …pe… nevarūpī nārūpī attā hoti… antavā attā hoti… anantavā attā hoti… antavā ca anantavā ca attā hoti… nevantavā nānantavā attā hoti arogo paraṃ maraṇā asaññī’ti naṃ paññapenti.

\paragraph{80.}
‘‘Imehi kho te, bhikkhave, samaṇabrāhmaṇā uddhamāghātanikā asaññīvādā uddhamāghātanaṃ asaññiṃ attānaṃ paññapenti aṭṭhahi vatthūhi. Ye hi keci, bhikkhave, samaṇā vā brāhmaṇā vā uddhamāghātanikā asaññīvādā uddhamāghātanaṃ asaññiṃ attānaṃ paññapenti, sabbe te imeheva aṭṭhahi vatthūhi, etesaṃ vā aññatarena, natthi ito bahiddhā …pe… yehi tathāgatassa yathābhuccaṃ vaṇṇaṃ sammā vadamānā vadeyyuṃ.

\subsubsection{Nevasaññīnāsaññīvādo}

\paragraph{81.}
‘‘Santi, bhikkhave, eke samaṇabrāhmaṇā uddhamāghātanikā nevasaññīnāsaññīvādā, uddhamāghātanaṃ nevasaññīnāsaññiṃ attānaṃ paññapenti aṭṭhahi vatthūhi. Te ca bhonto samaṇabrāhmaṇā kimāgamma kimārabbha uddhamāghātanikā nevasaññīnāsaññīvādā uddhamāghātanaṃ nevasaññīnāsaññiṃ attānaṃ paññapenti aṭṭhahi vatthūhi?

\paragraph{82.}
‘‘‘Rūpī attā hoti arogo paraṃ maraṇā nevasaññīnāsaññī’ti naṃ paññapenti ‘arūpī attā hoti …pe… rūpī ca arūpī ca attā hoti… nevarūpī nārūpī attā hoti… antavā attā hoti… anantavā attā hoti… antavā ca anantavā ca attā hoti… nevantavā nānantavā attā hoti arogo paraṃ maraṇā nevasaññīnāsaññī’ti naṃ paññapenti.

\paragraph{83.}
‘‘Imehi kho te, bhikkhave, samaṇabrāhmaṇā uddhamāghātanikā nevasaññīnāsaññīvādā uddhamāghātanaṃ nevasaññīnāsaññiṃ attānaṃ paññapenti aṭṭhahi vatthūhi. Ye hi keci, bhikkhave, samaṇā vā brāhmaṇā vā uddhamāghātanikā nevasaññīnāsaññīvādā uddhamāghātanaṃ nevasaññīnāsaññiṃ attānaṃ paññapenti, sabbe te imeheva aṭṭhahi vatthūhi …pe… yehi tathāgatassa yathābhuccaṃ vaṇṇaṃ sammā vadamānā vadeyyuṃ.

\subsubsection{Ucchedavādo}

\paragraph{84.}
‘‘Santi, bhikkhave, eke samaṇabrāhmaṇā ucchedavādā sato sattassa ucchedaṃ vināsaṃ vibhavaṃ paññapenti sattahi vatthūhi. Te ca bhonto samaṇabrāhmaṇā kimāgamma kimārabbha ucchedavādā sato sattassa ucchedaṃ vināsaṃ vibhavaṃ paññapenti sattahi vatthūhi?

\paragraph{85.}
‘‘Idha, bhikkhave, ekacco samaṇo vā brāhmaṇo vā evaṃvādī hoti evaṃdiṭṭhi\footnote{evaṃdiṭṭhī (ka. pī.)} – ‘yato kho, bho, ayaṃ attā rūpī cātumahābhūtiko mātāpettikasambhavo kāyassa bhedā ucchijjati vinassati, na hoti paraṃ maraṇā, ettāvatā kho, bho, ayaṃ attā sammā samucchinno hotī’ti. Ittheke sato sattassa ucchedaṃ vināsaṃ vibhavaṃ paññapenti.

\paragraph{86.}
‘‘Tamañño evamāha – ‘atthi kho, bho, eso attā, yaṃ tvaṃ vadesi, neso natthīti vadāmi; no ca kho, bho, ayaṃ attā ettāvatā sammā samucchinno hoti. Atthi kho, bho, añño attā dibbo rūpī kāmāvacaro kabaḷīkārāhārabhakkho. Taṃ tvaṃ na jānāsi na passasi. Tamahaṃ jānāmi passāmi. So kho, bho, attā yato kāyassa bhedā ucchijjati vinassati, na hoti paraṃ maraṇā, ettāvatā kho, bho, ayaṃ attā sammā samucchinno hotī’ti. Ittheke sato sattassa ucchedaṃ vināsaṃ vibhavaṃ paññapenti.

\paragraph{87.}
‘‘Tamañño evamāha – ‘atthi kho, bho, eso attā, yaṃ tvaṃ vadesi, neso natthīti vadāmi; no ca kho, bho, ayaṃ attā ettāvatā sammā samucchinno hoti. Atthi kho, bho, añño attā dibbo rūpī manomayo sabbaṅgapaccaṅgī ahīnindriyo. Taṃ tvaṃ na jānāsi na passasi. Tamahaṃ jānāmi passāmi. So kho, bho, attā yato kāyassa bhedā ucchijjati vinassati, na hoti paraṃ maraṇā, ettāvatā kho, bho, ayaṃ attā sammā samucchinno hotī’ti. Ittheke sato sattassa ucchedaṃ vināsaṃ vibhavaṃ paññapenti.

\paragraph{88.}
‘‘Tamañño evamāha – ‘atthi kho, bho, eso attā, yaṃ tvaṃ vadesi, neso natthīti vadāmi; no ca kho, bho, ayaṃ attā ettāvatā sammā samucchinno hoti. Atthi kho, bho, añño attā sabbaso rūpasaññānaṃ samatikkamā paṭighasaññānaṃ atthaṅgamā nānattasaññānaṃ amanasikārā ‘‘ananto ākāso’’ti ākāsānañcāyatanūpago. Taṃ tvaṃ na jānāsi na passasi. Tamahaṃ jānāmi passāmi. So kho, bho, attā yato kāyassa bhedā ucchijjati vinassati, na hoti paraṃ maraṇā, ettāvatā kho, bho, ayaṃ attā sammā samucchinno hotī’ti. Ittheke sato sattassa ucchedaṃ vināsaṃ vibhavaṃ paññapenti.

\paragraph{89.}
‘‘Tamañño evamāha – ‘atthi kho, bho, eso attā yaṃ tvaṃ vadesi, neso natthīti vadāmi; no ca kho, bho, ayaṃ attā ettāvatā sammā samucchinno hoti. Atthi kho, bho, añño attā sabbaso ākāsānañcāyatanaṃ samatikkamma ‘‘anantaṃ viññāṇa’’nti viññāṇañcāyatanūpago. Taṃ tvaṃ na jānāsi na passasi. Tamahaṃ jānāmi passāmi. So kho, bho, attā yato kāyassa bhedā ucchijjati vinassati, na hoti paraṃ maraṇā, ettāvatā kho, bho, ayaṃ attā sammā samucchinno hotī’ti. Ittheke sato sattassa ucchedaṃ vināsaṃ vibhavaṃ paññapenti.

\paragraph{90.}
‘‘Tamañño evamāha – ‘atthi kho, bho, so attā, yaṃ tvaṃ vadesi, neso natthīti vadāmi; no ca kho, bho, ayaṃ attā ettāvatā sammā samucchinno hoti. Atthi kho, bho, añño attā sabbaso viññāṇañcāyatanaṃ samatikkamma ‘‘natthi kiñcī’’ti ākiñcaññāyatanūpago. Taṃ tvaṃ na jānāsi na passasi. Tamahaṃ jānāmi passāmi. So kho, bho, attā yato kāyassa bhedā ucchijjati vinassati, na hoti paraṃ maraṇā, ettāvatā kho, bho, ayaṃ attā sammā samucchinno hotī’’ti. Ittheke sato sattassa ucchedaṃ vināsaṃ vibhavaṃ paññapenti.

\paragraph{91.}
‘Tamañño evamāha – ‘‘atthi kho, bho, eso attā, yaṃ tvaṃ vadesi, neso natthīti vadāmi; no ca kho, bho, ayaṃ attā ettāvatā sammā samucchinno hoti. Atthi kho, bho, añño attā sabbaso ākiñcaññāyatanaṃ samatikkamma ‘‘santametaṃ paṇītameta’’nti nevasaññānāsaññāyatanūpago. Taṃ tvaṃ na jānāsi na passasi. Tamahaṃ jānāmi passāmi. So kho, bho, attā yato kāyassa bhedā ucchijjati vinassati, na hoti paraṃ maraṇā, ettāvatā kho, bho, ayaṃ attā sammā samucchinno hotī’ti. Ittheke sato sattassa ucchedaṃ vināsaṃ vibhavaṃ paññapenti.

\paragraph{92.}
‘‘Imehi kho te, bhikkhave, samaṇabrāhmaṇā ucchedavādā sato sattassa ucchedaṃ vināsaṃ vibhavaṃ paññapenti sattahi vatthūhi. Ye hi keci, bhikkhave, samaṇā vā brāhmaṇā vā ucchedavādā sato sattassa ucchedaṃ vināsaṃ vibhavaṃ paññapenti, sabbe te imeheva sattahi vatthūhi …pe… yehi tathāgatassa yathābhuccaṃ vaṇṇaṃ sammā vadamānā vadeyyuṃ.

\subsubsection{Diṭṭhadhammanibbānavādo}

\paragraph{93.}
‘‘Santi, bhikkhave, eke samaṇabrāhmaṇā diṭṭhadhammanibbānavādā sato sattassa paramadiṭṭhadhammanibbānaṃ paññapenti pañcahi vatthūhi. Te ca bhonto samaṇabrāhmaṇā kimāgamma kimārabbha diṭṭhadhammanibbānavādā sato sattassa paramadiṭṭhadhammanibbānaṃ paññapenti pañcahi vatthūhi?

\paragraph{94.}
‘‘Idha, bhikkhave, ekacco samaṇo vā brāhmaṇo vā evaṃvādī hoti evaṃdiṭṭhi – ‘‘yato kho, bho, ayaṃ attā pañcahi kāmaguṇehi samappito samaṅgībhūto paricāreti, ettāvatā kho, bho, ayaṃ attā paramadiṭṭhadhammanibbānaṃ patto hotī’ti. Ittheke sato sattassa paramadiṭṭhadhammanibbānaṃ paññapenti.

\paragraph{95.}
‘‘Tamañño evamāha –‘atthi kho, bho, eso attā, yaṃ tvaṃ vadesi, neso natthīti vadāmi; no ca kho, bho, ayaṃ attā ettāvatā paramadiṭṭhadhammanibbānaṃ patto hoti. Taṃ kissa hetu? Kāmā hi, bho, aniccā dukkhā vipariṇāmadhammā, tesaṃ vipariṇāmaññathābhāvā uppajjanti sokaparidevadukkhadomanassupāyāsā. Yato kho, bho, ayaṃ attā vivicceva kāmehi vivicca akusalehi dhammehi savitakkaṃ savicāraṃ vivekajaṃ pītisukhaṃ paṭhamaṃ jhānaṃ upasampajja viharati, ettāvatā kho, bho, ayaṃ attā paramadiṭṭhadhammanibbānaṃ patto hotī’ti. Ittheke sato sattassa paramadiṭṭhadhammanibbānaṃ paññapenti.

\paragraph{96.}
‘‘Tamañño evamāha – ‘atthi kho, bho, eso attā, yaṃ tvaṃ vadesi, neso natthīti vadāmi; no ca kho, bho, ayaṃ attā ettāvatā paramadiṭṭhadhammanibbānaṃ patto hoti. Taṃ kissa hetu? Yadeva tattha vitakkitaṃ vicāritaṃ, etenetaṃ oḷārikaṃ akkhāyati. Yato kho, bho, ayaṃ attā vitakkavicārānaṃ vūpasamā ajjhattaṃ sampasādanaṃ cetaso ekodibhāvaṃ avitakkaṃ avicāraṃ samādhijaṃ pītisukhaṃ dutiyaṃ jhānaṃ upasampajja viharati, ettāvatā kho, bho, ayaṃ attā paramadiṭṭhadhammanibbānaṃ patto hotī’ti. Ittheke sato sattassa paramadiṭṭhadhammanibbānaṃ paññapenti.

\paragraph{97.}
‘‘Tamañño evamāha – ‘atthi kho, bho, eso attā, yaṃ tvaṃ vadesi, neso natthīti vadāmi; no ca kho, bho, ayaṃ attā ettāvatā paramadiṭṭhadhammanibbānaṃ patto hoti. Taṃ kissa hetu? Yadeva tattha pītigataṃ cetaso uppilāvitattaṃ, etenetaṃ oḷārikaṃ akkhāyati. Yato kho, bho, ayaṃ attā pītiyā ca virāgā upekkhako ca viharati, sato ca sampajāno, sukhañca kāyena paṭisaṃvedeti, yaṃ taṃ ariyā ācikkhanti ‘‘upekkhako satimā sukhavihārī’’ti, tatiyaṃ jhānaṃ upasampajja viharati, ettāvatā kho, bho, ayaṃ attā paramadiṭṭhadhammanibbānaṃ patto hotī’ti. Ittheke sato sattassa paramadiṭṭhadhammanibbānaṃ paññapenti.

\paragraph{98.}
‘‘Tamañño evamāha – ‘atthi kho, bho, eso attā, yaṃ tvaṃ vadesi, neso natthīti vadāmi; no ca kho, bho, ayaṃ attā ettāvatā paramadiṭṭhadhammanibbānaṃ patto hoti. Taṃ kissa hetu? Yadeva tattha sukhamiti cetaso ābhogo, etenetaṃ oḷārikaṃ akkhāyati. Yato kho, bho, ayaṃ attā sukhassa ca pahānā dukkhassa ca pahānā pubbeva somanassadomanassānaṃ atthaṅgamā adukkhamasukhaṃ upekkhāsatipārisuddhiṃ catutthaṃ jhānaṃ upasampajja viharati, ettāvatā kho, bho, ayaṃ attā paramadiṭṭhadhammanibbānaṃ patto hotī’ti. Ittheke sato sattassa paramadiṭṭhadhammanibbānaṃ paññapenti.

\paragraph{99.}
‘‘Imehi kho te, bhikkhave, samaṇabrāhmaṇā diṭṭhadhammanibbānavādā sato sattassa paramadiṭṭhadhammanibbānaṃ paññapenti pañcahi vatthūhi. Ye hi keci, bhikkhave, samaṇā vā brāhmaṇā vā diṭṭhadhammanibbānavādā sato sattassa paramadiṭṭhadhammanibbānaṃ paññapenti, sabbe te imeheva pañcahi vatthūhi …pe… yehi tathāgatassa yathābhuccaṃ vaṇṇaṃ sammā vadamānā vadeyyuṃ.

\paragraph{100.}
‘‘Imehi kho te, bhikkhave, samaṇabrāhmaṇā aparantakappikā aparantānudiṭṭhino aparantaṃ ārabbha anekavihitāni adhimuttipadāni abhivadanti catucattārīsāya vatthūhi. Ye hi keci, bhikkhave, samaṇā vā brāhmaṇā vā aparantakappikā aparantānudiṭṭhino aparantaṃ ārabbha anekavihitāni adhimuttipadāni abhivadanti, sabbe te imeheva catucattārīsāya vatthūhi …pe… yehi tathāgatassa yathābhuccaṃ vaṇṇaṃ sammā vadamānā vadeyyuṃ.

\paragraph{101.}
‘‘Imehi kho te, bhikkhave, samaṇabrāhmaṇā pubbantakappikā ca aparantakappikā ca pubbantāparantakappikā ca pubbantāparantānudiṭṭhino pubbantāparantaṃ ārabbha anekavihitāni adhimuttipadāni abhivadanti dvāsaṭṭhiyā vatthūhi.

\paragraph{102.}
‘‘Ye hi keci, bhikkhave, samaṇā vā brāhmaṇā vā pubbantakappikā vā aparantakappikā vā pubbantāparantakappikā vā pubbantāparantānudiṭṭhino pubbantāparantaṃ ārabbha anekavihitāni adhimuttipadāni abhivadanti, sabbe te imeheva dvāsaṭṭhiyā vatthūhi, etesaṃ vā aññatarena; natthi ito bahiddhā.

\paragraph{103.}
‘‘Tayidaṃ, bhikkhave, tathāgato pajānāti – ‘ime diṭṭhiṭṭhānā evaṃgahitā evaṃparāmaṭṭhā evaṃgatikā bhavanti evaṃabhisamparāyā’ti. Tañca tathāgato pajānāti, tato ca uttaritaraṃ pajānāti, tañca pajānanaṃ na parāmasati, aparāmasato cassa paccattaññeva nibbuti viditā. Vedanānaṃ samudayañca atthaṅgamañca assādañca ādīnavañca nissaraṇañca yathābhūtaṃ viditvā anupādāvimutto, bhikkhave, tathāgato.

\paragraph{104.}
‘‘Ime kho te, bhikkhave, dhammā gambhīrā duddasā duranubodhā santā paṇītā atakkāvacarā nipuṇā paṇḍitavedanīyā, ye tathāgato sayaṃ abhiññā sacchikatvā pavedeti, yehi tathāgatassa yathābhuccaṃ vaṇṇaṃ sammā vadamānā vadeyyuṃ.

\subsubsection{Paritassitavipphanditavāro}

\paragraph{105.}
‘‘Tatra, bhikkhave, ye te samaṇabrāhmaṇā sassatavādā sassataṃ attānañca lokañca paññapenti catūhi vatthūhi, tadapi tesaṃ bhavataṃ samaṇabrāhmaṇānaṃ ajānataṃ apassataṃ vedayitaṃ taṇhāgatānaṃ paritassitavipphanditameva.

\paragraph{106.}
‘‘Tatra, bhikkhave, ye te samaṇabrāhmaṇā ekaccasassatikā ekaccaasassatikā ekaccaṃ sassataṃ ekaccaṃ asassataṃ attānañca lokañca paññapenti catūhi vatthūhi, tadapi tesaṃ bhavataṃ samaṇabrāhmaṇānaṃ ajānataṃ apassataṃ vedayitaṃ taṇhāgatānaṃ paritassitavipphanditameva.

\paragraph{107.}
‘‘Tatra, bhikkhave, ye te samaṇabrāhmaṇā antānantikā antānantaṃ lokassa paññapenti catūhi vatthūhi, tadapi tesaṃ bhavataṃ samaṇabrāhmaṇānaṃ ajānataṃ apassataṃ vedayitaṃ taṇhāgatānaṃ paritassitavipphanditameva.

\paragraph{108.}
‘‘Tatra, bhikkhave, ye te samaṇabrāhmaṇā amarāvikkhepikā tattha tattha pañhaṃ puṭṭhā samānā vācāvikkhepaṃ āpajjanti amarāvikkhepaṃ catūhi vatthūhi, tadapi tesaṃ bhavataṃ samaṇabrāhmaṇānaṃ ajānataṃ apassataṃ vedayitaṃ taṇhāgatānaṃ paritassitavipphanditameva.

\paragraph{109.}
‘‘Tatra, bhikkhave, ye te samaṇabrāhmaṇā adhiccasamuppannikā adhiccasamuppannaṃ attānañca lokañca paññapenti dvīhi vatthūhi, tadapi tesaṃ bhavataṃ samaṇabrāhmaṇānaṃ ajānataṃ apassataṃ vedayitaṃ taṇhāgatānaṃ paritassitavipphanditameva.

\paragraph{110.}
‘‘Tatra, bhikkhave, ye te samaṇabrāhmaṇā pubbantakappikā pubbantānudiṭṭhino pubbantaṃ ārabbha anekavihitāni adhimuttipadāni abhivadanti aṭṭhārasahi vatthūhi, tadapi tesaṃ bhavataṃ samaṇabrāhmaṇānaṃ ajānataṃ apassataṃ vedayitaṃ taṇhāgatānaṃ paritassitavipphanditameva.

\paragraph{111.}
‘‘Tatra, bhikkhave, ye te samaṇabrāhmaṇā uddhamāghātanikā saññīvādā uddhamāghātanaṃ saññiṃ attānaṃ paññapenti soḷasahi vatthūhi, tadapi tesaṃ bhavataṃ samaṇabrāhmaṇānaṃ ajānataṃ apassataṃ vedayitaṃ taṇhāgatānaṃ paritassitavipphanditameva.

\paragraph{112.}
‘‘Tatra, bhikkhave, ye te samaṇabrāhmaṇā uddhamāghātanikā asaññīvādā uddhamāghātanaṃ asaññiṃ attānaṃ paññapenti aṭṭhahi vatthūhi, tadapi tesaṃ bhavataṃ samaṇabrāhmaṇānaṃ ajānataṃ apassataṃ vedayitaṃ taṇhāgatānaṃ paritassitavipphanditameva.

\paragraph{113.}
‘‘Tatra, bhikkhave, ye te samaṇabrāhmaṇā uddhamāghātanikā nevasaññīnāsaññīvādā uddhamāghātanaṃ nevasaññīnāsaññiṃ attānaṃ paññapenti aṭṭhahi vatthūhi, tadapi tesaṃ bhavataṃ samaṇabrāhmaṇānaṃ ajānataṃ apassataṃ vedayitaṃ taṇhāgatānaṃ paritassitavipphanditameva.

\paragraph{114.}
‘‘Tatra, bhikkhave, ye te samaṇabrāhmaṇā ucchedavādā sato sattassa ucchedaṃ vināsaṃ vibhavaṃ paññapenti sattahi vatthūhi, tadapi tesaṃ bhavataṃ samaṇabrāhmaṇānaṃ ajānataṃ apassataṃ vedayitaṃ taṇhāgatānaṃ paritassitavipphanditameva.

\paragraph{115.}
‘‘Tatra, bhikkhave, ye te samaṇabrāhmaṇā diṭṭhadhammanibbānavādā sato sattassa paramadiṭṭhadhammanibbānaṃ paññapenti pañcahi vatthūhi, tadapi tesaṃ bhavataṃ samaṇabrāhmaṇānaṃ ajānataṃ apassataṃ vedayitaṃ taṇhāgatānaṃ paritassitavipphanditameva.

\paragraph{116.}
‘‘Tatra, bhikkhave, ye te samaṇabrāhmaṇā aparantakappikā aparantānudiṭṭhino aparantaṃ ārabbha anekavihitāni adhimuttipadāni abhivadanti catucattārīsāya vatthūhi, tadapi tesaṃ bhavataṃ samaṇabrāhmaṇānaṃ ajānataṃ apassataṃ vedayitaṃ taṇhāgatānaṃ paritassitavipphanditameva.

\paragraph{117.}
‘‘Tatra, bhikkhave, ye te samaṇabrāhmaṇā pubbantakappikā ca aparantakappikā ca pubbantāparantakappikā ca pubbantāparantānudiṭṭhino pubbantāparantaṃ ārabbha anekavihitāni adhimuttipadāni abhivadanti dvāsaṭṭhiyā vatthūhi, tadapi tesaṃ bhavataṃ samaṇabrāhmaṇānaṃ ajānataṃ apassataṃ vedayitaṃ taṇhāgatānaṃ paritassitavipphanditameva.

\subsubsection{Phassapaccayāvāro}

\paragraph{118.}
‘‘Tatra, bhikkhave, ye te samaṇabrāhmaṇā sassatavādā sassataṃ attānañca lokañca paññapenti catūhi vatthūhi, tadapi phassapaccayā.

\paragraph{119.}
‘‘Tatra, bhikkhave, ye te samaṇabrāhmaṇā ekaccasassatikā ekaccaasassatikā ekaccaṃ sassataṃ ekaccaṃ asassataṃ attānañca lokañca paññapenti catūhi vatthūhi, tadapi phassapaccayā.

\paragraph{120.}
‘‘Tatra, bhikkhave, ye te samaṇabrāhmaṇā antānantikā antānantaṃ lokassa paññapenti catūhi vatthūhi, tadapi phassapaccayā.

\paragraph{121.}
‘‘Tatra, bhikkhave, ye te samaṇabrāhmaṇā amarāvikkhepikā tattha tattha pañhaṃ puṭṭhā samānā vācāvikkhepaṃ āpajjanti amarāvikkhepaṃ catūhi vatthūhi, tadapi phassapaccayā.

\paragraph{122.}
‘‘Tatra, bhikkhave, ye te samaṇabrāhmaṇā adhiccasamuppannikā adhiccasamuppannaṃ attānañca lokañca paññapenti dvīhi vatthūhi, tadapi phassapaccayā.

\paragraph{123.}
‘‘Tatra, bhikkhave, ye te samaṇabrāhmaṇā pubbantakappikā pubbantānudiṭṭhino pubbantaṃ ārabbha anekavihitāni adhimuttipadāni abhivadanti aṭṭhārasahi vatthūhi, tadapi phassapaccayā.

\paragraph{124.}
‘‘Tatra, bhikkhave, ye te samaṇabrāhmaṇā uddhamāghātanikā saññīvādā uddhamāghātanaṃ saññiṃ attānaṃ paññapenti soḷasahi vatthūhi, tadapi phassapaccayā.

\paragraph{125.}
‘‘Tatra, bhikkhave, ye te samaṇabrāhmaṇā uddhamāghātanikā asaññīvādā uddhamāghātanaṃ asaññiṃ attānaṃ paññapenti aṭṭhahi vatthūhi, tadapi phassapaccayā.

\paragraph{126.}
‘‘Tatra, bhikkhave, ye te samaṇabrāhmaṇā uddhamāghātanikā nevasaññīnāsaññīvādā uddhamāghātanaṃ nevasaññīnāsaññiṃ attānaṃ paññapenti aṭṭhahi vatthūhi, tadapi phassapaccayā.

\paragraph{127.}
‘‘Tatra, bhikkhave, ye te samaṇabrāhmaṇā ucchedavādā sato sattassa ucchedaṃ vināsaṃ vibhavaṃ paññapenti sattahi vatthūhi, tadapi phassapaccayā.

\paragraph{128.}
‘‘Tatra, bhikkhave, ye te samaṇabrāhmaṇā diṭṭhadhammanibbānavādā sato sattassa paramadiṭṭhadhammanibbānaṃ paññapenti pañcahi vatthūhi, tadapi phassapaccayā.

\paragraph{129.}
‘‘Tatra, bhikkhave, ye te samaṇabrāhmaṇā aparantakappikā aparantānudiṭṭhino aparantaṃ ārabbha anekavihitāni adhimuttipadāni abhivadanti catucattārīsāya vatthūhi, tadapi phassapaccayā.

\paragraph{130.}
‘‘Tatra, bhikkhave, ye te samaṇabrāhmaṇā pubbantakappikā ca aparantakappikā ca pubbantāparantakappikā ca pubbantāparantānudiṭṭhino pubbantāparantaṃ ārabbha anekavihitāni adhimuttipadāni abhivadanti dvāsaṭṭhiyā vatthūhi, tadapi phassapaccayā.

\subsubsection{Netaṃ ṭhānaṃ vijjativāro}

\paragraph{131.}
‘‘Tatra, bhikkhave, ye te samaṇabrāhmaṇā sassatavādā sassataṃ attānañca lokañca paññapenti catūhi vatthūhi, te vata aññatra phassā paṭisaṃvedissantīti netaṃ ṭhānaṃ vijjati.

\paragraph{132.}
‘‘Tatra, bhikkhave, ye te samaṇabrāhmaṇā ekaccasassatikā ekacca asassatikā ekaccaṃ sassataṃ ekaccaṃ asassataṃ attānañca lokañca paññapenti catūhi vatthūhi, te vata aññatra phassā paṭisaṃvedissantīti netaṃ ṭhānaṃ vijjati.

\paragraph{133.}
‘‘Tatra, bhikkhave, ye te samaṇabrāhmaṇā antānantikā antānantaṃ lokassa paññapenti catūhi vatthūhi, te vata aññatra phassā paṭisaṃvedissantīti netaṃ ṭhānaṃ vijjati.

\paragraph{134.}
‘‘Tatra, bhikkhave, ye te samaṇabrāhmaṇā amarāvikkhepikā tattha tattha pañhaṃ puṭṭhā samānā vācāvikkhepaṃ āpajjanti amarāvikkhepaṃ catūhi vatthūhi, te vata aññatra phassā paṭisaṃvedissantīti netaṃ ṭhānaṃ vijjati.

\paragraph{135.}
‘‘Tatra, bhikkhave, ye te samaṇabrāhmaṇā adhiccasamuppannikā adhiccasamuppannaṃ attānañca lokañca paññapenti dvīhi vatthūhi, te vata aññatra phassā paṭisaṃvedissantīti netaṃ ṭhānaṃ vijjati.

\paragraph{136.}
‘‘Tatra, bhikkhave, ye te samaṇabrāhmaṇā pubbantakappikā pubbantānudiṭṭhino pubbantaṃ ārabbha anekavihitāni adhimuttipadāni abhivadanti aṭṭhārasahi vatthūhi, te vata aññatra phassā paṭisaṃvedissantīti netaṃ ṭhānaṃ vijjati.

\paragraph{137.}
‘‘Tatra, bhikkhave, ye te samaṇabrāhmaṇā uddhamāghātanikā saññīvādā uddhamāghātanaṃ saññiṃ attānaṃ paññapenti soḷasahi vatthūhi, te vata aññatra phassā paṭisaṃvedissantīti netaṃ ṭhānaṃ vijjati.

\paragraph{138.}
‘‘Tatra, bhikkhave, ye te samaṇabrāhmaṇā uddhamāghātanikā asaññīvādā, uddhamāghātanaṃ asaññiṃ attānaṃ paññapenti aṭṭhahi vatthūhi, te vata aññatra phassā paṭisaṃvedissantīti netaṃ ṭhānaṃ vijjati.

\paragraph{139.}
‘‘Tatra, bhikkhave, ye te samaṇabrāhmaṇā uddhamāghātanikā nevasaññīnāsaññīvādā uddhamāghātanaṃ nevasaññīnāsaññiṃ attānaṃ paññapenti aṭṭhahi vatthūhi, te vata aññatra phassā paṭisaṃvedissantīti netaṃ ṭhānaṃ vijjati.

\paragraph{140.}
‘‘Tatra, bhikkhave, ye te samaṇabrāhmaṇā ucchedavādā sato sattassa ucchedaṃ vināsaṃ vibhavaṃ paññapenti sattahi vatthūhi, te vata aññatra phassā paṭisaṃvedissantīti netaṃ ṭhānaṃ vijjati.

\paragraph{141.}
‘‘Tatra, bhikkhave, ye te samaṇabrāhmaṇā diṭṭhadhammanibbānavādā sato sattassa paramadiṭṭhadhammanibbānaṃ paññapenti pañcahi vatthūhi, te vata aññatra phassā paṭisaṃvedissantīti netaṃ ṭhānaṃ vijjati.

\paragraph{142.}
‘‘Tatra, bhikkhave, ye te samaṇabrāhmaṇā aparantakappikā aparantānudiṭṭhino aparantaṃ ārabbha anekavihitāni adhimuttipadāni abhivadanti catucattārīsāya vatthūhi, te vata aññatra phassā paṭisaṃvedissantīti netaṃ ṭhānaṃ vijjati.

\paragraph{143.}
‘‘Tatra, bhikkhave, ye te samaṇabrāhmaṇā pubbantakappikā ca aparantakappikā ca pubbantāparantakappikā ca pubbantāparantānudiṭṭhino pubbantāparantaṃ ārabbha anekavihitāni adhimuttipadāni abhivadanti dvāsaṭṭhiyā vatthūhi, te vata aññatra phassā paṭisaṃvedissantīti netaṃ ṭhānaṃ vijjati.

\subsubsection{Diṭṭhigatikādhiṭṭhānavaṭṭakathā}

\paragraph{144.}
‘‘Tatra, bhikkhave, ye te samaṇabrāhmaṇā sassatavādā sassataṃ attānañca lokañca paññapenti catūhi vatthūhi, yepi te samaṇabrāhmaṇā ekaccasassatikā ekaccaasassatikā …pe… yepi te samaṇabrāhmaṇā antānantikā… yepi te samaṇabrāhmaṇā amarāvikkhepikā… yepi te samaṇabrāhmaṇā adhiccasamuppannikā… yepi te samaṇabrāhmaṇā pubbantakappikā… yepi te samaṇabrāhmaṇā uddhamāghātanikā saññīvādā… yepi te samaṇabrāhmaṇā uddhamāghātanikā asaññīvādā… yepi te samaṇabrāhmaṇā uddhamāghātanikā nevasaññīnāsaññīvādā… yepi te samaṇabrāhmaṇā ucchedavādā… yepi te samaṇabrāhmaṇā diṭṭhadhammanibbānavādā… yepi te samaṇabrāhmaṇā aparantakappikā… yepi te samaṇabrāhmaṇā pubbantakappikā ca aparantakappikā ca pubbantāparantakappikā ca pubbantāparantānudiṭṭhino pubbantāparantaṃ ārabbha anekavihitāni adhimuttipadāni abhivadanti dvāsaṭṭhiyā vatthūhi, sabbe te chahi phassāyatanehi phussa phussa paṭisaṃvedenti tesaṃ vedanāpaccayā taṇhā, taṇhāpaccayā upādānaṃ, upādānapaccayā bhavo, bhavapaccayā jāti, jātipaccayā jarāmaraṇaṃ sokaparidevadukkhadomanassupāyāsā sambhavanti.

\subsubsection{Vivaṭṭakathādi}

\paragraph{145.}
‘‘Yato kho, bhikkhave, bhikkhu channaṃ phassāyatanānaṃ samudayañca atthaṅgamañca assādañca ādīnavañca nissaraṇañca yathābhūtaṃ pajānāti, ayaṃ imehi sabbeheva uttaritaraṃ pajānāti.

\paragraph{146.}
‘‘Ye hi keci, bhikkhave, samaṇā vā brāhmaṇā vā pubbantakappikā vā aparantakappikā vā pubbantāparantakappikā vā pubbantāparantānudiṭṭhino pubbantāparantaṃ ārabbha anekavihitāni adhimuttipadāni abhivadanti, sabbe te imeheva dvāsaṭṭhiyā vatthūhi antojālīkatā, ettha sitāva ummujjamānā ummujjanti, ettha pariyāpannā antojālīkatāva ummujjamānā ummujjanti. ‘‘Seyyathāpi, bhikkhave, dakkho kevaṭṭo vā kevaṭṭantevāsī vā sukhumacchikena jālena parittaṃ udakadahaṃ\footnote{udakarahadaṃ (sī. syā. pī.)} otthareyya. Tassa evamassa – ‘ye kho keci imasmiṃ udakadahe oḷārikā pāṇā, sabbe te antojālīkatā. Ettha sitāva ummujjamānā ummujjanti; ettha pariyāpannā antojālīkatāva ummujjamānā ummujjantī’ti; evameva kho, bhikkhave, ye hi keci samaṇā vā brāhmaṇā vā pubbantakappikā vā aparantakappikā vā pubbantāparantakappikā vā pubbantāparantānudiṭṭhino pubbantāparantaṃ ārabbha anekavihitāni adhimuttipadāni abhivadanti, sabbe te imeheva dvāsaṭṭhiyā vatthūhi antojālīkatā ettha sitāva ummujjamānā ummujjanti, ettha pariyāpannā antojālīkatāva ummujjamānā ummujjanti.

\paragraph{147.}
‘‘Ucchinnabhavanettiko, bhikkhave, tathāgatassa kāyo tiṭṭhati. Yāvassa kāyo ṭhassati, tāva naṃ dakkhanti devamanussā. Kāyassa bhedā uddhaṃ jīvitapariyādānā na naṃ dakkhanti devamanussā. ‘‘Seyyathāpi, bhikkhave, ambapiṇḍiyā vaṇṭacchinnāya yāni kānici ambāni vaṇṭapaṭibandhāni\footnote{vaṇṭūpanibandhanāni (sī. pī.), vaṇḍapaṭibaddhāni (ka.)}, sabbāni tāni tadanvayāni bhavanti; evameva kho, bhikkhave, ucchinnabhavanettiko tathāgatassa kāyo tiṭṭhati, yāvassa kāyo ṭhassati, tāva naṃ dakkhanti devamanussā, kāyassa bhedā uddhaṃ jīvitapariyādānā na naṃ dakkhanti devamanussā’’ti.

\paragraph{148.}
Evaṃ vutte āyasmā ānando bhagavantaṃ etadavoca – ‘‘acchariyaṃ, bhante, abbhutaṃ, bhante, ko nāmo ayaṃ, bhante, dhammapariyāyo’’ti? ‘‘Tasmātiha tvaṃ, ānanda, imaṃ dhammapariyāyaṃ atthajālantipi naṃ dhārehi, dhammajālantipi naṃ dhārehi, brahmajālantipi naṃ dhārehi, diṭṭhijālantipi naṃ dhārehi, anuttaro saṅgāmavijayotipi naṃ dhārehī’’ti. Idamavoca bhagavā.

\paragraph{149.}
Attamanā te bhikkhū bhagavato bhāsitaṃ abhinandunti. Imasmiñca pana veyyākaraṇasmiṃ bhaññamāne dasasahassī\footnote{sahassī (katthaci)} lokadhātu akampitthāti.

\xsectionEnd{Brahmajālasuttaṃ niṭṭhitaṃ paṭhamaṃ.}


\clearpage
\section{Sāmaññaphalasuttaṃ}

\subsubsection{Rājāmaccakathā}

\paragraph{150.}
Evaṃ me sutaṃ – ekaṃ samayaṃ bhagavā rājagahe viharati jīvakassa komārabhaccassa ambavane mahatā bhikkhusaṅghena saddhiṃ aḍḍhateḷasehi bhikkhusatehi. Tena kho pana samayena rājā māgadho ajātasattu vedehiputto tadahuposathe pannarase komudiyā cātumāsiniyā puṇṇāya puṇṇamāya rattiyā rājāmaccaparivuto uparipāsādavaragato nisinno hoti. Atha kho rājā māgadho ajātasattu vedehiputto tadahuposathe udānaṃ udānesi – ‘‘ramaṇīyā vata bho dosinā ratti, abhirūpā vata bho dosinā ratti, dassanīyā vata bho dosinā ratti, pāsādikā vata bho dosinā ratti, lakkhaññā vata bho dosinā ratti. Kaṃ nu khvajja samaṇaṃ vā brāhmaṇaṃ vā payirupāseyyāma, yaṃ no payirupāsato cittaṃ pasīdeyyā’’ti?

\paragraph{151.}
Evaṃ vutte, aññataro rājāmacco rājānaṃ māgadhaṃ ajātasattuṃ vedehiputtaṃ etadavoca – ‘‘ayaṃ, deva, pūraṇo kassapo saṅghī ceva gaṇī ca gaṇācariyo ca ñāto yasassī titthakaro sādhusammato bahujanassa rattaññū cirapabbajito addhagato vayoanuppatto. Taṃ devo pūraṇaṃ kassapaṃ payirupāsatu. Appeva nāma devassa pūraṇaṃ kassapaṃ payirupāsato cittaṃ pasīdeyyā’’ti. Evaṃ vutte, rājā māgadho ajātasattu vedehiputto tuṇhī ahosi.

\paragraph{152.}
Aññataropi kho rājāmacco rājānaṃ māgadhaṃ ajātasattuṃ vedehiputtaṃ etadavoca – ‘‘ayaṃ, deva, makkhali gosālo saṅghī ceva gaṇī ca gaṇācariyo ca ñāto yasassī titthakaro sādhusammato bahujanassa rattaññū cirapabbajito addhagato vayoanuppatto. Taṃ devo makkhaliṃ gosālaṃ payirupāsatu. Appeva nāma devassa makkhaliṃ gosālaṃ payirupāsato cittaṃ pasīdeyyā’’ti. Evaṃ vutte, rājā māgadho ajātasattu vedehiputto tuṇhī ahosi.

\paragraph{153.} Aññataropi kho rājāmacco rājānaṃ māgadhaṃ ajātasattuṃ vedehiputtaṃ etadavoca – ‘‘ayaṃ, deva, ajito kesakambalo saṅghī ceva gaṇī ca gaṇācariyo ca ñāto yasassī titthakaro sādhusammato bahujanassa rattaññū cirapabbajito addhagato vayoanuppatto. Taṃ devo ajitaṃ kesakambalaṃ payirupāsatu. Appeva nāma devassa ajitaṃ kesakambalaṃ payirupāsato cittaṃ pasīdeyyā’’ti. Evaṃ vutte, rājā māgadho ajātasattu vedehiputto tuṇhī ahosi.

\paragraph{154.} Aññataropi kho rājāmacco rājānaṃ māgadhaṃ ajātasattuṃ vedehiputtaṃ etadavoca – ‘‘ayaṃ, deva, pakudho\footnote{pakuddho (sī.)} kaccāyano saṅghī ceva gaṇī ca gaṇācariyo ca ñāto yasassī titthakaro sādhusammato bahujanassa rattaññū cirapabbajito addhagato vayoanuppatto. Taṃ devo pakudhaṃ kaccāyanaṃ payirupāsatu. Appeva nāma devassa pakudhaṃ kaccāyanaṃ payirupāsato cittaṃ pasīdeyyā’’ti. Evaṃ vutte, rājā māgadho ajātasattu vedehiputto tuṇhī ahosi.

\paragraph{155.} Aññataropi kho rājāmacco rājānaṃ māgadhaṃ ajātasattuṃ vedehiputtaṃ etadavoca – ‘‘ayaṃ, deva, sañcayo\footnote{sañjayo (sī. syā.)} belaṭṭhaputto\footnote{bellaṭṭhiputto (sī.), velaṭṭhaputto (syā.)} saṅghī ceva gaṇī ca gaṇācariyo ca ñāto yasassī titthakaro sādhusammato bahujanassa rattaññū cirapabbajito addhagato vayoanuppatto. Taṃ devo sañcayaṃ belaṭṭhaputtaṃ payirupāsatu. Appeva nāma devassa sañcayaṃ belaṭṭhaputtaṃ payirupāsato cittaṃ pasīdeyyā’’ti. Evaṃ vutte, rājā māgadho ajātasattu vedehiputto tuṇhī ahosi.

\paragraph{156.} Aññataropi kho rājāmacco rājānaṃ māgadhaṃ ajātasattuṃ vedehiputtaṃ etadavoca – ‘‘ayaṃ, deva, nigaṇṭho nāṭaputto\footnote{nāthaputto (sī.), nātaputto (pī.)} saṅghī ceva gaṇī ca gaṇācariyo ca ñāto yasassī titthakaro sādhusammato bahujanassa rattaññū cirapabbajito addhagato vayoanuppatto. Taṃ devo nigaṇṭhaṃ nāṭaputtaṃ payirupāsatu. Appeva nāma devassa nigaṇṭhaṃ nāṭaputtaṃ payirupāsato cittaṃ pasīdeyyā’’ti. Evaṃ vutte, rājā māgadho ajātasattu vedehiputto tuṇhī ahosi.

\subsubsection{Komārabhaccajīvakakathā}

\paragraph{157.} Tena kho pana samayena jīvako komārabhacco rañño māgadhassa ajātasattussa vedehiputtassa avidūre tuṇhībhūto nisinno hoti. Atha kho rājā māgadho ajātasattu vedehiputto jīvakaṃ komārabhaccaṃ etadavoca – ‘‘tvaṃ pana, samma jīvaka, kiṃ tuṇhī’’ti? ‘‘Ayaṃ, deva, bhagavā arahaṃ sammāsambuddho amhākaṃ ambavane viharati mahatā bhikkhusaṅghena saddhiṃ aḍḍhateḷasehi bhikkhusatehi. Taṃ kho pana bhagavantaṃ\footnote{bhagavantaṃ gotamaṃ (sī. ka. pī.)} evaṃ kalyāṇo kittisaddo abbhuggato – ‘itipi so bhagavā arahaṃ sammāsambuddho vijjācaraṇasampanno sugato lokavidū anuttaro purisadammasārathi satthā devamanussānaṃ buddho bhagavā’ti. Taṃ devo bhagavantaṃ payirupāsatu. Appeva nāma devassa bhagavantaṃ payirupāsato cittaṃ pasīdeyyā’ti.

\paragraph{158.} ‘‘Tena hi, samma jīvaka, hatthiyānāni kappāpehī’’ti. ‘‘Evaṃ, devā’’ti kho jīvako komārabhacco rañño māgadhassa ajātasattussa vedehiputtassa paṭissuṇitvā pañcamattāni hatthinikāsatāni kappāpetvā rañño ca ārohaṇīyaṃ nāgaṃ, rañño māgadhassa ajātasattussa vedehiputtassa paṭivedesi – ‘‘kappitāni kho te, deva, hatthiyānāni, yassadāni kālaṃ maññasī’’ti.

\paragraph{159.} Atha kho rājā māgadho ajātasattu vedehiputto pañcasu hatthinikāsatesu paccekā itthiyo āropetvā ārohaṇīyaṃ nāgaṃ abhiruhitvā ukkāsu dhāriyamānāsu rājagahamhā niyyāsi mahaccarājānubhāvena, yena jīvakassa komārabhaccassa ambavanaṃ tena pāyāsi. Atha kho rañño māgadhassa ajātasattussa vedehiputtassa avidūre ambavanassa ahudeva bhayaṃ, ahu chambhitattaṃ, ahu lomahaṃso. Atha kho rājā māgadho ajātasattu vedehiputto bhīto saṃviggo lomahaṭṭhajāto jīvakaṃ komārabhaccaṃ etadavoca – ‘‘kacci maṃ, samma jīvaka, na vañcesi? Kacci maṃ, samma jīvaka, na palambhesi? Kacci maṃ, samma jīvaka, na paccatthikānaṃ desi? Kathañhi nāma tāva mahato bhikkhusaṅghassa aḍḍhateḷasānaṃ bhikkhusatānaṃ neva khipitasaddo bhavissati, na ukkāsitasaddo na nigghoso’’ti. ‘‘Mā bhāyi, mahārāja, mā bhāyi, mahārāja. Na taṃ deva, vañcemi; na taṃ, deva, palambhāmi; na taṃ, deva, paccatthikānaṃ demi. Abhikkama, mahārāja, abhikkama, mahārāja, ete maṇḍalamāḷe dīpā\footnote{padīpā (sī. syā.)} jhāyantī’’ti.

\subsubsection{Sāmaññaphalapucchā}

\paragraph{160.} Atha kho rājā māgadho ajātasattu vedehiputto yāvatikā nāgassa bhūmi nāgena gantvā, nāgā paccorohitvā, pattikova\footnote{padikova (syā.)} yena maṇḍalamāḷassa dvāraṃ tenupasaṅkami; upasaṅkamitvā jīvakaṃ komārabhaccaṃ etadavoca – ‘‘kahaṃ pana, samma jīvaka, bhagavā’’ti? ‘‘Eso, mahārāja, bhagavā; eso, mahārāja, bhagavā majjhimaṃ thambhaṃ nissāya puratthābhimukho nisinno purakkhato bhikkhusaṅghassā’’ti.

\paragraph{161.} Atha kho rājā māgadho ajātasattu vedehiputto yena bhagavā tenupasaṅkami; upasaṅkamitvā ekamantaṃ aṭṭhāsi. Ekamantaṃ ṭhito kho rājā māgadho ajātasattu vedehiputto tuṇhībhūtaṃ tuṇhībhūtaṃ bhikkhusaṅghaṃ anuviloketvā rahadamiva vippasannaṃ udānaṃ udānesi – ‘‘iminā me upasamena udayabhaddo\footnote{udāyibhaddo (sī. pī.)} kumāro samannāgato hotu, yenetarahi upasamena bhikkhusaṅgho samannāgato’’ti. ‘‘Agamā kho tvaṃ, mahārāja, yathāpema’’nti. ‘‘Piyo me, bhante, udayabhaddo kumāro. Iminā me, bhante, upasamena udayabhaddo kumāro samannāgato hotu yenetarahi upasamena bhikkhusaṅgho samannāgato’’ti.

\paragraph{162.} Atha kho rājā māgadho ajātasattu vedehiputto bhagavantaṃ abhivādetvā, bhikkhusaṅghassa añjaliṃ paṇāmetvā, ekamantaṃ nisīdi. Ekamantaṃ nisinno kho rājā māgadho ajātasattu vedehiputto bhagavantaṃ etadavoca – ‘‘puccheyyāmahaṃ, bhante, bhagavantaṃ kiñcideva desaṃ\footnote{kiñcideva desaṃ lesamattaṃ (syā. kaṃ. ka.)}; sace me bhagavā okāsaṃ karoti pañhassa veyyākaraṇāyā’’ti. ‘‘Puccha, mahārāja, yadākaṅkhasī’’ti.

\paragraph{163.} ‘‘Yathā nu kho imāni, bhante, puthusippāyatanāni, seyyathidaṃ – hatthārohā assārohā rathikā dhanuggahā celakā calakā piṇḍadāyakā uggā rājaputtā pakkhandino mahānāgā sūrā cammayodhino dāsikaputtā āḷārikā kappakā nhāpakā\footnote{nahāpikā (sī.), nhāpikā (syā.)} sūdā mālākārā rajakā pesakārā naḷakārā kumbhakārā gaṇakā muddikā, yāni vā panaññānipi evaṃgatāni puthusippāyatanāni, te diṭṭheva dhamme sandiṭṭhikaṃ sippaphalaṃ upajīvanti; te tena attānaṃ sukhenti pīṇenti\footnote{pīnenti (katthaci)}, mātāpitaro sukhenti pīṇenti, puttadāraṃ sukhenti pīṇenti, mittāmacce sukhenti pīṇenti, samaṇabrāhmaṇesu\footnote{samaṇesu brāhmaṇesu (ka.)} uddhaggikaṃ dakkhiṇaṃ patiṭṭhapenti sovaggikaṃ sukhavipākaṃ saggasaṃvattanikaṃ. Sakkā nu kho, bhante, evameva diṭṭheva dhamme sandiṭṭhikaṃ sāmaññaphalaṃ paññapetu’’nti?

\paragraph{164.} ‘‘Abhijānāsi no tvaṃ, mahārāja, imaṃ pañhaṃ aññe samaṇabrāhmaṇe pucchitā’’ti? ‘‘Abhijānāmahaṃ, bhante, imaṃ pañhaṃ aññe samaṇabrāhmaṇe pucchitā’’ti. ‘‘Yathā kathaṃ pana te, mahārāja, byākariṃsu, sace te agaru bhāsassū’’ti. ‘‘Na kho me, bhante, garu, yatthassa bhagavā nisinno, bhagavantarūpo vā’’ti\footnote{cāti (sī. ka.)}. ‘‘Tena hi, mahārāja, bhāsassū’’ti.

\subsubsection{Pūraṇakassapavādo}

\paragraph{165.} ‘‘Ekamidāhaṃ, bhante, samayaṃ yena pūraṇo kassapo tenupasaṅkami; upasaṅkamitvā pūraṇena kassapena saddhiṃ sammodiṃ. Sammodanīyaṃ kathaṃ sāraṇīyaṃ vītisāretvā ekamantaṃ nisīdiṃ. Ekamantaṃ nisinno kho ahaṃ, bhante, pūraṇaṃ kassapaṃ etadavocaṃ – ‘yathā nu kho imāni, bho kassapa, puthusippāyatanāni, seyyathidaṃ – hatthārohā assārohā rathikā dhanuggahā celakā calakā piṇḍadāyakā uggā rājaputtā pakkhandino mahānāgā sūrā cammayodhino dāsikaputtā āḷārikā kappakā nhāpakā sūdā mālākārā rajakā pesakārā naḷakārā kumbhakārā gaṇakā muddikā, yāni vā panaññānipi evaṃgatāni puthusippāyatanāni- te diṭṭheva dhamme sandiṭṭhikaṃ sippaphalaṃ upajīvanti; te tena attānaṃ sukhenti pīṇenti, mātāpitaro sukhenti pīṇenti, puttadāraṃ sukhenti pīṇenti, mittāmacce sukhenti pīṇenti, samaṇabrāhmaṇesu uddhaggikaṃ dakkhiṇaṃ patiṭṭhapenti sovaggikaṃ sukhavipākaṃ saggasaṃvattanikaṃ. Sakkā nu kho, bho kassapa, evameva diṭṭheva dhamme sandiṭṭhikaṃ sāmaññaphalaṃ paññapetu’nti?

\paragraph{166.} ‘‘Evaṃ vutte, bhante, pūraṇo kassapo maṃ etadavoca – ‘karoto kho, mahārāja, kārayato, chindato chedāpayato, pacato pācāpayato socayato, socāpayato, kilamato kilamāpayato, phandato phandāpayato, pāṇamatipātāpayato, adinnaṃ ādiyato, sandhiṃ chindato, nillopaṃ harato, ekāgārikaṃ karoto, paripanthe tiṭṭhato, paradāraṃ gacchato, musā bhaṇato, karoto na karīyati pāpaṃ. Khurapariyantena cepi cakkena yo imissā pathaviyā pāṇe ekaṃ maṃsakhalaṃ ekaṃ maṃsapuñjaṃ kareyya, natthi tatonidānaṃ pāpaṃ, natthi pāpassa āgamo. Dakkhiṇaṃ cepi gaṅgāya tīraṃ gaccheyya hananto ghātento chindanto chedāpento pacanto pācāpento, natthi tatonidānaṃ pāpaṃ, natthi pāpassa āgamo. Uttarañcepi gaṅgāya tīraṃ gaccheyya dadanto dāpento yajanto yajāpento, natthi tatonidānaṃ puññaṃ, natthi puññassa āgamo. Dānena damena saṃyamena saccavajjena natthi puññaṃ, natthi puññassa āgamo’ti. Itthaṃ kho me, bhante, pūraṇo kassapo sandiṭṭhikaṃ sāmaññaphalaṃ puṭṭho samāno akiriyaṃ byākāsi. ‘‘Seyyathāpi, bhante, ambaṃ vā puṭṭho labujaṃ byākareyya, labujaṃ vā puṭṭho ambaṃ byākareyya; evameva kho me, bhante, pūraṇo kassapo sandiṭṭhikaṃ sāmaññaphalaṃ puṭṭho samāno akiriyaṃ byākāsi. Tassa mayhaṃ, bhante, etadahosi – ‘kathañhi nāma mādiso samaṇaṃ vā brāhmaṇaṃ vā vijite vasantaṃ apasādetabbaṃ maññeyyā’ti. So kho ahaṃ, bhante, pūraṇassa kassapassa bhāsitaṃ neva abhinandiṃ nappaṭikkosiṃ. Anabhinanditvā appaṭikositvā anattamano, anattamanavācaṃ anicchāretvā, tameva vācaṃ anuggaṇhanto anikkujjanto\footnote{anikkujjento (syā. kaṃ. ka.)} uṭṭhāyāsanā pakkamiṃ\footnote{pakkāmiṃ (sī. syā. kaṃ. pī.)}.

\subsubsection{Makkhaligosālavādo}

\paragraph{167.} ‘‘Ekamidāhaṃ, bhante, samayaṃ yena makkhali gosālo tenupasaṅkamiṃ; upasaṅkamitvā makkhalinā gosālena saddhiṃ sammodiṃ. Sammodanīyaṃ kathaṃ sāraṇīyaṃ vītisāretvā ekamantaṃ nisīdiṃ. Ekamantaṃ nisinno kho ahaṃ, bhante, makkhaliṃ gosālaṃ etadavocaṃ – ‘yathā nu kho imāni, bho gosāla, puthusippāyatanāni …pe… sakkā nu kho, bho gosāla, evameva diṭṭheva dhamme sandiṭṭhikaṃ sāmaññaphalaṃ paññapetu’nti?

\paragraph{168.} ‘‘Evaṃ vutte, bhante, makkhali gosālo maṃ etadavoca – ‘natthi mahārāja hetu natthi paccayo sattānaṃ saṃkilesāya, ahetū\footnote{ahetu (katthaci)} apaccayā sattā saṃkilissanti. Natthi hetu, natthi paccayo sattānaṃ visuddhiyā, ahetū apaccayā sattā visujjhanti. Natthi attakāre, natthi parakāre, natthi purisakāre, natthi balaṃ, natthi vīriyaṃ, natthi purisathāmo, natthi purisaparakkamo. Sabbe sattā sabbe pāṇā sabbe bhūtā sabbe jīvā avasā abalā avīriyā niyatisaṅgatibhāvapariṇatā chasvevābhijātīsu sukhadukkhaṃ\footnote{sukhañca dukkhañca (syā.)} paṭisaṃvedenti. Cuddasa kho panimāni yonipamukhasatasahassāni saṭṭhi ca satāni cha ca satāni pañca ca kammuno satāni pañca ca kammāni tīṇi ca kammāni kamme ca aḍḍhakamme ca dvaṭṭhipaṭipadā dvaṭṭhantarakappā chaḷābhijātiyo aṭṭha purisabhūmiyo ekūnapaññāsa ājīvakasate ekūnapaññāsa paribbājakasate ekūnapaññāsa nāgāvāsasate vīse indriyasate tiṃse nirayasate chattiṃsa rajodhātuyo satta saññīgabbhā satta asaññīgabbhā satta nigaṇṭhigabbhā satta devā satta mānusā satta pisācā satta sarā satta pavuṭā\footnote{sapuṭā (ka.), pabuṭā (sī.)} satta pavuṭasatāni satta papātā satta papātasatāni satta supinā satta supinasatāni cullāsīti mahākappino\footnote{mahākappuno (ka. sī. pī.)} satasahassāni, yāni bāle ca paṇḍite ca sandhāvitvā saṃsaritvā dukkhassantaṃ karissanti. Tattha natthi ‘‘imināhaṃ sīlena vā vatena vā tapena vā brahmacariyena vā aparipakkaṃ vā kammaṃ paripācessāmi, paripakkaṃ vā kammaṃ phussa phussa byantiṃ karissāmī’ti hevaṃ natthi. Doṇamite sukhadukkhe pariyantakate saṃsāre, natthi hāyanavaḍḍhane, natthi ukkaṃsāvakaṃse. Seyyathāpi nāma suttaguḷe khitte nibbeṭhiyamānameva paleti, evameva bāle ca paṇḍite ca sandhāvitvā saṃsaritvā dukkhassantaṃ karissantī’ti.

\paragraph{169.} ‘‘Itthaṃ kho me, bhante, makkhali gosālo sandiṭṭhikaṃ sāmaññaphalaṃ puṭṭho samāno saṃsārasuddhiṃ byākāsi. Seyyathāpi, bhante, ambaṃ vā puṭṭho labujaṃ byākareyya, labujaṃ vā puṭṭho ambaṃ byākareyya; evameva kho me, bhante, makkhali gosālo sandiṭṭhikaṃ sāmaññaphalaṃ puṭṭho samāno saṃsārasuddhiṃ byākāsi. Tassa mayhaṃ, bhante, etadahosi – ‘kathañhi nāma mādiso samaṇaṃ vā brāhmaṇaṃ vā vijite vasantaṃ apasādetabbaṃ maññeyyā’ti. So kho ahaṃ, bhante, makkhalissa gosālassa bhāsitaṃ neva abhinandiṃ nappaṭikkosiṃ. Anabhinanditvā appaṭikkositvā anattamano, anattamanavācaṃ anicchāretvā, tameva vācaṃ anuggaṇhanto anikkujjanto uṭṭhāyāsanā pakkamiṃ.

\subsubsection{Ajitakesakambalavādo}

\paragraph{170.} ‘‘Ekamidāhaṃ, bhante, samayaṃ yena ajito kesakambalo tenupasaṅkamiṃ; upasaṅkamitvā ajitena kesakambalena saddhiṃ sammodiṃ. Sammodanīyaṃ kathaṃ sāraṇīyaṃ vītisāretvā ekamantaṃ nisīdiṃ. Ekamantaṃ nisinno kho ahaṃ, bhante, ajitaṃ kesakambalaṃ etadavocaṃ – ‘yathā nu kho imāni, bho ajita, puthusippāyatanāni …pe… sakkā nu kho, bho ajita, evameva diṭṭheva dhamme sandiṭṭhikaṃ sāmaññaphalaṃ paññapetu’nti?

\paragraph{171.} ‘‘Evaṃ vutte, bhante, ajito kesakambalo maṃ etadavoca – ‘natthi, mahārāja, dinnaṃ, natthi yiṭṭhaṃ, natthi hutaṃ, natthi sukatadukkaṭānaṃ kammānaṃ phalaṃ vipāko, natthi ayaṃ loko\footnote{paraloko (syā.)}, natthi paro loko, natthi mātā, natthi pitā, natthi sattā opapātikā, natthi loke samaṇabrāhmaṇā sammaggatā\footnote{samaggatā (ka.), samaggatā (syā.)} sammāpaṭipannā, ye imañca lokaṃ parañca lokaṃ sayaṃ abhiññā sacchikatvā pavedenti. Cātumahābhūtiko ayaṃ puriso, yadā kālaṅkaroti, pathavī pathavikāyaṃ anupeti anupagacchati, āpo āpokāyaṃ anupeti anupagacchati, tejo tejokāyaṃ anupeti anupagacchati, vāyo vāyokāyaṃ anupeti anupagacchati, ākāsaṃ indriyāni saṅkamanti. Āsandipañcamā purisā mataṃ ādāya gacchanti. Yāvāḷāhanā padāni paññāyanti. Kāpotakāni aṭṭhīni bhavanti, bhassantā āhutiyo. Dattupaññattaṃ yadidaṃ dānaṃ. Tesaṃ tucchaṃ musā vilāpo ye keci atthikavādaṃ vadanti. Bāle ca paṇḍite ca kāyassa bhedā ucchijjanti vinassanti, na honti paraṃ maraṇā’ti.

\paragraph{172.} ‘‘Itthaṃ kho me, bhante, ajito kesakambalo sandiṭṭhikaṃ sāmaññaphalaṃ puṭṭho samāno ucchedaṃ byākāsi. Seyyathāpi, bhante, ambaṃ vā puṭṭho labujaṃ byākareyya, labujaṃ vā puṭṭho ambaṃ byākareyya; evameva kho me, bhante, ajito kesakambalo sandiṭṭhikaṃ sāmaññaphalaṃ puṭṭho samāno ucchedaṃ byākāsi. Tassa mayhaṃ, bhante, etadahosi – ‘kathañhi nāma mādiso samaṇaṃ vā brāhmaṇaṃ vā vijite vasantaṃ apasādetabbaṃ maññeyyā’ti. So kho ahaṃ, bhante, ajitassa kesakambalassa bhāsitaṃ neva abhinandiṃ nappaṭikkosiṃ. Anabhinanditvā appaṭikkositvā anattamano anattamanavācaṃ anicchāretvā tameva vācaṃ anuggaṇhanto anikkujjanto uṭṭhāyāsanā pakkamiṃ.

\subsubsection{Pakudhakaccāyanavādo}

\paragraph{173.} ‘‘Ekamidāhaṃ, bhante, samayaṃ yena pakudho kaccāyano tenupasaṅkamiṃ; upasaṅkamitvā pakudhena kaccāyanena saddhiṃ sammodiṃ. Sammodanīyaṃ kathaṃ sāraṇīyaṃ vītisāretvā ekamantaṃ nisīdiṃ. Ekamantaṃ nisinno kho ahaṃ, bhante, pakudhaṃ kaccāyanaṃ etadavocaṃ – ‘yathā nu kho imāni, bho kaccāyana, puthusippāyatanāni …pe… sakkā nu kho, bho kaccāyana, evameva diṭṭheva dhamme sandiṭṭhikaṃ sāmaññaphalaṃ paññapetu’nti?

\paragraph{174.} ‘‘Evaṃ vutte, bhante, pakudho kaccāyano maṃ etadavoca – ‘sattime, mahārāja, kāyā akaṭā akaṭavidhā animmitā animmātā vañjhā kūṭaṭṭhā esikaṭṭhāyiṭṭhitā. Te na iñjanti, na vipariṇamanti, na aññamaññaṃ byābādhenti, nālaṃ aññamaññassa sukhāya vā dukkhāya vā sukhadukkhāya vā. Katame satta? Pathavikāyo, āpokāyo, tejokāyo, vāyokāyo, sukhe, dukkhe, jīve sattame – ime satta kāyā akaṭā akaṭavidhā animmitā animmātā vañjhā kūṭaṭṭhā esikaṭṭhāyiṭṭhitā. Te na iñjanti, na vipariṇamanti, na aññamaññaṃ byābādhenti, nālaṃ aññamaññassa sukhāya vā dukkhāya vā sukhadukkhāya vā. Tattha natthi hantā vā ghātetā vā, sotā vā sāvetā vā, viññātā vā viññāpetā vā. Yopi tiṇhena satthena sīsaṃ chindati, na koci kiñci\footnote{kañci (kaṃ.)} jīvitā voropeti; sattannaṃ tveva\footnote{sattannaṃ yeva (sī. syā. kaṃ. pī.)} kāyānamantarena satthaṃ vivaramanupatatī’ti.

\paragraph{175.} ‘‘Itthaṃ kho me, bhante, pakudho kaccāyano sandiṭṭhikaṃ sāmaññaphalaṃ puṭṭho samāno aññena aññaṃ byākāsi. Seyyathāpi, bhante, ambaṃ vā puṭṭho labujaṃ byākareyya, labujaṃ vā puṭṭho ambaṃ byākareyya; evameva kho me, bhante, pakudho kaccāyano sandiṭṭhikaṃ sāmaññaphalaṃ puṭṭho samāno aññena aññaṃ byākāsi. Tassa mayhaṃ, bhante, etadahosi – ‘kathañhi nāma mādiso samaṇaṃ vā brāhmaṇaṃ vā vijite vasantaṃ apasādetabbaṃ maññeyyā’ti. So kho ahaṃ, bhante, pakudhassa kaccāyanassa bhāsitaṃ neva abhinandiṃ nappaṭikkosiṃ, anabhinanditvā appaṭikkositvā anattamano, anattamanavācaṃ anicchāretvā tameva vācaṃ anuggaṇhanto anikkujjanto uṭṭhāyāsanā pakkamiṃ.

\subsubsection{Nigaṇṭhanāṭaputtavādo}

\paragraph{176.} ‘‘Ekamidāhaṃ, bhante, samayaṃ yena nigaṇṭho nāṭaputto tenupasaṅkamiṃ; upasaṅkamitvā nigaṇṭhena nāṭaputtena saddhiṃ sammodiṃ. Sammodanīyaṃ kathaṃ sāraṇīyaṃ vītisāretvā ekamantaṃ nisīdiṃ. Ekamantaṃ nisinno kho ahaṃ, bhante, nigaṇṭhaṃ nāṭaputtaṃ etadavocaṃ – ‘yathā nu kho imāni, bho aggivessana, puthusippāyatanāni …pe… sakkā nu kho, bho aggivessana, evameva diṭṭheva dhamme sandiṭṭhikaṃ sāmaññaphalaṃ paññapetu’nti?

\paragraph{177.} ‘‘Evaṃ vutte, bhante, nigaṇṭho nāṭaputto maṃ etadavoca – ‘idha, mahārāja, nigaṇṭho cātuyāmasaṃvarasaṃvuto hoti. Kathañca, mahārāja, nigaṇṭho cātuyāmasaṃvarasaṃvuto hoti? Idha, mahārāja, nigaṇṭho sabbavārivārito ca hoti, sabbavāriyutto ca, sabbavāridhuto ca, sabbavāriphuṭo ca. Evaṃ kho, mahārāja, nigaṇṭho cātuyāmasaṃvarasaṃvuto hoti. Yato kho, mahārāja, nigaṇṭho evaṃ cātuyāmasaṃvarasaṃvuto hoti; ayaṃ vuccati, mahārāja, nigaṇṭho\footnote{nigaṇṭho nāṭaputto (syā. ka.)} gatatto ca yatatto ca ṭhitatto cā’ti.

\paragraph{178.} ‘‘Itthaṃ kho me, bhante, nigaṇṭho nāṭaputto sandiṭṭhikaṃ sāmaññaphalaṃ puṭṭho samāno cātuyāmasaṃvaraṃ byākāsi. Seyyathāpi, bhante, ambaṃ vā puṭṭho labujaṃ byākareyya, labujaṃ vā puṭṭho ambaṃ byākareyya; evameva kho me, bhante, nigaṇṭho nāṭaputto sandiṭṭhikaṃ sāmaññaphalaṃ puṭṭho samāno cātuyāmasaṃvaraṃ byākāsi. Tassa mayhaṃ, bhante, etadahosi – ‘kathañhi nāma mādiso samaṇaṃ vā brāhmaṇaṃ vā vijite vasantaṃ apasādetabbaṃ maññeyyā’ti. So kho ahaṃ, bhante, nigaṇṭhassa nāṭaputtassa bhāsitaṃ neva abhinandiṃ nappaṭikkosiṃ. Anabhinanditvā appaṭikkositvā anattamano anattamanavācaṃ anicchāretvā tameva vācaṃ anuggaṇhanto anikkujjanto uṭṭhāyāsanā pakkamiṃ.

\subsubsection{Sañcayabelaṭṭhaputtavādo}

\paragraph{179.} ‘‘Ekamidāhaṃ, bhante, samayaṃ yena sañcayo belaṭṭhaputto tenupasaṅkamiṃ; upasaṅkamitvā sañcayena belaṭṭhaputtena saddhiṃ sammodiṃ. Sammodanīyaṃ kathaṃ sāraṇīyaṃ vītisāretvā ekamantaṃ nisīdiṃ. Ekamantaṃ nisinno kho ahaṃ bhante, sañcayaṃ belaṭṭhaputtaṃ etadavocaṃ – ‘yathā nu kho imāni, bho sañcaya, puthusippāyatanāni …pe… sakkā nu kho, bho sañcaya, evameva diṭṭheva dhamme sandiṭṭhikaṃ sāmaññaphalaṃ paññapetu’nti?

\paragraph{180.} ‘‘Evaṃ vutte, bhante, sañcayo belaṭṭhaputto maṃ etadavoca – ‘atthi paro lokoti iti ce maṃ pucchasi, atthi paro lokoti iti ce me assa, atthi paro lokoti iti te naṃ byākareyyaṃ. Evantipi me no, tathātipi me no, aññathātipi me no, notipi me no, no notipi me no. Natthi paro loko …pe… atthi ca natthi ca paro loko …pe… nevatthi na natthi paro loko …pe… atthi sattā opapātikā …pe… natthi sattā opapātikā …pe… atthi ca natthi ca sattā opapātikā …pe… nevatthi na natthi sattā opapātikā …pe… atthi sukatadukkaṭānaṃ kammānaṃ phalaṃ vipāko …pe… natthi sukatadukkaṭānaṃ kammānaṃ phalaṃ vipāko… pe…atthi ca natthi ca sukatadukkaṭānaṃ kammānaṃ phalaṃ vipāko …pe… nevatthi na natthi sukatadukkaṭānaṃ kammānaṃ phalaṃ vipāko …pe… hoti tathāgato paraṃ maraṇā… pe… na hoti tathāgato paraṃ maraṇā …pe… hoti ca na ca hoti tathāgato paraṃ maraṇā… pe… neva hoti na na hoti tathāgato paraṃ maraṇāti iti ce maṃ pucchasi, neva hoti na na hoti tathāgato paraṃ maraṇāti iti ce me assa, neva hoti na na hoti tathāgato paraṃ maraṇāti iti te naṃ byākareyyaṃ. Evantipi me no, tathātipi me no, aññathātipi me no, notipi me no, no notipi me no’ti.

\paragraph{181.} ‘‘Itthaṃ kho me, bhante, sañcayo belaṭṭhaputto sandiṭṭhikaṃ sāmaññaphalaṃ puṭṭho samāno vikkhepaṃ byākāsi. Seyyathāpi, bhante, ambaṃ vā puṭṭho labujaṃ byākareyya, labujaṃ vā puṭṭho ambaṃ byākareyya; evameva kho me, bhante, sañcayo belaṭṭhaputto sandiṭṭhikaṃ sāmaññaphalaṃ puṭṭho samāno vikkhepaṃ byākāsi. Tassa mayhaṃ, bhante, etadahosi – ‘ayañca imesaṃ samaṇabrāhmaṇānaṃ sabbabālo sabbamūḷho. Kathañhi nāma sandiṭṭhikaṃ sāmaññaphalaṃ puṭṭho samāno vikkhepaṃ byākarissatī’ti. Tassa mayhaṃ, bhante, etadahosi – ‘kathañhi nāma mādiso samaṇaṃ vā brāhmaṇaṃ vā vijite vasantaṃ apasādetabbaṃ maññeyyā’ti. So kho ahaṃ, bhante, sañcayassa belaṭṭhaputtassa bhāsitaṃ neva abhinandiṃ nappaṭikkosiṃ. Anabhinanditvā appaṭikkositvā anattamano anattamanavācaṃ anicchāretvā tameva vācaṃ anuggaṇhanto anikkujjanto uṭṭhāyāsanā pakkamiṃ.

\subsubsection{Paṭhamasandiṭṭhikasāmaññaphalaṃ}

\paragraph{182.} ‘‘Sohaṃ, bhante, bhagavantampi pucchāmi – ‘yathā nu kho imāni, bhante, puthusippāyatanāni seyyathidaṃ – hatthārohā assārohā rathikā dhanuggahā celakā calakā piṇḍadāyakā uggā rājaputtā pakkhandino mahānāgā sūrā cammayodhino dāsikaputtā āḷārikā kappakā nhāpakā sūdā mālākārā rajakā pesakārā naḷakārā kumbhakārā gaṇakā muddikā, yāni vā panaññānipi evaṃgatāni puthusippāyatanāni, te diṭṭheva dhamme sandiṭṭhikaṃ sippaphalaṃ upajīvanti, te tena attānaṃ sukhenti pīṇenti, mātāpitaro sukhenti pīṇenti, puttadāraṃ sukhenti pīṇenti, mittāmacce sukhenti pīṇenti, samaṇabrāhmaṇesu uddhaggikaṃ dakkhiṇaṃ patiṭṭhapenti sovaggikaṃ sukhavipākaṃ saggasaṃvattanikaṃ. Sakkā nu kho me, bhante, evameva diṭṭheva dhamme sandiṭṭhikaṃ sāmaññaphalaṃ paññapetu’nti?

\paragraph{183.} ‘‘Sakkā, mahārāja. Tena hi, mahārāja, taññevettha paṭipucchissāmi. Yathā te khameyya, tathā naṃ byākareyyāsi. Taṃ kiṃ maññasi, mahārāja, idha te assa puriso dāso kammakāro\footnote{kammakaro (sī. syā. kaṃ. pī.)} pubbuṭṭhāyī pacchānipātī kiṅkārapaṭissāvī manāpacārī piyavādī mukhullokako\footnote{mukhullokiko (syā. kaṃ. ka.)}. Tassa evamassa – ‘acchariyaṃ, vata bho, abbhutaṃ, vata bho, puññānaṃ gati, puññānaṃ vipāko. Ayañhi rājā māgadho ajātasattu vedehiputto manusso; ahampi manusso. Ayañhi rājā māgadho ajātasattu vedehiputto pañcahi kāmaguṇehi samappito samaṅgībhūto paricāreti, devo maññe. Ahaṃ panamhissa dāso kammakāro pubbuṭṭhāyī pacchānipātī kiṅkārapaṭissāvī manāpacārī piyavādī mukhullokako. So vatassāhaṃ puññāni kareyyaṃ. Yaṃnūnāhaṃ kesamassuṃ ohāretvā kāsāyāni vatthāni acchādetvā agārasmā anagāriyaṃ pabbajeyya’nti. So aparena samayena kesamassuṃ ohāretvā kāsāyāni vatthāni acchādetvā agārasmā anagāriyaṃ pabbajeyya. So evaṃ pabbajito samāno kāyena saṃvuto vihareyya, vācāya saṃvuto vihareyya, manasā saṃvuto vihareyya, ghāsacchādanaparamatāya santuṭṭho, abhirato paviveke. Taṃ ce te purisā evamāroceyyuṃ – ‘yagghe deva jāneyyāsi, yo te so puriso\footnote{yo te puriso (sī. ka.)} dāso kammakāro pubbuṭṭhāyī pacchānipātī kiṅkārapaṭissāvī manāpacārī piyavādī mukhullokako; so, deva, kesamassuṃ ohāretvā kāsāyāni vatthāni acchādetvā agārasmā anagāriyaṃ pabbajito. So evaṃ pabbajito samāno kāyena saṃvuto viharati, vācāya saṃvuto viharati, manasā saṃvuto viharati, ghāsacchādanaparamatāya santuṭṭho, abhirato paviveke’ti. Api nu tvaṃ evaṃ vadeyyāsi – ‘etu me, bho, so puriso, punadeva hotu dāso kammakāro pubbuṭṭhāyī pacchānipātī kiṅkārapaṭissāvī manāpacārī piyavādī mukhullokako’ti?

\paragraph{184.} ‘‘No hetaṃ, bhante. Atha kho naṃ mayameva abhivādeyyāmapi, paccuṭṭheyyāmapi, āsanenapi nimanteyyāma, abhinimanteyyāmapi naṃ cīvarapiṇḍapātasenāsanagilānappaccayabhesajjaparikkhārehi, dhammikampissa rakkhāvaraṇaguttiṃ saṃvidaheyyāmā’’ti.

\paragraph{185.} ‘‘Taṃ kiṃ maññasi, mahārāja, yadi evaṃ sante hoti vā sandiṭṭhikaṃ sāmaññaphalaṃ no vā’’ti? ‘‘Addhā kho, bhante, evaṃ sante hoti sandiṭṭhikaṃ sāmaññaphala’’nti. ‘‘Idaṃ kho te, mahārāja, mayā paṭhamaṃ diṭṭheva dhamme sandiṭṭhikaṃ sāmaññaphalaṃ paññatta’’nti.

\subsubsection{Dutiyasandiṭṭhikasāmaññaphalaṃ}

\paragraph{186.} ‘‘Sakkā pana, bhante, aññampi evameva diṭṭheva dhamme sandiṭṭhikaṃ sāmaññaphalaṃ paññapetu’’nti? ‘‘Sakkā, mahārāja. Tena hi, mahārāja, taññevettha paṭipucchissāmi. Yathā te khameyya, tathā naṃ byākareyyāsi. Taṃ kiṃ maññasi, mahārāja, idha te assa puriso kassako gahapatiko karakārako rāsivaḍḍhako. Tassa evamassa – ‘acchariyaṃ vata bho, abbhutaṃ vata bho, puññānaṃ gati, puññānaṃ vipāko. Ayañhi rājā māgadho ajātasattu vedehiputto manusso, ahampi manusso. Ayañhi rājā māgadho ajātasattu vedehiputto pañcahi kāmaguṇehi samappito samaṅgībhūto paricāreti, devo maññe. Ahaṃ panamhissa kassako gahapatiko karakārako rāsivaḍḍhako. So vatassāhaṃ puññāni kareyyaṃ. Yaṃnūnāhaṃ kesamassuṃ ohāretvā kāsāyāni vatthāni acchādetvā agārasmā anagāriyaṃ pabbajeyya’nti. ‘‘So aparena samayena appaṃ vā bhogakkhandhaṃ pahāya mahantaṃ vā bhogakkhandhaṃ pahāya, appaṃ vā ñātiparivaṭṭaṃ pahāya mahantaṃ vā ñātiparivaṭṭaṃ pahāya kesamassuṃ ohāretvā kāsāyāni vatthāni acchādetvā agārasmā anagāriyaṃ pabbajeyya. So evaṃ pabbajito samāno kāyena saṃvuto vihareyya, vācāya saṃvuto vihareyya, manasā saṃvuto vihareyya, ghāsacchādanaparamatāya santuṭṭho, abhirato paviveke. Taṃ ce te purisā evamāroceyyuṃ – ‘yagghe, deva jāneyyāsi, yo te so puriso\footnote{yo te puriso (sī.)} kassako gahapatiko karakārako rāsivaḍḍhako; so deva kesamassuṃ ohāretvā kāsāyāni vatthāni acchādetvā agārasmā anagāriyaṃ pabbajito. So evaṃ pabbajito samāno kāyena saṃvuto viharati, vācāya saṃvuto viharati, manasā saṃvuto viharati, ghāsacchādanaparamatāya santuṭṭho, abhirato paviveke’’ti. Api nu tvaṃ evaṃ vadeyyāsi – ‘etu me, bho, so puriso, punadeva hotu kassako gahapatiko karakārako rāsivaḍḍhako’ti?

\paragraph{187.} ‘‘No hetaṃ, bhante. Atha kho naṃ mayameva abhivādeyyāmapi, paccuṭṭheyyāmapi, āsanenapi nimanteyyāma, abhinimanteyyāmapi naṃ cīvarapiṇḍapātasenāsanagilānappaccayabhesajjaparikkhārehi, dhammikampissa rakkhāvaraṇaguttiṃ saṃvidaheyyāmā’’ti.

\paragraph{188.} ‘‘Taṃ kiṃ maññasi, mahārāja? Yadi evaṃ sante hoti vā sandiṭṭhikaṃ sāmaññaphalaṃ no vā’’ti? ‘‘Addhā kho, bhante, evaṃ sante hoti sandiṭṭhikaṃ sāmaññaphala’’nti. ‘‘Idaṃ kho te, mahārāja, mayā dutiyaṃ diṭṭheva dhamme sandiṭṭhikaṃ sāmaññaphalaṃ paññatta’’nti.

\subsubsection{Paṇītatarasāmaññaphalaṃ}

\paragraph{189.} ‘‘Sakkā pana, bhante, aññampi diṭṭheva dhamme sandiṭṭhikaṃ sāmaññaphalaṃ paññapetuṃ imehi sandiṭṭhikehi sāmaññaphalehi abhikkantatarañca paṇītatarañcā’’ti? ‘‘Sakkā, mahārāja. Tena hi, mahārāja, suṇohi, sādhukaṃ manasi karohi, bhāsissāmī’’ti. ‘‘Evaṃ, bhante’’ti kho rājā māgadho ajātasattu vedehiputto bhagavato paccassosi.

\paragraph{190.} Bhagavā etadavoca – ‘‘idha, mahārāja, tathāgato loke uppajjati arahaṃ sammāsambuddho vijjācaraṇasampanno sugato lokavidū anuttaro purisadammasārathi satthā devamanussānaṃ buddho bhagavā. So imaṃ lokaṃ sadevakaṃ samārakaṃ sabrahmakaṃ sassamaṇabrāhmaṇiṃ pajaṃ sadevamanussaṃ sayaṃ abhiññā sacchikatvā pavedeti. So dhammaṃ deseti ādikalyāṇaṃ majjhekalyāṇaṃ pariyosānakalyāṇaṃ sātthaṃ sabyañjanaṃ, kevalaparipuṇṇaṃ parisuddhaṃ brahmacariyaṃ pakāseti.

\paragraph{191.} ‘‘Taṃ dhammaṃ suṇāti gahapati vā gahapatiputto vā aññatarasmiṃ vā kule paccājāto. So taṃ dhammaṃ sutvā tathāgate saddhaṃ paṭilabhati. So tena saddhāpaṭilābhena samannāgato iti paṭisañcikkhati – ‘sambādho gharāvāso rajopatho, abbhokāso pabbajjā. Nayidaṃ sukaraṃ agāraṃ ajjhāvasatā ekantaparipuṇṇaṃ ekantaparisuddhaṃ saṅkhalikhitaṃ brahmacariyaṃ carituṃ. Yaṃnūnāhaṃ kesamassuṃ ohāretvā kāsāyāni vatthāni acchādetvā agārasmā anagāriyaṃ pabbajeyya’nti.

\paragraph{192.} ‘‘So aparena samayena appaṃ vā bhogakkhandhaṃ pahāya mahantaṃ vā bhogakkhandhaṃ pahāya appaṃ vā ñātiparivaṭṭaṃ pahāya mahantaṃ vā ñātiparivaṭṭaṃ pahāya kesamassuṃ ohāretvā kāsāyāni vatthāni acchādetvā agārasmā anagāriyaṃ pabbajati.

\paragraph{193.} ‘‘So evaṃ pabbajito samāno pātimokkhasaṃvarasaṃvuto viharati ācāragocarasampanno, aṇumattesu vajjesu bhayadassāvī, samādāya sikkhati sikkhāpadesu, kāyakammavacīkammena samannāgato kusalena, parisuddhājīvo sīlasampanno, indriyesu guttadvāro\footnote{guttadvāro, bhojane mattaññū (ka.)}, satisampajaññena samannāgato, santuṭṭho.

\subsubsection{Cūḷasīlaṃ}

\paragraph{194.} ‘‘Kathañca, mahārāja, bhikkhu sīlasampanno hoti? Idha, mahārāja, bhikkhu pāṇātipātaṃ pahāya pāṇātipātā paṭivirato hoti. Nihitadaṇḍo nihitasattho lajjī dayāpanno sabbapāṇabhūtahitānukampī viharati. Idampissa hoti sīlasmiṃ. ‘‘Adinnādānaṃ pahāya adinnādānā paṭivirato hoti dinnādāyī dinnapāṭikaṅkhī, athenena sucibhūtena attanā viharati. Idampissa hoti sīlasmiṃ. ‘‘Abrahmacariyaṃ pahāya brahmacārī hoti ārācārī virato methunā gāmadhammā. Idampissa hoti sīlasmiṃ. ‘‘Musāvādaṃ pahāya musāvādā paṭivirato hoti saccavādī saccasandho theto paccayiko avisaṃvādako lokassa. Idampissa hoti sīlasmiṃ. ‘‘Pisuṇaṃ vācaṃ pahāya pisuṇāya vācāya paṭivirato hoti; ito sutvā na amutra akkhātā imesaṃ bhedāya; amutra vā sutvā na imesaṃ akkhātā, amūsaṃ bhedāya. Iti bhinnānaṃ vā sandhātā, sahitānaṃ vā anuppadātā, samaggārāmo samaggarato samagganandī samaggakaraṇiṃ vācaṃ bhāsitā hoti. Idampissa hoti sīlasmiṃ. ‘‘Pharusaṃ vācaṃ pahāya pharusāya vācāya paṭivirato hoti; yā sā vācā nelā kaṇṇasukhā pemanīyā hadayaṅgamā porī bahujanakantā bahujanamanāpā tathārūpiṃ vācaṃ bhāsitā hoti. Idampissa hoti sīlasmiṃ. ‘‘Samphappalāpaṃ pahāya samphappalāpā paṭivirato hoti kālavādī bhūtavādī atthavādī dhammavādī vinayavādī, nidhānavatiṃ vācaṃ bhāsitā hoti kālena sāpadesaṃ pariyantavatiṃ atthasaṃhitaṃ. Idampissa hoti sīlasmiṃ. ‘‘Bījagāmabhūtagāmasamārambhā paṭivirato hoti …pe… ekabhattiko hoti rattūparato virato vikālabhojanā. Naccagītavāditavisūkadassanā paṭivirato hoti. Mālāgandhavilepanadhāraṇamaṇḍanavibhūsanaṭṭhānā paṭivirato hoti. Uccāsayanamahāsayanā paṭivirato hoti. Jātarūparajatapaṭiggahaṇā paṭivirato hoti. Āmakadhaññapaṭiggahaṇā paṭivirato hoti. Āmakamaṃsapaṭiggahaṇā paṭivirato hoti. Itthikumārikapaṭiggahaṇā paṭivirato hoti. Dāsidāsapaṭiggahaṇā paṭivirato hoti. Ajeḷakapaṭiggahaṇā paṭivirato hoti. Kukkuṭasūkarapaṭiggahaṇā paṭivirato hoti. Hatthigavassavaḷavapaṭiggahaṇā paṭivirato hoti. Khettavatthupaṭiggahaṇā paṭivirato hoti. Dūteyyapahiṇagamanānuyogā paṭivirato hoti. Kayavikkayā paṭivirato hoti. Tulākūṭakaṃsakūṭamānakūṭā paṭivirato hoti. Ukkoṭanavañcananikatisāciyogā paṭivirato hoti. Chedanavadhabandhanaviparāmosaālopasahasākārā paṭivirato hoti. Idampissa hoti sīlasmiṃ.

\xsubsubsectionEnd{Cūḷasīlaṃ niṭṭhitaṃ.}

\subsubsection{Majjhimasīlaṃ}

\paragraph{195.} ‘‘Yathā vā paneke bhonto samaṇabrāhmaṇā saddhādeyyāni bhojanāni bhuñjitvā te evarūpaṃ bījagāmabhūtagāmasamārambhaṃ anuyuttā viharanti. Seyyathidaṃ – mūlabījaṃ khandhabījaṃ phaḷubījaṃ aggabījaṃ bījabījameva pañcamaṃ, iti evarūpā bījagāmabhūtagāmasamārambhā paṭivirato hoti. Idampissa hoti sīlasmiṃ.

\paragraph{196.} ‘‘Yathā vā paneke bhonto samaṇabrāhmaṇā saddhādeyyāni bhojanāni bhuñjitvā te evarūpaṃ sannidhikāraparibhogaṃ anuyuttā viharanti. Seyyathidaṃ – annasannidhiṃ pānasannidhiṃ vatthasannidhiṃ yānasannidhiṃ sayanasannidhiṃ gandhasannidhiṃ āmisasannidhiṃ, iti vā iti evarūpā sannidhikāraparibhogā paṭivirato hoti. Idampissa hoti sīlasmiṃ.

\paragraph{197.} ‘‘Yathā vā paneke bhonto samaṇabrāhmaṇā saddhādeyyāni bhojanāni bhuñjitvā te evarūpaṃ visūkadassanaṃ anuyuttā viharanti. Seyyathidaṃ – naccaṃ gītaṃ vāditaṃ pekkhaṃ akkhānaṃ pāṇissaraṃ vetāḷaṃ kumbhathūṇaṃ sobhanakaṃ caṇḍālaṃ vaṃsaṃ dhovanaṃ hatthiyuddhaṃ assayuddhaṃ mahiṃsayuddhaṃ usabhayuddhaṃ ajayuddhaṃ meṇḍayuddhaṃ kukkuṭayuddhaṃ vaṭṭakayuddhaṃ daṇḍayuddhaṃ muṭṭhiyuddhaṃ nibbuddhaṃ uyyodhikaṃ balaggaṃ senābyūhaṃ anīkadassanaṃ iti vā iti evarūpā visūkadassanā paṭivirato hoti. Idampissa hoti sīlasmiṃ.

\paragraph{198.} ‘‘Yathā vā paneke bhonto samaṇabrāhmaṇā saddhādeyyāni bhojanāni bhuñjitvā te evarūpaṃ jūtappamādaṭṭhānānuyogaṃ anuyuttā viharanti. Seyyathidaṃ – aṭṭhapadaṃ dasapadaṃ ākāsaṃ parihārapathaṃ santikaṃ khalikaṃ ghaṭikaṃ salākahatthaṃ akkhaṃ paṅgacīraṃ vaṅkakaṃ mokkhacikaṃ ciṅgulikaṃ pattāḷhakaṃ rathakaṃ dhanukaṃ akkharikaṃ manesikaṃ yathāvajjaṃ iti vā iti evarūpā jūtappamādaṭṭhānānuyogā paṭivirato hoti. Idampissa hoti sīlasmiṃ.

\paragraph{199.} ‘‘Yathā vā paneke bhonto samaṇabrāhmaṇā saddhādeyyāni bhojanāni bhuñjitvā te evarūpaṃ uccāsayanamahāsayanaṃ anuyuttā viharanti. Seyyathidaṃ – āsandiṃ pallaṅkaṃ gonakaṃ cittakaṃ paṭikaṃ paṭalikaṃ tūlikaṃ vikatikaṃ uddalomiṃ ekantalomiṃ kaṭṭissaṃ koseyyaṃ kuttakaṃ hatthattharaṃ assattharaṃ rathattharaṃ ajinappaveṇiṃ kadalimigapavarapaccattharaṇaṃ sauttaracchadaṃ ubhatolohitakūpadhānaṃ iti vā iti evarūpā uccāsayanamahāsayanā paṭivirato hoti. Idampissa hoti sīlasmiṃ.

\paragraph{200.} ‘‘Yathā vā paneke bhonto samaṇabrāhmaṇā saddhādeyyāni bhojanāni bhuñjitvā te evarūpaṃ maṇḍanavibhūsanaṭṭhānānuyogaṃ anuyuttā viharanti. Seyyathidaṃ – ucchādanaṃ parimaddanaṃ nhāpanaṃ sambāhanaṃ ādāsaṃ añjanaṃ mālāgandhavilepanaṃ mukhacuṇṇaṃ mukhalepanaṃ hatthabandhaṃ sikhābandhaṃ daṇḍaṃ nāḷikaṃ asiṃ\footnote{khaggaṃ (sī. pī.), asiṃ khaggaṃ (syā. kaṃ.), khaggaṃ asiṃ (ka.)} chattaṃ citrupāhanaṃ uṇhīsaṃ maṇiṃ vālabījaniṃ odātāni vatthāni dīghadasāni iti vā iti evarūpā maṇḍanavibhūsanaṭṭhānānuyogā paṭivirato hoti. Idampissa hoti sīlasmiṃ.

\paragraph{201.} ‘‘Yathā vā paneke bhonto samaṇabrāhmaṇā saddhādeyyāni bhojanāni bhuñjitvā te evarūpaṃ tiracchānakathaṃ anuyuttā viharanti. Seyyathidaṃ – rājakathaṃ corakathaṃ mahāmattakathaṃ senākathaṃ bhayakathaṃ yuddhakathaṃ annakathaṃ pānakathaṃ vatthakathaṃ sayanakathaṃ mālākathaṃ gandhakathaṃ ñātikathaṃ yānakathaṃ gāmakathaṃ nigamakathaṃ nagarakathaṃ janapadakathaṃ itthikathaṃ\footnote{itthikathaṃ purisakathaṃ kumārakathaṃ kumārikathaṃ (ka.)} sūrakathaṃ visikhākathaṃ kumbhaṭṭhānakathaṃ pubbapetakathaṃ nānattakathaṃ lokakkhāyikaṃ samuddakkhāyikaṃ itibhavābhavakathaṃ iti vā iti evarūpāya tiracchānakathāya paṭivirato hoti. Idampissa hoti sīlasmiṃ.

\paragraph{202.} ‘‘Yathā vā paneke bhonto samaṇabrāhmaṇā saddhādeyyāni bhojanāni bhuñjitvā te evarūpaṃ viggāhikakathaṃ anuyuttā viharanti. Seyyathidaṃ – na tvaṃ imaṃ dhammavinayaṃ ājānāsi, ahaṃ imaṃ dhammavinayaṃ ājānāmi, kiṃ tvaṃ imaṃ dhammavinayaṃ ājānissasi, micchā paṭipanno tvamasi, ahamasmi sammā paṭipanno, sahitaṃ me, asahitaṃ te, pure vacanīyaṃ pacchā avaca, pacchā vacanīyaṃ pure avaca, adhiciṇṇaṃ te viparāvattaṃ, āropito te vādo, niggahito tvamasi, cara vādappamokkhāya, nibbeṭhehi vā sace pahosīti iti vā iti evarūpāya viggāhikakathāya paṭivirato hoti. Idampissa hoti sīlasmiṃ.

\paragraph{203.} ‘‘Yathā vā paneke bhonto samaṇabrāhmaṇā saddhādeyyāni bhojanāni bhuñjitvā te evarūpaṃ dūteyyapahiṇagamanānuyogaṃ anuyuttā viharanti. Seyyathidaṃ – raññaṃ, rājamahāmattānaṃ, khattiyānaṃ, brāhmaṇānaṃ, gahapatikānaṃ, kumārānaṃ – ‘idha gaccha, amutrāgaccha, idaṃ hara, amutra idaṃ āharā’ti iti vā iti evarūpā dūteyyapahiṇagamanānuyogā paṭivirato hoti. Idampissa hoti sīlasmiṃ.

\paragraph{204.} ‘‘Yathā vā paneke bhonto samaṇabrāhmaṇā saddhādeyyāni bhojanāni bhuñjitvā te kuhakā ca honti lapakā ca nemittikā ca nippesikā ca lābhena lābhaṃ nijigīṃsitāro ca. Iti evarūpā kuhanalapanā paṭivirato hoti. Idampissa hoti sīlasmiṃ’’.

\xsubsubsectionEnd{Majjhimasīlaṃ niṭṭhitaṃ.}

\subsubsection{Mahāsīlaṃ}

\paragraph{205.} ‘‘Yathā vā paneke bhonto samaṇabrāhmaṇā saddhādeyyāni bhojanāni bhuñjitvā te evarūpāya tiracchānavijjāya micchājīvena jīvitaṃ kappenti. Seyyathidaṃ – aṅgaṃ nimittaṃ uppātaṃ supinaṃ lakkhaṇaṃ mūsikacchinnaṃ aggihomaṃ dabbihomaṃ thusahomaṃ kaṇahomaṃ taṇḍulahomaṃ sappihomaṃ telahomaṃ mukhahomaṃ lohitahomaṃ aṅgavijjā vatthuvijjā khattavijjā sivavijjā bhūtavijjā bhūrivijjā ahivijjā visavijjā vicchikavijjā mūsikavijjā sakuṇavijjā vāyasavijjā pakkajjhānaṃ saraparittāṇaṃ migacakkaṃ iti vā iti evarūpāya tiracchānavijjāya micchājīvā paṭivirato hoti. Idampissa hoti sīlasmiṃ.

\paragraph{206.} ‘‘Yathā vā paneke bhonto samaṇabrāhmaṇā saddhādeyyāni bhojanāni bhuñjitvā te evarūpāya tiracchānavijjāya micchājīvena jīvitaṃ kappenti. Seyyathidaṃ – maṇilakkhaṇaṃ vatthalakkhaṇaṃ daṇḍalakkhaṇaṃ satthalakkhaṇaṃ asilakkhaṇaṃ usulakkhaṇaṃ dhanulakkhaṇaṃ āvudhalakkhaṇaṃ itthilakkhaṇaṃ purisalakkhaṇaṃ kumāralakkhaṇaṃ kumārilakkhaṇaṃ dāsalakkhaṇaṃ dāsilakkhaṇaṃ hatthilakkhaṇaṃ assalakkhaṇaṃ mahiṃsalakkhaṇaṃ usabhalakkhaṇaṃ golakkhaṇaṃ ajalakkhaṇaṃ meṇḍalakkhaṇaṃ kukkuṭalakkhaṇaṃ vaṭṭakalakkhaṇaṃ godhālakkhaṇaṃ kaṇṇikalakkhaṇaṃ kacchapalakkhaṇaṃ migalakkhaṇaṃ iti vā iti evarūpāya tiracchānavijjāya micchājīvā paṭivirato hoti. Idampissa hoti sīlasmiṃ.

\paragraph{207.} ‘‘Yathā vā paneke bhonto samaṇabrāhmaṇā saddhādeyyāni bhojanāni bhuñjitvā te evarūpāya tiracchānavijjāya micchājīvena jīvitaṃ kappenti. Seyyathidaṃ – raññaṃ niyyānaṃ bhavissati, raññaṃ aniyyānaṃ bhavissati, abbhantarānaṃ raññaṃ upayānaṃ bhavissati, bāhirānaṃ raññaṃ apayānaṃ bhavissati, bāhirānaṃ raññaṃ upayānaṃ bhavissati, abbhantarānaṃ raññaṃ apayānaṃ bhavissati, abbhantarānaṃ raññaṃ jayo bhavissati, bāhirānaṃ raññaṃ parājayo bhavissati, bāhirānaṃ raññaṃ jayo bhavissati, abbhantarānaṃ raññaṃ parājayo bhavissati, iti imassa jayo bhavissati, imassa parājayo bhavissati iti vā iti evarūpāya tiracchānavijjāya micchājīvā paṭivirato hoti. Idampissa hoti sīlasmiṃ.

\paragraph{208.} ‘‘Yathā vā paneke bhonto samaṇabrāhmaṇā saddhādeyyāni bhojanāni bhuñjitvā te evarūpāya tiracchānavijjāya micchājīvena jīvitaṃ kappenti. Seyyathidaṃ – candaggāho bhavissati, sūriyaggāho bhavissati, nakkhattaggāho bhavissati, candimasūriyānaṃ pathagamanaṃ bhavissati, candimasūriyānaṃ uppathagamanaṃ bhavissati, nakkhattānaṃ pathagamanaṃ bhavissati, nakkhattānaṃ uppathagamanaṃ bhavissati, ukkāpāto bhavissati, disāḍāho bhavissati, bhūmicālo bhavissati, devadudrabhi bhavissati, candimasūriyanakkhattānaṃ uggamanaṃ ogamanaṃ saṃkilesaṃ vodānaṃ bhavissati, evaṃvipāko candaggāho bhavissati, evaṃvipāko sūriyaggāho bhavissati, evaṃvipāko nakkhattaggāho bhavissati, evaṃvipākaṃ candimasūriyānaṃ pathagamanaṃ bhavissati, evaṃvipākaṃ candimasūriyānaṃ uppathagamanaṃ bhavissati, evaṃvipākaṃ nakkhattānaṃ pathagamanaṃ bhavissati, evaṃvipākaṃ nakkhattānaṃ uppathagamanaṃ bhavissati, evaṃvipāko ukkāpāto bhavissati, evaṃvipāko disāḍāho bhavissati, evaṃvipāko bhūmicālo bhavissati, evaṃvipāko devadudrabhi bhavissati, evaṃvipākaṃ candimasūriyanakkhattānaṃ uggamanaṃ ogamanaṃ saṃkilesaṃ vodānaṃ bhavissati iti vā iti evarūpāya tiracchānavijjāya micchājīvā paṭivirato hoti. Idampissa hoti sīlasmiṃ.

\paragraph{209.} ‘‘Yathā vā paneke bhonto samaṇabrāhmaṇā saddhādeyyāni bhojanāni bhuñjitvā te evarūpāya tiracchānavijjāya micchājīvena jīvitaṃ kappenti. Seyyathidaṃ – suvuṭṭhikā bhavissati, dubbuṭṭhikā bhavissati, subhikkhaṃ bhavissati, dubbhikkhaṃ bhavissati, khemaṃ bhavissati, bhayaṃ bhavissati, rogo bhavissati, ārogyaṃ bhavissati, muddā, gaṇanā, saṅkhānaṃ, kāveyyaṃ, lokāyataṃ iti vā iti evarūpāya tiracchānavijjāya micchājīvā paṭivirato hoti. Idampissa hoti sīlasmiṃ.

\paragraph{210.} ‘‘Yathā vā paneke bhonto samaṇabrāhmaṇā saddhādeyyāni bhojanāni bhuñjitvā te evarūpāya tiracchānavijjāya micchājīvena jīvitaṃ kappenti. Seyyathidaṃ – āvāhanaṃ vivāhanaṃ saṃvaraṇaṃ vivaraṇaṃ saṅkiraṇaṃ vikiraṇaṃ subhagakaraṇaṃ dubbhagakaraṇaṃ viruddhagabbhakaraṇaṃ jivhānibandhanaṃ hanusaṃhananaṃ hatthābhijappanaṃ hanujappanaṃ kaṇṇajappanaṃ ādāsapañhaṃ kumārikapañhaṃ devapañhaṃ ādiccupaṭṭhānaṃ mahatupaṭṭhānaṃ abbhujjalanaṃ sirivhāyanaṃ iti vā iti evarūpāya tiracchānavijjāya micchājīvā paṭivirato hoti. Idampissa hoti sīlasmiṃ.

\paragraph{211.} ‘‘Yathā vā paneke bhonto samaṇabrāhmaṇā saddhādeyyāni bhojanāni bhuñjitvā te evarūpāya tiracchānavijjāya micchājīvena jīvitaṃ kappenti. Seyyathidaṃ – santikammaṃ paṇidhikammaṃ bhūtakammaṃ bhūrikammaṃ vassakammaṃ vossakammaṃ vatthukammaṃ vatthuparikammaṃ ācamanaṃ nhāpanaṃ juhanaṃ vamanaṃ virecanaṃ uddhaṃvirecanaṃ adhovirecanaṃ sīsavirecanaṃ kaṇṇatelaṃ nettatappanaṃ natthukammaṃ añjanaṃ paccañjanaṃ sālākiyaṃ sallakattiyaṃ dārakatikicchā, mūlabhesajjānaṃ anuppadānaṃ, osadhīnaṃ paṭimokkho iti vā iti evarūpāya tiracchānavijjāya micchājīvā paṭivirato hoti. Idampissa hoti sīlasmiṃ.

\paragraph{212.} ‘‘Sa kho so, mahārāja, bhikkhu evaṃ sīlasampanno na kutoci bhayaṃ samanupassati, yadidaṃ sīlasaṃvarato. Seyyathāpi – mahārāja, rājā khattiyo muddhābhisitto nihatapaccāmitto na kutoci bhayaṃ samanupassati, yadidaṃ paccatthikato; evameva kho, mahārāja, bhikkhu evaṃ sīlasampanno na kutoci bhayaṃ samanupassati, yadidaṃ sīlasaṃvarato. So iminā ariyena sīlakkhandhena samannāgato ajjhattaṃ anavajjasukhaṃ paṭisaṃvedeti. Evaṃ kho, mahārāja, bhikkhu sīlasampanno hoti.

\xsubsubsectionEnd{Mahāsīlaṃ niṭṭhitaṃ.}

\subsubsection{Indriyasaṃvaro}

\paragraph{213.} ‘‘Kathañca, mahārāja, bhikkhu indriyesu guttadvāro hoti? Idha, mahārāja, bhikkhu cakkhunā rūpaṃ disvā na nimittaggāhī hoti nānubyañjanaggāhī. Yatvādhikaraṇamenaṃ cakkhundriyaṃ asaṃvutaṃ viharantaṃ abhijjhā domanassā pāpakā akusalā dhammā anvāssaveyyuṃ, tassa saṃvarāya paṭipajjati, rakkhati cakkhundriyaṃ, cakkhundriye saṃvaraṃ āpajjati. Sotena saddaṃ sutvā …pe… ghānena gandhaṃ ghāyitvā… pe… jivhāya rasaṃ sāyitvā …pe… kāyena phoṭṭhabbaṃ phusitvā …pe… manasā dhammaṃ viññāya na nimittaggāhī hoti nānubyañjanaggāhī. Yatvādhikaraṇamenaṃ manindriyaṃ asaṃvutaṃ viharantaṃ abhijjhā domanassā pāpakā akusalā dhammā anvāssaveyyuṃ, tassa saṃvarāya paṭipajjati, rakkhati manindriyaṃ, manindriye saṃvaraṃ āpajjati. So iminā ariyena indriyasaṃvarena samannāgato ajjhattaṃ abyāsekasukhaṃ paṭisaṃvedeti. Evaṃ kho, mahārāja, bhikkhu indriyesu guttadvāro hoti.

\subsubsection{Satisampajaññaṃ}

\paragraph{214.} ‘‘Kathañca, mahārāja, bhikkhu satisampajaññena samannāgato hoti? Idha, mahārāja, bhikkhu abhikkante paṭikkante sampajānakārī hoti, ālokite vilokite sampajānakārī hoti, samiñjite pasārite sampajānakārī hoti, saṅghāṭipattacīvaradhāraṇe sampajānakārī hoti, asite pīte khāyite sāyite sampajānakārī hoti, uccārapassāvakamme sampajānakārī hoti, gate ṭhite nisinne sutte jāgarite bhāsite tuṇhībhāve sampajānakārī hoti. Evaṃ kho, mahārāja, bhikkhu satisampajaññena samannāgato hoti.

\subsubsection{Santoso}

\paragraph{215.} ‘‘Kathañca, mahārāja, bhikkhu santuṭṭho hoti? Idha, mahārāja, bhikkhu santuṭṭho hoti kāyaparihārikena cīvarena, kucchiparihārikena piṇḍapātena. So yena yeneva pakkamati, samādāyeva pakkamati. Seyyathāpi, mahārāja, pakkhī sakuṇo yena yeneva ḍeti, sapattabhārova ḍeti. Evameva kho, mahārāja, bhikkhu santuṭṭho hoti kāyaparihārikena cīvarena kucchiparihārikena piṇḍapātena. So yena yeneva pakkamati, samādāyeva pakkamati. Evaṃ kho, mahārāja, bhikkhu santuṭṭho hoti.

\subsubsection{Nīvaraṇappahānaṃ}

\paragraph{216.} ‘‘So iminā ca ariyena sīlakkhandhena samannāgato, iminā ca ariyena indriyasaṃvarena samannāgato, iminā ca ariyena satisampajaññena samannāgato, imāya ca ariyāya santuṭṭhiyā samannāgato, vivittaṃ senāsanaṃ bhajati araññaṃ rukkhamūlaṃ pabbataṃ kandaraṃ giriguhaṃ susānaṃ vanapatthaṃ abbhokāsaṃ palālapuñjaṃ. So pacchābhattaṃ piṇḍapātappaṭikkanto nisīdati pallaṅkaṃ ābhujitvā ujuṃ kāyaṃ paṇidhāya parimukhaṃ satiṃ upaṭṭhapetvā.

\paragraph{217.} ‘‘So abhijjhaṃ loke pahāya vigatābhijjhena cetasā viharati, abhijjhāya cittaṃ parisodheti. Byāpādapadosaṃ pahāya abyāpannacitto viharati sabbapāṇabhūtahitānukampī, byāpādapadosā cittaṃ parisodheti. Thinamiddhaṃ pahāya vigatathinamiddho viharati ālokasaññī, sato sampajāno, thinamiddhā cittaṃ parisodheti. Uddhaccakukkuccaṃ pahāya anuddhato viharati, ajjhattaṃ vūpasantacitto, uddhaccakukkuccā cittaṃ parisodheti. Vicikicchaṃ pahāya tiṇṇavicikiccho viharati, akathaṃkathī kusalesu dhammesu, vicikicchāya cittaṃ parisodheti.

\paragraph{218.} ‘‘Seyyathāpi, mahārāja, puriso iṇaṃ ādāya kammante payojeyya. Tassa te kammantā samijjheyyuṃ. So yāni ca porāṇāni iṇamūlāni, tāni ca byantiṃ kareyya\footnote{byantīkareyya (sī. syā. kaṃ.)}, siyā cassa uttariṃ avasiṭṭhaṃ dārabharaṇāya. Tassa evamassa – ‘ahaṃ kho pubbe iṇaṃ ādāya kammante payojesiṃ. Tassa me te kammantā samijjhiṃsu. Sohaṃ yāni ca porāṇāni iṇamūlāni, tāni ca byantiṃ akāsiṃ, atthi ca me uttariṃ avasiṭṭhaṃ dārabharaṇāyā’ti. So tatonidānaṃ labhetha pāmojjaṃ, adhigaccheyya somanassaṃ.

\paragraph{219.} ‘‘Seyyathāpi, mahārāja, puriso ābādhiko assa dukkhito bāḷhagilāno; bhattañcassa nacchādeyya, na cassa kāye balamattā. So aparena samayena tamhā ābādhā mucceyya; bhattaṃ cassa chādeyya, siyā cassa kāye balamattā. Tassa evamassa – ‘ahaṃ kho pubbe ābādhiko ahosiṃ dukkhito bāḷhagilāno; bhattañca me nacchādesi, na ca me āsi\footnote{na cassa me (ka.)} kāye balamattā. Somhi etarahi tamhā ābādhā mutto; bhattañca me chādeti, atthi ca me kāye balamattā’ti. So tatonidānaṃ labhetha pāmojjaṃ, adhigaccheyya somanassaṃ.

\paragraph{220.} ‘‘Seyyathāpi, mahārāja, puriso bandhanāgāre baddho assa. So aparena samayena tamhā bandhanāgārā mucceyya sotthinā abbhayena\footnote{ubbayena (sī. ka.)}, na cassa kiñci bhogānaṃ vayo. Tassa evamassa – ‘ahaṃ kho pubbe bandhanāgāre baddho ahosiṃ, somhi etarahi tamhā bandhanāgārā mutto sotthinā abbhayena. Natthi ca me kiñci bhogānaṃ vayo’ti. So tatonidānaṃ labhetha pāmojjaṃ, adhigaccheyya somanassaṃ.

\paragraph{221.} ‘‘Seyyathāpi, mahārāja, puriso dāso assa anattādhīno parādhīno na yenakāmaṃgamo. So aparena samayena tamhā dāsabyā mucceyya attādhīno aparādhīno bhujisso yenakāmaṃgamo. Tassa evamassa – ‘ahaṃ kho pubbe dāso ahosiṃ anattādhīno parādhīno na yenakāmaṃgamo. Somhi etarahi tamhā dāsabyā mutto attādhīno aparādhīno bhujisso yenakāmaṃgamo’ti. So tatonidānaṃ labhetha pāmojjaṃ, adhigaccheyya somanassaṃ.

\paragraph{222.} ‘‘Seyyathāpi, mahārāja, puriso sadhano sabhogo kantāraddhānamaggaṃ paṭipajjeyya dubbhikkhaṃ sappaṭibhayaṃ. So aparena samayena taṃ kantāraṃ nitthareyya sotthinā, gāmantaṃ anupāpuṇeyya khemaṃ appaṭibhayaṃ. Tassa evamassa – ‘ahaṃ kho pubbe sadhano sabhogo kantāraddhānamaggaṃ paṭipajjiṃ dubbhikkhaṃ sappaṭibhayaṃ. Somhi etarahi taṃ kantāraṃ nitthiṇṇo sotthinā, gāmantaṃ anuppatto khemaṃ appaṭibhaya’nti. So tatonidānaṃ labhetha pāmojjaṃ, adhigaccheyya somanassaṃ.

\paragraph{223.} ‘‘Evameva kho, mahārāja, bhikkhu yathā iṇaṃ yathā rogaṃ yathā bandhanāgāraṃ yathā dāsabyaṃ yathā kantāraddhānamaggaṃ, evaṃ ime pañca nīvaraṇe appahīne attani samanupassati.

\paragraph{224.} ‘‘Seyyathāpi, mahārāja, yathā āṇaṇyaṃ yathā ārogyaṃ yathā bandhanāmokkhaṃ yathā bhujissaṃ yathā khemantabhūmiṃ; evameva kho, mahārāja, bhikkhu ime pañca nīvaraṇe pahīne attani samanupassati.

\paragraph{225.} ‘‘Tassime pañca nīvaraṇe pahīne attani samanupassato pāmojjaṃ jāyati, pamuditassa pīti jāyati, pītimanassa kāyo passambhati, passaddhakāyo sukhaṃ vedeti, sukhino cittaṃ samādhiyati.

\subsubsection{Paṭhamajjhānaṃ}

\paragraph{226.} ‘‘So vivicceva kāmehi, vivicca akusalehi dhammehi savitakkaṃ savicāraṃ vivekajaṃ pītisukhaṃ paṭhamaṃ jhānaṃ upasampajja viharati. So imameva kāyaṃ vivekajena pītisukhena abhisandeti parisandeti paripūreti parippharati, nāssa kiñci sabbāvato kāyassa vivekajena pītisukhena apphuṭaṃ hoti.

\paragraph{227.} ‘‘Seyyathāpi, mahārāja, dakkho nhāpako vā nhāpakantevāsī vā kaṃsathāle nhānīyacuṇṇāni ākiritvā udakena paripphosakaṃ paripphosakaṃ sanneyya, sāyaṃ nhānīyapiṇḍi snehānugatā snehaparetā santarabāhirā phuṭā snehena, na ca paggharaṇī; evameva kho, mahārāja, bhikkhu imameva kāyaṃ vivekajena pītisukhena abhisandeti parisandeti paripūreti parippharati, nāssa kiñci sabbāvato kāyassa vivekajena pītisukhena apphuṭaṃ hoti. Idampi kho, mahārāja, sandiṭṭhikaṃ sāmaññaphalaṃ purimehi sandiṭṭhikehi sāmaññaphalehi abhikkantatarañca paṇītatarañca.

\subsubsection{Dutiyajjhānaṃ}

\paragraph{228.} ‘‘Puna caparaṃ, mahārāja, bhikkhu vitakkavicārānaṃ vūpasamā ajjhattaṃ sampasādanaṃ cetaso ekodibhāvaṃ avitakkaṃ avicāraṃ samādhijaṃ pītisukhaṃ dutiyaṃ jhānaṃ upasampajja viharati. So imameva kāyaṃ samādhijena pītisukhena abhisandeti parisandeti paripūreti parippharati, nāssa kiñci sabbāvato kāyassa samādhijena pītisukhena apphuṭaṃ hoti.

\paragraph{229.} ‘‘Seyyathāpi, mahārāja, udakarahado gambhīro ubbhidodako\footnote{ubbhitodako (syā. kaṃ. ka.)} tassa nevassa puratthimāya disāya udakassa āyamukhaṃ, na dakkhiṇāya disāya udakassa āyamukhaṃ, na pacchimāya disāya udakassa āyamukhaṃ, na uttarāya disāya udakassa āyamukhaṃ, devo ca na kālenakālaṃ sammādhāraṃ anuppaveccheyya. Atha kho tamhāva udakarahadā sītā vāridhārā ubbhijjitvā tameva udakarahadaṃ sītena vārinā abhisandeyya parisandeyya paripūreyya paripphareyya, nāssa kiñci sabbāvato udakarahadassa sītena vārinā apphuṭaṃ assa. Evameva kho, mahārāja, bhikkhu imameva kāyaṃ samādhijena pītisukhena abhisandeti parisandeti paripūreti parippharati, nāssa kiñci sabbāvato kāyassa samādhijena pītisukhena apphuṭaṃ hoti. Idampi kho, mahārāja, sandiṭṭhikaṃ sāmaññaphalaṃ purimehi sandiṭṭhikehi sāmaññaphalehi abhikkantatarañca paṇītatarañca.

\subsubsection{Tatiyajjhānaṃ}

\paragraph{230.} ‘‘Puna caparaṃ, mahārāja, bhikkhu pītiyā ca virāgā upekkhako ca viharati sato sampajāno, sukhañca kāyena paṭisaṃvedeti, yaṃ taṃ ariyā ācikkhanti – ‘upekkhako satimā sukhavihārī’ti, tatiyaṃ jhānaṃ upasampajja viharati. So imameva kāyaṃ nippītikena sukhena abhisandeti parisandeti paripūreti parippharati, nāssa kiñci sabbāvato kāyassa nippītikena sukhena apphuṭaṃ hoti.

\paragraph{231.} ‘‘Seyyathāpi, mahārāja, uppaliniyaṃ vā paduminiyaṃ vā puṇḍarīkiniyaṃ vā appekaccāni uppalāni vā padumāni vā puṇḍarīkāni vā udake jātāni udake saṃvaḍḍhāni udakānuggatāni antonimuggaposīni, tāni yāva caggā yāva ca mūlā sītena vārinā abhisannāni parisannāni\footnote{abhisandāni parisandāni (ka.)} paripūrāni paripphuṭāni\footnote{paripphuṭṭhāni (pī.)}, nāssa kiñci sabbāvataṃ uppalānaṃ vā padumānaṃ vā puṇḍarīkānaṃ vā sītena vārinā apphuṭaṃ assa; evameva kho, mahārāja, bhikkhu imameva kāyaṃ nippītikena sukhena abhisandeti parisandeti paripūreti parippharati, nāssa kiñci sabbāvato kāyassa nippītikena sukhena apphuṭaṃ hoti. Idampi kho, mahārāja, sandiṭṭhikaṃ sāmaññaphalaṃ purimehi sandiṭṭhikehi sāmaññaphalehi abhikkantatarañca paṇītatarañca.

\subsubsection{Catutthajjhānaṃ}

\paragraph{232.} ‘‘Puna caparaṃ, mahārāja, bhikkhu sukhassa ca pahānā dukkhassa ca pahānā, pubbeva somanassadomanassānaṃ atthaṅgamā adukkhamasukhaṃ upekkhāsatipārisuddhiṃ catutthaṃ jhānaṃ upasampajja viharati, so imameva kāyaṃ parisuddhena cetasā pariyodātena pharitvā nisinno hoti, nāssa kiñci sabbāvato kāyassa parisuddhena cetasā pariyodātena apphuṭaṃ hoti.

\paragraph{233.} ‘‘Seyyathāpi, mahārāja, puriso odātena vatthena sasīsaṃ pārupitvā nisinno assa, nāssa kiñci sabbāvato kāyassa odātena vatthena apphuṭaṃ assa; evameva kho, mahārāja, bhikkhu imameva kāyaṃ parisuddhena cetasā pariyodātena pharitvā nisinno hoti, nāssa kiñci sabbāvato kāyassa parisuddhena cetasā pariyodātena apphuṭaṃ hoti. Idampi kho, mahārāja, sandiṭṭhikaṃ sāmaññaphalaṃ purimehi sandiṭṭhikehi sāmaññaphalehi abhikkantatarañca paṇītatarañca.

\subsubsection{Vipassanāñāṇaṃ}

\paragraph{234.} ‘‘So\footnote{puna caparaṃ mahārāja bhikkhu so (ka.)} evaṃ samāhite citte parisuddhe pariyodāte anaṅgaṇe vigatūpakkilese mudubhūte kammaniye ṭhite āneñjappatte ñāṇadassanāya cittaṃ abhinīharati abhininnāmeti. So evaṃ pajānāti – ‘ayaṃ kho me kāyo rūpī cātumahābhūtiko mātāpettikasambhavo odanakummāsūpacayo aniccucchādanaparimaddana\hyp{}bhedana\hyp{}viddhaṃsana\hyp{}dhammo; idañca pana me viññāṇaṃ ettha sitaṃ ettha paṭibaddha’nti.

\paragraph{235.} ‘‘Seyyathāpi, mahārāja, maṇi veḷuriyo subho jātimā aṭṭhaṃso suparikammakato accho vippasanno anāvilo sabbākārasampanno. Tatrāssa suttaṃ āvutaṃ nīlaṃ vā pītaṃ vā lohitaṃ vā\footnote{pītakaṃ vā lohitakaṃ vā (ka.)} odātaṃ vā paṇḍusuttaṃ vā. Tamenaṃ cakkhumā puriso hatthe karitvā paccavekkheyya – ‘ayaṃ kho maṇi veḷuriyo subho jātimā aṭṭhaṃso suparikammakato accho vippasanno anāvilo sabbākārasampanno; tatridaṃ suttaṃ āvutaṃ nīlaṃ vā pītaṃ vā lohitaṃ vā odātaṃ vā paṇḍusuttaṃ vā’ti. Evameva kho, mahārāja, bhikkhu evaṃ samāhite citte parisuddhe pariyodāte anaṅgaṇe vigatūpakkilese mudubhūte kammaniye ṭhite āneñjappatte ñāṇadassanāya cittaṃ abhinīharati abhininnāmeti. So evaṃ pajānāti – ‘ayaṃ kho me kāyo rūpī cātumahābhūtiko mātāpettikasambhavo odanakummāsūpacayo aniccucchādanaparimaddanabhedanaviddhaṃsanadhammo; idañca pana me viññāṇaṃ ettha sitaṃ ettha paṭibaddha’nti. Idampi kho, mahārāja, sandiṭṭhikaṃ sāmaññaphalaṃ purimehi sandiṭṭhikehi sāmaññaphalehi abhikkantatarañca paṇītatarañca.

\subsubsection{Manomayiddhiñāṇaṃ}

\paragraph{236.} ‘‘So evaṃ samāhite citte parisuddhe pariyodāte anaṅgaṇe vigatūpakkilese mudubhūte kammaniye ṭhite āneñjappatte manomayaṃ kāyaṃ abhinimmānāya cittaṃ abhinīharati abhininnāmeti. So imamhā kāyā aññaṃ kāyaṃ abhinimmināti rūpiṃ manomayaṃ sabbaṅgapaccaṅgiṃ ahīnindriyaṃ.

\paragraph{237.} ‘‘Seyyathāpi, mahārāja, puriso muñjamhā īsikaṃ pavāheyya\footnote{pabbāheyya (syā. ka.)}. Tassa evamassa – ‘ayaṃ muñjo, ayaṃ īsikā, añño muñjo, aññā īsikā, muñjamhā tveva īsikā pavāḷhā’ti\footnote{pabbāḷhāti (syā. ka.)}. Seyyathā vā pana, mahārāja, puriso asiṃ kosiyā pavāheyya. Tassa evamassa – ‘ayaṃ asi, ayaṃ kosi, añño asi, aññā kosi, kosiyā tveva asi pavāḷho’’ti. Seyyathā vā pana, mahārāja, puriso ahiṃ karaṇḍā uddhareyya. Tassa evamassa – ‘ayaṃ ahi, ayaṃ karaṇḍo. Añño ahi, añño karaṇḍo, karaṇḍā tveva ahi ubbhato’ti\footnote{uddharito (syā. kaṃ.)}. Evameva kho, mahārāja, bhikkhu evaṃ samāhite citte parisuddhe pariyodāte anaṅgaṇe vigatūpakkilese mudubhūte kammaniye ṭhite āneñjappatte manomayaṃ kāyaṃ abhinimmānāya cittaṃ abhinīharati abhininnāmeti. So imamhā kāyā aññaṃ kāyaṃ abhinimmināti rūpiṃ manomayaṃ sabbaṅgapaccaṅgiṃ ahīnindriyaṃ. Idampi kho, mahārāja, sandiṭṭhikaṃ sāmaññaphalaṃ purimehi sandiṭṭhikehi sāmaññaphalehi abhikkantatarañca paṇītatarañca.

\subsubsection{Iddhividhañāṇaṃ}

\paragraph{238.} ‘‘So evaṃ samāhite citte parisuddhe pariyodāte anaṅgaṇe vigatūpakkilese mudubhūte kammaniye ṭhite āneñjappatte iddhividhāya cittaṃ abhinīharati abhininnāmeti. So anekavihitaṃ iddhividhaṃ paccanubhoti – ekopi hutvā bahudhā hoti, bahudhāpi hutvā eko hoti; āvibhāvaṃ tirobhāvaṃ tirokuṭṭaṃ tiropākāraṃ tiropabbataṃ asajjamāno gacchati seyyathāpi ākāse. Pathaviyāpi ummujjanimujjaṃ karoti seyyathāpi udake. Udakepi abhijjamāne gacchati\footnote{abhijjamāno (sī. ka.)} seyyathāpi pathaviyā. Ākāsepi pallaṅkena kamati seyyathāpi pakkhī sakuṇo. Imepi candimasūriye evaṃmahiddhike evaṃmahānubhāve pāṇinā parāmasati parimajjati. Yāva brahmalokāpi kāyena vasaṃ vatteti.

\paragraph{239.} ‘‘Seyyathāpi, mahārāja, dakkho kumbhakāro vā kumbhakārantevāsī vā suparikammakatāya mattikāya yaṃ yadeva bhājanavikatiṃ ākaṅkheyya, taṃ tadeva kareyya abhinipphādeyya. Seyyathā vā pana, mahārāja, dakkho dantakāro vā dantakārantevāsī vā suparikammakatasmiṃ dantasmiṃ yaṃ yadeva dantavikatiṃ ākaṅkheyya, taṃ tadeva kareyya abhinipphādeyya. Seyyathā vā pana, mahārāja, dakkho suvaṇṇakāro vā suvaṇṇakārantevāsī vā suparikammakatasmiṃ suvaṇṇasmiṃ yaṃ yadeva suvaṇṇavikatiṃ ākaṅkheyya, taṃ tadeva kareyya abhinipphādeyya. Evameva kho, mahārāja, bhikkhu evaṃ samāhite citte parisuddhe pariyodāte anaṅgaṇe vigatūpakkilese mudubhūte kammaniye ṭhite āneñjappatte iddhividhāya cittaṃ abhinīharati abhininnāmeti. So anekavihitaṃ iddhividhaṃ paccanubhoti – ekopi hutvā bahudhā hoti, bahudhāpi hutvā eko hoti; āvibhāvaṃ tirobhāvaṃ tirokuṭṭaṃ tiropākāraṃ tiropabbataṃ asajjamāno gacchati seyyathāpi ākāse. Pathaviyāpi ummujjanimujjaṃ karoti seyyathāpi udake. Udakepi abhijjamāne gacchati seyyathāpi pathaviyā. Ākāsepi pallaṅkena kamati seyyathāpi pakkhī sakuṇo. Imepi candimasūriye evaṃmahiddhike evaṃmahānubhāve pāṇinā parāmasati parimajjati. Yāva brahmalokāpi kāyena vasaṃ vatteti. Idampi kho, mahārāja, sandiṭṭhikaṃ sāmaññaphalaṃ purimehi sandiṭṭhikehi sāmaññaphalehi abhikkantatarañca paṇītatarañca.

\subsubsection{Dibbasotañāṇaṃ}

\paragraph{240.} ‘‘So evaṃ samāhite citte parisuddhe pariyodāte anaṅgaṇe vigatūpakkilese mudubhūte kammaniye ṭhite āneñjappatte dibbāya sotadhātuyā cittaṃ abhinīharati abhininnāmeti. So dibbāya sotadhātuyā visuddhāya atikkantamānusikāya ubho sadde suṇāti dibbe ca mānuse ca ye dūre santike ca.

\paragraph{241.} ‘‘Seyyathāpi, mahārāja, puriso addhānamaggappaṭipanno. So suṇeyya bherisaddampi mudiṅgasaddampi\footnote{mutiṅgasaddampi (sī. pī.)} saṅkhapaṇavadindimasaddampi\footnote{saṅkhapaṇavadeṇḍimasaddampi (sī. pī.), saṅkhasaddaṃpi paṇavasaddaṃpi dendimasaddaṃpi (syā. kaṃ.)}. Tassa evamassa – ‘bherisaddo’ itipi, ‘mudiṅgasaddo’ itipi, ‘saṅkhapaṇavadindimasaddo’ itipi\footnote{saṅkhasaddo itipi paṇavasaddo itipi dendimasaddo itipi (syā. kaṃ.)}. Evameva kho, mahārāja, bhikkhu evaṃ samāhite citte parisuddhe pariyodāte anaṅgaṇe vigatūpakkilese mudubhūte kammaniye ṭhite āneñjappatte dibbāya sotadhātuyā cittaṃ abhinīharati abhininnāmeti. So dibbāya sotadhātuyā visuddhāya atikkantamānusikāya ubho sadde suṇāti dibbe ca mānuse ca ye dūre santike ca. Idampi kho, mahārāja, sandiṭṭhikaṃ sāmaññaphalaṃ purimehi sandiṭṭhikehi sāmaññaphalehi abhikkantatarañca paṇītatarañca.

\subsubsection{Cetopariyañāṇaṃ}

\paragraph{242.} ‘‘So evaṃ samāhite citte parisuddhe pariyodāte anaṅgaṇe vigatūpakkilese mudubhūte kammaniye ṭhite āneñjappatte cetopariyañāṇāya cittaṃ abhinīharati abhininnāmeti. So parasattānaṃ parapuggalānaṃ cetasā ceto paricca pajānāti – sarāgaṃ vā cittaṃ ‘sarāgaṃ citta’nti pajānāti, vītarāgaṃ vā cittaṃ ‘vītarāgaṃ citta’nti pajānāti, sadosaṃ vā cittaṃ ‘sadosaṃ citta’nti pajānāti, vītadosaṃ vā cittaṃ ‘vītadosaṃ citta’nti pajānāti, samohaṃ vā cittaṃ ‘samohaṃ citta’nti pajānāti, vītamohaṃ vā cittaṃ ‘vītamohaṃ citta’nti pajānāti, saṅkhittaṃ vā cittaṃ ‘saṅkhittaṃ citta’nti pajānāti, vikkhittaṃ vā cittaṃ ‘vikkhittaṃ citta’nti pajānāti, mahaggataṃ vā cittaṃ ‘mahaggataṃ citta’nti pajānāti, amahaggataṃ vā cittaṃ ‘amahaggataṃ citta’nti pajānāti, sauttaraṃ vā cittaṃ ‘sauttaraṃ citta’nti pajānāti, anuttaraṃ vā cittaṃ ‘anuttaraṃ citta’nti pajānāti, samāhitaṃ vā cittaṃ ‘samāhitaṃ citta’nti pajānāti, asamāhitaṃ vā cittaṃ ‘asamāhitaṃ citta’nti pajānāti, vimuttaṃ vā cittaṃ ‘vimuttaṃ citta’nti pajānāti, avimuttaṃ vā cittaṃ ‘avimuttaṃ citta’nti pajānāti.

\paragraph{243.} ‘‘Seyyathāpi, mahārāja, itthī vā puriso vā daharo yuvā maṇḍanajātiko ādāse vā parisuddhe pariyodāte acche vā udakapatte sakaṃ mukhanimittaṃ paccavekkhamāno sakaṇikaṃ vā ‘sakaṇika’nti jāneyya, akaṇikaṃ vā ‘akaṇika’nti jāneyya; evameva kho, mahārāja, bhikkhu evaṃ samāhite citte parisuddhe pariyodāte anaṅgaṇe vigatūpakkilese mudubhūte kammaniye ṭhite āneñjappatte cetopariyañāṇāya cittaṃ abhinīharati abhininnāmeti. So parasattānaṃ parapuggalānaṃ cetasā ceto paricca pajānāti – sarāgaṃ vā cittaṃ ‘sarāgaṃ citta’nti pajānāti, vītarāgaṃ vā cittaṃ ‘vītarāgaṃ citta’nti pajānāti, sadosaṃ vā cittaṃ ‘sadosaṃ citta’nti pajānāti, vītadosaṃ vā cittaṃ ‘vītadosaṃ citta’nti pajānāti, samohaṃ vā cittaṃ ‘samohaṃ citta’nti pajānāti, vītamohaṃ vā cittaṃ ‘vītamohaṃ citta’nti pajānāti, saṅkhittaṃ vā cittaṃ ‘saṅkhittaṃ citta’nti pajānāti, vikkhittaṃ vā cittaṃ ‘vikkhittaṃ citta’nti pajānāti, mahaggataṃ vā cittaṃ ‘mahaggataṃ citta’nti pajānāti, amahaggataṃ vā cittaṃ ‘amahaggataṃ citta’nti pajānāti, sauttaraṃ vā cittaṃ ‘sauttaraṃ citta’nti pajānāti, anuttaraṃ vā cittaṃ ‘anuttaraṃ citta’nti pajānāti, samāhitaṃ vā cittaṃ ‘samāhitaṃ citta’nti pajānāti, asamāhitaṃ vā cittaṃ ‘asamāhitaṃ citta’nti pajānāti, vimuttaṃ vā cittaṃ ‘vimuttaṃ citta’’nti pajānāti, avimuttaṃ vā cittaṃ ‘avimuttaṃ citta’nti pajānāti. Idampi kho, mahārāja, sandiṭṭhikaṃ sāmaññaphalaṃ purimehi sandiṭṭhikehi sāmaññaphalehi abhikkantatarañca paṇītatarañca.

\subsubsection{Pubbenivāsānussatiñāṇaṃ}

\paragraph{244.} ‘‘So evaṃ samāhite citte parisuddhe pariyodāte anaṅgaṇe vigatūpakkilese mudubhūte kammaniye ṭhite āneñjappatte pubbenivāsānussatiñāṇāya cittaṃ abhinīharati abhininnāmeti. So anekavihitaṃ pubbenivāsaṃ anussarati, seyyathidaṃ – ekampi jātiṃ dvepi jātiyo tissopi jātiyo catassopi jātiyo pañcapi jātiyo dasapi jātiyo vīsampi jātiyo tiṃsampi jātiyo cattālīsampi jātiyo paññāsampi jātiyo jātisatampi jātisahassampi jātisatasahassampi anekepi saṃvaṭṭakappe anekepi vivaṭṭakappe anekepi saṃvaṭṭavivaṭṭakappe, ‘amutrāsiṃ evaṃnāmo evaṃgotto evaṃvaṇṇo evamāhāro evaṃsukhadukkhappaṭisaṃvedī evamāyupariyanto, so tato cuto amutra udapādiṃ; tatrāpāsiṃ evaṃnāmo evaṃgotto evaṃvaṇṇo evamāhāro evaṃsukhadukkhappaṭisaṃvedī evamāyupariyanto, so tato cuto idhūpapanno’ti. Iti sākāraṃ sauddesaṃ anekavihitaṃ pubbenivāsaṃ anussarati.

\paragraph{245.} ‘‘Seyyathāpi, mahārāja, puriso sakamhā gāmā aññaṃ gāmaṃ gaccheyya, tamhāpi gāmā aññaṃ gāmaṃ gaccheyya. So tamhā gāmā sakaṃyeva gāmaṃ paccāgaccheyya. Tassa evamassa – ‘ahaṃ kho sakamhā gāmā amuṃ gāmaṃ agacchiṃ\footnote{agañchiṃ (syā. kaṃ.)}, tatrāpi evaṃ aṭṭhāsiṃ, evaṃ nisīdiṃ, evaṃ abhāsiṃ, evaṃ tuṇhī ahosiṃ, tamhāpi gāmā amuṃ gāmaṃ agacchiṃ, tatrāpi evaṃ aṭṭhāsiṃ, evaṃ nisīdiṃ, evaṃ abhāsiṃ, evaṃ tuṇhī ahosiṃ, somhi tamhā gāmā sakaṃyeva gāmaṃ paccāgato’ti. Evameva kho, mahārāja, bhikkhu evaṃ samāhite citte parisuddhe pariyodāte anaṅgaṇe vigatūpakkilese mudubhūte kammaniye ṭhite āneñjappatte pubbenivāsānussatiñāṇāya cittaṃ abhinīharati abhininnāmeti. So anekavihitaṃ pubbenivāsaṃ anussarati, seyyathidaṃ – ekampi jātiṃ dvepi jātiyo tissopi jātiyo catassopi jātiyo pañcapi jātiyo dasapi jātiyo vīsampi jātiyo tiṃsampi jātiyo cattālīsampi jātiyo paññāsampi jātiyo jātisatampi jātisahassampi jātisatasahassampi anekepi saṃvaṭṭakappe anekepi vivaṭṭakappe anekepi saṃvaṭṭavivaṭṭakappe, ‘amutrāsiṃ evaṃnāmo evaṃgotto evaṃvaṇṇo evamāhāro evaṃsukhadukkhappaṭisaṃvedī evamāyupariyanto, so tato cuto amutra udapādiṃ; tatrāpāsiṃ evaṃnāmo evaṃgotto evaṃvaṇṇo evamāhāro evaṃsukhadukkhappaṭisaṃvedī evamāyupariyanto, so tato cuto idhūpapanno’ti, iti sākāraṃ sauddesaṃ anekavihitaṃ pubbenivāsaṃ anussarati. Idampi kho, mahārāja, sandiṭṭhikaṃ sāmaññaphalaṃ purimehi sandiṭṭhikehi sāmaññaphalehi abhikkantatarañca paṇītatarañca.

\subsubsection{Dibbacakkhuñāṇaṃ}

\paragraph{246.} ‘‘So evaṃ samāhite citte parisuddhe pariyodāte anaṅgaṇe vigatūpakkilese mudubhūte kammaniye ṭhite āneñjappatte sattānaṃ cutūpapātañāṇāya cittaṃ abhinīharati abhininnāmeti. So dibbena cakkhunā visuddhena atikkantamānusakena satte passati cavamāne upapajjamāne hīne paṇīte suvaṇṇe dubbaṇṇe sugate duggate, yathākammūpage satte pajānāti – ‘ime vata bhonto sattā kāyaduccaritena samannāgatā vacīduccaritena samannāgatā manoduccaritena samannāgatā ariyānaṃ upavādakā micchādiṭṭhikā micchādiṭṭhikammasamādānā. Te kāyassa bhedā paraṃ maraṇā apāyaṃ duggatiṃ vinipātaṃ nirayaṃ upapannā. Ime vā pana bhonto sattā kāyasucaritena samannāgatā vacīsucaritena samannāgatā manosucaritena samannāgatā ariyānaṃ anupavādakā sammādiṭṭhikā sammādiṭṭhikammasamādānā, te kāyassa bhedā paraṃ maraṇā sugatiṃ saggaṃ lokaṃ upapannā’ti. Iti dibbena cakkhunā visuddhena atikkantamānusakena satte passati cavamāne upapajjamāne hīne paṇīte suvaṇṇe dubbaṇṇe sugate duggate, yathākammūpage satte pajānāti.

\paragraph{247.} ‘‘Seyyathāpi, mahārāja, majjhe siṅghāṭake pāsādo. Tattha cakkhumā puriso ṭhito passeyya manusse gehaṃ pavisantepi nikkhamantepi rathikāyapi vīthiṃ sañcarante\footnote{rathiyāpī rathiṃ sañcarante (sī.), rathiyāya vithiṃ sañcarantepi (syā.)} majjhe siṅghāṭake nisinnepi. Tassa evamassa – ‘ete manussā gehaṃ pavisanti, ete nikkhamanti, ete rathikāya vīthiṃ sañcaranti, ete majjhe siṅghāṭake nisinnā’ti. Evameva kho, mahārāja, bhikkhu evaṃ samāhite citte parisuddhe pariyodāte anaṅgaṇe vigatūpakkilese mudubhūte kammaniye ṭhite āneñjappatte sattānaṃ cutūpapātañāṇāya cittaṃ abhinīharati abhininnāmeti. So dibbena cakkhunā visuddhena atikkantamānusakena satte passati cavamāne upapajjamāne hīne paṇīte suvaṇṇe dubbaṇṇe sugate duggate, yathākammūpage satte pajānāti – ‘ime vata bhonto sattā kāyaduccaritena samannāgatā vacīduccaritena samannāgatā manoduccaritena samannāgatā ariyānaṃ upavādakā micchādiṭṭhikā micchādiṭṭhikammasamādānā, te kāyassa bhedā paraṃ maraṇā apāyaṃ duggatiṃ vinipātaṃ nirayaṃ upapannā. Ime vā pana bhonto sattā kāyasucaritena samannāgatā vacīsucaritena samannāgatā manosucaritena samannāgatā ariyānaṃ anupavādakā sammādiṭṭhikā sammādiṭṭhikammasamādānā. Te kāyassa bhedā paraṃ maraṇā sugatiṃ saggaṃ lokaṃ upapannā’ti. Iti dibbena cakkhunā visuddhena atikkantamānusakena satte passati cavamāne upapajjamāne hīne paṇīte suvaṇṇe dubbaṇṇe sugate duggate; yathākammūpage satte pajānāti. ‘Idampi kho, mahārāja, sandiṭṭhikaṃ sāmaññaphalaṃ purimehi sandiṭṭhikehi sāmaññaphalehi abhikkantatarañca paṇītatarañca.

\subsubsection{Āsavakkhayañāṇaṃ}

\paragraph{248.} ‘‘So evaṃ samāhite citte parisuddhe pariyodāte anaṅgaṇe vigatūpakkilese mudubhūte kammaniye ṭhite āneñjappatte āsavānaṃ khayañāṇāya cittaṃ abhinīharati abhininnāmeti. So idaṃ dukkhanti yathābhūtaṃ pajānāti, ayaṃ dukkhasamudayoti yathābhūtaṃ pajānāti, ayaṃ dukkhanirodhoti yathābhūtaṃ pajānāti, ayaṃ dukkhanirodhagāminī paṭipadāti yathābhūtaṃ pajānāti. Ime āsavāti yathābhūtaṃ pajānāti, ayaṃ āsavasamudayoti yathābhūtaṃ pajānāti, ayaṃ āsavanirodhoti yathābhūtaṃ pajānāti, ayaṃ āsavanirodhagāminī paṭipadāti yathābhūtaṃ pajānāti. Tassa evaṃ jānato evaṃ passato kāmāsavāpi cittaṃ vimuccati, bhavāsavāpi cittaṃ vimuccati, avijjāsavāpi cittaṃ vimuccati, ‘vimuttasmiṃ vimuttami’ti ñāṇaṃ hoti, ‘khīṇā jāti, vusitaṃ brahmacariyaṃ, kataṃ karaṇīyaṃ, nāparaṃ itthattāyā’ti pajānāti.

\paragraph{249.} ‘‘Seyyathāpi, mahārāja, pabbatasaṅkhepe udakarahado accho vippasanno anāvilo. Tattha cakkhumā puriso tīre ṭhito passeyya sippisambukampi sakkharakathalampi macchagumbampi carantampi tiṭṭhantampi. Tassa evamassa – ‘ayaṃ kho udakarahado accho vippasanno anāvilo. Tatrime sippisambukāpi sakkharakathalāpi macchagumbāpi carantipi tiṭṭhantipī’ti. Evameva kho, mahārāja, bhikkhu evaṃ samāhite citte parisuddhe pariyodāte anaṅgaṇe vigatūpakkilese mudubhūte kammaniye ṭhite āneñjappatte āsavānaṃ khayañāṇāya cittaṃ abhinīharati abhininnāmeti. ‘So idaṃ dukkha’nti yathābhūtaṃ pajānāti, ‘ayaṃ dukkhasamudayo’ti yathābhūtaṃ pajānāti, ‘ayaṃ dukkhanirodho’ti yathābhūtaṃ pajānāti, ‘ayaṃ dukkhanirodhagāminī paṭipadā’ti yathābhūtaṃ pajānāti. ‘Ime āsavāti yathābhūtaṃ pajānāti, ‘ayaṃ āsavasamudayo’ti yathābhūtaṃ pajānāti, ‘ayaṃ āsavanirodho’ti yathābhūtaṃ pajānāti, ‘ayaṃ āsavanirodhagāminī paṭipadāti yathābhūtaṃ pajānāti. Tassa evaṃ jānato evaṃ passato kāmāsavāpi cittaṃ vimuccati, bhavāsavāpi cittaṃ vimuccati, avijjāsavāpi cittaṃ vimuccati, ‘vimuttasmiṃ vimuttamiti ñāṇaṃ hoti, ‘khīṇā jāti, vusitaṃ brahmacariyaṃ, kataṃ karaṇīyaṃ, nāparaṃ itthattāyā’ti pajānāti. Idaṃ kho, mahārāja, sandiṭṭhikaṃ sāmaññaphalaṃ purimehi sandiṭṭhikehi sāmaññaphalehi abhikkantatarañca paṇītatarañca. Imasmā ca pana, mahārāja, sandiṭṭhikā sāmaññaphalā aññaṃ sandiṭṭhikaṃ sāmaññaphalaṃ uttaritaraṃ vā paṇītataraṃ vā natthī’’ti.

\subsubsection{Ajātasattuupāsakattapaṭivedanā}

\paragraph{250.} Evaṃ vutte, rājā māgadho ajātasattu vedehiputto bhagavantaṃ etadavoca – ‘‘abhikkantaṃ, bhante, abhikkantaṃ, bhante. Seyyathāpi, bhante, nikkujjitaṃ vā ukkujjeyya, paṭicchannaṃ vā vivareyya, mūḷhassa vā maggaṃ ācikkheyya, andhakāre vā telapajjotaṃ dhāreyya ‘cakkhumanto rūpāni dakkhantī’ti; evamevaṃ, bhante, bhagavatā anekapariyāyena dhammo pakāsito. Esāhaṃ, bhante, bhagavantaṃ saraṇaṃ gacchāmi dhammañca bhikkhusaṅghañca. Upāsakaṃ maṃ bhagavā dhāretu ajjatagge pāṇupetaṃ saraṇaṃ gataṃ. Accayo maṃ, bhante, accagamā yathābālaṃ yathāmūḷhaṃ yathāakusalaṃ, yohaṃ pitaraṃ dhammikaṃ dhammarājānaṃ issariyakāraṇā jīvitā voropesiṃ. Tassa me, bhante bhagavā accayaṃ accayato paṭiggaṇhātu āyatiṃ saṃvarāyā’’ti.

\paragraph{251.} ‘‘Taggha tvaṃ, mahārāja, accayo accagamā yathābālaṃ yathāmūḷhaṃ yathāakusalaṃ, yaṃ tvaṃ pitaraṃ dhammikaṃ dhammarājānaṃ jīvitā voropesi. Yato ca kho tvaṃ, mahārāja, accayaṃ accayato disvā yathādhammaṃ paṭikarosi, taṃ te mayaṃ paṭiggaṇhāma. Vuddhihesā, mahārāja, ariyassa vinaye, yo accayaṃ accayato disvā yathādhammaṃ paṭikaroti, āyatiṃ saṃvaraṃ āpajjatī’’ti.

\paragraph{252.} Evaṃ vutte, rājā māgadho ajātasattu vedehiputto bhagavantaṃ etadavoca – ‘‘handa ca dāni mayaṃ, bhante, gacchāma bahukiccā mayaṃ bahukaraṇīyā’’ti. ‘‘Yassadāni tvaṃ, mahārāja, kālaṃ maññasī’’ti. Atha kho rājā māgadho ajātasattu vedehiputto bhagavato bhāsitaṃ abhinanditvā anumoditvā uṭṭhāyāsanā bhagavantaṃ abhivādetvā padakkhiṇaṃ katvā pakkāmi.

\paragraph{253.} Atha kho bhagavā acirapakkantassa rañño māgadhassa ajātasattussa vedehiputtassa bhikkhū āmantesi – ‘‘khatāyaṃ, bhikkhave, rājā. Upahatāyaṃ, bhikkhave, rājā. Sacāyaṃ, bhikkhave, rājā pitaraṃ dhammikaṃ dhammarājānaṃ jīvitā na voropessatha, imasmiññeva āsane virajaṃ vītamalaṃ dhammacakkhuṃ uppajjissathā’’ti. Idamavoca bhagavā. Attamanā te bhikkhū bhagavato bhāsitaṃ abhinandunti.

\xsectionEnd{Sāmaññaphalasuttaṃ niṭṭhitaṃ dutiyaṃ.}


\clearpage
\section{Ambaṭṭhasuttaṃ}

\paragraph{254.} Evaṃ me sutaṃ – ekaṃ samayaṃ bhagavā kosalesu cārikaṃ caramāno mahatā bhikkhusaṅghena saddhiṃ pañcamattehi bhikkhusatehi yena icchānaṅgalaṃ nāma kosalānaṃ brāhmaṇagāmo tadavasari. Tatra sudaṃ bhagavā icchānaṅgale viharati icchānaṅgalavanasaṇḍe.

\subsubsection{Pokkharasātivatthu}

\paragraph{255.} Tena kho pana samayena brāhmaṇo pokkharasāti ukkaṭṭhaṃ\footnote{pokkharasātī (sī.), pokkharasādi (pī.)} ajjhāvasati sattussadaṃ satiṇakaṭṭhodakaṃ sadhaññaṃ rājabhoggaṃ raññā pasenadinā kosalena dinnaṃ rājadāyaṃ brahmadeyyaṃ. Assosi kho brāhmaṇo pokkharasāti – ‘‘samaṇo khalu, bho, gotamo sakyaputto sakyakulā pabbajito kosalesu cārikaṃ caramāno mahatā bhikkhusaṅghena saddhiṃ pañcamattehi bhikkhusatehi icchānaṅgalaṃ anuppatto icchānaṅgale viharati icchānaṅgalavanasaṇḍe. Taṃ kho pana bhavantaṃ gotamaṃ evaṃ kalyāṇo kittisaddo abbhuggato – ‘itipi so bhagavā arahaṃ sammāsambuddho vijjācaraṇasampanno sugato lokavidū anuttaro purisadammasārathi satthā devamanussānaṃ buddho bhagavā’\footnote{bhagavāti (syā. kaṃ.), uparisoṇadaṇḍasuttādīsupi buddhaguṇakathāyaṃ evameva dissati}. So imaṃ lokaṃ sadevakaṃ samārakaṃ sabrahmakaṃ sassamaṇabrāhmaṇiṃ pajaṃ sadevamanussaṃ sayaṃ abhiññā sacchikatvā pavedeti. So dhammaṃ deseti ādikalyāṇaṃ majjhekalyāṇaṃ pariyosānakalyāṇaṃ, sātthaṃ sabyañjanaṃ, kevalaparipuṇṇaṃ parisuddhaṃ brahmacariyaṃ pakāseti. Sādhu kho pana tathārūpānaṃ arahataṃ dassanaṃ hotī’’ti.

\subsubsection{Ambaṭṭhamāṇavo}

\paragraph{256.} Tena kho pana samayena brāhmaṇassa pokkharasātissa ambaṭṭho nāma māṇavo antevāsī hoti ajjhāyako mantadharo tiṇṇaṃ vedānaṃ\footnote{bedānaṃ (ka.)} pāragū sanighaṇḍukeṭubhānaṃ sākkharappabhedānaṃ itihāsapañcamānaṃ padako veyyākaraṇo lokāyatamahāpurisalakkhaṇesu anavayo anuññātapaṭiññāto sake ācariyake tevijjake pāvacane – ‘‘yamahaṃ jānāmi, taṃ tvaṃ jānāsi; yaṃ tvaṃ jānāsi tamahaṃ jānāmī’’ti.

\paragraph{257.} Atha kho brāhmaṇo pokkharasāti ambaṭṭhaṃ māṇavaṃ āmantesi – ‘‘ayaṃ, tāta ambaṭṭha, samaṇo gotamo sakyaputto sakyakulā pabbajito kosalesu cārikaṃ caramāno mahatā bhikkhusaṅghena saddhiṃ pañcamattehi bhikkhusatehi icchānaṅgalaṃ anuppatto icchānaṅgale viharati icchānaṅgalavanasaṇḍe. Taṃ kho pana bhavantaṃ gotamaṃ evaṃ kalyāṇo kittisaddo abbhuggato – ‘itipi so bhagavā, arahaṃ sammāsambuddho vijjācaraṇasampanno sugato lokavidū anuttaro purisadammasārathi satthā devamanussānaṃ buddho bhagavā’. So imaṃ lokaṃ sadevakaṃ samārakaṃ sabrahmakaṃ sassamaṇabrāhmaṇiṃ pajaṃ sadevamanussaṃ sayaṃ abhiññā sacchikatvā pavedeti. So dhammaṃ deseti ādikalyāṇaṃ majjhekalyāṇaṃ pariyosānakalyāṇaṃ, sātthaṃ sabyañjanaṃ kevalaparipuṇṇaṃ parisuddhaṃ brahmacariyaṃ pakāseti. Sādhu kho pana tathārūpānaṃ arahataṃ dassanaṃ hotīti. Ehi tvaṃ tāta ambaṭṭha, yena samaṇo gotamo tenupasaṅkama; upasaṅkamitvā samaṇaṃ gotamaṃ jānāhi, yadi vā taṃ bhavantaṃ gotamaṃ tathāsantaṃyeva saddo abbhuggato, yadi vā no tathā. Yadi vā so bhavaṃ gotamo tādiso, yadi vā na tādiso, tathā mayaṃ taṃ bhavantaṃ gotamaṃ vedissāmā’’ti.

\paragraph{258.} ‘‘Yathā kathaṃ panāhaṃ, bho, taṃ bhavantaṃ gotamaṃ jānissāmi – ‘yadi vā taṃ bhavantaṃ gotamaṃ tathāsantaṃyeva saddo abbhuggato, yadi vā no tathā. Yadi vā so bhavaṃ gotamo tādiso, yadi vā na tādiso’’’ti? ‘‘Āgatāni kho, tāta ambaṭṭha, amhākaṃ mantesu dvattiṃsa mahāpurisalakkhaṇāni, yehi samannāgatassa mahāpurisassa dveyeva gatiyo bhavanti anaññā. Sace agāraṃ ajjhāvasati, rājā hoti cakkavattī dhammiko dhammarājā cāturanto vijitāvī janapadatthāvariyappatto sattaratanasamannāgato. Tassimāni satta ratanāni bhavanti. Seyyathidaṃ – cakkaratanaṃ, hatthiratanaṃ, assaratanaṃ, maṇiratanaṃ, itthiratanaṃ, gahapatiratanaṃ, pariṇāyakaratanameva sattamaṃ. Parosahassaṃ kho panassa puttā bhavanti sūrā vīraṅgarūpā parasenappamaddanā. So imaṃ pathaviṃ sāgarapariyantaṃ adaṇḍena asatthena dhammena abhivijiya ajjhāvasati. Sace kho pana agārasmā anagāriyaṃ pabbajati, arahaṃ hoti sammāsambuddho loke vivaṭṭacchado. Ahaṃ kho pana, tāta ambaṭṭha, mantānaṃ dātā; tvaṃ mantānaṃ paṭiggahetā’’ti.

\paragraph{259.} ‘‘Evaṃ, bho’’ti kho ambaṭṭho māṇavo brāhmaṇassa pokkharasātissa paṭissutvā uṭṭhāyāsanā brāhmaṇaṃ pokkharasātiṃ abhivādetvā padakkhiṇaṃ katvā vaḷavārathamāruyha sambahulehi māṇavakehi saddhiṃ yena icchānaṅgalavanasaṇḍo tena pāyāsi. Yāvatikā yānassa bhūmi yānena gantvā yānā paccorohitvā pattikova ārāmaṃ pāvisi. Tena kho pana samayena sambahulā bhikkhū abbhokāse caṅkamanti. Atha kho ambaṭṭho māṇavo yena te bhikkhū tenupasaṅkami; upasaṅkamitvā te bhikkhū etadavoca – ‘‘kahaṃ nu kho, bho, etarahi so bhavaṃ gotamo viharati? Tañhi mayaṃ bhavantaṃ gotamaṃ dassanāya idhūpasaṅkantā’’ti.

\paragraph{260.} Atha kho tesaṃ bhikkhūnaṃ etadahosi – ‘‘ayaṃ kho ambaṭṭho māṇavo abhiññātakolañño ceva abhiññātassa ca brāhmaṇassa pokkharasātissa antevāsī. Agaru kho pana bhagavato evarūpehi kulaputtehi saddhiṃ kathāsallāpo hotī’’ti. Te ambaṭṭhaṃ māṇavaṃ etadavocuṃ – ‘‘eso ambaṭṭha vihāro saṃvutadvāro, tena appasaddo upasaṅkamitvā ataramāno āḷindaṃ pavisitvā ukkāsitvā aggaḷaṃ ākoṭehi, vivarissati te bhagavā dvāra’’nti.

\paragraph{261.} Atha kho ambaṭṭho māṇavo yena so vihāro saṃvutadvāro, tena appasaddo upasaṅkamitvā ataramāno āḷindaṃ pavisitvā ukkāsitvā aggaḷaṃ ākoṭesi. Vivari bhagavā dvāraṃ. Pāvisi ambaṭṭho māṇavo. Māṇavakāpi pavisitvā bhagavatā saddhiṃ sammodiṃsu, sammodanīyaṃ kathaṃ sāraṇīyaṃ vītisāretvā ekamantaṃ nisīdiṃsu. Ambaṭṭho pana māṇavo caṅkamantopi nisinnena bhagavatā kañci kañci\footnote{kiñci kiñci (ka.)} kathaṃ sāraṇīyaṃ vītisāreti, ṭhitopi nisinnena bhagavatā kiñci kiñci kathaṃ sāraṇīyaṃ vītisāreti.

\paragraph{262.} Atha kho bhagavā ambaṭṭhaṃ māṇavaṃ etadavoca – ‘‘evaṃ nu te, ambaṭṭha, brāhmaṇehi vuddhehi mahallakehi ācariyapācariyehi saddhiṃ kathāsallāpo hoti, yathayidaṃ caraṃ tiṭṭhaṃ nisinnena mayā kiñci kiñci kathaṃ sāraṇīyaṃ vītisāretī’’ti?

\subsubsection{Paṭhamaibbhavādo}

\paragraph{263.} ‘‘No hidaṃ, bho gotama. Gacchanto vā hi, bho gotama, gacchantena brāhmaṇo brāhmaṇena saddhiṃ sallapitumarahati, ṭhito vā hi, bho gotama, ṭhitena brāhmaṇo brāhmaṇena saddhiṃ sallapitumarahati, nisinno vā hi, bho gotama, nisinnena brāhmaṇo brāhmaṇena saddhiṃ sallapitumarahati, sayāno vā hi, bho gotama, sayānena brāhmaṇo brāhmaṇena saddhiṃ sallapitumarahati. Ye ca kho te, bho gotama, muṇḍakā samaṇakā ibbhā kaṇhā\footnote{kiṇhā (ka. sī. pī.)} bandhupādāpaccā, tehipi me saddhiṃ evaṃ kathāsallāpo hoti, yathariva bhotā gotamenā’’ti. ‘‘Atthikavato kho pana te, ambaṭṭha, idhāgamanaṃ ahosi, yāyeva kho panatthāya āgaccheyyātha\footnote{āgaccheyyātho (sī. pī.)}, tameva atthaṃ sādhukaṃ manasi kareyyātha\footnote{manasikareyyātho (sī. pī.)}. Avusitavāyeva kho pana bho ayaṃ ambaṭṭho māṇavo vusitamānī kimaññatra avusitattā’’ti.

\paragraph{264.} Atha kho ambaṭṭho māṇavo bhagavatā avusitavādena vuccamāno kupito anattamano bhagavantaṃyeva khuṃsento bhagavantaṃyeva vambhento bhagavantaṃyeva upavadamāno – ‘‘samaṇo ca me, bho, gotamo pāpito bhavissatī’’ti bhagavantaṃ etadavoca – ‘‘caṇḍā, bho gotama, sakyajāti; pharusā, bho gotama, sakyajāti; lahusā, bho gotama, sakyajāti; bhassā, bho gotama, sakyajāti; ibbhā santā ibbhā samānā na brāhmaṇe sakkaronti, na brāhmaṇe garuṃ karonti\footnote{garukaronti (sī. syā. kaṃ. pī.)}, na brāhmaṇe mānenti, na brāhmaṇe pūjenti, na brāhmaṇe apacāyanti. Tayidaṃ, bho gotama, nacchannaṃ, tayidaṃ nappatirūpaṃ, yadime sakyā ibbhā santā ibbhā samānā na brāhmaṇe sakkaronti, na brāhmaṇe garuṃ karonti, na brāhmaṇe mānenti, na brāhmaṇe pūjenti, na brāhmaṇe apacāyantī’’ti. Itiha ambaṭṭho māṇavo idaṃ paṭhamaṃ sakyesu ibbhavādaṃ nipātesi.

\subsubsection{Dutiyaibbhavādo}

\paragraph{265.} ‘‘Kiṃ pana te, ambaṭṭha, sakyā aparaddhu’’nti? ‘‘Ekamidāhaṃ, bho gotama, samayaṃ ācariyassa brāhmaṇassa pokkharasātissa kenacideva karaṇīyena kapilavatthuṃ agamāsiṃ. Yena sakyānaṃ sandhāgāraṃ tenupasaṅkamiṃ. Tena kho pana samayena sambahulā sakyā ceva sakyakumārā ca sandhāgāre\footnote{santhāgāre (sī. pī.)} uccesu āsanesu nisinnā honti aññamaññaṃ aṅgulipatodakehi\footnote{aṅgulipatodakena (pī.)} sañjagghantā saṃkīḷantā, aññadatthu mamaññeva maññe anujagghantā, na maṃ koci āsanenapi nimantesi. Tayidaṃ, bho gotama, nacchannaṃ, tayidaṃ nappatirūpaṃ, yadime sakyā ibbhā santā ibbhā samānā na brāhmaṇe sakkaronti, na brāhmaṇe garuṃ karonti, na brāhmaṇe mānenti, na brāhmaṇe pūjenti, na brāhmaṇe apacāyantī’’ti. Itiha ambaṭṭho māṇavo idaṃ dutiyaṃ sakyesu ibbhavādaṃ nipātesi.

\subsubsection{Tatiyaibbhavādo}

\paragraph{266.} ‘‘Laṭukikāpi kho, ambaṭṭha, sakuṇikā sake kulāvake kāmalāpinī hoti. Sakaṃ kho panetaṃ, ambaṭṭha, sakyānaṃ yadidaṃ kapilavatthuṃ, nārahatāyasmā ambaṭṭho imāya appamattāya abhisajjitu’’nti. ‘‘Cattārome, bho gotama, vaṇṇā – khattiyā brāhmaṇā vessā suddā. Imesañhi, bho gotama, catunnaṃ vaṇṇānaṃ tayo vaṇṇā – khattiyā ca vessā ca suddā ca – aññadatthu brāhmaṇasseva paricārakā sampajjanti. Tayidaṃ, bho gotama, nacchannaṃ, tayidaṃ nappatirūpaṃ, yadime sakyā ibbhā santā ibbhā samānā na brāhmaṇe sakkaronti, na brāhmaṇe garuṃ karonti, na brāhmaṇe mānenti, na brāhmaṇe pūjenti, na brāhmaṇe apacāyantī’’ti. Itiha ambaṭṭho māṇavo idaṃ tatiyaṃ sakyesu ibbhavādaṃ nipātesi.

\subsubsection{Dāsiputtavādo}

\paragraph{267.} Atha kho bhagavato etadahosi – ‘‘atibāḷhaṃ kho ayaṃ ambaṭṭho māṇavo sakyesu ibbhavādena nimmādeti, yaṃnūnāhaṃ gottaṃ puccheyya’’nti. Atha kho bhagavā ambaṭṭhaṃ māṇavaṃ etadavoca – ‘‘kathaṃ gottosi, ambaṭṭhā’’ti? ‘‘Kaṇhāyanohamasmi, bho gotamā’’ti. Porāṇaṃ kho pana te ambaṭṭha mātāpettikaṃ nāmagottaṃ anussarato ayyaputtā sakyā bhavanti; dāsiputto tvamasi sakyānaṃ. Sakyā kho pana, ambaṭṭha, rājānaṃ okkākaṃ pitāmahaṃ dahanti. ‘‘Bhūtapubbaṃ, ambaṭṭha, rājā okkāko yā sā mahesī piyā manāpā, tassā puttassa rajjaṃ pariṇāmetukāmo jeṭṭhakumāre raṭṭhasmā pabbājesi – okkāmukhaṃ karakaṇḍaṃ\footnote{ukkāmukhaṃ karakaṇḍuṃ (sī. syā.)} hatthinikaṃ sinisūraṃ\footnote{sinipuraṃ (sī. syā.)}. Te raṭṭhasmā pabbājitā himavantapasse pokkharaṇiyā tīre mahāsākasaṇḍo, tattha vāsaṃ kappesuṃ. Te jātisambhedabhayā sakāhi bhaginīhi saddhiṃ saṃvāsaṃ kappesuṃ. ‘‘Atha kho, ambaṭṭha, rājā okkāko amacce pārisajje āmantesi – ‘kahaṃ nu kho, bho, etarahi kumārā sammantī’ti? ‘Atthi, deva, himavantapasse pokkharaṇiyā tīre mahāsākasaṇḍo, tatthetarahi kumārā sammanti. Te jātisambhedabhayā sakāhi bhaginīhi saddhiṃ saṃvāsaṃ kappentī’ti. Atha kho, ambaṭṭha, rājā okkāko udānaṃ udānesi – ‘sakyā vata, bho, kumārā, paramasakyā vata, bho, kumārā’ti. Tadagge kho pana ambaṭṭha sakyā paññāyanti; so ca nesaṃ pubbapuriso. ‘‘Rañño kho pana, ambaṭṭha, okkākassa disā nāma dāsī ahosi. Sā kaṇhaṃ nāma\footnote{sā kaṇhaṃ (pī.)} janesi. Jāto kaṇho pabyāhāsi – ‘dhovatha maṃ, amma, nahāpetha maṃ amma, imasmā maṃ asucismā parimocetha, atthāya vo bhavissāmī’ti. Yathā kho pana ambaṭṭha etarahi manussā pisāce disvā ‘pisācā’ti sañjānanti; evameva kho, ambaṭṭha, tena kho pana samayena manussā pisāce ‘kaṇhā’ti sañjānanti. Te evamāhaṃsu – ‘ayaṃ jāto pabyāhāsi, kaṇho jāto, pisāco jāto’ti. Tadagge kho pana, ambaṭṭha kaṇhāyanā paññāyanti, so ca kaṇhāyanānaṃ pubbapuriso. Iti kho te, ambaṭṭha, porāṇaṃ mātāpettikaṃ nāmagottaṃ anussarato ayyaputtā sakyā bhavanti, dāsiputto tvamasi sakyāna’’nti.

\paragraph{268.} Evaṃ vutte, te māṇavakā bhagavantaṃ etadavocuṃ – ‘‘mā bhavaṃ gotamo ambaṭṭhaṃ atibāḷhaṃ dāsiputtavādena nimmādesi. Sujāto ca, bho gotama ambaṭṭho māṇavo, kulaputto ca ambaṭṭho māṇavo, bahussuto ca ambaṭṭho māṇavo, kalyāṇavākkaraṇo ca ambaṭṭho māṇavo, paṇḍito ca ambaṭṭho māṇavo, pahoti ca ambaṭṭho māṇavo bhotā gotamena saddhiṃ asmiṃ vacane paṭimantetu’’nti.

\paragraph{269.} Atha kho bhagavā te māṇavake etadavoca – ‘‘sace kho tumhākaṃ māṇavakānaṃ evaṃ hoti – ‘dujjāto ca ambaṭṭho māṇavo, akulaputto ca ambaṭṭho māṇavo, appassuto ca ambaṭṭho māṇavo, akalyāṇavākkaraṇo ca ambaṭṭho māṇavo, duppañño ca ambaṭṭho māṇavo, na ca pahoti ambaṭṭho māṇavo samaṇena gotamena saddhiṃ asmiṃ vacane paṭimantetu’nti, tiṭṭhatu ambaṭṭho māṇavo, tumhe mayā saddhiṃ mantavho asmiṃ vacane. Sace pana tumhākaṃ māṇavakānaṃ evaṃ hoti – ‘sujāto ca ambaṭṭho māṇavo, kulaputto ca ambaṭṭho māṇavo, bahussuto ca ambaṭṭho māṇavo, kalyāṇavākkaraṇo ca ambaṭṭho māṇavo, paṇḍito ca ambaṭṭho māṇavo, pahoti ca ambaṭṭho māṇavo samaṇena gotamena saddhiṃ asmiṃ vacane paṭimantetu’nti, tiṭṭhatha tumhe; ambaṭṭho māṇavo mayā saddhiṃ paṭimantetū’’ti. ‘‘Sujāto ca, bho gotama, ambaṭṭho māṇavo, kulaputto ca ambaṭṭho māṇavo, bahussuto ca ambaṭṭho māṇavo, kalyāṇavākkaraṇo ca ambaṭṭho māṇavo, paṇḍito ca ambaṭṭho māṇavo, pahoti ca ambaṭṭho māṇavo bhotā gotamena saddhiṃ asmiṃ vacane paṭimantetuṃ, tuṇhī mayaṃ bhavissāma, ambaṭṭho māṇavo bhotā gotamena saddhiṃ asmiṃ vacane paṭimantetū’’ti.

\paragraph{270.} Atha kho bhagavā ambaṭṭhaṃ māṇavaṃ etadavoca – ‘‘ayaṃ kho pana te, ambaṭṭha, sahadhammiko pañho āgacchati, akāmā byākātabbo. Sace tvaṃ na byākarissasi, aññena vā aññaṃ paṭicarissasi, tuṇhī vā bhavissasi, pakkamissasi vā ettheva te sattadhā muddhā phalissati. Taṃ kiṃ maññasi, ambaṭṭha, kinti te sutaṃ brāhmaṇānaṃ vuddhānaṃ mahallakānaṃ ācariyapācariyānaṃ bhāsamānānaṃ kutopabhutikā kaṇhāyanā, ko ca kaṇhāyanānaṃ pubbapuriso’’ti? Evaṃ vutte, ambaṭṭho māṇavo tuṇhī ahosi. Dutiyampi kho bhagavā ambaṭṭhaṃ māṇavaṃ etadavoca – ‘‘taṃ kiṃ maññasi, ambaṭṭha, kinti te sutaṃ brāhmaṇānaṃ vuddhānaṃ mahallakānaṃ ācariyapācariyānaṃ bhāsamānānaṃ kutopabhutikā kaṇhāyanā, ko ca kaṇhāyanānaṃ pubbapuriso’’ti? Dutiyampi kho ambaṭṭho māṇavo tuṇhī ahosi. Atha kho bhagavā ambaṭṭhaṃ māṇavaṃ etadavoca – ‘‘byākarohi dāni ambaṭṭha, na dāni, te tuṇhībhāvassa kālo. Yo kho, ambaṭṭha, tathāgatena yāvatatiyakaṃ sahadhammikaṃ pañhaṃ puṭṭho na byākaroti, etthevassa sattadhā muddhā phalissatī’’ti.

\paragraph{271.} Tena kho pana samayena vajirapāṇī yakkho mahantaṃ ayokūṭaṃ ādāya ādittaṃ sampajjalitaṃ sajotibhūtaṃ\footnote{sañjotibhūtaṃ (syā.)} ambaṭṭhassa māṇavassa upari vehāsaṃ ṭhito hoti – ‘‘sacāyaṃ ambaṭṭho māṇavo bhagavatā yāvatatiyakaṃ sahadhammikaṃ pañhaṃ puṭṭho na byākarissati, etthevassa sattadhā muddhaṃ phālessāmī’’ti. Taṃ kho pana vajirapāṇiṃ yakkhaṃ bhagavā ceva passati ambaṭṭho ca māṇavo.

\paragraph{272.} Atha kho ambaṭṭho māṇavo bhīto saṃviggo lomahaṭṭhajāto bhagavantaṃyeva tāṇaṃ gavesī bhagavantaṃyeva leṇaṃ gavesī bhagavantaṃyeva saraṇaṃ gavesī – upanisīditvā bhagavantaṃ etadavoca – ‘‘kimetaṃ\footnote{kiṃ me taṃ (ka.)} bhavaṃ gotamo āha? Punabhavaṃ gotamo bravitū’’ti\footnote{brūtu (syā.)}. ‘‘Taṃ kiṃ maññasi, ambaṭṭha, kinti te sutaṃ brāhmaṇānaṃ vuddhānaṃ mahallakānaṃ ācariyapācariyānaṃ bhāsamānānaṃ kutopabhutikā kaṇhāyanā, ko ca kaṇhāyanānaṃ pubbapuriso’’ti? ‘‘Evameva me, bho gotama, sutaṃ yatheva bhavaṃ gotamo āha. Tatopabhutikā kaṇhāyanā; so ca kaṇhāyanānaṃ pubbapuriso’’ti.

\subsubsection{Ambaṭṭhavaṃsakathā}

\paragraph{273.} Evaṃ vutte, te māṇavakā unnādino uccāsaddamahāsaddā ahesuṃ – ‘‘dujjāto kira, bho, ambaṭṭho māṇavo; akulaputto kira, bho, ambaṭṭho māṇavo; dāsiputto kira, bho, ambaṭṭho māṇavo sakyānaṃ. Ayyaputtā kira, bho, ambaṭṭhassa māṇavassa sakyā bhavanti. Dhammavādiṃyeva kira mayaṃ samaṇaṃ gotamaṃ apasādetabbaṃ amaññimhā’’ti.

\paragraph{274.} Atha kho bhagavato etadahosi – ‘‘atibāḷhaṃ kho ime māṇavakā ambaṭṭhaṃ māṇavaṃ dāsiputtavādena nimmādenti, yaṃnūnāhaṃ parimoceyya’’nti. Atha kho bhagavā te māṇavake etadavoca – ‘‘mā kho tumhe, māṇavakā, ambaṭṭhaṃ māṇavaṃ atibāḷhaṃ dāsiputtavādena nimmādetha. Uḷāro so kaṇho isi ahosi. So dakkhiṇajanapadaṃ gantvā brahmamante adhīyitvā rājānaṃ okkākaṃ upasaṅkamitvā maddarūpiṃ dhītaraṃ yāci. Tassa rājā okkāko – ‘ko nevaṃ re ayaṃ mayhaṃ dāsiputto samāno maddarūpiṃ dhītaraṃ yācatī’’’ ti, kupito anattamano khurappaṃ sannayhi\footnote{sannahi (ka.)}. So taṃ khurappaṃ neva asakkhi muñcituṃ, no paṭisaṃharituṃ. ‘‘Atha kho, māṇavakā, amaccā pārisajjā kaṇhaṃ isiṃ upasaṅkamitvā etadavocuṃ – ‘sotthi, bhaddante\footnote{bhadante (sī. syā.)}, hotu rañño; sotthi, bhaddante, hotu rañño’ti. ‘Sotthi bhavissati rañño, api ca rājā yadi adho khurappaṃ muñcissati, yāvatā rañño vijitaṃ, ettāvatā pathavī undriyissatī’ti. ‘Sotthi, bhaddante, hotu rañño, sotthi janapadassā’ti. ‘Sotthi bhavissati rañño, sotthi janapadassa, api ca rājā yadi uddhaṃ khurappaṃ muñcissati, yāvatā rañño vijitaṃ, ettāvatā satta vassāni devo na vassissatī’ti. ‘Sotthi, bhaddante, hotu rañño sotthi janapadassa devo ca vassatū’ti. ‘Sotthi bhavissati rañño sotthi janapadassa devo ca vassissati, api ca rājā jeṭṭhakumāre khurappaṃ patiṭṭhāpetu, sotthi kumāro pallomo bhavissatī’ti. Atha kho, māṇavakā, amaccā okkākassa ārocesuṃ – ‘okkāko jeṭṭhakumāre khurappaṃ patiṭṭhāpetu. Sotthi kumāro pallomo bhavissatī’ti. Atha kho rājā okkāko jeṭṭhakumāre khurappaṃ patiṭṭhapesi, sotthi kumāro pallomo samabhavi. Atha kho tassa rājā okkāko bhīto saṃviggo lomahaṭṭhajāto brahmadaṇḍena tajjito maddarūpiṃ dhītaraṃ adāsi. Mā kho tumhe, māṇavakā, ambaṭṭhaṃ māṇavaṃ atibāḷhaṃ dāsiputtavādena nimmādetha, uḷāro so kaṇho isi ahosī’’ti.

\subsubsection{Khattiyaseṭṭhabhāvo}

\paragraph{275.} Atha kho bhagavā ambaṭṭhaṃ māṇavaṃ āmantesi – ‘‘taṃ kiṃ maññasi, ambaṭṭha, idha khattiyakumāro brāhmaṇakaññāya saddhiṃ saṃvāsaṃ kappeyya, tesaṃ saṃvāsamanvāya putto jāyetha. Yo so khattiyakumārena brāhmaṇakaññāya putto uppanno, api nu so labhetha brāhmaṇesu āsanaṃ vā udakaṃ vā’’ti? ‘‘Labhetha, bho gotama’’. ‘‘Apinu naṃ brāhmaṇā bhojeyyuṃ saddhe vā thālipāke vā yaññe vā pāhune vā’’ti? ‘‘Bhojeyyuṃ, bho gotama’’. ‘‘Apinu naṃ brāhmaṇā mante vāceyyuṃ vā no vā’’ti? ‘‘Vāceyyuṃ, bho gotama’’. ‘‘Apinussa itthīsu āvaṭaṃ vā assa anāvaṭaṃ vā’’ti? ‘‘Anāvaṭaṃ hissa, bho gotama’’. ‘‘Apinu naṃ khattiyā khattiyābhisekena abhisiñceyyu’’nti? ‘‘No hidaṃ, bho gotama’’. ‘‘Taṃ kissa hetu’’? ‘‘Mātito hi, bho gotama, anupapanno’’ti. ‘‘Taṃ kiṃ maññasi, ambaṭṭha, idha brāhmaṇakumāro khattiyakaññāya saddhiṃ saṃvāsaṃ kappeyya, tesaṃ saṃvāsamanvāya putto jāyetha. Yo so brāhmaṇakumārena khattiyakaññāya putto uppanno, apinu so labhetha brāhmaṇesu āsanaṃ vā udakaṃ vā’’ti? ‘‘Labhetha, bho gotama’’. ‘‘Apinu naṃ brāhmaṇā bhojeyyuṃ saddhe vā thālipāke vā yaññe vā pāhune vā’’ti? ‘‘Bhojeyyuṃ, bho gotama’’. ‘‘Apinu naṃ brāhmaṇā mante vāceyyuṃ vā no vā’’ti? ‘‘Vāceyyuṃ, bho gotama’’. ‘‘Apinussa itthīsu āvaṭaṃ vā assa anāvaṭaṃ vā’’ti? ‘‘Anāvaṭaṃ hissa, bho gotama’’. ‘‘Apinu naṃ khattiyā khattiyābhisekena abhisiñceyyu’’nti? ‘‘No hidaṃ, bho gotama’’. ‘‘Taṃ kissa hetu’’? ‘‘Pitito hi, bho gotama, anupapanno’’ti.

\paragraph{276.} ‘‘Iti kho, ambaṭṭha, itthiyā vā itthiṃ karitvā purisena vā purisaṃ karitvā khattiyāva seṭṭhā, hīnā brāhmaṇā. Taṃ kiṃ maññasi, ambaṭṭha, idha brāhmaṇā brāhmaṇaṃ kismiñcideva pakaraṇe khuramuṇḍaṃ karitvā bhassapuṭena vadhitvā raṭṭhā vā nagarā vā pabbājeyyuṃ. Apinu so labhetha brāhmaṇesu āsanaṃ vā udakaṃ vā’’ti? ‘‘No hidaṃ, bho gotama’’. ‘‘Apinu naṃ brāhmaṇā bhojeyyuṃ saddhe vā thālipāke vā yaññe vā pāhune vā’’ti? ‘‘No hidaṃ, bho gotama’’. ‘‘Apinu naṃ brāhmaṇā mante vāceyyuṃ vā no vā’’ti? ‘‘No hidaṃ, bho gotama’’. ‘‘Apinussa itthīsu āvaṭaṃ vā assa anāvaṭaṃ vā’’ti? ‘‘Āvaṭaṃ hissa, bho gotama’’. ‘‘Taṃ kiṃ maññasi, ambaṭṭha, idha khattiyā khattiyaṃ kismiñcideva pakaraṇe khuramuṇḍaṃ karitvā bhassapuṭena vadhitvā raṭṭhā vā nagarā vā pabbājeyyuṃ. Apinu so labhetha brāhmaṇesu āsanaṃ vā udakaṃ vā’’ti? ‘‘Labhetha, bho gotama’’. ‘‘Apinu naṃ brāhmaṇā bhojeyyuṃ saddhe vā thālipāke vā yaññe vā pāhune vā’’ti? ‘‘Bhojeyyuṃ, bho gotama’’. ‘‘Apinu naṃ brāhmaṇā mante vāceyyuṃ vā no vā’’ti? ‘‘Vāceyyuṃ, bho gotama’’. ‘‘Apinussa itthīsu āvaṭaṃ vā assa anāvaṭaṃ vā’’ti? ‘‘Anāvaṭaṃ hissa, bho gotama’’.

\paragraph{277.} ‘‘Ettāvatā kho, ambaṭṭha, khattiyo paramanihīnataṃ patto hoti, yadeva naṃ khattiyā khuramuṇḍaṃ karitvā bhassapuṭena vadhitvā raṭṭhā vā nagarā vā pabbājenti. Iti kho, ambaṭṭha, yadā khattiyo paramanihīnataṃ patto hoti, tadāpi khattiyāva seṭṭhā, hīnā brāhmaṇā. Brahmunā pesā, ambaṭṭha\footnote{brahmunāpi ambaṭṭha (ka.), brahmunāpi eso ambaṭṭha (pī.)}, sanaṅkumārena gāthā bhāsitā –

\begin{verse}
  ‘Khattiyo seṭṭho janetasmiṃ,\\
  Ye gottapaṭisārino;\\
  Vijjācaraṇasampanno,\\
  So seṭṭho devamānuse’ti.\\
\end{verse}
‘‘Sā kho panesā, ambaṭṭha, brahmunā sanaṅkumārena gāthā sugītā no duggītā, subhāsitā no dubbhāsitā, atthasaṃhitā no anatthasaṃhitā, anumatā mayā. Ahampi hi, ambaṭṭha, evaṃ vadāmi –

\begin{verse}
  ‘Khattiyo seṭṭho janetasmiṃ,\\
  Ye gottapaṭisārino;\\
  Vijjācaraṇasampanno,\\
  So seṭṭho devamānuse’ti.\\
\end{verse}

\xsubsubsectionEnd{Bhāṇavāro paṭhamo.}

\subsubsection{Vijjācaraṇakathā}

\paragraph{278.} ‘‘Katamaṃ pana taṃ, bho gotama, caraṇaṃ, katamā ca pana sā vijjā’’ti? ‘‘Na kho, ambaṭṭha, anuttarāya vijjācaraṇasampadāya jātivādo vā vuccati, gottavādo vā vuccati, mānavādo vā vuccati – ‘arahasi vā maṃ tvaṃ, na vā maṃ tvaṃ arahasī’ti. Yattha kho, ambaṭṭha, āvāho vā hoti, vivāho vā hoti, āvāhavivāho vā hoti, etthetaṃ vuccati jātivādo vā itipi gottavādo vā itipi mānavādo vā itipi – ‘arahasi vā maṃ tvaṃ, na vā maṃ tvaṃ arahasī’ti. Ye hi keci ambaṭṭha jātivādavinibaddhā vā gottavādavinibaddhā vā mānavādavinibaddhā vā āvāhavivāhavinibaddhā vā, ārakā te anuttarāya vijjācaraṇasampadāya. Pahāya kho, ambaṭṭha, jātivādavinibaddhañca gottavādavinibaddhañca mānavādavinibaddhañca āvāhavivāhavinibaddhañca anuttarāya vijjācaraṇasampadāya sacchikiriyā hotī’’ti.

\paragraph{279.} ‘‘Katamaṃ pana taṃ, bho gotama, caraṇaṃ, katamā ca sā vijjā’’ti? ‘‘Idha, ambaṭṭha, tathāgato loke uppajjati arahaṃ sammāsambuddho vijjācaraṇasampanno sugato lokavidū anuttaro purisadammasārathi satthā devamanussānaṃ buddho bhagavā. So imaṃ lokaṃ sadevakaṃ samārakaṃ sabrahmakaṃ sassamaṇabrāhmaṇiṃ pajaṃ sadevamanussaṃ sayaṃ abhiññā sacchikatvā pavedeti. So dhammaṃ deseti ādikalyāṇaṃ majjhekalyāṇaṃ pariyosānakalyāṇaṃ sātthaṃ sabyañjanaṃ kevalaparipuṇṇaṃ parisuddhaṃ brahmacariyaṃ pakāseti. Taṃ dhammaṃ suṇāti gahapati vā gahapatiputto vā aññatarasmiṃ vā kule paccājāto. So taṃ dhammaṃ sutvā tathāgate saddhaṃ paṭilabhati. So tena saddhāpaṭilābhena samannāgato iti paṭisañcikkhati…pe… (yathā 191 ādayo anucchedā, evaṃ vitthāretabbaṃ).… ‘‘So vivicceva kāmehi, vivicca akusalehi dhammehi, savitakkaṃ savicāraṃ vivekajaṃ pītisukhaṃ paṭhamaṃ jhānaṃ upasampajja viharati…pe… idampissa hoti caraṇasmiṃ. ‘‘Puna caparaṃ, ambaṭṭha, bhikkhu vitakkavicārānaṃ vūpasamā ajjhattaṃ sampasādanaṃ cetaso ekodibhāvaṃ avitakkaṃ avicāraṃ samādhijaṃ pītisukhaṃ dutiyaṃ jhānaṃ upasampajja viharati…pe… idampissa hoti caraṇasmiṃ. ‘‘Puna caparaṃ, ambaṭṭha, bhikkhu pītiyā ca virāgā upekkhako ca viharati sato ca sampajāno, sukhañca kāyena paṭisaṃvedeti, yaṃ taṃ ariyā ācikkhanti – ‘‘upekkhako satimā sukhavihārī’ti, tatiyaṃ jhānaṃ upasampajja viharati…pe… idampissa hoti caraṇasmiṃ. ‘‘Puna caparaṃ, ambaṭṭha, bhikkhu sukhassa ca pahānā dukkhassa ca pahānā, pubbeva somanassadomanassānaṃ atthaṅgamā adukkhamasukhaṃ upekkhāsatipārisuddhiṃ catutthaṃ jhānaṃ upasampajja viharati…pe… idampissa hoti caraṇasmiṃ. Idaṃ kho taṃ, ambaṭṭha, caraṇaṃ. ‘‘So evaṃ samāhite citte parisuddhe pariyodāte anaṅgaṇe vigatūpakkilese mudubhūte kammaniye ṭhite āneñjappatte ñāṇadassanāya cittaṃ abhinīharati abhininnāmeti…pe… idampissa hoti vijjāya…pe… nāparaṃ itthattāyāti pajānāti, idampissa hoti vijjāya. Ayaṃ kho sā, ambaṭṭha, vijjā. ‘‘Ayaṃ vuccati, ambaṭṭha, bhikkhu ‘vijjāsampanno’ itipi, ‘caraṇasampanno’ itipi, ‘vijjācaraṇasampanno’ itipi. Imāya ca ambaṭṭha vijjāsampadāya caraṇasampadāya ca aññā vijjāsampadā ca caraṇasampadā ca uttaritarā vā paṇītatarā vā natthi.

\subsubsection{Catuapāyamukhaṃ}

\paragraph{280.} ‘‘Imāya kho, ambaṭṭha, anuttarāya vijjācaraṇasampadāya cattāri apāyamukhāni bhavanti. Katamāni cattāri? Idha, ambaṭṭha, ekacco samaṇo vā brāhmaṇo vā imaññeva anuttaraṃ vijjācaraṇasampadaṃ anabhisambhuṇamāno khārividhamādāya\footnote{khārivividhamādāya (sī. syā. pī.)} araññāyatanaṃ ajjhogāhati – ‘pavattaphalabhojano bhavissāmī’ti. So aññadatthu vijjācaraṇasampannasseva paricārako sampajjati. Imāya kho, ambaṭṭha, anuttarāya vijjācaraṇasampadāya idaṃ paṭhamaṃ apāyamukhaṃ bhavati. ‘‘Puna caparaṃ, ambaṭṭha, idhekacco samaṇo vā brāhmaṇo vā imañceva anuttaraṃ vijjācaraṇasampadaṃ anabhisambhuṇamāno pavattaphalabhojanatañca anabhisambhuṇamāno kudālapiṭakaṃ\footnote{kuddālapiṭakaṃ (sī. syā. pī.)} ādāya araññavanaṃ ajjhogāhati – ‘kandamūlaphalabhojano bhavissāmī’ti. So aññadatthu vijjācaraṇasampannasseva paricārako sampajjati. Imāya kho, ambaṭṭha, anuttarāya vijjācaraṇasampadāya idaṃ dutiyaṃ apāyamukhaṃ bhavati. ‘‘Puna caparaṃ, ambaṭṭha, idhekacco samaṇo vā brāhmaṇo vā imañceva anuttaraṃ vijjācaraṇasampadaṃ anabhisambhuṇamāno pavattaphalabhojanatañca anabhisambhuṇamāno kandamūlaphalabhojanatañca anabhisambhuṇamāno gāmasāmantaṃ vā nigamasāmantaṃ vā agyāgāraṃ karitvā aggiṃ paricaranto acchati. So aññadatthu vijjācaraṇasampannasseva paricārako sampajjati. Imāya kho, ambaṭṭha, anuttarāya vijjācaraṇasampadāya idaṃ tatiyaṃ apāyamukhaṃ bhavati. ‘‘Puna caparaṃ, ambaṭṭha, idhekacco samaṇo vā brāhmaṇo vā imaṃ ceva anuttaraṃ vijjācaraṇasampadaṃ anabhisambhuṇamāno pavattaphalabhojanatañca anabhisambhuṇamāno kandamūlaphalabhojanatañca anabhisambhuṇamāno aggipāricariyañca anabhisambhuṇamāno cātumahāpathe catudvāraṃ agāraṃ karitvā acchati – ‘yo imāhi catūhi disāhi āgamissati samaṇo vā brāhmaṇo vā, tamahaṃ yathāsatti yathābalaṃ paṭipūjessāmī’ti. So aññadatthu vijjācaraṇasampannasseva paricārako sampajjati. Imāya kho, ambaṭṭha, anuttarāya vijjācaraṇasampadāya idaṃ catutthaṃ apāyamukhaṃ bhavati. Imāya kho, ambaṭṭha, anuttarāya vijjācaraṇasampadāya imāni cattāri apāyamukhāni bhavanti.

\paragraph{281.} ‘‘Taṃ kiṃ maññasi, ambaṭṭha, apinu tvaṃ imāya anuttarāya vijjācaraṇasampadāya sandissasi sācariyako’’ti? ‘‘No hidaṃ, bho gotama’’. ‘Kocāhaṃ, bho gotama, sācariyako, kā ca anuttarā vijjācaraṇasampadā? Ārakāhaṃ, bho gotama, anuttarāya vijjācaraṇasampadāya sācariyako’’ti. ‘‘Taṃ kiṃ maññasi, ambaṭṭha, apinu tvaṃ imañceva anuttaraṃ vijjācaraṇasampadaṃ anabhisambhuṇamāno khārividhamādāya araññavanamajjhogāhasi sācariyako – ‘pavattaphalabhojano bhavissāmī’’’ti? ‘‘No hidaṃ, bho gotama’’. ‘‘Taṃ kiṃ maññasi, ambaṭṭha, apinu tvaṃ imañceva anuttaraṃ vijjācaraṇasampadaṃ anabhisambhuṇamāno pavattaphalabhojanatañca anabhisambhuṇamāno kudālapiṭakaṃ ādāya araññavanamajjhogāhasi sācariyako – ‘kandamūlaphalabhojano bhavissāmī’’’ti? ‘‘No hidaṃ, bho gotama’’. ‘‘Taṃ kiṃ maññasi, ambaṭṭha, apinu tvaṃ imañceva anuttaraṃ vijjācaraṇasampadaṃ anabhisambhuṇamāno pavattaphalabhojanatañca anabhisambhuṇamāno kandamūlaphalabhojanatañca anabhisambhuṇamāno gāmasāmantaṃ vā nigamasāmantaṃ vā agyāgāraṃ karitvā aggiṃ paricaranto acchasi sācariyako’’ti? ‘‘No hidaṃ, bho gotama’’. ‘‘Taṃ kiṃ maññasi, ambaṭṭha, apinu tvaṃ imañceva anuttaraṃ vijjācaraṇasampadaṃ anabhisambhuṇamāno pavattaphalabhojanatañca anabhisambhuṇamāno kandamūlaphalabhojanatañca anabhisambhuṇamāno aggipāricariyañca anabhisambhuṇamāno cātumahāpathe catudvāraṃ agāraṃ karitvā acchasi sācariyako – ‘yo imāhi catūhi disāhi āgamissati samaṇo vā brāhmaṇo vā, taṃ mayaṃ yathāsatti yathābalaṃ paṭipūjessāmā’’’ti? ‘‘No hidaṃ, bho gotama’’.

\paragraph{282.} ‘‘Iti kho, ambaṭṭha, imāya ceva tvaṃ anuttarāya vijjācaraṇasampadāya parihīno sācariyako. Ye cime anuttarāya vijjācaraṇasampadāya cattāri apāyamukhāni bhavanti, tato ca tvaṃ parihīno sācariyako. Bhāsitā kho pana te esā, ambaṭṭha, ācariyena brāhmaṇena pokkharasātinā vācā – ‘ke ca muṇḍakā samaṇakā ibbhā kaṇhā bandhupādāpaccā, kā ca tevijjānaṃ brāhmaṇānaṃ sākacchā’ti attanā āpāyikopi aparipūramāno. Passa, ambaṭṭha, yāva aparaddhañca te idaṃ ācariyassa brāhmaṇassa pokkharasātissa.

\subsubsection{Pubbakaisibhāvānuyogo}

\paragraph{283.} ‘‘Brāhmaṇo kho pana, ambaṭṭha, pokkharasāti rañño pasenadissa kosalassa dattikaṃ bhuñjati. Tassa rājā pasenadi kosalo sammukhībhāvampi na dadāti. Yadāpi tena manteti, tirodussantena manteti. Yassa kho pana, ambaṭṭha, dhammikaṃ payātaṃ bhikkhaṃ paṭiggaṇheyya, kathaṃ tassa rājā pasenadi kosalo sammukhībhāvampi na dadeyya. Passa, ambaṭṭha, yāva aparaddhañca te idaṃ ācariyassa brāhmaṇassa pokkharasātissa.

\paragraph{284.} ‘‘Taṃ kiṃ maññasi, ambaṭṭha, idha rājā pasenadi kosalo hatthigīvāya vā nisinno assapiṭṭhe vā nisinno rathūpatthare vā ṭhito uggehi vā rājaññehi vā kiñcideva mantanaṃ manteyya. So tamhā padesā apakkamma ekamantaṃ tiṭṭheyya. Atha āgaccheyya suddo vā suddadāso vā, tasmiṃ padese ṭhito tadeva mantanaṃ manteyya – ‘evampi rājā pasenadi kosalo āha, evampi rājā pasenadi kosalo āhā’ti. Apinu so rājabhaṇitaṃ vā bhaṇati rājamantanaṃ vā manteti? Ettāvatā so assa rājā vā rājamatto vā’’ti? ‘‘No hidaṃ, bho gotama’’.

\paragraph{285.} ‘‘Evameva kho tvaṃ, ambaṭṭha, ye te ahesuṃ brāhmaṇānaṃ pubbakā isayo mantānaṃ kattāro mantānaṃ pavattāro, yesamidaṃ etarahi brāhmaṇā porāṇaṃ mantapadaṃ gītaṃ pavuttaṃ samihitaṃ, tadanugāyanti tadanubhāsanti bhāsitamanubhāsanti vācitamanuvācenti, seyyathidaṃ – aṭṭhako vāmako vāmadevo vessāmitto yamataggi\footnote{yamadaggi (ka.)} aṅgīraso bhāradvājo vāseṭṭho kassapo bhagu – ‘tyāhaṃ mante adhiyāmi sācariyako’ti, tāvatā tvaṃ bhavissasi isi vā isitthāya vā paṭipannoti netaṃ ṭhānaṃ vijjati.

\paragraph{286.} ‘‘Taṃ kiṃ maññasi, ambaṭṭha, kinti te sutaṃ brāhmaṇānaṃ vuddhānaṃ mahallakānaṃ ācariyapācariyānaṃ bhāsamānānaṃ – ye te ahesuṃ brāhmaṇānaṃ pubbakā isayo mantānaṃ kattāro mantānaṃ pavattāro, yesamidaṃ etarahi brāhmaṇā porāṇaṃ mantapadaṃ gītaṃ pavuttaṃ samihitaṃ, tadanugāyanti tadanubhāsanti bhāsitamanubhāsanti vācitamanuvācenti, seyyathidaṃ – aṭṭhako vāmako vāmadevo vessāmitto yamataggi aṅgīraso bhāradvājo vāseṭṭho kassapo bhagu, evaṃ su te sunhātā suvilittā kappitakesamassū āmukkamaṇikuṇḍalābharaṇā\footnote{āmuttamālābharaṇā (sī. syā. pī.)} odātavatthavasanā pañcahi kāmaguṇehi samappitā samaṅgībhūtā paricārenti, seyyathāpi tvaṃ etarahi sācariyako’’ti? ‘‘No hidaṃ, bho gotama’’. ‘‘…Pe… evaṃ su te sālīnaṃ odanaṃ sucimaṃsūpasecanaṃ vicitakāḷakaṃ anekasūpaṃ anekabyañjanaṃ paribhuñjanti, seyyathāpi tvaṃ etarahi sācariyako’’ti? ‘‘No hidaṃ, bho gotama’’. ‘‘…Pe… evaṃ su te veṭhakanatapassāhi nārīhi paricārenti, seyyathāpi tvaṃ etarahi sācariyako’’ti? ‘‘No hidaṃ, bho gotama’’. ‘‘…Pe… evaṃ su te kuttavālehi vaḷavārathehi dīghāhi patodalaṭṭhīhi vāhane vitudentā vipariyāyanti, seyyathāpi tvaṃ etarahi sācariyako’’ti? ‘‘No hidaṃ, bho gotama’’. ‘‘…Pe… evaṃ su te ukkiṇṇaparikhāsu okkhittapalighāsu nagarūpakārikāsu dīghāsivudhehi\footnote{dīghāsibaddhehi (syā. pī.)} purisehi rakkhāpenti, seyyathāpi tvaṃ etarahi sācariyako’’ti? ‘‘No hidaṃ, bho gotama’’. ‘‘Iti kho, ambaṭṭha, neva tvaṃ isi na isitthāya paṭipanno sācariyako. Yassa kho pana, ambaṭṭha, mayi kaṅkhā vā vimati vā so maṃ pañhena, ahaṃ veyyākaraṇena sodhissāmī’’ti.

\subsubsection{Dvelakkhaṇādassanaṃ}

\paragraph{287.} Atha kho bhagavā vihārā nikkhamma caṅkamaṃ abbhuṭṭhāsi. Ambaṭṭhopi māṇavo vihārā nikkhamma caṅkamaṃ abbhuṭṭhāsi. Atha kho ambaṭṭho māṇavo bhagavantaṃ caṅkamantaṃ anucaṅkamamāno bhagavato kāye dvattiṃsamahāpurisalakkhaṇāni samannesi. Addasā kho ambaṭṭho māṇavo bhagavato kāye dvattiṃsamahāpurisalakkhaṇāni yebhuyyena ṭhapetvā dve. Dvīsu mahāpurisalakkhaṇesu kaṅkhati vicikicchati nādhimuccati na sampasīdati – kosohite ca vatthaguyhe pahūtajivhatāya ca.

\paragraph{288.} Atha kho bhagavato etadahosi – ‘‘passati kho me ayaṃ ambaṭṭho māṇavo dvattiṃsamahāpurisalakkhaṇāni yebhuyyena ṭhapetvā dve. Dvīsu mahāpurisalakkhaṇesu kaṅkhati vicikicchati nādhimuccati na sampasīdati – kosohite ca vatthaguyhe pahūtajivhatāya cā’’ti. Atha kho bhagavā tathārūpaṃ iddhābhisaṅkhāraṃ abhisaṅkhāsi yathā addasa ambaṭṭho māṇavo bhagavato kosohitaṃ vatthaguyhaṃ. Atha kho bhagavā jivhaṃ ninnāmetvā ubhopi kaṇṇasotāni anumasi paṭimasi, ubhopi nāsikasotāni anumasi paṭimasi, kevalampi nalāṭamaṇḍalaṃ jivhāya chādesi. Atha kho ambaṭṭhassa māṇavassa etadahosi – ‘‘samannāgato kho samaṇo gotamo dvattiṃsamahāpurisalakkhaṇehi paripuṇṇehi, no aparipuṇṇehī’’ti. Bhagavantaṃ etadavoca – ‘‘handa ca dāni mayaṃ, bho gotama, gacchāma, bahukiccā mayaṃ bahukaraṇīyā’’ti. ‘‘Yassadāni tvaṃ, ambaṭṭha, kālaṃ maññasī’’ti. Atha kho ambaṭṭho māṇavo vaḷavārathamāruyha pakkāmi.

\paragraph{289.} Tena kho pana samayena brāhmaṇo pokkharasāti ukkaṭṭhāya nikkhamitvā mahatā brāhmaṇagaṇena saddhiṃ sake ārāme nisinno hoti ambaṭṭhaṃyeva māṇavaṃ paṭimānento. Atha kho ambaṭṭho māṇavo yena sako ārāmo tena pāyāsi. Yāvatikā yānassa bhūmi, yānena gantvā yānā paccorohitvā pattikova yena brāhmaṇo pokkharasāti tenupasaṅkami; upasaṅkamitvā brāhmaṇaṃ pokkharasātiṃ abhivādetvā ekamantaṃ nisīdi.

\paragraph{290.} Ekamantaṃ nisinnaṃ kho ambaṭṭhaṃ māṇavaṃ brāhmaṇo pokkharasāti etadavoca – ‘‘kacci, tāta ambaṭṭha, addasa taṃ bhavantaṃ gotama’’nti? ‘‘Addasāma kho mayaṃ, bho, taṃ bhavantaṃ gotama’’nti. ‘‘Kacci, tāta ambaṭṭha, taṃ bhavantaṃ gotamaṃ tathā santaṃyeva saddo abbhuggato no aññathā; kacci pana so bhavaṃ gotamo tādiso no aññādiso’’ti? ‘‘Tathā santaṃyeva, bho, taṃ bhavantaṃ gotamaṃ saddo abbhuggato no aññathā, tādisova so bhavaṃ gotamo no aññādiso. Samannāgato ca so bhavaṃ gotamo dvattiṃsamahāpurisalakkhaṇehi paripuṇṇehi no aparipuṇṇehī’’ti. ‘‘Ahu pana te, tāta ambaṭṭha, samaṇena gotamena saddhiṃ kocideva kathāsallāpo’’ti? ‘‘Ahu kho me, bho, samaṇena gotamena saddhiṃ kocideva kathāsallāpo’’ti. ‘‘Yathā kathaṃ pana te, tāta ambaṭṭha, ahu samaṇena gotamena saddhiṃ kocideva kathāsallāpo’’ti? Atha kho ambaṭṭho māṇavo yāvatako\footnote{yāvatiko (ka. pī.)} ahosi bhagavatā saddhiṃ kathāsallāpo, taṃ sabbaṃ brāhmaṇassa pokkharasātissa ārocesi.

\paragraph{291.} Evaṃ vutte, brāhmaṇo pokkharasāti ambaṭṭhaṃ māṇavaṃ etadavoca – ‘‘aho vata re amhākaṃ paṇḍitaka\footnote{paṇḍitakā}, aho vata re amhākaṃ bahussutaka\footnote{bahussutakā}, aho vata re amhākaṃ tevijjaka\footnote{tevijjakā}, evarūpena kira, bho, puriso atthacarakena kāyassa bhedā paraṃ maraṇā apāyaṃ duggatiṃ vinipātaṃ nirayaṃ upapajjeyya. Yadeva kho tvaṃ, ambaṭṭha, taṃ bhavantaṃ gotamaṃ evaṃ āsajja āsajja avacāsi, atha kho so bhavaṃ gotamo amhepi evaṃ upaneyya upaneyya avaca. Aho vata re amhākaṃ paṇḍitaka, aho vata re amhākaṃ bahussutaka, aho vata re amhākaṃ tevijjaka, evarūpena kira, bho, puriso atthacarakena kāyassa bhedā paraṃ maraṇā apāyaṃ duggatiṃ vinipātaṃ nirayaṃ upapajjeyyā’’ti, kupito\footnote{so kupito (pī.)} anattamano ambaṭṭhaṃ māṇavaṃ padasāyeva pavattesi. Icchati ca tāvadeva bhagavantaṃ dassanāya upasaṅkamituṃ.

\subsubsection{Pokkharasātibuddhupasaṅkamanaṃ}

\paragraph{292.} Atha kho te brāhmaṇā brāhmaṇaṃ pokkharasātiṃ etadavocuṃ – ‘‘ativikālo kho, bho, ajja samaṇaṃ gotamaṃ dassanāya upasaṅkamituṃ. Svedāni\footnote{dāni sve (sī. ka.)} bhavaṃ pokkharasāti samaṇaṃ gotamaṃ dassanāya upasaṅkamissatī’’ti. Atha kho brāhmaṇo pokkharasāti sake nivesane paṇītaṃ khādanīyaṃ bhojanīyaṃ paṭiyādāpetvā yāne āropetvā ukkāsu dhāriyamānāsu ukkaṭṭhāya niyyāsi, yena icchānaṅgalavanasaṇḍo tena pāyāsi. Yāvatikā yānassa bhūmi yānena gantvā, yānā paccorohitvā pattikova yena bhagavā tenupasaṅkami. Upasaṅkamitvā bhagavatā saddhiṃ sammodi, sammodanīyaṃ kathaṃ sāraṇīyaṃ vītisāretvā ekamantaṃ nisīdi.

\paragraph{293.} Ekamantaṃ nisinno kho brāhmaṇo pokkharasāti bhagavantaṃ etadavoca – ‘‘āgamā nu kho idha, bho gotama, amhākaṃ antevāsī ambaṭṭho māṇavo’’ti? ‘‘Āgamā kho te\footnote{tedha (syā.), te idha (pī.)}, brāhmaṇa, antevāsī ambaṭṭho māṇavo’’ti. ‘‘Ahu pana te, bho gotama, ambaṭṭhena māṇavena saddhiṃ kocideva kathāsallāpo’’ti? ‘‘Ahu kho me, brāhmaṇa, ambaṭṭhena māṇavena saddhiṃ kocideva kathāsallāpo’’ti. ‘‘Yathākathaṃ pana te, bho gotama, ahu ambaṭṭhena māṇavena saddhiṃ kocideva kathāsallāpo’’ti? Atha kho bhagavā yāvatako ahosi ambaṭṭhena māṇavena saddhiṃ kathāsallāpo, taṃ sabbaṃ brāhmaṇassa pokkharasātissa ārocesi. Evaṃ vutte, brāhmaṇo pokkharasāti bhagavantaṃ etadavoca – ‘‘bālo, bho gotama, ambaṭṭho māṇavo, khamatu bhavaṃ gotamo ambaṭṭhassa māṇavassā’’ti. ‘‘Sukhī hotu, brāhmaṇa, ambaṭṭho māṇavo’’ti.

\paragraph{294.} Atha kho brāhmaṇo pokkharasāti bhagavato kāye dvattiṃsamahāpurisalakkhaṇāni samannesi. Addasā kho brāhmaṇo pokkharasāti bhagavato kāye dvattiṃsamahāpurisalakkhaṇāni yebhuyyena ṭhapetvā dve. Dvīsu mahāpurisalakkhaṇesu kaṅkhati vicikicchati nādhimuccati na sampasīdati – kosohite ca vatthaguyhe pahūtajivhatāya ca.

\paragraph{295.} Atha kho bhagavato etadahosi – ‘‘passati kho me ayaṃ brāhmaṇo pokkharasāti dvattiṃsamahāpurisalakkhaṇāni yebhuyyena ṭhapetvā dve. Dvīsu mahāpurisalakkhaṇesu kaṅkhati vicikicchati nādhimuccati na sampasīdati – kosohite ca vatthaguyhe, pahūtajivhatāya cā’’ti. Atha kho bhagavā tathārūpaṃ iddhābhisaṅkhāraṃ abhisaṅkhāsi yathā addasa brāhmaṇo pokkharasāti bhagavato kosohitaṃ vatthaguyhaṃ. Atha kho bhagavā jivhaṃ ninnāmetvā ubhopi kaṇṇasotāni anumasi paṭimasi, ubhopi nāsikasotāni anumasi paṭimasi, kevalampi nalāṭamaṇḍalaṃ jivhāya chādesi.

\paragraph{296.} Atha kho brāhmaṇassa pokkharasātissa etadahosi – ‘‘samannāgato kho samaṇo gotamo dvattiṃsamahāpurisalakkhaṇehi paripuṇṇehi no aparipuṇṇehī’’ti. Bhagavantaṃ etadavoca – ‘‘adhivāsetu me bhavaṃ gotamo ajjatanāya bhattaṃ saddhiṃ bhikkhusaṅghenā’’ti. Adhivāsesi bhagavā tuṇhībhāvena.

\paragraph{297.} Atha kho brāhmaṇo pokkharasāti bhagavato adhivāsanaṃ viditvā bhagavato kālaṃ ārocesi – ‘‘kālo, bho gotama, niṭṭhitaṃ bhatta’’nti. Atha kho bhagavā pubbaṇhasamayaṃ nivāsetvā pattacīvaramādāya saddhiṃ bhikkhusaṅghena yena brāhmaṇassa pokkharasātissa nivesanaṃ tenupasaṅkami; upasaṅkamitvā paññatte āsane nisīdi. Atha kho brāhmaṇo pokkharasāti bhagavantaṃ paṇītena khādanīyena bhojanīyena sahatthā santappesi sampavāresi, māṇavakāpi bhikkhusaṅghaṃ. Atha kho brāhmaṇo pokkharasāti bhagavantaṃ bhuttāviṃ onītapattapāṇiṃ aññataraṃ nīcaṃ āsanaṃ gahetvā ekamantaṃ nisīdi.

\paragraph{298.} Ekamantaṃ nisinnassa kho brāhmaṇassa pokkharasātissa bhagavā anupubbiṃ kathaṃ kathesi, seyyathidaṃ – dānakathaṃ sīlakathaṃ saggakathaṃ; kāmānaṃ ādīnavaṃ okāraṃ saṃkilesaṃ, nekkhamme ānisaṃsaṃ pakāsesi. Yadā bhagavā aññāsi brāhmaṇaṃ pokkharasātiṃ kallacittaṃ muducittaṃ vinīvaraṇacittaṃ udaggacittaṃ pasannacittaṃ, atha yā buddhānaṃ sāmukkaṃsikā dhammadesanā, taṃ pakāsesi – dukkhaṃ samudayaṃ nirodhaṃ maggaṃ. Seyyathāpi nāma suddhaṃ vatthaṃ apagatakāḷakaṃ sammadeva rajanaṃ paṭiggaṇheyya; evameva brāhmaṇassa pokkharasātissa tasmiññeva āsane virajaṃ vītamalaṃ dhammacakkhuṃ udapādi – ‘‘yaṃ kiñci samudayadhammaṃ, sabbaṃ taṃ nirodhadhamma’’nti.

\subsubsection{Pokkharasātiupāsakattapaṭivedanā}

\paragraph{299.} Atha kho brāhmaṇo pokkharasāti diṭṭhadhammo pattadhammo viditadhammo pariyogāḷhadhammo tiṇṇavicikiccho vigatakathaṃkatho vesārajjappatto aparappaccayo satthusāsane bhagavantaṃ etadavoca – ‘‘abhikkantaṃ, bho gotama, abhikkantaṃ, bho gotama. Seyyathāpi, bho gotama, nikkujjitaṃ vā ukkujjeyya, paṭicchannaṃ vā vivareyya, mūḷhassa vā maggaṃ ācikkheyya, andhakāre vā telapajjotaṃ dhāreyya, ‘cakkhumanto rūpāni dakkhantī’ti; evamevaṃ bhotā gotamena anekapariyāyena dhammo pakāsito. Esāhaṃ, bho gotama, saputto sabhariyo sapariso sāmacco bhavantaṃ gotamaṃ saraṇaṃ gacchāmi dhammañca bhikkhusaṅghañca. Upāsakaṃ maṃ bhavaṃ gotamo dhāretu ajjatagge pāṇupetaṃ saraṇaṃ gataṃ. Yathā ca bhavaṃ gotamo ukkaṭṭhāya aññāni upāsakakulāni upasaṅkamati, evameva bhavaṃ gotamo pokkharasātikulaṃ upasaṅkamatu. Tattha ye te māṇavakā vā māṇavikā vā bhavantaṃ gotamaṃ abhivādessanti vā paccuṭṭhissanti\footnote{paccuṭṭhassanti (pī.)} vā āsanaṃ vā udakaṃ vā dassanti cittaṃ vā pasādessanti, tesaṃ taṃ bhavissati dīgharattaṃ hitāya sukhāyā’’ti. ‘‘Kalyāṇaṃ vuccati, brāhmaṇā’’ti.

\xsectionEnd{Ambaṭṭhasuttaṃ niṭṭhitaṃ tatiyaṃ.}


\clearpage
\section{Soṇadaṇḍasuttaṃ}

\subsubsection{Campeyyakabrāhmaṇagahapatikā}

\paragraph{300.} Evaṃ me sutaṃ – ekaṃ samayaṃ bhagavā aṅgesu cārikaṃ caramāno mahatā bhikkhusaṅghena saddhiṃ pañcamattehi bhikkhusatehi yena campā tadavasari. Tatra sudaṃ bhagavā campāyaṃ viharati gaggarāya pokkharaṇiyā tīre. Tena kho pana samayena soṇadaṇḍo brāhmaṇo campaṃ ajjhāvasati sattussadaṃ satiṇakaṭṭhodakaṃ sadhaññaṃ rājabhoggaṃ raññā māgadhena seniyena bimbisārena dinnaṃ rājadāyaṃ brahmadeyyaṃ.

\paragraph{301.} Assosuṃ kho campeyyakā brāhmaṇagahapatikā – ‘‘samaṇo khalu bho gotamo sakyaputto sakyakulā pabbajito aṅgesu cārikaṃ caramāno mahatā bhikkhusaṅghena saddhiṃ pañcamattehi bhikkhusatehi campaṃ anuppatto campāyaṃ viharati gaggarāya pokkharaṇiyā tīre. Taṃ kho pana bhavantaṃ gotamaṃ evaṃ kalyāṇo kittisaddo abbhuggato – ‘itipi so bhagavā arahaṃ sammāsambuddho vijjācaraṇasampanno sugato lokavidū anuttaro purisadammasārathi satthā devamanussānaṃ buddho bhagavā’ti. So imaṃ lokaṃ sadevakaṃ samārakaṃ sabrahmakaṃ sassamaṇabrāhmaṇiṃ pajaṃ sadevamanussaṃ sayaṃ abhiññā sacchikatvā pavedeti. So dhammaṃ deseti ādikalyāṇaṃ majjhekalyāṇaṃ pariyosānakalyāṇaṃ sātthaṃ sabyañjanaṃ kevalaparipuṇṇaṃ parisuddhaṃ brahmacariyaṃ pakāseti. Sādhu kho pana tathārūpānaṃ arahataṃ dassanaṃ hotī’’ti. Atha kho campeyyakā brāhmaṇagahapatikā campāya nikkhamitvā saṅghasaṅghī\footnote{saṅghā saṅghī (sī. syā. pī.)} gaṇībhūtā yena gaggarā pokkharaṇī tenupasaṅkamanti.

\paragraph{302.} Tena kho pana samayena soṇadaṇḍo brāhmaṇo uparipāsāde divāseyyaṃ upagato hoti. Addasā kho soṇadaṇḍo brāhmaṇo campeyyake brāhmaṇagahapatike campāya nikkhamitvā saṅghasaṅghī\footnote{saṅghe saṅghī (sī. pī.) saṅghā saṅghī (syā.)} gaṇībhūte yena gaggarā pokkharaṇī tenupasaṅkamante. Disvā khattaṃ āmantesi – ‘‘kiṃ nu kho, bho khatte, campeyyakā brāhmaṇagahapatikā campāya nikkhamitvā saṅghasaṅghī gaṇībhūtā yena gaggarā pokkharaṇī tenupasaṅkamantī’’ti? ‘‘Atthi kho, bho, samaṇo gotamo sakyaputto sakyakulā pabbajito aṅgesu cārikaṃ caramāno mahatā bhikkhusaṅghena saddhiṃ pañcamattehi bhikkhusatehi campaṃ anuppatto campāyaṃ viharati gaggarāya pokkharaṇiyā tīre. Taṃ kho pana bhavantaṃ gotamaṃ evaṃ kalyāṇo kittisaddo abbhuggato – ‘itipi so bhagavā arahaṃ sammāsambuddho vijjācaraṇasampanno sugato lokavidū anuttaro purisadammasārathi satthā devamanussānaṃ buddho bhagavā’ti. Tamete bhavantaṃ gotamaṃ dassanāya upasaṅkamantī’’ti. ‘‘Tena hi, bho khatte, yena campeyyakā brāhmaṇagahapatikā tenupasaṅkama, upasaṅkamitvā campeyyake brāhmaṇagahapatike evaṃ vadehi – ‘soṇadaṇḍo, bho, brāhmaṇo evamāha – āgamentu kira bhavanto, soṇadaṇḍopi brāhmaṇo samaṇaṃ gotamaṃ dassanāya upasaṅkamissatī’’’ti. ‘‘Evaṃ, bho’’ti kho so khattā soṇadaṇḍassa brāhmaṇassa paṭissutvā yena campeyyakā brāhmaṇagahapatikā tenupasaṅkami; upasaṅkamitvā campeyyake brāhmaṇagahapatike etadavoca – ‘‘soṇadaṇḍo bho brāhmaṇo evamāha – ‘āgamentu kira bhavanto, soṇadaṇḍopi brāhmaṇo samaṇaṃ gotamaṃ dassanāya upasaṅkamissatī’’’ti.

\subsubsection{Soṇadaṇḍaguṇakathā}

\paragraph{303.} Tena kho pana samayena nānāverajjakānaṃ brāhmaṇānaṃ pañcamattāni brāhmaṇasatāni campāyaṃ paṭivasanti kenacideva karaṇīyena. Assosuṃ kho te brāhmaṇā – ‘‘soṇadaṇḍo kira brāhmaṇo samaṇaṃ gotamaṃ dassanāya upasaṅkamissatī’’ti. Atha kho te brāhmaṇā yena soṇadaṇḍo brāhmaṇo tenupasaṅkamiṃsu; upasaṅkamitvā soṇadaṇḍaṃ brāhmaṇaṃ etadavocuṃ – ‘‘saccaṃ kira bhavaṃ soṇadaṇḍo samaṇaṃ gotamaṃ dassanāya upasaṅkamissatī’’ti? ‘‘Evaṃ kho me, bho, hoti – ‘ahampi samaṇaṃ gotamaṃ dassanāya upasaṅkamissāmī’’’ti. ‘‘Mā bhavaṃ soṇadaṇḍo samaṇaṃ gotamaṃ dassanāya upasaṅkami. Na arahati bhavaṃ soṇadaṇḍo samaṇaṃ gotamaṃ dassanāya upasaṅkamituṃ. Sace bhavaṃ soṇadaṇḍo samaṇaṃ gotamaṃ dassanāya upasaṅkamissati, bhoto soṇadaṇḍassa yaso hāyissati, samaṇassa gotamassa yaso abhivaḍḍhissati. Yampi bhoto soṇadaṇḍassa yaso hāyissati, samaṇassa gotamassa yaso abhivaḍḍhissati, imināpaṅgena na arahati bhavaṃ soṇadaṇḍo samaṇaṃ gotamaṃ dassanāya upasaṅkamituṃ; samaṇotveva gotamo arahati bhavantaṃ soṇadaṇḍaṃ dassanāya upasaṅkamituṃ. ‘‘Bhavañhi soṇadaṇḍo ubhato sujāto mātito ca pitito ca, saṃsuddhagahaṇiko yāva sattamā pitāmahayugā akkhitto anupakkuṭṭho jātivādena. Yampi bhavaṃ soṇadaṇḍo ubhato sujāto mātito ca pitito ca, saṃsuddhagahaṇiko yāva sattamā pitāmahayugā akkhitto anupakkuṭṭho jātivādena, imināpaṅgena na arahati bhavaṃ soṇadaṇḍo samaṇaṃ gotamaṃ dassanāya upasaṅkamituṃ; samaṇotveva gotamo arahati bhavantaṃ soṇadaṇḍaṃ dassanāya upasaṅkamituṃ. ‘‘Bhavañhi soṇadaṇḍo aḍḍho mahaddhano mahābhogo…pe… ‘‘Bhavañhi soṇadaṇḍo ajjhāyako, mantadharo, tiṇṇaṃ vedānaṃ pāragū sanighaṇḍukeṭubhānaṃ sākkharappabhedānaṃ itihāsapañcamānaṃ padako veyyākaraṇo, lokāyatamahāpurisalakkhaṇesu anavayo…pe… ‘‘Bhavañhi soṇadaṇḍo abhirūpo dassanīyo pāsādiko paramāya vaṇṇapokkharatāya samannāgato brahmavaṇṇī brahmavacchasī\footnote{brahmaḍḍhī (sī.), brahmavaccasī (pī.)} akhuddāvakāso dassanāya…pe… ‘‘Bhavañhi soṇadaṇḍo sīlavā vuddhasīlī vuddhasīlena samannāgato…pe… ‘‘Bhavañhi soṇadaṇḍo kalyāṇavāco kalyāṇavākkaraṇo poriyā vācāya samannāgato vissaṭṭhāya anelagalāya\footnote{aneḷagalāya (sī. pī.), anelagaḷāya (ka)} atthassa viññāpaniyā…pe… ‘‘Bhavañhi soṇadaṇḍo bahūnaṃ ācariyapācariyo tīṇi māṇavakasatāni mante vāceti. Bahū kho pana nānādisā nānājanapadā māṇavakā āgacchanti bhoto soṇadaṇḍassa santike mantatthikā mante adhiyitukāmā …pe… ‘‘Bhavañhi soṇadaṇḍo jiṇṇo vuddho mahallako addhagato vayoanuppatto; samaṇo gotamo taruṇo ceva taruṇapabbajito ca…pe… ‘‘Bhavañhi soṇadaṇḍo rañño māgadhassa seniyassa bimbisārassa sakkato garukato mānito pūjito apacito…pe… ‘‘Bhavañhi soṇadaṇḍo brāhmaṇassa pokkharasātissa sakkato garukato mānito pūjito apacito…pe… ‘‘Bhavañhi soṇadaṇḍo campaṃ ajjhāvasati sattussadaṃ satiṇakaṭṭhodakaṃ sadhaññaṃ rājabhoggaṃ, raññā māgadhena seniyena bimbisārena dinnaṃ, rājadāyaṃ brahmadeyyaṃ. Yampi bhavaṃ soṇadaṇḍo campaṃ ajjhāvasati sattussadaṃ satiṇakaṭṭhodakaṃ sadhaññaṃ rājabhoggaṃ, raññā māgadhena seniyena bimbisārena dinnaṃ, rājadāyaṃ brahmadeyyaṃ. Imināpaṅgena na arahati bhavaṃ soṇadaṇḍo samaṇaṃ gotamaṃ dassanāya upasaṅkamituṃ; samaṇotveva gotamo arahati bhavantaṃ soṇadaṇḍaṃ dassanāya upasaṅkamitu’’nti.

\subsubsection{Buddhaguṇakathā}

\paragraph{304.} Evaṃ vutte, soṇadaṇḍo brāhmaṇo te brāhmaṇe etadavoca – ‘‘Tena hi, bho, mamapi suṇātha, yathā mayameva arahāma taṃ bhavantaṃ gotamaṃ dassanāya upasaṅkamituṃ; natveva arahati so bhavaṃ gotamo amhākaṃ dassanāya upasaṅkamituṃ. Samaṇo khalu, bho, gotamo ubhato sujāto mātito ca pitito ca, saṃsuddhagahaṇiko yāva sattamā pitāmahayugā, akkhitto anupakkuṭṭho jātivādena. Yampi bho samaṇo gotamo ubhato sujāto mātito ca pitito ca saṃsuddhagahaṇiko yāva sattamā pitāmahayugā, akkhitto anupakkuṭṭho jātivādena, imināpaṅgena na arahati so bhavaṃ gotamo amhākaṃ dassanāya upasaṅkamituṃ; atha kho mayameva arahāma taṃ bhavantaṃ gotamaṃ dassanāya upasaṅkamituṃ. ‘‘Samaṇo khalu, bho, gotamo mahantaṃ ñātisaṅghaṃ ohāya pabbajito…pe… ‘‘Samaṇo khalu, bho, gotamo pahūtaṃ hiraññasuvaṇṇaṃ ohāya pabbajito bhūmigatañca vehāsaṭṭhaṃ ca…pe… ‘‘Samaṇo khalu, bho, gotamo daharova samāno yuvā susukāḷakeso bhadrena yobbanena samannāgato paṭhamena vayasā agārasmā anagāriyaṃ pabbajito…pe… ‘‘Samaṇo khalu, bho, gotamo akāmakānaṃ mātāpitūnaṃ assumukhānaṃ rudantānaṃ kesamassuṃ ohāretvā kāsāyāni vatthāni acchādetvā agārasmā anagāriyaṃ pabbajito…pe… ‘‘Samaṇo khalu, bho, gotamo abhirūpo dassanīyo pāsādiko paramāya vaṇṇapokkharatāya samannāgato, brahmavaṇṇī, brahmavacchasī, akhuddāvakāso dassanāya…pe… ‘‘Samaṇo khalu, bho, gotamo sīlavā ariyasīlī kusalasīlī kusalasīlena samannāgato…pe… ‘‘Samaṇo khalu, bho, gotamo kalyāṇavāco kalyāṇavākkaraṇo poriyā vācāya samannāgato vissaṭṭhāya anelagalāya atthassa viññāpaniyā…pe… ‘‘Samaṇo khalu, bho, gotamo bahūnaṃ ācariyapācariyo…pe… ‘‘Samaṇo khalu, bho, gotamo khīṇakāmarāgo vigatacāpallo…pe… ‘‘Samaṇo khalu, bho, gotamo kammavādī kiriyavādī apāpapurekkhāro brahmaññāya pajāya…pe… ‘‘Samaṇo khalu, bho, gotamo uccā kulā pabbajito asambhinnakhattiyakulā…pe… ‘‘Samaṇo khalu, bho, gotamo aḍḍhā kulā pabbajito mahaddhanā mahābhogā…pe… ‘‘Samaṇaṃ khalu, bho, gotamaṃ tiroraṭṭhā tirojanapadā pañhaṃ pucchituṃ āgacchanti…pe… ‘‘Samaṇaṃ khalu, bho, gotamaṃ anekāni devatāsahassāni pāṇehi saraṇaṃ gatāni… pe… ‘‘Samaṇaṃ khalu, bho, gotamaṃ evaṃ kalyāṇo kittisaddo abbhuggato – ‘itipi so bhagavā arahaṃ sammāsambuddho vijjācaraṇasampanno sugato lokavidū anuttaro purisadammasārathi satthā devamanussānaṃ buddho bhagavā’ ti…pe… ‘‘Samaṇo khalu, bho, gotamo dvattiṃsamahāpurisalakkhaṇehi samannāgato…pe… ‘‘Samaṇo khalu, bho, gotamo ehisvāgatavādī sakhilo sammodako abbhākuṭiko uttānamukho pubbabhāsī…pe… ‘‘Samaṇo khalu, bho, gotamo catunnaṃ parisānaṃ sakkato garukato mānito pūjito apacito…pe… ‘‘Samaṇe khalu, bho, gotame bahū devā ca manussā ca abhippasannā…pe… ‘‘Samaṇo khalu, bho, gotamo yasmiṃ gāme vā nigame vā paṭivasati, na tasmiṃ gāme vā nigame vā amanussā manusse viheṭhenti…pe… ‘‘Samaṇo khalu, bho, gotamo saṅghī gaṇī gaṇācariyo puthutitthakarānaṃ aggamakkhāyati. Yathā kho pana, bho, etesaṃ samaṇabrāhmaṇānaṃ yathā vā tathā vā yaso samudāgacchati, na hevaṃ samaṇassa gotamassa yaso samudāgato. Atha kho anuttarāya vijjācaraṇasampadāya samaṇassa gotamassa yaso samudāgato…pe… ‘‘Samaṇaṃ khalu, bho, gotamaṃ rājā māgadho seniyo bimbisāro saputto sabhariyo sapariso sāmacco pāṇehi saraṇaṃ gato…pe… ‘‘Samaṇaṃ khalu, bho, gotamaṃ rājā pasenadi kosalo saputto sabhariyo sapariso sāmacco pāṇehi saraṇaṃ gato…pe… ‘‘Samaṇaṃ khalu, bho, gotamaṃ brāhmaṇo pokkharasāti saputto sabhariyo sapariso sāmacco pāṇehi saraṇaṃ gato…pe… ‘‘Samaṇo khalu, bho, gotamo rañño māgadhassa seniyassa bimbisārassa sakkato garukato mānito pūjito apacito…pe… ‘‘Samaṇo khalu, bho, gotamo rañño pasenadissa kosalassa sakkato garukato mānito pūjito apacito…pe… ‘‘Samaṇo khalu, bho, gotamo brāhmaṇassa pokkharasātissa sakkato garukato mānito pūjito apacito…pe… ‘‘Samaṇo khalu, bho, gotamo campaṃ anuppatto, campāyaṃ viharati gaggarāya pokkharaṇiyā tīre. Ye kho pana, bho, keci samaṇā vā brāhmaṇā vā amhākaṃ gāmakhettaṃ āgacchanti atithī no te honti. Atithī kho panamhehi sakkātabbā garukātabbā mānetabbā pūjetabbā apacetabbā. Yampi, bho, samaṇo gotamo campaṃ anuppatto campāyaṃ viharati gaggarāya pokkharaṇiyā tīre, atithimhākaṃ samaṇo gotamo; atithi kho panamhehi sakkātabbo garukātabbo mānetabbo pūjetabbo apacetabbo. Imināpaṅgena na arahati so bhavaṃ gotamo amhākaṃ dassanāya upasaṅkamituṃ. Atha kho mayameva arahāma taṃ bhavantaṃ gotamaṃ dassanāya upasaṅkamituṃ. Ettake kho ahaṃ, bho, tassa bhoto gotamassa vaṇṇe pariyāpuṇāmi, no ca kho so bhavaṃ gotamo ettakavaṇṇo. Aparimāṇavaṇṇo hi so bhavaṃ gotamo’’ti.

\paragraph{305.} Evaṃ vutte, te brāhmaṇā soṇadaṇḍaṃ brāhmaṇaṃ etadavocuṃ – ‘‘yathā kho bhavaṃ soṇadaṇḍo samaṇassa gotamassa vaṇṇe bhāsati ito cepi so bhavaṃ gotamo yojanasate viharati, alameva saddhena kulaputtena dassanāya upasaṅkamituṃ api puṭosenā’’ti. ‘‘Tena hi, bho, sabbeva mayaṃ samaṇaṃ gotamaṃ dassanāya upasaṅkamissāmā’’ti.

\subsubsection{Soṇadaṇḍaparivitakko}

\paragraph{306.} Atha kho soṇadaṇḍo brāhmaṇo mahatā brāhmaṇagaṇena saddhiṃ yena gaggarā pokkharaṇī tenupasaṅkami. Atha kho soṇadaṇḍassa brāhmaṇassa tirovanasaṇḍagatassa evaṃ cetaso parivitakko udapādi – ‘‘ahañceva kho pana samaṇaṃ gotamaṃ pañhaṃ puccheyyaṃ; tatra ce maṃ samaṇo gotamo evaṃ vadeyya – ‘na kho esa, brāhmaṇa, pañho evaṃ pucchitabbo, evaṃ nāmesa, brāhmaṇa, pañho pucchitabbo’ti, tena maṃ ayaṃ parisā paribhaveyya – ‘bālo soṇadaṇḍo brāhmaṇo abyatto, nāsakkhi samaṇaṃ gotamaṃ yoniso pañhaṃ pucchitu’nti. Yaṃ kho panāyaṃ parisā paribhaveyya, yasopi tassa hāyetha. Yassa kho pana yaso hāyetha, bhogāpi tassa hāyeyyuṃ. Yasoladdhā kho panamhākaṃ bhogā. Mamañceva kho pana samaṇo gotamo pañhaṃ puccheyya, tassa cāhaṃ pañhassa veyyākaraṇena cittaṃ na ārādheyyaṃ; tatra ce maṃ samaṇo gotamo evaṃ vadeyya – ‘na kho esa, brāhmaṇa, pañho evaṃ byākātabbo, evaṃ nāmesa, brāhmaṇa, pañho byākātabbo’ti, tena maṃ ayaṃ parisā paribhaveyya – ‘bālo soṇadaṇḍo brāhmaṇo abyatto, nāsakkhi samaṇassa gotamassa pañhassa veyyākaraṇena cittaṃ ārādhetu’nti. Yaṃ kho panāyaṃ parisā paribhaveyya, yasopi tassa hāyetha. Yassa kho pana yaso hāyetha, bhogāpi tassa hāyeyyuṃ. Yasoladdhā kho panamhākaṃ bhogā. Ahañceva kho pana evaṃ samīpagato samāno adisvāva samaṇaṃ gotamaṃ nivatteyyaṃ, tena maṃ ayaṃ parisā paribhaveyya – ‘bālo soṇadaṇḍo brāhmaṇo abyatto mānathaddho bhīto ca, no visahati samaṇaṃ gotamaṃ dassanāya upasaṅkamituṃ, kathañhi nāma evaṃ samīpagato samāno adisvā samaṇaṃ gotamaṃ nivattissatī’ti. Yaṃ kho panāyaṃ parisā paribhaveyya, yasopi tassa hāyetha. Yassa kho pana yaso hāyetha, bhogāpi tassa hāyeyyuṃ, yasoladdhā kho panamhākaṃ bhogā’’ti.

\paragraph{307.} Atha kho soṇadaṇḍo brāhmaṇo yena bhagavā tenupasaṅkami; upasaṅkamitvā bhagavatā saddhiṃ sammodi. Sammodanīyaṃ kathaṃ sāraṇīyaṃ vītisāretvā ekamantaṃ nisīdi. Campeyyakāpi kho brāhmaṇagahapatikā appekacce bhagavantaṃ abhivādetvā ekamantaṃ nisīdiṃsu; appekacce bhagavatā saddhiṃ sammodiṃsu; sammodanīyaṃ kathaṃ sāraṇīyaṃ vītisāretvā ekamantaṃ nisīdiṃsu; appekacce yena bhagavā tenañjaliṃ paṇāmetvā ekamantaṃ nisīdiṃsu; appekacce nāmagottaṃ sāvetvā ekamantaṃ nisīdiṃsu; appekacce tuṇhībhūtā ekamantaṃ nisīdiṃsu.

\paragraph{308.} Tatrapi sudaṃ soṇadaṇḍo brāhmaṇo etadeva bahulamanuvitakkento nisinno hoti – ‘‘ahañceva kho pana samaṇaṃ gotamaṃ pañhaṃ puccheyyaṃ; tatra ce maṃ samaṇo gotamo evaṃ vadeyya – ‘na kho esa, brāhmaṇa, pañho evaṃ pucchitabbo, evaṃ nāmesa, brāhmaṇa, pañho pucchitabbo’ti, tena maṃ ayaṃ parisā paribhaveyya – ‘bālo soṇadaṇḍo brāhmaṇo abyatto, nāsakkhi samaṇaṃ gotamaṃ yoniso pañhaṃ pucchitu’nti. Yaṃ kho panāyaṃ parisā paribhaveyya, yasopi tassa hāyetha. Yassa kho pana yaso hāyetha, bhogāpi tassa hāyeyyuṃ. Yasoladdhā kho panamhākaṃ bhogā. Mamañceva kho pana samaṇo gotamo pañhaṃ puccheyya, tassa cāhaṃ pañhassa veyyākaraṇena cittaṃ na ārādheyyaṃ; tatra ce maṃ samaṇo gotamo evaṃ vadeyya – ‘na kho esa, brāhmaṇa, pañho evaṃ byākātabbo, evaṃ nāmesa, brāhmaṇa, pañho byākātabbo’ti, tena maṃ ayaṃ parisā paribhaveyya – ‘bālo soṇadaṇḍo brāhmaṇo abyatto, nāsakkhi samaṇassa gotamassa pañhassa veyyākaraṇena cittaṃ ārādhetu’nti. Yaṃ kho panāyaṃ parisā paribhaveyya, yasopi tassa hāyetha. Yassa kho pana yaso hāyetha, bhogāpi tassa hāyeyyuṃ. Yasoladdhā kho panamhākaṃ bhogā. Aho vata maṃ samaṇo gotamo sake ācariyake tevijjake pañhaṃ puccheyya, addhā vatassāhaṃ cittaṃ ārādheyyaṃ pañhassa veyyākaraṇenā’’ti.

\subsubsection{Brāhmaṇapaññatti}

\paragraph{309.} Atha kho bhagavato soṇadaṇḍassa brāhmaṇassa cetasā cetoparivitakkamaññāya etadahosi – ‘‘vihaññati kho ayaṃ soṇadaṇḍo brāhmaṇo sakena cittena. Yaṃnūnāhaṃ soṇadaṇḍaṃ brāhmaṇaṃ sake ācariyake tevijjake pañhaṃ puccheyya’’nti. Atha kho bhagavā soṇadaṇḍaṃ brāhmaṇaṃ etadavoca – ‘‘katihi pana, brāhmaṇa, aṅgehi samannāgataṃ brāhmaṇā brāhmaṇaṃ paññapenti; ‘brāhmaṇosmī’ti ca vadamāno sammā vadeyya, na ca pana musāvādaṃ āpajjeyyā’’ti?

\paragraph{310.} Atha kho soṇadaṇḍassa brāhmaṇassa etadahosi – ‘‘yaṃ vata no ahosi icchitaṃ, yaṃ ākaṅkhitaṃ, yaṃ adhippetaṃ, yaṃ abhipatthitaṃ – ‘aho vata maṃ samaṇo gotamo sake ācariyake tevijjake pañhaṃ puccheyya, addhā vatassāhaṃ cittaṃ ārādheyyaṃ pañhassa veyyākaraṇenā’ti, tatra maṃ samaṇo gotamo sake ācariyake tevijjake pañhaṃ pucchati. Addhā vatassāhaṃ cittaṃ ārādhessāmi pañhassa veyyākaraṇenā’’ti.

\paragraph{311.} Atha kho soṇadaṇḍo brāhmaṇo abbhunnāmetvā kāyaṃ anuviloketvā parisaṃ bhagavantaṃ etadavoca – ‘‘pañcahi, bho gotama, aṅgehi samannāgataṃ brāhmaṇā brāhmaṇaṃ paññapenti; ‘brāhmaṇosmī’ti ca vadamāno sammā vadeyya, na ca pana musāvādaṃ āpajjeyya. Katamehi pañcahi? Idha, bho gotama, brāhmaṇo ubhato sujāto hoti mātito ca pitito ca, saṃsuddhagahaṇiko yāva sattamā pitāmahayugā akkhitto anupakkuṭṭho jātivādena; ajjhāyako hoti mantadharo tiṇṇaṃ vedānaṃ pāragū sanighaṇḍukeṭubhānaṃ sākkharappabhedānaṃ itihāsapañcamānaṃ padako veyyākaraṇo lokāyatamahāpurisalakkhaṇesu anavayo; abhirūpo hoti dassanīyo pāsādiko paramāya vaṇṇapokkharatāya samannāgato brahmavaṇṇī brahmavacchasī akhuddāvakāso dassanāya; sīlavā hoti vuddhasīlī vuddhasīlena samannāgato; paṇḍito ca hoti medhāvī paṭhamo vā dutiyo vā sujaṃ paggaṇhantānaṃ. Imehi kho, bho gotama, pañcahi aṅgehi samannāgataṃ brāhmaṇā brāhmaṇaṃ paññapenti; ‘brāhmaṇosmī’ti ca vadamāno sammā vadeyya, na ca pana musāvādaṃ āpajjeyyā’’ti. ‘‘Imesaṃ pana, brāhmaṇa, pañcannaṃ aṅgānaṃ sakkā ekaṃ aṅgaṃ ṭhapayitvā catūhaṅgehi samannāgataṃ brāhmaṇā brāhmaṇaṃ paññapetuṃ; ‘brāhmaṇosmī’ti ca vadamāno sammā vadeyya, na ca pana musāvādaṃ āpajjeyyā’’ti? ‘‘Sakkā, bho gotama. Imesañhi, bho gotama, pañcannaṃ aṅgānaṃ vaṇṇaṃ ṭhapayāma. Kiñhi vaṇṇo karissati? Yato kho, bho gotama, brāhmaṇo ubhato sujāto hoti mātito ca pitito ca saṃsuddhagahaṇiko yāva sattamā pitāmahayugā akkhitto anupakkuṭṭho jātivādena; ajjhāyako ca hoti mantadharo ca tiṇṇaṃ vedānaṃ pāragū sanighaṇḍukeṭubhānaṃ sākkharappabhedānaṃ itihāsapañcamānaṃ padako veyyākaraṇo lokāyatamahāpurisalakkhaṇesu anavayo; sīlavā ca hoti vuddhasīlī vuddhasīlena samannāgato; paṇḍito ca hoti medhāvī paṭhamo vā dutiyo vā sujaṃ paggaṇhantānaṃ. Imehi kho bho gotama catūhaṅgehi samannāgataṃ brāhmaṇā brāhmaṇaṃ paññapenti; ‘brāhmaṇosmī’ti ca vadamāno sammā vadeyya, na ca pana musāvādaṃ āpajjeyyā’’ti.

\paragraph{312.} ‘‘Imesaṃ pana, brāhmaṇa, catunnaṃ aṅgānaṃ sakkā ekaṃ aṅgaṃ ṭhapayitvā tīhaṅgehi samannāgataṃ brāhmaṇā brāhmaṇaṃ paññapetuṃ; ‘brāhmaṇosmī’ti ca vadamāno sammā vadeyya, na ca pana musāvādaṃ āpajjeyyā’’ti? ‘‘Sakkā, bho gotama. Imesañhi, bho gotama, catunnaṃ aṅgānaṃ mante ṭhapayāma. Kiñhi mantā karissanti? Yato kho, bho gotama, brāhmaṇo ubhato sujāto hoti mātito ca pitito ca saṃsuddhagahaṇiko yāva sattamā pitāmahayugā akkhitto anupakkuṭṭho jātivādena; sīlavā ca hoti vuddhasīlī vuddhasīlena samannāgato; paṇḍito ca hoti medhāvī paṭhamo vā dutiyo vā sujaṃ paggaṇhantānaṃ. Imehi kho, bho gotama, tīhaṅgehi samannāgataṃ brāhmaṇā brāhmaṇaṃ paññapenti; ‘brāhmaṇosmī’ti ca vadamāno sammā vadeyya, na ca pana musāvādaṃ āpajjeyyā’’ti. ‘‘Imesaṃ pana, brāhmaṇa, tiṇṇaṃ aṅgānaṃ sakkā ekaṃ aṅgaṃ ṭhapayitvā dvīhaṅgehi samannāgataṃ brāhmaṇā brāhmaṇaṃ paññapetuṃ; ‘brāhmaṇosmī’ti ca vadamāno sammā vadeyya, na ca pana musāvādaṃ āpajjeyyā’’ti? ‘‘Sakkā, bho gotama. Imesañhi, bho gotama, tiṇṇaṃ aṅgānaṃ jātiṃ ṭhapayāma. Kiñhi jāti karissati? Yato kho, bho gotama, brāhmaṇo sīlavā hoti vuddhasīlī vuddhasīlena samannāgato; paṇḍito ca hoti medhāvī paṭhamo vā dutiyo vā sujaṃ paggaṇhantānaṃ. Imehi kho, bho gotama, dvīhaṅgehi samannāgataṃ brāhmaṇā brāhmaṇaṃ paññapenti; ‘brāhmaṇosmī’ti ca vadamāno sammā vadeyya, na ca pana musāvādaṃ āpajjeyyā’’ti.

\paragraph{313.} Evaṃ vutte, te brāhmaṇā soṇadaṇḍaṃ brāhmaṇaṃ etadavocuṃ – ‘‘mā bhavaṃ soṇadaṇḍo evaṃ avaca, mā bhavaṃ soṇadaṇḍo evaṃ avaca. Apavadateva bhavaṃ soṇadaṇḍo vaṇṇaṃ, apavadati mante, apavadati jātiṃ ekaṃsena. Bhavaṃ soṇadaṇḍo samaṇasseva gotamassa vādaṃ anupakkhandatī’’ti.

\paragraph{314.} Atha kho bhagavā te brāhmaṇe etadavoca – ‘‘sace kho tumhākaṃ brāhmaṇānaṃ evaṃ hoti – ‘appassuto ca soṇadaṇḍo brāhmaṇo, akalyāṇavākkaraṇo ca soṇadaṇḍo brāhmaṇo, duppañño ca soṇadaṇḍo brāhmaṇo, na ca pahoti soṇadaṇḍo brāhmaṇo samaṇena gotamena saddhiṃ asmiṃ vacane paṭimantetu’nti, tiṭṭhatu soṇadaṇḍo brāhmaṇo, tumhe mayā saddhiṃ mantavho asmiṃ vacane. Sace pana tumhākaṃ brāhmaṇānaṃ evaṃ hoti – ‘bahussuto ca soṇadaṇḍo brāhmaṇo, kalyāṇavākkaraṇo ca soṇadaṇḍo brāhmaṇo, paṇḍito ca soṇadaṇḍo brāhmaṇo, pahoti ca soṇadaṇḍo brāhmaṇo samaṇena gotamena saddhiṃ asmiṃ vacane paṭimantetu’nti, tiṭṭhatha tumhe, soṇadaṇḍo brāhmaṇo mayā saddhiṃ paṭimantetū’’ti.

\paragraph{315.} Evaṃ vutte, soṇadaṇḍo brāhmaṇo bhagavantaṃ etadavoca – ‘‘tiṭṭhatu bhavaṃ gotamo, tuṇhī bhavaṃ gotamo hotu, ahameva tesaṃ sahadhammena paṭivacanaṃ karissāmī’’ti. Atha kho soṇadaṇḍo brāhmaṇo te brāhmaṇe etadavoca – ‘‘mā bhavanto evaṃ avacuttha, mā bhavanto evaṃ avacuttha – ‘apavadateva bhavaṃ soṇadaṇḍo vaṇṇaṃ, apavadati mante, apavadati jātiṃ ekaṃsena. Bhavaṃ soṇadaṇḍo samaṇasseva gotamassa vādaṃ anupakkhandatī’ti. Nāhaṃ, bho, apavadāmi vaṇṇaṃ vā mante vā jātiṃ vā’’ti.

\paragraph{316.} Tena kho pana samayena soṇadaṇḍassa brāhmaṇassa bhāgineyyo aṅgako nāma māṇavako tassaṃ parisāyaṃ nisinno hoti. Atha kho soṇadaṇḍo brāhmaṇo te brāhmaṇe etadavoca – ‘‘passanti no bhonto imaṃ aṅgakaṃ māṇavakaṃ amhākaṃ bhāgineyya’’nti? ‘‘Evaṃ, bho’’. ‘‘Aṅgako kho, bho, māṇavako abhirūpo dassanīyo pāsādiko paramāya vaṇṇapokkharatāya samannāgato brahmavaṇṇī brahmavacchasī akhuddāvakāso dassanāya, nāssa imissaṃ parisāyaṃ samasamo atthi vaṇṇena ṭhapetvā samaṇaṃ gotamaṃ. Aṅgako kho māṇavako ajjhāyako mantadharo, tiṇṇaṃ vedānaṃ pāragū sanighaṇḍukeṭubhānaṃ sākkharappabhedānaṃ itihāsapañcamānaṃ padako veyyākaraṇo lokāyatamahāpurisalakkhaṇesu anavayo. Ahamassa mante vācetā. Aṅgako kho māṇavako ubhato sujāto mātito ca pitito ca saṃsuddhagahaṇiko yāva sattamā pitāmahayugā akkhitto anupakkuṭṭho jātivādena. Ahamassa mātāpitaro jānāmi. Aṅgako kho māṇavako pāṇampi haneyya, adinnampi ādiyeyya, paradārampi gaccheyya, musāvādampi bhaṇeyya, majjampi piveyya, ettha dāni, bho, kiṃ vaṇṇo karissati, kiṃ mantā, kiṃ jāti? Yato kho, bho, brāhmaṇo sīlavā ca hoti vuddhasīlī vuddhasīlena samannāgato, paṇḍito ca hoti medhāvī paṭhamo vā dutiyo vā sujaṃ paggaṇhantānaṃ. Imehi kho, bho, dvīhaṅgehi samannāgataṃ brāhmaṇā brāhmaṇaṃ paññapenti; ‘brāhmaṇosmī’ti ca vadamāno sammā vadeyya, na ca pana musāvādaṃ āpajjeyyā’’ti.

\subsubsection{Sīlapaññākathā}

\paragraph{317.} ‘‘Imesaṃ pana, brāhmaṇa, dvinnaṃ aṅgānaṃ sakkā ekaṃ aṅgaṃ ṭhapayitvā ekena aṅgena samannāgataṃ brāhmaṇā brāhmaṇaṃ paññapetuṃ; ‘brāhmaṇosmī’ti ca vadamāno sammā vadeyya, na ca pana musāvādaṃ āpajjeyyā’’ti? ‘‘No hidaṃ, bho gotama. Sīlaparidhotā hi, bho gotama, paññā; paññāparidhotaṃ sīlaṃ. Yattha sīlaṃ tattha paññā, yattha paññā tattha sīlaṃ. Sīlavato paññā, paññavato sīlaṃ. Sīlapaññāṇañca pana lokasmiṃ aggamakkhāyati. Seyyathāpi, bho gotama, hatthena vā hatthaṃ dhoveyya, pādena vā pādaṃ dhoveyya; evameva kho, bho gotama, sīlaparidhotā paññā, paññāparidhotaṃ sīlaṃ. Yattha sīlaṃ tattha paññā, yattha paññā tattha sīlaṃ. Sīlavato paññā, paññavato sīlaṃ. Sīlapaññāṇañca pana lokasmiṃ aggamakkhāyatī’’ti. ‘‘Evametaṃ, brāhmaṇa, evametaṃ, brāhmaṇa, sīlaparidhotā hi, brāhmaṇa, paññā, paññāparidhotaṃ sīlaṃ. Yattha sīlaṃ tattha paññā, yattha paññā tattha sīlaṃ. Sīlavato paññā, paññavato sīlaṃ. Sīlapaññāṇañca pana lokasmiṃ aggamakkhāyati. Seyyathāpi, brāhmaṇa, hatthena vā hatthaṃ dhoveyya, pādena vā pādaṃ dhoveyya; evameva kho, brāhmaṇa, sīlaparidhotā paññā, paññāparidhotaṃ sīlaṃ. Yattha sīlaṃ tattha paññā, yattha paññā tattha sīlaṃ. Sīlavato paññā, paññavato sīlaṃ. Sīlapaññāṇañca pana lokasmiṃ aggamakkhāyati.

\paragraph{318.} ‘‘Katamaṃ pana taṃ, brāhmaṇa, sīlaṃ? Katamā sā paññā’’ti? ‘‘Ettakaparamāva mayaṃ, bho gotama, etasmiṃ atthe. Sādhu vata bhavantaṃyeva gotamaṃ paṭibhātu etassa bhāsitassa attho’’ti. ‘‘Tena hi, brāhmaṇa, suṇohi; sādhukaṃ manasikarohi; bhāsissāmī’’ti. ‘‘Evaṃ, bho’’ti kho soṇadaṇḍo brāhmaṇo bhagavato paccassosi. Bhagavā etadavoca – ‘‘idha, brāhmaṇa, tathāgato loke uppajjati arahaṃ sammāsambuddho…pe… (yathā 190-212 anucchedesu tathā vitthāretabbaṃ). Evaṃ kho, brāhmaṇa, bhikkhu sīlasampanno hoti. Idaṃ kho taṃ, brāhmaṇa, sīlaṃ…pe… paṭhamaṃ jhānaṃ upasampajja viharati…pe… dutiyaṃ jhānaṃ…pe… tatiyaṃ jhānaṃ…pe… catutthaṃ jhānaṃ upasampajja viharati…pe… ñāṇadassanāya cittaṃ abhinīharati, abhininnāmeti. Idampissa hoti paññāya… pe… nāparaṃ itthattāyāti pajānāti, idampissa hoti paññāya ayaṃ kho sā, brāhmaṇa, paññā’’ti.

\subsubsection{Soṇadaṇḍaupāsakattapaṭivedanā}

\paragraph{319.} Evaṃ vutte, soṇadaṇḍo brāhmaṇo bhagavantaṃ etadavoca – ‘‘abhikkantaṃ, bho gotama, abhikkantaṃ, bho gotama. Seyyathāpi, bho gotama, nikkujjitaṃ vā ukkujjeyya, paṭicchannaṃ vā vivareyya, mūḷhassa vā maggaṃ ācikkheyya, andhakāre vā telapajjotaṃ dhāreyya, ‘cakkhumanto rūpāni dakkhantī’ti; evamevaṃ bhotā gotamena anekapariyāyena dhammo pakāsito. Esāhaṃ bhavantaṃ gotamaṃ saraṇaṃ gacchāmi, dhammañca, bhikkhusaṅghañca. Upāsakaṃ maṃ bhavaṃ gotamo dhāretu ajjatagge pāṇupetaṃ saraṇaṃ gataṃ. Adhivāsetu ca me bhavaṃ gotamo svātanāya bhattaṃ saddhiṃ bhikkhusaṅghenā’’ti. Adhivāsesi bhagavā tuṇhībhāvena.

\paragraph{320.} Atha kho soṇadaṇḍo brāhmaṇo bhagavato adhivāsanaṃ viditvā uṭṭhāyāsanā bhagavantaṃ abhivādetvā padakkhiṇaṃ katvā pakkāmi. Atha kho soṇadaṇḍo brāhmaṇo tassā rattiyā accayena sake nivesane paṇītaṃ khādanīyaṃ bhojanīyaṃ paṭiyādāpetvā bhagavato kālaṃ ārocāpesi – ‘‘kālo, bho gotama, niṭṭhitaṃ bhatta’’nti. Atha kho bhagavā pubbaṇhasamayaṃ nivāsetvā pattacīvaramādāya saddhiṃ bhikkhusaṅghena yena soṇadaṇḍassa brāhmaṇassa nivesanaṃ tenupasaṅkami; upasaṅkamitvā paññatte āsane nisīdi. Atha kho soṇadaṇḍo brāhmaṇo buddhappamukhaṃ bhikkhusaṅghaṃ paṇītena khādanīyena bhojanīyena sahatthā santappesi sampavāresi.

\paragraph{321.} Atha kho soṇadaṇḍo brāhmaṇo bhagavantaṃ bhuttāviṃ onītapattapāṇiṃ aññataraṃ nīcaṃ āsanaṃ gahetvā ekamantaṃ nisīdi. Ekamantaṃ nisinno kho soṇadaṇḍo brāhmaṇo bhagavantaṃ etadavoca – ‘‘ahañceva kho pana, bho gotama, parisagato samāno āsanā vuṭṭhahitvā bhavantaṃ gotamaṃ abhivādeyyaṃ, tena maṃ sā parisā paribhaveyya. Yaṃ kho pana sā parisā paribhaveyya, yasopi tassa hāyetha. Yassa kho pana yaso hāyetha, bhogāpi tassa hāyeyyuṃ. Yasoladdhā kho panamhākaṃ bhogā. Ahañceva kho pana, bho gotama, parisagato samāno añjaliṃ paggaṇheyyaṃ, āsanā me taṃ bhavaṃ gotamo paccuṭṭhānaṃ dhāretu. Ahañceva kho pana, bho gotama, parisagato samāno veṭhanaṃ omuñceyyaṃ, sirasā me taṃ bhavaṃ gotamo abhivādanaṃ dhāretu. Ahañceva kho pana, bho gotama, yānagato samāno yānā paccorohitvā bhavantaṃ gotamaṃ abhivādeyyaṃ, tena maṃ sā parisā paribhaveyya. Yaṃ kho pana sā parisā paribhaveyya, yasopi tassa hāyetha, yassa kho pana yaso hāyetha, bhogāpi tassa hāyeyyuṃ. Yasoladdhā kho panamhākaṃ bhogā. Ahañceva kho pana, bho gotama, yānagato samāno patodalaṭṭhiṃ abbhunnāmeyyaṃ, yānā me taṃ bhavaṃ gotamo paccorohanaṃ dhāretu. Ahañceva kho pana, bho gotama, yānagato samāno chattaṃ apanāmeyyaṃ, sirasā me taṃ bhavaṃ gotamo abhivādanaṃ dhāretū’’ti.

\paragraph{322.} Atha kho bhagavā soṇadaṇḍaṃ brāhmaṇaṃ dhammiyā kathāya sandassetvā samādapetvā samuttejetvā sampahaṃsetvā uṭṭhāyāsanā pakkāmīti.

\xsectionEnd{Soṇadaṇḍasuttaṃ niṭṭhitaṃ catutthaṃ.}


\clearpage
\section{Kūṭadantasuttaṃ}

\subsubsection{Khāṇumatakabrāhmaṇagahapatikā}

\paragraph{323.} Evaṃ me sutaṃ – ekaṃ samayaṃ bhagavā magadhesu cārikaṃ caramāno mahatā bhikkhusaṅghena saddhiṃ pañcamattehi bhikkhusatehi yena khāṇumataṃ nāma magadhānaṃ brāhmaṇagāmo tadavasari. Tatra sudaṃ bhagavā khāṇumate viharati ambalaṭṭhikāyaṃ. Tena kho pana samayena kūṭadanto brāhmaṇo khāṇumataṃ ajjhāvasati sattussadaṃ satiṇakaṭṭhodakaṃ sadhaññaṃ rājabhoggaṃ raññā māgadhena seniyena bimbisārena dinnaṃ rājadāyaṃ brahmadeyyaṃ. Tena kho pana samayena kūṭadantassa brāhmaṇassa mahāyañño upakkhaṭo hoti. Satta ca usabhasatāni satta ca vacchatarasatāni satta ca vacchatarīsatāni satta ca ajasatāni satta ca urabbhasatāni thūṇūpanītāni honti yaññatthāya.

\paragraph{324.} Assosuṃ kho khāṇumatakā brāhmaṇagahapatikā – ‘‘samaṇo khalu, bho, gotamo sakyaputto sakyakulā pabbajito magadhesu cārikaṃ caramāno mahatā bhikkhusaṅghena saddhiṃ pañcamattehi bhikkhusatehi khāṇumataṃ anuppatto khāṇumate viharati ambalaṭṭhikāyaṃ. Taṃ kho pana bhavantaṃ gotamaṃ evaṃ kalyāṇo kittisaddo abbhuggato – ‘itipi so bhagavā arahaṃ sammāsambuddho vijjācaraṇasampanno sugato lokavidū anuttaro purisadammasārathi satthā devamanussānaṃ buddho bhagavā’ti. So imaṃ lokaṃ sadevakaṃ samārakaṃ sabrahmakaṃ sassamaṇabrāhmaṇiṃ pajaṃ sadevamanussaṃ sayaṃ abhiññā sacchikatvā pavedeti. So dhammaṃ deseti ādikalyāṇaṃ majjhekalyāṇaṃ pariyosānakalyāṇaṃ sātthaṃ sabyañjanaṃ kevalaparipuṇṇaṃ parisuddhaṃ brahmacariyaṃ pakāseti. Sādhu kho pana tathārūpānaṃ arahataṃ dassanaṃ hotī’’ti.

\paragraph{325.} Atha kho khāṇumatakā brāhmaṇagahapatikā khāṇumatā nikkhamitvā saṅghasaṅghī gaṇībhūtā yena ambalaṭṭhikā tenupasaṅkamanti.

\paragraph{326.} Tena kho pana samayena kūṭadanto brāhmaṇo uparipāsāde divāseyyaṃ upagato hoti. Addasā kho kūṭadanto brāhmaṇo khāṇumatake brāhmaṇagahapatike khāṇumatā nikkhamitvā saṅghasaṅghī gaṇībhūte yena ambalaṭṭhikā tenupasaṅkamante. Disvā khattaṃ āmantesi – ‘‘kiṃ nu kho, bho khatte, khāṇumatakā brāhmaṇagahapatikā khāṇumatā nikkhamitvā saṅghasaṅghī gaṇībhūtā yena ambalaṭṭhikā tenupasaṅkamantī’’ti?

\paragraph{327.} ‘‘Atthi kho, bho, samaṇo gotamo sakyaputto sakyakulā pabbajito magadhesu cārikaṃ caramāno mahatā bhikkhusaṅghena saddhiṃ pañcamattehi bhikkhusatehi khāṇumataṃ anuppatto, khāṇumate viharati ambalaṭṭhikāyaṃ. Taṃ kho pana bhavantaṃ gotamaṃ evaṃ kalyāṇo kittisaddo abbhuggato – ‘itipi so bhagavā arahaṃ sammāsambuddho vijjācaraṇasampanno sugato lokavidū anuttaro purisadammasārathi satthā devamanussānaṃ buddho bhagavā’ti. Tamete bhavantaṃ gotamaṃ dassanāya upasaṅkamantī’’ti.

\paragraph{328.} Atha kho kūṭadantassa brāhmaṇassa etadahosi – ‘‘sutaṃ kho pana metaṃ – ‘samaṇo gotamo tividhaṃ yaññasampadaṃ soḷasaparikkhāraṃ jānātī’ti. Na kho panāhaṃ jānāmi tividhaṃ yaññasampadaṃ soḷasaparikkhāraṃ. Icchāmi cāhaṃ mahāyaññaṃ yajituṃ. Yaṃnūnāhaṃ samaṇaṃ gotamaṃ upasaṅkamitvā tividhaṃ yaññasampadaṃ soḷasaparikkhāraṃ puccheyya’’nti.

\paragraph{329.} Atha kho kūṭadanto brāhmaṇo khattaṃ āmantesi – ‘‘tena hi, bho khatte, yena khāṇumatakā brāhmaṇagahapatikā tenupasaṅkama. Upasaṅkamitvā khāṇumatake brāhmaṇagahapatike evaṃ vadehi – ‘kūṭadanto, bho, brāhmaṇo evamāha – ‘‘āgamentu kira bhavanto, kūṭadantopi brāhmaṇo samaṇaṃ gotamaṃ dassanāya upasaṅkamissatī’’’ti. ‘‘Evaṃ, bho’’ti kho so khattā kūṭadantassa brāhmaṇassa paṭissutvā yena khāṇumatakā brāhmaṇagahapatikā tenupasaṅkami. Upasaṅkamitvā khāṇumatake brāhmaṇagahapatike etadavoca – ‘‘kūṭadanto, bho, brāhmaṇo evamāha – ‘āgamentu kira bhonto, kūṭadantopi brāhmaṇo samaṇaṃ gotamaṃ dassanāya upasaṅkamissatī’’’ti.

\subsubsection{Kūṭadantaguṇakathā}

\paragraph{330.} Tena kho pana samayena anekāni brāhmaṇasatāni khāṇumate paṭivasanti – ‘‘kūṭadantassa brāhmaṇassa mahāyaññaṃ anubhavissāmā’’ti. Assosuṃ kho te brāhmaṇā – ‘‘kūṭadanto kira brāhmaṇo samaṇaṃ gotamaṃ dassanāya upasaṅkamissatī’’ti. Atha kho te brāhmaṇā yena kūṭadanto brāhmaṇo tenupasaṅkamiṃsu.

\paragraph{331.} Upasaṅkamitvā kūṭadantaṃ brāhmaṇaṃ etadavocuṃ – ‘‘saccaṃ kira bhavaṃ kūṭadanto samaṇaṃ gotamaṃ dassanāya upasaṅkamissatī’’ti? ‘‘Evaṃ kho me, bho, hoti – ‘ahampi samaṇaṃ gotamaṃ dassanāya upasaṅkamissāmī’’’ti. ‘‘Mā bhavaṃ kūṭadanto samaṇaṃ gotamaṃ dassanāya upasaṅkami. Na arahati bhavaṃ kūṭadanto samaṇaṃ gotamaṃ dassanāya upasaṅkamituṃ. Sace bhavaṃ kūṭadanto samaṇaṃ gotamaṃ dassanāya upasaṅkamissati, bhoto kūṭadantassa yaso hāyissati, samaṇassa gotamassa yaso abhivaḍḍhissati. Yampi bhoto kūṭadantassa yaso hāyissati, samaṇassa gotamassa yaso abhivaḍḍhissati, imināpaṅgena na arahati bhavaṃ kūṭadanto samaṇaṃ gotamaṃ dassanāya upasaṅkamituṃ. Samaṇo tveva gotamo arahati bhavantaṃ kūṭadantaṃ dassanāya upasaṅkamituṃ. ‘‘Bhavañhi kūṭadanto ubhato sujāto mātito ca pitito ca saṃsuddhagahaṇiko yāva sattamā pitāmahayugā akkhitto anupakkuṭṭho jātivādena. Yampi bhavaṃ kūṭadanto ubhato sujāto mātito ca pitito ca saṃsuddhagahaṇiko yāva sattamā pitāmahayugā akkhitto anupakkuṭṭho jātivādena, imināpaṅgena na arahati bhavaṃ kūṭadanto samaṇaṃ gotamaṃ dassanāya upasaṅkamituṃ. Samaṇo tveva gotamo arahati bhavantaṃ kūṭadantaṃ dassanāya upasaṅkamituṃ. ‘‘Bhavañhi kūṭadanto aḍḍho mahaddhano mahābhogo pahūtavittūpakaraṇo pahūtajātarūparajato…pe… ‘‘Bhavañhi kūṭadanto ajjhāyako mantadharo tiṇṇaṃ vedānaṃ pāragū sanighaṇḍukeṭubhānaṃ sākkharappabhedānaṃ itihāsapañcamānaṃ padako veyyākaraṇo lokāyatamahāpurisalakkhaṇesu anavayo…pe… ‘‘Bhavañhi kūṭadanto abhirūpo dassanīyo pāsādiko paramāya vaṇṇapokkharatāya samannāgato brahmavaṇṇī brahmavacchasī akhuddāvakāso dassanāya…pe… ‘‘Bhavañhi kūṭadanto sīlavā vuddhasīlī vuddhasīlena samannāgato…pe… ‘‘Bhavañhi kūṭadanto kalyāṇavāco kalyāṇavākkaraṇo poriyā vācāya samannāgato vissaṭṭhāya anelagalāya atthassa viññāpaniyā…pe… ‘‘Bhavañhi kūṭadanto bahūnaṃ ācariyapācariyo tīṇi māṇavakasatāni mante vāceti, bahū kho pana nānādisā nānājanapadā māṇavakā āgacchanti bhoto kūṭadantassa santike mantatthikā mante adhiyitukāmā…pe… ‘‘Bhavañhi kūṭadanto jiṇṇo vuddho mahallako addhagato vayoanuppatto. Samaṇo gotamo taruṇo ceva taruṇapabbajito ca…pe… ‘‘Bhavañhi kūṭadanto rañño māgadhassa seniyassa bimbisārassa sakkato garukato mānito pūjito apacito…pe… ‘‘Bhavañhi kūṭadanto brāhmaṇassa pokkharasātissa sakkato garukato mānito pūjito apacito…pe… ‘‘Bhavañhi kūṭadanto khāṇumataṃ ajjhāvasati sattussadaṃ satiṇakaṭṭhodakaṃ sadhaññaṃ rājabhoggaṃ raññā māgadhena seniyena bimbisārena dinnaṃ rājadāyaṃ brahmadeyyaṃ. Yampi bhavaṃ kūṭadanto khāṇumataṃ ajjhāvasati sattussadaṃ satiṇakaṭṭhodakaṃ sadhaññaṃ rājabhoggaṃ, raññā māgadhena seniyena bimbisārena dinnaṃ rājadāyaṃ brahmadeyyaṃ, imināpaṅgena na arahati bhavaṃ kūṭadanto samaṇaṃ gotamaṃ dassanāya upasaṅkamituṃ. Samaṇotveva gotamo arahati bhavantaṃ kūṭadantaṃ dassanāya upasaṅkamitu’’nti.

\subsubsection{Buddhaguṇakathā}

\paragraph{332.} Evaṃ vutte kūṭadanto brāhmaṇo te brāhmaṇe etadavoca – ‘‘Tena hi, bho, mamapi suṇātha, yathā mayameva arahāma taṃ bhavantaṃ gotamaṃ dassanāya upasaṅkamituṃ, na tveva arahati so bhavaṃ gotamo amhākaṃ dassanāya upasaṅkamituṃ. Samaṇo khalu, bho, gotamo ubhato sujāto mātito ca pitito ca saṃsuddhagahaṇiko yāva sattamā pitāmahayugā akkhitto anupakkuṭṭho jātivādena. Yampi, bho, samaṇo gotamo ubhato sujāto mātito ca pitito ca saṃsuddhagahaṇiko yāva sattamā pitāmahayugā akkhitto anupakkuṭṭho jātivādena, imināpaṅgena na arahati so bhavaṃ gotamo amhākaṃ dassanāya upasaṅkamituṃ. Atha kho mayameva arahāma taṃ bhavantaṃ gotamaṃ dassanāya upasaṅkamituṃ. ‘‘Samaṇo khalu, bho, gotamo mahantaṃ ñātisaṅghaṃ ohāya pabbajito…pe… ‘‘Samaṇo khalu, bho, gotamo pahūtaṃ hiraññasuvaṇṇaṃ ohāya pabbajito bhūmigatañca vehāsaṭṭhaṃ ca…pe… ‘‘Samaṇo khalu, bho, gotamo daharova samāno yuvā susukāḷakeso bhadrena yobbanena samannāgato paṭhamena vayasā agārasmā anagāriyaṃ pabbajito…pe… ‘‘Samaṇo khalu, bho, gotamo akāmakānaṃ mātāpitūnaṃ assumukhānaṃ rudantānaṃ kesamassuṃ ohāretvā kāsāyāni vatthāni acchādetvā agārasmā anagāriyaṃ pabbajito…pe… ‘‘Samaṇo khalu, bho, gotamo abhirūpo dassanīyo pāsādiko paramāya vaṇṇapokkharatāya samannāgato brahmavaṇṇī brahmavacchasī akhuddāvakāso dassanāya …pe… ‘‘Samaṇo khalu, bho, gotamo sīlavā ariyasīlī kusalasīlī kusalasīlena samannāgato…pe… ‘‘Samaṇo khalu, bho, gotamo kalyāṇavāco kalyāṇavākkaraṇo poriyā vācāya samannāgato vissaṭṭhāya anelagalāya atthassa viññāpaniyā…pe… ‘‘Samaṇo khalu, bho, gotamo bahūnaṃ ācariyapācariyo…pe… ‘‘Samaṇo khalu, bho, gotamo khīṇakāmarāgo vigatacāpallo…pe… ‘‘Samaṇo khalu, bho, gotamo kammavādī kiriyavādī apāpapurekkhāro brahmaññāya pajāya…pe… ‘‘Samaṇo khalu, bho, gotamo uccā kulā pabbajito asambhinnakhattiyakulā…pe… ‘‘Samaṇo khalu, bho, gotamo aḍḍhā kulā pabbajito mahaddhanā mahābhogā…pe… ‘‘Samaṇaṃ khalu, bho, gotamaṃ tiroraṭṭhā tirojanapadā pañhaṃ pucchituṃ āgacchanti…pe… ‘‘Samaṇaṃ khalu, bho, gotamaṃ anekāni devatāsahassāni pāṇehi saraṇaṃ gatāni… pe… ‘‘Samaṇaṃ khalu, bho, gotamaṃ evaṃ kalyāṇo kittisaddo abbhuggato – ‘itipi so bhagavā arahaṃ sammāsambuddho vijjācaraṇasampanno sugato lokavidū anuttaro purisadammasārathi satthā devamanussānaṃ buddho bhagavā’ ti…pe… ‘‘Samaṇo khalu, bho, gotamo dvattiṃsamahāpurisalakkhaṇehi samannāgato…pe… ‘‘Samaṇo khalu, bho, gotamo ehisvāgatavādī sakhilo sammodako abbhākuṭiko uttānamukho pubbabhāsī…pe… ‘‘Samaṇo khalu, bho, gotamo catunnaṃ parisānaṃ sakkato garukato mānito pūjito apacito…pe… ‘‘Samaṇe khalu, bho, gotame bahū devā ca manussā ca abhippasannā…pe… ‘‘Samaṇo khalu, bho, gotamo yasmiṃ gāme vā nigame vā paṭivasati na tasmiṃ gāme vā nigame vā amanussā manusse viheṭhenti…pe… ‘‘Samaṇo khalu, bho, gotamo saṅghī gaṇī gaṇācariyo puthutitthakarānaṃ aggamakkhāyati, yathā kho pana, bho, etesaṃ samaṇabrāhmaṇānaṃ yathā vā tathā vā yaso samudāgacchati, na hevaṃ samaṇassa gotamassa yaso samudāgato. Atha kho anuttarāya vijjācaraṇasampadāya samaṇassa gotamassa yaso samudāgato…pe… ‘‘Samaṇaṃ khalu, bho, gotamaṃ rājā māgadho seniyo bimbisāro saputto sabhariyo sapariso sāmacco pāṇehi saraṇaṃ gato…pe… ‘‘Samaṇaṃ khalu, bho, gotamaṃ rājā pasenadi kosalo saputto sabhariyo sapariso sāmacco pāṇehi saraṇaṃ gato…pe… ‘‘Samaṇaṃ khalu, bho, gotamaṃ brāhmaṇo pokkharasāti saputto sabhariyo sapariso sāmacco pāṇehi saraṇaṃ gato…pe… ‘‘Samaṇo khalu, bho, gotamo rañño māgadhassa seniyassa bimbisārassa sakkato garukato mānito pūjito apacito…pe… ‘‘Samaṇo khalu, bho, gotamo rañño pasenadissa kosalassa sakkato garukato mānito pūjito apacito…pe… ‘‘Samaṇo khalu, bho, gotamo brāhmaṇassa pokkharasātissa sakkato garukato mānito pūjito apacito…pe… ‘‘Samaṇo khalu, bho, gotamo khāṇumataṃ anuppatto khāṇumate viharati ambalaṭṭhikāyaṃ. Ye kho pana, bho, keci samaṇā vā brāhmaṇā vā amhākaṃ gāmakhettaṃ āgacchanti, atithī no te honti. Atithī kho panamhehi sakkātabbā garukātabbā mānetabbā pūjetabbā apacetabbā. Yampi, bho, samaṇo gotamo khāṇumataṃ anuppatto khāṇumate viharati ambalaṭṭhikāyaṃ, atithimhākaṃ samaṇo gotamo. Atithi kho panamhehi sakkātabbo garukātabbo mānetabbo pūjetabbo apacetabbo. Imināpaṅgena nārahati so bhavaṃ gotamo amhākaṃ dassanāya upasaṅkamituṃ. Atha kho mayameva arahāma taṃ bhavantaṃ gotamaṃ dassanāya upasaṅkamituṃ. Ettake kho ahaṃ, bho, tassa bhoto gotamassa vaṇṇe pariyāpuṇāmi, no ca kho so bhavaṃ gotamo ettakavaṇṇo. Aparimāṇavaṇṇo hi so bhavaṃ gotamo’’ti.

\paragraph{333.} Evaṃ vutte, te brāhmaṇā kūṭadantaṃ brāhmaṇaṃ etadavocuṃ – ‘‘yathā kho bhavaṃ kūṭadanto samaṇassa gotamassa vaṇṇe bhāsati, ito cepi so bhavaṃ gotamo yojanasate viharati, alameva saddhena kulaputtena dassanāya upasaṅkamituṃ api puṭosenā’’ti. ‘‘Tena hi, bho, sabbeva mayaṃ samaṇaṃ gotamaṃ dassanāya upasaṅkamissāmā’’ti.

\subsubsection{Mahāvijitarājayaññakathā}

\paragraph{334.} Atha kho kūṭadanto brāhmaṇo mahatā brāhmaṇagaṇena saddhiṃ yena ambalaṭṭhikā yena bhagavā tenupasaṅkami, upasaṅkamitvā bhagavatā saddhiṃ sammodi. Sammodanīyaṃ kathaṃ sāraṇīyaṃ vītisāretvā ekamantaṃ nisīdi. Khāṇumatakāpi kho brāhmaṇagahapatikā appekacce bhagavantaṃ abhivādetvā ekamantaṃ nisīdiṃsu; appekacce bhagavatā saddhiṃ sammodiṃsu, sammodanīyaṃ kathaṃ sāraṇīyaṃ vītisāretvā ekamantaṃ nisīdiṃsu; appekacce yena bhagavā tenañjaliṃ paṇāmetvā ekamantaṃ nisīdiṃsu; appekacce nāmagottaṃ sāvetvā ekamantaṃ nisīdiṃsu; appekacce tuṇhībhūtā ekamantaṃ nisīdiṃsu.

\paragraph{335.} Ekamantaṃ nisinno kho kūṭadanto brāhmaṇo bhagavantaṃ etadavoca – ‘‘sutaṃ metaṃ, bho gotama – ‘samaṇo gotamo tividhaṃ yaññasampadaṃ soḷasaparikkhāraṃ jānātī’ti. Na kho panāhaṃ jānāmi tividhaṃ yaññasampadaṃ soḷasaparikkhāraṃ. Icchāmi cāhaṃ mahāyaññaṃ yajituṃ. Sādhu me bhavaṃ gotamo tividhaṃ yaññasampadaṃ soḷasaparikkhāraṃ desetū’’ti.

\paragraph{336.} ‘‘Tena hi, brāhmaṇa, suṇāhi sādhukaṃ manasikarohi, bhāsissāmī’’ti. ‘‘Evaṃ, bho’’ti kho kūṭadanto brāhmaṇo bhagavato paccassosi. Bhagavā etadavoca – ‘‘bhūtapubbaṃ, brāhmaṇa, rājā mahāvijito nāma ahosi aḍḍho mahaddhano mahābhogo pahūtajātarūparajato pahūtavittūpakaraṇo pahūtadhanadhañño paripuṇṇakosakoṭṭhāgāro. Atha kho, brāhmaṇa, rañño mahāvijitassa rahogatassa paṭisallīnassa evaṃ cetaso parivitakko udapādi – ‘adhigatā kho me vipulā mānusakā bhogā, mahantaṃ pathavimaṇḍalaṃ abhivijiya ajjhāvasāmi, yaṃnūnāhaṃ mahāyaññaṃ yajeyyaṃ, yaṃ mama assa dīgharattaṃ hitāya sukhāyā’ti.

\paragraph{337.} ‘‘Atha kho, brāhmaṇa, rājā mahāvijito purohitaṃ brāhmaṇaṃ āmantetvā etadavoca – ‘idha mayhaṃ, brāhmaṇa, rahogatassa paṭisallīnassa evaṃ cetaso parivitakko udapādi – adhigatā kho me vipulā mānusakā bhogā, mahantaṃ pathavimaṇḍalaṃ abhivijiya ajjhāvasāmi. Yaṃnūnāhaṃ mahāyaññaṃ yajeyyaṃ yaṃ mama assa dīgharattaṃ hitāya sukhāyā’ti. Icchāmahaṃ, brāhmaṇa, mahāyaññaṃ yajituṃ. Anusāsatu maṃ bhavaṃ yaṃ mama assa dīgharattaṃ hitāya sukhāyā’’’ti.

\paragraph{338.} ‘‘Evaṃ vutte, brāhmaṇa, purohito brāhmaṇo rājānaṃ mahāvijitaṃ etadavoca – ‘bhoto kho rañño janapado sakaṇṭako sauppīḷo, gāmaghātāpi dissanti, nigamaghātāpi dissanti, nagaraghātāpi dissanti, panthaduhanāpi dissanti. Bhavaṃ kho pana rājā evaṃ sakaṇṭake janapade sauppīḷe balimuddhareyya, akiccakārī assa tena bhavaṃ rājā. Siyā kho pana bhoto rañño evamassa – ‘‘ahametaṃ dassukhīlaṃ vadhena vā bandhena vā jāniyā vā garahāya vā pabbājanāya vā samūhanissāmī’’ti, na kho panetassa dassukhīlassa evaṃ sammā samugghāto hoti. Ye te hatāvasesakā bhavissanti, te pacchā rañño janapadaṃ viheṭhessanti. Api ca kho idaṃ saṃvidhānaṃ āgamma evametassa dassukhīlassa sammā samugghāto hoti. Tena hi bhavaṃ rājā ye bhoto rañño janapade ussahanti kasigorakkhe, tesaṃ bhavaṃ rājā bījabhattaṃ anuppadetu. Ye bhoto rañño janapade ussahanti vāṇijjāya, tesaṃ bhavaṃ rājā pābhataṃ anuppadetu. Ye bhoto rañño janapade ussahanti rājaporise, tesaṃ bhavaṃ rājā bhattavetanaṃ pakappetu. Te ca manussā sakammapasutā rañño janapadaṃ na viheṭhessanti; mahā ca rañño rāsiko bhavissati. Khemaṭṭhitā janapadā akaṇṭakā anuppīḷā. Manussā mudā modamānā ure putte naccentā apārutagharā maññe viharissantī’ti. ‘Evaṃ, bho’ti kho, brāhmaṇa, rājā mahāvijito purohitassa brāhmaṇassa paṭissutvā ye rañño janapade ussahiṃsu kasigorakkhe, tesaṃ rājā mahāvijito bījabhattaṃ anuppadāsi. Ye ca rañño janapade ussahiṃsu vāṇijjāya, tesaṃ rājā mahāvijito pābhataṃ anuppadāsi. Ye ca rañño janapade ussahiṃsu rājaporise, tesaṃ rājā mahāvijito bhattavetanaṃ pakappesi. Te ca manussā sakammapasutā rañño janapadaṃ na viheṭhiṃsu, mahā ca rañño rāsiko ahosi. Khemaṭṭhitā janapadā akaṇṭakā anuppīḷā manussā mudā modamānā ure putte naccentā apārutagharā maññe vihariṃsu. Atha kho, brāhmaṇa, rājā mahāvijito purohitaṃ brāhmaṇaṃ āmantetvā etadavoca – ‘samūhato kho me bhoto dassukhīlo, bhoto saṃvidhānaṃ āgamma mahā ca me rāsiko. Khemaṭṭhitā janapadā akaṇṭakā anuppīḷā manussā mudā modamānā ure putte naccentā apārutagharā maññe viharanti. Icchāmahaṃ brāhmaṇa mahāyaññaṃ yajituṃ. Anusāsatu maṃ bhavaṃ yaṃ mama assa dīgharattaṃ hitāya sukhāyā’ti.

\subsubsection{Catuparikkhāraṃ}

\paragraph{339.} ‘‘Tena hi bhavaṃ rājā ye bhoto rañño janapade khattiyā ānuyantā negamā ceva jānapadā ca te bhavaṃ rājā āmantayataṃ – ‘icchāmahaṃ, bho, mahāyaññaṃ yajituṃ, anujānantu me bhavanto yaṃ mama assa dīgharattaṃ hitāya sukhāyā’ti. Ye bhoto rañño janapade amaccā pārisajjā negamā ceva jānapadā ca…pe… brāhmaṇamahāsālā negamā ceva jānapadā ca…pe… gahapatinecayikā negamā ceva jānapadā ca, te bhavaṃ rājā āmantayataṃ – ‘icchāmahaṃ, bho, mahāyaññaṃ yajituṃ, anujānantu me bhavanto yaṃ mama assa dīgharattaṃ hitāya sukhāyā’ti. ‘Evaṃ, bho’ti kho, brāhmaṇa, rājā mahāvijito purohitassa brāhmaṇassa paṭissutvā ye rañño janapade khattiyā ānuyantā negamā ceva jānapadā ca, te rājā mahāvijito āmantesi – ‘icchāmahaṃ, bho, mahāyaññaṃ yajituṃ, anujānantu me bhavanto yaṃ mama assa dīgharattaṃ hitāya sukhāyā’’ti. ‘Yajataṃ bhavaṃ rājā yaññaṃ, yaññakālo mahārājā’ti. Ye rañño janapade amaccā pārisajjā negamā ceva jānapadā ca…pe… brāhmaṇamahāsālā negamā ceva jānapadā ca…pe… gahapatinecayikā negamā ceva jānapadā ca, te rājā mahāvijito āmantesi – ‘icchāmahaṃ, bho, mahāyaññaṃ yajituṃ. Anujānantu me bhavanto yaṃ mama assa dīgharattaṃ hitāya sukhāyā’ti. ‘Yajataṃ bhavaṃ rājā yaññaṃ, yaññakālo mahārājā’ti. Itime cattāro anumatipakkhā tasseva yaññassa parikkhārā bhavanti.

\subsubsection{Aṭṭha parikkhārā}

\paragraph{340.} ‘‘Rājā mahāvijito aṭṭhahaṅgehi samannāgato, ubhato sujāto mātito ca pitito ca saṃsuddhagahaṇiko yāva sattamā pitāmahayugā akkhitto anupakkuṭṭho jātivādena abhirūpo dassanīyo pāsādiko paramāya vaṇṇapokkharatāya samannāgato brahmavaṇṇī brahmavacchasī akhuddāvakāso dassanāya; aḍḍho mahaddhano mahābhogo pahūtajātarūparajato pahūtavittūpakaraṇo pahūtadhanadhañño paripuṇṇakosakoṭṭhāgāro; balavā caturaṅginiyā senāya samannāgato assavāya ovādapaṭikarāya sahati\footnote{patapati (sī. pī.), tapati (syā.)} maññe paccatthike yasasā; saddho dāyako dānapati anāvaṭadvāro samaṇabrāhmaṇakapaṇaddhikavaṇibbakayācakānaṃ opānabhūto puññāni karoti; bahussuto tassa tassa sutajātassa, tassa tasseva kho pana bhāsitassa atthaṃ jānāti ‘ayaṃ imassa bhāsitassa attho ayaṃ imassa bhāsitassa attho’ti; paṇḍito, viyatto, medhāvī, paṭibalo, atītānāgatapaccuppanne atthe cintetuṃ. Rājā mahāvijito imehi aṭṭhahaṅgehi samannāgato. Iti imānipi aṭṭhaṅgāni tasseva yaññassa parikkhārā bhavanti.

\subsubsection{Catuparikkhāraṃ}

\paragraph{341.} ‘‘Purohito\footnote{purohitopi (ka. sī. ka.)} brāhmaṇo catuhaṅgehi samannāgato. Ubhato sujāto mātito ca pitito ca saṃsuddhagahaṇiko yāva sattamā pitāmahayugā akkhitto anupakkuṭṭho jātivādena; ajjhāyako mantadharo tiṇṇaṃ vedānaṃ pāragū sanighaṇḍukeṭubhānaṃ sākkharappabhedānaṃ itihāsapañcamānaṃ padako veyyākaraṇo lokāyatamahāpurisalakkhaṇesu anavayo; sīlavā vuddhasīlī vuddhasīlena samannāgato; paṇḍito viyatto medhāvī paṭhamo vā dutiyo vā sujaṃ paggaṇhantānaṃ. Purohito brāhmaṇo imehi catūhaṅgehi samannāgato. Iti imāni cattāri aṅgāni tasseva yaññassa parikkhārā bhavanti.

\subsubsection{Tisso vidhā}

\paragraph{342.} ‘‘Atha kho, brāhmaṇa, purohito brāhmaṇo rañño mahāvijitassa pubbeva yaññā tisso vidhā desesi. Siyā kho pana bhoto rañño mahāyaññaṃ yiṭṭhukāmassa\footnote{yiṭṭhakāmassa (ka.)} kocideva vippaṭisāro – ‘mahā vata me bhogakkhandho vigacchissatī’ti, so bhotā raññā vippaṭisāro na karaṇīyo. Siyā kho pana bhoto rañño mahāyaññaṃ yajamānassa kocideva vippaṭisāro – ‘mahā vata me bhogakkhandho vigacchatī’ti, so bhotā raññā vippaṭisāro na karaṇīyo. Siyā kho pana bhoto rañño mahāyaññaṃ yiṭṭhassa kocideva vippaṭisāro – ‘mahā vata me bhogakkhandho vigato’ti, so bhotā raññā vippaṭisāro na karaṇīyo’’ti. Imā kho, brāhmaṇa, purohito brāhmaṇo rañño mahāvijitassa pubbeva yaññā tisso vidhā desesi.

\subsubsection{Dasa ākārā}

\paragraph{343.} ‘‘Atha kho, brāhmaṇa, purohito brāhmaṇo rañño mahāvijitassa pubbeva yaññā dasahākārehi paṭiggāhakesu vippaṭisāraṃ paṭivinesi. ‘Āgamissanti kho bhoto yaññaṃ pāṇātipātinopi pāṇātipātā paṭiviratāpi. Ye tattha pāṇātipātino, tesaññeva tena. Ye tattha pāṇātipātā paṭiviratā, te ārabbha yajataṃ bhavaṃ, sajjataṃ bhavaṃ, modataṃ bhavaṃ, cittameva bhavaṃ antaraṃ pasādetu. Āgamissanti kho bhoto yaññaṃ adinnādāyinopi adinnādānā paṭiviratāpi…pe… kāmesu micchācārinopi kāmesumicchācārā paṭiviratāpi… musāvādinopi musāvādā paṭiviratāpi… pisuṇavācinopi pisuṇāya vācāya paṭiviratāpi… pharusavācinopi pharusāya vācāya paṭiviratāpi… samphappalāpinopi samphappalāpā paṭiviratāpi … abhijjhālunopi anabhijjhālunopi… byāpannacittāpi abyāpannacittāpi… micchādiṭṭhikāpi sammādiṭṭhikāpi…. Ye tattha micchādiṭṭhikā, tesaññeva tena. Ye tattha sammādiṭṭhikā, te ārabbha yajataṃ bhavaṃ, sajjataṃ bhavaṃ, modataṃ bhavaṃ, cittameva bhavaṃ antaraṃ pasādetū’ti. Imehi kho, brāhmaṇa, purohito brāhmaṇo rañño mahāvijitassa pubbeva yaññā dasahākārehi paṭiggāhakesu vippaṭisāraṃ paṭivinesi.

\subsubsection{Soḷasa ākārā}

\paragraph{344.} ‘‘Atha kho, brāhmaṇa, purohito brāhmaṇo rañño mahāvijitassa mahāyaññaṃ yajamānassa soḷasahākārehi cittaṃ sandassesi samādapesi samuttejesi sampahaṃsesi siyā kho pana bhoto rañño mahāyaññaṃ yajamānassa kocideva vattā – ‘rājā kho mahāvijito mahāyaññaṃ yajati, no ca kho tassa āmantitā khattiyā ānuyantā negamā ceva jānapadā ca; atha ca pana bhavaṃ rājā evarūpaṃ mahāyaññaṃ yajatī’ti. Evampi bhoto rañño vattā dhammato natthi. Bhotā kho pana raññā āmantitā khattiyā ānuyantā negamā ceva jānapadā ca. Imināpetaṃ bhavaṃ rājā jānātu, yajataṃ bhavaṃ, sajjataṃ bhavaṃ, modataṃ bhavaṃ, cittameva bhavaṃ antaraṃ pasādetu. ‘‘Siyā kho pana bhoto rañño mahāyaññaṃ yajamānassa kocideva vattā – ‘rājā kho mahāvijito mahāyaññaṃ yajati, no ca kho tassa āmantitā amaccā pārisajjā negamā ceva jānapadā ca…pe… brāhmaṇamahāsālā negamā ceva jānapadā ca…pe… gahapatinecayikā negamā ceva jānapadā ca, atha ca pana bhavaṃ rājā evarūpaṃ mahāyaññaṃ yajatī’ti. Evampi bhoto rañño vattā dhammato natthi. Bhotā kho pana raññā āmantitā gahapatinecayikā negamā ceva jānapadā ca. Imināpetaṃ bhavaṃ rājā jānātu, yajataṃ bhavaṃ, sajjataṃ bhavaṃ, modataṃ bhavaṃ, cittameva bhavaṃ antaraṃ pasādetu. ‘‘Siyā kho pana bhoto rañño mahāyaññaṃ yajamānassa kocideva vattā – ‘rājā kho mahāvijito mahāyaññaṃ yajati, no ca kho ubhato sujāto mātito ca pitito ca saṃsuddhagahaṇiko yāva sattamā pitāmahayugā akkhitto anupakkuṭṭho jātivādena, atha ca pana bhavaṃ rājā evarūpaṃ mahāyaññaṃ yajatī’ti. Evampi bhoto rañño vattā dhammato natthi. Bhavaṃ kho pana rājā ubhato sujāto mātito ca pitito ca saṃsuddhagahaṇiko yāva sattamā pitāmahayugā akkhitto anupakkuṭṭho jātivādena. Imināpetaṃ bhavaṃ rājā jānātu, yajataṃ bhavaṃ, sajjataṃ bhavaṃ, modataṃ bhavaṃ, cittameva bhavaṃ antaraṃ pasādetu. ‘‘Siyā kho pana bhoto rañño mahāyaññaṃ yajamānassa kocideva vattā – ‘rājā kho mahāvijito mahāyaññaṃ yajati no ca kho abhirūpo dassanīyo pāsādiko paramāya vaṇṇapokkharatāya samannāgato brahmavaṇṇī brahmavacchasī akhuddāvakāso dassanāya…pe… no ca kho aḍḍho mahaddhano mahābhogo pahūtajātarūparajato pahūtavittūpakaraṇo pahūtadhanadhañño paripuṇṇakosakoṭṭhāgāro…pe… no ca kho balavā caturaṅginiyā senāya samannāgato assavāya ovādapaṭikarāya sahati maññe paccatthike yasasā…pe… no ca kho saddho dāyako dānapati anāvaṭadvāro samaṇabrāhmaṇakapaṇaddhikavaṇibbakayācakānaṃ opānabhūto puññāni karoti…pe… no ca kho bahussuto tassa tassa sutajātassa…pe… no ca kho tassa tasseva kho pana bhāsitassa atthaṃ jānāti ‘‘ayaṃ imassa bhāsitassa attho, ayaṃ imassa bhāsitassa attho’’ti… pe… no ca kho paṇḍito viyatto medhāvī paṭibalo atītānāgatapaccuppanne atthe cintetuṃ, atha ca pana bhavaṃ rājā evarūpaṃ mahāyaññaṃ yajatī’ti. Evampi bhoto rañño vattā dhammato natthi. Bhavaṃ kho pana rājā paṇḍito viyatto medhāvī paṭibalo atītānāgatapaccuppanne atthe cintetuṃ. Imināpetaṃ bhavaṃ rājā jānātu, yajataṃ bhavaṃ, sajjataṃ bhavaṃ, modataṃ bhavaṃ, cittameva bhavaṃ antaraṃ pasādetu. ‘‘Siyā kho pana bhoto rañño mahāyaññaṃ yajamānassa kocideva vattā – ‘rājā kho mahāvijito mahāyaññaṃ yajati. No ca khvassa purohito brāhmaṇo ubhato sujāto mātito ca pitito ca saṃsuddhagahaṇiko yāva sattamā pitāmahayugā akkhitto anupakkuṭṭho jātivādena; atha ca pana bhavaṃ rājā evarūpaṃ mahāyaññaṃ yajatī’ti. Evampi bhoto rañño vattā dhammato natthi. Bhoto kho pana rañño purohito brāhmaṇo ubhato sujāto mātito ca pitito ca saṃsuddhagahaṇiko yāva sattamā pitāmahayugā akkhitto anupakkuṭṭho jātivādena. Imināpetaṃ bhavaṃ rājā jānātu, yajataṃ bhavaṃ, sajjataṃ bhavaṃ, modataṃ bhavaṃ, cittameva bhavaṃ antaraṃ pasādetu. ‘‘Siyā kho pana bhoto rañño mahāyaññaṃ yajamānassa kocideva vattā – ‘rājā kho mahāvijito mahāyaññaṃ yajati. No ca khvassa purohito brāhmaṇo ajjhāyako mantadharo tiṇṇaṃ vedānaṃ pāragū sanighaṇḍukeṭubhānaṃ sākkharappabhedānaṃ itihāsapañcamānaṃ padako veyyākaraṇo lokāyatamahāpurisalakkhaṇesu anavayo…pe… no ca khvassa purohito brāhmaṇo sīlavā vuddhasīlī vuddhasīlena samannāgato…pe… no ca khvassa purohito brāhmaṇo paṇḍito viyatto medhāvī paṭhamo vā dutiyo vā sujaṃ paggaṇhantānaṃ, atha ca pana bhavaṃ rājā evarūpaṃ mahāyaññaṃ yajatī’ti. Evampi bhoto rañño vattā dhammato natthi. Bhoto kho pana rañño purohito brāhmaṇo paṇḍito viyatto medhāvī paṭhamo vā dutiyo vā sujaṃ paggaṇhantānaṃ. Imināpetaṃ bhavaṃ rājā jānātu, yajataṃ bhavaṃ, sajjataṃ bhavaṃ, modataṃ bhavaṃ, cittameva bhavaṃ antaraṃ pasādetūti. Imehi kho, brāhmaṇa, purohito brāhmaṇo rañño mahāvijitassa mahāyaññaṃ yajamānassa soḷasahi ākārehi cittaṃ sandassesi samādapesi samuttejesi sampahaṃsesi.

\paragraph{345.} ‘‘Tasmiṃ kho, brāhmaṇa, yaññe neva gāvo haññiṃsu, na ajeḷakā haññiṃsu, na kukkuṭasūkarā haññiṃsu, na vividhā pāṇā saṃghātaṃ āpajjiṃsu, na rukkhā chijjiṃsu yūpatthāya, na dabbhā lūyiṃsu barihisatthāya\footnote{parihiṃsatthāya (syā. ka. sī. ka.), parahiṃsatthāya (ka.)}. Yepissa ahesuṃ dāsāti vā pessāti vā kammakarāti vā, tepi na daṇḍatajjitā na bhayatajjitā na assumukhā rudamānā parikammāni akaṃsu. Atha kho ye icchiṃsu, te akaṃsu, ye na icchiṃsu, na te akaṃsu; yaṃ icchiṃsu, taṃ akaṃsu, yaṃ na icchiṃsu, na taṃ akaṃsu. Sappitelanavanītadadhimadhuphāṇitena ceva so yañño niṭṭhānamagamāsi.

\paragraph{346.} ‘‘Atha kho, brāhmaṇa, khattiyā ānuyantā negamā ceva jānapadā ca, amaccā pārisajjā negamā ceva jānapadā ca, brāhmaṇamahāsālā negamā ceva jānapadā ca, gahapatinecayikā negamā ceva jānapadā ca pahūtaṃ sāpateyyaṃ ādāya rājānaṃ mahāvijitaṃ upasaṅkamitvā evamāhaṃsu – ‘idaṃ, deva, pahūtaṃ sāpateyyaṃ devaññeva uddissābhataṃ, taṃ devo paṭiggaṇhātū’ti. ‘Alaṃ, bho, mamāpidaṃ pahūtaṃ sāpateyyaṃ dhammikena balinā abhisaṅkhataṃ; tañca vo hotu, ito ca bhiyyo harathā’ti. Te raññā paṭikkhittā ekamantaṃ apakkamma evaṃ samacintesuṃ – ‘na kho etaṃ amhākaṃ patirūpaṃ, yaṃ mayaṃ imāni sāpateyyāni punadeva sakāni gharāni paṭihareyyāma. Rājā kho mahāvijito mahāyaññaṃ yajati, handassa mayaṃ anuyāgino homā’ti.

\paragraph{347.} ‘‘Atha kho, brāhmaṇa, puratthimena yaññavāṭassa\footnote{yaññāvāṭassa (sī. pī. ka.)} khattiyā ānuyantā negamā ceva jānapadā ca dānāni paṭṭhapesuṃ. Dakkhiṇena yaññavāṭassa amaccā pārisajjā negamā ceva jānapadā ca dānāni paṭṭhapesuṃ. Pacchimena yaññavāṭassa brāhmaṇamahāsālā negamā ceva jānapadā ca dānāni paṭṭhapesuṃ. Uttarena yaññavāṭassa gahapatinecayikā negamā ceva jānapadā ca dānāni paṭṭhapesuṃ. ‘‘Tesupi kho, brāhmaṇa, yaññesu neva gāvo haññiṃsu, na ajeḷakā haññiṃsu, na kukkuṭasūkarā haññiṃsu, na vividhā pāṇā saṃghātaṃ āpajjiṃsu, na rukkhā chijjiṃsu yūpatthāya, na dabbhā lūyiṃsu barihisatthāya. Yepi nesaṃ ahesuṃ dāsāti vā pessāti vā kammakarāti vā, tepi na daṇḍatajjitā na bhayatajjitā na assumukhā rudamānā parikammāni akaṃsu. Atha kho ye icchiṃsu, te akaṃsu, ye na icchiṃsu, na te akaṃsu; yaṃ icchiṃsu, taṃ akaṃsu, yaṃ na icchiṃsu na taṃ akaṃsu. Sappitelanavanītadadhimadhuphāṇitena ceva te yaññā niṭṭhānamagamaṃsu. ‘‘Iti cattāro ca anumatipakkhā, rājā mahāvijito aṭṭhahaṅgehi samannāgato, purohito brāhmaṇo catūhaṅgehi samannāgato; tisso ca vidhā ayaṃ vuccati brāhmaṇa tividhā yaññasampadā soḷasaparikkhārā’’ti.

\paragraph{348.} Evaṃ vutte, te brāhmaṇā unnādino uccāsaddamahāsaddā ahesuṃ – ‘‘aho yañño, aho yaññasampadā’’ti! Kūṭadanto pana brāhmaṇo tūṇhībhūtova nisinno hoti. Atha kho te brāhmaṇā kūṭadantaṃ brāhmaṇaṃ etadavocuṃ – ‘‘kasmā pana bhavaṃ kūṭadanto samaṇassa gotamassa subhāsitaṃ subhāsitato nābbhanumodatī’’ti? ‘‘Nāhaṃ, bho, samaṇassa gotamassa subhāsitaṃ subhāsitato nābbhanumodāmi. Muddhāpi tassa vipateyya, yo samaṇassa gotamassa subhāsitaṃ subhāsitato nābbhanumodeyya. Api ca me, bho, evaṃ hoti – samaṇo gotamo na evamāha – ‘evaṃ me suta’nti vā ‘evaṃ arahati bhavitu’nti vā; api ca samaṇo gotamo – ‘evaṃ tadā āsi, itthaṃ tadā āsi’ tveva bhāsati. Tassa mayhaṃ bho evaṃ hoti – ‘addhā samaṇo gotamo tena samayena rājā vā ahosi mahāvijito yaññassāmi purohito vā brāhmaṇo tassa yaññassa yājetā’ti. Abhijānāti pana bhavaṃ gotamo evarūpaṃ yaññaṃ yajitvā vā yājetvā vā kāyassa bhedā paraṃ maraṇā sugatiṃ saggaṃ lokaṃ upapajjitāti’’? ‘‘Abhijānāmahaṃ, brāhmaṇa, evarūpaṃ yaññaṃ yajitvā vā yājetvā vā kāyassa bhedā paraṃ maraṇā sugatiṃ saggaṃ lokaṃ upapajjitā, ahaṃ tena samayena purohito brāhmaṇo ahosiṃ tassa yaññassa yājetā’’ti.

\subsubsection{Niccadānaanukulayaññaṃ}

\paragraph{349.} ‘‘Atthi pana, bho gotama, añño yañño imāya tividhāya yaññasampadāya\footnote{tividhayaññasampadāya (ka.)} soḷasaparikkhārāya appaṭṭhataro\footnote{appatthataro (syā. kaṃ.)} ca appasamārambhataro\footnote{appasamārabbhataro (sī. pī. ka.)} ca mahapphalataro ca mahānisaṃsataro cā’’ti? ‘‘Atthi kho, brāhmaṇa, añño yañño imāya tividhāya yaññasampadāya soḷasaparikkhārāya appaṭṭhataro ca appasamārambhataro ca mahapphalataro ca mahānisaṃsataro cā’’ti. ‘‘Katamo pana so, bho gotama, yañño imāya tividhāya yaññasampadāya soḷasaparikkhārāya appaṭṭhataro ca appasamārambhataro ca mahapphalataro ca mahānisaṃsataro cā’’ti? ‘‘Yāni kho pana tāni, brāhmaṇa, niccadānāni anukulayaññāni sīlavante pabbajite uddissa diyyanti; ayaṃ kho, brāhmaṇa, yañño imāya tividhāya yaññasampadāya soḷasaparikkhārāya appaṭṭhataro ca appasamārambhataro ca mahapphalataro ca mahānisaṃsataro cā’’ti. ‘‘Ko nu kho, bho gotama, hetu ko paccayo, yena taṃ niccadānaṃ anukulayaññaṃ imāya tividhāya yaññasampadāya soḷasaparikkhārāya appaṭṭhatarañca appasamārambhatarañca mahapphalatarañca mahānisaṃsatarañcā’’ti? ‘‘Na kho, brāhmaṇa, evarūpaṃ yaññaṃ upasaṅkamanti arahanto vā arahattamaggaṃ vā samāpannā. Taṃ kissa hetu? Dissanti hettha, brāhmaṇa, daṇḍappahārāpi galaggahāpi, tasmā evarūpaṃ yaññaṃ na upasaṅkamanti arahanto vā arahattamaggaṃ vā samāpannā. Yāni kho pana tāni, brāhmaṇa, niccadānāni anukulayaññāni sīlavante pabbajite uddissa diyyanti; evarūpaṃ kho, brāhmaṇa, yaññaṃ upasaṅkamanti arahanto vā arahattamaggaṃ vā samāpannā. Taṃ kissa hetu? Na hettha, brāhmaṇa, dissanti daṇḍappahārāpi galaggahāpi, tasmā evarūpaṃ yaññaṃ upasaṅkamanti arahanto vā arahattamaggaṃ vā samāpannā. Ayaṃ kho, brāhmaṇa, hetu ayaṃ paccayo, yena taṃ niccadānaṃ anukulayaññaṃ imāya tividhāya yaññasampadāya soḷasaparikkhārāya appaṭṭhatarañca appasamārambhatarañca mahapphalatarañca mahānisaṃsatarañcā’’ti.

\paragraph{350.} ‘‘Atthi pana, bho gotama, añño yañño imāya ca tividhāya yaññasampadāya soḷasaparikkhārāya iminā ca niccadānena anukulayaññena appaṭṭhataro ca appasamārambhataro ca mahapphalataro ca mahānisaṃsataro cā’’ti? ‘‘Atthi kho, brāhmaṇa, añño yañño imāya ca tividhāya yaññasampadāya soḷasaparikkhārāya iminā ca niccadānena anukulayaññena appaṭṭhataro ca appasamārambhataro ca mahapphalataro ca mahānisaṃsataro cā’’ti. ‘‘Katamo pana so, bho gotama, yañño imāya ca tividhāya yaññasampadāya soḷasaparikkhārāya iminā ca niccadānena anukulayaññena appaṭṭhataro ca appasamārambhataro ca mahapphalataro ca mahānisaṃsataro cā’’ti? ‘‘Yo kho, brāhmaṇa, cātuddisaṃ saṅghaṃ uddissa vihāraṃ karoti, ayaṃ kho, brāhmaṇa, yañño imāya ca tividhāya yaññasampadāya soḷasaparikkhārāya iminā ca niccadānena anukulayaññena appaṭṭhataro ca appasamārambhataro ca mahapphalataro ca mahānisaṃsataro cā’’ti.

\paragraph{351.} ‘‘Atthi pana, bho gotama, añño yañño imāya ca tividhāya yaññasampadāya soḷasaparikkhārāya iminā ca niccadānena anukulayaññena iminā ca vihāradānena appaṭṭhataro ca appasamārambhataro ca mahapphalataro ca mahānisaṃsataro cā’’ti? ‘‘Atthi kho, brāhmaṇa, añño yañño imāya ca tividhāya yaññasampadāya soḷasaparikkhārāya iminā ca niccadānena anukulayaññena iminā ca vihāradānena appaṭṭhataro ca appasamārambhataro ca mahapphalataro ca mahānisaṃsataro cā’’ti. ‘‘Katamo pana so, bho gotama, yañño imāya ca tividhāya yaññasampadāya soḷasaparikkhārāya iminā ca niccadānena anukulayaññena iminā ca vihāradānena appaṭṭhataro ca appasamārambhataro ca mahapphalataro ca mahānisaṃsataro cā’’ti? ‘‘Yo kho, brāhmaṇa, pasannacitto buddhaṃ saraṇaṃ gacchati, dhammaṃ saraṇaṃ gacchati, saṅghaṃ saraṇaṃ gacchati; ayaṃ kho, brāhmaṇa, yañño imāya ca tividhāya yaññasampadāya soḷasaparikkhārāya iminā ca niccadānena anukulayaññena iminā ca vihāradānena appaṭṭhataro ca appasamārambhataro ca mahapphalataro ca mahānisaṃsataro cā’’ti.

\paragraph{352.} ‘‘Atthi pana, bho gotama, añño yañño imāya ca tividhāya yaññasampadāya soḷasaparikkhārāya iminā ca niccadānena anukulayaññena iminā ca vihāradānena imehi ca saraṇagamanehi appaṭṭhataro ca appasamārambhataro ca mahapphalataro ca mahānisaṃsataro cā’’ti? ‘‘Atthi kho, brāhmaṇa, añño yañño imāya ca tividhāya yaññasampadāya soḷasaparikkhārāya iminā ca niccadānena anukulayaññena iminā ca vihāradānena imehi ca saraṇagamanehi appaṭṭhataro ca appasamārambhataro ca mahapphalataro ca mahānisaṃsataro cā’’ti. ‘‘Katamo pana so, bho gotama, yañño imāya ca tividhāya yaññasampadāya soḷasaparikkhārāya iminā ca niccadānena anukulayaññena iminā ca vihāradānena imehi ca saraṇagamanehi appaṭṭhataro ca appasamārambhataro ca mahapphalataro ca mahānisaṃsataro cā’’ti? ‘‘Yo kho, brāhmaṇa, pasannacitto sikkhāpadāni samādiyati – pāṇātipātā veramaṇiṃ, adinnādānā veramaṇiṃ, kāmesumicchācārā veramaṇiṃ, musāvādā veramaṇiṃ, surāmerayamajjapamādaṭṭhānā veramaṇiṃ. Ayaṃ kho, brāhmaṇa, yañño imāya ca tividhāya yaññasampadāya soḷasaparikkhārāya iminā ca niccadānena anukulayaññena iminā ca vihāradānena imehi ca saraṇagamanehi appaṭṭhataro ca appasamārambhataro ca mahapphalataro ca mahānisaṃsataro cā’’ti.

\paragraph{353.} ‘‘Atthi pana, bho gotama, añño yañño imāya ca tividhāya yaññasampadāya soḷasaparikkhārāya iminā ca niccadānena anukulayaññena iminā ca vihāradānena imehi ca saraṇagamanehi imehi ca sikkhāpadehi appaṭṭhataro ca appasamārambhataro ca mahapphalataro ca mahānisaṃsataro cā’’ti? ‘‘Atthi kho, brāhmaṇa, añño yañño imāya ca tividhāya yaññasampadāya soḷasaparikkhārāya iminā ca niccadānena anukulayaññena iminā ca vihāradānena imehi ca saraṇagamanehi imehi ca sikkhāpadehi appaṭṭhataro ca appasamārambhataro ca mahapphalataro ca mahānisaṃsataro cā’’ti. ‘‘Katamo pana so, bho gotama, yañño imāya ca tividhāya yaññasampadāya soḷasaparikkhārāya iminā ca niccadānena anukulayaññena iminā ca vihāradānena imehi ca saraṇagamanehi imehi ca sikkhāpadehi appaṭṭhataro ca appasamārambhataro ca mahapphalataro ca mahānisaṃsataro cā’’ti? ‘‘Idha, brāhmaṇa, tathāgato loke uppajjati arahaṃ sammāsambuddho…pe… (yathā 190-212 anucchedesu, evaṃ vitthāretabbaṃ). Evaṃ kho, brāhmaṇa, bhikkhu sīlasampanno hoti…pe… paṭhamaṃ jhānaṃ upasampajja viharati. Ayaṃ kho, brāhmaṇa, yañño purimehi yaññehi appaṭṭhataro ca appasamārambhataro ca mahapphalataro ca mahānisaṃsataro ca…pe… dutiyaṃ jhānaṃ…pe… tatiyaṃ jhānaṃ…pe… catutthaṃ jhānaṃ upasampajja viharati. Ayampi kho, brāhmaṇa, yañño purimehi yaññehi appaṭṭhataro ca appasamārambhataro ca mahapphalataro ca mahānisaṃsataro cāti. Ñāṇadassanāya cittaṃ abhinīharati abhininnāmeti…pe… ayampi kho, brāhmaṇa, yañño purimehi yaññehi appaṭṭhataro ca appasamārambhataro ca mahapphalataro ca mahānisaṃsataro ca…pe… nāparaṃ itthattāyāti pajānāti. Ayampi kho, brāhmaṇa, yañño purimehi yaññehi appaṭṭhataro ca appasamārambhataro ca mahapphalataro ca mahānisaṃsataro ca. Imāya ca, brāhmaṇa, yaññasampadāya aññā yaññasampadā uttaritarā vā paṇītatarā vā natthī’’ti.

\subsubsection{Kūṭadantaupāsakattapaṭivedanā}

\paragraph{354.} Evaṃ vutte, kūṭadanto brāhmaṇo bhagavantaṃ etadavoca – ‘‘abhikkantaṃ, bho gotama, abhikkantaṃ, bho gotama! Seyyathāpi bho gotama, nikkujjitaṃ vā ukkujjeyya, paṭicchannaṃ vā vivareyya, mūḷhassa vā maggaṃ ācikkheyya, andhakāre vā telapajjotaṃ dhāreyya ‘cakkhumanto rūpāni dakkhantī’ti; evamevaṃ bhotā gotamena anekapariyāyena dhammo pakāsito. Esāhaṃ bhavantaṃ gotamaṃ saraṇaṃ gacchāmi dhammañca bhikkhusaṅghañca. Upāsakaṃ maṃ bhavaṃ gotamo dhāretu ajjatagge pāṇupetaṃ saraṇaṃ gataṃ. Esāhaṃ bho gotama satta ca usabhasatāni satta ca vacchatarasatāni satta ca vacchatarīsatāni satta ca ajasatāni satta ca urabbhasatāni muñcāmi, jīvitaṃ demi, haritāni ceva tiṇāni khādantu, sītāni ca pānīyāni pivantu, sīto ca nesaṃ vāto upavāyatū’’ti.

\subsubsection{Sotāpattiphalasacchikiriyā}

\paragraph{355.} Atha kho bhagavā kūṭadantassa brāhmaṇassa anupubbiṃ kathaṃ kathesi, seyyathidaṃ, dānakathaṃ sīlakathaṃ saggakathaṃ; kāmānaṃ ādīnavaṃ okāraṃ saṃkilesaṃ nekkhamme ānisaṃsaṃ pakāsesi. Yadā bhagavā aññāsi kūṭadantaṃ brāhmaṇaṃ kallacittaṃ muducittaṃ vinīvaraṇacittaṃ udaggacittaṃ pasannacittaṃ, atha yā buddhānaṃ sāmukkaṃsikā dhammadesanā, taṃ pakāsesi – dukkhaṃ samudayaṃ nirodhaṃ maggaṃ. Seyyathāpi nāma suddhaṃ vatthaṃ apagatakāḷakaṃ sammadeva rajanaṃ paṭiggaṇheyya, evameva kūṭadantassa brāhmaṇassa tasmiññeva āsane virajaṃ vītamalaṃ dhammacakkhuṃ udapādi – ‘‘yaṃ kiñci samudayadhammaṃ, sabbaṃ taṃ nirodhadhamma’’nti.

\paragraph{356.} Atha kho kūṭadanto brāhmaṇo diṭṭhadhammo pattadhammo viditadhammo pariyogāḷhadhammo tiṇṇavicikiccho vigatakathaṃkatho vesārajjappatto aparappaccayo satthusāsane bhagavantaṃ etadavoca – ‘‘adhivāsetu me bhavaṃ gotamo svātanāya bhattaṃ saddhiṃ bhikkhusaṅghenā’’ti. Adhivāsesi bhagavā tuṇhībhāvena.

\paragraph{357.} Atha kho kūṭadanto brāhmaṇo bhagavato adhivāsanaṃ viditvā uṭṭhāyāsanā bhagavantaṃ abhivādetvā padakkhiṇaṃ katvā pakkāmi. Atha kho kūṭadanto brāhmaṇo tassā rattiyā accayena sake yaññavāṭe paṇītaṃ khādanīyaṃ bhojanīyaṃ paṭiyādāpetvā bhagavato kālaṃ ārocāpesi – ‘‘kālo, bho gotama; niṭṭhitaṃ bhatta’’nti.

\paragraph{358.} Atha kho bhagavā pubbaṇhasamayaṃ nivāsetvā pattacīvaramādāya saddhiṃ bhikkhusaṅghena yena kūṭadantassa brāhmaṇassa yaññavāṭo tenupasaṅkami; upasaṅkamitvā paññatte āsane nisīdi. Atha kho kūṭadanto brāhmaṇo buddhappamukhaṃ bhikkhusaṅghaṃ paṇītena khādanīyena bhojanīyena sahatthā santappesi sampavāresi. Atha kho kūṭadanto brāhmaṇo bhagavantaṃ bhuttāviṃ onītapattapāṇiṃ aññataraṃ nīcaṃ āsanaṃ gahetvā ekamantaṃ nisīdi. Ekamantaṃ nisinnaṃ kho kūṭadantaṃ brāhmaṇaṃ bhagavā dhammiyā kathāya sandassetvā samādapetvā samuttejetvā sampahaṃsetvā uṭṭhāyāsanā pakkāmīti.

\xsectionEnd{Kūṭadantasuttaṃ niṭṭhitaṃ pañcamaṃ.}


\clearpage
\section{Mahālisuttaṃ}

\subsubsection{Brāhmaṇadūtavatthu}

\paragraph{359.} Evaṃ me sutaṃ – uekaṃ samayaṃ bhagavā vesāliyaṃ viharati mahāvane kūṭāgārasālāyaṃ. Tena kho pana samayena sambahulā kosalakā ca brāhmaṇadūtā māgadhakā ca brāhmaṇadūtā vesāliyaṃ paṭivasanti kenacideva karaṇīyena. Assosuṃ kho te kosalakā ca brāhmaṇadūtā māgadhakā ca brāhmaṇadūtā – ‘‘samaṇo khalu, bho, gotamo sakyaputto sakyakulā pabbajito vesāliyaṃ viharati mahāvane kūṭāgārasālāyaṃ. Taṃ kho pana bhavantaṃ gotamaṃ evaṃ kalyāṇo kittisaddo abbhuggato – ‘itipi so bhagavā arahaṃ sammāsambuddho vijjācaraṇasampanno sugato lokavidū anuttaro purisadammasārathi satthā devamanussānaṃ buddho bhagavā’. So imaṃ lokaṃ sadevakaṃ samārakaṃ sabrahmakaṃ sassamaṇabrāhmaṇiṃ pajaṃ sadevamanussaṃ sayaṃ abhiññā sacchikatvā pavedeti. So dhammaṃ deseti ādikalyāṇaṃ majjhekalyāṇaṃ pariyosānakalyāṇaṃ sātthaṃ sabyañjanaṃ kevalaparipuṇṇaṃ parisuddhaṃ brahmacariyaṃ pakāseti. Sādhu kho pana tathārūpānaṃ arahataṃ dassanaṃ hotī’’ti.

\paragraph{360.} Atha kho te kosalakā ca brāhmaṇadūtā māgadhakā ca brāhmaṇadūtā yena mahāvanaṃ kūṭāgārasālā tenupasaṅkamiṃsu. Tena kho pana samayena āyasmā nāgito bhagavato upaṭṭhāko hoti. Atha kho te kosalakā ca brāhmaṇadūtā māgadhakā ca brāhmaṇadūtā yenāyasmā nāgito tenupasaṅkamiṃsu. Upasaṅkamitvā āyasmantaṃ nāgitaṃ etadavocuṃ – ‘‘kahaṃ nu kho, bho nāgita, etarahi so bhavaṃ gotamo viharati? Dassanakāmā hi mayaṃ taṃ bhavantaṃ gotama’’nti. ‘‘Akālo kho, āvuso, bhagavantaṃ dassanāya, paṭisallīno bhagavā’’ti. Atha kho te kosalakā ca brāhmaṇadūtā māgadhakā ca brāhmaṇadūtā tattheva ekamantaṃ nisīdiṃsu – ‘‘disvāva mayaṃ taṃ bhavantaṃ gotamaṃ gamissāmā’’ti.

\subsubsection{Oṭṭhaddhalicchavīvatthu}

\paragraph{361.} Oṭṭhaddhopi licchavī mahatiyā licchavīparisāya saddhiṃ yena mahāvanaṃ kūṭāgārasālā yenāyasmā nāgito tenupasaṅkami; upasaṅkamitvā āyasmantaṃ nāgitaṃ abhivādetvā ekamantaṃ aṭṭhāsi. Ekamantaṃ ṭhito kho oṭṭhaddhopi licchavī āyasmantaṃ nāgitaṃ etadavoca – ‘‘kahaṃ nu kho, bhante nāgita, etarahi so bhagavā viharati arahaṃ sammāsambuddho, dassanakāmā hi mayaṃ taṃ bhagavantaṃ arahantaṃ sammāsambuddha’’nti. ‘‘Akālo kho, mahāli, bhagavantaṃ dassanāya, paṭisallīno bhagavā’’ti. Oṭṭhaddhopi licchavī tattheva ekamantaṃ nisīdi – ‘‘disvāva ahaṃ taṃ bhagavantaṃ gamissāmi arahantaṃ sammāsambuddha’’nti.

\paragraph{362.} Atha kho sīho samaṇuddeso yenāyasmā nāgito tenupasaṅkami; upasaṅkamitvā āyasmantaṃ nāgitaṃ abhivādetvā ekamantaṃ aṭṭhāsi. Ekamantaṃ ṭhito kho sīho samaṇuddeso āyasmantaṃ nāgitaṃ etadavoca – ‘‘ete, bhante kassapa, sambahulā kosalakā ca brāhmaṇadūtā māgadhakā ca brāhmaṇadūtā idhūpasaṅkantā bhagavantaṃ dassanāya; oṭṭhaddhopi licchavī mahatiyā licchavīparisāya saddhiṃ idhūpasaṅkanto bhagavantaṃ dassanāya, sādhu, bhante kassapa, labhataṃ esā janatā bhagavantaṃ dassanāyā’’ti. ‘‘Tena hi, sīha, tvaññeva bhagavato ārocehī’’ti. ‘‘Evaṃ, bhante’’ti kho sīho samaṇuddeso āyasmato nāgitassa paṭissutvā yena bhagavā tenupasaṅkami; upasaṅkamitvā bhagavantaṃ abhivādetvā ekamantaṃ aṭṭhāsi. Ekamantaṃ ṭhito kho sīho samaṇuddeso bhagavantaṃ etadavoca – ‘‘ete, bhante, sambahulā kosalakā ca brāhmaṇadūtā māgadhakā ca brāhmaṇadūtā idhūpasaṅkantā bhagavantaṃ dassanāya, oṭṭhaddhopi licchavī mahatiyā licchavīparisāya saddhiṃ idhūpasaṅkanto bhagavantaṃ dassanāya. Sādhu, bhante, labhataṃ esā janatā bhagavantaṃ dassanāyā’’ti. ‘‘Tena hi, sīha, vihārapacchāyāyaṃ āsanaṃ paññapehī’’ti. ‘‘Evaṃ, bhante’’ti kho sīho samaṇuddeso bhagavato paṭissutvā vihārapacchāyāyaṃ āsanaṃ paññapesi.

\paragraph{363.} Atha kho bhagavā vihārā nikkhamma vihārapacchāyāyaṃ paññatte āsane nisīdi. Atha kho te kosalakā ca brāhmaṇadūtā māgadhakā ca brāhmaṇadūtā yena bhagavā tenupasaṅkamiṃsu; upasaṅkamitvā bhagavatā saddhiṃ sammodiṃsu. Sammodanīyaṃ kathaṃ sāraṇīyaṃ vītisāretvā ekamantaṃ nisīdiṃsu. Oṭṭhaddhopi licchavī mahatiyā licchavīparisāya saddhiṃ yena bhagavā tenupasaṅkami; upasaṅkamitvā bhagavantaṃ abhivādetvā ekamantaṃ nisīdi.

\paragraph{364.} Ekamantaṃ nisinno kho oṭṭhaddho licchavī bhagavantaṃ etadavoca – ‘‘purimāni, bhante, divasāni purimatarāni sunakkhatto licchaviputto yenāhaṃ tenupasaṅkami; upasaṅkamitvā maṃ etadavoca – ‘yadagge ahaṃ, mahāli, bhagavantaṃ upanissāya viharāmi, na ciraṃ tīṇi vassāni, dibbāni hi kho rūpāni passāmi piyarūpāni kāmūpasaṃhitāni rajanīyāni, no ca kho dibbāni saddāni suṇāmi piyarūpāni kāmūpasaṃhitāni rajanīyānī’ti. Santāneva nu kho, bhante, sunakkhatto licchaviputto dibbāni saddāni nāssosi piyarūpāni kāmūpasaṃhitāni rajanīyāni, udāhu asantānī’’ti?

\subsubsection{Ekaṃsabhāvitasamādhi}

\paragraph{365.} ‘‘Santāneva kho, mahāli, sunakkhatto licchaviputto dibbāni saddāni nāssosi piyarūpāni kāmūpasaṃhitāni rajanīyāni, no asantānī’’ti. ‘‘Ko nu kho, bhante, hetu, ko paccayo, yena santāneva sunakkhatto licchaviputto dibbāni saddāni nāssosi piyarūpāni kāmūpasaṃhitāni rajanīyāni, no asantānī’’ti?

\paragraph{366.} ‘‘Idha, mahāli, bhikkhuno puratthimāya disāya ekaṃsabhāvito samādhi hoti dibbānaṃ rūpānaṃ dassanāya piyarūpānaṃ kāmūpasaṃhitānaṃ rajanīyānaṃ, no ca kho dibbānaṃ saddānaṃ savanāya piyarūpānaṃ kāmūpasaṃhitānaṃ rajanīyānaṃ. So puratthimāya disāya ekaṃsabhāvite samādhimhi dibbānaṃ rūpānaṃ dassanāya piyarūpānaṃ kāmūpasaṃhitānaṃ rajanīyānaṃ, no ca kho dibbānaṃ saddānaṃ savanāya piyarūpānaṃ kāmūpasaṃhitānaṃ rajanīyānaṃ. Puratthimāya disāya dibbāni rūpāni passati piyarūpāni kāmūpasaṃhitāni rajanīyāni, no ca kho dibbāni saddāni suṇāti piyarūpāni kāmūpasaṃhitāni rajanīyāni. Taṃ kissa hetu? Evañhetaṃ, mahāli, hoti bhikkhuno puratthimāya disāya ekaṃsabhāvite samādhimhi dibbānaṃ rūpānaṃ dassanāya piyarūpānaṃ kāmūpasaṃhitānaṃ rajanīyānaṃ, no ca kho dibbānaṃ saddānaṃ savanāya piyarūpānaṃ kāmūpasaṃhitānaṃ rajanīyānaṃ.

\paragraph{367.} ‘‘Puna caparaṃ, mahāli, bhikkhuno dakkhiṇāya disāya…pe… pacchimāya disāya … uttarāya disāya… uddhamadho tiriyaṃ ekaṃsabhāvito samādhi hoti dibbānaṃ rūpānaṃ dassanāya piyarūpānaṃ kāmūpasaṃhitānaṃ rajanīyānaṃ, no ca kho dibbānaṃ saddānaṃ savanāya piyarūpānaṃ kāmūpasaṃhitānaṃ rajanīyānaṃ. So uddhamadho tiriyaṃ ekaṃsabhāvite samādhimhi dibbānaṃ rūpānaṃ dassanāya piyarūpānaṃ kāmūpasaṃhitānaṃ rajanīyānaṃ, no ca kho dibbānaṃ saddānaṃ savanāya piyarūpānaṃ kāmūpasaṃhitānaṃ rajanīyānaṃ. Uddhamadho tiriyaṃ dibbāni rūpāni passati piyarūpāni kāmūpasaṃhitāni rajanīyāni, no ca kho dibbāni saddāni suṇāti piyarūpāni kāmūpasaṃhitāni rajanīyāni. Taṃ kissa hetu? Evañhetaṃ, mahāli, hoti bhikkhuno uddhamadho tiriyaṃ ekaṃsabhāvite samādhimhi dibbānaṃ rūpānaṃ dassanāya piyarūpānaṃ kāmūpasaṃhitānaṃ rajanīyānaṃ, no ca kho dibbānaṃ saddānaṃ savanāya piyarūpānaṃ kāmūpasaṃhitānaṃ rajanīyānaṃ.

\paragraph{368.} ‘‘Idha, mahāli, bhikkhuno puratthimāya disāya ekaṃsabhāvito samādhi hoti dibbānaṃ saddānaṃ savanāya piyarūpānaṃ kāmūpasaṃhitānaṃ rajanīyānaṃ, no ca kho dibbānaṃ rūpānaṃ dassanāya piyarūpānaṃ kāmūpasaṃhitānaṃ rajanīyānaṃ. So puratthimāya disāya ekaṃsabhāvite samādhimhi dibbānaṃ saddānaṃ savanāya piyarūpānaṃ kāmūpasaṃhitānaṃ rajanīyānaṃ, no ca kho dibbānaṃ rūpānaṃ dassanāya piyarūpānaṃ kāmūpasaṃhitānaṃ rajanīyānaṃ. Puratthimāya disāya dibbāni saddāni suṇāti piyarūpāni kāmūpasaṃhitāni rajanīyāni, no ca kho dibbāni rūpāni passati piyarūpāni kāmūpasaṃhitāni rajanīyāni. Taṃ kissa hetu? Evañhetaṃ, mahāli, hoti bhikkhuno puratthimāya disāya ekaṃsabhāvite samādhimhi dibbānaṃ saddānaṃ savanāya piyarūpānaṃ kāmūpasaṃhitānaṃ rajanīyānaṃ, no ca kho dibbānaṃ rūpānaṃ dassanāya piyarūpānaṃ kāmūpasaṃhitānaṃ rajanīyānaṃ.

\paragraph{369.} ‘‘Puna caparaṃ, mahāli, bhikkhuno dakkhiṇāya disāya…pe… pacchimāya disāya… uttarāya disāya… uddhamadho tiriyaṃ ekaṃsabhāvito samādhi hoti dibbānaṃ saddānaṃ savanāya piyarūpānaṃ kāmūpasaṃhitānaṃ rajanīyānaṃ, no ca kho dibbānaṃ rūpānaṃ dassanāya piyarūpānaṃ kāmūpasaṃhitānaṃ rajanīyānaṃ. So uddhamadho tiriyaṃ ekaṃsabhāvite samādhimhi dibbānaṃ saddānaṃ savanāya piyarūpānaṃ kāmūpasaṃhitānaṃ rajanīyānaṃ, no ca kho dibbānaṃ rūpānaṃ dassanāya piyarūpānaṃ kāmūpasaṃhitānaṃ rajanīyānaṃ. Uddhamadho tiriyaṃ dibbāni saddāni suṇāti piyarūpāni kāmūpasaṃhitāni rajanīyāni, no ca kho dibbāni rūpāni passati piyarūpāni kāmūpasaṃhitāni rajanīyāni. Taṃ kissa hetu? Evañhetaṃ, mahāli, hoti bhikkhuno uddhamadho tiriyaṃ ekaṃsabhāvite samādhimhi dibbānaṃ saddānaṃ savanāya piyarūpānaṃ kāmūpasaṃhitānaṃ rajanīyānaṃ, no ca kho dibbānaṃ rūpānaṃ dassanāya piyarūpānaṃ kāmūpasaṃhitānaṃ rajanīyānaṃ.

\paragraph{370.} ‘‘Idha, mahāli, bhikkhuno puratthimāya disāya ubhayaṃsabhāvito samādhi hoti dibbānañca rūpānaṃ dassanāya piyarūpānaṃ kāmūpasaṃhitānaṃ rajanīyānaṃ dibbānañca saddānaṃ savanāya piyarūpānaṃ kāmūpasaṃhitānaṃ rajanīyānaṃ. So puratthimāya disāya ubhayaṃsabhāvite samādhimhi dibbānañca rūpānaṃ dassanāya piyarūpānaṃ kāmūpasaṃhitānaṃ rajanīyānaṃ, dibbānañca saddānaṃ savanāya piyarūpānaṃ kāmūpasaṃhitānaṃ rajanīyānaṃ. Puratthimāya disāya dibbāni ca rūpāni passati piyarūpāni kāmūpasaṃhitāni rajanīyāni, dibbāni ca saddāni suṇāti piyarūpāni kāmūpasaṃhitāni rajanīyāni. Taṃ kissa hetu? Evañhetaṃ, mahāli, hoti bhikkhuno puratthimāya disāya ubhayaṃsabhāvite samādhimhi dibbānañca rūpānaṃ dassanāya piyarūpānaṃ kāmūpasaṃhitānaṃ rajanīyānaṃ dibbānañca saddānaṃ savanāya piyarūpānaṃ kāmūpasaṃhitānaṃ rajanīyānaṃ.

\paragraph{371.} ‘‘Puna caparaṃ, mahāli, bhikkhuno dakkhiṇāya disāya…pe… pacchimāya disāya… uttarāya disāya… uddhamadho tiriyaṃ ubhayaṃsabhāvito samādhi hoti dibbānañca rūpānaṃ dassanāya piyarūpānaṃ kāmūpasaṃhitānaṃ rajanīyānaṃ, dibbānañca saddānaṃ savanāya piyarūpānaṃ kāmūpasaṃhitānaṃ rajanīyānaṃ. So uddhamadho tiriyaṃ ubhayaṃsabhāvite samādhimhi dibbānañca rūpānaṃ dassanāya piyarūpānaṃ kāmūpasaṃhitānaṃ rajanīyānaṃ dibbānañca saddānaṃ savanāya piyarūpānaṃ kāmūpasaṃhitānaṃ rajanīyānaṃ. Uddhamadho tiriyaṃ dibbāni ca rūpāni passati piyarūpāni kāmūpasaṃhitāni rajanīyāni, dibbāni ca saddāni suṇāti piyarūpāni kāmūpasaṃhitāni rajanīyāni. Taṃ kissa hetu? Evañhetaṃ, mahāli, hoti bhikkhuno uddhamadho tiriyaṃ ubhayaṃsabhāvite samādhimhi dibbānañca rūpānaṃ dassanāya piyarūpānaṃ kāmūpasaṃhitānaṃ rajanīyānaṃ, dibbānañca saddānaṃ savanāya piyarūpānaṃ kāmūpasaṃhitānaṃ rajanīyānaṃ. Ayaṃ kho mahāli, hetu, ayaṃ paccayo, yena santāneva sunakkhatto licchaviputto dibbāni saddāni nāssosi piyarūpāni kāmūpasaṃhitāni rajanīyāni, no asantānī’’ti.

\paragraph{372.} ‘‘Etāsaṃ nūna, bhante, samādhibhāvanānaṃ sacchikiriyāhetu bhikkhū bhagavati brahmacariyaṃ carantī’’ti. ‘‘Na kho, mahāli, etāsaṃ samādhibhāvanānaṃ sacchikiriyāhetu bhikkhū mayi brahmacariyaṃ caranti. Atthi kho, mahāli, aññeva dhammā uttaritarā ca paṇītatarā ca, yesaṃ sacchikiriyāhetu bhikkhū mayi brahmacariyaṃ carantī’’ti.

\subsubsection{Catuariyaphalaṃ}

\paragraph{373.} ‘‘Katame pana te, bhante, dhammā uttaritarā ca paṇītatarā ca, yesaṃ sacchikiriyāhetu bhikkhū bhagavati brahmacariyaṃ carantī’’ti? ‘‘Idha, mahāli, bhikkhu tiṇṇaṃ saṃyojanānaṃ parikkhayā sotāpanno hoti avinipātadhammo niyato sambodhiparāyaṇo. Ayampi kho, mahāli, dhammo uttaritaro ca paṇītataro ca, yassa sacchikiriyāhetu bhikkhū mayi brahmacariyaṃ caranti. ‘‘Puna caparaṃ, mahāli, bhikkhu tiṇṇaṃ saṃyojanānaṃ parikkhayā rāgadosamohānaṃ tanuttā sakadāgāmī hoti, sakideva\footnote{sakiṃdeva (ka.)} imaṃ lokaṃ āgantvā dukkhassantaṃ karoti. Ayampi kho, mahāli, dhammo uttaritaro ca paṇītataro ca, yassa sacchikiriyāhetu bhikkhū mayi brahmacariyaṃ caranti. ‘‘Puna caparaṃ, mahāli, bhikkhu pañcannaṃ orambhāgiyānaṃ saṃyojanānaṃ parikkhayā opapātiko hoti, tattha parinibbāyī, anāvattidhammo tasmā lokā. Ayampi kho, mahāli, dhammo uttaritaro ca paṇītataro ca, yassa sacchikiriyāhetu bhikkhū mayi brahmacariyaṃ caranti. ‘‘Puna caparaṃ, mahāli, bhikkhu āsavānaṃ khayā anāsavaṃ cetovimuttiṃ paññāvimuttiṃ diṭṭheva dhamme sayaṃ abhiññā sacchikatvā upasampajja viharati. Ayampi kho, mahāli, dhammo uttaritaro ca paṇītataro ca, yassa sacchikiriyāhetu bhikkhū mayi brahmacariyaṃ caranti. Ime kho te, mahāli, dhammā uttaritarā ca paṇītatarā ca, yesaṃ sacchikiriyāhetu bhikkhū mayi brahmacariyaṃ carantī’’ti.

\subsubsection{Ariyaaṭṭhaṅgikamaggo}

\paragraph{374.} ‘‘Atthi pana, bhante, maggo atthi paṭipadā etesaṃ dhammānaṃ sacchikiriyāyā’’ti? ‘‘Atthi kho, mahāli, maggo atthi paṭipadā etesaṃ dhammānaṃ sacchikiriyāyā’’ti.

\paragraph{375.} ‘‘Katamo pana, bhante, maggo katamā paṭipadā etesaṃ dhammānaṃ sacchikiriyāyā’’ti? ‘‘Ayameva ariyo aṭṭhaṅgiko maggo. Seyyathidaṃ – sammādiṭṭhi sammāsaṅkappo sammāvācā sammākammanto sammāājīvo sammāvāyāmo sammāsati sammāsamādhi. Ayaṃ kho, mahāli, maggo ayaṃ paṭipadā etesaṃ dhammānaṃ sacchikiriyāya.

\subsubsection{Dvepabbajitavatthu}

\paragraph{376.} ‘‘Ekamidāhaṃ, mahāli, samayaṃ kosambiyaṃ viharāmi ghositārāme. Atha kho dve pabbajitā – muṇḍiyo ca paribbājako jāliyo ca dārupattikantevāsī yenāhaṃ tenupasaṅkamiṃsu. Upasaṅkamitvā mayā saddhiṃ sammodiṃsu. Sammodanīyaṃ kathaṃ sāraṇīyaṃ vītisāretvā ekamantaṃ aṭṭhaṃsu. Ekamantaṃ ṭhitā kho te dve pabbajitā maṃ etadavocuṃ – ‘kiṃ nu kho, āvuso gotama, taṃ jīvaṃ taṃ sarīraṃ, udāhu aññaṃ jīvaṃ aññaṃ sarīra’nti?

\paragraph{377.} ‘‘‘Tena hāvuso, suṇātha sādhukaṃ manasi karotha bhāsissāmī’’ti. ‘Evamāvuso’ti kho te dve pabbajitā mama paccassosuṃ. Ahaṃ etadavocaṃ – idhāvuso tathāgato loke uppajjati arahaṃ sammāsambuddho…pe… (yathā 190-212 anucchedesu evaṃ vitthāretabbaṃ). Evaṃ kho, āvuso, bhikkhu sīlasampanno hoti…pe… paṭhamaṃ jhānaṃ upasampajja viharati. Yo kho, āvuso, bhikkhu evaṃ jānāti evaṃ passati, kallaṃ nu kho tassetaṃ vacanāya – ‘taṃ jīvaṃ taṃ sarīra’nti vā ‘aññaṃ jīvaṃ aññaṃ sarīra’nti vāti? Yo so, āvuso, bhikkhu evaṃ jānāti evaṃ passati, kallaṃ tassetaṃ vacanāya – ‘taṃ jīvaṃ taṃ sarīra’nti vā, ‘aññaṃ jīvaṃ aññaṃ sarīra’nti vāti. Ahaṃ kho panetaṃ, āvuso, evaṃ jānāmi evaṃ passāmi. Atha ca panāhaṃ na vadāmi – ‘taṃ jīvaṃ taṃ sarīra’nti vā ‘aññaṃ jīvaṃ aññaṃ sarīra’nti vā…pe… dutiyaṃ jhānaṃ…pe… tatiyaṃ jhānaṃ…pe… catutthaṃ jhānaṃ upasampajja viharati. Yo kho, āvuso, bhikkhu evaṃ jānāti evaṃ passati, kallaṃ nu kho tassetaṃ vacanāya – ‘taṃ jīvaṃ taṃ sarīra’nti vā ‘aññaṃ jīvaṃ aññaṃ sarīra’nti vāti? Yo so, āvuso, bhikkhu evaṃ jānāti evaṃ passati, kallaṃ tassetaṃ vacanāya – ‘taṃ jīvaṃ taṃ sarīra’nti vā ‘aññaṃ jīvaṃ aññaṃ sarīra’nti vāti. Ahaṃ kho panetaṃ, āvuso, evaṃ jānāmi evaṃ passāmi. Atha ca panāhaṃ na vadāmi – ‘taṃ jīvaṃ taṃ sarīra’nti vā ‘aññaṃ jīvaṃ aññaṃ sarīra’nti vā…pe… ñāṇadassanāya cittaṃ abhinīharati abhininnāmeti…pe… yo kho, āvuso, bhikkhu evaṃ jānāti evaṃ passati, kallaṃ nu kho tassetaṃ vacanāya – ‘taṃ jīvaṃ taṃ sarīra’nti vā ‘aññaṃ jīvaṃ aññaṃ sarīra’nti vāti? Yo so, āvuso, bhikkhu evaṃ jānāti evaṃ passati, kallaṃ\footnote{na kallaṃ (sī. syā. kaṃ. ka.)} tassetaṃ vacanāya – ‘taṃ jīvaṃ taṃ sarīra’’nti vā ‘aññaṃ jīvaṃ aññaṃ sarīra’nti vāti. Ahaṃ kho panetaṃ, āvuso, evaṃ jānāmi evaṃ passāmi. Atha ca panāhaṃ na vadāmi – ‘taṃ jīvaṃ taṃ sarīra’nti vā ‘aññaṃ jīvaṃ aññaṃ sarīra’nti vā…pe… nāparaṃ itthattāyāti pajānāti. Yo kho, āvuso, bhikkhu evaṃ jānāti evaṃ passati, kallaṃ nu kho tassetaṃ vacanāya – ‘taṃ jīvaṃ taṃ sarīra’nti vā ‘aññaṃ jīvaṃ aññaṃ sarīra’nti vāti? Yo so, āvuso, bhikkhu evaṃ jānāti evaṃ passati na kallaṃ tassetaṃ vacanāya – ‘taṃ jīvaṃ taṃ sarīra’nti vā ‘aññaṃ jīvaṃ aññaṃ sarīra’nti vāti. Ahaṃ kho panetaṃ, āvuso, evaṃ jānāmi evaṃ passāmi. Atha ca panāhaṃ na vadāmi – ‘taṃ jīvaṃ taṃ sarīra’nti vā ‘aññaṃ jīvaṃ aññaṃ sarīra’nti vā’’ti. Idamavoca bhagavā. Attamano oṭṭhaddho licchavī bhagavato bhāsitaṃ abhinandīti.

\xsectionEnd{Mahālisuttaṃ niṭṭhitaṃ chaṭṭhaṃ.}


\clearpage
\section{Jāliyasuttaṃ}

\subsubsection{Dvepabbajitavatthu}

\paragraph{378.} Evaṃ me sutaṃ – ekaṃ samayaṃ bhagavā kosambiyaṃ viharati ghositārāme. Tena kho pana samayena dve pabbajitā – muṇḍiyo ca paribbājako jāliyo ca dārupattikantevāsī yena bhagavā tenupasaṅkamiṃsu; upasaṅkamitvā bhagavatā saddhiṃ sammodiṃsu. Sammodanīyaṃ kathaṃ sāraṇīyaṃ vītisāretvā ekamantaṃ aṭṭhaṃsu. Ekamantaṃ ṭhitā kho te dve pabbajitā bhagavantaṃ etadavocuṃ – ‘‘kiṃ nu kho, āvuso gotama, taṃ jīvaṃ taṃ sarīraṃ, udāhu aññaṃ jīvaṃ aññaṃ sarīra’’nti?

\paragraph{379.} ‘‘Tena hāvuso, suṇātha sādhukaṃ manasi karotha; bhāsissāmī’’ti. ‘‘Evamāvuso’’ti kho te dve pabbajitā bhagavato paccassosuṃ. Bhagavā etadavoca – ‘‘idhāvuso, tathāgato loke uppajjati arahaṃ, sammāsambuddho…pe… (yathā 190-212 anucchedesu evaṃ vitthāretabbaṃ). Evaṃ kho, āvuso, bhikkhu sīlasampanno hoti…pe… paṭhamaṃ jhānaṃ upasampajja viharati. Yo kho, āvuso, bhikkhu evaṃ jānāti evaṃ passati, kallaṃ nu kho tassetaṃ vacanāya – ‘taṃ jīvaṃ taṃ sarīra’nti vā ‘aññaṃ jīvaṃ aññaṃ sarīra’nti vāti. Yo so, āvuso, bhikkhu evaṃ jānāti evaṃ passati, kallaṃ tassetaṃ vacanāya – ‘taṃ jīvaṃ taṃ sarīra’nti vā ‘aññaṃ jīvaṃ aññaṃ sarīra’nti vāti. Ahaṃ kho panetaṃ, āvuso, evaṃ jānāmi evaṃ passāmi. Atha ca panāhaṃ na vadāmi – ‘taṃ jīvaṃ taṃ sarīra’nti vā ‘aññaṃ jīvaṃ aññaṃ sarīra’nti vā…pe… dutiyaṃ jhānaṃ…pe… tatiyaṃ jhānaṃ…pe… catutthaṃ jhānaṃ upasampajja viharati. Yo kho, āvuso, bhikkhu evaṃ jānāti evaṃ passati, kallaṃ nu kho tassetaṃ vacanāya – ‘taṃ jīvaṃ taṃ sarīra’nti vā ‘aññaṃ jīvaṃ aññaṃ sarīra’nti vāti? Yo so, āvuso, bhikkhu evaṃ jānāti evaṃ passati kallaṃ, tassetaṃ vacanāya – ‘taṃ jīvaṃ taṃ sarīra’nti vā ‘aññaṃ jīvaṃ aññaṃ sarīra’nti vāti. Ahaṃ kho panetaṃ, āvuso, evaṃ jānāmi evaṃ passāmi. Atha ca panāhaṃ na vadāmi – ‘taṃ jīvaṃ taṃ sarīra’nti vā ‘aññaṃ jīvaṃ aññaṃ sarīra’nti vā…pe… ñāṇadassanāya cittaṃ abhinīharati abhininnāmeti…pe… yo kho, āvuso, bhikkhu evaṃ jānāti evaṃ passati, kallaṃ nu kho tassetaṃ vacanāya – ‘taṃ jīvaṃ taṃ sarīra’nti vā ‘aññaṃ jīvaṃ aññaṃ sarīra’nti vāti. Yo so, āvuso, bhikkhu evaṃ jānāti evaṃ passati kallaṃ tassetaṃ vacanāya – ‘taṃ jīvaṃ taṃ sarīra’nti vā ‘aññaṃ jīvaṃ aññaṃ sarīra’nti vāti. Ahaṃ kho panetaṃ, āvuso, evaṃ jānāmi evaṃ passāmi. Atha ca panāhaṃ na vadāmi – ‘taṃ jīvaṃ taṃ sarīra’nti vā ‘aññaṃ jīvaṃ aññaṃ sarīra’nti vā…pe….

\paragraph{380.} …Pe… nāparaṃ itthattāyāti pajānāti. Yo kho, āvuso, bhikkhu evaṃ jānāti evaṃ passati, kallaṃ nu kho tassetaṃ vacanāya – ‘taṃ jīvaṃ taṃ sarīra’nti vā ‘aññaṃ jīvaṃ aññaṃ sarīra’nti vāti? Yo so, āvuso, bhikkhu evaṃ jānāti evaṃ passati, na kallaṃ tassetaṃ vacanāya – ‘taṃ jīvaṃ taṃ sarīra’nti vā ‘aññaṃ jīvaṃ aññaṃ sarīra’nti vāti. Ahaṃ kho panetaṃ, āvuso, evaṃ jānāmi evaṃ passāmi. Atha ca panāhaṃ na vadāmi – ‘taṃ jīvaṃ taṃ sarīra’nti vā ‘aññaṃ jīvaṃ aññaṃ sarīra’nti vā’’ti. Idamavoca bhagavā. Attamanā te dve pabbajitā bhagavato bhāsitaṃ abhinandunti.

\xsectionEnd{Jāliyasuttaṃ niṭṭhitaṃ sattamaṃ.}


\clearpage
\section{Mahāsīhanādasuttaṃ}

\subsubsection{Acelakassapavatthu}

\paragraph{381.} Evaṃ me sutaṃ – ekaṃ samayaṃ bhagavā uruññāyaṃ\footnote{ujuññāyaṃ (sī. syā. kaṃ. pī.)} viharati kaṇṇakatthale migadāye. Atha kho acelo kassapo yena bhagavā tenupasaṅkami; upasaṅkamitvā bhagavatā saddhiṃ sammodi. Sammodanīyaṃ kathaṃ sāraṇīyaṃ vītisāretvā ekamantaṃ aṭṭhāsi. Ekamantaṃ ṭhito kho acelo kassapo bhagavantaṃ etadavoca – ‘‘sutaṃ metaṃ, bho gotama – ‘samaṇo gotamo sabbaṃ tapaṃ garahati, sabbaṃ tapassiṃ lūkhājīviṃ ekaṃsena upakkosati upavadatī’ti. Ye te, bho gotama, evamāhaṃsu – ‘samaṇo gotamo sabbaṃ tapaṃ garahati, sabbaṃ tapassiṃ lūkhājīviṃ ekaṃsena upakkosati upavadatī’ti, kacci te bhoto gotamassa vuttavādino, na ca bhavantaṃ gotamaṃ abhūtena abbhācikkhanti, dhammassa cānudhammaṃ byākaronti, na ca koci sahadhammiko vādānuvādo gārayhaṃ ṭhānaṃ āgacchati? Anabbhakkhātukāmā hi mayaṃ bhavantaṃ gotama’’nti.

\paragraph{382.} ‘‘Ye te, kassapa, evamāhaṃsu – ‘samaṇo gotamo sabbaṃ tapaṃ garahati, sabbaṃ tapassiṃ lūkhājīviṃ ekaṃsena upakkosati upavadatī’ti, na me te vuttavādino, abbhācikkhanti ca pana maṃ te asatā abhūtena. Idhāhaṃ, kassapa, ekaccaṃ tapassiṃ lūkhājīviṃ passāmi dibbena cakkhunā visuddhena atikkantamānusakena kāyassa bhedā paraṃ maraṇā apāyaṃ duggatiṃ vinipātaṃ nirayaṃ upapannaṃ. Idha panāhaṃ, kassapa, ekaccaṃ tapassiṃ lūkhājīviṃ passāmi dibbena cakkhunā visuddhena atikkantamānusakena kāyassa bhedā paraṃ maraṇā sugatiṃ saggaṃ lokaṃ upapannaṃ.

\paragraph{383.} ‘‘Idhāhaṃ, kassapa, ekaccaṃ tapassiṃ appadukkhavihāriṃ passāmi dibbena cakkhunā visuddhena atikkantamānusakena kāyassa bhedā paraṃ maraṇā apāyaṃ duggatiṃ vinipātaṃ nirayaṃ upapannaṃ. Idha panāhaṃ, kassapa, ekaccaṃ tapassiṃ appadukkhavihāriṃ passāmi dibbena cakkhunā visuddhena atikkantamānusakena kāyassa bhedā paraṃ maraṇā sugatiṃ saggaṃ lokaṃ upapannaṃ. Yohaṃ, kassapa, imesaṃ tapassīnaṃ evaṃ āgatiñca gatiñca cutiñca upapattiñca yathābhūtaṃ pajānāmi, sohaṃ kiṃ sabbaṃ tapaṃ garahissāmi, sabbaṃ vā tapassiṃ lūkhājīviṃ ekaṃsena upakkosissāmi upavadissāmi?

\paragraph{384.} ‘‘Santi, kassapa, eke samaṇabrāhmaṇā paṇḍitā nipuṇā kataparappavādā vālavedhirūpā. Te bhindantā maññe caranti paññāgatena diṭṭhigatāni. Tehipi me saddhiṃ ekaccesu ṭhānesu sameti, ekaccesu ṭhānesu na sameti. Yaṃ te ekaccaṃ vadanti ‘sādhū’ti, mayampi taṃ ekaccaṃ vadema ‘sādhū’ti. Yaṃ te ekaccaṃ vadanti ‘na sādhū’ti, mayampi taṃ ekaccaṃ vadema ‘na sādhū’ti. Yaṃ te ekaccaṃ vadanti ‘sādhū’ti, mayaṃ taṃ ekaccaṃ vadema ‘na sādhū’ti. Yaṃ te ekaccaṃ vadanti ‘na sādhū’ti, mayaṃ taṃ ekaccaṃ vadema ‘sādhū’ti. ‘‘Yaṃ mayaṃ ekaccaṃ vadema ‘sādhū’ti, parepi taṃ ekaccaṃ vadanti ‘sādhū’ti. Yaṃ mayaṃ ekaccaṃ vadema ‘na sādhū’ti, parepi taṃ ekaccaṃ vadanti ‘na sādhū’ti. Yaṃ mayaṃ ekaccaṃ vadema ‘na sādhū’ti, pare taṃ ekaccaṃ vadanti ‘sādhū’ti. Yaṃ mayaṃ ekaccaṃ vadema ‘sādhū’ti, pare taṃ ekaccaṃ vadanti ‘na sādhū’ti.

\subsubsection{Samanuyuñjāpanakathā}

\paragraph{385.} ‘‘Tyāhaṃ upasaṅkamitvā evaṃ vadāmi – yesu no, āvuso, ṭhānesu na sameti, tiṭṭhantu tāni ṭhānāni. Yesu ṭhānesu sameti, tattha viññū samanuyuñjantaṃ samanugāhantaṃ samanubhāsantaṃ satthārā vā satthāraṃ saṅghena vā saṅghaṃ – ‘ye imesaṃ bhavataṃ dhammā akusalā akusalasaṅkhātā, sāvajjā sāvajjasaṅkhātā, asevitabbā asevitabbasaṅkhātā, na alamariyā na alamariyasaṅkhātā, kaṇhā kaṇhasaṅkhātā. Ko ime dhamme anavasesaṃ pahāya vattati, samaṇo vā gotamo, pare vā pana bhonto gaṇācariyā’ti?

\paragraph{386.} ‘‘Ṭhānaṃ kho panetaṃ, kassapa, vijjati, yaṃ viññū samanuyuñjantā samanugāhantā samanubhāsantā evaṃ vadeyyuṃ – ‘ye imesaṃ bhavataṃ dhammā akusalā akusalasaṅkhātā, sāvajjā sāvajjasaṅkhātā, asevitabbā asevitabbasaṅkhātā, na alamariyā na alamariyasaṅkhātā, kaṇhā kaṇhasaṅkhātā. Samaṇo gotamo ime dhamme anavasesaṃ pahāya vattati, yaṃ vā pana bhonto pare gaṇācariyā’ti. Itiha, kassapa, viññū samanuyuñjantā samanugāhantā samanubhāsantā amheva tattha yebhuyyena pasaṃseyyuṃ.

\paragraph{387.} ‘‘Aparampi no, kassapa, viññū samanuyuñjantaṃ samanugāhantaṃ samanubhāsantaṃ satthārā vā satthāraṃ saṅghena vā saṅghaṃ – ‘ye imesaṃ bhavataṃ dhammā kusalā kusalasaṅkhātā, anavajjā anavajjasaṅkhātā, sevitabbā sevitabbasaṅkhātā, alamariyā alamariyasaṅkhātā, sukkā sukkasaṅkhātā. Ko ime dhamme anavasesaṃ samādāya vattati, samaṇo vā gotamo, pare vā pana bhonto gaṇācariyā’ ti?

\paragraph{388.} ‘‘Ṭhānaṃ kho panetaṃ, kassapa, vijjati, yaṃ viññū samanuyuñjantā samanugāhantā samanubhāsantā evaṃ vadeyyuṃ – ‘ye imesaṃ bhavataṃ dhammā kusalā kusalasaṅkhātā, anavajjā anavajjasaṅkhātā, sevitabbā sevitabbasaṅkhātā, alamariyā alamariyasaṅkhātā, sukkā sukkasaṅkhātā. Samaṇo gotamo ime dhamme anavasesaṃ samādāya vattati, yaṃ vā pana bhonto pare gaṇācariyā’ti. Itiha, kassapa, viññū samanuyuñjantā samanugāhantā samanubhāsantā amheva tattha yebhuyyena pasaṃseyyuṃ.

\paragraph{389.} ‘‘Aparampi no, kassapa, viññū samanuyuñjantaṃ samanugāhantaṃ samanubhāsantaṃ satthārā vā satthāraṃ saṅghena vā saṅghaṃ – ‘ye imesaṃ bhavataṃ dhammā akusalā akusalasaṅkhātā, sāvajjā sāvajjasaṅkhātā, asevitabbā asevitabbasaṅkhātā, na alamariyā na alamariyasaṅkhātā, kaṇhā kaṇhasaṅkhātā. Ko ime dhamme anavasesaṃ pahāya vattati, gotamasāvakasaṅgho vā, pare vā pana bhonto gaṇācariyasāvakasaṅghā’ti?

\paragraph{390.} ‘‘Ṭhānaṃ kho panetaṃ, kassapa, vijjati, yaṃ viññū samanuyuñjantā samanugāhantā samanubhāsantā evaṃ vadeyyuṃ – ‘ye imesaṃ bhavataṃ dhammā akusalā akusalasaṅkhātā, sāvajjā sāvajjasaṅkhātā, asevitabbā asevitabbasaṅkhātā, na alamariyā na alamariyasaṅkhātā, kaṇhā kaṇhasaṅkhātā. Gotamasāvakasaṅgho ime dhamme anavasesaṃ pahāya vattati, yaṃ vā pana bhonto pare gaṇācariyasāvakasaṅghā’ti. Itiha, kassapa, viññū samanuyuñjantā samanugāhantā samanubhāsantā amheva tattha yebhuyyena pasaṃseyyuṃ.

\paragraph{391.} ‘‘Aparampi no, kassapa, viññū samanuyuñjantaṃ samanugāhantaṃ samanubhāsantaṃ satthārā vā satthāraṃ saṅghena vā saṅghaṃ. ‘Ye imesaṃ bhavataṃ dhammā kusalā kusalasaṅkhātā, anavajjā anavajjasaṅkhātā, sevitabbā sevitabbasaṅkhātā, alamariyā alamariyasaṅkhātā, sukkā sukkasaṅkhātā. Ko ime dhamme anavasesaṃ samādāya vattati, gotamasāvakasaṅgho vā, pare vā pana bhonto gaṇācariyasāvakasaṅghā’ti?

\paragraph{392.} ‘‘Ṭhānaṃ kho panetaṃ, kassapa, vijjati, yaṃ viññū samanuyuñjantā samanugāhantā samanubhāsantā evaṃ vadeyyuṃ – ‘ye imesaṃ bhavataṃ dhammā kusalā kusalasaṅkhātā, anavajjā anavajjasaṅkhātā, sevitabbā sevitabbasaṅkhātā, alamariyā alamariyasaṅkhātā, sukkā sukkasaṅkhātā. Gotamasāvakasaṅgho ime dhamme anavasesaṃ samādāya vattati, yaṃ vā pana bhonto pare gaṇācariyasāvakasaṅghā’ti. Itiha, kassapa, viññū samanuyuñjantā samanugāhantā samanubhāsantā amheva tattha yebhuyyena pasaṃseyyuṃ.

\subsubsection{Ariyo aṭṭhaṅgiko maggo}

\paragraph{393.} ‘‘Atthi, kassapa, maggo atthi paṭipadā, yathāpaṭipanno sāmaṃyeva ñassati sāmaṃ dakkhati\footnote{dakkhiti (sī.)} – ‘samaṇova gotamo kālavādī bhūtavādī atthavādī dhammavādī vinayavādī’ti. Katamo ca, kassapa, maggo, katamā ca paṭipadā, yathāpaṭipanno sāmaṃyeva ñassati sāmaṃ dakkhati – ‘samaṇova gotamo kālavādī bhūtavādī atthavādī dhammavādī vinayavādī’ti? Ayameva ariyo aṭṭhaṅgiko maggo. Seyyathidaṃ – sammādiṭṭhi sammāsaṅkappo sammāvācā sammākammanto sammāājīvo sammāvāyāmo sammāsati sammāsamādhi. Ayaṃ kho, kassapa, maggo, ayaṃ paṭipadā, yathāpaṭipanno sāmaṃyeva ñassati sāmaṃ dakkhati ‘samaṇova gotamo kālavādī bhūtavādī atthavādī dhammavādī vinayavādī’’’ti.

\subsubsection{Tapopakkamakathā}

\paragraph{394.} Evaṃ vutte, acelo kassapo bhagavantaṃ etadavoca – ‘‘imepi kho, āvuso gotama, tapopakkamā etesaṃ samaṇabrāhmaṇānaṃ sāmaññasaṅkhātā ca brahmaññasaṅkhātā ca. Acelako hoti, muttācāro, hatthāpalekhano, na ehibhaddantiko, na tiṭṭhabhaddantiko, nābhihaṭaṃ, na uddissakataṃ, na nimantanaṃ sādiyati. So na kumbhimukhā paṭiggaṇhāti, na kaḷopimukhā paṭiggaṇhāti, na eḷakamantaraṃ, na daṇḍamantaraṃ, na musalamantaraṃ, na dvinnaṃ bhuñjamānānaṃ, na gabbhiniyā, na pāyamānāya, na purisantaragatāya, na saṅkittīsu, na yattha sā upaṭṭhito hoti, na yattha makkhikā saṇḍasaṇḍacārinī, na macchaṃ, na maṃsaṃ, na suraṃ, na merayaṃ, na thusodakaṃ pivati. So ekāgāriko vā hoti ekālopiko, dvāgāriko vā hoti dvālopiko…pe… sattāgāriko vā hoti sattālopiko; ekissāpi dattiyā yāpeti, dvīhipi dattīhi yāpeti… sattahipi dattīhi yāpeti; ekāhikampi āhāraṃ āhāreti, dvīhikampi āhāraṃ āhāreti… sattāhikampi āhāraṃ āhāreti. Iti evarūpaṃ addhamāsikampi pariyāyabhattabhojanānuyogamanuyutto viharati.

\paragraph{395.} ‘‘Imepi kho, āvuso gotama, tapopakkamā etesaṃ samaṇabrāhmaṇānaṃ sāmaññasaṅkhātā ca brahmaññasaṅkhātā ca. Sākabhakkho vā hoti, sāmākabhakkho vā hoti, nīvārabhakkho vā hoti, daddulabhakkho vā hoti, haṭabhakkho vā hoti, kaṇabhakkho vā hoti, ācāmabhakkho vā hoti, piññākabhakkho vā hoti, tiṇabhakkho vā hoti, gomayabhakkho vā hoti, vanamūlaphalāhāro yāpeti pavattaphalabhojī.

\paragraph{396.} ‘‘Imepi kho, āvuso gotama, tapopakkamā etesaṃ samaṇabrāhmaṇānaṃ sāmaññasaṅkhātā ca brahmaññasaṅkhātā ca. Sāṇānipi dhāreti, masāṇānipi dhāreti, chavadussānipi dhāreti, paṃsukūlānipi dhāreti, tirīṭānipi dhāreti, ajinampi dhāreti, ajinakkhipampi dhāreti, kusacīrampi dhāreti, vākacīrampi dhāreti, phalakacīrampi dhāreti, kesakambalampi dhāreti, vāḷakambalampi dhāreti, ulūkapakkhikampi dhāreti, kesamassulocakopi hoti kesamassulocanānuyogamanuyutto, ubbhaṭṭhakopi\footnote{ubbhaṭṭhikopi (ka.)} hoti āsanapaṭikkhitto, ukkuṭikopi hoti ukkuṭikappadhānamanuyutto, kaṇṭakāpassayikopi hoti kaṇṭakāpassaye seyyaṃ kappeti, phalakaseyyampi kappeti, thaṇḍilaseyyampi kappeti, ekapassayikopi hoti rajojalladharo, abbhokāsikopi hoti yathāsanthatiko, vekaṭikopi hoti vikaṭabhojanānuyogamanuyutto, apānakopi hoti apānakattamanuyutto, sāyatatiyakampi udakorohanānuyogamanuyutto viharatī’’ti.

\subsubsection{Tapopakkamaniratthakathā}

\paragraph{397.} ‘‘Acelako cepi, kassapa, hoti, muttācāro, hatthāpalekhano…pe… iti evarūpaṃ addhamāsikampi pariyāyabhattabhojanānuyogamanuyutto viharati. Tassa cāyaṃ sīlasampadā cittasampadā paññāsampadā abhāvitā hoti asacchikatā. Atha kho so ārakāva sāmaññā ārakāva brahmaññā. Yato kho, kassapa, bhikkhu averaṃ abyāpajjaṃ mettacittaṃ bhāveti, āsavānañca khayā anāsavaṃ cetovimuttiṃ paññāvimuttiṃ diṭṭheva dhamme sayaṃ abhiññā sacchikatvā upasampajja viharati. Ayaṃ vuccati, kassapa, bhikkhu samaṇo itipi brāhmaṇo itipi. ‘‘Sākabhakkho cepi, kassapa, hoti, sāmākabhakkho…pe… vanamūlaphalāhāro yāpeti pavattaphalabhojī. Tassa cāyaṃ sīlasampadā cittasampadā paññāsampadā abhāvitā hoti asacchikatā. Atha kho so ārakāva sāmaññā ārakāva brahmaññā. Yato kho, kassapa, bhikkhu averaṃ abyāpajjaṃ mettacittaṃ bhāveti, āsavānañca khayā anāsavaṃ cetovimuttiṃ paññāvimuttiṃ diṭṭheva dhamme sayaṃ abhiññā sacchikatvā upasampajja viharati. Ayaṃ vuccati, kassapa, bhikkhu samaṇo itipi brāhmaṇo itipi. ‘‘Sāṇāni cepi, kassapa, dhāreti, masāṇānipi dhāreti…pe… sāyatatiyakampi udakorohanānuyogamanuyutto viharati. Tassa cāyaṃ sīlasampadā cittasampadā paññāsampadā abhāvitā hoti asacchikatā. Atha kho so ārakāva sāmaññā ārakāva brahmaññā. Yato kho, kassapa, bhikkhu averaṃ abyāpajjaṃ mettacittaṃ bhāveti, āsavānañca khayā anāsavaṃ cetovimuttiṃ paññāvimuttiṃ diṭṭheva dhamme sayaṃ abhiññā sacchikatvā upasampajja viharati. Ayaṃ vuccati, kassapa, bhikkhu samaṇo itipi brāhmaṇo itipī’’ti.

\paragraph{398.} Evaṃ vutte, acelo kassapo bhagavantaṃ etadavoca – ‘‘dukkaraṃ, bho gotama, sāmaññaṃ dukkaraṃ brahmañña’’nti. ‘‘Pakati kho esā, kassapa, lokasmiṃ ‘dukkaraṃ sāmaññaṃ dukkaraṃ brahmañña’nti. Acelako cepi, kassapa, hoti, muttācāro, hatthāpalekhano…pe… iti evarūpaṃ addhamāsikampi pariyāyabhattabhojanānuyogamanuyutto viharati. Imāya ca, kassapa, mattāya iminā tapopakkamena sāmaññaṃ vā abhavissa brahmaññaṃ vā dukkaraṃ sudukkaraṃ, netaṃ abhavissa kallaṃ vacanāya – ‘dukkaraṃ sāmaññaṃ dukkaraṃ brahmañña’nti. ‘‘Sakkā ca panetaṃ abhavissa kātuṃ gahapatinā vā gahapatiputtena vā antamaso kumbhadāsiyāpi – ‘handāhaṃ acelako homi, muttācāro, hatthāpalekhano…pe… iti evarūpaṃ addhamāsikampi pariyāyabhattabhojanānuyogamanuyutto viharāmī’ti. ‘‘Yasmā ca kho, kassapa, aññatreva imāya mattāya aññatra iminā tapopakkamena sāmaññaṃ vā hoti brahmaññaṃ vā dukkaraṃ sudukkaraṃ, tasmā etaṃ kallaṃ vacanāya – ‘dukkaraṃ sāmaññaṃ dukkaraṃ brahmañña’nti. Yato kho, kassapa, bhikkhu averaṃ abyāpajjaṃ mettacittaṃ bhāveti, āsavānañca khayā anāsavaṃ cetovimuttiṃ paññāvimuttiṃ diṭṭheva dhamme sayaṃ abhiññā sacchikatvā upasampajja viharati. Ayaṃ vuccati, kassapa, bhikkhu samaṇo itipi brāhmaṇo itipi. ‘‘Sākabhakkho cepi, kassapa, hoti, sāmākabhakkho…pe… vanamūlaphalāhāro yāpeti pavattaphalabhojī. Imāya ca, kassapa, mattāya iminā tapopakkamena sāmaññaṃ vā abhavissa brahmaññaṃ vā dukkaraṃ sudukkaraṃ, netaṃ abhavissa kallaṃ vacanāya – ‘dukkaraṃ sāmaññaṃ dukkaraṃ brahmañña’nti. ‘‘Sakkā ca panetaṃ abhavissa kātuṃ gahapatinā vā gahapatiputtena vā antamaso kumbhadāsiyāpi – ‘handāhaṃ sākabhakkho vā homi, sāmākabhakkho vā…pe… vanamūlaphalāhāro yāpemi pavattaphalabhojī’ti. ‘‘Yasmā ca kho, kassapa, aññatreva imāya mattāya aññatra iminā tapopakkamena sāmaññaṃ vā hoti brahmaññaṃ vā dukkaraṃ sudukkaraṃ, tasmā etaṃ kallaṃ vacanāya – ‘dukkaraṃ sāmaññaṃ dukkaraṃ brahmañña’nti. Yato kho, kassapa, bhikkhu averaṃ abyāpajjaṃ mettacittaṃ bhāveti, āsavānañca khayā anāsavaṃ cetovimuttiṃ paññāvimuttiṃ diṭṭheva dhamme sayaṃ abhiññā sacchikatvā upasampajja viharati. Ayaṃ vuccati, kassapa, bhikkhu samaṇo itipi brāhmaṇo itipi. ‘‘Sāṇāni cepi, kassapa, dhāreti, masāṇānipi dhāreti…pe… sāyatatiyakampi udakorohanānuyogamanuyutto viharati. Imāya ca, kassapa, mattāya iminā tapopakkamena sāmaññaṃ vā abhavissa brahmaññaṃ vā dukkaraṃ sudukkaraṃ, netaṃ abhavissa kallaṃ vacanāya – ‘dukkaraṃ sāmaññaṃ dukkaraṃ brahmañña’nti. ‘‘Sakkā ca panetaṃ abhavissa kātuṃ gahapatinā vā gahapatiputtena vā antamaso kumbhadāsiyāpi – ‘handāhaṃ sāṇānipi dhāremi, masāṇānipi dhāremi…pe… sāyatatiyakampi udakorohanānuyogamanuyutto viharāmī’ti. ‘‘Yasmā ca kho, kassapa, aññatreva imāya mattāya aññatra iminā tapopakkamena sāmaññaṃ vā hoti brahmaññaṃ vā dukkaraṃ sudukkaraṃ, tasmā etaṃ kallaṃ vacanāya – ‘dukkaraṃ sāmaññaṃ dukkaraṃ brahmañña’nti. Yato kho, kassapa, bhikkhu averaṃ abyāpajjaṃ mettacittaṃ bhāveti, āsavānañca khayā anāsavaṃ cetovimuttiṃ paññāvimuttiṃ diṭṭheva dhamme sayaṃ abhiññā sacchikatvā upasampajja viharati. Ayaṃ vuccati, kassapa, bhikkhu samaṇo itipi brāhmaṇo itipī’’ti.

\paragraph{399.} Evaṃ vutte, acelo kassapo bhagavantaṃ etadavoca – ‘‘dujjāno, bho gotama, samaṇo, dujjāno brāhmaṇo’’ti. ‘‘Pakati kho esā, kassapa, lokasmiṃ ‘dujjāno samaṇo dujjāno brāhmaṇo’ti. Acelako cepi, kassapa, hoti, muttācāro, hatthāpalekhano…pe… iti evarūpaṃ addhamāsikampi pariyāyabhattabhojanānuyogamanuyutto viharati. Imāya ca, kassapa, mattāya iminā tapopakkamena samaṇo vā abhavissa brāhmaṇo vā dujjāno sudujjāno, netaṃ abhavissa kallaṃ vacanāya – ‘dujjāno samaṇo dujjāno brāhmaṇo’ti. ‘‘Sakkā ca paneso abhavissa ñātuṃ gahapatinā vā gahapatiputtena vā antamaso kumbhadāsiyāpi – ‘ayaṃ acelako hoti, muttācāro, hatthāpalekhano…pe… iti evarūpaṃ addhamāsikampi pariyāyabhattabhojanānuyogamanuyutto viharatī’ti. ‘‘Yasmā ca kho, kassapa, aññatreva imāya mattāya aññatra iminā tapopakkamena samaṇo vā hoti brāhmaṇo vā dujjāno sudujjāno, tasmā etaṃ kallaṃ vacanāya – ‘dujjāno samaṇo dujjāno brāhmaṇo’ti. Yato kho\footnote{yato ca kho (ka.)}, kassapa, bhikkhu averaṃ abyāpajjaṃ mettacittaṃ bhāveti, āsavānañca khayā anāsavaṃ cetovimuttiṃ paññāvimuttiṃ diṭṭheva dhamme sayaṃ abhiññā sacchikatvā upasampajja viharati. Ayaṃ vuccati, kassapa, bhikkhu samaṇo itipi brāhmaṇo itipi. ‘‘Sākabhakkho cepi, kassapa, hoti sāmākabhakkho…pe… vanamūlaphalāhāro yāpeti pavattaphalabhojī. Imāya ca, kassapa, mattāya iminā tapopakkamena samaṇo vā abhavissa brāhmaṇo vā dujjāno sudujjāno, netaṃ abhavissa kallaṃ vacanāya – ‘dujjāno samaṇo dujjāno brāhmaṇo’ti. ‘‘Sakkā ca paneso abhavissa ñātuṃ gahapatinā vā gahapatiputtena vā antamaso kumbhadāsiyāpi – ‘ayaṃ sākabhakkho vā hoti sāmākabhakkho…pe… vanamūlaphalāhāro yāpeti pavattaphalabhojī’ti. ‘‘Yasmā ca kho, kassapa, aññatreva imāya mattāya aññatra iminā tapopakkamena samaṇo vā hoti brāhmaṇo vā dujjāno sudujjāno, tasmā etaṃ kallaṃ vacanāya – ‘dujjāno samaṇo dujjāno brāhmaṇo’ti. Yato kho, kassapa, bhikkhu averaṃ abyāpajjaṃ mettacittaṃ bhāveti, āsavānañca khayā anāsavaṃ cetovimuttiṃ paññāvimuttiṃ diṭṭheva dhamme sayaṃ abhiññā sacchikatvā upasampajja viharati. Ayaṃ vuccati, kassapa, bhikkhu samaṇo itipi brāhmaṇo itipi. ‘‘Sāṇāni cepi, kassapa, dhāreti, masāṇānipi dhāreti…pe… sāyatatiyakampi udakorohanānuyogamanuyutto viharati. Imāya ca, kassapa, mattāya iminā tapopakkamena samaṇo vā abhavissa brāhmaṇo vā dujjāno sudujjāno, netaṃ abhavissa kallaṃ vacanāya – ‘dujjāno samaṇo dujjāno brāhmaṇo’ti. ‘‘Sakkā ca paneso abhavissa ñātuṃ gahapatinā vā gahapatiputtena vā antamaso kumbhadāsiyāpi – ‘ayaṃ sāṇānipi dhāreti, masāṇānipi dhāreti…pe… sāyatatiyakampi udakorohanānuyogamanuyutto viharatī’ti. ‘‘Yasmā ca kho, kassapa, aññatreva imāya mattāya aññatra iminā tapopakkamena samaṇo vā hoti brāhmaṇo vā dujjāno sudujjāno, tasmā etaṃ kallaṃ vacanāya – ‘dujjāno samaṇo dujjāno brāhmaṇo’ti. Yato kho, kassapa, bhikkhu averaṃ abyāpajjaṃ mettacittaṃ bhāveti, āsavānañca khayā anāsavaṃ cetovimuttiṃ paññāvimuttiṃ diṭṭheva dhamme sayaṃ abhiññā sacchikatvā upasampajja viharati. Ayaṃ vuccati, kassapa, bhikkhu samaṇo itipi brāhmaṇo itipī’’ti.

\subsubsection{Sīlasamādhipaññāsampadā}

\paragraph{400.} Evaṃ vutte, acelo kassapo bhagavantaṃ etadavoca – ‘‘katamā pana sā, bho gotama, sīlasampadā, katamā cittasampadā, katamā paññāsampadā’’ti? ‘‘Idha, kassapa, tathāgato loke uppajjati arahaṃ, sammāsambuddho…pe… (yathā 190-193 anucchedesu, evaṃ vitthāretabbaṃ) bhayadassāvī samādāya sikkhati sikkhāpadesu, kāyakammavacīkammena samannāgato kusalena parisuddhājīvo sīlasampanno indriyesu guttadvāro satisampajaññena samannāgato santuṭṭho.

\paragraph{401.} ‘‘Kathañca, kassapa, bhikkhu sīlasampanno hoti? Idha, kassapa, bhikkhu pāṇātipātaṃ pahāya pāṇātipātā paṭivirato hoti nihitadaṇḍo nihitasattho lajjī dayāpanno, sabbapāṇabhūtahitānukampī viharati. Idampissa hoti sīlasampadāya …pe… (yathā 194 yāva 210 anucchedesu) ‘‘Yathā vā paneke bhonto samaṇabrāhmaṇā saddhādeyyāni bhojanāni bhuñjitvā te evarūpāya tiracchānavijjāya micchājīvena jīvitaṃ kappenti. Seyyathidaṃ – santikammaṃ paṇidhikammaṃ…pe… (yathā 211 anucchede) osadhīnaṃ patimokkho iti vā iti, evarūpāya tiracchānavijjāya micchājīvā paṭivirato hoti. Idampissa hoti sīlasampadāya. ‘‘Sa kho so\footnote{ayaṃ kho (ka.)}, kassapa, bhikkhu evaṃ sīlasampanno na kutoci bhayaṃ samanupassati, yadidaṃ sīlasaṃvarato. Seyyathāpi, kassapa, rājā khattiyo muddhāvasitto nihatapaccāmitto na kutoci bhayaṃ samanupassati, yadidaṃ paccatthikato. Evameva kho, kassapa, bhikkhu evaṃ sīlasampanno na kutoci bhayaṃ samanupassati, yadidaṃ sīlasaṃvarato. So iminā ariyena sīlakkhandhena samannāgato ajjhattaṃ anavajjasukhaṃ paṭisaṃvedeti. Evaṃ kho, kassapa, bhikkhu sīlasampanno hoti. Ayaṃ kho, kassapa, sīlasampadā…pe… paṭhamaṃ jhānaṃ upasampajja viharati. Idampissa hoti cittasampadāya…pe… dutiyaṃ jhānaṃ…pe… tatiyaṃ jhānaṃ…pe… catutthaṃ jhānaṃ upasampajja viharati. Idampissa hoti cittasampadāya. Ayaṃ kho, kassapa, cittasampadā. ‘‘So evaṃ samāhite citte…pe… ñāṇadassanāya cittaṃ abhinīharati abhininnāmeti…pe… idampissa hoti paññāsampadāya…pe… nāparaṃ itthattāyāti pajānāti…pe… idampissa hoti paññāsampadāya. Ayaṃ kho, kassapa, paññāsampadā. ‘‘Imāya ca, kassapa, sīlasampadāya cittasampadāya paññāsampadāya aññā sīlasampadā cittasampadā paññāsampadā uttaritarā vā paṇītatarā vā natthi. 
\subsubsection{Sīhanādakathā}

\paragraph{402.} ‘‘Santi, kassapa, eke samaṇabrāhmaṇā sīlavādā. Te anekapariyāyena sīlassa vaṇṇaṃ bhāsanti. Yāvatā, kassapa, ariyaṃ paramaṃ sīlaṃ, nāhaṃ tattha attano samasamaṃ samanupassāmi, kuto bhiyyo! Atha kho ahameva tattha bhiyyo, yadidaṃ adhisīlaṃ. ‘‘Santi, kassapa, eke samaṇabrāhmaṇā tapojigucchāvādā. Te anekapariyāyena tapojigucchāya vaṇṇaṃ bhāsanti. Yāvatā, kassapa, ariyā paramā tapojigucchā, nāhaṃ tattha attano samasamaṃ samanupassāmi, kuto bhiyyo! Atha kho ahameva tattha bhiyyo, yadidaṃ adhijegucchaṃ. ‘‘Santi, kassapa, eke samaṇabrāhmaṇā paññāvādā. Te anekapariyāyena paññāya vaṇṇaṃ bhāsanti. Yāvatā, kassapa, ariyā paramā paññā, nāhaṃ tattha attano samasamaṃ samanupassāmi, kuto bhiyyo! Atha kho ahameva tattha bhiyyo, yadidaṃ adhipaññaṃ. ‘‘Santi, kassapa, eke samaṇabrāhmaṇā vimuttivādā. Te anekapariyāyena vimuttiyā vaṇṇaṃ bhāsanti. Yāvatā, kassapa, ariyā paramā vimutti, nāhaṃ tattha attano samasamaṃ samanupassāmi, kuto bhiyyo! Atha kho ahameva tattha bhiyyo, yadidaṃ adhivimutti.

\paragraph{403.} ‘‘Ṭhānaṃ kho panetaṃ, kassapa, vijjati, yaṃ aññatitthiyā paribbājakā evaṃ vadeyyuṃ – ‘sīhanādaṃ kho samaṇo gotamo nadati, tañca kho suññāgāre nadati, no parisāsū’ti. Te – ‘mā heva’ntissu vacanīyā. ‘Sīhanādañca samaṇo gotamo nadati, parisāsu ca nadatī’ti evamassu, kassapa, vacanīyā. ‘‘Ṭhānaṃ kho panetaṃ, kassapa, vijjati, yaṃ aññatitthiyā paribbājakā evaṃ vadeyyuṃ – ‘sīhanādañca samaṇo gotamo nadati, parisāsu ca nadati, no ca kho visārado nadatī’ti. Te – ‘mā heva’ntissu vacanīyā. ‘Sīhanādañca samaṇo gotamo nadati, parisāsu ca nadati, visārado ca nadatī’’ti evamassu, kassapa, vacanīyā. ‘‘Ṭhānaṃ kho panetaṃ, kassapa, vijjati, yaṃ aññatitthiyā paribbājakā evaṃ vadeyyuṃ – ‘sīhanādañca samaṇo gotamo nadati, parisāsu ca nadati, visārado ca nadati, no ca kho naṃ pañhaṃ pucchanti…pe… pañhañca naṃ pucchanti; no ca kho nesaṃ pañhaṃ puṭṭho byākaroti…pe… pañhañca nesaṃ puṭṭho byākaroti; no ca kho pañhassa veyyākaraṇena cittaṃ ārādheti…pe… pañhassa ca veyyākaraṇena cittaṃ ārādheti; no ca kho sotabbaṃ maññanti…pe… sotabbañcassa maññanti; no ca kho sutvā pasīdanti…pe… sutvā cassa pasīdanti; no ca kho pasannākāraṃ karonti…pe… pasannākārañca karonti; no ca kho tathattāya paṭipajjanti…pe… tathattāya ca paṭipajjanti; no ca kho paṭipannā ārādhentī’ti. Te – ‘mā heva’ntissu vacanīyā. ‘Sīhanādañca samaṇo gotamo nadati, parisāsu ca nadati, visārado ca nadati, pañhañca naṃ pucchanti, pañhañca nesaṃ puṭṭho byākaroti, pañhassa ca veyyākaraṇena cittaṃ ārādheti, sotabbañcassa maññanti, sutvā cassa pasīdanti, pasannākārañca karonti, tathattāya ca paṭipajjanti, paṭipannā ca ārādhentī’ti evamassu, kassapa, vacanīyā.

\subsubsection{Titthiyaparivāsakathā}

\paragraph{404.} ‘‘Ekamidāhaṃ, kassapa, samayaṃ rājagahe viharāmi gijjhakūṭe pabbate. Tatra maṃ aññataro tapabrahmacārī nigrodho nāma adhijegucche pañhaṃ apucchi. Tassāhaṃ adhijegucche pañhaṃ puṭṭho byākāsiṃ. Byākate ca pana me attamano ahosi paraṃ viya mattāyā’’ti. ‘‘Ko hi, bhante, bhagavato dhammaṃ sutvā na attamano assa paraṃ viya mattāya? Ahampi hi, bhante, bhagavato dhammaṃ sutvā attamano paraṃ viya mattāya. Abhikkantaṃ, bhante, abhikkantaṃ, bhante. Seyyathāpi, bhante, nikkujjitaṃ vā ukkujjeyya, paṭicchannaṃ vā vivareyya, mūḷhassa vā maggaṃ ācikkheyya, andhakāre vā telapajjotaṃ dhāreyya – ‘cakkhumanto rūpāni dakkhantī’ti; evamevaṃ bhagavatā anekapariyāyena dhammo pakāsito. Esāhaṃ, bhante, bhagavantaṃ saraṇaṃ gacchāmi, dhammañca bhikkhusaṅghañca. Labheyyāhaṃ, bhante, bhagavato santike pabbajjaṃ, labheyyaṃ upasampada’’nti.

\paragraph{405.} ‘‘Yo kho, kassapa, aññatitthiyapubbo imasmiṃ dhammavinaye ākaṅkhati pabbajjaṃ, ākaṅkhati upasampadaṃ, so cattāro māse parivasati, catunnaṃ māsānaṃ accayena āraddhacittā bhikkhū pabbājenti, upasampādenti bhikkhubhāvāya. Api ca mettha puggalavemattatā viditā’’ti. ‘‘Sace, bhante, aññatitthiyapubbā imasmiṃ dhammavinaye ākaṅkhanti pabbajjaṃ, ākaṅkhanti upasampadaṃ, cattāro māse parivasanti, catunnaṃ māsānaṃ accayena āraddhacittā bhikkhū pabbājenti, upasampādenti bhikkhubhāvāya. Ahaṃ cattāri vassāni parivasissāmi, catunnaṃ vassānaṃ accayena āraddhacittā bhikkhū pabbājentu, upasampādentu bhikkhubhāvāyā’’ti. Alattha kho acelo kassapo bhagavato santike pabbajjaṃ, alattha upasampadaṃ. Acirūpasampanno kho panāyasmā kassapo eko vūpakaṭṭho appamatto ātāpī pahitatto viharanto na cirasseva – yassatthāya kulaputtā sammadeva agārasmā anagāriyaṃ pabbajanti, tadanuttaraṃ – brahmacariyapariyosānaṃ diṭṭheva dhamme sayaṃ abhiññā sacchikatvā upasampajja vihāsi. ‘Khīṇā jāti, vusitaṃ brahmacariyaṃ, kataṃ karaṇīyaṃ, nāparaṃ itthattāyā’ti – abbhaññāsi. Aññataro kho panāyasmā kassapo arahataṃ ahosīti.

\xsectionEnd{Mahāsīhanādasuttaṃ niṭṭhitaṃ aṭṭhamaṃ.}


\clearpage
\section{Poṭṭhapādasuttaṃ}

\subsubsection{Poṭṭhapādaparibbājakavatthu}

\paragraph{406.} Evaṃ me sutaṃ – ekaṃ samayaṃ bhagavā sāvatthiyaṃ viharati jetavane anāthapiṇḍikassa ārāme. Tena kho pana samayena poṭṭhapādo paribbājako samayappavādake tindukācīre ekasālake mallikāya ārāme paṭivasati mahatiyā paribbājakaparisāya saddhiṃ tiṃsamattehi paribbājakasatehi. Atha kho bhagavā pubbaṇhasamayaṃ nivāsetvā pattacīvaramādāya sāvatthiṃ piṇḍāya pāvisi.

\paragraph{407.} Atha kho bhagavato etadahosi – ‘‘atippago kho tāva sāvatthiyaṃ piṇḍāya carituṃ. Yaṃnūnāhaṃ yena samayappavādako tindukācīro ekasālako mallikāya ārāmo, yena poṭṭhapādo paribbājako tenupasaṅkameyya’’nti. Atha kho bhagavā yena samayappavādako tindukācīro ekasālako mallikāya ārāmo tenupasaṅkami.

\paragraph{408.} Tena kho pana samayena poṭṭhapādo paribbājako mahatiyā paribbājakaparisāya saddhiṃ nisinno hoti unnādiniyā uccāsaddamahāsaddāya anekavihitaṃ tiracchānakathaṃ kathentiyā. Seyyathidaṃ – rājakathaṃ corakathaṃ mahāmattakathaṃ senākathaṃ bhayakathaṃ yuddhakathaṃ annakathaṃ pānakathaṃ vatthakathaṃ sayanakathaṃ mālākathaṃ gandhakathaṃ ñātikathaṃ yānakathaṃ gāmakathaṃ nigamakathaṃ nagarakathaṃ janapadakathaṃ itthikathaṃ sūrakathaṃ visikhākathaṃ kumbhaṭṭhānakathaṃ pubbapetakathaṃ nānattakathaṃ lokakkhāyikaṃ samuddakkhāyikaṃ itibhavābhavakathaṃ iti vā.

\paragraph{409.} Addasā kho poṭṭhapādo paribbājako bhagavantaṃ dūratova āgacchantaṃ; disvāna sakaṃ parisaṃ saṇṭhapesi – ‘‘appasaddā bhonto hontu, mā bhonto saddamakattha. Ayaṃ samaṇo gotamo āgacchati. Appasaddakāmo kho so āyasmā appasaddassa vaṇṇavādī. Appeva nāma appasaddaṃ parisaṃ viditvā upasaṅkamitabbaṃ maññeyyā’’ti. Evaṃ vutte te paribbājakā tuṇhī ahesuṃ.

\paragraph{410.} Atha kho bhagavā yena poṭṭhapādo paribbājako tenupasaṅkami. Atha kho poṭṭhapādo paribbājako bhagavantaṃ etadavoca – ‘‘etu kho, bhante, bhagavā. Svāgataṃ, bhante, bhagavato. Cirassaṃ kho, bhante, bhagavā imaṃ pariyāyamakāsi, yadidaṃ idhāgamanāya. Nisīdatu, bhante, bhagavā, idaṃ āsanaṃ paññatta’’nti. Nisīdi bhagavā paññatte āsane. Poṭṭhapādopi kho paribbājako aññataraṃ nīcaṃ āsanaṃ gahetvā ekamantaṃ nisīdi. Ekamantaṃ nisinnaṃ kho poṭṭhapādaṃ paribbājakaṃ bhagavā etadavoca – ‘‘kāya nuttha\footnote{kāya nottha (syā. ka.)}, poṭṭhapāda, etarahi kathāya sannisinnā, kā ca pana vo antarākathā vippakatā’’ti?

\subsubsection{Abhisaññānirodhakathā}

\paragraph{411.} Evaṃ vutte poṭṭhapādo paribbājako bhagavantaṃ etadavoca – ‘‘tiṭṭhatesā, bhante, kathā, yāya mayaṃ etarahi kathāya sannisinnā. Nesā, bhante, kathā bhagavato dullabhā bhavissati pacchāpi savanāya. Purimāni, bhante, divasāni purimatarāni, nānātitthiyānaṃ samaṇabrāhmaṇānaṃ kotūhalasālāya sannisinnānaṃ sannipatitānaṃ abhisaññānirodhe kathā udapādi – ‘kathaṃ nu kho, bho, abhisaññānirodho hotī’ti? Tatrekacce evamāhaṃsu – ‘ahetū appaccayā purisassa saññā uppajjantipi nirujjhantipi. Yasmiṃ samaye uppajjanti, saññī tasmiṃ samaye hoti. Yasmiṃ samaye nirujjhanti, asaññī tasmiṃ samaye hotī’ti. Ittheke abhisaññānirodhaṃ paññapenti. ‘‘Tamañño evamāha – ‘na kho pana metaṃ\footnote{na kho nāmetaṃ (sī. pī.)}, bho, evaṃ bhavissati. Saññā hi, bho, purisassa attā. Sā ca kho upetipi apetipi. Yasmiṃ samaye upeti, saññī tasmiṃ samaye hoti. Yasmiṃ samaye apeti, asaññī tasmiṃ samaye hotī’ti. Ittheke abhisaññānirodhaṃ paññapenti. ‘‘Tamañño evamāha – ‘na kho pana metaṃ, bho, evaṃ bhavissati. Santi hi, bho, samaṇabrāhmaṇā mahiddhikā mahānubhāvā. Te imassa purisassa saññaṃ upakaḍḍhantipi apakaḍḍhantipi. Yasmiṃ samaye upakaḍḍhanti, saññī tasmiṃ samaye hoti. Yasmiṃ samaye apakaḍḍhanti, asaññī tasmiṃ samaye hotī’ti. Ittheke abhisaññānirodhaṃ paññapenti. ‘‘Tamañño evamāha – ‘na kho pana metaṃ, bho, evaṃ bhavissati. Santi hi, bho, devatā mahiddhikā mahānubhāvā. Tā imassa purisassa saññaṃ upakaḍḍhantipi apakaḍḍhantipi. Yasmiṃ samaye upakaḍḍhanti, saññī tasmiṃ samaye hoti. Yasmiṃ samaye apakaḍḍhanti, asaññī tasmiṃ samaye hotī’ti. Ittheke abhisaññānirodhaṃ paññapenti. ‘‘Tassa mayhaṃ, bhante, bhagavantaṃyeva ārabbha sati udapādi – ‘aho nūna bhagavā, aho nūna sugato, yo imesaṃ dhammānaṃ sukusalo’ti. Bhagavā, bhante, kusalo, bhagavā pakataññū abhisaññānirodhassa. Kathaṃ nu kho, bhante, abhisaññānirodho hotī’’ti?

\subsubsection{Sahetukasaññuppādanirodhakathā}

\paragraph{412.} ‘‘Tatra, poṭṭhapāda, ye te samaṇabrāhmaṇā evamāhaṃsu – ‘ahetū appaccayā purisassa saññā uppajjantipi nirujjhantipī’ti, āditova tesaṃ aparaddhaṃ. Taṃ kissa hetu? Sahetū hi, poṭṭhapāda, sappaccayā purisassa saññā uppajjantipi nirujjhantipi. Sikkhā ekā saññā uppajjati, sikkhā ekā saññā nirujjhati’’.

\paragraph{413.} ‘‘Kā ca sikkhā’’ti? Bhagavā avoca – ‘‘idha, poṭṭhapāda, tathāgato loke uppajjati arahaṃ, sammāsambuddho…pe… (yathā 190-212 anucchedesu, evaṃ vitthāretabbaṃ). Evaṃ kho, poṭṭhapāda, bhikkhu sīlasampanno hoti…pe… tassime pañcanīvaraṇe pahīne attani samanupassato pāmojjaṃ jāyati, pamuditassa pīti jāyati, pītimanassa kāyo passambhati, passaddhakāyo sukhaṃ vedeti, sukhino cittaṃ samādhiyati. So vivicceva kāmehi, vivicca akusalehi dhammehi, savitakkaṃ savicāraṃ vivekajaṃ pītisukhaṃ paṭhamaṃ jhānaṃ upasampajja viharati. Tassa yā purimā kāmasaññā, sā nirujjhati. Vivekajapītisukhasukhumasaccasaññā tasmiṃ samaye hoti, vivekajapītisukhasukhumasaccasaññīyeva tasmiṃ samaye hoti. Evampi sikkhā ekā saññā uppajjati, sikkhā ekā saññā nirujjhati. Ayaṃ sikkhā’’ti bhagavā avoca. ‘‘Puna caparaṃ, poṭṭhapāda, bhikkhu vitakkavicārānaṃ vūpasamā ajjhattaṃ sampasādanaṃ cetaso ekodibhāvaṃ avitakkaṃ avicāraṃ samādhijaṃ pītisukhaṃ dutiyaṃ jhānaṃ upasampajja viharati. Tassa yā purimā vivekajapītisukhasukhumasaccasaññā, sā nirujjhati. Samādhijapītisukhasukhumasaccasaññā tasmiṃ samaye hoti, samādhijapītisukhasukhumasaccasaññīyeva tasmiṃ samaye hoti. Evampi sikkhā ekā saññā uppajjati, sikkhā ekā saññā nirujjhati. Ayampi sikkhā’’ti bhagavā avoca. ‘‘Puna caparaṃ, poṭṭhapāda, bhikkhu pītiyā ca virāgā upekkhako ca viharati sato ca sampajāno, sukhañca kāyena paṭisaṃvedeti, yaṃ taṃ ariyā ācikkhanti – ‘‘upekkhako satimā sukhavihārī’’ti, tatiyaṃ jhānaṃ upasampajja viharati. Tassa yā purimā samādhijapītisukhasukhumasaccasaññā, sā nirujjhati. Upekkhāsukhasukhumasaccasaññā tasmiṃ samaye hoti, upekkhāsukhasukhumasaccasaññīyeva tasmiṃ samaye hoti. Evampi sikkhā ekā saññā uppajjati, sikkhā ekā saññā nirujjhati. Ayampi sikkhā’’ti bhagavā avoca. ‘‘Puna caparaṃ, poṭṭhapāda, bhikkhu sukhassa ca pahānā dukkhassa ca pahānā pubbeva somanassadomanassānaṃ atthaṅgamā adukkhamasukhaṃ upekkhāsatipārisuddhiṃ catutthaṃ jhānaṃ upasampajja viharati. Tassa yā purimā upekkhāsukhasukhumasaccasaññā, sā nirujjhati. Adukkhamasukhasukhumasaccasaññā tasmiṃ samaye hoti, adukkhamasukhasukhumasaccasaññīyeva tasmiṃ samaye hoti. Evampi sikkhā ekā saññā uppajjati, sikkhā ekā saññā nirujjhati. Ayampi sikkhā’’ti bhagavā avoca. ‘‘Puna caparaṃ, poṭṭhapāda, bhikkhu sabbaso rūpasaññānaṃ samatikkamā paṭighasaññānaṃ atthaṅgamā nānattasaññānaṃ amanasikārā ‘ananto ākāso’ti ākāsānañcāyatanaṃ upasampajja viharati. Tassa yā purimā rūpasaññā\footnote{purimasaññā (ka.)}, sā nirujjhati. Ākāsānañcāyatanasukhumasaccasaññā tasmiṃ samaye hoti, ākāsānañcāyatanasukhumasaccasaññīyeva tasmiṃ samaye hoti. Evampi sikkhā ekā saññā uppajjati, sikkhā ekā saññā nirujjhati. Ayampi sikkhā’’ti bhagavā avoca. ‘‘Puna caparaṃ, poṭṭhapāda, bhikkhu sabbaso ākāsānañcāyatanaṃ samatikkamma ‘anantaṃ viññāṇa’nti viññāṇañcāyatanaṃ upasampajja viharati. Tassa yā purimā ākāsānañcāyatanasukhumasaccasaññā, sā nirujjhati. Viññāṇañcāyatanasukhumasaccasaññā tasmiṃ samaye hoti, viññāṇañcāyatanasukhumasaccasaññīyeva tasmiṃ samaye hoti. Evampi sikkhā ekā saññā uppajjati, sikkhā ekā saññā nirujjhati. Ayampi sikkhā’’ti bhagavā avoca. ‘‘Puna caparaṃ, poṭṭhapāda, bhikkhu sabbaso viññāṇañcāyatanaṃ samatikkamma ‘natthi kiñcī’ti ākiñcaññāyatanaṃ upasampajja viharati. Tassa yā purimā viññāṇañcāyatanasukhumasaccasaññā, sā nirujjhati. Ākiñcaññāyatanasukhumasaccasaññā tasmiṃ samaye hoti, ākiñcaññāyatanasukhumasaccasaññīyeva tasmiṃ samaye hoti. Evampi sikkhā ekā saññā uppajjati, sikkhā ekā saññā nirujjhati. Ayampi sikkhā’’ti bhagavā avoca.

\paragraph{414.} ‘‘Yato kho, poṭṭhapāda, bhikkhu idha sakasaññī hoti, so tato amutra tato amutra anupubbena saññaggaṃ phusati. Tassa saññagge ṭhitassa evaṃ hoti – ‘cetayamānassa me pāpiyo, acetayamānassa me seyyo. Ahañceva kho pana ceteyyaṃ, abhisaṅkhareyyaṃ, imā ca me saññā nirujjheyyuṃ, aññā ca oḷārikā saññā uppajjeyyuṃ; yaṃnūnāhaṃ na ceva ceteyyaṃ na ca abhisaṅkhareyya’nti. So na ceva ceteti, na ca abhisaṅkharoti. Tassa acetayato anabhisaṅkharoto tā ceva saññā nirujjhanti, aññā ca oḷārikā saññā na uppajjanti. So nirodhaṃ phusati. Evaṃ kho, poṭṭhapāda, anupubbābhisaññānirodha-sampajānasamāpatti hoti. ‘‘Taṃ kiṃ maññasi, poṭṭhapāda, api nu te ito pubbe evarūpā anupubbābhisaññānirodha-sampajāna-samāpatti sutapubbā’’ti? ‘‘No hetaṃ, bhante. Evaṃ kho ahaṃ, bhante, bhagavato bhāsitaṃ ājānāmi – ‘yato kho, poṭṭhapāda, bhikkhu idha sakasaññī hoti, so tato amutra tato amutra anupubbena saññaggaṃ phusati, tassa saññagge ṭhitassa evaṃ hoti – ‘‘cetayamānassa me pāpiyo, acetayamānassa me seyyo. Ahañceva kho pana ceteyyaṃ abhisaṅkhareyyaṃ, imā ca me saññā nirujjheyyuṃ, aññā ca oḷārikā saññā uppajjeyyuṃ; yaṃnūnāhaṃ na ceva ceteyyaṃ, na ca abhisaṅkhareyya’’nti. So na ceva ceteti, na cābhisaṅkharoti, tassa acetayato anabhisaṅkharoto tā ceva saññā nirujjhanti, aññā ca oḷārikā saññā na uppajjanti. So nirodhaṃ phusati. Evaṃ kho, poṭṭhapāda, anupubbābhisaññānirodha-sampajāna-samāpatti hotī’’’ti. ‘‘Evaṃ, poṭṭhapādā’’ti.

\paragraph{415.} ‘‘Ekaññeva nu kho, bhante, bhagavā saññaggaṃ paññapeti, udāhu puthūpi saññagge paññapetī’’ti? ‘‘Ekampi kho ahaṃ, poṭṭhapāda, saññaggaṃ paññapemi, puthūpi saññagge paññapemī’’ti. ‘‘Yathā kathaṃ pana, bhante, bhagavā ekampi saññaggaṃ paññapeti, puthūpi saññagge paññapetī’’ti? ‘‘Yathā yathā kho, poṭṭhapāda, nirodhaṃ phusati, tathā tathāhaṃ saññaggaṃ paññapemi. Evaṃ kho ahaṃ, poṭṭhapāda, ekampi saññaggaṃ paññapemi, puthūpi saññagge paññapemī’’ti.

\paragraph{416.} ‘‘Saññā nu kho, bhante, paṭhamaṃ uppajjati, pacchā ñāṇaṃ, udāhu ñāṇaṃ paṭhamaṃ uppajjati, pacchā saññā, udāhu saññā ca ñāṇañca apubbaṃ acarimaṃ uppajjantī’’ti? ‘‘Saññā kho, poṭṭhapāda, paṭhamaṃ uppajjati, pacchā ñāṇaṃ, saññuppādā ca pana ñāṇuppādo hoti. So evaṃ pajānāti – ‘idappaccayā kira me ñāṇaṃ udapādī’ti. Iminā kho etaṃ, poṭṭhapāda, pariyāyena veditabbaṃ – yathā saññā paṭhamaṃ uppajjati, pacchā ñāṇaṃ, saññuppādā ca pana ñāṇuppādo hotī’’ti.

\subsubsection{Saññāattakathā}

\paragraph{417.} ‘‘Saññā nu kho, bhante, purisassa attā, udāhu aññā saññā añño attā’’ti? ‘‘Kaṃ pana tvaṃ, poṭṭhapāda, attānaṃ paccesī’’ti? ‘‘Oḷārikaṃ kho ahaṃ, bhante, attānaṃ paccemi rūpiṃ cātumahābhūtikaṃ kabaḷīkārāhārabhakkha’’nti\footnote{kabaḷīkārabhakkhanti (syā. ka.)}. ‘‘Oḷāriko ca hi te, poṭṭhapāda, attā abhavissa rūpī cātumahābhūtiko kabaḷīkārāhārabhakkho. Evaṃ santaṃ kho te, poṭṭhapāda, aññāva saññā bhavissati añño attā. Tadamināpetaṃ, poṭṭhapāda, pariyāyena veditabbaṃ yathā aññāva saññā bhavissati añño attā. Tiṭṭhateva sāyaṃ\footnote{tiṭṭhatevāyaṃ (sī. pī.)}, poṭṭhapāda, oḷāriko attā rūpī cātumahābhūtiko kabaḷīkārāhārabhakkho, atha imassa purisassa aññā ca saññā uppajjanti, aññā ca saññā nirujjhanti. Iminā kho etaṃ, poṭṭhapāda, pariyāyena veditabbaṃ yathā aññāva saññā bhavissati añño attā’’ti.

\paragraph{418.} ‘‘Manomayaṃ kho ahaṃ, bhante, attānaṃ paccemi sabbaṅgapaccaṅgiṃ ahīnindriya’’nti. ‘‘Manomayo ca hi te, poṭṭhapāda, attā abhavissa sabbaṅgapaccaṅgī ahīnindriyo, evaṃ santampi kho te, poṭṭhapāda, aññāva saññā bhavissati añño attā. Tadamināpetaṃ, poṭṭhapāda, pariyāyena veditabbaṃ yathā aññāva saññā bhavissati añño attā. Tiṭṭhateva sāyaṃ, poṭṭhapāda, manomayo attā sabbaṅgapaccaṅgī ahīnindriyo, atha imassa purisassa aññā ca saññā uppajjanti, aññā ca saññā nirujjhanti. Imināpi kho etaṃ, poṭṭhapāda, pariyāyena veditabbaṃ yathā aññāva saññā bhavissati añño attā’’ti.

\paragraph{419.} ‘‘Arūpiṃ kho ahaṃ, bhante, attānaṃ paccemi saññāmaya’’nti. ‘‘Arūpī ca hi te, poṭṭhapāda, attā abhavissa saññāmayo, evaṃ santampi kho te, poṭṭhapāda, aññāva saññā bhavissati añño attā. Tadamināpetaṃ, poṭṭhapāda, pariyāyena veditabbaṃ yathā aññāva saññā bhavissati añño attā. Tiṭṭhateva sāyaṃ, poṭṭhapāda, arūpī attā saññāmayo, atha imassa purisassa aññā ca saññā uppajjanti, aññā ca saññā nirujjhanti. Imināpi kho etaṃ, poṭṭhapāda, pariyāyena veditabbaṃ yathā aññāva saññā bhavissati añño attā’’ti.

\paragraph{420.} ‘‘Sakkā panetaṃ, bhante, mayā ñātuṃ – ‘saññā purisassa attā’ti vā ‘aññāva saññā añño attāti vā’ti? ‘‘Dujjānaṃ kho etaṃ\footnote{evaṃ (ka.)}, poṭṭhapāda, tayā aññadiṭṭhikena aññakhantikena aññarucikena aññatrāyogena aññatrācariyakena – ‘saññā purisassa attā’ti vā, ‘aññāva saññā añño attāti vā’’’ti. ‘‘Sace taṃ, bhante, mayā dujjānaṃ aññadiṭṭhikena aññakhantikena aññarucikena aññatrāyogena aññatrācariyakena – ‘saññā purisassa attā’ti vā, ‘aññāva saññā añño attā’ti vā; ‘kiṃ pana, bhante, sassato loko, idameva saccaṃ moghamañña’nti? Abyākataṃ kho etaṃ, poṭṭhapāda, mayā – ‘sassato loko, idameva saccaṃ moghamañña’nti. ‘‘Kiṃ pana, bhante, ‘asassato loko, idameva saccaṃ moghamañña’’’nti? ‘‘Etampi kho, poṭṭhapāda, mayā abyākataṃ – ‘asassato loko, idameva saccaṃ moghamañña’’’nti. ‘‘Kiṃ pana, bhante, ‘antavā loko…pe… ‘anantavā loko … ‘taṃ jīvaṃ taṃ sarīraṃ… ‘aññaṃ jīvaṃ aññaṃ sarīraṃ… ‘hoti tathāgato paraṃ maraṇā… ‘na hoti tathāgato paraṃ maraṇā… ‘hoti ca na ca hoti tathāgato paraṃ maraṇā… ‘neva hoti na na hoti tathāgato paraṃ maraṇā, idameva saccaṃ moghamañña’’’nti? ‘‘Etampi kho, poṭṭhapāda, mayā abyākataṃ – ‘neva hoti na na hoti tathāgato paraṃ maraṇā, idameva saccaṃ moghamañña’’’nti. ‘‘Kasmā panetaṃ, bhante, bhagavatā abyākata’’nti? ‘‘Na hetaṃ, poṭṭhapāda, atthasaṃhitaṃ na dhammasaṃhitaṃ nādibrahmacariyakaṃ, na nibbidāya na virāgāya na nirodhāya na upasamāya na abhiññāya na sambodhāya na nibbānāya saṃvattati, tasmā etaṃ mayā abyākata’’nti. ‘‘Kiṃ pana, bhante, bhagavatā byākata’’nti? ‘‘Idaṃ dukkhanti kho, poṭṭhapāda, mayā byākataṃ. Ayaṃ dukkhasamudayoti kho, poṭṭhapāda, mayā byākataṃ. Ayaṃ dukkhanirodhoti kho, poṭṭhapāda, mayā byākataṃ. Ayaṃ dukkhanirodhagāminī paṭipadāti kho, poṭṭhapāda, mayā byākata’’nti. ‘‘Kasmā panetaṃ, bhante, bhagavatā byākata’’nti? ‘‘Etañhi, poṭṭhapāda, atthasaṃhitaṃ, etaṃ dhammasaṃhitaṃ, etaṃ ādibrahmacariyakaṃ, etaṃ nibbidāya virāgāya nirodhāya upasamāya abhiññāya sambodhāya nibbānāya saṃvattati; tasmā etaṃ mayā byākata’’nti. ‘‘Evametaṃ, bhagavā, evametaṃ, sugata. Yassadāni, bhante, bhagavā kālaṃ maññatī’’ti. Atha kho bhagavā uṭṭhāyāsanā pakkāmi.

\paragraph{421.} Atha kho te paribbājakā acirapakkantassa bhagavato poṭṭhapādaṃ paribbājakaṃ samantato vācā\footnote{vācāya (syā. ka.)} sannitodakena sañjhabbharimakaṃsu – ‘‘evameva panāyaṃ bhavaṃ poṭṭhapādo yaññadeva samaṇo gotamo bhāsati, taṃ tadevassa abbhanumodati – ‘evametaṃ bhagavā evametaṃ, sugatā’ti. Na kho pana mayaṃ kiñci\footnote{kañci (pī.)} samaṇassa gotamassa ekaṃsikaṃ dhammaṃ desitaṃ ājānāma – ‘sassato loko’ti vā, ‘asassato loko’ti vā, ‘antavā loko’ti vā, ‘anantavā loko’ti vā, ‘taṃ jīvaṃ taṃ sarīra’nti vā, ‘aññaṃ jīvaṃ aññaṃ sarīra’nti vā, ‘hoti tathāgato paraṃ maraṇā’ti vā, ‘na hoti tathāgato paraṃ maraṇā’ti vā, ‘hoti ca na ca hoti tathāgato paraṃ maraṇā’ti vā, ‘neva hoti na na hoti tathāgato paraṃ maraṇā’ti vā’’ti. Evaṃ vutte poṭṭhapādo paribbājako te paribbājake etadavoca – ‘‘ahampi kho, bho, na kiñci samaṇassa gotamassa ekaṃsikaṃ dhammaṃ desitaṃ ājānāmi – ‘sassato loko’ti vā, ‘asassato loko’ti vā…pe… ‘neva hoti na na hoti tathāgato paraṃ maraṇā’ti vā; api ca samaṇo gotamo bhūtaṃ tacchaṃ tathaṃ paṭipadaṃ paññapeti dhammaṭṭhitataṃ dhammaniyāmataṃ. Bhūtaṃ kho pana tacchaṃ tathaṃ paṭipadaṃ paññapentassa dhammaṭṭhitataṃ dhammaniyāmataṃ, kathañhi nāma mādiso viññū samaṇassa gotamassa subhāsitaṃ subhāsitato nābbhanumodeyyā’’ti?

\subsubsection{Cittahatthisāriputtapoṭṭhapādavatthu}

\paragraph{422.} Atha kho dvīhatīhassa accayena citto ca hatthisāriputto poṭṭhapādo ca paribbājako yena bhagavā tenupasaṅkamiṃsu; upasaṅkamitvā citto hatthisāriputto bhagavantaṃ abhivādetvā ekamantaṃ nisīdi. Poṭṭhapādo pana paribbājako bhagavatā saddhiṃ sammodi. Sammodanīyaṃ kathaṃ sāraṇīyaṃ vītisāretvā ekamantaṃ nisīdi. Ekamantaṃ nisinno kho poṭṭhapādo paribbājako bhagavantaṃ etadavoca – ‘‘tadā maṃ, bhante, te paribbājakā acirapakkantassa bhagavato samantato vācāsannitodakena sañjhabbharimakaṃsu – ‘evameva panāyaṃ bhavaṃ poṭṭhapādo yaññadeva samaṇo gotamo bhāsati, taṃ tadevassa abbhanumodati – ‘evametaṃ bhagavā evametaṃ sugatā’’ti. Na kho pana mayaṃ kiñci samaṇassa gotamassa ekaṃsikaṃ dhammaṃ desitaṃ ājānāma – ‘‘sassato loko’’ti vā, ‘‘asassato loko’’ti vā, ‘‘antavā loko’’ti vā, ‘‘anantavā loko’’ti vā, ‘‘taṃ jīvaṃ taṃ sarīra’’nti vā, ‘‘aññaṃ jīvaṃ aññaṃ sarīra’’nti vā, ‘‘hoti tathāgato paraṃ maraṇā’’ti vā, ‘‘na hoti tathāgato paraṃ maraṇā’’ti vā, ‘‘hoti ca na ca hoti tathāgato paraṃ maraṇā’’ti vā, ‘‘neva hoti na na hoti tathāgato paraṃ maraṇā’’ti vā’ti. Evaṃ vuttāhaṃ, bhante, te paribbājake etadavocaṃ – ‘ahampi kho, bho, na kiñci samaṇassa gotamassa ekaṃsikaṃ dhammaṃ desitaṃ ājānāmi – ‘‘sassato loko’’ti vā, ‘‘asassato loko’’ti vā…pe… ‘‘neva hoti na na hoti tathāgato paraṃ maraṇā’’ti vā; api ca samaṇo gotamo bhūtaṃ tacchaṃ tathaṃ paṭipadaṃ paññapeti dhammaṭṭhitataṃ dhammaniyāmataṃ. Bhūtaṃ kho pana tacchaṃ tathaṃ paṭipadaṃ paññapentassa dhammaṭṭhitataṃ dhammaniyāmataṃ, kathañhi nāma mādiso viññū samaṇassa gotamassa subhāsitaṃ subhāsitato nābbhanumodeyyā’’ti?

\paragraph{423.} ‘‘Sabbeva kho ete, poṭṭhapāda, paribbājakā andhā acakkhukā; tvaṃyeva nesaṃ eko cakkhumā. Ekaṃsikāpi hi kho, poṭṭhapāda, mayā dhammā desitā paññattā; anekaṃsikāpi hi kho, poṭṭhapāda, mayā dhammā desitā paññattā. ‘‘Katame ca te, poṭṭhapāda, mayā anekaṃsikā dhammā desitā paññattā? ‘Sassato loko’ti\footnote{lokoti vā (sī. ka.)} kho, poṭṭhapāda, mayā anekaṃsiko dhammo desito paññatto; ‘asassato loko’ti\footnote{lokoti vā (sī. ka.)} kho, poṭṭhapāda, mayā anekaṃsiko dhammo desito paññatto; ‘antavā loko’ti\footnote{lokoti vā (sī. ka.)} kho poṭṭhapāda…pe… ‘anantavā loko’ti\footnote{lokoti vā (sī. ka.)} kho poṭṭhapāda… ‘taṃ jīvaṃ taṃ sarīra’nti kho poṭṭhapāda… ‘aññaṃ jīvaṃ aññaṃ sarīra’nti kho poṭṭhapāda… ‘hoti tathāgato paraṃ maraṇā’ti kho poṭṭhapāda… na hoti tathāgato paraṃ maraṇā’ti kho poṭṭhapāda… ‘hoti ca na ca hoti tathāgato paraṃ maraṇā’ti kho poṭṭhapāda… ‘neva hoti na na hoti tathāgato paraṃ maraṇā’ti kho, poṭṭhapāda, mayā anekaṃsiko dhammo desito paññatto. ‘‘Kasmā ca te, poṭṭhapāda, mayā anekaṃsikā dhammā desitā paññattā? Na hete, poṭṭhapāda, atthasaṃhitā na dhammasaṃhitā na ādibrahmacariyakā na nibbidāya na virāgāya na nirodhāya na upasamāya na abhiññāya na sambodhāya na nibbānāya saṃvattanti. Tasmā te mayā anekaṃsikā dhammā desitā paññattā’’.

\subsubsection{Ekaṃsikadhammo}

\paragraph{424.} ‘‘Katame ca te, poṭṭhapāda, mayā ekaṃsikā dhammā desitā paññattā? Idaṃ dukkhanti kho, poṭṭhapāda, mayā ekaṃsiko dhammo desito paññatto. Ayaṃ dukkhasamudayoti kho, poṭṭhapāda, mayā ekaṃsiko dhammo desito paññatto. Ayaṃ dukkhanirodhoti kho, poṭṭhapāda, mayā ekaṃsiko dhammo desito paññatto. Ayaṃ dukkhanirodhagāminī paṭipadāti kho, poṭṭhapāda, mayā ekaṃsiko dhammo desito paññatto. ‘‘Kasmā ca te, poṭṭhapāda, mayā ekaṃsikā dhammā desitā paññattā? Ete, poṭṭhapāda, atthasaṃhitā, ete dhammasaṃhitā, ete ādibrahmacariyakā ete nibbidāya virāgāya nirodhāya upasamāya abhiññāya sambodhāya nibbānāya saṃvattanti. Tasmā te mayā ekaṃsikā dhammā desitā paññattā.

\paragraph{425.} ‘‘Santi, poṭṭhapāda, eke samaṇabrāhmaṇā evaṃvādino evaṃdiṭṭhino – ‘ekantasukhī attā hoti arogo paraṃ maraṇā’ti. Tyāhaṃ upasaṅkamitvā evaṃ vadāmi – ‘saccaṃ kira tumhe āyasmanto evaṃvādino evaṃdiṭṭhino – ‘‘ekantasukhī attā hoti arogo paraṃ maraṇā’ti? Te ce me evaṃ puṭṭhā ‘āmā’ti paṭijānanti. Tyāhaṃ evaṃ vadāmi – ‘api pana tumhe āyasmanto ekantasukhaṃ lokaṃ jānaṃ passaṃ viharathā’ti? Iti puṭṭhā ‘no’ti vadanti. ‘‘Tyāhaṃ evaṃ vadāmi – ‘api pana tumhe āyasmanto ekaṃ vā rattiṃ ekaṃ vā divasaṃ upaḍḍhaṃ vā rattiṃ upaḍḍhaṃ vā divasaṃ ekantasukhiṃ attānaṃ sañjānāthā’ti\footnote{sampajānāthāti (sī. syā. ka.)}? Iti puṭṭhā ‘no’ti vadanti. Tyāhaṃ evaṃ vadāmi – ‘api pana tumhe āyasmanto jānātha – ‘‘ayaṃ maggo ayaṃ paṭipadā ekantasukhassa lokassa sacchikiriyāyā’’’ti? Iti puṭṭhā ‘no’ti vadanti. ‘‘Tyāhaṃ evaṃ vadāmi – ‘api pana tumhe āyasmanto yā tā devatā ekantasukhaṃ lokaṃ upapannā, tāsaṃ bhāsamānānaṃ saddaṃ suṇātha – ‘‘suppaṭipannāttha, mārisā, ujuppaṭipannāttha, mārisā, ekantasukhassa lokassa sacchikiriyāya; mayampi hi, mārisā, evaṃpaṭipannā ekantasukhaṃ lokaṃ upapannā’ti? Iti puṭṭhā ‘no’ti vadanti. ‘‘Taṃ kiṃ maññasi, poṭṭhapāda, nanu evaṃ sante tesaṃ samaṇabrāhmaṇānaṃ appāṭihīrakataṃ bhāsitaṃ sampajjatī’’ti? ‘‘Addhā kho, bhante, evaṃ sante tesaṃ samaṇabrāhmaṇānaṃ appāṭihīrakataṃ bhāsitaṃ sampajjatī’’ti.

\paragraph{426.} ‘‘Seyyathāpi, poṭṭhapāda, puriso evaṃ vadeyya – ‘ahaṃ yā imasmiṃ janapade janapadakalyāṇī, taṃ icchāmi taṃ kāmemī’ti. Tamenaṃ evaṃ vadeyyuṃ – ‘ambho purisa, yaṃ tvaṃ janapadakalyāṇiṃ icchasi kāmesi, jānāsi taṃ janapadakalyāṇiṃ khattiyī vā brāhmaṇī vā vessī vā suddī vā’ti? Iti puṭṭho ‘no’ti vadeyya. Tamenaṃ evaṃ vadeyyuṃ – ‘ambho purisa, yaṃ tvaṃ janapadakalyāṇiṃ icchasi kāmesi, jānāsi taṃ janapadakalyāṇiṃ evaṃnāmā evaṃgottāti vā, dīghā vā rassā vā majjhimā vā kāḷī vā sāmā vā maṅguracchavī vāti, amukasmiṃ gāme vā nigame vā nagare vā’ti? Iti puṭṭho ‘no’ti vadeyya. Tamenaṃ evaṃ vadeyyuṃ – ‘ambho purisa, yaṃ tvaṃ na jānāsi na passasi, taṃ tvaṃ icchasi kāmesī’ti? Iti puṭṭho ‘āmā’ti vadeyya. ‘‘Taṃ kiṃ maññasi, poṭṭhapāda, nanu evaṃ sante tassa purisassa appāṭihīrakataṃ bhāsitaṃ sampajjatī’’ti? ‘‘Addhā kho, bhante, evaṃ sante tassa purisassa appāṭihīrakataṃ bhāsitaṃ sampajjatī’’ti. ‘‘Evameva kho, poṭṭhapāda, ye te samaṇabrāhmaṇā evaṃvādino evaṃdiṭṭhino – ‘ekantasukhī attā hoti arogo paraṃ maraṇā’ti. Tyāhaṃ upasaṅkamitvā evaṃ vadāmi – ‘saccaṃ kira tumhe āyasmanto evaṃvādino evaṃdiṭṭhino – ‘‘ekantasukhī attā hoti arogo paraṃ maraṇā’’’ti? Te ce me evaṃ puṭṭhā ‘āmā’ti paṭijānanti. Tyāhaṃ evaṃ vadāmi – ‘api pana tumhe āyasmanto ekantasukhaṃ lokaṃ jānaṃ passaṃ viharathā’ti? Iti puṭṭhā ‘no’ti vadanti. ‘‘Tyāhaṃ evaṃ vadāmi – ‘api pana tumhe āyasmanto ekaṃ vā rattiṃ ekaṃ vā divasaṃ upaḍḍhaṃ vā rattiṃ upaḍḍhaṃ vā divasaṃ ekantasukhiṃ attānaṃ sañjānāthā’ti? Iti puṭṭhā ‘no’ti vadanti. Tyāhaṃ evaṃ vadāmi – ‘api pana tumhe āyasmanto jānātha – ‘‘ayaṃ maggo ayaṃ paṭipadā ekantasukhassa lokassa sacchikiriyāyā’ti? Iti puṭṭhā ‘no’ti vadanti. ‘‘Tyāhaṃ evaṃ vadāmi – ‘api pana tumhe āyasmanto yā tā devatā ekantasukhaṃ lokaṃ upapannā, tāsaṃ bhāsamānānaṃ saddaṃ suṇātha – ‘‘suppaṭipannāttha, mārisā, ujuppaṭipannāttha, mārisā, ekantasukhassa lokassa sacchikiriyāya; mayampi hi, mārisā, evaṃpaṭipannā ekantasukhaṃ lokaṃ upapannā’’’ti? Iti puṭṭhā ‘no’ti vadanti. ‘‘Taṃ kiṃ maññasi, poṭṭhapāda, nanu evaṃ sante tesaṃ samaṇabrāhmaṇānaṃ appāṭihīrakataṃ bhāsitaṃ sampajjatī’’ti? ‘‘Addhā kho, bhante, evaṃ sante tesaṃ samaṇabrāhmaṇānaṃ appāṭihīrakataṃ bhāsitaṃ sampajjatī’’ti.

\paragraph{427.} ‘‘Seyyathāpi, poṭṭhapāda, puriso cātumahāpathe nisseṇiṃ kareyya pāsādassa ārohaṇāya. Tamenaṃ evaṃ vadeyyuṃ – ‘ambho purisa, yassa tvaṃ\footnote{yaṃ tvaṃ (sī. ka.)} pāsādassa ārohaṇāya nisseṇiṃ karosi, jānāsi taṃ pāsādaṃ puratthimāya vā disāya dakkhiṇāya vā disāya pacchimāya vā disāya uttarāya vā disāya ucco vā nīco vā majjhimo vā’ti? Iti puṭṭho ‘no’ti vadeyya. Tamenaṃ evaṃ vadeyyuṃ – ‘ambho purisa, yaṃ tvaṃ na jānāsi na passasi, tassa tvaṃ pāsādassa ārohaṇāya nisseṇiṃ karosī’ti? Iti puṭṭho ‘āmā’ti vadeyya. ‘‘Taṃ kiṃ maññasi, poṭṭhapāda, nanu evaṃ sante tassa purisassa appāṭihīrakataṃ bhāsitaṃ sampajjatī’’ti? ‘‘Addhā kho, bhante, evaṃ sante tassa purisassa appāṭihīrakataṃ bhāsitaṃ sampajjatī’’ti. ‘‘Evameva kho, poṭṭhapāda, ye te samaṇabrāhmaṇā evaṃvādino evaṃdiṭṭhino – ‘ekantasukhī attā hoti arogo paraṃ maraṇā’ti. Tyāhaṃ upasaṅkamitvā evaṃ vadāmi – ‘saccaṃ kira tumhe āyasmanto evaṃvādino evaṃdiṭṭhino – ‘‘ekantasukhī attā hoti arogo paraṃ maraṇā’ti? Te ce me evaṃ puṭṭhā ‘āmā’ti paṭijānanti. Tyāhaṃ evaṃ vadāmi – ‘api pana tumhe āyasmanto ekantasukhaṃ lokaṃ jānaṃ passaṃ viharathā’ti? Iti puṭṭhā ‘no’ti vadanti. ‘‘Tyāhaṃ evaṃ vadāmi – ‘api pana tumhe āyasmanto ekaṃ vā rattiṃ ekaṃ vā divasaṃ upaḍḍhaṃ vā rattiṃ upaḍḍhaṃ vā divasaṃ ekantasukhiṃ attānaṃ sañjānāthā’ti? Iti puṭṭhā ‘no’ti vadanti. Tyāhaṃ evaṃ vadāmi – ‘api pana tumhe āyasmanto jānātha ayaṃ maggo ayaṃ paṭipadā ekantasukhassa lokassa sacchikiriyāyā’ti? Iti puṭṭhā ‘no’ti vadanti. ‘‘Tyāhaṃ evaṃ vadāmi – ‘api pana tumhe āyasmanto yā tā devatā ekantasukhaṃ lokaṃ upapannā’ tāsaṃ devatānaṃ bhāsamānānaṃ saddaṃ suṇātha- ‘‘suppaṭipannāttha, mārisā, ujuppaṭipannāttha, mārisā, ekantasukhassa lokassa sacchikiriyāya; mayampi hi, mārisā, evaṃ paṭipannā ekantasukhaṃ lokaṃ upapannā’ti? Iti puṭṭhā ‘‘no’’ti vadanti. ‘‘Taṃ kiṃ maññasi, poṭṭhapāda, nanu evaṃ sante tesaṃ samaṇabrāhmaṇānaṃ appāṭihīrakataṃ bhāsitaṃ sampajjatī’’ti? ‘‘Addhā kho, bhante, evaṃ sante tesaṃ samaṇabrāhmaṇānaṃ appāṭihīrakataṃ bhāsitaṃ sampajjatī’’ti.

\subsubsection{Tayo attapaṭilābhā}

\paragraph{428.} ‘‘Tayo kho me, poṭṭhapāda, attapaṭilābhā – oḷāriko attapaṭilābho, manomayo attapaṭilābho, arūpo attapaṭilābho. Katamo ca, poṭṭhapāda, oḷāriko attapaṭilābho? Rūpī cātumahābhūtiko kabaḷīkārāhārabhakkho\footnote{kabaḷīkārabhakkho (syā. ka.)}, ayaṃ oḷāriko attapaṭilābho. Katamo manomayo attapaṭilābho? Rūpī manomayo sabbaṅgapaccaṅgī ahīnindriyo, ayaṃ manomayo attapaṭilābho. Katamo arūpo attapaṭilābho? Arūpī saññāmayo, ayaṃ arūpo attapaṭilābho.

\paragraph{429.} ‘‘Oḷārikassapi kho ahaṃ, poṭṭhapāda, attapaṭilābhassa pahānāya dhammaṃ desemi – yathāpaṭipannānaṃ vo saṃkilesikā dhammā pahīyissanti, vodāniyā dhammā abhivaḍḍhissanti, paññāpāripūriṃ vepullattañca diṭṭheva dhamme sayaṃ abhiññā sacchikatvā upasampajja viharissathāti. Siyā kho pana te, poṭṭhapāda, evamassa – saṃkilesikā dhammā pahīyissanti, vodāniyā dhammā abhivaḍḍhissanti, paññāpāripūriṃ vepullattañca diṭṭheva dhamme sayaṃ abhiññā sacchikatvā upasampajja viharissati, dukkho ca kho vihāroti, na kho panetaṃ, poṭṭhapāda, evaṃ daṭṭhabbaṃ. Saṃkilesikā ceva dhammā pahīyissanti, vodāniyā ca dhammā abhivaḍḍhissanti, paññāpāripūriṃ vepullattañca diṭṭheva dhamme sayaṃ abhiññā sacchikatvā upasampajja viharissati, pāmujjaṃ ceva bhavissati pīti ca passaddhi ca sati ca sampajaññañca sukho ca vihāro.

\paragraph{430.} ‘‘Manomayassapi kho ahaṃ, poṭṭhapāda, attapaṭilābhassa pahānāya dhammaṃ desemi yathāpaṭipannānaṃ vo saṃkilesikā dhammā pahīyissanti, vodāniyā dhammā abhivaḍḍhissanti, paññāpāripūriṃ vepullattañca diṭṭheva dhamme sayaṃ abhiññā sacchikatvā upasampajja viharissathāti. Siyā kho pana te, poṭṭhapāda, evamassa – ‘saṃkilesikā dhammā pahīyissanti, vodāniyā dhammā abhivaḍḍhissanti, paññāpāripūriṃ vepullattañca diṭṭheva dhamme sayaṃ abhiññā sacchikatvā upasampajja viharissati, dukkho ca kho vihāro’ti, na kho panetaṃ, poṭṭhapāda, evaṃ daṭṭhabbaṃ. Saṃkilesikā ceva dhammā pahīyissanti, vodāniyā ca dhammā abhivaḍḍhissanti, paññāpāripūriṃ vepullattañca diṭṭheva dhamme sayaṃ abhiññā sacchikatvā upasampajja viharissati, pāmujjaṃ ceva bhavissati pīti ca passaddhi ca sati ca sampajaññañca sukho ca vihāro.

\paragraph{431.} ‘‘Arūpassapi kho ahaṃ, poṭṭhapāda, attapaṭilābhassa pahānāya dhammaṃ desemi yathāpaṭipannānaṃ vo saṃkilesikā dhammā pahīyissanti, vodāniyā dhammā abhivaḍḍhissanti, paññāpāripūriṃ vepullattañca diṭṭheva dhamme sayaṃ abhiññā sacchikatvā upasampajja viharissathāti. Siyā kho pana te, poṭṭhapāda, evamassa – ‘saṃkilesikā dhammā pahīyissanti, vodāniyā dhammā abhivaḍḍhissanti, paññāpāripūriṃ vepullattañca diṭṭheva dhamme sayaṃ abhiññā sacchikatvā upasampajja viharissati, dukkho ca kho vihāro’ti, na kho panetaṃ, poṭṭhapāda, evaṃ daṭṭhabbaṃ. Saṃkilesikā ceva dhammā pahīyissanti, vodāniyā ca dhammā abhivaḍḍhissanti, paññāpāripūriṃ vepullattañca diṭṭheva dhamme sayaṃ abhiññā sacchikatvā upasampajja viharissati, pāmujjaṃ ceva bhavissati pīti ca passaddhi ca sati ca sampajaññañca sukho ca vihāro.

\paragraph{432.} ‘‘Pare ce, poṭṭhapāda, amhe evaṃ puccheyyuṃ – ‘katamo pana so, āvuso, oḷāriko attapaṭilābho, yassa tumhe pahānāya dhammaṃ desetha, yathāpaṭipannānaṃ vo saṃkilesikā dhammā pahīyissanti, vodāniyā dhammā abhivaḍḍhissanti, paññāpāripūriṃ vepullattañca diṭṭheva dhamme sayaṃ abhiññā sacchikatvā upasampajja viharissathā’ti, tesaṃ mayaṃ evaṃ puṭṭhā evaṃ byākareyyāma – ‘ayaṃ vā so, āvuso, oḷāriko attapaṭilābho, yassa mayaṃ pahānāya dhammaṃ desema, yathāpaṭipannānaṃ vo saṃkilesikā dhammā pahīyissanti, vodāniyā dhammā abhivaḍḍhissanti, paññāpāripūriṃ vepullattañca diṭṭheva dhamme sayaṃ abhiññā sacchikatvā upasampajja viharissathā’ti.

\paragraph{433.} ‘‘Pare ce, poṭṭhapāda, amhe evaṃ puccheyyuṃ – ‘katamo pana so, āvuso, manomayo attapaṭilābho, yassa tumhe pahānāya dhammaṃ desetha, yathāpaṭipannānaṃ vo saṃkilesikā dhammā pahīyissanti, vodāniyā dhammā abhivaḍḍhissanti, paññāpāripūriṃ vepullattañca diṭṭheva dhamme sayaṃ abhiññā sacchikatvā upasampajja viharissathā’ti? Tesaṃ mayaṃ evaṃ puṭṭhā evaṃ byākareyyāma – ‘ayaṃ vā so, āvuso, manomayo attapaṭilābho yassa mayaṃ pahānāya dhammaṃ desema, yathāpaṭipannānaṃ vo saṃkilesikā dhammā pahīyissanti, vodāniyā dhammā abhivaḍḍhissanti, paññāpāripūriṃ vepullattañca diṭṭheva dhamme sayaṃ abhiññā sacchikatvā upasampajja viharissathā’ti.

\paragraph{434.} ‘‘Pare ce, poṭṭhapāda, amhe evaṃ puccheyyuṃ – ‘katamo pana so, āvuso, arūpo attapaṭilābho, yassa tumhe pahānāya dhammaṃ desetha, yathāpaṭipannānaṃ vo saṃkilesikā dhammā pahīyissanti, vodāniyā dhammā abhivaḍḍhissanti, paññāpāripūriṃ vepullattañca diṭṭheva dhamme sayaṃ abhiññā sacchikatvā upasampajja viharissathā’ti, tesaṃ mayaṃ evaṃ puṭṭhā evaṃ byākareyyāma – ‘ayaṃ vā so, āvuso, arūpo attapaṭilābho yassa mayaṃ pahānāya dhammaṃ desema, yathāpaṭipannānaṃ vo saṃkilesikā dhammā pahīyissanti, vodāniyā dhammā abhivaḍḍhissanti, paññāpāripūriṃ vepullattañca diṭṭheva dhamme sayaṃ abhiññā sacchikatvā upasampajja viharissathā’ti. ‘‘Taṃ kiṃ maññasi, poṭṭhapāda, nanu evaṃ sante sappāṭihīrakataṃ bhāsitaṃ sampajjatī’’ti? ‘‘Addhā kho, bhante, evaṃ sante sappāṭihīrakataṃ bhāsitaṃ sampajjatī’’ti.

\paragraph{435.} ‘‘Seyyathāpi, poṭṭhapāda, puriso nisseṇiṃ kareyya pāsādassa ārohaṇāya tasseva pāsādassa heṭṭhā. Tamenaṃ evaṃ vadeyyuṃ – ‘ambho purisa, yassa tvaṃ pāsādassa ārohaṇāya nisseṇiṃ karosi, jānāsi taṃ pāsādaṃ, puratthimāya vā disāya dakkhiṇāya vā disāya pacchimāya vā disāya uttarāya vā disāya ucco vā nīco vā majjhimo vā’ti? So evaṃ vadeyya – ‘ayaṃ vā so, āvuso, pāsādo, yassāhaṃ ārohaṇāya nisseṇiṃ karomi, tasseva pāsādassa heṭṭhā’ti. ‘‘Taṃ kiṃ maññasi, poṭṭhapāda, nanu evaṃ sante tassa purisassa sappāṭihīrakataṃ bhāsitaṃ sampajjatī’’ti? ‘‘Addhā kho, bhante, evaṃ sante tassa purisassa sappāṭihīrakataṃ bhāsitaṃ sampajjatī’’ti.

\paragraph{436.} ‘‘Evameva kho, poṭṭhapāda, pare ce amhe evaṃ puccheyyuṃ – ‘katamo pana so, āvuso, oḷāriko attapaṭilābho…pe… katamo pana so, āvuso, manomayo attapaṭilābho…pe… katamo pana so, āvuso, arūpo attapaṭilābho, yassa tumhe pahānāya dhammaṃ desetha, yathāpaṭipannānaṃ vo saṃkilesikā dhammā pahīyissanti, vodāniyā dhammā abhivaḍḍhissanti, paññāpāripūriṃ vepullattañca diṭṭheva dhamme sayaṃ abhiññā sacchikatvā upasampajja viharissathā’ti, tesaṃ mayaṃ evaṃ puṭṭhā evaṃ byākareyyāma – ‘ayaṃ vā so, āvuso, arūpo attapaṭilābho, yassa mayaṃ pahānāya dhammaṃ desema, yathāpaṭipannānaṃ vo saṃkilesikā dhammā pahīyissanti, vodāniyā dhammā abhivaḍḍhissanti, paññāpāripūriṃ vepullattañca diṭṭheva dhamme sayaṃ abhiññā sacchikatvā upasampajja viharissathā’ti. ‘‘Taṃ kiṃ maññasi, poṭṭhapāda, nanu evaṃ sante sappāṭihīrakataṃ bhāsitaṃ sampajjatī’’ti? ‘‘Addhā kho, bhante, evaṃ sante sappāṭihīrakataṃ bhāsitaṃ sampajjatī’’ti.

\paragraph{437.} Evaṃ vutte citto hatthisāriputto bhagavantaṃ etadavoca – ‘‘yasmiṃ, bhante, samaye oḷāriko attapaṭilābho hoti, moghassa tasmiṃ samaye manomayo attapaṭilābho hoti, mogho arūpo attapaṭilābho hoti; oḷāriko vāssa attapaṭilābho tasmiṃ samaye sacco hoti. Yasmiṃ, bhante, samaye manomayo attapaṭilābho hoti, moghassa tasmiṃ samaye oḷāriko attapaṭilābho hoti, mogho arūpo attapaṭilābho hoti; manomayo vāssa attapaṭilābho tasmiṃ samaye sacco hoti. Yasmiṃ, bhante, samaye arūpo attapaṭilābho hoti, moghassa tasmiṃ samaye oḷāriko attapaṭilābho hoti, mogho manomayo attapaṭilābho hoti; arūpo vāssa attapaṭilābho tasmiṃ samaye sacco hotī’’ti. ‘‘Yasmiṃ, citta, samaye oḷāriko attapaṭilābho hoti, neva tasmiṃ samaye manomayo attapaṭilābhoti saṅkhaṃ gacchati, na arūpo attapaṭilābhoti saṅkhaṃ gacchati; oḷāriko attapaṭilābhotveva tasmiṃ samaye saṅkhaṃ gacchati. Yasmiṃ, citta, samaye manomayo attapaṭilābho hoti, neva tasmiṃ samaye oḷāriko attapaṭilābhoti saṅkhaṃ gacchati, na arūpo attapaṭilābhoti saṅkhaṃ gacchati; manomayo attapaṭilābhotveva tasmiṃ samaye saṅkhaṃ gacchati. Yasmiṃ, citta, samaye arūpo attapaṭilābho hoti, neva tasmiṃ samaye oḷāriko attapaṭilābhoti saṅkhaṃ gacchati, na manomayo attapaṭilābhoti saṅkhaṃ gacchati; arūpo attapaṭilābhotveva tasmiṃ samaye saṅkhaṃ gacchati.

\paragraph{438.} ‘‘Sace taṃ, citta, evaṃ puccheyyuṃ – ‘ahosi tvaṃ atītamaddhānaṃ, na tvaṃ nāhosi; bhavissasi tvaṃ anāgatamaddhānaṃ, na tvaṃ na bhavissasi; atthi tvaṃ etarahi, na tvaṃ natthī’ti, evaṃ puṭṭho tvaṃ, citta, kinti byākareyyāsī’’ti? ‘‘Sace maṃ, bhante, evaṃ puccheyyuṃ – ‘ahosi tvaṃ atītamaddhānaṃ, na tvaṃ na ahosi; bhavissasi tvaṃ anāgatamaddhānaṃ, na tvaṃ na bhavissasi; atthi tvaṃ etarahi, na tvaṃ natthī’ti. Evaṃ puṭṭho ahaṃ, bhante, evaṃ byākareyyaṃ – ‘ahosāhaṃ atītamaddhānaṃ, nāhaṃ na ahosiṃ; bhavissāmahaṃ anāgatamaddhānaṃ, nāhaṃ na bhavissāmi; atthāhaṃ etarahi, nāhaṃ natthī’ti. Evaṃ puṭṭho ahaṃ, bhante, evaṃ byākareyya’’nti. ‘‘Sace pana taṃ, citta, evaṃ puccheyyuṃ – ‘yo te ahosi atīto attapaṭilābho, sova\footnote{sveva (sī. pī.), soyeva (syā.)} te attapaṭilābho sacco, mogho anāgato, mogho paccuppanno? Yo\footnote{yo vā (pī.)} te bhavissati anāgato attapaṭilābho, sova te attapaṭilābho sacco, mogho atīto, mogho paccuppanno? Yo\footnote{yo vā (pī.)} te etarahi paccuppanno attapaṭilābho, sova\footnote{so ca (ka.)} te attapaṭilābho sacco, mogho atīto, mogho anāgato’ti. Evaṃ puṭṭho tvaṃ, citta, kinti byākareyyāsī’’ti? ‘‘Sace pana maṃ, bhante, evaṃ puccheyyuṃ – ‘yo te ahosi atīto attapaṭilābho, sova te attapaṭilābho sacco, mogho anāgato, mogho paccuppanno. Yo te bhavissati anāgato attapaṭilābho, sova te attapaṭilābho sacco, mogho atīto, mogho paccuppanno. Yo te etarahi paccuppanno attapaṭilābho, sova te attapaṭilābho sacco, mogho atīto, mogho anāgato’ti. Evaṃ puṭṭho ahaṃ, bhante, evaṃ byākareyyaṃ – ‘yo me ahosi atīto attapaṭilābho, sova me attapaṭilābho tasmiṃ samaye sacco ahosi, mogho anāgato, mogho paccuppanno. Yo me bhavissati anāgato attapaṭilābho, sova me attapaṭilābho tasmiṃ samaye sacco bhavissati, mogho atīto, mogho paccuppanno. Yo me etarahi paccuppanno attapaṭilābho, sova me attapaṭilābho sacco, mogho atīto, mogho anāgato’ti. Evaṃ puṭṭho ahaṃ, bhante, evaṃ byākareyya’’nti.

\paragraph{439.} ‘‘Evameva kho, citta, yasmiṃ samaye oḷāriko attapaṭilābho hoti, neva tasmiṃ samaye manomayo attapaṭilābhoti saṅkhaṃ gacchati, na arūpo attapaṭilābhoti saṅkhaṃ gacchati. Oḷāriko attapaṭilābho tveva tasmiṃ samaye saṅkhaṃ gacchati. Yasmiṃ, citta, samaye manomayo attapaṭilābho hoti…pe… yasmiṃ, citta, samaye arūpo attapaṭilābho hoti, neva tasmiṃ samaye oḷāriko attapaṭilābhoti saṅkhaṃ gacchati, na manomayo attapaṭilābhoti saṅkhaṃ gacchati; arūpo attapaṭilābho tveva tasmiṃ samaye saṅkhaṃ gacchati.

\paragraph{440.} ‘‘Seyyathāpi, citta, gavā khīraṃ, khīramhā dadhi, dadhimhā navanītaṃ, navanītamhā sappi, sappimhā sappimaṇḍo. Yasmiṃ samaye khīraṃ hoti, neva tasmiṃ samaye dadhīti saṅkhaṃ gacchati, na navanītanti saṅkhaṃ gacchati, na sappīti saṅkhaṃ gacchati, na sappimaṇḍoti saṅkhaṃ gacchati; khīraṃ tveva tasmiṃ samaye saṅkhaṃ gacchati. Yasmiṃ samaye dadhi hoti…pe… navanītaṃ hoti… sappi hoti… sappimaṇḍo hoti, neva tasmiṃ samaye khīranti saṅkhaṃ gacchati, na dadhīti saṅkhaṃ gacchati, na navanītanti saṅkhaṃ gacchati, na sappīti saṅkhaṃ gacchati; sappimaṇḍo tveva tasmiṃ samaye saṅkhaṃ gacchati. Evameva kho, citta, yasmiṃ samaye oḷāriko attapaṭilābho hoti… pe… yasmiṃ, citta, samaye manomayo attapaṭilābho hoti…pe… yasmiṃ, citta, samaye arūpo attapaṭilābho hoti, neva tasmiṃ samaye oḷāriko attapaṭilābhoti saṅkhaṃ gacchati, na manomayo attapaṭilābhoti saṅkhaṃ gacchati; arūpo attapaṭilābho tveva tasmiṃ samaye saṅkhaṃ gacchati. Imā kho citta, lokasamaññā lokaniruttiyo lokavohārā lokapaññattiyo, yāhi tathāgato voharati aparāmasa’’nti.

\paragraph{441.} Evaṃ vutte, poṭṭhapādo paribbājako bhagavantaṃ etadavoca – ‘‘abhikkantaṃ, bhante! Abhikkantaṃ, bhante, seyyathāpi, bhante, nikkujjitaṃ vā ukkujjeyya, paṭicchannaṃ vā vivareyya, mūḷhassa vā maggaṃ ācikkheyya, andhakāre vā telapajjotaṃ dhāreyya – ‘cakkhumanto rūpāni dakkhantī’ti. Evamevaṃ bhagavatā anekapariyāyena dhammo pakāsito. Esāhaṃ, bhante, bhagavantaṃ saraṇaṃ gacchāmi dhammañca bhikkhusaṅghañca. Upāsakaṃ maṃ bhagavā dhāretu ajjatagge pāṇupetaṃ saraṇaṃ gata’’nti.

\subsubsection{Cittahatthisāriputtaupasampadā}

\paragraph{442.} Citto pana hatthisāriputto bhagavantaṃ etadavoca – ‘‘abhikkantaṃ, bhante; abhikkantaṃ, bhante! Seyyathāpi, bhante, nikkujjitaṃ vā ukkujjeyya, paṭicchannaṃ vā vivareyya, mūḷhassa vā maggaṃ ācikkheyya, andhakāre vā telapajjotaṃ dhāreyya – ‘cakkhumanto rūpāni dakkhantī’ti. Evamevaṃ bhagavatā anekapariyāyena dhammo pakāsito. Esāhaṃ, bhante, bhagavantaṃ saraṇaṃ gacchāmi dhammañca bhikkhusaṅghañca. Labheyyāhaṃ, bhante, bhagavato santike pabbajjaṃ, labheyyaṃ upasampada’’nti.

\paragraph{443.} Alattha kho citto hatthisāriputto bhagavato santike pabbajjaṃ, alattha upasampadaṃ. Acirūpasampanno kho panāyasmā citto hatthisāriputto eko vūpakaṭṭho appamatto ātāpī pahitatto viharanto na cirasseva – yassatthāya kulaputtā sammadeva agārasmā anagāriyaṃ pabbajanti, tadanuttaraṃ – brahmacariyapariyosānaṃ diṭṭheva dhamme sayaṃ abhiññā sacchikatvā upasampajja vihāsi. ‘Khīṇā jāti, vusitaṃ brahmacariyaṃ, kataṃ karaṇīyaṃ, nāparaṃ itthattāyā’ti – abbhaññāsi. Aññataro kho panāyasmā citto hatthisāriputto arahataṃ ahosīti.

\xsectionEnd{Poṭṭhapādasuttaṃ niṭṭhitaṃ navamaṃ.}


\clearpage
\section{Subhasuttaṃ}

\subsubsection{Subhamāṇavavatthu}

\paragraph{444.} Evaṃ me sutaṃ – ekaṃ samayaṃ āyasmā ānando sāvatthiyaṃ viharati jetavane anāthapiṇḍikassa ārāme aciraparinibbute bhagavati. Tena kho pana samayena subho māṇavo todeyyaputto sāvatthiyaṃ paṭivasati kenacideva karaṇīyena.

\paragraph{445.} Atha kho subho māṇavo todeyyaputto aññataraṃ māṇavakaṃ āmantesi – ‘‘ehi tvaṃ, māṇavaka, yena samaṇo ānando tenupasaṅkama; upasaṅkamitvā mama vacanena samaṇaṃ ānandaṃ appābādhaṃ appātaṅkaṃ lahuṭṭhānaṃ balaṃ phāsuvihāraṃ puccha – ‘subho māṇavo todeyyaputto bhavantaṃ ānandaṃ appābādhaṃ appātaṅkaṃ lahuṭṭhānaṃ balaṃ phāsuvihāraṃ pucchatī’ti. Evañca vadehi – ‘sādhu kira bhavaṃ ānando yena subhassa māṇavassa todeyyaputtassa nivesanaṃ tenupasaṅkamatu anukampaṃ upādāyā’’’ti.

\paragraph{446.} ‘‘Evaṃ, bho’’ti kho so māṇavako subhassa māṇavassa todeyyaputtassa paṭissutvā yenāyasmā ānando tenupasaṅkami; upasaṅkamitvā āyasmatā ānandena saddhiṃ sammodi. Sammodanīyaṃ kathaṃ sāraṇīyaṃ vītisāretvā ekamantaṃ nisīdi. Ekamantaṃ nisinno kho so māṇavako āyasmantaṃ ānandaṃ etadavoca – ‘‘subho māṇavo todeyyaputto bhavantaṃ ānandaṃ appābādhaṃ appātaṅkaṃ lahuṭṭhānaṃ balaṃ phāsuvihāraṃ pucchati; evañca vadeti – ‘sādhu kira bhavaṃ ānando yena subhassa māṇavassa todeyyaputtassa nivesanaṃ tenupasaṅkamatu anukampaṃ upādāyā’’’ti.

\paragraph{447.} Evaṃ vutte, āyasmā ānando taṃ māṇavakaṃ etadavoca – ‘‘akālo kho, māṇavaka. Atthi me ajja bhesajjamattā pītā. Appevanāma svepi upasaṅkameyyāma kālañca samayañca upādāyā’’ti. ‘‘Evaṃ, bho’’ti kho so māṇavako āyasmato ānandassa paṭissutvā uṭṭhāyāsanā yena subho māṇavo todeyyaputto tenupasaṅkami; upasaṅkamitvā subhaṃ māṇavaṃ todeyyaputtaṃ etadavoca, ‘‘avocumhā kho mayaṃ bhoto vacanena taṃ bhavantaṃ ānandaṃ – ‘subho māṇavo todeyyaputto bhavantaṃ ānandaṃ appābādhaṃ appātaṅkaṃ lahuṭṭhānaṃ balaṃ phāsuvihāraṃ pucchati, evañca vadeti – ‘‘sādhu kira bhavaṃ ānando yena subhassa māṇavassa todeyyaputtassa nivesanaṃ tenupasaṅkamatu anukampaṃ upādāyā’’’ti. Evaṃ vutte, bho, samaṇo ānando maṃ etadavoca – ‘akālo kho, māṇavaka. Atthi me ajja bhesajjamattā pītā. Appevanāma svepi upasaṅkameyyāma kālañca samayañca upādāyā’ti. Ettāvatāpi kho, bho, katameva etaṃ, yato kho so bhavaṃ ānando okāsamakāsi svātanāyapi upasaṅkamanāyā’’ti.

\paragraph{448.} Atha kho āyasmā ānando tassā rattiyā accayena pubbaṇhasamayaṃ nivāsetvā pattacīvaramādāya cetakena bhikkhunā pacchāsamaṇena yena subhassa māṇavassa todeyyaputtassa nivesanaṃ tenupasaṅkami; upasaṅkamitvā paññatte āsane nisīdi. Atha kho subho māṇavo todeyyaputto yenāyasmā ānando tenupasaṅkami; upasaṅkamitvā āyasmatā ānandena saddhiṃ sammodi. Sammodanīyaṃ kathaṃ sāraṇīyaṃ vītisāretvā ekamantaṃ nisīdi. Ekamantaṃ nisinno kho subho māṇavo todeyyaputto āyasmantaṃ ānandaṃ etadavoca – ‘‘bhavañhi ānando tassa bhoto gotamassa dīgharattaṃ upaṭṭhāko santikāvacaro samīpacārī. Bhavametaṃ ānando jāneyya, yesaṃ so bhavaṃ gotamo dhammānaṃ vaṇṇavādī ahosi, yattha ca imaṃ janataṃ samādapesi nivesesi patiṭṭhāpesi. Katamesānaṃ kho, bho ānanda, dhammānaṃ so bhavaṃ gotamo vaṇṇavādī ahosi; kattha ca imaṃ janataṃ samādapesi nivesesi patiṭṭhāpesī’’ti?

\paragraph{449.} ‘‘Tiṇṇaṃ kho, māṇava, khandhānaṃ so bhagavā vaṇṇavādī ahosi; ettha ca imaṃ janataṃ samādapesi nivesesi patiṭṭhāpesi. Katamesaṃ tiṇṇaṃ? Ariyassa sīlakkhandhassa, ariyassa samādhikkhandhassa, ariyassa paññākkhandhassa. Imesaṃ kho, māṇava, tiṇṇaṃ khandhānaṃ so bhagavā vaṇṇavādī ahosi; ettha ca imaṃ janataṃ samādapesi nivesesi patiṭṭhāpesī’’ti.

\subsubsection{Sīlakkhandho}

\paragraph{450.} ‘‘Katamo pana so, bho ānanda, ariyo sīlakkhandho, yassa so bhavaṃ gotamo vaṇṇavādī ahosi, yattha ca imaṃ janataṃ samādapesi nivesesi patiṭṭhāpesī’’ti? ‘‘Idha, māṇava, tathāgato loke uppajjati arahaṃ sammāsambuddho vijjācaraṇasampanno sugato lokavidū anuttaro purisadammasārathi satthā devamanussānaṃ buddho bhagavā. So imaṃ lokaṃ sadevakaṃ samārakaṃ sabrahmakaṃ sassamaṇabrāhmaṇiṃ pajaṃ sadevamanussaṃ sayaṃ abhiññā sacchikatvā pavedeti. So dhammaṃ deseti ādikalyāṇaṃ majjhekalyāṇaṃ pariyosānakalyāṇaṃ sātthaṃ sabyañjanaṃ kevalaparipuṇṇaṃ parisuddhaṃ brahmacariyaṃ pakāseti. Taṃ dhammaṃ suṇāti gahapati vā gahapatiputto vā aññatarasmiṃ vā kule paccājāto. So taṃ dhammaṃ sutvā tathāgate saddhaṃ paṭilabhati. So tena saddhāpaṭilābhena samannāgato iti paṭisañcikkhati – ‘sambādho gharāvāso rajopatho, abbhokāso pabbajjā, nayidaṃ sukaraṃ agāraṃ ajjhāvasatā ekantaparipuṇṇaṃ ekantaparisuddhaṃ saṅkhalikhitaṃ brahmacariyaṃ carituṃ. Yaṃnūnāhaṃ kesamassuṃ ohāretvā kāsāyāni vatthāni acchādetvā agārasmā anagāriyaṃ pabbajeyya’nti. So aparena samayena appaṃ vā bhogakkhandhaṃ pahāya mahantaṃ vā bhogakkhandhaṃ pahāya appaṃ vā ñātiparivaṭṭaṃ pahāya mahantaṃ vā ñātiparivaṭṭaṃ pahāya kesamassuṃ ohāretvā kāsāyāni vatthāni acchādetvā agārasmā anagāriyaṃ pabbajati. So evaṃ pabbajito samāno pātimokkhasaṃvarasaṃvuto viharati, ācāragocarasampanno, anumattesu vajjesu bhayadassāvī, samādāya sikkhati sikkhāpadesu, kāyakammavacīkammena samannāgato kusalena, parisuddhājīvo, sīlasampanno, indriyesu guttadvāro, satisampajaññena samannāgato, santuṭṭho.

\paragraph{451.} ‘‘Kathañca, māṇava, bhikkhu sīlasampanno hoti? Idha, māṇava, bhikkhu pāṇātipātaṃ pahāya pāṇātipātā paṭivirato hoti, nihitadaṇḍo nihitasattho lajjī dayāpanno, sabbapāṇabhūtahitānukampī viharati. Yampi, māṇava, bhikkhu pāṇātipātaṃ pahāya pāṇātipātā paṭivirato hoti, nihitadaṇḍo nihitasattho lajjī dayāpanno, sabbapāṇabhūtahitānukampī viharati; idampissa hoti sīlasmiṃ. (Yathā 194 yāva 210 anucchedesu evaṃ vitthāretabbaṃ). ‘‘Yathā vā paneke bhonto samaṇabrāhmaṇā saddhādeyyāni bhojanāni bhuñjitvā te evarūpāya tiracchānavijjāya micchājīvena jīvitaṃ kappenti, seyyathidaṃ – santikammaṃ paṇidhikammaṃ bhūtakammaṃ bhūrikammaṃ vassakammaṃ vossakammaṃ vatthukammaṃ vatthuparikammaṃ ācamanaṃ nhāpanaṃ juhanaṃ vamanaṃ virecanaṃ uddhaṃvirecanaṃ adhovirecanaṃ sīsavirecanaṃ kaṇṇatelaṃ nettatappanaṃ natthukammaṃ añjanaṃ paccañjanaṃ sālākiyaṃ sallakattiyaṃ dārakatikicchā mūlabhesajjānaṃ anuppadānaṃ osadhīnaṃ paṭimokkho iti vā iti evarūpāya tiracchānavijjāya micchājīvā paṭivirato hoti. Yampi, māṇava, bhikkhu yathā vā paneke bhonto samaṇabrāhmaṇā saddhādeyyāni bhojanāni bhuñjitvā te evarūpāya tiracchānavijjāya micchājīvena jīvitaṃ kappenti, seyyathidaṃ, santikammaṃ paṇidhikammaṃ…pe… osadhīnaṃ paṭimokkho iti vā iti evarūpāya tiracchānavijjāya micchājīvā paṭivirato hoti. Idampissa hoti sīlasmiṃ.

\paragraph{452.} ‘‘Sa kho so\footnote{ayaṃ kho so (ka.)}, māṇava, bhikkhu evaṃ sīlasampanno na kutoci bhayaṃ samanupassati, yadidaṃ sīlasaṃvarato. Seyyathāpi, māṇava, rājā khattiyo muddhāvasitto nihatapaccāmitto na kutoci bhayaṃ samanupassati, yadidaṃ paccatthikato. Evameva kho, māṇava, bhikkhu evaṃ sīlasampanno na kutoci bhayaṃ samanupassati, yadidaṃ sīlasaṃvarato. So iminā ariyena sīlakkhandhena samannāgato ajjhattaṃ anavajjasukhaṃ paṭisaṃvedeti. Evaṃ kho, māṇava, bhikkhu sīlasampanno hoti.

\paragraph{453.} ‘‘Ayaṃ kho so, māṇava, ariyo sīlakkhandho yassa so bhagavā vaṇṇavādī ahosi, yattha ca imaṃ janataṃ samādapesi nivesesi patiṭṭhāpesi. Atthi cevettha uttarikaraṇīya’’nti. ‘‘Acchariyaṃ, bho ānanda, abbhutaṃ, bho ānanda! So cāyaṃ, bho ānanda, ariyo sīlakkhandho paripuṇṇo, no aparipuṇṇo. Evaṃ paripuṇṇaṃ cāhaṃ, bho, ānanda, ariyaṃ sīlakkhandhaṃ ito bahiddhā aññesu samaṇabrāhmaṇesu na samanupassāmi. Evaṃ paripuṇṇañca, bho ānanda, ariyaṃ sīlakkhandhaṃ ito bahiddhā aññe samaṇabrāhmaṇā attani samanupasseyyuṃ, te tāvatakeneva attamanā assu – ‘alamettāvatā, katamettāvatā, anuppatto no sāmaññattho, natthi no kiñci uttarikaraṇīya’nti. Atha ca pana bhavaṃ ānando evamāha – ‘atthi cevettha uttarikaraṇīya’’’nti\footnote{imassa anantaraṃ sī. pī. potthakesu ‘‘paṭhamabhāṇavāraṃ’’ti pāṭho dissati}.

\subsubsection{Samādhikkhandho}

\paragraph{454.} ‘‘Katamo pana so, bho ānanda, ariyo samādhikkhandho, yassa so bhavaṃ gotamo vaṇṇavādī ahosi, yattha ca imaṃ janataṃ samādapesi nivesesi patiṭṭhāpesī’’ti? ‘‘Kathañca, māṇava, bhikkhu indriyesu guttadvāro hoti? Idha, māṇava, bhikkhu cakkhunā rūpaṃ disvā na nimittaggāhī hoti nānubyañjanaggāhī; yatvādhikaraṇamenaṃ cakkhundriyaṃ asaṃvutaṃ viharantaṃ abhijjhādomanassā pāpakā akusalā dhammā anvāssaveyyuṃ tassa saṃvarāya paṭipajjati, rakkhati cakkhundriyaṃ, cakkhundriye saṃvaraṃ āpajjati. Sotena saddaṃ sutvā…pe… ghānena gandhaṃ ghāyitvā… jivhāya rasaṃ sāyitvā… kāyena phoṭṭhabbaṃ phusitvā… manasā dhammaṃ viññāya na nimittaggāhī hoti nānubyañjanaggāhī; yatvādhikaraṇamenaṃ manindriyaṃ asaṃvutaṃ viharantaṃ abhijjhādomanassā pāpakā akusalā dhammā anvāssaveyyuṃ tassa saṃvarāya paṭipajjati, rakkhati manindriyaṃ, manindriye saṃvaraṃ āpajjati. So iminā ariyena indriyasaṃvarena samannāgato ajjhattaṃ abyāsekasukhaṃ paṭisaṃvedeti. Evaṃ kho, māṇava, bhikkhu indriyesu guttadvāro hoti.

\paragraph{455.} ‘‘Kathañca, māṇava, bhikkhu satisampajaññena samannāgato hoti? Idha, māṇava, bhikkhu abhikkante paṭikkante sampajānakārī hoti, ālokite vilokite sampajānakārī hoti, samiñjite pasārite sampajānakārī hoti, saṅghāṭipattacīvaradhāraṇe sampajānakārī hoti, asite pīte khāyite sāyite sampajānakārī hoti, uccārapassāvakamme sampajānakārī hoti, gate ṭhite nisinne sutte jāgarite bhāsite tuṇhībhāve sampajānakārī hoti. Evaṃ kho, māṇava, bhikkhu satisampajaññena samannāgato hoti.

\paragraph{456.} ‘‘Kathañca, māṇava, bhikkhu santuṭṭho hoti? Idha, māṇava, bhikkhu santuṭṭho hoti kāyaparihārikena cīvarena kucchiparihārikena piṇḍapātena. So yena yeneva pakkamati, samādāyeva pakkamati. Seyyathāpi, māṇava, pakkhī sakuṇo yena yeneva ḍeti, sapattabhārova ḍeti; evameva kho, māṇava, bhikkhu santuṭṭho hoti kāyaparihārikena cīvarena kucchiparihārikena piṇḍapātena. So yena yeneva pakkamati, samādāyeva pakkamati. Evaṃ kho, māṇava, bhikkhu santuṭṭho hoti.

\paragraph{457.} ‘‘So iminā ca ariyena sīlakkhandhena samannāgato, iminā ca ariyena indriyasaṃvarena samannāgato, iminā ca ariyena satisampajaññena samannāgato, imāya ca ariyāya santuṭṭhiyā samannāgato vivittaṃ senāsanaṃ bhajati araññaṃ rukkhamūlaṃ pabbataṃ kandaraṃ giriguhaṃ susānaṃ vanapatthaṃ abbhokāsaṃ palālapuñjaṃ. So pacchābhattaṃ piṇḍapātappaṭikkanto nisīdati pallaṅkaṃ ābhujitvā, ujuṃ kāyaṃ paṇidhāya, parimukhaṃ satiṃ upaṭṭhapetvā.

\paragraph{458.} ‘‘So abhijjhaṃ loke pahāya vigatābhijjhena cetasā viharati abhijjhāya cittaṃ parisodheti. Byāpādapadosaṃ pahāya abyāpannacitto viharati sabbapāṇabhūtahitānukampī byāpādapadosā cittaṃ parisodheti. Thinamiddhaṃ pahāya vigatathinamiddho viharati ālokasaññī sato sampajāno, thinamiddhā cittaṃ parisodheti. Uddhaccakukkuccaṃ pahāya anuddhato viharati ajjhattaṃ vūpasantacitto uddhaccakukkuccā cittaṃ parisodheti. Vicikicchaṃ pahāya tiṇṇavicikiccho viharati akathaṃkathī kusalesu dhammesu, vicikicchāya cittaṃ parisodheti.

\paragraph{459.} ‘‘Seyyathāpi, māṇava, puriso iṇaṃ ādāya kammante payojeyya. Tassa te kammantā samijjheyyuṃ. So yāni ca porāṇāni iṇamūlāni tāni ca byantiṃ kareyya, siyā cassa uttariṃ avasiṭṭhaṃ dārabharaṇāya. Tassa evamassa – ‘ahaṃ kho pubbe iṇaṃ ādāya kammante payojesiṃ. Tassa me te kammantā samijjhiṃsu. Sohaṃ yāni ca porāṇāni iṇamūlāni tāni ca byantiṃ akāsiṃ, atthi ca me uttariṃ avasiṭṭhaṃ dārabharaṇāyā’ti. So tatonidānaṃ labhetha pāmojjaṃ, adhigaccheyya somanassaṃ.

\paragraph{460.} ‘‘Seyyathāpi, māṇava, puriso ābādhiko assa dukkhito bāḷhagilāno; bhattañcassa nacchādeyya, na cassa kāye balamattā. So aparena samayena tamhā ābādhā mucceyya, bhattañcassa chādeyya, siyā cassa kāye balamattā. Tassa evamassa – ‘ahaṃ kho pubbe ābādhiko ahosiṃ dukkhito bāḷhagilāno, bhattañca me nacchādesi, na ca me āsi kāye balamattā. Somhi etarahi tamhā ābādhā mutto bhattañca me chādeti, atthi ca me kāye balamattā’ti. So tatonidānaṃ labhetha pāmojjaṃ, adhigaccheyya somanassaṃ.

\paragraph{461.} ‘‘Seyyathāpi, māṇava, puriso bandhanāgāre baddho assa. So aparena samayena tamhā bandhanāgārā mucceyya sotthinā abbhayena, na cassa kiñci bhogānaṃ vayo. Tassa evamassa – ‘ahaṃ kho pubbe bandhanāgāre baddho ahosiṃ. Somhi etarahi tamhā bandhanāgārā mutto sotthinā abbhayena, natthi ca me kiñci bhogānaṃ vayo’ti. So tatonidānaṃ labhetha pāmojjaṃ, adhigaccheyya somanassaṃ.

\paragraph{462.} ‘‘Seyyathāpi, māṇava, puriso dāso assa anattādhīno parādhīno na yenakāmaṃgamo. So aparena samayena tamhā dāsabyā mucceyya, attādhīno aparādhīno bhujisso yenakāmaṃgamo. Tassa evamassa – ‘ahaṃ kho pubbe dāso ahosiṃ anattādhīno parādhīno na yenakāmaṃgamo. Somhi etarahi tamhā dāsabyā mutto attādhīno aparādhīno bhujisso yenakāmaṃgamo’ti. So tatonidānaṃ labhetha pāmojjaṃ, adhigaccheyya somanassaṃ.

\paragraph{463.} ‘‘Seyyathāpi, māṇava, puriso sadhano sabhogo kantāraddhānamaggaṃ paṭipajjeyya dubbhikkhaṃ sappaṭibhayaṃ. So aparena samayena taṃ kantāraṃ nitthareyya, sotthinā gāmantaṃ anupāpuṇeyya khemaṃ appaṭibhayaṃ. Tassa evamassa – ‘ahaṃ kho pubbe sadhano sabhogo kantāraddhānamaggaṃ paṭipajjiṃ dubbhikkhaṃ sappaṭibhayaṃ. Somhi etarahi kantāraṃ nitthiṇṇo, sotthinā gāmantaṃ anuppatto khemaṃ appaṭibhaya’nti. So tatonidānaṃ labhetha pāmojjaṃ, adhigaccheyya somanassaṃ.

\paragraph{464.} ‘‘Evameva kho, māṇava, bhikkhu yathā iṇaṃ yathā rogaṃ yathā bandhanāgāraṃ yathā dāsabyaṃ yathā kantāraddhānamaggaṃ, evaṃ ime pañca nīvaraṇe appahīne attani samanupassati.

\paragraph{465.} ‘‘Seyyathāpi, māṇava, yathā āṇaṇyaṃ yathā ārogyaṃ yathā bandhanāmokkhaṃ yathā bhujissaṃ yathā khemantabhūmiṃ. Evameva bhikkhu ime pañca nīvaraṇe pahīne attani samanupassati.

\paragraph{466.} ‘‘Tassime pañca nīvaraṇe pahīne attani samanupassato pāmojjaṃ jāyati, pamuditassa pīti jāyati, pītimanassa kāyo passambhati, passaddhakāyo sukhaṃ vedeti, sukhino cittaṃ samādhiyati.

\paragraph{467.} ‘‘So vivicceva kāmehi vivicca akusalehi dhammehi savitakkaṃ savicāraṃ vivekajaṃ pītisukhaṃ paṭhamaṃ jhānaṃ upasampajja viharati. So imameva kāyaṃ vivekajena pītisukhena abhisandeti parisandeti paripūreti parippharati, nāssa kiñci sabbāvato kāyassa vivekajena pītisukhena apphuṭaṃ hoti. ‘‘Seyyathāpi, māṇava, dakkho nhāpako vā nhāpakantevāsī vā kaṃsathāle nhānīyacuṇṇāni ākiritvā udakena paripphosakaṃ paripphosakaṃ sandeyya. Sāyaṃ nhānīyapiṇḍi snehānugatā snehaparetā santarabāhirā phuṭā snehena, na ca paggharaṇī. Evameva kho, māṇava, bhikkhu imameva kāyaṃ vivekajena pītisukhena abhisandeti parisandeti paripūreti parippharati, nāssa kiñci sabbāvato kāyassa vivekajena pītisukhena apphuṭaṃ hoti. Yampi, māṇava, bhikkhu vivicceva kāmehi vivicca akusalehi dhammehi savitakkaṃ savicāraṃ vivekajaṃ pītisukhaṃ paṭhamaṃ jhānaṃ upasampajja viharati. So imameva kāyaṃ vivekajena pītisukhena abhisandeti parisandeti paripūreti parippharati, nāssa kiñci sabbāvato kāyassa vivekajena pītisukhena apphuṭaṃ hoti. Idampissa hoti samādhismiṃ.

\paragraph{468.} ‘‘Puna caparaṃ, māṇava, bhikkhu vitakkavicārānaṃ vūpasamā ajjhattaṃ sampasādanaṃ cetaso ekodibhāvaṃ avitakkaṃ avicāraṃ samādhijaṃ pītisukhaṃ dutiyaṃ jhānaṃ upasampajja viharati. So imameva kāyaṃ samādhijena pītisukhena abhisandeti parisandeti paripūreti parippharati, nāssa kiñci sabbāvato kāyassa samādhijena pītisukhena apphuṭaṃ hoti. ‘‘Seyyathāpi, māṇava, udakarahado gambhīro ubbhidodako. Tassa nevassa puratthimāya disāya udakassa āyamukhaṃ, na dakkhiṇāya disāya udakassa āyamukhaṃ, na pacchimāya disāya udakassa āyamukhaṃ, na uttarāya disāya udakassa āyamukhaṃ, devo ca na kālena kālaṃ sammā dhāraṃ anupaveccheyya. Atha kho tamhāva udakarahadā sītā vāridhārā ubbhijjitvā tameva udakarahadaṃ sītena vārinā abhisandeyya parisandeyya paripūreyya paripphareyya, nāssa kiñci sabbāvato udakarahadassa sītena vārinā apphuṭaṃ assa. Evameva kho, māṇava, bhikkhu…pe… yampi, māṇava, bhikkhu vitakkavicārānaṃ vūpasamā… pe… dutiyaṃ jhānaṃ upasampajja viharati, so imameva kāyaṃ samādhijena pītisukhena abhisandeti parisandeti paripūreti parippharati, nāssa kiñci sabbāvato kāyassa samādhijena pītisukhena apphuṭaṃ hoti. Idampissa hoti samādhismiṃ.

\paragraph{469.} ‘‘Puna caparaṃ, māṇava, bhikkhu pītiyā ca virāgā upekkhako ca viharati sato sampajāno, sukhañca kāyena paṭisaṃvedeti, yaṃ taṃ ariyā ācikkhanti – ‘‘upekkhako satimā sukhavihārī’’ti, tatiyaṃ jhānaṃ upasampajja viharati. So imameva kāyaṃ nippītikena sukhena abhisandeti parisandeti paripūreti parippharati, nāssa kiñci sabbāvato kāyassa nippītikena sukhena apphuṭaṃ hoti. ‘‘Seyyathāpi, māṇava, uppaliniyaṃ vā paduminiyaṃ vā puṇḍarīkiniyaṃ vā appekaccāni uppalāni vā padumāni vā puṇḍarīkāni vā udake jātāni udake saṃvaḍḍhāni udakānuggatāni antonimuggaposīni, tāni yāva caggā yāva ca mūlā sītena vārinā abhisannāni parisannāni paripūrāni paripphuṭāni, nāssa kiñci sabbāvataṃ uppalānaṃ vā padumānaṃ vā puṇḍarīkānaṃ vā sītena vārinā apphuṭaṃ assa. Evameva kho, māṇava, bhikkhu…pe… yampi, māṇava, bhikkhu pītiyā ca virāgā…pe… tatiyaṃ jhānaṃ upasampajja viharati. So imameva kāyaṃ nippītikena sukhena abhisandeti parisandeti paripūreti parippharati, nāssa kiñci sabbāvato kāyassa nippītikena sukhena apphuṭaṃ hoti. Idampissa hoti samādhismiṃ.

\paragraph{470.} ‘‘Puna caparaṃ, māṇava, bhikkhu sukhassa ca pahānā dukkhassa ca pahānā pubbeva somanassadomanassānaṃ atthaṅgamā adukkhamasukhaṃ upekkhāsatipārisuddhiṃ catutthaṃ jhānaṃ upasampajja viharati. So imameva kāyaṃ parisuddhena cetasā pariyodātena pharitvā nisinno hoti; nāssa kiñci sabbāvato kāyassa parisuddhena cetasā pariyodātena apphuṭaṃ hoti. ‘‘Seyyathāpi, māṇava, puriso odātena vatthena sasīsaṃ pārupitvā nisinno assa, nāssa kiñci sabbāvato kāyassa odātena vatthena apphuṭaṃ assa. Evameva kho, māṇava, bhikkhu…pe… yampi, māṇava, bhikkhu sukhassa ca pahānā dukkhassa ca pahānā pubbeva somanassadomanassānaṃ atthaṅgamā adukkhamasukhaṃ upekkhāsatipārisuddhiṃ catutthaṃ jhānaṃ upasampajja viharati. So imameva kāyaṃ parisuddhena cetasā pariyodātena pharitvā nisinno hoti; nāssa kiñci sabbāvato kāyassa parisuddhena cetasā pariyodātena apphuṭaṃ hoti. Idampissa hoti samādhismiṃ.

\paragraph{471.} ‘‘Ayaṃ kho so, māṇava, ariyo samādhikkhandho yassa so bhagavā vaṇṇavādī ahosi, yattha ca imaṃ janataṃ samādapesi nivesesi patiṭṭhāpesi. Atthi cevettha uttarikaraṇīya’’nti. ‘‘Acchariyaṃ, bho ānanda, abbhutaṃ, bho ānanda! So cāyaṃ, bho ānanda, ariyo samādhikkhandho paripuṇṇo, no aparipuṇṇo. Evaṃ paripuṇṇaṃ cāhaṃ, bho ānanda, ariyaṃ samādhikkhandhaṃ ito bahiddhā aññesu samaṇabrāhmaṇesu na samanupassāmi. Evaṃ paripuṇṇañca, bho ānanda, ariyaṃ samādhikkhandhaṃ ito bahiddhā aññe samaṇabrāhmaṇā attani samanupasseyyuṃ, te tāvatakeneva attamanā assu – ‘alamettāvatā, katamettāvatā, anuppatto no sāmaññattho, natthi no kiñci uttarikaraṇīya’nti. Atha ca pana bhavaṃ ānando evamāha – ‘atthi cevettha uttarikaraṇīya’’’nti.

\subsubsection{Paññākkhandho}

\paragraph{472.} ‘‘Katamo pana so, bho ānanda, ariyo paññākkhandho, yassa bho bhavaṃ gotamo vaṇṇavādī ahosi, yattha ca imaṃ janataṃ samādapesi nivesesi patiṭṭhāpesī’’ti? ‘‘So evaṃ samāhite citte parisuddhe pariyodāte anaṅgaṇe vigatūpakkilese mudubhūte kammaniye ṭhite āneñjappatte ñāṇadassanāya cittaṃ abhinīharati abhininnāmeti. So evaṃ pajānāti – ‘ayaṃ kho me kāyo rūpī cātumahābhūtiko mātāpettikasambhavo odanakummāsūpacayo aniccucchādanaparimaddanabhedanaviddhaṃsanadhammo; idañca pana me viññāṇaṃ ettha sitaṃ ettha paṭibaddha’nti. ‘‘Seyyathāpi, māṇava, maṇi veḷuriyo subho jātimā aṭṭhaṃso suparikammakato accho vippasanno anāvilo sabbākārasampanno. Tatrāssa suttaṃ āvutaṃ nīlaṃ vā pītaṃ vā lohitaṃ vā odātaṃ vā paṇḍusuttaṃ vā. Tamenaṃ cakkhumā puriso hatthe karitvā paccavekkheyya – ‘ayaṃ kho maṇi veḷuriyo subho jātimā aṭṭhaṃso suparikammakato accho vippasanno anāvilo sabbākārasampanno. Tatridaṃ suttaṃ āvutaṃ nīlaṃ vā pītaṃ vā lohitaṃ vā odātaṃ vā paṇḍusuttaṃ vā’ti. Evameva kho, māṇava, bhikkhu evaṃ samāhite citte parisuddhe pariyodāte anaṅgaṇe vigatūpakkilese mudubhūte kammaniye ṭhite āneñjappatte ñāṇadassanāya cittaṃ abhinīharati abhininnāmeti. So evaṃ pajānāti – ‘ayaṃ kho me kāyo rūpī cātumahābhūtiko mātāpettikasambhavo odanakummāsūpacayo aniccucchādanaparimaddanabhedana-viddhaṃsanadhammo. Idañca pana me viññāṇaṃ ettha sitaṃ ettha paṭibaddha’nti. Yampi, māṇava, bhikkhu evaṃ samāhite citte…pe… āneñjappatte ñāṇadassanāya cittaṃ abhinīharati abhininnāmeti. So evaṃ pajānāti…pe… ettha paṭibaddhanti. Idampissa hoti paññāya.

\paragraph{473.} ‘‘So evaṃ samāhite citte parisuddhe pariyodāte anaṅgaṇe vigatūpakkilese mudubhūte kammaniye ṭhite āneñjappatte manomayaṃ kāyaṃ abhinimmānāya cittaṃ abhinīharati abhininnāmeti. So imamhā kāyā aññaṃ kāyaṃ abhinimmināti rūpiṃ manomayaṃ sabbaṅgapaccaṅgiṃ ahīnindriyaṃ. ‘‘Seyyathāpi, māṇava, puriso muñjamhā īsikaṃ pavāheyya. Tassa evamassa – ‘ayaṃ muñjo ayaṃ īsikā; añño muñjo aññā īsikā; muñjamhā tveva īsikā pavāḷhā’ti. Seyyathā vā pana, māṇava, puriso asiṃ kosiyā pavāheyya. Tassa evamassa – ‘ayaṃ asi, ayaṃ kosi; añño asi, aññā kosi; kosiyā tveva asi pavāḷho’ti. Seyyathā vā pana, māṇava, puriso ahiṃ karaṇḍā uddhareyya. Tassa evamassa – ‘ayaṃ ahi, ayaṃ karaṇḍo; añño ahi, añño karaṇḍo; karaṇḍā tveva ahi ubbhato’ti. Evameva kho, māṇava, bhikkhu…pe… yampi, māṇava, bhikkhu evaṃ samāhite citte parisuddhe pariyodāte anaṅgaṇe vigatūpakkilese mudubhūte kammaniye ṭhite āneñjappatte manomayaṃ kāyaṃ abhinimmānāya cittaṃ abhinīharati abhininnāmeti…pe…. Idampissa hoti paññāya.

\paragraph{474.} ‘‘So evaṃ samāhite citte parisuddhe pariyodāte anaṅgaṇe vigatūpakkilese mudubhūte kammaniye ṭhite āneñjappatte iddhividhāya cittaṃ abhinīharati abhininnāmeti. So anekavihitaṃ iddhividhaṃ paccanubhoti. Ekopi hutvā bahudhā hoti, bahudhāpi hutvā eko hoti. Āvibhāvaṃ tirobhāvaṃ tirokuṭṭaṃ tiropākāraṃ tiropabbataṃ asajjamāno gacchati seyyathāpi ākāse. Pathaviyāpi ummujjanimujjaṃ karoti, seyyathāpi udake. Udakepi abhijjamāne gacchati seyyathāpi pathaviyaṃ. Ākāsepi pallaṅkena kamati seyyathāpi pakkhī sakuṇo. Imepi candimasūriye evaṃ mahiddhike evaṃ mahānubhāve pāṇinā parāmasati parimajjati. Yāva brahmalokāpi kāyena vasaṃ vatteti. ‘‘Seyyathāpi, māṇava, dakkho kumbhakāro vā kumbhakārantevāsī vā suparikammakatāya mattikāya yaññadeva bhājanavikatiṃ ākaṅkheyya, taṃ tadeva kareyya abhinipphādeyya. Seyyathā vā pana, māṇava, dakkho dantakāro vā dantakārantevāsī vā suparikammakatasmiṃ dantasmiṃ yaññadeva dantavikatiṃ ākaṅkheyya, taṃ tadeva kareyya abhinipphādeyya. Seyyathā vā pana, māṇava, dakkho suvaṇṇakāro vā suvaṇṇakārantevāsī vā suparikammakatasmiṃ suvaṇṇasmiṃ yaññadeva suvaṇṇavikatiṃ ākaṅkheyya, taṃ tadeva kareyya abhinipphādeyya. Evameva kho, māṇava, bhikkhu …pe… yampi māṇava bhikkhu evaṃ samāhite citte parisuddhe pariyodāte anaṅgaṇe vigatūpakkilese mudubhūte kammaniye ṭhite āneñjappatte iddhividhāya cittaṃ abhinīharati abhininnāmeti. So anekavihitaṃ iddhividhaṃ paccanubhoti. Ekopi hutvā bahudhā hoti … pe… yāva brahmalokāpi kāyena vasaṃ vatteti. Idampissa hoti paññāya.

\paragraph{475.} ‘‘So evaṃ samāhite citte…pe… āneñjappatte dibbāya sotadhātuyā cittaṃ abhinīharati abhininnāmeti. So dibbāya sotadhātuyā visuddhāya atikkantamānusikāya ubho sadde suṇāti dibbe ca mānuse ca ye dūre santike ca. Seyyathāpi, māṇava, puriso addhānamaggappaṭipanno. So suṇeyya bherisaddampi mudiṅgasaddampi saṅkhapaṇavadindimasaddampi. Tassa evamassa – ‘bherisaddo itipi mudiṅgasaddo itipi saṅkhapaṇavadindimasaddo iti’pi\footnote{itipīti (ka.)}. Evameva kho, māṇava, bhikkhu…pe…. Yampi māṇava, bhikkhu evaṃ samāhite citte…pe… āneñjappatte dibbāya sotadhātuyā cittaṃ abhinīharati abhininnāmeti. So dibbāya sotadhātuyā visuddhāya atikkantamānusikāya ubho sadde suṇāti dibbe ca mānuse ca ye dūre santike ca. Idampissa hoti paññāya.

\paragraph{476.} ‘‘So evaṃ samāhite citte parisuddhe pariyodāte anaṅgaṇe vigatūpakkilese mudubhūte kammaniye ṭhite āneñjappatte cetopariyañāṇāya cittaṃ abhinīharati abhininnāmeti. So parasattānaṃ parapuggalānaṃ cetasā ceto paricca pajānāti, ‘sarāgaṃ vā cittaṃ sarāgaṃ citta’nti pajānāti, ‘vītarāgaṃ vā cittaṃ vītarāgaṃ citta’nti pajānāti, ‘sadosaṃ vā cittaṃ sadosaṃ citta’nti pajānāti, ‘vītadosaṃ vā cittaṃ vītadosaṃ citta’nti pajānāti, ‘samohaṃ vā cittaṃ samohaṃ citta’nti pajānāti, ‘vītamohaṃ vā cittaṃ vītamohaṃ citta’nti pajānāti, ‘saṅkhittaṃ vā cittaṃ saṅkhittaṃ citta’nti pajānāti, ‘vikkhittaṃ vā cittaṃ vikkhittaṃ citta’nti pajānāti, ‘mahaggataṃ vā cittaṃ mahaggataṃ citta’nti pajānāti, ‘amahaggataṃ vā cittaṃ amahaggataṃ citta’nti pajānāti, ‘sauttaraṃ vā cittaṃ sauttaraṃ citta’nti pajānāti, ‘anuttaraṃ vā cittaṃ anuttaraṃ citta’nti pajānāti, ‘samāhitaṃ vā cittaṃ samāhitaṃ citta’nti pajānāti, ‘asamāhitaṃ vā cittaṃ asamāhitaṃ citta’nti pajānāti, ‘vimuttaṃ vā cittaṃ vimuttaṃ citta’nti pajānāti, ‘avimuttaṃ vā cittaṃ avimuttaṃ citta’nti pajānāti. ‘‘Seyyathāpi, māṇava, itthī vā puriso vā daharo yuvā maṇḍanajātiko ādāse vā parisuddhe pariyodāte acche vā udakapatte sakaṃ mukhanimittaṃ paccavekkhamāno sakaṇikaṃ vā sakaṇikanti jāneyya, akaṇikaṃ vā akaṇikanti jāneyya. Evameva kho, māṇava, bhikkhu…pe… yampi, māṇava, bhikkhu evaṃ samāhite…pe… āneñjappatte cetopariyañāṇāya cittaṃ abhinīharati abhininnāmeti. So parasattānaṃ purapuggalānaṃ cetasā ceto paricca pajānāti, sarāgaṃ vā cittaṃ sarāgaṃ cittanti pajānāti…pe… avimuttaṃ vā cittaṃ avimuttaṃ cittanti pajānāti. Idampissa hoti paññāya.

\paragraph{477.} ‘‘So evaṃ samāhite citte…pe… āneñjappatte pubbenivāsānussatiñāṇāya cittaṃ abhinīharati abhininnāmeti. So anekavihitaṃ pubbenivāsaṃ anussarati. Seyyathidaṃ, ekampi jātiṃ dvepi jātiyo tissopi jātiyo catassopi jātiyo pañcapi jātiyo dasapi jātiyo vīsampi jātiyo tiṃsampi jātiyo cattālīsampi jātiyo paññāsampi jātiyo jātisatampi jātisahassampi jātisatasahassampi anekepi saṃvaṭṭakappe anekepi vivaṭṭakappe anekepi saṃvaṭṭavivaṭṭakappe – ‘amutrāsiṃ evaṃnāmo evaṃgotto evaṃvaṇṇo evamāhāro evaṃsukhadukkhappaṭisaṃvedī evamāyupariyanto. So tato cuto amutra udapādiṃ; tatrāpāsiṃ evaṃnāmo evaṃgotto evaṃvaṇṇo evamāhāro evaṃsukhadukkhappaṭisaṃvedī evamāyupariyanto; so tato cuto idhūpapanno’ti. Iti sākāraṃ sauddesaṃ anekavihitaṃ pubbenivāsaṃ anussarati. ‘‘Seyyathāpi, māṇava, puriso sakamhā gāmā aññaṃ gāmaṃ gaccheyya; tamhāpi gāmā aññaṃ gāmaṃ gaccheyya; so tamhā gāmā sakaṃyeva gāmaṃ paccāgaccheyya. Tassa evamassa – ‘ahaṃ kho sakamhā gāmā amuṃ gāmaṃ agacchiṃ, tatra evaṃ aṭṭhāsiṃ evaṃ nisīdiṃ evaṃ abhāsiṃ evaṃ tuṇhī ahosiṃ. So tamhāpi gāmā amuṃ gāmaṃ gacchiṃ, tatrāpi evaṃ aṭṭhāsiṃ evaṃ nisīdiṃ evaṃ abhāsiṃ evaṃ tuṇhī ahosiṃ. Somhi tamhā gāmā sakaṃyeva gāmaṃ paccāgato’ti. Evameva kho, māṇava, bhikkhu…pe… yampi, māṇava, bhikkhu evaṃ samāhite citte…pe… āneñjappatte pubbenivāsānussatiñāṇāya cittaṃ abhinīharati abhininnāmeti. So anekavihitaṃ pubbenivāsaṃ anussarati. Seyyathidaṃ – ekampi jātiṃ…pe… iti sākāraṃ sauddesaṃ anekavihitaṃ pubbenivāsaṃ anussarati. Idampissa hoti paññāya.

\paragraph{478.} ‘‘So evaṃ samāhite citte…pe… āneñjappatte sattānaṃ cutūpapātañāṇāya cittaṃ abhinīharati abhininnāmeti. So dibbena cakkhunā visuddhena atikkantamānusakena satte passati cavamāne upapajjamāne hīne paṇīte suvaṇṇe dubbaṇṇe sugate duggate, yathākammūpage satte pajānāti – ‘ime vata bhonto sattā kāyaduccaritena samannāgatā vacīduccaritena samannāgatā manoduccaritena samannāgatā ariyānaṃ upavādakā micchādiṭṭhikā micchādiṭṭhikammasamādānā. Te kāyassa bhedā paraṃ maraṇā apāyaṃ duggatiṃ vinipātaṃ nirayaṃ upapannā. Ime vā pana bhonto sattā kāyasucaritena samannāgatā vacīsucaritena samannāgatā manosucaritena samannāgatā ariyānaṃ anupavādakā sammādiṭṭhikā sammādiṭṭhikammasamādānā. Te kāyassa bhedā paraṃ maraṇā sugatiṃ saggaṃ lokaṃ upapannā’ti. Iti dibbena cakkhunā visuddhena atikkantamānusakena satte passati cavamāne upapajjamāne hīne paṇīte suvaṇṇe dubbaṇṇe sugate duggate, yathākammūpage satte pajānāti. ‘‘Seyyathāpi, māṇava, majjhesiṅghāṭake pāsādo, tattha cakkhumā puriso ṭhito passeyya manusse gehaṃ pavisantepi nikkhamantepi rathikāyapi vīthiṃ sañcarante majjhesiṅghāṭake nisinnepi. Tassa evamassa – ‘ete manussā gehaṃ pavisanti, ete nikkhamanti, ete rathikāya vīthiṃ sañcaranti, ete majjhesiṅghāṭake nisinnā’ti. Evameva kho, māṇava, bhikkhu…pe… yampi, māṇava, bhikkhu evaṃ samāhite citte…pe… āneñjappatte sattānaṃ cutūpapātañāṇāya cittaṃ abhinīharati abhininnāmeti. So dibbena cakkhunā visuddhena atikkantamānusakena satte passati cavamāne upapajjamāne hīne paṇīte suvaṇṇe dubbaṇṇe sugate duggate, yathākammūpage satte pajānāti. Idampissa hoti paññāya.

\paragraph{479.} ‘‘So evaṃ samāhite citte parisuddhe pariyodāte anaṅgaṇe vigatūpakkilese mudubhūte kammaniye ṭhite āneñjappatte āsavānaṃ khayañāṇāya cittaṃ abhinīharati abhininnāmeti. So idaṃ dukkhanti yathābhūtaṃ pajānāti, ayaṃ dukkhasamudayoti yathābhūtaṃ pajānāti, ayaṃ dukkhanirodhoti yathābhūtaṃ pajānāti, ayaṃ dukkhanirodhagāminī paṭipadāti yathābhūtaṃ pajānāti; ime āsavāti yathābhūtaṃ pajānāti, ayaṃ āsavasamudayoti yathābhūtaṃ pajānāti, ayaṃ āsavanirodhoti yathābhūtaṃ pajānāti, ayaṃ āsavanirodhagāminī paṭipadāti yathābhūtaṃ pajānāti. Tassa evaṃ jānato evaṃ passato kāmāsavāpi cittaṃ vimuccati, bhavāsavāpi cittaṃ vimuccati, avijjāsavāpi cittaṃ vimuccati, vimuttasmiṃ vimuttamiti ñāṇaṃ hoti. ‘Khīṇā jāti, vusitaṃ brahmacariyaṃ, kataṃ karaṇīyaṃ, nāparaṃ itthattāyā’ti pajānāti. ‘‘Seyyathāpi, māṇava, pabbatasaṅkhepe udakarahado accho vippasanno anāvilo. Tattha cakkhumā puriso tīre ṭhito passeyya sippikasambukampi sakkharakathalampi macchagumbampi carantampi tiṭṭhantampi. Tassa evamassa – ‘ayaṃ kho udakarahado accho vippasanno anāvilo. Tatrime sippikasambukāpi sakkharakathalāpi macchagumbāpi carantipi tiṭṭhantipī’ti. Evameva kho, māṇava, bhikkhu…pe… yampi, māṇava, bhikkhu evaṃ samāhite citte…pe… āneñjappatte āsavānaṃ khayañāṇāya cittaṃ abhinīharati abhininnāmeti. So idaṃ dukkhanti yathābhūtaṃ pajānāti…pe… āsavanirodhagāminī paṭipadāti yathābhūtaṃ pajānāti. Tassa evaṃ jānato evaṃ passato kāmāsavāpi cittaṃ vimuccati, bhavāsavāpi cittaṃ vimuccati, avijjāsavāpi cittaṃ vimuccati, vimuttasmiṃ vimuttamiti ñāṇaṃ hoti, ‘khīṇā jāti, vusitaṃ brahmacariyaṃ, kataṃ karaṇīyaṃ, nāparaṃ itthattāyā’ti pajānāti. Idampissa hoti paññāya.

\paragraph{480.} ‘‘Ayaṃ kho, so māṇava, ariyo paññākkhandho yassa so bhagavā vaṇṇavādī ahosi, yattha ca imaṃ janataṃ samādapesi nivesesi patiṭṭhāpesi. Natthi cevettha uttarikaraṇīya’’nti. ‘‘Acchariyaṃ, bho ānanda, abbhutaṃ, bho ānanda! So cāyaṃ, bho ānanda, ariyo paññākkhandho paripuṇṇo, no aparipuṇṇo. Evaṃ paripuṇṇaṃ cāhaṃ, bho ānanda, ariyaṃ paññākkhandhaṃ ito bahiddhā aññesu samaṇabrāhmaṇesu na samanupassāmi. Natthi cevettha\footnote{na samanupassāmi…pe… natthi no kiñci (syā. ka.)} uttarikaraṇīyaṃ\footnote{uttariṃ karaṇīyanti (sī. syā. pī.) uttarikaraṇīyanti (ka.)}. Abhikkantaṃ, bho ānanda, abhikkantaṃ, bho ānanda! Seyyathāpi, bho ānanda, nikkujjitaṃ vā ukkujjeyya, paṭicchannaṃ vā vivareyya, mūḷhassa vā maggaṃ ācikkheyya, andhakāre vā telapajjotaṃ dhāreyya ‘cakkhumanto rūpāni dakkhantī’ti. Evamevaṃ bhotā ānandena anekapariyāyena dhammo pakāsito. Esāhaṃ, bho ānanda, taṃ bhavantaṃ gotamaṃ saraṇaṃ gacchāmi dhammañca bhikkhusaṅghañca. Upāsakaṃ maṃ bhavaṃ ānando dhāretu ajjatagge pāṇupetaṃ saraṇaṃ gata’’nti.

\xsectionEnd{Subhasuttaṃ niṭṭhitaṃ dasamaṃ.}


\clearpage
\section{Kevaṭṭasuttaṃ}

\subsubsection{Kevaṭṭagahapatiputtavatthu}

\paragraph{481.} Evaṃ me sutaṃ – ekaṃ samayaṃ bhagavā nāḷandāyaṃ viharati pāvārikambavane. Atha kho kevaṭṭo gahapatiputto yena bhagavā tenupasaṅkami; upasaṅkamitvā bhagavantaṃ abhivādetvā ekamantaṃ nisīdi. Ekamantaṃ nisinno kho kevaṭṭo gahapatiputto bhagavantaṃ etadavoca – ‘‘ayaṃ, bhante, nāḷandā iddhā ceva phītā ca bahujanā ākiṇṇamanussā bhagavati abhippasannā. Sādhu, bhante, bhagavā ekaṃ bhikkhuṃ samādisatu, yo uttarimanussadhammā, iddhipāṭihāriyaṃ karissati; evāyaṃ nāḷandā bhiyyoso mattāya bhagavati abhippasīdissatī’’ti. Evaṃ vutte, bhagavā kevaṭṭaṃ gahapatiputtaṃ etadavoca – ‘‘na kho ahaṃ, kevaṭṭa, bhikkhūnaṃ evaṃ dhammaṃ desemi – etha tumhe, bhikkhave, gihīnaṃ odātavasanānaṃ uttarimanussadhammā iddhipāṭihāriyaṃ karothā’’ti.

\paragraph{482.} Dutiyampi kho kevaṭṭo gahapatiputto bhagavantaṃ etadavoca – ‘‘nāhaṃ, bhante, bhagavantaṃ dhaṃsemi; api ca, evaṃ vadāmi – ‘ayaṃ, bhante, nāḷandā iddhā ceva phītā ca bahujanā ākiṇṇamanussā bhagavati abhippasannā. Sādhu, bhante, bhagavā ekaṃ bhikkhuṃ samādisatu, yo uttarimanussadhammā iddhipāṭihāriyaṃ karissati; evāyaṃ nāḷandā bhiyyoso mattāya bhagavati abhippasīdissatī’’’ti. Dutiyampi kho bhagavā kevaṭṭaṃ gahapatiputtaṃ etadavoca – ‘‘na kho ahaṃ, kevaṭṭa, bhikkhūnaṃ evaṃ dhammaṃ desemi – etha tumhe, bhikkhave, gihīnaṃ odātavasanānaṃ uttarimanussadhammā iddhipāṭihāriyaṃ karothā’’’ti. Tatiyampi kho kevaṭṭo gahapatiputto bhagavantaṃ etadavoca – ‘‘nāhaṃ, bhante, bhagavantaṃ dhaṃsemi; api ca, evaṃ vadāmi – ‘ayaṃ, bhante, nāḷandā iddhā ceva phītā ca bahujanā ākiṇṇamanussā bhagavati abhippasannā. Sādhu, bhante, bhagavā ekaṃ bhikkhuṃ samādisatu, yo uttarimanussadhammā iddhipāṭihāriyaṃ karissati. Evāyaṃ nāḷandā bhiyyoso mattāya bhagavati abhippasīdissatī’ti.

\subsubsection{Iddhipāṭihāriyaṃ}

\paragraph{483.} ‘‘Tīṇi kho imāni, kevaṭṭa, pāṭihāriyāni mayā sayaṃ abhiññā sacchikatvā paveditāni. Katamāni tīṇi? Iddhipāṭihāriyaṃ, ādesanāpāṭihāriyaṃ, anusāsanīpāṭihāriyaṃ.

\paragraph{484.} ‘‘Katamañca, kevaṭṭa, iddhipāṭihāriyaṃ? Idha, kevaṭṭa, bhikkhu anekavihitaṃ iddhividhaṃ paccanubhoti. Ekopi hutvā bahudhā hoti, bahudhāpi hutvā eko hoti; āvibhāvaṃ tirobhāvaṃ tirokuṭṭaṃ tiropākāraṃ tiropabbataṃ asajjamāno gacchati seyyathāpi ākāse; pathaviyāpi ummujjanimujjaṃ karoti seyyathāpi udake; udakepi abhijjamāne gacchati seyyathāpi pathaviyaṃ; ākāsepi pallaṅkena kamati seyyathāpi pakkhī sakuṇo; imepi candimasūriye evaṃ mahiddhike evaṃ mahānubhāve pāṇinā parāmasati parimajjati; yāva brahmalokāpi kāyena vasaṃ vatteti. ‘‘Tamenaṃ aññataro saddho pasanno passati taṃ bhikkhuṃ anekavihitaṃ iddhividhaṃ paccanubhontaṃ – ekopi hutvā bahudhā hontaṃ, bahudhāpi hutvā eko hontaṃ; āvibhāvaṃ tirobhāvaṃ; tirokuṭṭaṃ tiropākāraṃ tiropabbataṃ asajjamānaṃ gacchantaṃ seyyathāpi ākāse; pathaviyāpi ummujjanimujjaṃ karontaṃ seyyathāpi udake; udakepi abhijjamāne gacchantaṃ seyyathāpi pathaviyaṃ; ākāsepi pallaṅkena kamantaṃ seyyathāpi pakkhī sakuṇo; imepi candimasūriye evaṃ mahiddhike evaṃ mahānubhāve pāṇinā parāmasantaṃ parimajjantaṃ yāva brahmalokāpi kāyena vasaṃ vattentaṃ. ‘‘Tamenaṃ so saddho pasanno aññatarassa assaddhassa appasannassa āroceti – ‘acchariyaṃ vata, bho, abbhutaṃ vata, bho, samaṇassa mahiddhikatā mahānubhāvatā. Amāhaṃ bhikkhuṃ addasaṃ anekavihitaṃ iddhividhaṃ paccanubhontaṃ – ekopi hutvā bahudhā hontaṃ, bahudhāpi hutvā eko hontaṃ…pe… yāva brahmalokāpi kāyena vasaṃ vattenta’nti. ‘‘Tamenaṃ so assaddho appasanno taṃ saddhaṃ pasannaṃ evaṃ vadeyya – ‘atthi kho, bho, gandhārī nāma vijjā. Tāya so bhikkhu anekavihitaṃ iddhividhaṃ paccanubhoti – ekopi hutvā bahudhā hoti, bahudhāpi hutvā eko hoti…pe… yāva brahmalokāpi kāyena vasaṃ vattetī’ti. ‘‘Taṃ kiṃ maññasi, kevaṭṭa, api nu so assaddho appasanno taṃ saddhaṃ pasannaṃ evaṃ vadeyyā’’ti? ‘‘Vadeyya, bhante’’ti. ‘‘Imaṃ kho ahaṃ, kevaṭṭa, iddhipāṭihāriye ādīnavaṃ sampassamāno iddhipāṭihāriyena aṭṭīyāmi harāyāmi jigucchāmi’’.

\subsubsection{Ādesanāpāṭihāriyaṃ}

\paragraph{485.} ‘‘Katamañca, kevaṭṭa, ādesanāpāṭihāriyaṃ? Idha, kevaṭṭa, bhikkhu parasattānaṃ parapuggalānaṃ cittampi ādisati, cetasikampi ādisati, vitakkitampi ādisati, vicāritampi ādisati – ‘evampi te mano, itthampi te mano, itipi te citta’nti. ‘‘Tamenaṃ aññataro saddho pasanno passati taṃ bhikkhuṃ parasattānaṃ parapuggalānaṃ cittampi ādisantaṃ, cetasikampi ādisantaṃ, vitakkitampi ādisantaṃ, vicāritampi ādisantaṃ – ‘evampi te mano, itthampi te mano, itipi te citta’nti. Tamenaṃ so saddho pasanno aññatarassa assaddhassa appasannassa āroceti – ‘acchariyaṃ vata, bho, abbhutaṃ vata, bho, samaṇassa mahiddhikatā mahānubhāvatā. Amāhaṃ bhikkhuṃ addasaṃ parasattānaṃ parapuggalānaṃ cittampi ādisantaṃ, cetasikampi ādisantaṃ, vitakkitampi ādisantaṃ, vicāritampi ādisantaṃ – ‘‘evampi te mano, itthampi te mano, itipi te citta’’’nti. ‘‘Tamenaṃ so assaddho appasanno taṃ saddhaṃ pasannaṃ evaṃ vadeyya – ‘atthi kho, bho, maṇikā nāma vijjā; tāya so bhikkhu parasattānaṃ parapuggalānaṃ cittampi ādisati, cetasikampi ādisati, vitakkitampi ādisati, vicāritampi ādisati – ‘evampi te mano, itthampi te mano, itipi te citta’’’nti. ‘‘Taṃ kiṃ maññasi, kevaṭṭa, api nu so assaddho appasanno taṃ saddhaṃ pasannaṃ evaṃ vadeyyā’’ti? ‘‘Vadeyya, bhante’’ti. ‘‘Imaṃ kho ahaṃ, kevaṭṭa, ādesanāpāṭihāriye ādīnavaṃ sampassamāno ādesanāpāṭihāriyena aṭṭīyāmi harāyāmi jigucchāmi’’.

\subsubsection{Anusāsanīpāṭihāriyaṃ}

\paragraph{486.} ‘‘Katamañca, kevaṭṭa, anusāsanīpāṭihāriyaṃ? Idha, kevaṭṭa, bhikkhu evamanusāsati – ‘evaṃ vitakketha, mā evaṃ vitakkayittha, evaṃ manasikarotha, mā evaṃ manasākattha, idaṃ pajahatha, idaṃ upasampajja viharathā’ti. Idaṃ vuccati, kevaṭṭa, anusāsanīpāṭihāriyaṃ. ‘‘Puna caparaṃ, kevaṭṭa, idha tathāgato loke uppajjati arahaṃ sammāsambuddho … pe… (yathā 190-212 anucchedesu evaṃ vitthāretabbaṃ). Evaṃ kho, kevaṭṭa, bhikkhu sīlasampanno hoti…pe… paṭhamaṃ jhānaṃ upasampajja viharati. Idampi vuccati, kevaṭṭa, anusāsanīpāṭihāriyaṃ…pe… dutiyaṃ jhānaṃ…pe… tatiyaṃ jhānaṃ…pe… catutthaṃ jhānaṃ upasampajja viharati. Idampi vuccati, kevaṭṭa, anusāsanīpāṭihāriyaṃ…pe… ñāṇadassanāya cittaṃ abhinīharati abhininnāmeti…pe… idampi vuccati, kevaṭṭa, anusāsanīpāṭihāriyaṃ…pe… nāparaṃ itthattāyāti pajānāti…pe… idampi vuccati, kevaṭṭa, anusāsanīpāṭihāriyaṃ. ‘‘Imāni kho, kevaṭṭa, tīṇi pāṭihāriyāni mayā sayaṃ abhiññā sacchikatvā paveditāni’’.

\subsubsection{Bhūtanirodhesakabhikkhuvatthu}

\paragraph{487.} ‘‘Bhūtapubbaṃ, kevaṭṭa, imasmiññeva bhikkhusaṅghe aññatarassa bhikkhuno evaṃ cetaso parivitakko udapādi – ‘kattha nu kho ime cattāro mahābhūtā aparisesā nirujjhanti, seyyathidaṃ – pathavīdhātu āpodhātu tejodhātu vāyodhātū’ti?

\paragraph{488.} ‘‘Atha kho so, kevaṭṭa, bhikkhu tathārūpaṃ samādhiṃ samāpajji, yathāsamāhite citte devayāniyo maggo pāturahosi. Atha kho so, kevaṭṭa, bhikkhu yena cātumahārājikā devā tenupasaṅkami; upasaṅkamitvā cātumahārājike deve etadavoca – ‘kattha nu kho, āvuso, ime cattāro mahābhūtā aparisesā nirujjhanti, seyyathidaṃ – pathavīdhātu āpodhātu tejodhātu vāyodhātū’ti? ‘‘Evaṃ vutte, kevaṭṭa, cātumahārājikā devā taṃ bhikkhuṃ etadavocuṃ – ‘mayampi kho, bhikkhu, na jānāma, yatthime cattāro mahābhūtā aparisesā nirujjhanti, seyyathidaṃ – pathavīdhātu āpodhātu tejodhātu vāyodhātūti\footnote{vāyodhātu. atthi kho (pī. evamuparipi)}. Atthi kho\footnote{vāyodhātu. atthi kho (pī. evamuparipi)}, bhikkhu, cattāro mahārājāno amhehi abhikkantatarā ca paṇītatarā ca. Te kho etaṃ jāneyyuṃ, yatthime cattāro mahābhūtā aparisesā nirujjhanti, seyyathidaṃ – pathavīdhātu āpodhātu tejodhātu vāyodhātū’ti.

\paragraph{489.} ‘‘Atha kho so, kevaṭṭa, bhikkhu yena cattāro mahārājāno tenupasaṅkami; upasaṅkamitvā cattāro mahārāje etadavoca – ‘kattha nu kho, āvuso, ime cattāro mahābhūtā aparisesā nirujjhanti, seyyathidaṃ – pathavīdhātu āpodhātu tejodhātu vāyodhātū’ti? Evaṃ vutte, kevaṭṭa, cattāro mahārājāno taṃ bhikkhuṃ etadavocuṃ – ‘mayampi kho, bhikkhu, na jānāma, yatthime cattāro mahābhūtā aparisesā nirujjhanti, seyyathidaṃ – pathavīdhātu, āpodhātu tejodhātu vāyodhātūti. Atthi kho, bhikkhu, tāvatiṃsā nāma devā amhehi abhikkantatarā ca paṇītatarā ca. Te kho etaṃ jāneyyuṃ, yatthime cattāro mahābhūtā aparisesā nirujjhanti, seyyathidaṃ – pathavīdhātu āpodhātu tejodhātu vāyodhātū’ti.

\paragraph{490.} ‘‘Atha kho so, kevaṭṭa, bhikkhu yena tāvatiṃsā devā tenupasaṅkami; upasaṅkamitvā tāvatiṃse deve etadavoca – ‘kattha nu kho, āvuso, ime cattāro mahābhūtā aparisesā nirujjhanti, seyyathidaṃ – pathavīdhātu āpodhātu tejodhātu vāyodhātū’ti? Evaṃ vutte, kevaṭṭa, tāvatiṃsā devā taṃ bhikkhuṃ etadavocuṃ – ‘mayampi kho, bhikkhu, na jānāma, yatthime cattāro mahābhūtā aparisesā nirujjhanti, seyyathidaṃ – pathavīdhātu āpodhātu tejodhātu vāyodhātūti. Atthi kho, bhikkhu, sakko nāma devānamindo amhehi abhikkantataro ca paṇītataro ca. So kho etaṃ jāneyya, yatthime cattāro mahābhūtā aparisesā nirujjhanti, seyyathidaṃ – pathavīdhātu āpodhātu tejodhātu vāyodhātū’ti.

\paragraph{491.} ‘‘Atha kho so, kevaṭṭa, bhikkhu yena sakko devānamindo tenupasaṅkami; upasaṅkamitvā sakkaṃ devānamindaṃ etadavoca – ‘kattha nu kho, āvuso, ime cattāro mahābhūtā aparisesā nirujjhanti, seyyathidaṃ – pathavīdhātu āpodhātu tejodhātu vāyodhātū’ti? Evaṃ vutte, kevaṭṭa, sakko devānamindo taṃ bhikkhuṃ etadavoca – ‘ahampi kho, bhikkhu, na jānāmi, yatthime cattāro mahābhūtā aparisesā nirujjhanti, seyyathidaṃ – pathavīdhātu āpodhātu tejodhātu vāyodhātūti. Atthi kho, bhikkhu, yāmā nāma devā…pe… suyāmo nāma devaputto… tusitā nāma devā… santussito nāma devaputto… nimmānaratī nāma devā … sunimmito nāma devaputto… paranimmitavasavattī nāma devā… vasavattī nāma devaputto amhehi abhikkantataro ca paṇītataro ca. So kho etaṃ jāneyya, yatthime cattāro mahābhūtā aparisesā nirujjhanti, seyyathidaṃ – pathavīdhātu āpodhātu tejodhātu vāyodhātū’ti.

\paragraph{492.} ‘‘Atha kho so, kevaṭṭa, bhikkhu yena vasavattī devaputto tenupasaṅkami; upasaṅkamitvā vasavattiṃ devaputtaṃ etadavoca – ‘kattha nu kho, āvuso, ime cattāro mahābhūtā aparisesā nirujjhanti, seyyathidaṃ – pathavīdhātu āpodhātu tejodhātu vāyodhātū’ti? Evaṃ vutte, kevaṭṭa, vasavattī devaputto taṃ bhikkhuṃ etadavoca – ‘ahampi kho, bhikkhu, na jānāmi yatthime cattāro mahābhūtā aparisesā nirujjhanti, seyyathidaṃ – pathavīdhātu āpodhātu tejodhātu vāyodhātūti. Atthi kho, bhikkhu, brahmakāyikā nāma devā amhehi abhikkantatarā ca paṇītatarā ca. Te kho etaṃ jāneyyuṃ, yatthime cattāro mahābhūtā aparisesā nirujjhanti, seyyathidaṃ – pathavīdhātu āpodhātu tejodhātu vāyodhātū’ti.

\paragraph{493.} ‘‘Atha kho so, kevaṭṭa, bhikkhu tathārūpaṃ samādhiṃ samāpajji, yathāsamāhite citte brahmayāniyo maggo pāturahosi. Atha kho so, kevaṭṭa, bhikkhu yena brahmakāyikā devā tenupasaṅkami; upasaṅkamitvā brahmakāyike deve etadavoca – ‘kattha nu kho, āvuso, ime cattāro mahābhūtā aparisesā nirujjhanti, seyyathidaṃ – pathavīdhātu āpodhātu tejodhātu vāyodhātū’ti? Evaṃ vutte, kevaṭṭa, brahmakāyikā devā taṃ bhikkhuṃ etadavocuṃ – ‘mayampi kho, bhikkhu, na jānāma, yatthime cattāro mahābhūtā aparisesā nirujjhanti, seyyathidaṃ – pathavīdhātu āpodhātu tejodhātu vāyodhātūti. Atthi kho, bhikkhu, brahmā mahābrahmā abhibhū anabhibhūto aññadatthudaso vasavattī issaro kattā nimmātā seṭṭho sajitā vasī pitā bhūtabhabyānaṃ amhehi abhikkantataro ca paṇītataro ca. So kho etaṃ jāneyya, yatthime cattāro mahābhūtā aparisesā nirujjhanti, seyyathidaṃ – pathavīdhātu āpodhātu tejodhātu vāyodhātū’’ti. ‘‘‘Kahaṃ panāvuso, etarahi so mahābrahmā’ti? ‘Mayampi kho, bhikkhu, na jānāma, yattha vā brahmā yena vā brahmā yahiṃ vā brahmā; api ca, bhikkhu, yathā nimittā dissanti, āloko sañjāyati, obhāso pātubhavati, brahmā pātubhavissati, brahmuno hetaṃ pubbanimittaṃ pātubhāvāya, yadidaṃ āloko sañjāyati, obhāso pātubhavatī’ti. Atha kho so, kevaṭṭa, mahābrahmā nacirasseva pāturahosi.

\paragraph{494.} ‘‘Atha kho so, kevaṭṭa, bhikkhu yena so mahābrahmā tenupasaṅkami; upasaṅkamitvā taṃ mahābrahmānaṃ etadavoca – ‘kattha nu kho, āvuso, ime cattāro mahābhūtā aparisesā nirujjhanti, seyyathidaṃ – pathavīdhātu āpodhātu tejodhātu vāyodhātū’’ti? Evaṃ vutte, kevaṭṭa, so mahābrahmā taṃ bhikkhuṃ etadavoca – ‘ahamasmi, bhikkhu, brahmā mahābrahmā abhibhū anabhibhūto aññadatthudaso vasavattī issaro kattā nimmātā seṭṭho sajitā vasī pitā bhūtabhabyāna’nti. ‘‘Dutiyampi kho so, kevaṭṭa, bhikkhu taṃ mahābrahmānaṃ etadavoca – ‘na khohaṃ taṃ, āvuso, evaṃ pucchāmi – ‘‘tvamasi brahmā mahābrahmā abhibhū anabhibhūto aññadatthudaso vasavattī issaro kattā nimmātā seṭṭho sajitā vasī pitā bhūtabhabyāna’’nti. Evañca kho ahaṃ taṃ, āvuso, pucchāmi – ‘‘kattha nu kho, āvuso, ime cattāro mahābhūtā aparisesā nirujjhanti, seyyathidaṃ – pathavīdhātu āpodhātu tejodhātu vāyodhātū’’’ti? ‘‘Dutiyampi kho so, kevaṭṭa, mahābrahmā taṃ bhikkhuṃ etadavoca – ‘ahamasmi, bhikkhu, brahmā mahābrahmā abhibhū anabhibhūto aññadatthudaso vasavattī issaro kattā nimmātā seṭṭho sajitā vasī pitā bhūtabhabyāna’nti. Tatiyampi kho so, kevaṭṭa, bhikkhu taṃ mahābrahmānaṃ etadavoca – ‘na khohaṃ taṃ, āvuso, evaṃ pucchāmi – ‘‘tvamasi brahmā mahābrahmā abhibhū anabhibhūto aññadatthudaso vasavattī issaro kattā nimmātā seṭṭho sajitā vasī pitā bhūtabhabyāna’’nti. Evañca kho ahaṃ taṃ, āvuso, pucchāmi – ‘‘kattha nu kho, āvuso, ime cattāro mahābhūtā aparisesā nirujjhanti, seyyathidaṃ – pathavīdhātu āpodhātu tejodhātu vāyodhātū’’’ti?

\paragraph{495.} ‘‘Atha kho so, kevaṭṭa, mahābrahmā taṃ bhikkhuṃ bāhāyaṃ gahetvā ekamantaṃ apanetvā taṃ bhikkhuṃ etadavoca – ‘ime kho maṃ, bhikkhu, brahmakāyikā devā evaṃ jānanti, ‘‘natthi kiñci brahmuno aññātaṃ, natthi kiñci brahmuno adiṭṭhaṃ, natthi kiñci brahmuno aviditaṃ, natthi kiñci brahmuno asacchikata’’nti. Tasmāhaṃ tesaṃ sammukhā na byākāsiṃ. Ahampi kho, bhikkhu, na jānāmi yatthime cattāro mahābhūtā aparisesā nirujjhanti, seyyathidaṃ – pathavīdhātu āpodhātu tejodhātu vāyodhātūti. Tasmātiha, bhikkhu, tuyhevetaṃ dukkaṭaṃ, tuyhevetaṃ aparaddhaṃ, yaṃ tvaṃ taṃ bhagavantaṃ atidhāvitvā bahiddhā pariyeṭṭhiṃ āpajjasi imassa pañhassa veyyākaraṇāya. Gaccha tvaṃ, bhikkhu, tameva bhagavantaṃ upasaṅkamitvā imaṃ pañhaṃ puccha, yathā ca te bhagavā byākaroti, tathā naṃ dhāreyyāsī’ti.

\paragraph{496.} ‘‘Atha kho so, kevaṭṭa, bhikkhu – seyyathāpi nāma balavā puriso samiñjitaṃ vā bāhaṃ pasāreyya, pasāritaṃ vā bāhaṃ samiñjeyya evameva brahmaloke antarahito mama purato pāturahosi. Atha kho so, kevaṭṭa, bhikkhu maṃ abhivādetvā ekamantaṃ nisīdi, ekamantaṃ nisinno kho, kevaṭṭa, so bhikkhu maṃ etadavoca – ‘kattha nu kho, bhante, ime cattāro mahābhūtā aparisesā nirujjhanti, seyyathidaṃ – pathavīdhātu āpodhātu tejodhātu vāyodhātū’ti?

\subsubsection{Tīradassisakuṇupamā}

\paragraph{497.} ‘‘Evaṃ vutte, ahaṃ, kevaṭṭa, taṃ bhikkhuṃ etadavocaṃ – ‘bhūtapubbaṃ, bhikkhu, sāmuddikā vāṇijā tīradassiṃ sakuṇaṃ gahetvā nāvāya samuddaṃ ajjhogāhanti. Te atīradakkhiniyā nāvāya tīradassiṃ sakuṇaṃ muñcanti. So gacchateva puratthimaṃ disaṃ, gacchati dakkhiṇaṃ disaṃ, gacchati pacchimaṃ disaṃ, gacchati uttaraṃ disaṃ, gacchati uddhaṃ disaṃ, gacchati anudisaṃ. Sace so samantā tīraṃ passati, tathāgatakova\footnote{tathāpakkantova (syā.)} hoti. Sace pana so samantā tīraṃ na passati, tameva nāvaṃ paccāgacchati. Evameva kho tvaṃ, bhikkhu, yato yāva brahmalokā pariyesamāno imassa pañhassa veyyākaraṇaṃ nājjhagā, atha mamaññeva santike paccāgato. Na kho eso, bhikkhu, pañho evaṃ pucchitabbo – ‘kattha nu kho, bhante, ime cattāro mahābhūtā aparisesā nirujjhanti, seyyathidaṃ – pathavīdhātu āpodhātu tejodhātu vāyodhātū’ti?

\paragraph{498.} ‘‘Evañca kho eso, bhikkhu, pañho pucchitabbo –
\begin{verse}
  \small
  ‘Kattha āpo ca pathavī, tejo vāyo na gādhati;\\
  Kattha dīghañca rassañca, aṇuṃ thūlaṃ subhāsubhaṃ;\\
  Kattha nāmañca rūpañca, asesaṃ uparujjhatī’ti.\\
\end{verse}

\paragraph{499.} ‘‘Tatra veyyākaraṇaṃ bhavati –
\begin{verse}
  \small
  ‘Viññāṇaṃ anidassanaṃ, anantaṃ sabbatopabhaṃ;\\
  Ettha āpo ca pathavī, tejo vāyo na gādhati.\\[0.5cm]
  
  Ettha dīghañca rassañca, aṇuṃ thūlaṃ subhāsubhaṃ;\\
  Ettha nāmañca rūpañca, asesaṃ uparujjhati;\\
  Viññāṇassa nirodhena, etthetaṃ uparujjhatī’ti.\\
\end{verse}

\paragraph{500.} Idamavoca bhagavā. Attamano kevaṭṭo gahapatiputto bhagavato bhāsitaṃ abhinandīti.

\xsectionEnd{Kevaṭṭasuttaṃ niṭṭhitaṃ ekādasamaṃ.}


\clearpage
\section{Lohiccasuttaṃ}

\subsubsection{Lohiccabrāhmaṇavatthu}

\paragraph{501.} Evaṃ me sutaṃ – ekaṃ samayaṃ bhagavā kosalesu cārikaṃ caramāno mahatā bhikkhusaṅghena saddhiṃ pañcamattehi bhikkhusatehi yena sālavatikā tadavasari. Tena kho pana samayena lohicco brāhmaṇo sālavatikaṃ ajjhāvasati sattussadaṃ satiṇakaṭṭhodakaṃ sadhaññaṃ rājabhoggaṃ raññā pasenadinā kosalena dinnaṃ rājadāyaṃ, brahmadeyyaṃ.

\paragraph{502.} Tena kho pana samayena lohiccassa brāhmaṇassa evarūpaṃ pāpakaṃ diṭṭhigataṃ uppannaṃ hoti – ‘‘idha samaṇo vā brāhmaṇo vā kusalaṃ dhammaṃ adhigaccheyya, kusalaṃ dhammaṃ adhigantvā na parassa āroceyya, kiñhi paro parassa karissati. Seyyathāpi nāma purāṇaṃ bandhanaṃ chinditvā aññaṃ navaṃ bandhanaṃ kareyya, evaṃsampadamidaṃ pāpakaṃ lobhadhammaṃ vadāmi, kiñhi paro parassa karissatī’’ti.

\paragraph{503.} Assosi kho lohicco brāhmaṇo – ‘‘samaṇo khalu, bho, gotamo sakyaputto sakyakulā pabbajito kosalesu cārikaṃ caramāno mahatā bhikkhusaṅghena saddhiṃ pañcamattehi bhikkhusatehi sālavatikaṃ anuppatto. Taṃ kho pana bhavantaṃ gotamaṃ evaṃ kalyāṇo kittisaddo abbhuggato – ‘itipi so bhagavā arahaṃ sammāsambuddho vijjācaraṇasampanno sugato lokavidū anuttaro purisadammasārathi satthā devamanussānaṃ buddho bhagavā’. So imaṃ lokaṃ sadevakaṃ samārakaṃ sabrahmakaṃ sassamaṇabrāhmaṇiṃ pajaṃ sadevamanussaṃ sayaṃ abhiññā sacchikatvā pavedeti. So dhammaṃ deseti ādikalyāṇaṃ majjhekalyāṇaṃ pariyosānakalyāṇaṃ sātthaṃ sabyañjanaṃ kevalaparipuṇṇaṃ parisuddhaṃ brahmacariyaṃ pakāseti. Sādhu kho pana tathārūpānaṃ arahataṃ dassanaṃ hotī’’ti.

\paragraph{504.} Atha kho lohicco brāhmaṇo rosikaṃ\footnote{bhesikaṃ (sī. pī.)} nhāpitaṃ āmantesi – ‘‘ehi tvaṃ, samma rosike, yena samaṇo gotamo tenupasaṅkama; upasaṅkamitvā mama vacanena samaṇaṃ gotamaṃ appābādhaṃ appātaṅkaṃ lahuṭṭhānaṃ balaṃ phāsuvihāraṃ puccha – lohicco, bho gotama, brāhmaṇo bhavantaṃ gotamaṃ appābādhaṃ appātaṅkaṃ lahuṭṭhānaṃ balaṃ phāsuvihāraṃ pucchatī’’ti. Evañca vadehi – ‘‘adhivāsetu kira bhavaṃ gotamo lohiccassa brāhmaṇassa svātanāya bhattaṃ saddhiṃ bhikkhusaṅghenā’’ti.

\paragraph{505.} ‘‘Evaṃ, bho’’ti\footnote{evaṃ bhanteti (sī. pī.)} kho rosikā nhāpito lohiccassa brāhmaṇassa paṭissutvā yena bhagavā tenupasaṅkami; upasaṅkamitvā bhagavantaṃ abhivādetvā ekamantaṃ nisīdi. Ekamantaṃ nisinno kho rosikā nhāpito bhagavantaṃ etadavoca – ‘‘lohicco, bhante, brāhmaṇo bhagavantaṃ appābādhaṃ appātaṅkaṃ lahuṭṭhānaṃ balaṃ phāsuvihāraṃ pucchati; evañca vadeti – adhivāsetu kira, bhante, bhagavā lohiccassa brāhmaṇassa svātanāya bhattaṃ saddhiṃ bhikkhusaṅghenā’’ti. Adhivāsesi bhagavā tuṇhībhāvena.

\paragraph{506.} Atha kho rosikā nhāpito bhagavato adhivāsanaṃ viditvā uṭṭhāyāsanā bhagavantaṃ abhivādetvā padakkhiṇaṃ katvā yena lohicco brāhmaṇo tenupasaṅkami; upasaṅkamitvā lohiccaṃ brāhmaṇaṃ etadavoca – ‘‘avocumhā kho mayaṃ bhoto\footnote{mayaṃ bhante tava (sī. pī.)} vacanena taṃ bhagavantaṃ – ‘lohicco, bhante, brāhmaṇo bhagavantaṃ appābādhaṃ appātaṅkaṃ lahuṭṭhānaṃ balaṃ phāsuvihāraṃ pucchati; evañca vadeti – adhivāsetu kira, bhante, bhagavā lohiccassa brāhmaṇassa svātanāya bhattaṃ saddhiṃ bhikkhusaṅghenā’ti. Adhivutthañca pana tena bhagavatā’’ti.

\paragraph{507.} Atha kho lohicco brāhmaṇo tassā rattiyā accayena sake nivesane paṇītaṃ khādanīyaṃ bhojanīyaṃ paṭiyādāpetvā rosikaṃ nhāpitaṃ āmantesi – ‘‘ehi tvaṃ, samma rosike, yena samaṇo gotamo tenupasaṅkama; upasaṅkamitvā samaṇassa gotamassa kālaṃ ārocehi – kālo bho, gotama, niṭṭhitaṃ bhatta’’nti. ‘‘Evaṃ, bho’’ti kho rosikā nhāpito lohiccassa brāhmaṇassa paṭissutvā yena bhagavā tenupasaṅkami; upasaṅkamitvā bhagavantaṃ abhivādetvā ekamantaṃ aṭṭhāsi. Ekamantaṃ ṭhito kho rosikā nhāpito bhagavato kālaṃ ārocesi – ‘‘kālo, bhante, niṭṭhitaṃ bhatta’’nti.

\paragraph{508.} Atha kho bhagavā pubbaṇhasamayaṃ nivāsetvā pattacīvaramādāya saddhiṃ bhikkhusaṅghena yena sālavatikā tenupasaṅkami. Tena kho pana samayena rosikā nhāpito bhagavantaṃ piṭṭhito piṭṭhito anubandho hoti. Atha kho rosikā nhāpito bhagavantaṃ etadavoca – ‘‘lohiccassa, bhante, brāhmaṇassa evarūpaṃ pāpakaṃ diṭṭhigataṃ uppannaṃ – ‘idha samaṇo vā brāhmaṇo vā kusalaṃ dhammaṃ adhigaccheyya, kusalaṃ dhammaṃ adhigantvā na parassa āroceyya – kiñhi paro parassa karissati. Seyyathāpi nāma purāṇaṃ bandhanaṃ chinditvā aññaṃ navaṃ bandhanaṃ kareyya, evaṃ sampadamidaṃ pāpakaṃ lobhadhammaṃ vadāmi – kiñhi paro parassa karissatī’ti. Sādhu, bhante, bhagavā lohiccaṃ brāhmaṇaṃ etasmā pāpakā diṭṭhigatā vivecetū’’ti. ‘‘Appeva nāma siyā rosike, appeva nāma siyā rosike’’ti. Atha kho bhagavā yena lohiccassa brāhmaṇassa nivesanaṃ tenupasaṅkami; upasaṅkamitvā paññatte āsane nisīdi. Atha kho lohicco brāhmaṇo buddhappamukhaṃ bhikkhusaṅghaṃ paṇītena khādanīyena bhojanīyena sahatthā santappesi sampavāresi.

\subsubsection{Lohiccabrāhmaṇānuyogo}

\paragraph{509.} Atha kho lohicco brāhmaṇo bhagavantaṃ bhuttāviṃ onītapattapāṇiṃ aññataraṃ nīcaṃ āsanaṃ gahetvā ekamantaṃ nisīdi. Ekamantaṃ nisinnaṃ kho lohiccaṃ brāhmaṇaṃ bhagavā etadavoca – ‘‘saccaṃ kira te, lohicca, evarūpaṃ pāpakaṃ diṭṭhigataṃ uppannaṃ – ‘idha samaṇo vā brāhmaṇo vā kusalaṃ dhammaṃ adhigaccheyya, kusalaṃ dhammaṃ adhigantvā na parassa āroceyya – kiñhi paro parassa karissati. Seyyathāpi nāma purāṇaṃ bandhanaṃ chinditvā aññaṃ navaṃ bandhanaṃ kareyya, evaṃ sampadamidaṃ pāpakaṃ lobhadhammaṃ vadāmi, kiñhi paro parassa karissatī’’’ ti? ‘‘Evaṃ, bho gotama’’. ‘‘Taṃ kiṃ maññasi lohicca nanu tvaṃ sālavatikaṃ ajjhāvasasī’’ti? ‘‘Evaṃ, bho gotama’’. ‘‘Yo nu kho, lohicca, evaṃ vadeyya – ‘lohicco brāhmaṇo sālavatikaṃ ajjhāvasati. Yā sālavatikāya samudayasañjāti lohiccova taṃ brāhmaṇo ekako paribhuñjeyya, na aññesaṃ dadeyyā’ti. Evaṃ vādī so ye taṃ upajīvanti, tesaṃ antarāyakaro vā hoti, no vā’’ti? ‘‘Antarāyakaro, bho gotama’’. ‘‘Antarāyakaro samāno hitānukampī vā tesaṃ hoti ahitānukampī vā’’ti? ‘‘Ahitānukampī, bho gotama’’. ‘‘Ahitānukampissa mettaṃ vā tesu cittaṃ paccupaṭṭhitaṃ hoti sapattakaṃ vā’’ti? ‘‘Sapattakaṃ, bho gotama’’. ‘‘Sapattake citte paccupaṭṭhite micchādiṭṭhi vā hoti sammādiṭṭhi vā’’ti? ‘‘Micchādiṭṭhi, bho gotama’’. ‘‘Micchādiṭṭhissa kho ahaṃ, lohicca, dvinnaṃ gatīnaṃ aññataraṃ gatiṃ vadāmi – nirayaṃ vā tiracchānayoniṃ vā’’.

\paragraph{510.} ‘‘Taṃ kiṃ maññasi, lohicca, nanu rājā pasenadi kosalo kāsikosalaṃ ajjhāvasatī’’ti? ‘‘Evaṃ, bho gotama’’. ‘‘Yo nu kho, lohicca, evaṃ vadeyya – ‘rājā pasenadi kosalo kāsikosalaṃ ajjhāvasati; yā kāsikosale samudayasañjāti, rājāva taṃ pasenadi kosalo ekako paribhuñjeyya, na aññesaṃ dadeyyā’ti. Evaṃ vādī so ye rājānaṃ pasenadiṃ kosalaṃ upajīvanti tumhe ceva aññe ca, tesaṃ antarāyakaro vā hoti, no vā’’ti? ‘‘Antarāyakaro, bho gotama’’. ‘‘Antarāyakaro samāno hitānukampī vā tesaṃ hoti ahitānukampī vā’’ti? ‘‘Ahitānukampī, bho gotama’’. ‘‘Ahitānukampissa mettaṃ vā tesu cittaṃ paccupaṭṭhitaṃ hoti sapattakaṃ vā’’ti? ‘‘Sapattakaṃ, bho gotama’’. ‘‘Sapattake citte paccupaṭṭhite micchādiṭṭhi vā hoti sammādiṭṭhi vā’’ti? ‘‘Micchādiṭṭhi, bho gotama’’. ‘‘Micchādiṭṭhissa kho ahaṃ, lohicca, dvinnaṃ gatīnaṃ aññataraṃ gatiṃ vadāmi – nirayaṃ vā tiracchānayoniṃ vā’’.

\paragraph{511.} ‘‘Iti kira, lohicca, yo evaṃ vadeyya – ‘‘lohicco brāhmaṇo sālavatikaṃ ajjhāvasati; yā sālavatikāya samudayasañjāti, lohiccova taṃ brāhmaṇo ekako paribhuñjeyya, na aññesaṃ dadeyyā’’ti. Evaṃvādī so ye taṃ upajīvanti, tesaṃ antarāyakaro hoti. Antarāyakaro samāno ahitānukampī hoti, ahitānukampissa sapattakaṃ cittaṃ paccupaṭṭhitaṃ hoti, sapattake citte paccupaṭṭhite micchādiṭṭhi hoti. Evameva kho, lohicca, yo evaṃ vadeyya – ‘‘idha samaṇo vā brāhmaṇo vā kusalaṃ dhammaṃ adhigaccheyya, kusalaṃ dhammaṃ adhigantvā na parassa āroceyya, kiñhi paro parassa karissati. Seyyathāpi nāma purāṇaṃ bandhanaṃ chinditvā aññaṃ navaṃ bandhanaṃ kareyya…pe… karissatī’’ti. Evaṃvādī so ye te kulaputtā tathāgatappaveditaṃ dhammavinayaṃ āgamma evarūpaṃ uḷāraṃ visesaṃ adhigacchanti, sotāpattiphalampi sacchikaronti, sakadāgāmiphalampi sacchikaronti, anāgāmiphalampi sacchikaronti, arahattampi sacchikaronti, ye cime dibbā gabbhā paripācenti dibbānaṃ bhavānaṃ abhinibbattiyā, tesaṃ antarāyakaro hoti, antarāyakaro samāno ahitānukampī hoti, ahitānukampissa sapattakaṃ cittaṃ paccupaṭṭhitaṃ hoti, sapattake citte paccupaṭṭhite micchādiṭṭhi hoti. Micchādiṭṭhissa kho ahaṃ, lohicca, dvinnaṃ gatīnaṃ aññataraṃ gatiṃ vadāmi – nirayaṃ vā tiracchānayoniṃ vā.

\paragraph{512.} ‘‘Iti kira, lohicca, yo evaṃ vadeyya – ‘‘rājā pasenadi kosalo kāsikosalaṃ ajjhāvasati; yā kāsikosale samudayasañjāti, rājāva taṃ pasenadi kosalo ekako paribhuñjeyya, na aññesaṃ dadeyyā’’ti. Evaṃvādī so ye rājānaṃ pasenadiṃ kosalaṃ upajīvanti tumhe ceva aññe ca, tesaṃ antarāyakaro hoti. Antarāyakaro samāno ahitānukampī hoti, ahitānukampissa sapattakaṃ cittaṃ paccupaṭṭhitaṃ hoti, sapattake citte paccupaṭṭhite micchādiṭṭhi hoti. Evameva kho, lohicca, yo evaṃ vadeyya – ‘‘idha samaṇo vā brāhmaṇo vā kusalaṃ dhammaṃ adhigaccheyya, kusalaṃ dhammaṃ adhigantvā na parassa āroceyya, kiñhi paro parassa karissati. Seyyathāpi nāma…pe… kiñhi paro parassa karissatī’’ti, evaṃ vādī so ye te kulaputtā tathāgatappaveditaṃ dhammavinayaṃ āgamma evarūpaṃ uḷāraṃ visesaṃ adhigacchanti, sotāpattiphalampi sacchikaronti, sakadāgāmiphalampi sacchikaronti, anāgāmiphalampi sacchikaronti, arahattampi sacchikaronti. Ye cime dibbā gabbhā paripācenti dibbānaṃ bhavānaṃ abhinibbattiyā, tesaṃ antarāyakaro hoti, antarāyakaro samāno ahitānukampī hoti, ahitānukampissa sapattakaṃ cittaṃ paccupaṭṭhitaṃ hoti, sapattake citte paccupaṭṭhite micchādiṭṭhi hoti. Micchādiṭṭhissa kho ahaṃ, lohicca, dvinnaṃ gatīnaṃ aññataraṃ gatiṃ vadāmi – nirayaṃ vā tiracchānayoniṃ vā.

\subsubsection{Tayo codanārahā}

\paragraph{513.} ‘‘Tayo khome, lohicca, satthāro, ye loke codanārahā; yo ca panevarūpe satthāro codeti, sā codanā bhūtā tacchā dhammikā anavajjā. Katame tayo? Idha, lohicca, ekacco satthā yassatthāya agārasmā anagāriyaṃ pabbajito hoti, svāssa sāmaññattho ananuppatto hoti. So taṃ sāmaññatthaṃ ananupāpuṇitvā sāvakānaṃ dhammaṃ deseti – ‘‘idaṃ vo hitāya idaṃ vo sukhāyā’’ti. Tassa sāvakā na sussūsanti, na sotaṃ odahanti, na aññā cittaṃ upaṭṭhapenti, vokkamma ca satthusāsanā vattanti. So evamassa codetabbo – ‘‘āyasmā kho yassatthāya agārasmā anagāriyaṃ pabbajito, so te sāmaññattho ananuppatto, taṃ tvaṃ sāmaññatthaṃ ananupāpuṇitvā sāvakānaṃ dhammaṃ desesi – ‘idaṃ vo hitāya idaṃ vo sukhāyā’ti. Tassa te sāvakā na sussūsanti, na sotaṃ odahanti, na aññā cittaṃ upaṭṭhapenti, vokkamma ca satthusāsanā vattanti. Seyyathāpi nāma osakkantiyā vā ussakkeyya, parammukhiṃ vā āliṅgeyya, evaṃ sampadamidaṃ pāpakaṃ lobhadhammaṃ vadāmi – kiñhi paro parassa karissatī’’ti. Ayaṃ kho, lohicca, paṭhamo satthā, yo loke codanāraho; yo ca panevarūpaṃ satthāraṃ codeti, sā codanā bhūtā tacchā dhammikā anavajjā.

\paragraph{514.} ‘‘Puna caparaṃ, lohicca, idhekacco satthā yassatthāya agārasmā anagāriyaṃ pabbajito hoti, svāssa sāmaññattho ananuppatto hoti. So taṃ sāmaññatthaṃ ananupāpuṇitvā sāvakānaṃ dhammaṃ deseti – ‘‘idaṃ vo hitāya, idaṃ vo sukhāyā’’ti. Tassa sāvakā sussūsanti, sotaṃ odahanti, aññā cittaṃ upaṭṭhapenti, na ca vokkamma satthusāsanā vattanti. So evamassa codetabbo – ‘‘āyasmā kho yassatthāya agārasmā anagāriyaṃ pabbajito, so te sāmaññattho ananuppatto. Taṃ tvaṃ sāmaññatthaṃ ananupāpuṇitvā sāvakānaṃ dhammaṃ desesi – ‘idaṃ vo hitāya idaṃ vo sukhāyā’ti. Tassa te sāvakā sussūsanti, sotaṃ odahanti, aññā cittaṃ upaṭṭhapenti, na ca vokkamma satthusāsanā vattanti. Seyyathāpi nāma sakaṃ khettaṃ ohāya paraṃ khettaṃ niddāyitabbaṃ maññeyya, evaṃ sampadamidaṃ pāpakaṃ lobhadhammaṃ vadāmi – kiñhi paro parassa karissatī’’ti. Ayaṃ kho, lohicca, dutiyo satthā, yo, loke codanāraho; yo ca panevarūpaṃ satthāraṃ codeti, sā codanā bhūtā tacchā dhammikā anavajjā.

\paragraph{515.} ‘‘Puna caparaṃ, lohicca, idhekacco satthā yassatthāya agārasmā anagāriyaṃ pabbajito hoti, svāssa sāmaññattho anuppatto hoti. So taṃ sāmaññatthaṃ anupāpuṇitvā sāvakānaṃ dhammaṃ deseti – ‘‘idaṃ vo hitāya idaṃ vo sukhāyā’’ti. Tassa sāvakā na sussūsanti, na sotaṃ odahanti, na aññā cittaṃ upaṭṭhapenti, vokkamma ca satthusāsanā vattanti. So evamassa codetabbo – ‘‘āyasmā kho yassatthāya agārasmā anagāriyaṃ pabbajito, so te sāmaññattho anuppatto. Taṃ tvaṃ sāmaññatthaṃ anupāpuṇitvā sāvakānaṃ dhammaṃ desesi – ‘idaṃ vo hitāya idaṃ vo sukhāyā’ti. Tassa te sāvakā na sussūsanti, na sotaṃ odahanti, na aññā cittaṃ upaṭṭhapenti, vokkamma ca satthusāsanā vattanti. Seyyathāpi nāma purāṇaṃ bandhanaṃ chinditvā aññaṃ navaṃ bandhanaṃ kareyya, evaṃ sampadamidaṃ pāpakaṃ lobhadhammaṃ vadāmi, kiñhi paro parassa karissatī’’ti. Ayaṃ kho, lohicca, tatiyo satthā, yo loke codanāraho; yo ca panevarūpaṃ satthāraṃ codeti, sā codanā bhūtā tacchā dhammikā anavajjā. Ime kho, lohicca, tayo satthāro, ye loke codanārahā, yo ca panevarūpe satthāro codeti, sā codanā bhūtā tacchā dhammikā anavajjāti.

\subsubsection{Nacodanārahasatthu}

\paragraph{516.} Evaṃ vutte, lohicco brāhmaṇo bhagavantaṃ etadavoca – ‘‘atthi pana, bho gotama, koci satthā, yo loke nacodanāraho’’ti? ‘‘Atthi kho, lohicca, satthā, yo loke nacodanāraho’’ti. ‘‘Katamo pana so, bho gotama, satthā, yo loke nacodanāraho’’ti? ‘‘Idha, lohicca, tathāgato loke uppajjati arahaṃ, sammāsambuddho…pe… (yathā 190212 anucchedesu evaṃ vitthāretabbaṃ). Evaṃ kho, lohicca, bhikkhu sīlasampanno hoti… pe… paṭhamaṃ jhānaṃ upasampajja viharati… yasmiṃ kho, lohicca, satthari sāvako evarūpaṃ uḷāraṃ visesaṃ adhigacchati, ayampi kho, lohicca, satthā, yo loke nacodanāraho. Yo ca panevarūpaṃ satthāraṃ codeti, sā codanā abhūtā atacchā adhammikā sāvajjā…pe… dutiyaṃ jhānaṃ…pe… tatiyaṃ jhānaṃ…pe… catutthaṃ jhānaṃ upasampajja viharati. Yasmiṃ kho, lohicca, satthari sāvako evarūpaṃ uḷāraṃ visesaṃ adhigacchati, ayampi kho, lohicca, satthā, yo loke nacodanāraho, yo ca panevarūpaṃ satthāraṃ codeti, sā codanā abhūtā atacchā adhammikā sāvajjā… ñāṇadassanāya cittaṃ abhinīharati abhininnāmeti…pe… yasmiṃ kho, lohicca, satthari sāvako evarūpaṃ uḷāraṃ visesaṃ adhigacchati, ayampi kho, lohicca, satthā, yo loke nacodanāraho, yo ca panevarūpaṃ satthāraṃ codeti, sā codanā abhūtā atacchā adhammikā sāvajjā… nāparaṃ itthattāyāti pajānāti. Yasmiṃ kho, lohicca, satthari sāvako evarūpaṃ uḷāraṃ visesaṃ adhigacchati, ayampi kho, lohicca, satthā, yo loke nacodanāraho, yo ca panevarūpaṃ satthāraṃ codeti, sā codanā abhūtā atacchā adhammikā sāvajjā’’ti.

\paragraph{517.} Evaṃ vutte, lohicco brāhmaṇo bhagavantaṃ etadavoca – ‘‘seyyathāpi, bho gotama, puriso purisaṃ narakapapātaṃ patantaṃ kesesu gahetvā uddharitvā thale patiṭṭhapeyya, evamevāhaṃ bhotā gotamena narakapapātaṃ papatanto uddharitvā thale patiṭṭhāpito. Abhikkantaṃ, bho gotama, abhikkantaṃ, bho gotama, seyyathāpi, bho gotama, nikkujjitaṃ vā ukkujjeyya, paṭicchannaṃ vā vivareyya, mūḷhassa vā maggaṃ ācikkheyya, andhakāre vā telapajjotaṃ dhāreyya, ‘cakkhumanto rūpāni dakkhantī’ti. Evamevaṃ bhotā gotamena anekapariyāyena dhammo pakāsito. Esāhaṃ bhavantaṃ gotamaṃ saraṇaṃ gacchāmi dhammañca bhikkhusaṅghañca. Upāsakaṃ maṃ bhavaṃ gotamo dhāretu ajjatagge pāṇupetaṃ saraṇaṃ gata’’nti.

\xsectionEnd{Lohiccasuttaṃ niṭṭhitaṃ dvādasamaṃ.}


\clearpage
\section{Tevijjasuttaṃ}

\paragraph{518.} Evaṃ me sutaṃ – ekaṃ samayaṃ bhagavā kosalesu cārikaṃ caramāno mahatā bhikkhusaṅghena saddhiṃ pañcamattehi bhikkhusatehi yena manasākaṭaṃ nāma kosalānaṃ brāhmaṇagāmo tadavasari. Tatra sudaṃ bhagavā manasākaṭe viharati uttarena manasākaṭassa aciravatiyā nadiyā tīre ambavane.

\paragraph{519.} Tena kho pana samayena sambahulā abhiññātā abhiññātā brāhmaṇamahāsālā manasākaṭe paṭivasanti, seyyathidaṃ – caṅkī brāhmaṇo tārukkho brāhmaṇo pokkharasāti brāhmaṇo jāṇusoṇi brāhmaṇo todeyyo brāhmaṇo aññe ca abhiññātā abhiññātā brāhmaṇamahāsālā.

\paragraph{520.} Atha kho vāseṭṭhabhāradvājānaṃ māṇavānaṃ jaṅghavihāraṃ anucaṅkamantānaṃ anuvicarantānaṃ maggāmagge kathā udapādi. Atha kho vāseṭṭho māṇavo evamāha – ‘‘ayameva ujumaggo, ayamañjasāyano niyyāniko niyyāti takkarassa brahmasahabyatāya, yvāyaṃ akkhāto brāhmaṇena pokkharasātinā’’ti. Bhāradvājopi māṇavo evamāha – ‘‘ayameva ujumaggo, ayamañjasāyano niyyāniko, niyyāti takkarassa brahmasahabyatāya, yvāyaṃ akkhāto brāhmaṇena tārukkhenā’’ti. Neva kho asakkhi vāseṭṭho māṇavo bhāradvājaṃ māṇavaṃ saññāpetuṃ, na pana asakkhi bhāradvājo māṇavopi vāseṭṭhaṃ māṇavaṃ saññāpetuṃ.

\paragraph{521.} Atha kho vāseṭṭho māṇavo bhāradvājaṃ māṇavaṃ āmantesi – ‘‘ayaṃ kho, bhāradvāja, samaṇo gotamo sakyaputto sakyakulā pabbajito manasākaṭe viharati uttarena manasākaṭassa aciravatiyā nadiyā tīre ambavane. Taṃ kho pana bhavantaṃ gotamaṃ evaṃ kalyāṇo kittisaddo abbhuggato – ‘‘itipi so bhagavā arahaṃ sammāsambuddho vijjācaraṇasampanno sugato lokavidū anuttaro purisadammasārathi satthā devamanussānaṃ buddho bhagavā’’ti. Āyāma, bho bhāradvāja, yena samaṇo gotamo tenupasaṅkamissāma; upasaṅkamitvā etamatthaṃ samaṇaṃ gotamaṃ pucchissāma. Yathā no samaṇo gotamo byākarissati, tathā naṃ dhāressāmā’’ti. ‘‘Evaṃ, bho’’ti kho bhāradvājo māṇavo vāseṭṭhassa māṇavassa paccassosi.

\subsubsection{Maggāmaggakathā}

\paragraph{522.} Atha kho vāseṭṭhabhāradvājā māṇavā yena bhagavā tenupasaṅkamiṃsu; upasaṅkamitvā bhagavatā saddhiṃ sammodiṃsu. Sammodanīyaṃ kathaṃ sāraṇīyaṃ vītisāretvā ekamantaṃ nisīdiṃsu. Ekamantaṃ nisinno kho vāseṭṭho māṇavo bhagavantaṃ etadavoca – ‘‘idha, bho gotama, amhākaṃ jaṅghavihāraṃ anucaṅkamantānaṃ anuvicarantānaṃ maggāmagge kathā udapādi. Ahaṃ evaṃ vadāmi – ‘ayameva ujumaggo, ayamañjasāyano niyyāniko niyyāti takkarassa brahmasahabyatāya, yvāyaṃ akkhāto brāhmaṇena pokkharasātinā’ti. Bhāradvājo māṇavo evamāha – ‘ayameva ujumaggo ayamañjasāyano niyyāniko niyyāti takkarassa brahmasahabyatāya, yvāyaṃ akkhāto brāhmaṇena tārukkhenā’ti. Ettha, bho gotama, attheva viggaho, atthi vivādo, atthi nānāvādo’’ti.

\paragraph{523.} ‘‘Iti kira, vāseṭṭha, tvaṃ evaṃ vadesi – ‘‘ayameva ujumaggo, ayamañjasāyano niyyāniko niyyāti takkarassa brahmasahabyatāya, yvāyaṃ akkhāto brāhmaṇena pokkharasātinā’’ti. Bhāradvājo māṇavo evamāha – ‘‘ayameva ujumaggo ayamañjasāyano niyyāniko niyyāti takkarassa brahmasahabyatāya, yvāyaṃ akkhāto brāhmaṇena tārukkhenā’’ti. Atha kismiṃ pana vo, vāseṭṭha, viggaho, kismiṃ vivādo, kismiṃ nānāvādo’’ti?

\paragraph{524.} ‘‘Maggāmagge, bho gotama. Kiñcāpi, bho gotama, brāhmaṇā nānāmagge paññapenti, addhariyā brāhmaṇā tittiriyā brāhmaṇā chandokā brāhmaṇā bavhārijjhā brāhmaṇā, atha kho sabbāni tāni niyyānikā niyyanti takkarassa brahmasahabyatāya. ‘‘Seyyathāpi, bho gotama, gāmassa vā nigamassa vā avidūre bahūni cepi nānāmaggāni bhavanti, atha kho sabbāni tāni gāmasamosaraṇāni bhavanti; evameva kho, bho gotama, kiñcāpi brāhmaṇā nānāmagge paññapenti, addhariyā brāhmaṇā tittiriyā brāhmaṇā chandokā brāhmaṇā bavhārijjhā brāhmaṇā, atha kho sabbāni tāni niyyānikā niyyanti takkarassa brahmasahabyatāyā’’ti.

\subsubsection{Vāseṭṭhamāṇavānuyogo}

\paragraph{525.} ‘‘Niyyantīti vāseṭṭha vadesi’’? ‘‘Niyyantīti, bho gotama, vadāmi’’. ‘‘Niyyantīti, vāseṭṭha, vadesi’’? ‘‘Niyyantīti, bho gotama, vadāmi’’. ‘‘Niyyantīti, vāseṭṭha, vadesi’’? ‘‘Niyyantī’’ti, bho gotama, vadāmi’’. ‘‘Kiṃ pana, vāseṭṭha, atthi koci tevijjānaṃ brāhmaṇānaṃ ekabrāhmaṇopi, yena brahmā sakkhidiṭṭho’’ti? ‘‘No hidaṃ, bho gotama’’. ‘‘Kiṃ pana, vāseṭṭha, atthi koci tevijjānaṃ brāhmaṇānaṃ ekācariyopi, yena brahmā sakkhidiṭṭho’’ti? ‘‘No hidaṃ, bho gotama’’. ‘‘Kiṃ pana, vāseṭṭha, atthi koci tevijjānaṃ brāhmaṇānaṃ ekācariyapācariyopi, yena brahmā sakkhidiṭṭho’’ti? ‘‘No hidaṃ, bho gotama’’. ‘‘Kiṃ pana, vāseṭṭha, atthi koci tevijjānaṃ brāhmaṇānaṃ yāva sattamā ācariyāmahayugā\footnote{sattamācariyamahayugā (syā.)} yena brahmā sakkhidiṭṭho’’ti? ‘‘No hidaṃ, bho gotama’’.

\paragraph{526.} ‘‘Kiṃ pana, vāseṭṭha, yepi tevijjānaṃ brāhmaṇānaṃ pubbakā isayo mantānaṃ kattāro mantānaṃ pavattāro, yesamidaṃ etarahi tevijjā brāhmaṇā porāṇaṃ mantapadaṃ gītaṃ pavuttaṃ samihitaṃ\footnote{samīhitaṃ (syā.)}, tadanugāyanti, tadanubhāsanti, bhāsitamanubhāsanti, vācitamanuvācenti, seyyathidaṃ – aṭṭhako vāmako vāmadevo vessāmitto yamataggi aṅgīraso bhāradvājo vāseṭṭho kassapo bhagu. Tepi evamāhaṃsu – ‘mayametaṃ jānāma, mayametaṃ passāma, yattha vā brahmā, yena vā brahmā, yahiṃ vā brahmā’’’ti? ‘‘No hidaṃ, bho gotama’’.

\paragraph{527.} ‘‘Iti kira, vāseṭṭha, natthi koci tevijjānaṃ brāhmaṇānaṃ ekabrāhmaṇopi, yena brahmā sakkhidiṭṭho. Natthi koci tevijjānaṃ brāhmaṇānaṃ ekācariyopi, yena brahmā sakkhidiṭṭho. Natthi koci tevijjānaṃ brāhmaṇānaṃ ekācariyapācariyopi, yena brahmā sakkhidiṭṭho. Natthi koci tevijjānaṃ brāhmaṇānaṃ yāva sattamā ācariyāmahayugā yena brahmā sakkhidiṭṭho. Yepi kira tevijjānaṃ brāhmaṇānaṃ pubbakā isayo mantānaṃ kattāro mantānaṃ pavattāro, yesamidaṃ etarahi tevijjā brāhmaṇā porāṇaṃ mantapadaṃ gītaṃ pavuttaṃ samihitaṃ, tadanugāyanti, tadanubhāsanti, bhāsitamanubhāsanti, vācitamanuvācenti, seyyathidaṃ – aṭṭhako vāmako vāmadevo vessāmitto yamataggi aṅgīraso bhāradvājo vāseṭṭho kassapo bhagu, tepi na evamāhaṃsu – ‘mayametaṃ jānāma, mayametaṃ passāma, yattha vā brahmā, yena vā brahmā, yahiṃ vā brahmā’ti. Teva tevijjā brāhmaṇā evamāhaṃsu – ‘yaṃ na jānāma, yaṃ na passāma, tassa sahabyatāya maggaṃ desema. Ayameva ujumaggo ayamañjasāyano niyyāniko, niyyāti takkarassa brahmasahabyatāyā’’’ti.

\paragraph{528.} ‘‘Taṃ kiṃ maññasi, vāseṭṭha, nanu evaṃ sante tevijjānaṃ brāhmaṇānaṃ appāṭihīrakataṃ bhāsitaṃ sampajjatī’’ti? ‘‘Addhā kho, bho gotama, evaṃ sante tevijjānaṃ brāhmaṇānaṃ appāṭihīrakataṃ bhāsitaṃ sampajjatī’’ti. ‘‘Sādhu, vāseṭṭha, te vata\footnote{teva (ka.)}, vāseṭṭha, tevijjā brāhmaṇā yaṃ na jānanti, yaṃ na passanti, tassa sahabyatāya maggaṃ desessanti. ‘Ayameva ujumaggo, ayamañjasāyano niyyāniko, niyyāti takkarassa brahmasahabyatāyā’ti, netaṃ ṭhānaṃ vijjati.

\paragraph{529.} ‘‘Seyyathāpi, vāseṭṭha, andhaveṇi paramparasaṃsattā purimopi na passati, majjhimopi na passati, pacchimopi na passati. Evameva kho, vāseṭṭha, andhaveṇūpamaṃ maññe tevijjānaṃ brāhmaṇānaṃ bhāsitaṃ, purimopi na passati, majjhimopi na passati, pacchimopi na passati. Tesamidaṃ tevijjānaṃ brāhmaṇānaṃ bhāsitaṃ hassakaññeva sampajjati, nāmakaññeva sampajjati, rittakaññeva sampajjati, tucchakaññeva sampajjati.

\paragraph{530.} ‘‘Taṃ kiṃ maññasi, vāseṭṭha, passanti tevijjā brāhmaṇā candimasūriye, aññe cāpi bahujanā, yato ca candimasūriyā uggacchanti, yattha ca ogacchanti, āyācanti thomayanti pañjalikā namassamānā anuparivattantī’’ti? ‘‘Evaṃ, bho gotama, passanti tevijjā brāhmaṇā candimasūriye, aññe cāpi bahujanā, yato ca candimasūriyā uggacchanti, yattha ca ogacchanti, āyācanti thomayanti pañjalikā namassamānā anuparivattantī’’ti.

\paragraph{531.} ‘‘Taṃ kiṃ maññasi, vāseṭṭha, yaṃ passanti tevijjā brāhmaṇā candimasūriye, aññe cāpi bahujanā, yato ca candimasūriyā uggacchanti, yattha ca ogacchanti, āyācanti thomayanti pañjalikā namassamānā anuparivattanti, pahonti tevijjā brāhmaṇā candimasūriyānaṃ sahabyatāya maggaṃ desetuṃ – ‘‘ayameva ujumaggo, ayamañjasāyano niyyāniko, niyyāti takkarassa candimasūriyānaṃ sahabyatāyā’’ti? ‘‘No hidaṃ, bho gotama’’. ‘‘Iti kira, vāseṭṭha, yaṃ passanti tevijjā brāhmaṇā candimasūriye, aññe cāpi bahujanā, yato ca candimasūriyā uggacchanti, yattha ca ogacchanti, āyācanti thomayanti pañjalikā namassamānā anuparivattanti, tesampi nappahonti candimasūriyānaṃ sahabyatāya maggaṃ desetuṃ – ‘‘ayameva ujumaggo, ayamañjasāyano niyyāniko, niyyāti takkarassa candimasūriyānaṃ sahabyatāyā’’ti.

\paragraph{532.} ‘‘Iti pana\footnote{kiṃ pana (sī. syā. pī.)} na kira tevijjehi brāhmaṇehi brahmā sakkhidiṭṭho. Napi kira tevijjānaṃ brāhmaṇānaṃ ācariyehi brahmā sakkhidiṭṭho. Napi kira tevijjānaṃ brāhmaṇānaṃ ācariyapācariyehi brahmā sakkhidiṭṭho. Napi kira tevijjānaṃ brāhmaṇānaṃ yāva sattamā\footnote{sattamehi (?)} ācariyāmahayugehi brahmā sakkhidiṭṭho. Yepi kira tevijjānaṃ brāhmaṇānaṃ pubbakā isayo mantānaṃ kattāro mantānaṃ pavattāro, yesamidaṃ etarahi tevijjā brāhmaṇā porāṇaṃ mantapadaṃ gītaṃ pavuttaṃ samihitaṃ, tadanugāyanti, tadanubhāsanti, bhāsitamanubhāsanti, vācitamanuvācenti, seyyathidaṃ – aṭṭhako vāmako vāmadevo vessāmitto yamataggi aṅgīraso bhāradvājo vāseṭṭho kassapo bhagu, tepi na evamāhaṃsu – ‘‘mayametaṃ jānāma, mayametaṃ passāma, yattha vā brahmā, yena vā brahmā, yahiṃ vā brahmā’’ti. Teva tevijjā brāhmaṇā evamāhaṃsu – ‘‘yaṃ na jānāma, yaṃ na passāma, tassa sahabyatāya maggaṃ desema – ayameva ujumaggo ayamañjasāyano niyyāniko niyyāti takkarassa brahmasahabyatāyā’’ti.

\paragraph{533.} ‘‘Taṃ kiṃ maññasi, vāseṭṭha, nanu evaṃ sante tevijjānaṃ brāhmaṇānaṃ appāṭihīrakataṃ bhāsitaṃ sampajjatī’’ti? ‘‘Addhā kho, bho gotama, evaṃ sante tevijjānaṃ brāhmaṇānaṃ appāṭihīrakataṃ bhāsitaṃ sampajjatī’’ti. ‘‘Sādhu, vāseṭṭha, te vata, vāseṭṭha, tevijjā brāhmaṇā yaṃ na jānanti, yaṃ na passanti, tassa sahabyatāya maggaṃ desessanti – ‘‘ayameva ujumaggo, ayamañjasāyano niyyāniko, niyyāti takkarassa brahmasahabyatāyā’’ti, netaṃ ṭhānaṃ vijjati.

\subsubsection{Janapadakalyāṇīupamā}

\paragraph{534.} ‘‘Seyyathāpi, vāseṭṭha, puriso evaṃ vadeyya – ‘‘ahaṃ yā imasmiṃ janapade janapadakalyāṇī, taṃ icchāmi, taṃ kāmemī’’ti. Tamenaṃ evaṃ vadeyyuṃ – ‘‘ambho purisa, yaṃ tvaṃ janapadakalyāṇiṃ icchasi kāmesi, jānāsi taṃ janapadakalyāṇiṃ – khattiyī vā brāhmaṇī vā vessī vā suddī vā’’ti? Iti puṭṭho ‘‘no’’ti vadeyya. ‘‘Tamenaṃ evaṃ vadeyyuṃ – ‘‘ambho purisa, yaṃ tvaṃ janapadakalyāṇiṃ icchasi kāmesi, jānāsi taṃ janapadakalyāṇiṃ – evaṃnāmā evaṃgottāti vā, dīghā vā rassā vā majjhimā vā kāḷī vā sāmā vā maṅguracchavī vāti, amukasmiṃ gāme vā nigame vā nagare vā’’ti? Iti puṭṭho ‘no’ti vadeyya. Tamenaṃ evaṃ vadeyyuṃ – ‘‘ambho purisa, yaṃ tvaṃ na jānāsi na passasi, taṃ tvaṃ icchasi kāmesī’’ti? Iti puṭṭho ‘‘āmā’’ti vadeyya.

\paragraph{535.} ‘‘Taṃ kiṃ maññasi, vāseṭṭha, nanu evaṃ sante tassa purisassa appāṭihīrakataṃ bhāsitaṃ sampajjatī’’ti? ‘‘Addhā kho, bho gotama, evaṃ sante tassa purisassa appāṭihīrakataṃ bhāsitaṃ sampajjatī’’ti.

\paragraph{536.} ‘‘Evameva kho, vāseṭṭha, na kira tevijjehi brāhmaṇehi brahmā sakkhidiṭṭho, napi kira tevijjānaṃ brāhmaṇānaṃ ācariyehi brahmā sakkhidiṭṭho, napi kira tevijjānaṃ brāhmaṇānaṃ ācariyapācariyehi brahmā sakkhidiṭṭho. Napi kira tevijjānaṃ brāhmaṇānaṃ yāva sattamā ācariyāmahayugehi brahmā sakkhidiṭṭho. Yepi kira tevijjānaṃ brāhmaṇānaṃ pubbakā isayo mantānaṃ kattāro mantānaṃ pavattāro, yesamidaṃ etarahi tevijjā brāhmaṇā porāṇaṃ mantapadaṃ gītaṃ pavuttaṃ samihitaṃ, tadanugāyanti, tadanubhāsanti, bhāsitamanubhāsanti, vācitamanuvācenti, seyyathidaṃ – aṭṭhako vāmako vāmadevo vessāmitto yamataggi aṅgīraso bhāradvājo vāseṭṭho kassapo bhagu, tepi na evamāhaṃsu – ‘‘mayametaṃ jānāma, mayametaṃ passāma, yattha vā brahmā, yena vā brahmā, yahiṃ vā brahmā’’ti. Teva tevijjā brāhmaṇā evamāhaṃsu – ‘‘yaṃ na jānāma, yaṃ na passāma, tassa sahabyatāya maggaṃ desema – ayameva ujumaggo ayamañjasāyano niyyāniko niyyāti takkarassa brahmasahabyatāyā’’ti.

\paragraph{537.} ‘‘Taṃ kiṃ maññasi, vāseṭṭha, nanu evaṃ sante tevijjānaṃ brāhmaṇānaṃ appāṭihīrakataṃ bhāsitaṃ sampajjatī’’ti? ‘‘Addhā kho, bho gotama, evaṃ sante tevijjānaṃ brāhmaṇānaṃ appāṭihīrakataṃ bhāsitaṃ sampajjatī’’ti. ‘‘Sādhu, vāseṭṭha, te vata, vāseṭṭha, tevijjā brāhmaṇā yaṃ na jānanti, yaṃ na passanti, tassa sahabyatāya maggaṃ desessanti – ayameva ujumaggo ayamañjasāyano niyyāniko niyyāti takkarassa brahmasahabyatāyāti netaṃ ṭhānaṃ vijjati.

\subsubsection{Nisseṇīupamā}

\paragraph{538.} ‘‘Seyyathāpi, vāseṭṭha, puriso cātumahāpathe nisseṇiṃ kareyya – pāsādassa ārohaṇāya. Tamenaṃ evaṃ vadeyyuṃ – ‘‘ambho purisa, yassa tvaṃ\footnote{yaṃ tvaṃ (syā.)} pāsādassa ārohaṇāya nisseṇiṃ karosi, jānāsi taṃ pāsādaṃ – puratthimāya vā disāya dakkhiṇāya vā disāya pacchimāya vā disāya uttarāya vā disāya ucco vā nīco vā majjhimo vā’’ti? Iti puṭṭho ‘‘no’’ti vadeyya. ‘‘Tamenaṃ evaṃ vadeyyuṃ – ‘‘ambho purisa, yaṃ tvaṃ na jānāsi, na passasi, tassa tvaṃ pāsādassa ārohaṇāya nisseṇiṃ karosī’’ti? Iti puṭṭho ‘‘āmā’’ti vadeyya.

\paragraph{539.} ‘‘Taṃ kiṃ maññasi, vāseṭṭha, nanu evaṃ sante tassa purisassa appāṭihīrakataṃ bhāsitaṃ sampajjatī’’ti? ‘‘Addhā kho, bho gotama, evaṃ sante tassa purisassa appāṭihīrakataṃ bhāsitaṃ sampajjatī’’ti.

\paragraph{540.} ‘‘Evameva kho, vāseṭṭha, na kira tevijjehi brāhmaṇehi brahmā sakkhidiṭṭho, napi kira tevijjānaṃ brāhmaṇānaṃ ācariyehi brahmā sakkhidiṭṭho, napi kira tevijjānaṃ brāhmaṇānaṃ ācariyapācariyehi brahmā sakkhidiṭṭho, napi kira tevijjānaṃ brāhmaṇānaṃ yāva sattamā ācariyāmahayugehi brahmā sakkhidiṭṭho. Yepi kira tevijjānaṃ brāhmaṇānaṃ pubbakā isayo mantānaṃ kattāro mantānaṃ pavattāro, yesamidaṃ etarahi tevijjā brāhmaṇā porāṇaṃ mantapadaṃ gītaṃ pavuttaṃ samihitaṃ, tadanugāyanti, tadanubhāsanti, bhāsitamanubhāsanti, vācitamanuvācenti, seyyathidaṃ – aṭṭhako vāmako vāmadevo vessāmitto yamataggi aṅgīraso bhāradvājo vāseṭṭho kassapo bhagu, tepi na evamāhaṃsu – mayametaṃ jānāma, mayametaṃ passāma, yattha vā brahmā, yena vā brahmā, yahiṃ vā brahmāti. Teva tevijjā brāhmaṇā evamāhaṃsu – ‘‘yaṃ na jānāma, yaṃ na passāma, tassa sahabyatāya maggaṃ desema, ayameva ujumaggo ayamañjasāyano niyyāniko niyyāti takkarassa brahmasahabyatāyā’’ti.

\paragraph{541.} ‘‘Taṃ kiṃ maññasi, vāseṭṭha, nanu evaṃ sante tevijjānaṃ brāhmaṇānaṃ appāṭihīrakataṃ bhāsitaṃ sampajjatī’’ti? ‘‘Addhā kho, bho gotama, evaṃ sante tevijjānaṃ brāhmaṇānaṃ appāṭihīrakataṃ bhāsitaṃ sampajjatī’’ti. ‘‘Sādhu, vāseṭṭha. Te vata, vāseṭṭha, tevijjā brāhmaṇā yaṃ na jānanti, yaṃ na passanti, tassa sahabyatāya maggaṃ desessanti. Ayameva ujumaggo ayamañjasāyano niyyāniko niyyāti takkarassa brahmasabyatāyāti, netaṃ ṭhānaṃ vijjati.

\subsubsection{Aciravatīnadīupamā}

\paragraph{542.} ‘‘Seyyathāpi, vāseṭṭha, ayaṃ aciravatī nadī pūrā udakassa samatittikā kākapeyyā. Atha puriso āgaccheyya pāratthiko pāragavesī pāragāmī pāraṃ taritukāmo. So orime tīre ṭhito pārimaṃ tīraṃ avheyya – ‘‘ehi pārāpāraṃ, ehi pārāpāra’’nti.

\paragraph{543.} ‘‘Taṃ kiṃ maññasi, vāseṭṭha, api nu tassa purisassa avhāyanahetu vā āyācanahetu vā patthanahetu vā abhinandanahetu vā aciravatiyā nadiyā pārimaṃ tīraṃ orimaṃ tīraṃ āgaccheyyā’’ti? ‘‘No hidaṃ, bho gotama’’.

\paragraph{544.} ‘‘Evameva kho, vāseṭṭha, tevijjā brāhmaṇā ye dhammā brāhmaṇakārakā te dhamme pahāya vattamānā, ye dhammā abrāhmaṇakārakā te dhamme samādāya vattamānā evamāhaṃsu – ‘‘indamavhayāma, somamavhayāma, varuṇamavhayāma, īsānamavhayāma, pajāpatimavhayāma, brahmamavhayāma, mahiddhimavhayāma, yamamavhayāmā’’ti. ‘‘Te vata, vāseṭṭha, tevijjā brāhmaṇā ye dhammā brāhmaṇakārakā te dhamme pahāya vattamānā, ye dhammā abrāhmaṇakārakā te dhamme samādāya vattamānā avhāyanahetu vā āyācanahetu vā patthanahetu vā abhinandanahetu vā kāyassa bhedā paraṃ maraṇā brahmānaṃ sahabyūpagā bhavissantī’’ti, netaṃ ṭhānaṃ vijjati.

\paragraph{545.} ‘‘Seyyathāpi, vāseṭṭha, ayaṃ aciravatī nadī pūrā udakassa samatittikā kākapeyyā. Atha puriso āgaccheyya pāratthiko pāragavesī pāragāmī pāraṃ taritukāmo. So orime tīre daḷhāya anduyā pacchābāhaṃ gāḷhabandhanaṃ baddho. ‘‘Taṃ kiṃ maññasi, vāseṭṭha, api nu so puriso aciravatiyā nadiyā orimā tīrā pārimaṃ tīraṃ gaccheyyā’’ti? ‘‘No hidaṃ, bho gotama’’.

\paragraph{546.} ‘‘Evameva kho, vāseṭṭha, pañcime kāmaguṇā ariyassa vinaye andūtipi vuccanti, bandhanantipi vuccanti. Katame pañca? Cakkhuviññeyyā rūpā iṭṭhā kantā manāpā piyarūpā kāmūpasaṃhitā rajanīyā. Sotaviññeyyā saddā…pe… ghānaviññeyyā gandhā… jivhāviññeyyā rasā… kāyaviññeyyā phoṭṭhabbā iṭṭhā kantā manāpā piyarūpā kāmūpasaṃhitā rajanīyā. ‘‘Ime kho, vāseṭṭha, pañca kāmaguṇā ariyassa vinaye andūtipi vuccanti, bandhanantipi vuccanti. Ime kho vāseṭṭha pañca kāmaguṇe tevijjā brāhmaṇā gadhitā mucchitā ajjhopannā anādīnavadassāvino anissaraṇapaññā paribhuñjanti. Te vata, vāseṭṭha, tevijjā brāhmaṇā ye dhammā brāhmaṇakārakā, te dhamme pahāya vattamānā, ye dhammā abrāhmaṇakārakā, te dhamme samādāya vattamānā pañca kāmaguṇe gadhitā mucchitā ajjhopannā anādīnavadassāvino anissaraṇapaññā paribhuñjantā kāmandubandhanabaddhā kāyassa bhedā paraṃ maraṇā brahmānaṃ sahabyūpagā bhavissantī’’ti, netaṃ ṭhānaṃ vijjati.

\paragraph{547.} ‘‘Seyyathāpi, vāseṭṭha, ayaṃ aciravatī nadī pūrā udakassa samatittikā kākapeyyā. Atha puriso āgaccheyya pāratthiko pāragavesī pāragāmī pāraṃ taritukāmo. So orime tīre sasīsaṃ pārupitvā nipajjeyya. ‘‘Taṃ kiṃ maññasi, vāseṭṭha, api nu so puriso aciravatiyā nadiyā orimā tīrā pārimaṃ tīraṃ gaccheyyā’’ti? ‘‘No hidaṃ, bho gotama’’.

\paragraph{548.} ‘‘Evameva kho, vāseṭṭha, pañcime nīvaraṇā ariyassa vinaye āvaraṇātipi vuccanti, nīvaraṇātipi vuccanti, onāhanātipi vuccanti, pariyonāhanātipi vuccanti. Katame pañca? Kāmacchandanīvaraṇaṃ, byāpādanīvaraṇaṃ, thinamiddhanīvaraṇaṃ, uddhaccakukkuccanīvaraṇaṃ, vicikicchānīvaraṇaṃ. Ime kho, vāseṭṭha, pañca nīvaraṇā ariyassa vinaye āvaraṇātipi vuccanti, nīvaraṇātipi vuccanti, onāhanātipi vuccanti, pariyonāhanātipi vuccanti.

\paragraph{549.} ‘‘Imehi kho, vāseṭṭha, pañcahi nīvaraṇehi tevijjā brāhmaṇā āvuṭā nivuṭā onaddhā\footnote{ophuṭā (sī. ka.), ophutā (syā.)} pariyonaddhā. Te vata, vāseṭṭha, tevijjā brāhmaṇā ye dhammā brāhmaṇakārakā te dhamme pahāya vattamānā, ye dhammā abrāhmaṇakārakā te dhamme samādāya vattamānā pañcahi nīvaraṇehi āvuṭā nivuṭā onaddhā pariyonaddhā\footnote{pariyonaddhā, te (syā. ka.)} kāyassa bhedā paraṃ maraṇā brahmānaṃ sahabyūpagā bhavissantī’’ti, netaṃ ṭhānaṃ vijjati.

\subsubsection{Saṃsandanakathā}

\paragraph{550.} ‘‘Taṃ kiṃ maññasi, vāseṭṭha, kinti te sutaṃ brāhmaṇānaṃ vuddhānaṃ mahallakānaṃ ācariyapācariyānaṃ bhāsamānānaṃ, sapariggaho vā brahmā apariggaho vā’’ti? ‘‘Apariggaho, bho gotama’’. ‘‘Saveracitto vā averacitto vā’’ti? ‘‘Averacitto, bho gotama’’. ‘‘Sabyāpajjacitto vā abyāpajjacitto vā’’ti? ‘‘Abyāpajjacitto, bho gotama’’. ‘‘Saṃkiliṭṭhacitto vā asaṃkiliṭṭhacitto vā’’ti? ‘‘Asaṃkiliṭṭhacitto, bho gotama’’. ‘‘Vasavattī vā avasavattī vā’’ti? ‘‘Vasavattī, bho gotama’’. ‘‘Taṃ kiṃ maññasi, vāseṭṭha, sapariggahā vā tevijjā brāhmaṇā apariggahā vā’’ti? ‘‘Sapariggahā, bho gotama’’. ‘‘Saveracittā vā averacittā vā’’ti? ‘‘Saveracittā, bho gotama’’. ‘‘Sabyāpajjacittā vā abyāpajjacittā vā’’ti? ‘‘Sabyāpajjacittā, bho gotama’’. ‘‘Saṃkiliṭṭhacittā vā asaṃkiliṭṭhacittā vā’’ti? ‘‘Saṃkiliṭṭhacittā, bho gotama’’. ‘‘Vasavattī vā avasavattī vā’’ti? ‘‘Avasavattī, bho gotama’’.

\paragraph{551.} ‘‘Iti kira, vāseṭṭha, sapariggahā tevijjā brāhmaṇā apariggaho brahmā. Api nu kho sapariggahānaṃ tevijjānaṃ brāhmaṇānaṃ apariggahena brahmunā saddhiṃ saṃsandati sametī’’ti? ‘‘No hidaṃ, bho gotama’’. ‘‘Sādhu, vāseṭṭha, te vata, vāseṭṭha, sapariggahā tevijjā brāhmaṇā kāyassa bhedā paraṃ maraṇā apariggahassa brahmuno sahabyūpagā bhavissantī’’ti, netaṃ ṭhānaṃ vijjati. ‘‘Iti kira, vāseṭṭha, saveracittā tevijjā brāhmaṇā, averacitto brahmā…pe… sabyāpajjacittā tevijjā brāhmaṇā abyāpajjacitto brahmā… saṃkiliṭṭhacittā tevijjā brāhmaṇā asaṃkiliṭṭhacitto brahmā… avasavattī tevijjā brāhmaṇā vasavattī brahmā, api nu kho avasavattīnaṃ tevijjānaṃ brāhmaṇānaṃ vasavattinā brahmunā saddhiṃ saṃsandati sametī’’ti? ‘‘No hidaṃ, bho gotama’’. ‘‘Sādhu, vāseṭṭha, te vata, vāseṭṭha, avasavattī tevijjā brāhmaṇā kāyassa bhedā paraṃ maraṇā vasavattissa brahmuno sahabyūpagā bhavissantī’’ti, netaṃ ṭhānaṃ vijjati.

\paragraph{552.} ‘‘Idha kho pana te, vāseṭṭha, tevijjā brāhmaṇā āsīditvā\footnote{ādisitvā (ka.)} saṃsīdanti, saṃsīditvā visāraṃ\footnote{visādaṃ (sī. pī.), visattaṃ (syā.)} pāpuṇanti, sukkhataraṃ\footnote{sukkhataraṇaṃ (ka.)} maññe taranti. Tasmā idaṃ tevijjānaṃ brāhmaṇānaṃ tevijjāiriṇantipi vuccati, tevijjāvivanantipi vuccati, tevijjābyasanantipi vuccatī’’ti.

\paragraph{553.} Evaṃ vutte, vāseṭṭho māṇavo bhagavantaṃ etadavoca – ‘‘sutaṃ metaṃ, bho gotama, samaṇo gotamo brahmānaṃ sahabyatāya maggaṃ jānātī’’ti. ‘‘Taṃ kiṃ maññasi, vāseṭṭha. Āsanne ito manasākaṭaṃ, na ito dūre manasākaṭa’’nti? ‘‘Evaṃ, bho gotama, āsanne ito manasākaṭaṃ, na ito dūre manasākaṭa’’nti.

\paragraph{554.} ‘‘Taṃ kiṃ maññasi, vāseṭṭha, idhassa puriso manasākaṭe jātasaṃvaddho. Tamenaṃ manasākaṭato tāvadeva avasaṭaṃ manasākaṭassa maggaṃ puccheyyuṃ. Siyā nu kho, vāseṭṭha, tassa purisassa manasākaṭe jātasaṃvaddhassa manasākaṭassa maggaṃ puṭṭhassa dandhāyitattaṃ vā vitthāyitattaṃ vā’’ti? ‘‘No hidaṃ, bho gotama’’. ‘‘Taṃ kissa hetu’’? ‘‘Amu hi, bho gotama, puriso manasākaṭe jātasaṃvaddho, tassa sabbāneva manasākaṭassa maggāni suviditānī’’ti. ‘‘Siyā kho, vāseṭṭha, tassa purisassa manasākaṭe jātasaṃvaddhassa manasākaṭassa maggaṃ puṭṭhassa dandhāyitattaṃ vā vitthāyitattaṃ vā, na tveva tathāgatassa brahmaloke vā brahmalokagāminiyā vā paṭipadāya puṭṭhassa dandhāyitattaṃ vā vitthāyitattaṃ vā. Brahmānaṃ cāhaṃ, vāseṭṭha, pajānāmi brahmalokañca brahmalokagāminiñca paṭipadaṃ, yathā paṭipanno ca brahmalokaṃ upapanno, tañca pajānāmī’’ti.

\paragraph{555.} Evaṃ vutte, vāseṭṭho māṇavo bhagavantaṃ etadavoca – ‘‘sutaṃ metaṃ, bho gotama, samaṇo gotamo brahmānaṃ sahabyatāya maggaṃ desetī’’ti. ‘‘Sādhu no bhavaṃ gotamo brahmānaṃ sahabyatāya maggaṃ desetu ullumpatu bhavaṃ gotamo brāhmaṇiṃ paja’’nti. ‘‘Tena hi, vāseṭṭha, suṇāhi; sādhukaṃ manasi karohi; bhāsissāmī’’ti. ‘‘Evaṃ bho’’ti kho vāseṭṭho māṇavo bhagavato paccassosi.

\subsubsection{Brahmalokamaggadesanā}

\paragraph{556.} Bhagavā etadavoca – ‘‘idha, vāseṭṭha, tathāgato loke uppajjati arahaṃ, sammāsambuddho…pe… (yathā 190-212 anucchedesu evaṃ vitthāretabbaṃ). Evaṃ kho, vāseṭṭha, bhikkhu sīlasampanno hoti…pe… tassime pañca nīvaraṇe pahīne attani samanupassato pāmojjaṃ jāyati, pamuditassa pīti jāyati, pītimanassa kāyo passambhati, passaddhakāyo sukhaṃ vedeti, sukhino cittaṃ samādhiyati. ‘‘So mettāsahagatena cetasā ekaṃ disaṃ pharitvā viharati. Tathā dutiyaṃ. Tathā tatiyaṃ. Tathā catutthaṃ. Iti uddhamadho tiriyaṃ sabbadhi sabbattatāya sabbāvantaṃ lokaṃ mettāsahagatena cetasā vipulena mahaggatena appamāṇena averena abyāpajjena pharitvā viharati. ‘‘Seyyathāpi, vāseṭṭha, balavā saṅkhadhamo appakasireneva catuddisā viññāpeyya; evameva kho, vāseṭṭha, evaṃ bhāvitāya mettāya cetovimuttiyā yaṃ pamāṇakataṃ kammaṃ na taṃ tatrāvasissati, na taṃ tatrāvatiṭṭhati. Ayampi kho, vāseṭṭha, brahmānaṃ sahabyatāya maggo. ‘‘Puna caparaṃ, vāseṭṭha, bhikkhu karuṇāsahagatena cetasā…pe… muditāsahagatena cetasā…pe… upekkhāsahagatena cetasā ekaṃ disaṃ pharitvā viharati. Tathā dutiyaṃ. Tathā tatiyaṃ. Tathā catutthaṃ. Iti uddhamadho tiriyaṃ sabbadhi sabbattatāya sabbāvantaṃ lokaṃ upekkhāsahagatena cetasā vipulena mahaggatena appamāṇena averena abyāpajjena pharitvā viharati. ‘‘Seyyathāpi, vāseṭṭha, balavā saṅkhadhamo appakasireneva catuddisā viññāpeyya. Evameva kho, vāseṭṭha, evaṃ bhāvitāya upekkhāya cetovimuttiyā yaṃ pamāṇakataṃ kammaṃ na taṃ tatrāvasissati, na taṃ tatrāvatiṭṭhati. Ayaṃ kho, vāseṭṭha, brahmānaṃ sahabyatāya maggo.

\paragraph{557.} ‘‘Taṃ kiṃ maññasi, vāseṭṭha, evaṃvihārī bhikkhu sapariggaho vā apariggaho vā’’ti? ‘‘Apariggaho, bho gotama’’. ‘‘Saveracitto vā averacitto vā’’ti? ‘‘Averacitto, bho gotama’’. ‘‘Sabyāpajjacitto vā abyāpajjacitto vā’’ti? ‘‘Abyāpajjacitto, bho gotama’’. ‘‘Saṃkiliṭṭhacitto vā asaṃkiliṭṭhacitto vā’’ti? ‘‘Asaṃkiliṭṭhacitto, bho gotama’’. ‘‘Vasavattī vā avasavattī vā’’ti? ‘‘Vasavattī, bho gotama’’. ‘‘Iti kira, vāseṭṭha, apariggaho bhikkhu, apariggaho brahmā. Api nu kho apariggahassa bhikkhuno apariggahena brahmunā saddhiṃ saṃsandati sametī’’ti? ‘‘Evaṃ, bho gotama’’. ‘‘Sādhu, vāseṭṭha, so vata vāseṭṭha apariggaho bhikkhu kāyassa bhedā paraṃ maraṇā apariggahassa brahmuno sahabyūpago bhavissatī’’ti, ṭhānametaṃ vijjati.

\paragraph{558.} ‘‘Iti kira, vāseṭṭha, averacitto bhikkhu, averacitto brahmā…pe… abyāpajjacitto bhikkhu, abyāpajjacitto brahmā… asaṃkiliṭṭhacitto bhikkhu, asaṃkiliṭṭhacitto brahmā… vasavattī bhikkhu, vasavattī brahmā, api nu kho vasavattissa bhikkhuno vasavattinā brahmunā saddhiṃ saṃsandati sametī’’ti? ‘‘Evaṃ, bho gotama’’. ‘‘Sādhu, vāseṭṭha, so vata, vāseṭṭha, vasavattī bhikkhu kāyassa bhedā paraṃ maraṇā vasavattissa brahmuno sahabyūpago bhavissatīti, ṭhānametaṃ vijjatī’’ti.

\paragraph{559.} Evaṃ vutte, vāseṭṭhabhāradvājā māṇavā bhagavantaṃ etadavocuṃ – ‘‘abhikkantaṃ, bho gotama, abhikkantaṃ, bho gotama! Seyyathāpi, bho gotama, nikkujjitaṃ vā ukkujjeyya, paṭicchannaṃ vā vivareyya, mūḷhassa vā maggaṃ ācikkheyya, andhakāre vā telapajjotaṃ dhāreyya ‘cakkhumanto rūpāni dakkhantī’ti. Evamevaṃ bhotā gotamena anekapariyāyena dhammo pakāsito. Ete mayaṃ bhavantaṃ gotamaṃ saraṇaṃ gacchāma, dhammañca bhikkhusaṅghañca. Upāsake no bhavaṃ gotamo dhāretu ajjatagge pāṇupete saraṇaṃ gate’’ti.

\xsectionEnd{Tevijjasuttaṃ niṭṭhitaṃ terasamaṃ. \\ Sīlakkhandhavaggo niṭṭhito.}

\paragraph{}
Tassuddānaṃ –
\begin{verse}
  Brahmāsāmaññaambaṭṭha,\\
  Soṇakūṭamahālijālinī;\\
  Sīhapoṭṭhapādasubho kevaṭṭo,\\
  Lohiccatevijjā terasāti.\\
\end{verse}

\xsectionEnd{Sīlakkhandhavaggapāḷi niṭṭhitā.}




\xchapter{2}{Mahāvaggapāḷi}

\section{Mahāpadānasuttaṃ}

\subsubsection{Pubbenivāsapaṭisaṃyuttakathā}

\paragraph{1.} Evaṃ me sutaṃ – ekaṃ samayaṃ bhagavā sāvatthiyaṃ viharati jetavane anāthapiṇḍikassa ārāme karerikuṭikāyaṃ. Atha kho sambahulānaṃ bhikkhūnaṃ pacchābhattaṃ piṇḍapātapaṭikkantānaṃ karerimaṇḍalamāḷe sannisinnānaṃ sannipatitānaṃ pubbenivāsapaṭisaṃyuttā dhammī kathā udapādi – ‘‘itipi pubbenivāso, itipi pubbenivāso’’ti.

\paragraph{2.} Assosi kho bhagavā dibbāya sotadhātuyā visuddhāya atikkantamānusikāya tesaṃ bhikkhūnaṃ imaṃ kathāsallāpaṃ. Atha kho bhagavā uṭṭhāyāsanā yena karerimaṇḍalamāḷo tenupasaṅkami; upasaṅkamitvā paññatte āsane nisīdi, nisajja kho bhagavā bhikkhū āmantesi – ‘‘kāyanuttha, bhikkhave, etarahi kathāya sannisinnā; kā ca pana vo antarākathā vippakatā’’ti?

\paragraph{3.} Evaṃ vutte te bhikkhū bhagavantaṃ etadavocuṃ – ‘‘idha, bhante, amhākaṃ pacchābhattaṃ piṇḍapātapaṭikkantānaṃ karerimaṇḍalamāḷe sannisinnānaṃ sannipatitānaṃ pubbenivāsapaṭisaṃyuttā dhammī kathā udapādi – ‘itipi pubbenivāso itipi pubbenivāso’ti. Ayaṃ kho no, bhante, antarākathā vippakatā. Atha bhagavā anuppatto’’ti.

\paragraph{4.} ‘‘Iccheyyātha no tumhe, bhikkhave, pubbenivāsapaṭisaṃyuttaṃ dhammiṃ kathaṃ sotu’’nti? ‘‘Etassa, bhagavā, kālo; etassa, sugata, kālo; yaṃ bhagavā pubbenivāsapaṭisaṃyuttaṃ dhammiṃ kathaṃ kareyya, bhagavato sutvā\footnote{bhagavato vacanaṃ sutvā (syā.)} bhikkhū dhāressantī’’ti. ‘‘Tena hi, bhikkhave, suṇātha,sādhukaṃ manasi karotha, bhāsissāmī’’ti. ‘‘Evaṃ, bhante’’ti kho te bhikkhū bhagavato paccassosuṃ. Bhagavā etadavoca –

\paragraph{5.} ‘‘Ito so, bhikkhave, ekanavutikappe yaṃ\footnote{ekanavuto kappo (syā. kaṃ. pī.)} vipassī bhagavā arahaṃ sammāsambuddho loke udapādi. Ito so, bhikkhave, ekatiṃse kappe\footnote{ekatiṃ sakappo (sī.) ekatiṃ so kappo (syā. kaṃ. pī.)} yaṃ sikhī bhagavā arahaṃ sammāsambuddho loke udapādi. Tasmiññeva kho, bhikkhave, ekatiṃse kappe vessabhū bhagavā arahaṃ sammāsambuddho loke udapādi. Imasmiññeva\footnote{imasmiṃ (katthacī)} kho, bhikkhave, bhaddakappe kakusandho bhagavā arahaṃ sammāsambuddho loke udapādi. Imasmiññeva kho, bhikkhave, bhaddakappe koṇāgamano bhagavā arahaṃ sammāsambuddho loke udapādi. Imasmiññeva kho, bhikkhave, bhaddakappe kassapo bhagavā arahaṃ sammāsambuddho loke udapādi. Imasmiññeva kho, bhikkhave, bhaddakappe ahaṃ etarahi arahaṃ sammāsambuddho loke uppanno.

\paragraph{6.} ‘‘Vipassī, bhikkhave, bhagavā arahaṃ sammāsambuddho khattiyo jātiyā ahosi, khattiyakule udapādi. Sikhī, bhikkhave, bhagavā arahaṃ sammāsambuddho khattiyo jātiyā ahosi, khattiyakule udapādi. Vessabhū, bhikkhave, bhagavā arahaṃ sammāsambuddho khattiyo jātiyā ahosi, khattiyakule udapādi. Kakusandho, bhikkhave, bhagavā arahaṃ sammāsambuddho brāhmaṇo jātiyā ahosi, brāhmaṇakule udapādi. Koṇāgamano, bhikkhave, bhagavā arahaṃ sammāsambuddho brāhmaṇo jātiyā ahosi, brāhmaṇakule udapādi. Kassapo, bhikkhave, bhagavā arahaṃ sammāsambuddho brāhmaṇo jātiyā ahosi, brāhmaṇakule udapādi. Ahaṃ, bhikkhave, etarahi arahaṃ sammāsambuddho khattiyo jātiyā ahosiṃ, khattiyakule uppanno.

\paragraph{7.} ‘‘Vipassī , bhikkhave, bhagavā arahaṃ sammāsambuddho koṇḍañño gottena ahosi. Sikhī, bhikkhave, bhagavā arahaṃ sammāsambuddho koṇḍañño gottena ahosi. Vessabhū, bhikkhave, bhagavā arahaṃ sammāsambuddho koṇḍañño gottena ahosi. Kakusandho, bhikkhave, bhagavā arahaṃ sammāsambuddho kassapo gottena ahosi. Koṇāgamano, bhikkhave, bhagavā arahaṃ sammāsambuddho kassapo gottena ahosi. Kassapo, bhikkhave, bhagavā arahaṃ sammāsambuddho kassapo gottena ahosi. Ahaṃ, bhikkhave, etarahi arahaṃ sammāsambuddho gotamo gottena ahosiṃ.

\paragraph{8.} ‘‘Vipassissa, bhikkhave, bhagavato arahato sammāsambuddhassa asītivassasahassāni āyuppamāṇaṃ ahosi. Sikhissa, bhikkhave, bhagavato arahato sammāsambuddhassa sattativassasahassāni āyuppamāṇaṃ ahosi. Vessabhussa, bhikkhave, bhagavato arahato sammāsambuddhassa saṭṭhivassasahassāni āyuppamāṇaṃ ahosi. Kakusandhassa, bhikkhave, bhagavato arahato sammāsambuddhassa cattālīsavassasahassāni āyuppamāṇaṃ ahosi. Koṇāgamanassa, bhikkhave, bhagavato arahato sammāsambuddhassa tiṃsavassasahassāni āyuppamāṇaṃ ahosi. Kassapassa, bhikkhave, bhagavato arahato sammāsambuddhassa vīsativassasahassāni āyuppamāṇaṃ ahosi. Mayhaṃ, bhikkhave, etarahi appakaṃ āyuppamāṇaṃ parittaṃ lahukaṃ; yo ciraṃ jīvati, so vassasataṃ appaṃ vā bhiyyo.

\paragraph{9.} ‘‘Vipassī, bhikkhave, bhagavā arahaṃ sammāsambuddho pāṭaliyā mūle abhisambuddho. Sikhī, bhikkhave, bhagavā arahaṃ sammāsambuddho puṇḍarīkassa mūle abhisambuddho. Vessabhū, bhikkhave, bhagavā arahaṃ sammāsambuddho sālassa mūle abhisambuddho. Kakusandho, bhikkhave, bhagavā arahaṃ sammāsambuddho sirīsassa mūle abhisambuddho. Koṇāgamano, bhikkhave, bhagavā arahaṃ sammāsambuddho udumbarassa mūle abhisambuddho. Kassapo, bhikkhave, bhagavā arahaṃ sammāsambuddho nigrodhassa mūle abhisambuddho. Ahaṃ, bhikkhave, etarahi arahaṃ sammāsambuddho assatthassa mūle abhisambuddho.

\paragraph{10.} ‘‘Vipassissa , bhikkhave, bhagavato arahato sammāsambuddhassa khaṇḍatissaṃ nāma sāvakayugaṃ ahosi aggaṃ bhaddayugaṃ. Sikhissa, bhikkhave, bhagavato arahato sammāsambuddhassa abhibhūsambhavaṃ nāma sāvakayugaṃ ahosi aggaṃ bhaddayugaṃ. Vessabhussa, bhikkhave, bhagavato arahato sammāsambuddhassa soṇuttaraṃ nāma sāvakayugaṃ ahosi aggaṃ bhaddayugaṃ. Kakusandhassa, bhikkhave, bhagavato arahato sammāsambuddhassa vidhurasañjīvaṃ nāma sāvakayugaṃ ahosi aggaṃ bhaddayugaṃ. Koṇāgamanassa, bhikkhave, bhagavato arahato sammāsambuddhassa bhiyyosuttaraṃ nāma sāvakayugaṃ ahosi aggaṃ bhaddayugaṃ. Kassapassa, bhikkhave, bhagavato arahato sammāsambuddhassa tissabhāradvājaṃ nāma sāvakayugaṃ ahosi aggaṃ bhaddayugaṃ. Mayhaṃ, bhikkhave, etarahi sāriputtamoggallānaṃ nāma sāvakayugaṃ ahosi aggaṃ bhaddayugaṃ.

\paragraph{11.} ‘‘Vipassissa, bhikkhave, bhagavato arahato sammāsambuddhassa tayo sāvakānaṃ sannipātā ahesuṃ. Eko sāvakānaṃ sannipāto ahosi aṭṭhasaṭṭhibhikkhusatasahassaṃ, eko sāvakānaṃ sannipāto ahosi bhikkhusatasahassaṃ, eko sāvakānaṃ sannipāto ahosi asītibhikkhusahassāni. Vipassissa, bhikkhave, bhagavato arahato sammāsambuddhassa ime tayo sāvakānaṃ sannipātā ahesuṃ sabbesaṃyeva khīṇāsavānaṃ.

\paragraph{12.} ‘‘Sikhissa, bhikkhave, bhagavato arahato sammāsambuddhassa tayo sāvakānaṃ sannipātā ahesuṃ. Eko sāvakānaṃ sannipāto ahosi bhikkhusatasahassaṃ, eko sāvakānaṃ sannipāto ahosi asītibhikkhusahassāni, eko sāvakānaṃ sannipāto ahosi sattatibhikkhusahassāni. Sikhissa, bhikkhave, bhagavato arahato sammāsambuddhassa ime tayo sāvakānaṃ sannipātā ahesuṃ sabbesaṃyeva khīṇāsavānaṃ.

\paragraph{13.} ‘‘Vessabhussa, bhikkhave, bhagavato arahato sammāsambuddhassa tayo sāvakānaṃ sannipātā ahesuṃ. Eko sāvakānaṃ sannipāto ahosi asītibhikkhusahassāni, eko sāvakānaṃ sannipāto ahosi sattatibhikkhusahassāni, eko sāvakānaṃ sannipāto ahosi saṭṭhibhikkhusahassāni. Vessabhussa, bhikkhave, bhagavato arahato sammāsambuddhassa ime tayo sāvakānaṃ sannipātā ahesuṃ sabbesaṃyeva khīṇāsavānaṃ.

\paragraph{14.} ‘‘Kakusandhassa, bhikkhave, bhagavato arahato sammāsambuddhassa eko sāvakānaṃ sannipāto ahosi cattālīsabhikkhusahassāni. Kakusandhassa, bhikkhave, bhagavato arahato sammāsambuddhassa ayaṃ eko sāvakānaṃ sannipāto ahosi sabbesaṃyeva khīṇāsavānaṃ.

\paragraph{15.} ‘‘Koṇāgamanassa, bhikkhave, bhagavato arahato sammāsambuddhassa eko sāvakānaṃ sannipāto ahosi tiṃsabhikkhusahassāni. Koṇāgamanassa, bhikkhave, bhagavato arahato sammāsambuddhassa ayaṃ eko sāvakānaṃ sannipāto ahosi sabbesaṃyeva khīṇāsavānaṃ.

\paragraph{16.} ‘‘Kassapassa, bhikkhave, bhagavato arahato sammāsambuddhassa eko sāvakānaṃ sannipāto ahosi vīsatibhikkhusahassāni. Kassapassa, bhikkhave, bhagavato arahato sammāsambuddhassa ayaṃ eko sāvakānaṃ sannipāto ahosi sabbesaṃyeva khīṇāsavānaṃ.

\paragraph{17.} ‘‘Mayhaṃ, bhikkhave, etarahi eko sāvakānaṃ sannipāto ahosi aḍḍhateḷasāni bhikkhusatāni. Mayhaṃ, bhikkhave, ayaṃ eko sāvakānaṃ sannipāto ahosi sabbesaṃyeva khīṇāsavānaṃ.

\paragraph{18.} ‘‘Vipassissa, bhikkhave, bhagavato arahato sammāsambuddhassa asoko nāma bhikkhu upaṭṭhāko ahosi aggupaṭṭhāko. Sikhissa, bhikkhave, bhagavato arahato sammāsambuddhassa khemaṅkaro nāma bhikkhu upaṭṭhāko ahosi aggupaṭṭhāko. Vessabhussa, bhikkhave, bhagavato arahato sammāsambuddhassa upasanto nāma bhikkhu upaṭṭhāko ahosi aggupaṭṭhāko. Kakusandhassa, bhikkhave, bhagavato arahato sammāsambuddhassa buddhijo nāma bhikkhu upaṭṭhāko ahosi aggupaṭṭhāko. Koṇāgamanassa, bhikkhave, bhagavato arahato sammāsambuddhassa sotthijo nāma bhikkhu upaṭṭhāko ahosi aggupaṭṭhāko. Kassapassa, bhikkhave, bhagavato arahato sammāsambuddhassa sabbamitto nāma bhikkhu upaṭṭhāko ahosi aggupaṭṭhāko. Mayhaṃ, bhikkhave, etarahi ānando nāma bhikkhu upaṭṭhāko ahosi aggupaṭṭhāko.

\paragraph{19.} ‘‘Vipassissa, bhikkhave, bhagavato arahato sammāsambuddhassa bandhumā nāma rājā pitā ahosi. Bandhumatī nāma devī mātā ahosi janetti\footnote{janettī (syā.)}. Bandhumassa rañño bandhumatī nāma nagaraṃ rājadhānī ahosi.

\paragraph{20.} ‘‘Sikhissa , bhikkhave, bhagavato arahato sammāsambuddhassa aruṇo nāma rājā pitā ahosi. Pabhāvatī nāma devī mātā ahosi janetti. Aruṇassa rañño aruṇavatī nāma nagaraṃ rājadhānī ahosi.

\paragraph{21.} ‘‘Vessabhussa, bhikkhave, bhagavato arahato sammāsambuddhassa suppatito nāma\footnote{suppatīto nāma (syā.)} rājā pitā ahosi. Vassavatī nāma\footnote{yasavatī nāma (syā. pī.)} devī mātā ahosi janetti. Suppatitassa rañño anomaṃ nāma nagaraṃ rājadhānī ahosi.

\paragraph{22.} ‘‘Kakusandhassa, bhikkhave, bhagavato arahato sammāsambuddhassa aggidatto nāma brāhmaṇo pitā ahosi. Visākhā nāma brāhmaṇī mātā ahosi janetti. Tena kho pana, bhikkhave, samayena khemo nāma rājā ahosi. Khemassa rañño khemavatī nāma nagaraṃ rājadhānī ahosi.

\paragraph{23.} ‘‘Koṇāgamanassa, bhikkhave, bhagavato arahato sammāsambuddhassa yaññadatto nāma brāhmaṇo pitā ahosi. Uttarā nāma brāhmaṇī mātā ahosi janetti. Tena kho pana, bhikkhave, samayena sobho nāma rājā ahosi. Sobhassa rañño sobhavatī nāma nagaraṃ rājadhānī ahosi.

\paragraph{24.} ‘‘Kassapassa, bhikkhave, bhagavato arahato sammāsambuddhassa brahmadatto nāma brāhmaṇo pitā ahosi. Dhanavatī nāma brāhmaṇī mātā ahosi janetti. Tena kho pana, bhikkhave, samayena kikī nāma\footnote{kiṃ kī nāma (syā.)} rājā ahosi. Kikissa rañño bārāṇasī nāma nagaraṃ rājadhānī ahosi.

\paragraph{25.} ‘‘Mayhaṃ, bhikkhave, etarahi suddhodano nāma rājā pitā ahosi. Māyā nāma devī mātā ahosi janetti. Kapilavatthu nāma nagaraṃ rājadhānī ahosī’’ti. Idamavoca bhagavā, idaṃ vatvāna sugato uṭṭhāyāsanā vihāraṃ pāvisi.

\paragraph{26.} Atha kho tesaṃ bhikkhūnaṃ acirapakkantassa bhagavato ayamantarākathā udapādi – ‘‘acchariyaṃ, āvuso, abbhutaṃ, āvuso, tathāgatassa mahiddhikatā mahānubhāvatā. Yatra hi nāma tathāgato atīte buddhe parinibbute chinnapapañce chinnavaṭume pariyādinnavaṭṭe sabbadukkhavītivatte jātitopi anussarissati, nāmatopi anussarissati, gottatopi anussarissati, āyuppamāṇatopi anussarissati, sāvakayugatopi anussarissati, sāvakasannipātatopi anussarissati – ‘evaṃjaccā te bhagavanto ahesuṃ itipi, evaṃnāmā evaṃgottā evaṃsīlā evaṃdhammā evaṃpaññā evaṃvihārī evaṃvimuttā te bhagavanto ahesuṃ itipī’’’ti.

\paragraph{27.} ‘‘Kiṃ nu kho, āvuso, tathāgatasseva nu kho esā dhammadhātu suppaṭividdhā, yassā dhammadhātuyā suppaṭividdhattā tathāgato atīte buddhe parinibbute chinnapapañce chinnavaṭume pariyādinnavaṭṭe sabbadukkhavītivatte jātitopi anussarati, nāmatopi anussarati, gottatopi anussarati, āyuppamāṇatopi anussarati, sāvakayugatopi anussarati, sāvakasannipātatopi anussarati – ‘evaṃjaccā te bhagavanto ahesuṃ itipi, evaṃnāmā evaṃgottā evaṃsīlā evaṃdhammā evaṃpaññā evaṃvihārī evaṃvimuttā te bhagavanto ahesuṃ itipī’ti, udāhu devatā tathāgatassa etamatthaṃ ārocesuṃ, yena tathāgato atīte buddhe parinibbute chinnapapañce chinnavaṭume pariyādinnavaṭṭe sabbadukkhavītivatte jātitopi anussarati, nāmatopi anussarati, gottatopi anussarati, āyuppamāṇatopi anussarati, sāvakayugatopi anussarati, sāvakasannipātatopi anussarati – ‘evaṃjaccā te bhagavanto ahesuṃ itipi, evaṃnāmā evaṃgottā evaṃsīlā evaṃdhammā evaṃpaññā evaṃvihārī evaṃvimuttā te bhagavanto ahesuṃ itipī’’’ti. Ayañca hidaṃ tesaṃ bhikkhūnaṃ antarākathā vippakatā hoti.

\paragraph{28.} Atha kho bhagavā sāyanhasamayaṃ paṭisallānā vuṭṭhito yena karerimaṇḍalamāḷo tenupasaṅkami; upasaṅkamitvā paññatte āsane nisīdi. Nisajja kho bhagavā bhikkhū āmantesi – ‘‘kāyanuttha, bhikkhave, etarahi kathāya sannisinnā; kā ca pana vo antarākathā vippakatā’’ti?

\paragraph{29.} Evaṃ vutte te bhikkhū bhagavantaṃ etadavocuṃ – ‘‘idha, bhante, amhākaṃ acirapakkantassa bhagavato ayaṃ antarākathā udapādi – ‘acchariyaṃ, āvuso, abbhutaṃ, āvuso, tathāgatassa mahiddhikatā mahānubhāvatā, yatra hi nāma tathāgato atīte buddhe parinibbute chinnapapañce chinnavaṭume pariyādinnavaṭṭe sabbadukkhavītivatte jātitopi anussarissati, nāmatopi anussarissati, gottatopi anussarissati, āyuppamāṇatopi anussarissati, sāvakayugatopi anussarissati, sāvakasannipātatopi anussarissati – ‘‘evaṃjaccā te bhagavanto ahesuṃ itipi , evaṃnāmā evaṃgottā evaṃsīlā evaṃdhammā evaṃpaññā evaṃvihārī evaṃvimuttā te bhagavanto ahesuṃ itipī’’ti. Kiṃ nu kho, āvuso, tathāgatasseva nu kho esā dhammadhātu suppaṭividdhā, yassā dhammadhātuyā suppaṭividdhattā tathāgato atīte buddhe parinibbute chinnapapañce chinnavaṭume pariyādinnavaṭṭe sabbadukkhavītivatte jātitopi anussarati, nāmatopi anussarati, gottatopi anussarati, āyuppamāṇatopi anussarati, sāvakayugatopi anussarati, sāvakasannipātatopi anussarati – ‘‘evaṃjaccā te bhagavanto ahesuṃ itipi, evaṃnāmā evaṃgottā evaṃsīlā evaṃdhammā evaṃpaññā evaṃvihārī evaṃvimuttā te bhagavanto ahesuṃ itipī’’ti. Udāhu devatā tathāgatassa etamatthaṃ ārocesuṃ, yena tathāgato atīte buddhe parinibbute chinnapapañce chinnavaṭume pariyādinnavaṭṭe sabbadukkhavītivatte jātitopi anussarati, nāmatopi anussarati, gottatopi anussarati, āyuppamāṇatopi anussarati, sāvakayugatopi anussarati, sāvakasannipātatopi anussarati – ‘evaṃjaccā te bhagavanto ahesuṃ itipi, evaṃnāmā evaṃgottā evaṃsīlā evaṃdhammā evaṃpaññā evaṃvihārī evaṃvimuttā te bhagavanto ahesuṃ itipī’ti? Ayaṃ kho no, bhante, antarākathā vippakatā, atha bhagavā anuppatto’’ti.

\paragraph{30.} ‘‘Tathāgatassevesā, bhikkhave, dhammadhātu suppaṭividdhā, yassā dhammadhātuyā suppaṭividdhattā tathāgato atīte buddhe parinibbute chinnapapañce chinnavaṭume pariyādinnavaṭṭe sabbadukkhavītivatte jātitopi anussarati, nāmatopi anussarati, gottatopi anussarati, āyuppamāṇatopi anussarati, sāvakayugatopi anussarati, sāvakasannipātatopi anussarati – ‘evaṃjaccā te bhagavanto ahesuṃ itipi, evaṃnāmā evaṃgottā evaṃsīlā evaṃdhammā evaṃpaññā evaṃvihārī evaṃvimuttā te bhagavanto ahesuṃ itipī’ti. Devatāpi tathāgatassa etamatthaṃ ārocesuṃ, yena tathāgato atīte buddhe parinibbute chinnapapañce chinnavaṭume pariyādinnavaṭṭe sabbadukkhavītivatte jātitopi anussarati, nāmatopi anussarati, gottatopi anussarati, āyuppamāṇatopi anussarati, sāvakayugatopi anussarati, sāvakasannipātatopi anussarati – ‘evaṃjaccā te bhagavanto ahesuṃ itipi, evaṃnāmā evaṃgottā evaṃsīlā evaṃdhammā evaṃpaññā evaṃvihārī evaṃvimuttā te bhagavanto ahesuṃ itipī’ti.

\paragraph{31.}‘‘Iccheyyātha no tumhe, bhikkhave, bhiyyosomattāya pubbenivāsapaṭisaṃyuttaṃ dhammiṃ kathaṃ sotu’’nti? ‘‘Etassa, bhagavā, kālo; etassa, sugata, kālo; yaṃ bhagavā bhiyyosomattāya pubbenivāsapaṭisaṃyuttaṃ dhammiṃ kathaṃ kareyya, bhagavato sutvā bhikkhū dhāressantī’’ti. ‘‘Tena hi, bhikkhave , suṇātha, sādhukaṃ manasi karotha, bhāsissāmī’’ti. ‘‘Evaṃ, bhante’’ti kho te bhikkhū bhagavato paccassosuṃ. Bhagavā etadavoca –

\paragraph{32.} ‘‘Ito so, bhikkhave, ekanavutikappe yaṃ vipassī bhagavā arahaṃ sammāsambuddho loke udapādi. Vipassī, bhikkhave, bhagavā arahaṃ sammāsambuddho khattiyo jātiyā ahosi, khattiyakule udapādi. Vipassī, bhikkhave, bhagavā arahaṃ sammāsambuddho koṇḍañño gottena ahosi. Vipassissa, bhikkhave, bhagavato arahato sammāsambuddhassa asītivassasahassāni āyuppamāṇaṃ ahosi. Vipassī, bhikkhave, bhagavā arahaṃ sammāsambuddho pāṭaliyā mūle abhisambuddho. Vipassissa, bhikkhave , bhagavato arahato sammāsambuddhassa khaṇḍatissaṃ nāma sāvakayugaṃ ahosi aggaṃ bhaddayugaṃ. Vipassissa, bhikkhave, bhagavato arahato sammāsambuddhassa tayo sāvakānaṃ sannipātā ahesuṃ. Eko sāvakānaṃ sannipāto ahosi aṭṭhasaṭṭhibhikkhusatasahassaṃ, eko sāvakānaṃ sannipāto ahosi bhikkhusatasahassaṃ, eko sāvakānaṃ sannipāto ahosi asītibhikkhusahassāni. Vipassissa, bhikkhave, bhagavato arahato sammāsambuddhassa ime tayo sāvakānaṃ sannipātā ahesuṃ sabbesaṃyeva khīṇāsavānaṃ. Vipassissa, bhikkhave, bhagavato arahato sammāsambuddhassa asoko nāma bhikkhu upaṭṭhāko ahosi aggupaṭṭhāko. Vipassissa, bhikkhave, bhagavato arahato sammāsambuddhassa bandhumā nāma rājā pitā ahosi. Bandhumatī nāma devī mātā ahosi janetti. Bandhumassa rañño bandhumatī nāma nagaraṃ rājadhānī ahosi.

\subsubsection{Bodhisattadhammatā}

\paragraph{33.} ‘‘Atha kho, bhikkhave, vipassī bodhisatto tusitā kāyā cavitvā sato sampajāno mātukucchiṃ okkami. Ayamettha dhammatā.

\paragraph{34.} ‘‘Dhammatā, esā, bhikkhave, yadā bodhisatto tusitā kāyā cavitvā mātukucchiṃ okkamati. Atha sadevake loke samārake sabrahmake sassamaṇabrāhmaṇiyā pajāya sadevamanussāya appamāṇo uḷāro obhāso pātubhavati atikkammeva devānaṃ devānubhāvaṃ. Yāpi tā lokantarikā aghā asaṃvutā andhakārā andhakāratimisā , yattha pime candimasūriyā evaṃmahiddhikā evaṃmahānubhāvā ābhāya nānubhonti, tatthapi appamāṇo uḷāro obhāso pātubhavati atikkammeva devānaṃ devānubhāvaṃ. Yepi tattha sattā upapannā, tepi tenobhāsena aññamaññaṃ sañjānanti – ‘aññepi kira, bho, santi sattā idhūpapannā’ti. Ayañca dasasahassī lokadhātu saṅkampati sampakampati sampavedhati. Appamāṇo ca uḷāro obhāso loke pātubhavati atikkammeva devānaṃ devānubhāvaṃ. Ayamettha dhammatā.

\paragraph{35.} ‘‘Dhammatā esā, bhikkhave, yadā bodhisatto mātukucchiṃ okkanto hoti, cattāro naṃ devaputtā catuddisaṃ\footnote{cātuddisaṃ (syā.)} rakkhāya upagacchanti – ‘mā naṃ bodhisattaṃ vā bodhisattamātaraṃ vā manusso vā amanusso vā koci vā viheṭhesī’ti. Ayamettha dhammatā.

\paragraph{36.} ‘‘Dhammatā esā, bhikkhave, yadā bodhisatto mātukucchiṃ okkanto hoti, pakatiyā sīlavatī bodhisattamātā hoti, viratā pāṇātipātā, viratā adinnādānā, viratā kāmesumicchācārā , viratā musāvādā, viratā surāmerayamajjappamādaṭṭhānā. Ayamettha dhammatā.

\paragraph{37.} ‘‘Dhammatā esā, bhikkhave, yadā bodhisatto mātukucchiṃ okkanto hoti, na bodhisattamātu purisesu mānasaṃ uppajjati kāmaguṇūpasaṃhitaṃ, anatikkamanīyā ca bodhisattamātā hoti kenaci purisena rattacittena. Ayamettha dhammatā.

\paragraph{38.} ‘‘Dhammatā esā, bhikkhave, yadā bodhisatto mātukucchiṃ okkanto hoti, lābhinī bodhisattamātā hoti pañcannaṃ kāmaguṇānaṃ. Sā pañcahi kāmaguṇehi samappitā samaṅgībhūtā paricāreti. Ayamettha dhammatā.

\paragraph{39.} ‘‘Dhammatā esā, bhikkhave, yadā bodhisatto mātukucchiṃ okkanto hoti, na bodhisattamātu kocideva ābādho uppajjati. Sukhinī bodhisattamātā hoti akilantakāyā, bodhisattañca bodhisattamātā tirokucchigataṃ passati sabbaṅgapaccaṅgiṃ ahīnindriyaṃ. Seyyathāpi, bhikkhave, maṇi veḷuriyo subho jātimā aṭṭhaṃso suparikammakato accho vippasanno anāvilo sabbākārasampanno. Tatrāssa\footnote{tatrassa (syā.)} suttaṃ āvutaṃ nīlaṃ vā pītaṃ vā lohitaṃ vā odātaṃ vā paṇḍusuttaṃ vā. Tamenaṃ cakkhumā puriso hatthe karitvā paccavekkheyya – ‘ayaṃ kho maṇi veḷuriyo subho jātimā aṭṭhaṃso suparikammakato accho vippasanno anāvilo sabbākārasampanno. Tatridaṃ suttaṃ āvutaṃ nīlaṃ vā pītaṃ vā lohitaṃ vā odātaṃ vā paṇḍusuttaṃ vā’ti. Evameva kho, bhikkhave, yadā bodhisatto mātukucchiṃ okkanto hoti, na bodhisattamātu kocideva ābādho uppajjati, sukhinī bodhisattamātā hoti akilantakāyā , bodhisattañca bodhisattamātā tirokucchigataṃ passati sabbaṅgapaccaṅgiṃ ahīnindriyaṃ. Ayamettha dhammatā.

\paragraph{40.} ‘‘Dhammatā esā, bhikkhave, sattāhajāte bodhisatte bodhisattamātā kālaṅkaroti tusitaṃ kāyaṃ upapajjati. Ayamettha dhammatā.

\paragraph{41.} ‘‘Dhammatā esā, bhikkhave, yathā aññā itthikā nava vā dasa vā māse gabbhaṃ kucchinā pariharitvā vijāyanti, na hevaṃ bodhisattaṃ bodhisattamātā vijāyati. Daseva māsāni bodhisattaṃ bodhisattamātā kucchinā pariharitvā vijāyati. Ayamettha dhammatā.

\paragraph{42.} ‘‘Dhammatā esā, bhikkhave, yathā aññā itthikā nisinnā vā nipannā vā vijāyanti, na hevaṃ bodhisattaṃ bodhisattamātā vijāyati. Ṭhitāva bodhisattaṃ bodhisattamātā vijāyati. Ayamettha dhammatā.

\paragraph{43.} ‘‘Dhammatā esā, bhikkhave, yadā bodhisatto mātukucchimhā nikkhamati, devā paṭhamaṃ paṭiggaṇhanti, pacchā manussā. Ayamettha dhammatā.

\paragraph{44.} ‘‘Dhammatā esā, bhikkhave, yadā bodhisatto mātukucchimhā nikkhamati, appattova bodhisatto pathaviṃ hoti, cattāro naṃ devaputtā paṭiggahetvā mātu purato ṭhapenti – ‘attamanā, devi, hohi; mahesakkho te putto uppanno’ti. Ayamettha dhammatā.

\paragraph{45.} ‘‘Dhammatā esā, bhikkhave, yadā bodhisatto mātukucchimhā nikkhamati, visadova nikkhamati amakkhito udena\footnote{uddena (syā.), udarena (katthaci)} amakkhito semhena amakkhito ruhirena amakkhito kenaci asucinā suddho\footnote{visuddho (syā.)} visado. Seyyathāpi, bhikkhave, maṇiratanaṃ kāsike vatthe nikkhittaṃ neva maṇiratanaṃ kāsikaṃ vatthaṃ makkheti, nāpi kāsikaṃ vatthaṃ maṇiratanaṃ makkheti. Taṃ kissa hetu? Ubhinnaṃ suddhattā. Evameva kho, bhikkhave, yadā bodhisatto mātukucchimhā nikkhamati, visadova nikkhamati amakkhito, udena amakkhito semhena amakkhito ruhirena amakkhito kenaci asucinā suddho visado. Ayamettha dhammatā.

\paragraph{46.} ‘‘Dhammatā esā, bhikkhave, yadā bodhisatto mātukucchimhā nikkhamati, dve udakassa dhārā antalikkhā pātubhavanti – ekā sītassa ekā uṇhassa yena bodhisattassa udakakiccaṃ karonti mātu ca. Ayamettha dhammatā.

\paragraph{47.} ‘‘Dhammatā esā, bhikkhave, sampatijāto bodhisatto samehi pādehi patiṭṭhahitvā uttarābhimukho\footnote{uttarenābhimukho (syā.) uttarenamukho (ka.)} sattapadavītihārena gacchati setamhi chatte anudhāriyamāne, sabbā ca disā anuviloketi, āsabhiṃ vācaṃ bhāsati ‘aggohamasmi lokassa, jeṭṭhohamasmi lokassa, seṭṭhohamasmi lokassa, ayamantimā jāti, natthidāni punabbhavo’ti. Ayamettha dhammatā.

\paragraph{48.} ‘‘Dhammatā esā, bhikkhave, yadā bodhisatto mātukucchimhā nikkhamati, atha sadevake loke samārake sabrahmake sassamaṇabrāhmaṇiyā pajāya sadevamanussāya appamāṇo uḷāro obhāso pātubhavati, atikkammeva devānaṃ devānubhāvaṃ. Yāpi tā lokantarikā aghā asaṃvutā andhakārā andhakāratimisā, yattha pime candimasūriyā evaṃmahiddhikā evaṃmahānubhāvā ābhāya nānubhonti, tatthapi appamāṇo uḷāro obhāso pātubhavati atikkammeva devānaṃ devānubhāvaṃ. Yepi tattha sattā upapannā, tepi tenobhāsena aññamaññaṃ sañjānanti – ‘aññepi kira, bho, santi sattā idhūpapannā’ti. Ayañca dasasahassī lokadhātu saṅkampati sampakampati sampavedhati appamāṇo ca uḷāro obhāso loke pātubhavati atikkammeva devānaṃ devānubhāvaṃ. Ayamettha dhammatā.

\subsubsection{Dvattiṃsamahāpurisalakkhaṇā}

\paragraph{49.} ‘‘Jāte kho pana, bhikkhave, vipassimhi kumāre bandhumato rañño paṭivedesuṃ – ‘putto te, deva\footnote{deva te (ka.)}, jāto, taṃ devo passatū’ti. Addasā kho, bhikkhave, bandhumā rājā vipassiṃ kumāraṃ, disvā nemitte brāhmaṇe āmantāpetvā etadavoca – ‘passantu bhonto nemittā brāhmaṇā kumāra’nti. Addasaṃsu kho, bhikkhave, nemittā brāhmaṇā vipassiṃ kumāraṃ, disvā bandhumantaṃ rājānaṃ etadavocuṃ – ‘attamano, deva, hohi, mahesakkho te putto uppanno, lābhā te, mahārāja, suladdhaṃ te, mahārāja, yassa te kule evarūpo putto uppanno. Ayañhi, deva, kumāro dvattiṃsamahāpurisalakkhaṇehi samannāgato, yehi samannāgatassa mahāpurisassa dveva gatiyo bhavanti anaññā. Sace agāraṃ ajjhāvasati, rājā hoti cakkavattī dhammiko dhammarājā cāturanto vijitāvī janapadatthāvariyappatto sattaratanasamannāgato. Tassimāni sattaratanāni bhavanti. Seyyathidaṃ – cakkaratanaṃ hatthiratanaṃ assaratanaṃ maṇiratanaṃ itthiratanaṃ gahapatiratanaṃ pariṇāyakaratanameva sattamaṃ. Parosahassaṃ kho panassa puttā bhavanti sūrā vīraṅgarūpā parasenappamaddanā. So imaṃ pathaviṃ sāgarapariyantaṃ adaṇḍena asatthena dhammena abhivijiya ajjhāvasati. Sace kho pana agārasmā anagāriyaṃ pabbajati, arahaṃ hoti sammāsambuddho loke vivaṭacchado.

\paragraph{50.} ‘Katamehi cāyaṃ, deva, kumāro dvattiṃsamahāpurisalakkhaṇehi samannāgato, yehi samannāgatassa mahāpurisassa dveva gatiyo bhavanti anaññā. Sace agāraṃ ajjhāvasati, rājā hoti cakkavattī dhammiko dhammarājā cāturanto vijitāpī janapadatthāvariyappatto sattaratanasamannāgato. Tassimāni sattaratanāni bhavanti . Seyyathidaṃ – cakkaratanaṃ hatthiratanaṃ assaratanaṃ maṇiratanaṃ itthiratanaṃ gahapatiratanaṃ pariṇāyakaratanameva sattamaṃ. Parosahassaṃ kho panassa puttā bhavanti sūrā vīraṅgarūpā parasenappamaddanā. So imaṃ pathaviṃ sāgarapariyantaṃ adaṇḍena asatthena dhammena abhivijiya ajjhāvasati. Sace kho pana agārasmā anagāriyaṃ pabbajati, arahaṃ hoti sammāsambuddho loke vivaṭacchado.

\paragraph{51.} ‘Ayañhi, deva, kumāro suppatiṭṭhitapādo. Yaṃ pāyaṃ, deva, kumāro suppatiṭṭhitapādo. Idampissa mahāpurisassa mahāpurisalakkhaṇaṃ bhavati.

\paragraph{52.} ‘Imassa, deva\footnote{imassa hi deva (?)}, kumārassa heṭṭhā pādatalesu cakkāni jātāni sahassārāni sanemikāni sanābhikāni sabbākāraparipūrāni. Yampi, imassa deva, kumārassa heṭṭhā pādatalesu cakkāni jātāni sahassārāni sanemikāni sanābhikāni sabbākāraparipūrāni, idampissa mahāpurisassa mahāpurisalakkhaṇaṃ bhavati.

\paragraph{53.} ‘Ayañhi deva, kumāro āyatapaṇhī…pe…

\paragraph{54.} ‘Ayañhi, deva, kumāro dīghaṅgulī…

\paragraph{55.} ‘Ayañhi, deva, kumāro mudutalunahatthapādo…

\paragraph{56.} ‘Ayañhi, deva kumāro jālahatthapādo…

\paragraph{57.} ‘Ayañhi, deva, kumāro ussaṅkhapādo…

\paragraph{58.} ‘Ayañhi, deva, kumāro eṇijaṅgho…

\paragraph{59.} ‘Ayañhi, deva, kumāro ṭhitakova anonamanto ubhohi pāṇitalehi jaṇṇukāni parimasati\footnote{parāmasati (ka.)} parimajjati…

\paragraph{60.} ‘Ayañhi , deva, kumāro kosohitavatthaguyho…

\paragraph{61.} ‘Ayañhi, deva, kumāro suvaṇṇavaṇṇo kañcanasannibhattaco…

\paragraph{62.} ‘Ayañhi, deva, kumāro sukhumacchavī; sukhumattā chaviyā rajojallaṃ kāye na upalimpati\footnote{upalippati (syā.)} …

\paragraph{63.} ‘Ayañhi, deva, kumāro ekekalomo; ekekāni lomāni lomakūpesu jātāni…

\paragraph{64.} ‘Ayañhi, deva, kumāro uddhaggalomo; uddhaggāni lomāni jātāni nīlāni añjanavaṇṇāni kuṇḍalāvaṭṭāni dakkhiṇāvaṭṭakajātāni…

\paragraph{65.} ‘Ayañhi, deva, kumāro brahmujugatto…

\paragraph{66.} ‘Ayañhi, deva, kumāro sattussado…

\paragraph{67.} ‘Ayañhi , deva, kumāro sīhapubbaddhakāyo…

\paragraph{68.} ‘Ayañhi, deva, kumāro citantaraṃso\footnote{pitantaraṃso (syā.)} …

\paragraph{69.} ‘Ayañhi, deva, kumāro nigrodhaparimaṇḍalo yāvatakvassa kāyo tāvatakvassa byāmo, yāvatakvassa byāmo, tāvatakvassa kāyo…

\paragraph{70.} ‘Ayañhi , deva, kumāro samavaṭṭakkhandho…

\paragraph{71.} ‘Ayañhi, deva, kumāro rasaggasaggī…

\paragraph{72.} ‘Ayañhi, deva, kumāro sīhahanu…

\paragraph{73.} ‘Ayañhi, deva, kumāro cattālīsadanto…

\paragraph{74.} ‘Ayañhi, deva, kumāro samadanto…

\paragraph{75.} ‘Ayañhi, deva, kumāro aviraḷadanto…

\paragraph{76.} ‘Ayañhi, deva, kumāro susukkadāṭho…

\paragraph{77.} ‘Ayañhi, deva, kumāro pahūtajivho…

\paragraph{78.} ‘Ayañhi, deva, kumāro brahmassaro karavīkabhāṇī…

\paragraph{79.} ‘Ayañhi, deva, kumāro abhinīlanetto…

\paragraph{80.} ‘Ayañhi, deva, kumāro gopakhumo…

\paragraph{81.} Imassa, deva, kumārassa uṇṇā bhamukantare jātā odātā mudutūlasannibhā. Yampi imassa deva kumārassa uṇṇā bhamukantare jātā odātā mudutūlasannibhā, idampimassa mahāpurisassa mahāpurisalakkhaṇaṃ bhavati.

\paragraph{82.} ‘Ayañhi , deva, kumāro uṇhīsasīso. Yaṃ pāyaṃ, deva, kumāro uṇhīsasīso, idampissa mahāpurisassa mahāpurisalakkhaṇaṃ bhavati.

\paragraph{83.} ‘Imehi kho ayaṃ, deva, kumāro dvattiṃsamahāpurisalakkhaṇehi samannāgato, yehi samannāgatassa mahāpurisassa dveva gatiyo bhavanti anaññā. Sace agāraṃ ajjhāvasati, rājā hoti cakkavattī dhammiko dhammarājā cāturanto vijitāvī janapadatthāvariyappatto sattaratanasamannāgato. Tassimāni sattaratanāni bhavanti. Seyyathidaṃ – cakkaratanaṃ hatthiratanaṃ assaratanaṃ maṇiratanaṃ itthiratanaṃ gahapatiratanaṃ pariṇāyakaratanameva sattamaṃ. Parosahassaṃ kho panassa puttā bhavanti sūrā vīraṅgarūpā parasenappamaddanā. So imaṃ pathaviṃ sāgarapariyantaṃ adaṇḍena asatthena dhammena\footnote{dhammena samena (syā.)} abhivijiya ajjhāvasati. Sace kho pana agārasmā anagāriyaṃ pabbajati, arahaṃ hoti sammāsambuddho loke vivaṭacchado’ti.

\subsubsection{Vipassīsamaññā}

\paragraph{84.} ‘‘Atha kho, bhikkhave, bandhumā rājā nemitte brāhmaṇe ahatehi vatthehi acchādāpetvā\footnote{acchādetvā (syā.)} sabbakāmehi santappesi. Atha kho, bhikkhave, bandhumā rājā vipassissa kumārassa dhātiyo upaṭṭhāpesi. Aññā khīraṃ pāyenti, aññā nhāpenti, aññā dhārenti, aññā aṅkena pariharanti. Jātassa kho pana, bhikkhave, vipassissa kumārassa setacchattaṃ dhārayittha divā ceva rattiñca – ‘mā naṃ sītaṃ vā uṇhaṃ vā tiṇaṃ vā rajo vā ussāvo vā bādhayitthā’ti. Jāto kho pana, bhikkhave, vipassī kumāro bahuno janassa piyo ahosi manāpo. Seyyathāpi, bhikkhave, uppalaṃ vā padumaṃ vā puṇḍarīkaṃ vā bahuno janassa piyaṃ manāpaṃ; evameva kho, bhikkhave, vipassī kumāro bahuno janassa piyo ahosi manāpo. Svāssudaṃ aṅkeneva aṅkaṃ parihariyati.

\paragraph{85.} ‘‘Jāto kho pana, bhikkhave, vipassī kumāro mañjussaro ca\footnote{kumāro brahmassaro mañjussaro ca (sī. ka.)} ahosi vaggussaro ca madhurassaro ca pemaniyassaro ca. Seyyathāpi, bhikkhave, himavante pabbate karavīkā nāma sakuṇajāti mañjussarā ca vaggussarā ca madhurassarā ca pemaniyassarā ca; evameva kho, bhikkhave, vipassī kumāro mañjussaro ca ahosi vaggussaro ca madhurassaro ca pemaniyassaro ca.

\paragraph{86.} ‘‘Jātassa kho pana, bhikkhave, vipassissa kumārassa kammavipākajaṃ dibbacakkhu pāturahosi yena sudaṃ\footnote{yena dūraṃ (syā.)} samantā yojanaṃ passati divā ceva rattiñca.

\paragraph{87.} ‘‘Jāto kho pana, bhikkhave, vipassī kumāro animisanto pekkhati seyyathāpi devā tāvatiṃsā. ‘Animisanto kumāro pekkhatī’ti kho, bhikkhave\footnote{animisanto pekkhati, jātassa kho pana bhikkhave (ka.)}, vipassissa kumārassa ‘vipassī vipassī’ tveva samaññā udapādi.

\paragraph{88.} ‘‘Atha kho, bhikkhave, bandhumā rājā atthakaraṇe\footnote{aṭṭa karaṇe (syā.)} nisinno vipassiṃ kumāraṃ aṅke nisīdāpetvā atthe anusāsati . Tatra sudaṃ, bhikkhave, vipassī kumāro pituaṅke nisinno viceyya viceyya atthe panāyati ñāyena\footnote{aṭṭe panāyati ñāṇena (syā.)}. Viceyya viceyya kumāro atthe panāyati ñāyenāti kho, bhikkhave, vipassissa kumārassa bhiyyosomattāya ‘vipassī vipassī’ tveva samaññā udapādi.

\paragraph{89.} ‘‘Atha kho, bhikkhave, bandhumā rājā vipassissa kumārassa tayo pāsāde kārāpesi, ekaṃ vassikaṃ ekaṃ hemantikaṃ ekaṃ gimhikaṃ; pañca kāmaguṇāni upaṭṭhāpesi. Tatra sudaṃ, bhikkhave, vipassī kumāro vassike pāsāde cattāro māse\footnote{vassike pāsāde vassike} nippurisehi tūriyehi paricārayamāno na heṭṭhāpāsādaṃ orohatī’’ti.

\xsubsubsectionEnd{Paṭhamabhāṇavāro.}

\subsubsection{Jiṇṇapuriso}

\paragraph{90.} ‘‘Atha kho, bhikkhave, vipassī kumāro bahūnaṃ vassānaṃ bahūnaṃ vassasatānaṃ bahūnaṃ vassasahassānaṃ accayena sārathiṃ āmantesi – ‘yojehi, samma sārathi, bhaddāni bhaddāni yānāni uyyānabhūmiṃ gacchāma subhūmidassanāyā’ti. ‘Evaṃ, devā’ti kho, bhikkhave, sārathi vipassissa kumārassa paṭissutvā bhaddāni bhaddāni yānāni yojetvā vipassissa kumārassa paṭivedesi – ‘yuttāni kho te, deva, bhaddāni bhaddāni yānāni, yassa dāni kālaṃ maññasī’ti . Atha kho, bhikkhave, vipassī kumāro bhaddaṃ bhaddaṃ yānaṃ\footnote{bhadraṃ yānaṃ (syā.), bhaddaṃ yānaṃ (pī.) cattāro māse (sī. pī.)} abhiruhitvā bhaddehi bhaddehi yānehi uyyānabhūmiṃ niyyāsi.

\paragraph{91.} ‘‘Addasā kho, bhikkhave, vipassī kumāro uyyānabhūmiṃ niyyanto purisaṃ jiṇṇaṃ gopānasivaṅkaṃ bhoggaṃ\footnote{bhaggaṃ (syā.)} daṇḍaparāyanaṃ pavedhamānaṃ gacchantaṃ āturaṃ gatayobbanaṃ. Disvā sārathiṃ āmantesi – ‘ayaṃ pana, samma sārathi, puriso kiṃkato? Kesāpissa na yathā aññesaṃ, kāyopissa na yathā aññesa’nti. ‘Eso kho, deva, jiṇṇo nāmā’ti. ‘Kiṃ paneso, samma sārathi, jiṇṇo nāmā’ti? ‘Eso kho, deva, jiṇṇo nāma. Na dāni tena ciraṃ jīvitabbaṃ bhavissatī’ti. ‘Kiṃ pana, samma sārathi, ahampi jarādhammo, jaraṃ anatīto’ti? ‘Tvañca, deva, mayañcamha sabbe jarādhammā, jaraṃ anatītā’ti. ‘Tena hi, samma sārathi, alaṃ dānajja uyyānabhūmiyā. Itova antepuraṃ paccaniyyāhī’ti. ‘Evaṃ, devā’ti kho, bhikkhave, sārathi vipassissa kumārassa paṭissutvā tatova antepuraṃ paccaniyyāsi. Tatra sudaṃ, bhikkhave, vipassī kumāro antepuraṃ gato dukkhī dummano pajjhāyati – ‘dhiratthu kira, bho, jāti nāma, yatra hi nāma jātassa jarā paññāyissatī’ti!

\paragraph{92.} ‘‘Atha kho, bhikkhave, bandhumā rājā sārathiṃ āmantāpetvā etadavoca – ‘kacci, samma sārathi, kumāro uyyānabhūmiyā abhiramittha? Kacci, samma sārathi, kumāro uyyānabhūmiyā attamano ahosī’ti? ‘Na kho, deva, kumāro uyyānabhūmiyā abhiramittha, na kho, deva, kumāro uyyānabhūmiyā attamano ahosī’ti. ‘Kiṃ pana, samma sārathi, addasa kumāro uyyānabhūmiṃ niyyanto’ti? ‘Addasā kho, deva, kumāro uyyānabhūmiṃ niyyanto purisaṃ jiṇṇaṃ gopānasivaṅkaṃ bhoggaṃ daṇḍaparāyanaṃ pavedhamānaṃ gacchantaṃ āturaṃ gatayobbanaṃ. Disvā maṃ etadavoca – ‘‘ayaṃ pana, samma sārathi, puriso kiṃkato, kesāpissa na yathā aññesaṃ, kāyopissa na yathā aññesa’’nti? ‘‘Eso kho, deva, jiṇṇo nāmā’’ti. ‘‘Kiṃ paneso, samma sārathi, jiṇṇo nāmā’’ti? ‘‘Eso kho, deva, jiṇṇo nāma na dāni tena ciraṃ jīvitabbaṃ bhavissatī’’ti. ‘‘Kiṃ pana, samma sārathi, ahampi jarādhammo, jaraṃ anatīto’’ti? ‘‘Tvañca, deva, mayañcamha sabbe jarādhammā, jaraṃ anatītā’’ti.

\paragraph{93.} ‘‘‘Tena hi, samma sārathi, alaṃ dānajja uyyānabhūmiyā, itova antepuraṃ paccaniyyāhī’’’ti. ‘‘Evaṃ, devā’’ti kho ahaṃ, deva, vipassissa kumārassa paṭissutvā tatova antepuraṃ paccaniyyāsiṃ. So kho, deva, kumāro antepuraṃ gato dukkhī dummano pajjhāyati – ‘‘dhiratthu kira bho jāti nāma, yatra hi nāma jātassa jarā paññāyissatī’’’ti.

\subsubsection{Byādhitapuriso}

\paragraph{94.} ‘‘Atha kho, bhikkhave, bandhumassa rañño etadahosi –

\paragraph{95.}‘Mā heva kho vipassī kumāro na rajjaṃ kāresi, mā heva vipassī kumāro agārasmā anagāriyaṃ pabbaji, mā heva nemittānaṃ brāhmaṇānaṃ saccaṃ assa vacana’nti. Atha kho, bhikkhave, bandhumā rājā vipassissa kumārassa bhiyyosomattāya pañca kāmaguṇāni upaṭṭhāpesi – ‘yathā vipassī kumāro rajjaṃ kareyya, yathā vipassī kumāro na agārasmā anagāriyaṃ pabbajeyya, yathā nemittānaṃ brāhmaṇānaṃ micchā assa vacana’nti.

\paragraph{96.} ‘‘Tatra sudaṃ, bhikkhave, vipassī kumāro pañcahi kāmaguṇehi samappito samaṅgībhūto paricāreti. Atha kho, bhikkhave, vipassī kumāro bahūnaṃ vassānaṃ…pe…

\paragraph{97.} ‘‘Addasā kho, bhikkhave, vipassī kumāro uyyānabhūmiṃ niyyanto purisaṃ ābādhikaṃ dukkhitaṃ bāḷhagilānaṃ sake muttakarīse palipannaṃ semānaṃ\footnote{sayamānaṃ (syā. ka.)} aññehi vuṭṭhāpiyamānaṃ aññehi saṃvesiyamānaṃ. Disvā sārathiṃ āmantesi – ‘ayaṃ pana, samma sārathi, puriso kiṃkato? Akkhīnipissa na yathā aññesaṃ, saropissa\footnote{siropissa (syā.)} na yathā aññesa’nti? ‘Eso kho, deva, byādhito nāmā’ti. ‘Kiṃ paneso, samma sārathi, byādhito nāmā’ti? ‘Eso kho, deva, byādhito nāma appeva nāma tamhā ābādhā vuṭṭhaheyyā’ti. ‘Kiṃ pana, samma sārathi, ahampi byādhidhammo, byādhiṃ anatīto’ti? ‘Tvañca, deva, mayañcamha sabbe byādhidhammā, byādhiṃ anatītā’ti. ‘Tena hi, samma sārathi, alaṃ dānajja uyyānabhūmiyā, itova antepuraṃ paccaniyyāhī’ti. ‘Evaṃ devā’ti kho, bhikkhave, sārathi vipassissa kumārassa paṭissutvā tatova antepuraṃ paccaniyyāsi. Tatra sudaṃ, bhikkhave, vipassī kumāro antepuraṃ gato dukkhī dummano pajjhāyati – ‘dhiratthu kira bho jāti nāma, yatra hi nāma jātassa jarā paññāyissati, byādhi paññāyissatī’ti.

\paragraph{98.} ‘‘Atha kho, bhikkhave, bandhumā rājā sārathiṃ āmantāpetvā etadavoca – ‘kacci, samma sārathi, kumāro uyyānabhūmiyā abhiramittha, kacci, samma sārathi, kumāro uyyānabhūmiyā attamano ahosī’ti? ‘Na kho, deva, kumāro uyyānabhūmiyā abhiramittha, na kho, deva, kumāro uyyānabhūmiyā attamano ahosī’ti. ‘Kiṃ pana, samma sārathi, addasa kumāro uyyānabhūmiṃ niyyanto’ti? ‘Addasā kho, deva, kumāro uyyānabhūmiṃ niyyanto purisaṃ ābādhikaṃ dukkhitaṃ bāḷhagilānaṃ sake muttakarīse palipannaṃ semānaṃ aññehi vuṭṭhāpiyamānaṃ aññehi saṃvesiyamānaṃ. Disvā maṃ etadavoca – ‘‘ayaṃ pana, samma sārathi, puriso kiṃkato, akkhīnipissa na yathā aññesaṃ, saropissa na yathā aññesa’’nti? ‘‘Eso kho, deva, byādhito nāmā’’ti. ‘‘Kiṃ paneso, samma sārathi, byādhito nāmā’’ti? ‘‘Eso kho, deva, byādhito nāma appeva nāma tamhā ābādhā vuṭṭhaheyyā’’ti. ‘‘Kiṃ pana, samma sārathi, ahampi byādhidhammo, byādhiṃ anatīto’’ti? ‘‘Tvañca, deva, mayañcamha sabbe byādhidhammā, byādhiṃ anatītā’’ti. ‘‘Tena hi, samma sārathi, alaṃ dānajja uyyānabhūmiyā, itova antepuraṃ paccaniyyāhī’’ti. ‘‘Evaṃ, devā’’ti kho ahaṃ, deva, vipassissa kumārassa paṭissutvā tatova antepuraṃ paccaniyyāsiṃ. So kho, deva, kumāro antepuraṃ gato dukkhī dummano pajjhāyati – ‘‘‘dhiratthu kira bho jāti nāma, yatra hi nāma jātassa jarā paññāyissati, byādhi paññāyissatī’’’ti.

\subsubsection{Kālaṅkatapuriso}

\paragraph{99.} ‘‘Atha kho, bhikkhave, bandhumassa rañño etadahosi – ‘mā heva kho vipassī kumāro na rajjaṃ kāresi, mā heva vipassī kumāro agārasmā anagāriyaṃ pabbaji, mā heva nemittānaṃ brāhmaṇānaṃ saccaṃ assa vacana’nti. Atha kho, bhikkhave, bandhumā rājā vipassissa kumārassa bhiyyosomattāya pañca kāmaguṇāni upaṭṭhāpesi – ‘yathā vipassī kumāro rajjaṃ kareyya, yathā vipassī kumāro na agārasmā anagāriyaṃ pabbajeyya, yathā nemittānaṃ brāhmaṇānaṃ micchā assa vacana’nti.

\paragraph{100.} ‘‘Tatra sudaṃ, bhikkhave, vipassī kumāro pañcahi kāmaguṇehi samappito samaṅgībhūto paricāreti. Atha kho, bhikkhave, vipassī kumāro bahūnaṃ vassānaṃ…pe…

\paragraph{101.} ‘‘Addasā kho, bhikkhave, vipassī kumāro uyyānabhūmiṃ niyyanto mahājanakāyaṃ sannipatitaṃ nānārattānañca dussānaṃ vilātaṃ kayiramānaṃ. Disvā sārathiṃ āmantesi – ‘kiṃ nu kho, so, samma sārathi, mahājanakāyo sannipatito nānārattānañca dussānaṃ vilātaṃ kayiratī’ti? ‘Eso kho, deva, kālaṅkato nāmā’ti. ‘Tena hi, samma sārathi, yena so kālaṅkato tena rathaṃ pesehī’ti. ‘Evaṃ, devā’ti kho, bhikkhave, sārathi vipassissa kumārassa paṭissutvā yena so kālaṅkato tena rathaṃ pesesi. Addasā kho, bhikkhave, vipassī kumāro petaṃ kālaṅkataṃ, disvā sārathiṃ āmantesi – ‘kiṃ panāyaṃ, samma sārathi, kālaṅkato nāmā’ti? ‘Eso kho, deva, kālaṅkato nāma. Na dāni taṃ dakkhanti mātā vā pitā vā aññe vā ñātisālohitā, sopi na dakkhissati mātaraṃ vā pitaraṃ vā aññe vā ñātisālohite’ti. ‘Kiṃ pana, samma sārathi, ahampi maraṇadhammo maraṇaṃ anatīto; mampi na dakkhanti devo vā devī vā aññe vā ñātisālohitā; ahampi na dakkhissāmi devaṃ vā deviṃ vā aññe vā ñātisālohite’ti? ‘Tvañca, deva, mayañcamha sabbe maraṇadhammā maraṇaṃ anatītā; tampi na dakkhanti devo vā devī vā aññe vā ñātisālohitā; tvampi na dakkhissasi devaṃ vā deviṃ vā aññe vā ñātisālohite’ti. ‘Tena hi, samma sārathi, alaṃ dānajja uyyānabhūmiyā, itova antepuraṃ paccaniyyāhī’ti. ‘Evaṃ, devā’ti kho, bhikkhave, sārathi vipassissa kumārassa paṭissutvā tatova antepuraṃ paccaniyyāsi. Tatra sudaṃ, bhikkhave, vipassī kumāro antepuraṃ gato dukkhī dummano pajjhāyati – ‘dhiratthu kira, bho, jāti nāma, yatra hi nāma jātassa jarā paññāyissati, byādhi paññāyissati, maraṇaṃ paññāyissatī’ti.

\paragraph{102.} ‘‘Atha kho, bhikkhave, bandhumā rājā sārathiṃ āmantāpetvā etadavoca – ‘kacci, samma sārathi, kumāro uyyānabhūmiyā abhiramittha, kacci, samma sārathi, kumāro uyyānabhūmiyā attamano ahosī’ti? ‘Na kho, deva, kumāro uyyānabhūmiyā abhiramittha, na kho, deva, kumāro uyyānabhūmiyā attamano ahosī’ti. ‘Kiṃ pana, samma sārathi, addasa kumāro uyyānabhūmiṃ niyyanto’ti? ‘Addasā kho, deva, kumāro uyyānabhūmiṃ niyyanto mahājanakāyaṃ sannipatitaṃ nānārattānañca dussānaṃ vilātaṃ kayiramānaṃ. Disvā maṃ etadavoca – ‘‘kiṃ nu kho, so , samma sārathi, mahājanakāyo sannipatito nānārattānañca dussānaṃ vilātaṃ kayiratī’’ti? ‘‘Eso kho, deva, kālaṅkato nāmā’’ti. ‘‘Tena hi, samma sārathi, yena so kālaṅkato tena rathaṃ pesehī’’ti. ‘‘Evaṃ devā’’ti kho ahaṃ, deva, vipassissa kumārassa paṭissutvā yena so kālaṅkato tena rathaṃ pesesiṃ. Addasā kho, deva, kumāro petaṃ kālaṅkataṃ, disvā maṃ etadavoca – ‘‘kiṃ panāyaṃ, samma sārathi, kālaṅkato nāmā’’ti ? ‘‘Eso kho, deva, kālaṅkato nāma. Na dāni taṃ dakkhanti mātā vā pitā vā aññe vā ñātisālohitā, sopi na dakkhissati mātaraṃ vā pitaraṃ vā aññe vā ñātisālohite’’ti. ‘‘Kiṃ pana, samma sārathi, ahampi maraṇadhammo maraṇaṃ anatīto; mampi na dakkhanti devo vā devī vā aññe vā ñātisālohitā; ahampi na dakkhissāmi devaṃ vā deviṃ vā aññe vā ñātisālohite’’ti? ‘‘Tvañca, deva, mayañcamha sabbe maraṇadhammā maraṇaṃ anatītā; tampi na dakkhanti devo vā devī vā aññe vā ñātisālohitā, tvampi na dakkhissasi devaṃ vā deviṃ vā aññe vā ñātisālohite’’ti. ‘‘Tena hi, samma sārathi, alaṃ dānajja uyyānabhūmiyā, itova antepuraṃ paccaniyyāhī’ti. ‘‘‘Evaṃ, devā’’ti kho ahaṃ, deva, vipassissa kumārassa paṭissutvā tatova antepuraṃ paccaniyyāsiṃ. So kho, deva, kumāro antepuraṃ gato dukkhī dummano pajjhāyati – ‘‘dhiratthu kira bho jāti nāma, yatra hi nāma jātassa jarā paññāyissati, byādhi paññāyissati, maraṇaṃ paññāyissatī’’’ti.

\subsubsection{Pabbajito}

\paragraph{103.} ‘‘Atha kho, bhikkhave, bandhumassa rañño etadahosi – ‘mā heva kho vipassī kumāro na rajjaṃ kāresi, mā heva vipassī kumāro agārasmā anagāriyaṃ pabbaji, mā heva nemittānaṃ brāhmaṇānaṃ saccaṃ assa vacana’nti. Atha kho, bhikkhave, bandhumā rājā vipassissa kumārassa bhiyyosomattāya pañca kāmaguṇāni upaṭṭhāpesi – ‘yathā vipassī kumāro rajjaṃ kareyya, yathā vipassī kumāro na agārasmā anagāriyaṃ pabbajeyya, yathā nemittānaṃ brāhmaṇānaṃ micchā assa vacana’nti.

\paragraph{104.} ‘‘Tatra sudaṃ, bhikkhave, vipassī kumāro pañcahi kāmaguṇehi samappito samaṅgībhūto paricāreti. Atha kho, bhikkhave, vipassī kumāro bahūnaṃ vassānaṃ bahūnaṃ vassasatānaṃ bahūnaṃ vassasahassānaṃ accayena sārathiṃ āmantesi – ‘yojehi, samma sārathi, bhaddāni bhaddāni yānāni, uyyānabhūmiṃ gacchāma subhūmidassanāyā’ti. ‘Evaṃ, devā’ti kho, bhikkhave, sārathi vipassissa kumārassa paṭissutvā bhaddāni bhaddāni yānāni yojetvā vipassissa kumārassa paṭivedesi – ‘yuttāni kho te, deva, bhaddāni bhaddāni yānāni, yassa dāni kālaṃ maññasī’ti. Atha kho, bhikkhave, vipassī kumāro bhaddaṃ bhaddaṃ yānaṃ abhiruhitvā bhaddehi bhaddehi yānehi uyyānabhūmiṃ niyyāsi.

\paragraph{105.} ‘‘Addasā kho, bhikkhave, vipassī kumāro uyyānabhūmiṃ niyyanto purisaṃ bhaṇḍuṃ pabbajitaṃ kāsāyavasanaṃ. Disvā sārathiṃ āmantesi – ‘ayaṃ pana, samma sārathi, puriso kiṃkato? Sīsaṃpissa na yathā aññesaṃ, vatthānipissa na yathā aññesa’nti? ‘Eso kho, deva, pabbajito nāmā’ti. ‘Kiṃ paneso, samma sārathi, pabbajito nāmā’ti? ‘Eso kho, deva, pabbajito nāma sādhu dhammacariyā sādhu samacariyā\footnote{sammacariyā (ka.)} sādhu kusalakiriyā\footnote{kusalacariyā (syā.)} sādhu puññakiriyā sādhu avihiṃsā sādhu bhūtānukampā’ti. ‘Sādhu kho so, samma sārathi, pabbajito nāma, sādhu dhammacariyā sādhu samacariyā sādhu kusalakiriyā sādhu puññakiriyā sādhu avihiṃsā sādhu bhūtānukampā. Tena hi, samma sārathi, yena so pabbajito tena rathaṃ pesehī’ti. ‘Evaṃ, devā’ti kho, bhikkhave, sārathi vipassissa kumārassa paṭissutvā yena so pabbajito tena rathaṃ pesesi. Atha kho, bhikkhave, vipassī kumāro taṃ pabbajitaṃ etadavoca – ‘tvaṃ pana, samma, kiṃkato, sīsampi te na yathā aññesaṃ, vatthānipi te na yathā aññesa’nti? ‘Ahaṃ kho, deva, pabbajito nāmā’ti. ‘Kiṃ pana tvaṃ, samma, pabbajito nāmā’ti? ‘Ahaṃ kho, deva, pabbajito nāma, sādhu dhammacariyā sādhu samacariyā sādhu kusalakiriyā sādhu puññakiriyā sādhu avihiṃsā sādhu bhūtānukampā’ti. ‘Sādhu kho tvaṃ, samma, pabbajito nāma sādhu dhammacariyā sādhu samacariyā sādhu kusalakiriyā sādhu puññakiriyā sādhu avihiṃsā sādhu bhūtānukampā’ti.

\subsubsection{Bodhisattapabbajjā}

\paragraph{106.} ‘‘Atha kho, bhikkhave, vipassī kumāro sārathiṃ āmantesi – ‘tena hi, samma sārathi, rathaṃ ādāya itova antepuraṃ paccaniyyāhi. Ahaṃ pana idheva kesamassuṃ ohāretvā kāsāyāni vatthāni acchādetvā agārasmā anagāriyaṃ pabbajissāmī’ti. ‘Evaṃ, devā’ti kho, bhikkhave, sārathi vipassissa kumārassa paṭissutvā rathaṃ ādāya tatova antepuraṃ paccaniyyāsi. Vipassī pana kumāro tattheva kesamassuṃ ohāretvā kāsāyāni vatthāni acchādetvā agārasmā anagāriyaṃ pabbaji.

\subsubsection{Mahājanakāyaanupabbajjā}

\paragraph{107.} ‘‘Assosi kho, bhikkhave, bandhumatiyā rājadhāniyā mahājanakāyo caturāsīti pāṇasahassāni – ‘vipassī kira kumāro kesamassuṃ ohāretvā kāsāyāni vatthāni acchādetvā agārasmā anagāriyaṃ pabbajito’ti. Sutvāna tesaṃ etadahosi – ‘na hi nūna so orako dhammavinayo, na sā orakā\footnote{orikā (sī. syā.)} pabbajjā, yattha vipassī kumāro kesamassuṃ ohāretvā kāsāyāni vatthāni acchādetvā agārasmā anagāriyaṃ pabbajito. Vipassīpi nāma kumāro kesamassuṃ ohāretvā kāsāyāni vatthāni acchādetvā agārasmā anagāriyaṃ pabbajissati, kimaṅgaṃ\footnote{kimaṅga (sī.)} pana maya’nti.

\paragraph{108.} ‘‘Atha kho, so bhikkhave, mahājanakāyo\footnote{mahājanakāyo (syā.)} caturāsīti pāṇasahassāni kesamassuṃ ohāretvā kāsāyāni vatthāni acchādetvā vipassiṃ bodhisattaṃ agārasmā anagāriyaṃ pabbajitaṃ anupabbajiṃsu. Tāya sudaṃ, bhikkhave, parisāya parivuto vipassī bodhisatto gāmanigamajanapadarājadhānīsu cārikaṃ carati.

\paragraph{109.} ‘‘Atha kho, bhikkhave, vipassissa bodhisattassa rahogatassa paṭisallīnassa evaṃ cetaso parivitakko udapādi – ‘na kho metaṃ\footnote{na kho panetaṃ (syā.)} patirūpaṃ yohaṃ ākiṇṇo viharāmi, yaṃnūnāhaṃ eko gaṇamhā vūpakaṭṭho vihareyya’nti. Atha kho, bhikkhave, vipassī bodhisatto aparena samayena eko gaṇamhā vūpakaṭṭho vihāsi , aññeneva tāni caturāsīti pabbajitasahassāni agamaṃsu, aññena maggena vipassī bodhisatto.

\subsubsection{Bodhisattaabhiniveso}

\paragraph{110.} ‘‘Atha kho, bhikkhave, vipassissa bodhisattassa vāsūpagatassa rahogatassa paṭisallīnassa evaṃ cetaso parivitakko udapādi – ‘kicchaṃ vatāyaṃ loko āpanno, jāyati ca jīyati ca mīyati ca\footnote{jiyyati ca miyyati ca (ka.)} cavati ca upapajjati ca, atha ca panimassa dukkhassa nissaraṇaṃ nappajānāti jarāmaraṇassa, kudāssu nāma imassa dukkhassa nissaraṇaṃ paññāyissati jarāmaraṇassā’ti?

\paragraph{111.} ‘‘Atha kho, bhikkhave, vipassissa bodhisattassa etadahosi – ‘kimhi nu kho sati jarāmaraṇaṃ hoti, kiṃpaccayā jarāmaraṇa’nti? Atha kho, bhikkhave, vipassissa bodhisattassa yoniso manasikārā ahu paññāya abhisamayo – ‘jātiyā kho sati jarāmaraṇaṃ hoti, jātipaccayā jarāmaraṇa’nti.

\paragraph{112.} ‘‘Atha kho, bhikkhave, vipassissa bodhisattassa etadahosi – ‘kimhi nu kho sati jāti hoti, kiṃpaccayā jātī’ti? Atha kho, bhikkhave, vipassissa bodhisattassa yoniso manasikārā ahu paññāya abhisamayo – ‘bhave kho sati jāti hoti, bhavapaccayā jātī’ti.

\paragraph{113.} ‘‘Atha kho, bhikkhave, vipassissa bodhisattassa etadahosi – ‘kimhi nu kho sati bhavo hoti, kiṃpaccayā bhavo’ti? Atha kho, bhikkhave, vipassissa bodhisattassa yoniso manasikārā ahu paññāya abhisamayo – ‘upādāne kho sati bhavo hoti, upādānapaccayā bhavo’ti.

\paragraph{114.} ‘‘Atha kho, bhikkhave, vipassissa bodhisattassa etadahosi – ‘kimhi nu kho sati upādānaṃ hoti, kiṃpaccayā upādāna’nti? Atha kho, bhikkhave, vipassissa bodhisattassa yoniso manasikārā ahu paññāya abhisamayo – ‘taṇhāya kho sati upādānaṃ hoti, taṇhāpaccayā upādāna’nti.

\paragraph{115.} ‘‘Atha kho, bhikkhave, vipassissa bodhisattassa etadahosi – ‘kimhi nu kho sati taṇhā hoti, kiṃpaccayā taṇhā’ti? Atha kho, bhikkhave, vipassissa bodhisattassa yoniso manasikārā ahu paññāya abhisamayo – ‘vedanāya kho sati taṇhā hoti, vedanāpaccayā taṇhā’ti.

\paragraph{116.} ‘‘Atha kho, bhikkhave, vipassissa bodhisattassa etadahosi – ‘kimhi nu kho sati vedanā hoti, kiṃpaccayā vedanā’ti? Atha kho, bhikkhave, vipassissa bodhisattassa yoniso manasikārā ahu paññāya abhisamayo – ‘phasse kho sati vedanā hoti, phassapaccayā vedanā’ti.

\paragraph{117.} ‘‘Atha kho, bhikkhave, vipassissa bodhisattassa etadahosi – ‘kimhi nu kho sati phasso hoti, kiṃpaccayā phasso’ti? Atha kho, bhikkhave, vipassissa bodhisattassa yoniso manasikārā ahu paññāya abhisamayo – ‘saḷāyatane kho sati phasso hoti, saḷāyatanapaccayā phasso’ti.

\paragraph{118.} ‘‘Atha kho, bhikkhave, vipassissa bodhisattassa etadahosi – ‘kimhi nu kho sati saḷāyatanaṃ hoti, kiṃpaccayā saḷāyatana’nti? Atha kho, bhikkhave, vipassissa bodhisattassa yoniso manasikārā ahu paññāya abhisamayo – ‘nāmarūpe kho sati saḷāyatanaṃ hoti, nāmarūpapaccayā saḷāyatana’nti.

\paragraph{119.} ‘‘Atha kho, bhikkhave, vipassissa bodhisattassa etadahosi – ‘kimhi nu kho sati nāmarūpaṃ hoti, kiṃpaccayā nāmarūpa’nti? Atha kho, bhikkhave, vipassissa bodhisattassa yoniso manasikārā ahu paññāya abhisamayo – ‘viññāṇe kho sati nāmarūpaṃ hoti, viññāṇapaccayā nāmarūpa’nti.

\paragraph{120.} ‘‘Atha kho, bhikkhave, vipassissa bodhisattassa etadahosi – ‘kimhi nu kho sati viññāṇaṃ hoti, kiṃpaccayā viññāṇa’nti? Atha kho, bhikkhave, vipassissa bodhisattassa yoniso manasikārā ahu paññāya abhisamayo – ‘nāmarūpe kho sati viññāṇaṃ hoti, nāmarūpapaccayā viññāṇa’nti.

\paragraph{121.} ‘‘Atha kho, bhikkhave, vipassissa bodhisattassa etadahosi – ‘paccudāvattati kho idaṃ viññāṇaṃ nāmarūpamhā, nāparaṃ gacchati. Ettāvatā jāyetha vā jiyyetha vā miyyetha vā cavetha vā upapajjetha vā, yadidaṃ nāmarūpapaccayā viññāṇaṃ, viññāṇapaccayā nāmarūpaṃ, nāmarūpapaccayā saḷāyatanaṃ, saḷāyatanapaccayā phasso, phassapaccayā vedanā, vedanāpaccayā taṇhā , taṇhāpaccayā upādānaṃ, upādānapaccayā bhavo, bhavapaccayā jāti, jātipaccayā jarāmaraṇaṃ sokaparidevadukkhadomanassupāyāsā sambhavanti. Evametassa kevalassa dukkhakkhandhassa samudayo hoti’.

\paragraph{122.} ‘‘‘Samudayo samudayo’ti kho, bhikkhave, vipassissa bodhisattassa pubbe ananussutesu dhammesu cakkhuṃ udapādi, ñāṇaṃ udapādi, paññā udapādi, vijjā udapādi, āloko udapādi.

\paragraph{123.} ‘‘Atha kho, bhikkhave, vipassissa bodhisattassa etadahosi – ‘kimhi nu kho asati jarāmaraṇaṃ na hoti, kissa nirodhā jarāmaraṇanirodho’ti? Atha kho, bhikkhave, vipassissa bodhisattassa yoniso manasikārā ahu paññāya abhisamayo – ‘jātiyā kho asati jarāmaraṇaṃ na hoti, jātinirodhā jarāmaraṇanirodho’ti.

\paragraph{124.} ‘‘Atha kho, bhikkhave, vipassissa bodhisattassa etadahosi – ‘kimhi nu kho asati jāti na hoti, kissa nirodhā jātinirodho’ti? Atha kho, bhikkhave, vipassissa bodhisattassa yoniso manasikārā ahu paññāya abhisamayo – ‘bhave kho asati jāti na hoti, bhavanirodhā jātinirodho’ti.

\paragraph{125.} ‘‘Atha kho, bhikkhave, vipassissa bodhisattassa etadahosi – ‘kimhi nu kho asati bhavo na hoti, kissa nirodhā bhavanirodho’ti? Atha kho, bhikkhave, vipassissa bodhisattassa yoniso manasikārā ahu paññāya abhisamayo – ‘upādāne kho asati bhavo na hoti, upādānanirodhā bhavanirodho’ti.

\paragraph{126.} ‘‘Atha kho, bhikkhave, vipassissa bodhisattassa etadahosi – ‘kimhi nu kho asati upādānaṃ na hoti, kissa nirodhā upādānanirodho’ti? Atha kho, bhikkhave, vipassissa bodhisattassa yoniso manasikārā ahu paññāya abhisamayo – ‘taṇhāya kho asati upādānaṃ na hoti, taṇhānirodhā upādānanirodho’ti.

\paragraph{127.} ‘‘Atha kho, bhikkhave, vipassissa bodhisattassa etadahosi – ‘kimhi nu kho asati taṇhā na hoti, kissa nirodhā taṇhānirodho’ti? Atha kho, bhikkhave, vipassissa bodhisattassa yoniso manasikārā ahu paññāya abhisamayo – ‘vedanāya kho asati taṇhā na hoti, vedanānirodhā taṇhānirodho’ti.

\paragraph{128.} ‘‘Atha kho, bhikkhave, vipassissa bodhisattassa etadahosi – ‘kimhi nu kho asati vedanā na hoti, kissa nirodhā vedanānirodho’ti? Atha kho, bhikkhave, vipassissa bodhisattassa yoniso manasikārā ahu paññāya abhisamayo – ‘phasse kho asati vedanā na hoti, phassanirodhā vedanānirodho’ti.

\paragraph{129.} ‘‘Atha kho, bhikkhave, vipassissa bodhisattassa etadahosi – ‘kimhi nu kho asati phasso na hoti, kissa nirodhā phassanirodho’ti? Atha kho, bhikkhave, vipassissa bodhisattassa yoniso manasikārā ahu paññāya abhisamayo – ‘saḷāyatane kho asati phasso na hoti, saḷāyatananirodhā phassanirodho’ti.

\paragraph{130.} ‘‘Atha kho, bhikkhave, vipassissa bodhisattassa etadahosi – ‘kimhi nu kho asati saḷāyatanaṃ na hoti, kissa nirodhā saḷāyatananirodho’ti? Atha kho, bhikkhave, vipassissa bodhisattassa yoniso manasikārā ahu paññāya abhisamayo – ‘nāmarūpe kho asati saḷāyatanaṃ na hoti, nāmarūpanirodhā saḷāyatananirodho’ti.

\paragraph{131.} ‘‘Atha kho, bhikkhave, vipassissa bodhisattassa etadahosi – ‘kimhi nu kho asati nāmarūpaṃ na hoti, kissa nirodhā nāmarūpanirodho’ti? Atha kho, bhikkhave, vipassissa bodhisattassa yoniso manasikārā ahu paññāya abhisamayo – ‘viññāṇe kho asati nāmarūpaṃ na hoti, viññāṇanirodhā nāmarūpanirodho’ti.

\paragraph{132.} ‘‘Atha kho, bhikkhave, vipassissa bodhisattassa etadahosi – ‘kimhi nu kho asati viññāṇaṃ na hoti, kissa nirodhā viññāṇanirodho’ti? Atha kho, bhikkhave, vipassissa bodhisattassa yoniso manasikārā ahu paññāya abhisamayo – ‘nāmarūpe kho asati viññāṇaṃ na hoti, nāmarūpanirodhā viññāṇanirodho’ti.

\paragraph{133.} ‘‘Atha kho, bhikkhave, vipassissa bodhisattassa etadahosi – ‘adhigato kho myāyaṃ maggo sambodhāya yadidaṃ – nāmarūpanirodhā viññāṇanirodho, viññāṇanirodhā nāmarūpanirodho, nāmarūpanirodhā saḷāyatananirodho, saḷāyatananirodhā phassanirodho, phassanirodhā vedanānirodho, vedanānirodhā taṇhānirodho, taṇhānirodhā upādānanirodho, upādānanirodhā bhavanirodho, bhavanirodhā jātinirodho, jātinirodhā jarāmaraṇaṃ sokaparidevadukkhadomanassupāyāsā nirujjhanti. Evametassa kevalassa dukkhakkhandhassa nirodho hoti’.

\paragraph{134.} ‘‘‘Nirodho nirodho’ti kho, bhikkhave, vipassissa bodhisattassa pubbe ananussutesu dhammesu cakkhuṃ udapādi, ñāṇaṃ udapādi, paññā udapādi, vijjā udapādi, āloko udapādi.

\paragraph{135.} ‘‘Atha kho, bhikkhave, vipassī bodhisatto aparena samayena pañcasu upādānakkhandhesu udayabbayānupassī vihāsi – ‘iti rūpaṃ, iti rūpassa samudayo, iti rūpassa atthaṅgamo; iti vedanā, iti vedanāya samudayo, iti vedanāya atthaṅgamo; iti saññā, iti saññāya samudayo, iti saññāya atthaṅgamo; iti saṅkhārā, iti saṅkhārānaṃ samudayo, iti saṅkhārānaṃ atthaṅgamo; iti viññāṇaṃ, iti viññāṇassa samudayo, iti viññāṇassa atthaṅgamo’ti, tassa pañcasu upādānakkhandhesu udayabbayānupassino viharato na cirasseva anupādāya āsavehi cittaṃ vimuccī’’ti.

\xsubsubsectionEnd{Dutiyabhāṇavāro.}

\subsubsection{Brahmayācanakathā}

\paragraph{136.} ‘‘Atha kho, bhikkhave, vipassissa bhagavato arahato sammāsambuddhassa etadahosi – ‘yaṃnūnāhaṃ dhammaṃ deseyya’nti. Atha kho, bhikkhave, vipassissa bhagavato arahato sammāsambuddhassa etadahosi – ‘adhigato kho myāyaṃ dhammo gambhīro duddaso duranubodho santo paṇīto atakkāvacaro nipuṇo paṇḍitavedanīyo. Ālayarāmā kho panāyaṃ pajā ālayaratā ālayasammuditā. Ālayarāmāya kho pana pajāya ālayaratāya ālayasammuditāya duddasaṃ idaṃ ṭhānaṃ yadidaṃ idappaccayatāpaṭiccasamuppādo. Idampi kho ṭhānaṃ duddasaṃ yadidaṃ sabbasaṅkhārasamatho sabbūpadhipaṭinissaggo taṇhākkhayo virāgo nirodho nibbānaṃ. Ahañceva kho pana dhammaṃ deseyyaṃ, pare ca me na ājāneyyuṃ; so mamassa kilamatho, sā mamassa vihesā’ti.

\paragraph{137.} ‘‘Apissu, bhikkhave, vipassiṃ bhagavantaṃ arahantaṃ sammāsambuddhaṃ imā anacchariyā gāthāyo paṭibhaṃsu pubbe assutapubbā –

\paragraph{138.}\begin{verse}
  ‘Kicchena me adhigataṃ, \\halaṃ dāni pakāsituṃ;\\
  Rāgadosaparetehi, \\nāyaṃ dhammo susambudho.
\end{verse}

\paragraph{139.}\begin{verse}
  ‘Paṭisotagāmiṃ nipuṇaṃ, \\gambhīraṃ duddasaṃ aṇuṃ;\\
  Rāgarattā na dakkhanti, \\tamokhandhena āvuṭā’ti.
\end{verse}

\paragraph{140.} ‘‘Itiha , bhikkhave, vipassissa bhagavato arahato sammāsambuddhassa paṭisañcikkhato appossukkatāya cittaṃ nami, no dhammadesanāya.

\paragraph{141.} ‘‘Atha kho, bhikkhave, aññatarassa mahābrahmuno vipassissa bhagavato arahato sammāsambuddhassa cetasā cetoparivitakkamaññāya etadahosi – ‘nassati vata bho loko, vinassati vata bho loko, yatra hi nāma vipassissa bhagavato arahato sammāsambuddhassa appossukkatāya cittaṃ namati\footnote{nami (syā. ka.), namissati (?)}, no dhammadesanāyā’ti. Atha kho so, bhikkhave, mahābrahmā seyyathāpi nāma balavā puriso samiñjitaṃ vā bāhaṃ pasāreyya, pasāritaṃ vā bāhaṃ samiñjeyya; evameva brahmaloke antarahito vipassissa bhagavato arahato sammāsambuddhassa purato pāturahosi. Atha kho so, bhikkhave, mahābrahmā ekaṃsaṃ uttarāsaṅgaṃ karitvā dakkhiṇaṃ jāṇumaṇḍalaṃ pathaviyaṃ nihantvā\footnote{nidahanto (syā.)} yena vipassī bhagavā arahaṃ sammāsambuddho tenañjaliṃ paṇāmetvā vipassiṃ bhagavantaṃ arahantaṃ sammāsambuddhaṃ etadavoca – ‘desetu, bhante, bhagavā dhammaṃ, desetu sugato dhammaṃ, santi\footnote{santī (syā.)} sattā apparajakkhajātikā; assavanatā dhammassa parihāyanti, bhavissanti dhammassa aññātāro’ti.

\paragraph{142.} ‘‘Evaṃ vutte\footnote{atha kho (ka.)}, bhikkhave, vipassī bhagavā arahaṃ sammāsambuddho taṃ mahābrahmānaṃ etadavoca – ‘mayhampi kho, brahme, etadahosi – ‘‘yaṃnūnāhaṃ dhammaṃ deseyya’’nti. Tassa mayhaṃ, brahme, etadahosi – ‘‘adhigato kho myāyaṃ dhammo gambhīro duddaso duranubodho santo paṇīto atakkāvacaro nipuṇo paṇḍitavedanīyo. Ālayarāmā kho panāyaṃ pajā ālayaratā ālayasammuditā. Ālayarāmāya kho pana pajāya ālayaratāya ālayasammuditāya duddasaṃ idaṃ ṭhānaṃ yadidaṃ idappaccayatāpaṭiccasamuppādo. Idampi kho ṭhānaṃ duddasaṃ yadidaṃ sabbasaṅkhārasamatho sabbūpadhipaṭinissaggo taṇhākkhayo virāgo nirodho nibbānaṃ. Ahañceva kho pana dhammaṃ deseyyaṃ, pare ca me na ājāneyyuṃ; so mamassa kilamatho, sā mamassa vihesā’’ti. Apissu maṃ, brahme , imā anacchariyā gāthāyo paṭibhaṃsu pubbe assutapubbā –

\paragraph{143.}\begin{verse}
  ‘‘Kicchena me adhigataṃ, \\halaṃ dāni pakāsituṃ;\\
  Rāgadosaparetehi, nāyaṃ \\dhammo susambudho.
\end{verse}

\paragraph{144.}\begin{verse}
  ‘‘Paṭisotagāmiṃ nipuṇaṃ, \\gambhīraṃ duddasaṃ aṇuṃ;\\
  Rāgarattā na dakkhanti, \\tamokhandhena āvuṭā’’ti.
\end{verse}

\paragraph{145.} ‘Itiha me, brahme, paṭisañcikkhato appossukkatāya cittaṃ nami, no dhammadesanāyā’ti.

\paragraph{146.} ‘‘Dutiyampi kho, bhikkhave, so mahābrahmā…pe… tatiyampi kho, bhikkhave, so mahābrahmā vipassiṃ bhagavantaṃ arahantaṃ sammāsambuddhaṃ etadavoca – ‘desetu, bhante, bhagavā dhammaṃ, desetu sugato dhammaṃ, santi sattā apparajakkhajātikā, assavanatā dhammassa parihāyanti, bhavissanti dhammassa aññātāro’ti.

\paragraph{147.} ‘‘Atha kho, bhikkhave, vipassī bhagavā arahaṃ sammāsambuddho brahmuno ca ajjhesanaṃ viditvā sattesu ca kāruññataṃ paṭicca buddhacakkhunā lokaṃ volokesi. Addasā kho, bhikkhave, vipassī bhagavā arahaṃ sammāsambuddho buddhacakkhunā lokaṃ volokento satte apparajakkhe mahārajakkhe tikkhindriye mudindriye svākāre dvākāre suviññāpaye duviññāpaye\footnote{duviññāpaye bhabbe abhabbe (syā.)} appekacce paralokavajjabhayadassāvine\footnote{dassāvino (sī. syā. kaṃ. ka.)} viharante, appekacce na paralokavajjabhayadassāvine\footnote{dassāvino (sī. syā. kaṃ. ka.)} viharante. Seyyathāpi nāma uppaliniyaṃ vā paduminiyaṃ vā puṇḍarīkiniyaṃ vā appekaccāni uppalāni vā padumāni vā puṇḍarīkāni vā udake jātāni udake saṃvaḍḍhāni udakānuggatāni anto nimuggaposīni. Appekaccāni uppalāni vā padumāni vā puṇḍarīkāni vā udake jātāni udake saṃvaḍḍhāni samodakaṃ ṭhitāni. Appekaccāni uppalāni vā padumāni vā puṇḍarīkāni vā udake jātāni udake saṃvaḍḍhāni udakā accuggamma ṭhitāni anupalittāni udakena. Evameva kho, bhikkhave, vipassī bhagavā arahaṃ sammāsambuddho buddhacakkhunā lokaṃ volokento addasa satte apparajakkhe mahārajakkhe tikkhindriye mudindriye svākāre dvākāre suviññāpaye duviññāpaye appekacce paralokavajjabhayadassāvine viharante, appekacce na paralokavajjabhayadassāvine viharante.

\paragraph{148.} ‘‘Atha kho so, bhikkhave, mahābrahmā vipassissa bhagavato arahato sammāsambuddhassa cetasā cetoparivitakkamaññāya vipassiṃ bhagavantaṃ arahantaṃ sammāsambuddhaṃ gāthāhi ajjhabhāsi –

\paragraph{149.}\begin{verse}
  ‘Sele yathā pabbatamuddhaniṭṭhito, \\yathāpi passe janataṃ samantato;\\
  Tathūpamaṃ dhammamayaṃ sumedha, \\pāsādamāruyha samantacakkhu.\\
  ‘Sokāvatiṇṇaṃ\footnote{sokāvakiṇṇaṃ (syā.)} janatamapetasoko,\\
  Avekkhassu jātijarābhibhūtaṃ;
\end{verse}

\paragraph{150.}\begin{verse}
  Uṭṭhehi vīra vijitasaṅgāma,\\
  Satthavāha aṇaṇa vicara loke.\\
  Desassu\footnote{desetu (syā. pī.)} bhagavā dhammaṃ,\\
  Aññātāro bhavissantī’ti.
\end{verse}

\paragraph{151.} ‘‘Atha kho, bhikkhave, vipassī bhagavā arahaṃ sammāsambuddho taṃ mahābrahmānaṃ gāthāya ajjhabhāsi –

\paragraph{152.}\begin{verse}
  ‘Apārutā tesaṃ amatassa dvārā,\\
  Ye sotavanto pamuñcantu saddhaṃ;\\
  Vihiṃsasaññī paguṇaṃ na bhāsiṃ,\\
  Dhammaṃ paṇītaṃ manujesu brahme’ti.\\
\end{verse}

\paragraph{153.} ‘‘Atha kho so, bhikkhave, mahābrahmā ‘katāvakāso khomhi vipassinā bhagavatā arahatā sammāsambuddhena dhammadesanāyā’ti vipassiṃ bhagavantaṃ arahantaṃ sammāsambuddhaṃ abhivādetvā padakkhiṇaṃ katvā tattheva antaradhāyi.

\subsubsection{Aggasāvakayugaṃ}

\paragraph{154.} ‘‘Atha kho, bhikkhave, vipassissa bhagavato arahato sammāsambuddhassa etadahosi – ‘kassa nu kho ahaṃ paṭhamaṃ dhammaṃ deseyyaṃ, ko imaṃ dhammaṃ khippameva ājānissatī’ti? Atha kho, bhikkhave, vipassissa bhagavato arahato sammāsambuddhassa etadahosi – ‘ayaṃ kho khaṇḍo ca rājaputto tisso ca purohitaputto bandhumatiyā rājadhāniyā paṭivasanti paṇḍitā viyattā medhāvino dīgharattaṃ apparajakkhajātikā. Yaṃnūnāhaṃ khaṇḍassa ca rājaputtassa, tissassa ca purohitaputtassa paṭhamaṃ dhammaṃ deseyyaṃ , te imaṃ dhammaṃ khippameva ājānissantī’ti.

\paragraph{155.} ‘‘Atha kho, bhikkhave, vipassī bhagavā arahaṃ sammāsambuddho seyyathāpi nāma balavā puriso samiñjitaṃ vā bāhaṃ pasāreyya, pasāritaṃ vā bāhaṃ samiñjeyya; evameva bodhirukkhamūle antarahito bandhumatiyā rājadhāniyā kheme migadāye pāturahosi. Atha kho, bhikkhave, vipassī bhagavā arahaṃ sammāsambuddho dāyapālaṃ\footnote{migadāyapālaṃ (syā.)} āmantesi – ‘ehi tvaṃ, samma dāyapāla, bandhumatiṃ rājadhāniṃ pavisitvā khaṇḍañca rājaputtaṃ tissañca purohitaputtaṃ evaṃ vadehi – vipassī, bhante, bhagavā arahaṃ sammāsambuddho bandhumatiṃ rājadhāniṃ anuppatto kheme migadāye viharati, so tumhākaṃ dassanakāmo’ti. ‘Evaṃ, bhante’ti kho, bhikkhave, dāyapālo vipassissa bhagavato arahato sammāsambuddhassa paṭissutvā bandhumatiṃ rājadhāniṃ pavisitvā khaṇḍañca rājaputtaṃ tissañca purohitaputtaṃ etadavoca – ‘vipassī, bhante, bhagavā arahaṃ sammāsambuddho bandhumatiṃ rājadhāniṃ anuppatto kheme migadāye viharati; so tumhākaṃ dassanakāmo’ti.

\paragraph{156.} ‘‘Atha kho, bhikkhave, khaṇḍo ca rājaputto tisso ca purohitaputto bhaddāni bhaddāni yānāni yojāpetvā bhaddaṃ bhaddaṃ yānaṃ abhiruhitvā bhaddehi bhaddehi yānehi bandhumatiyā rājadhāniyā niyyiṃsu. Yena khemo migadāyo tena pāyiṃsu. Yāvatikā yānassa bhūmi, yānena gantvā yānā paccorohitvā pattikāva\footnote{padikāva (syā.)} yena vipassī bhagavā arahaṃ sammāsambuddho tenupasaṅkamiṃsu. Upasaṅkamitvā vipassiṃ bhagavantaṃ arahantaṃ sammāsambuddhaṃ abhivādetvā ekamantaṃ nisīdiṃsu.

\paragraph{157.} ‘‘Tesaṃ vipassī bhagavā arahaṃ sammāsambuddho anupubbiṃ kathaṃ\footnote{ānupubbikathaṃ (sī. pī.)} kathesi, seyyathidaṃ – dānakathaṃ sīlakathaṃ saggakathaṃ kāmānaṃ ādīnavaṃ okāraṃ saṃkilesaṃ nekkhamme ānisaṃsaṃ pakāsesi. Yadā te bhagavā aññāsi kallacitte muducitte vinīvaraṇacitte udaggacitte pasannacitte, atha yā buddhānaṃ sāmukkaṃsikā dhammadesanā, taṃ pakāsesi – dukkhaṃ samudayaṃ nirodhaṃ maggaṃ. Seyyathāpi nāma suddhaṃ vatthaṃ apagatakāḷakaṃ sammadeva rajanaṃ paṭiggaṇheyya, evameva khaṇḍassa ca rājaputtassa tissassa ca purohitaputtassa tasmiṃyeva āsane virajaṃ vītamalaṃ dhammacakkhuṃ udapādi – ‘yaṃ kiñci samudayadhammaṃ, sabbaṃ taṃ nirodhadhamma’nti.

\paragraph{158.} ‘‘Te diṭṭhadhammā pattadhammā viditadhammā pariyogāḷhadhammā tiṇṇavicikicchā vigatakathaṃkathā vesārajjappattā aparappaccayā satthusāsane vipassiṃ bhagavantaṃ arahantaṃ sammāsambuddhaṃ etadavocuṃ – ‘abhikkantaṃ, bhante, abhikkantaṃ, bhante. Seyyathāpi, bhante, nikkujjitaṃ vā ukkujjeyya, paṭicchannaṃ vā vivareyya, mūḷhassa vā maggaṃ ācikkheyya, andhakāre vā telapajjotaṃ dhāreyya ‘‘cakkhumanto rūpāni dakkhantī’’ti. Evamevaṃ bhagavatā anekapariyāyena dhammo pakāsito. Ete mayaṃ, bhante, bhagavantaṃ saraṇaṃ gacchāma dhammañca. Labheyyāma mayaṃ, bhante, bhagavato santike pabbajjaṃ, labheyyāma upasampada’nti.

\paragraph{159.} ‘‘Alatthuṃ kho , bhikkhave, khaṇḍo ca rājaputto, tisso ca purohitaputto vipassissa bhagavato arahato sammāsambuddhassa santike pabbajjaṃ alatthuṃ upasampadaṃ. Te vipassī bhagavā arahaṃ sammāsambuddho dhammiyā kathāya sandassesi samādapesi samuttejesi sampahaṃsesi; saṅkhārānaṃ ādīnavaṃ okāraṃ saṃkilesaṃ nibbāne\footnote{nekkhamme (syā.)} ānisaṃsaṃ pakāsesi. Tesaṃ vipassinā bhagavatā arahatā sammāsambuddhena dhammiyā kathāya sandassiyamānānaṃ samādapiyamānānaṃ samuttejiyamānānaṃ sampahaṃsiyamānānaṃ nacirasseva anupādāya āsavehi cittāni vimucciṃsu.

\subsubsection{Mahājanakāyapabbajjā}

\paragraph{160.} ‘‘Assosi kho, bhikkhave, bandhumatiyā rājadhāniyā mahājanakāyo caturāsītipāṇasahassāni – ‘vipassī kira bhagavā arahaṃ sammāsambuddho bandhumatiṃ rājadhāniṃ anuppatto kheme migadāye viharati. Khaṇḍo ca kira rājaputto tisso ca purohitaputto vipassissa bhagavato arahato sammāsambuddhassa santike kesamassuṃ ohāretvā kāsāyāni vatthāni acchādetvā agārasmā anagāriyaṃ pabbajitā’ti. Sutvāna nesaṃ etadahosi – ‘na hi nūna so orako dhammavinayo, na sā orakā pabbajjā, yattha khaṇḍo ca rājaputto tisso ca purohitaputto kesamassuṃ ohāretvā kāsāyāni vatthāni acchādetvā agārasmā anagāriyaṃ pabbajitā. Khaṇḍo ca rājaputto tisso ca purohitaputto kesamassuṃ ohāretvā kāsāyāni vatthāni acchādetvā agārasmā anagāriyaṃ pabbajissanti, kimaṅgaṃ pana maya’nti. Atha kho so, bhikkhave, mahājanakāyo caturāsītipāṇasahassāni bandhumatiyā rājadhāniyā nikkhamitvā yena khemo migadāyo yena vipassī bhagavā arahaṃ sammāsambuddho tenupasaṅkamiṃsu; upasaṅkamitvā vipassiṃ bhagavantaṃ arahantaṃ sammāsambuddhaṃ abhivādetvā ekamantaṃ nisīdiṃsu.

\paragraph{161.} ‘‘Tesaṃ vipassī bhagavā arahaṃ sammāsambuddho anupubbiṃ kathaṃ kathesi. Seyyathidaṃ – dānakathaṃ sīlakathaṃ saggakathaṃ kāmānaṃ ādīnavaṃ okāraṃ saṃkilesaṃ nekkhamme ānisaṃsaṃ pakāsesi. Yadā te bhagavā aññāsi kallacitte muducitte vinīvaraṇacitte udaggacitte pasannacitte , atha yā buddhānaṃ sāmukkaṃsikā dhammadesanā, taṃ pakāsesi – dukkhaṃ samudayaṃ nirodhaṃ maggaṃ. Seyyathāpi nāma suddhaṃ vatthaṃ apagatakāḷakaṃ sammadeva rajanaṃ paṭiggaṇheyya, evameva tesaṃ caturāsītipāṇasahassānaṃ tasmiṃyeva āsane virajaṃ vītamalaṃ dhammacakkhuṃ udapādi – ‘yaṃ kiñci samudayadhammaṃ sabbaṃ taṃ nirodhadhamma’nti.

\paragraph{162.} ‘‘Te diṭṭhadhammā pattadhammā viditadhammā pariyogāḷhadhammā tiṇṇavicikicchā vigatakathaṃkathā vesārajjappattā aparappaccayā satthusāsane vipassiṃ bhagavantaṃ arahantaṃ sammāsambuddhaṃ etadavocuṃ – ‘abhikkantaṃ, bhante, abhikkantaṃ, bhante. Seyyathāpi, bhante, nikkujjitaṃ vā ukkujjeyya, paṭicchannaṃ vā vivareyya, mūḷhassa vā maggaṃ ācikkheyya, andhakāre vā telapajjotaṃ dhāreyya ‘‘cakkhumanto rūpāni dakkhantī’’ti. Evamevaṃ bhagavatā anekapariyāyena dhammo pakāsito. Ete mayaṃ, bhante, bhagavantaṃ saraṇaṃ gacchāma dhammañca bhikkhusaṅghañca\footnote{( ) natthi aṭṭhakathāyaṃ, pāḷiyaṃ pana sabbatthapi dissati}. Labheyyāma mayaṃ, bhante, bhagavato santike pabbajjaṃ labheyyāma upasampada’’nti.

\paragraph{163.} ‘‘Alatthuṃ kho, bhikkhave, tāni caturāsītipāṇasahassāni vipassissa bhagavato arahato sammāsambuddhassa santike pabbajjaṃ, alatthuṃ upasampadaṃ. Te vipassī bhagavā arahaṃ sammāsambuddho dhammiyā kathāya sandassesi samādapesi samuttejesi sampahaṃsesi; saṅkhārānaṃ ādīnavaṃ okāraṃ saṃkilesaṃ nibbāne ānisaṃsaṃ pakāsesi. Tesaṃ vipassinā bhagavatā arahatā sammāsambuddhena dhammiyā kathāya sandassiyamānānaṃ samādapiyamānānaṃ samuttejiyamānānaṃ sampahaṃsiyamānānaṃ nacirasseva anupādāya āsavehi cittāni vimucciṃsu.

\subsubsection{Purimapabbajitānaṃ dhammābhisamayo}

\paragraph{164.} ‘‘Assosuṃ kho, bhikkhave, tāni purimāni caturāsītipabbajitasahassāni – ‘vipassī kira bhagavā arahaṃ sammāsambuddho bandhumatiṃ rājadhāniṃ anuppatto kheme migadāye viharati, dhammañca kira desetī’ti. Atha kho, bhikkhave, tāni caturāsītipabbajitasahassāni yena bandhumatī rājadhānī yena khemo migadāyo yena vipassī bhagavā arahaṃ sammāsambuddho tenupasaṅkamiṃsu; upasaṅkamitvā vipassiṃ bhagavantaṃ arahantaṃ sammāsambuddhaṃ abhivādetvā ekamantaṃ nisīdiṃsu.

\paragraph{165.} ‘‘Tesaṃ vipassī bhagavā arahaṃ sammāsambuddho anupubbiṃ kathaṃ kathesi. Seyyathidaṃ – dānakathaṃ sīlakathaṃ saggakathaṃ kāmānaṃ ādīnavaṃ okāraṃ saṃkilesaṃ nekkhamme ānisaṃsaṃ pakāsesi. Yadā te bhagavā aññāsi kallacitte muducitte vinīvaraṇacitte udaggacitte pasannacitte, atha yā buddhānaṃ sāmukkaṃsikā dhammadesanā, taṃ pakāsesi – dukkhaṃ samudayaṃ nirodhaṃ maggaṃ. Seyyathāpi nāma suddhaṃ vatthaṃ apagatakāḷakaṃ sammadeva rajanaṃ paṭiggaṇheyya, evameva tesaṃ caturāsītipabbajitasahassānaṃ tasmiṃyeva āsane virajaṃ vītamalaṃ dhammacakkhuṃ udapādi – ‘yaṃ kiñci samudayadhammaṃ sabbaṃ taṃ nirodhadhamma’nti.

\paragraph{166.} ‘‘Te diṭṭhadhammā pattadhammā viditadhammā pariyogāḷhadhammā tiṇṇavicikicchā vigatakathaṃkathā vesārajjappattā aparappaccayā satthusāsane vipassiṃ bhagavantaṃ arahantaṃ sammāsambuddhaṃ etadavocuṃ – ‘abhikkantaṃ , bhante, abhikkantaṃ, bhante. Seyyathāpi, bhante, nikkujjitaṃ vā ukkujjeyya, paṭicchannaṃ vā vivareyya, mūḷhassa vā maggaṃ ācikkheyya, andhakāre vā telapajjotaṃ dhāreyya ‘‘cakkhumanto rūpāni dakkhantī’’ti. Evamevaṃ bhagavatā anekapariyāyena dhammo pakāsito. Ete mayaṃ, bhante, bhagavantaṃ saraṇaṃ gacchāma dhammañca bhikkhusaṅghañca. Labheyyāma mayaṃ, bhante, bhagavato santike pabbajjaṃ labheyyāma upasampada’’nti.

\paragraph{167.} ‘‘Alatthuṃ kho, bhikkhave, tāni caturāsītipabbajitasahassāni vipassissa bhagavato arahato sammāsambuddhassa santike pabbajjaṃ alatthuṃ upasampadaṃ. Te vipassī bhagavā arahaṃ sammāsambuddho dhammiyā kathāya sandassesi samādapesi samuttejesi sampahaṃsesi; saṅkhārānaṃ ādīnavaṃ okāraṃ saṃkilesaṃ nibbāne ānisaṃsaṃ pakāsesi. Tesaṃ vipassinā bhagavatā arahatā sammāsambuddhena dhammiyā kathāya sandassiyamānānaṃ samādapiyamānānaṃ samuttejiyamānānaṃ sampahaṃsiyamānānaṃ nacirasseva anupādāya āsavehi cittāni vimucciṃsu.

\subsubsection{Cārikāanujānanaṃ}

\paragraph{168.} ‘‘Tena kho pana, bhikkhave, samayena bandhumatiyā rājadhāniyā mahābhikkhusaṅgho paṭivasati aṭṭhasaṭṭhibhikkhusatasahassaṃ. Atha kho, bhikkhave, vipassissa bhagavato arahato sammāsambuddhassa rahogatassa paṭisallīnassa evaṃ cetaso parivitakko udapādi – ‘mahā kho etarahi bhikkhusaṅgho bandhumatiyā rājadhāniyā paṭivasati aṭṭhasaṭṭhibhikkhusatasahassaṃ, yaṃnūnāhaṃ bhikkhū anujāneyyaṃ – ‘caratha, bhikkhave, cārikaṃ bahujanahitāya bahujanasukhāya lokānukampāya atthāya hitāya sukhāya devamanussānaṃ; mā ekena dve agamittha; desetha, bhikkhave , dhammaṃ ādikalyāṇaṃ majjhekalyāṇaṃ pariyosānakalyāṇaṃ sātthaṃ sabyañjanaṃ kevalaparipuṇṇaṃ parisuddhaṃ brahmacariyaṃ pakāsetha. Santi sattā apparajakkhajātikā, assavanatā dhammassa parihāyanti, bhavissanti dhammassa aññātāro. Api ca channaṃ channaṃ vassānaṃ accayena bandhumatī rājadhānī upasaṅkamitabbā pātimokkhuddesāyā’’’ti.

\paragraph{169.} ‘‘Atha kho, bhikkhave, aññataro mahābrahmā vipassissa bhagavato arahato sammāsambuddhassa cetasā cetoparivitakkamaññāya seyyathāpi nāma balavā puriso samiñjitaṃ vā bāhaṃ pasāreyya, pasāritaṃ vā bāhaṃ samiñjeyya. Evameva brahmaloke antarahito vipassissa bhagavato arahato sammāsambuddhassa purato pāturahosi. Atha kho so, bhikkhave, mahābrahmā ekaṃsaṃ uttarāsaṅgaṃ karitvā yena vipassī bhagavā arahaṃ sammāsambuddho tenañjaliṃ paṇāmetvā vipassiṃ bhagavantaṃ arahantaṃ sammāsambuddhaṃ etadavoca – ‘evametaṃ, bhagavā, evametaṃ, sugata. Mahā kho, bhante, etarahi bhikkhusaṅgho bandhumatiyā rājadhāniyā paṭivasati aṭṭhasaṭṭhibhikkhusatasahassaṃ, anujānātu, bhante, bhagavā bhikkhū – ‘‘caratha, bhikkhave, cārikaṃ bahujanahitāya bahujanasukhāya lokānukampāya atthāya hitāya sukhāya devamanussānaṃ; mā ekena dve agamittha; desetha, bhikkhave, dhammaṃ ādikalyāṇaṃ majjhekalyāṇaṃ pariyosānakalyāṇaṃ sātthaṃ sabyañjanaṃ kevalaparipuṇṇaṃ parisuddhaṃ brahmacariyaṃ pakāsetha. Santi sattā apparajakkhajātikā, assavanatā dhammassa parihāyanti, bhavissanti dhammassa aññātāro’’ti\footnote{aññātāro (ssabbattha)}. Api ca, bhante, mayaṃ tathā karissāma yathā bhikkhū channaṃ channaṃ vassānaṃ accayena bandhumatiṃ rājadhāniṃ upasaṅkamissanti pātimokkhuddesāyā’ti. Idamavoca, bhikkhave, so mahābrahmā, idaṃ vatvā vipassiṃ bhagavantaṃ arahantaṃ sammāsambuddhaṃ abhivādetvā padakkhiṇaṃ katvā tattheva antaradhāyi.

\paragraph{170.} ‘‘Atha kho, bhikkhave, vipassī bhagavā arahaṃ sammāsambuddho sāyanhasamayaṃ paṭisallānā vuṭṭhito bhikkhū āmantesi – ‘idha mayhaṃ, bhikkhave, rahogatassa paṭisallīnassa evaṃ cetaso parivitakko udapādi – mahā kho etarahi bhikkhusaṅgho bandhumatiyā rājadhāniyā paṭivasati aṭṭhasaṭṭhibhikkhusatasahassaṃ . Yaṃnūnāhaṃ bhikkhū anujāneyyaṃ – ‘caratha, bhikkhave, cārikaṃ bahujanahitāya bahujanasukhāya lokānukampāya atthāya hitāya sukhāya devamanussānaṃ; mā ekena dve agamittha; desetha, bhikkhave, dhammaṃ ādikalyāṇaṃ majjhekalyāṇaṃ pariyosānakalyāṇaṃ sātthaṃ sabyañjanaṃ kevalaparipuṇṇaṃ parisuddhaṃ brahmacariyaṃ pakāsetha. Santi sattā apparajakkhajātikā, assavanatā dhammassa parihāyanti, bhavissanti dhammassa aññātāro. Api ca, channaṃ channaṃ vassānaṃ accayena bandhumatī rājadhānī upasaṅkamitabbā pātimokkhuddesāyāti.

\paragraph{171.} ‘‘‘Atha kho, bhikkhave, aññataro mahābrahmā mama cetasā cetoparivitakkamaññāya seyyathāpi nāma balavā puriso samiñjitaṃ vā bāhaṃ pasāreyya, pasāritaṃ vā bāhaṃ samiñjeyya, evameva brahmaloke antarahito mama purato pāturahosi. Atha kho so, bhikkhave, mahābrahmā ekaṃsaṃ uttarāsaṅgaṃ karitvā yenāhaṃ tenañjaliṃ paṇāmetvā maṃ etadavoca – ‘‘evametaṃ, bhagavā, evametaṃ, sugata. Mahā kho, bhante, etarahi bhikkhusaṅgho bandhumatiyā rājadhāniyā paṭivasati aṭṭhasaṭṭhibhikkhusatasahassaṃ. Anujānātu, bhante, bhagavā bhikkhū – ‘caratha, bhikkhave, cārikaṃ bahujanahitāya bahujanasukhāya lokānukampāya atthāya hitāya sukhāya devamanussānaṃ; mā ekena dve agamittha; desetha, bhikkhave, dhammaṃ…pe… santi sattā apparajakkhajātikā , assavanatā dhammassa parihāyanti, bhavissanti dhammassa aññātāro’ti. Api ca, bhante, mayaṃ tathā karissāma, yathā bhikkhū channaṃ channaṃ vassānaṃ accayena bandhumatiṃ rājadhāniṃ upasaṅkamissanti pātimokkhuddesāyā’’ti. Idamavoca, bhikkhave, so mahābrahmā, idaṃ vatvā maṃ abhivādetvā padakkhiṇaṃ katvā tattheva antaradhāyi’.

\paragraph{172.} ‘‘‘Anujānāmi, bhikkhave, caratha cārikaṃ bahujanahitāya bahujanasukhāya lokānukampāya atthāya hitāya sukhāya devamanussānaṃ; mā ekena dve agamittha; desetha, bhikkhave, dhammaṃ ādikalyāṇaṃ majjhekalyāṇaṃ pariyosānakalyāṇaṃ sātthaṃ sabyañjanaṃ kevalaparipuṇṇaṃ parisuddhaṃ brahmacariyaṃ pakāsetha. Santi sattā apparajakkhajātikā, assavanatā dhammassa parihāyanti, bhavissanti dhammassa aññātāro. Api ca, bhikkhave, channaṃ channaṃ vassānaṃ accayena bandhumatī rājadhānī upasaṅkamitabbā pātimokkhuddesāyā’ti. Atha kho, bhikkhave, bhikkhū yebhuyyena ekāheneva janapadacārikaṃ pakkamiṃsu.

\paragraph{173.} ‘‘Tena kho pana samayena jambudīpe caturāsīti āvāsasahassāni honti. Ekamhi hi vasse nikkhante devatā saddamanussāvesuṃ – ‘nikkhantaṃ kho, mārisā, ekaṃ vassaṃ; pañca dāni vassāni sesāni ; pañcannaṃ vassānaṃ accayena bandhumatī rājadhānī upasaṅkamitabbā pātimokkhuddesāyā’ti. Dvīsu vassesu nikkhantesu… tīsu vassesu nikkhantesu… catūsu vassesu nikkhantesu… pañcasu vassesu nikkhantesu devatā saddamanussāvesuṃ – ‘nikkhantāni kho , mārisā, pañcavassāni; ekaṃ dāni vassaṃ sesaṃ; ekassa vassassa accayena bandhumatī rājadhānī upasaṅkamitabbā pātimokkhuddesāyā’ti. Chasu vassesu nikkhantesu devatā saddamanussāvesuṃ – ‘nikkhantāni kho, mārisā, chabbassāni, samayo dāni bandhumatiṃ rājadhāniṃ upasaṅkamituṃ pātimokkhuddesāyā’ti. Atha kho te, bhikkhave, bhikkhū appekacce sakena iddhānubhāvena appekacce devatānaṃ iddhānubhāvena ekāheneva bandhumatiṃ rājadhāniṃ upasaṅkamiṃsu pātimokkhuddesāyāti\footnote{pātimokkhuddesāya (?)}.

\paragraph{174.} ‘‘Tatra sudaṃ, bhikkhave, vipassī bhagavā arahaṃ sammāsambuddho bhikkhusaṅghe evaṃ pātimokkhaṃ uddisati –

\paragraph{175.}\begin{verse}
  ‘Khantī paramaṃ tapo titikkhā,\\
  Nibbānaṃ paramaṃ vadanti buddhā;\\
  Na hi pabbajito parūpaghātī,\\
  Na samaṇo\footnote{samaṇo (sī. syā. pī.)} hoti paraṃ viheṭhayanto.\\
\end{verse}

\paragraph{176.}\begin{verse}
  ‘Sabbapāpassa akaraṇaṃ, \\kusalassa upasampadā;\\
  Sacittapariyodapanaṃ, \\etaṃ buddhānasāsanaṃ.
\end{verse}

\paragraph{177.}\begin{verse}
  ‘Anūpavādo anūpaghāto\footnote{anupavādo anupaghāto (pī. ka.)}, \\pātimokkhe ca saṃvaro;\\
  Mattaññutā ca bhattasmiṃ, \\pantañca sayanāsanaṃ;\\
  Adhicitte ca āyogo, \\etaṃ buddhānasāsana’nti.
\end{verse}

\xsubsubsectionEnd{Devatārocanaṃ}

\paragraph{178.} ‘‘Ekamidāhaṃ, bhikkhave, samayaṃ ukkaṭṭhāyaṃ viharāmi subhagavane sālarājamūle. Tassa mayhaṃ, bhikkhave, rahogatassa paṭisallīnassa evaṃ cetaso parivitakko udapādi – ‘na kho so sattāvāso sulabharūpo, yo mayā anāvutthapubbo\footnote{anajjhāvuṭṭhapubbo (ka. sī. ka.)} iminā dīghena addhunā aññatra suddhāvāsehi devehi. Yaṃnūnāhaṃ yena suddhāvāsā devā tenupasaṅkameyya’nti. Atha khvāhaṃ, bhikkhave, seyyathāpi nāma balavā puriso samiñjitaṃ vā bāhaṃ pasāreyya, pasāritaṃ vā bāhaṃ samiñjeyya, evameva ukkaṭṭhāyaṃ subhagavane sālarājamūle antarahito avihesu devesu pāturahosiṃ . Tasmiṃ, bhikkhave, devanikāye anekāni devatāsahassāni anekāni devatāsatasahassāni\footnote{anekāni devatāsatāni anekāni devatāsahassāni (syā.)} yenāhaṃ tenupasaṅkamiṃsu; upasaṅkamitvā maṃ abhivādetvā ekamantaṃ aṭṭhaṃsu. Ekamantaṃ ṭhitā kho, bhikkhave, tā devatā maṃ etadavocuṃ – ‘ito so, mārisā, ekanavutikappe yaṃ vipassī bhagavā arahaṃ sammāsambuddho loke udapādi. Vipassī, mārisā, bhagavā arahaṃ sammāsambuddho khattiyo jātiyā ahosi, khattiyakule udapādi. Vipassī, mārisā, bhagavā arahaṃ sammāsambuddho koṇḍañño gottena ahosi . Vipassissa, mārisā, bhagavato arahato sammāsambuddhassa asītivassasahassāni āyuppamāṇaṃ ahosi. Vipassī, mārisā, bhagavā arahaṃ sammāsambuddho pāṭaliyā mūle abhisambuddho. Vipassissa, mārisā, bhagavato arahato sammāsambuddhassa khaṇḍatissaṃ nāma sāvakayugaṃ ahosi aggaṃ bhaddayugaṃ. Vipassissa, mārisā, bhagavato arahato sammāsambuddhassa tayo sāvakānaṃ sannipātā ahesuṃ. Eko sāvakānaṃ sannipāto ahosi aṭṭhasaṭṭhibhikkhusatasahassaṃ. Eko sāvakānaṃ sannipāto ahosi bhikkhusatasahassaṃ. Eko sāvakānaṃ sannipāto ahosi asītibhikkhusahassāni. Vipassissa, mārisā, bhagavato arahato sammāsambuddhassa ime tayo sāvakānaṃ sannipātā ahesuṃ sabbesaṃyeva khīṇāsavānaṃ. Vipassissa, mārisā, bhagavato arahato sammāsambuddhassa asoko nāma bhikkhu upaṭṭhāko ahosi aggupaṭṭhāko. Vipassissa, mārisa, bhagavato arahato sammāsambuddhassa bandhumā nāma rājā pitā ahosi. Bandhumatī nāma devī mātā ahosi janetti. Bandhumassa rañño bandhumatī nāma nagaraṃ rājadhānī ahosi. Vipassissa, mārisā , bhagavato arahato sammāsambuddhassa evaṃ abhinikkhamanaṃ ahosi evaṃ pabbajjā evaṃ padhānaṃ evaṃ abhisambodhi evaṃ dhammacakkappavattanaṃ. Te mayaṃ, mārisā, vipassimhi bhagavati brahmacariyaṃ caritvā kāmesu kāmacchandaṃ virājetvā idhūpapannā’ti …pe…

\paragraph{179.} ‘‘Tasmiṃyeva kho, bhikkhave, devanikāye anekāni devatāsahassāni anekāni devatāsatasahassāni\footnote{anekāni devatāsatāni anekāni devatāsahassāni (syā. evamuparipi)} yenāhaṃ tenupasaṅkamiṃsu; upasaṅkamitvā maṃ abhivādetvā ekamantaṃ aṭṭhaṃsu. Ekamantaṃ ṭhitā kho, bhikkhave, tā devatā maṃ etadavocuṃ – ‘imasmiṃyeva kho, mārisā, bhaddakappe bhagavā etarahi arahaṃ sammāsambuddho loke uppanno. Bhagavā, mārisā, khattiyo jātiyā khattiyakule uppanno. Bhagavā, mārisā, gotamo gottena. Bhagavato, mārisā, appakaṃ āyuppamāṇaṃ parittaṃ lahukaṃ yo ciraṃ jīvati, so vassasataṃ appaṃ vā bhiyyo. Bhagavā, mārisā, assatthassa mūle abhisambuddho. Bhagavato, mārisā, sāriputtamoggallānaṃ nāma sāvakayugaṃ ahosi aggaṃ bhaddayugaṃ . Bhagavato, mārisā, eko sāvakānaṃ sannipāto ahosi aḍḍhateḷasāni bhikkhusatāni. Bhagavato, mārisā, ayaṃ eko sāvakānaṃ sannipāto ahosi sabbesaṃyeva khīṇāsavānaṃ. Bhagavato, mārisā, ānando nāma bhikkhu upaṭṭhāko ahosi aggupaṭṭhāko. Bhagavato, mārisā, suddhodano nāma rājā pitā ahosi. Māyā nāma devī mātā ahosi janetti. Kapilavatthu nāma nagaraṃ rājadhānī ahosi. Bhagavato, mārisā, evaṃ abhinikkhamanaṃ ahosi evaṃ pabbajjā evaṃ padhānaṃ evaṃ abhisambodhi evaṃ dhammacakkappavattanaṃ. Te mayaṃ, mārisā, bhagavati brahmacariyaṃ caritvā kāmesu kāmacchandaṃ virājetvā idhūpapannā’ti.

\paragraph{180.} ‘‘Atha khvāhaṃ, bhikkhave, avihehi devehi saddhiṃ yena atappā devā tenupasaṅkamiṃ…pe… atha khvāhaṃ, bhikkhave, avihehi ca devehi atappehi ca devehi saddhiṃ yena sudassā devā tenupasaṅkamiṃ. Atha khvāhaṃ, bhikkhave, avihehi ca devehi atappehi ca devehi sudassehi ca devehi saddhiṃ yena sudassī devā tenupasaṅkamiṃ. Atha khvāhaṃ, bhikkhave, avihehi ca devehi atappehi ca devehi sudassehi ca devehi sudassīhi ca devehi saddhiṃ yena akaniṭṭhā devā tenupasaṅkamiṃ. Tasmiṃ, bhikkhave, devanikāye anekāni devatāsahassāni anekāni devatāsatasahassāni yenāhaṃ tenupasaṅkamiṃsu, upasaṅkamitvā maṃ abhivādetvā ekamantaṃ aṭṭhaṃsu .

\paragraph{181.} ‘‘Ekamantaṃ ṭhitā kho, bhikkhave, tā devatā maṃ etadavocuṃ – ‘ito so, mārisā, ekanavutikappe yaṃ vipassī bhagavā arahaṃ sammāsambuddho loke udapādi. Vipassī, mārisā, bhagavā arahaṃ sammāsambuddho khattiyo jātiyā ahosi. Khattiyakule udapādi. Vipassī, mārisā, bhagavā arahaṃ sammāsambuddho koṇḍañño gottena ahosi. Vipassissa, mārisā, bhagavato arahato sammāsambuddhassa asītivassasahassāni āyuppamāṇaṃ ahosi. Vipassī, mārisā, bhagavā arahaṃ sammāsambuddho pāṭaliyā mūle abhisambuddho. Vipassissa, mārisā, bhagavato arahato sammāsambuddhassa khaṇḍatissaṃ nāma sāvakayugaṃ ahosi aggaṃ bhaddayugaṃ. Vipassissa, mārisā, bhagavato arahato sammāsambuddhassa tayo sāvakānaṃ sannipātā ahesuṃ. Eko sāvakānaṃ sannipāto ahosi aṭṭhasaṭṭhibhikkhusatasahassaṃ. Eko sāvakānaṃ sannipāto ahosi bhikkhusatasahassaṃ. Eko sāvakānaṃ sannipāto ahosi asītibhikkhusahassāni. Vipassissa, mārisā, bhagavato arahato sammāsambuddhassa ime tayo sāvakānaṃ sannipātā ahesuṃ sabbesaṃyeva khīṇāsavānaṃ. Vipassissa, mārisā, bhagavato arahato sammāsambuddhassa asoko nāma bhikkhu upaṭṭhāko ahosi aggupaṭṭhāko. Vipassissa, mārisā, bhagavato arahato sammāsambuddhassa bandhumā nāma rājā pitā ahosi bandhumatī nāma devī mātā ahosi janetti. Bandhumassa rañño bandhumatī nāma nagaraṃ rājadhānī ahosi. Vipassissa, mārisā, bhagavato arahato sammāsambuddhassa evaṃ abhinikkhamanaṃ ahosi evaṃ pabbajjā evaṃ padhānaṃ evaṃ abhisambodhi, evaṃ dhammacakkappavattanaṃ. Te mayaṃ, mārisā, vipassimhi bhagavati brahmacariyaṃ caritvā kāmesu kāmacchandaṃ virājetvā idhūpapannā’ti. Tasmiṃyeva kho, bhikkhave, devanikāye anekāni devatāsahassāni anekāni devatāsatasahassāni yenāhaṃ tenupasaṅkamiṃsu; upasaṅkamitvā maṃ abhivādetvā ekamantaṃ aṭṭhaṃsu. Ekamantaṃ ṭhitā kho, bhikkhave, tā devatā maṃ etadavocuṃ – ‘ito so, mārisā, ekatiṃse kappe yaṃ sikhī bhagavā…pe… te mayaṃ, mārisā, sikhimhi bhagavati tasmiññeva kho mārisā, ekatiṃse kappe yaṃ vessabhū bhagavā…pe… te mayaṃ, mārisā, vessabhumhi bhagavati…pe… imasmiṃyeva kho, mārisā, bhaddakappe kakusandho koṇāgamano kassapo bhagavā…pe… te mayaṃ, mārisā, kakusandhamhi koṇāgamanamhi kassapamhi bhagavati brahmacariyaṃ caritvā kāmesu kāmacchandaṃ virājetvā idhūpapannā’ti.

\paragraph{182.} ‘‘Tasmiṃyeva kho, bhikkhave, devanikāye anekāni devatāsahassāni anekāni devatāsatasahassāni yenāhaṃ tenupasaṅkamiṃsu; upasaṅkamitvā maṃ abhivādetvā ekamantaṃ aṭṭhaṃsu. Ekamantaṃ ṭhitā kho, bhikkhave, tā devatā maṃ etadavocuṃ – ‘imasmiṃyeva kho, mārisā, bhaddakappe bhagavā etarahi arahaṃ sammāsambuddho loke uppanno. Bhagavā, mārisā, khattiyo jātiyā, khattiyakule uppanno. Bhagavā, mārisā, gotamo gottena. Bhagavato, mārisā, appakaṃ āyuppamāṇaṃ parittaṃ lahukaṃ yo ciraṃ jīvati, so vassasataṃ appaṃ vā bhiyyo. Bhagavā, mārisā, assatthassa mūle abhisambuddho. Bhagavato, mārisā, sāriputtamoggallānaṃ nāma sāvakayugaṃ ahosi aggaṃ bhaddayugaṃ. Bhagavato , mārisā, eko sāvakānaṃ sannipāto ahosi aḍḍhateḷasāni bhikkhusatāni. Bhagavato, mārisā, ayaṃ eko sāvakānaṃ sannipāto ahosi sabbesaṃyeva khīṇāsavānaṃ. Bhagavato, mārisā, ānando nāma bhikkhu upaṭṭhāko aggupaṭṭhāko ahosi. Bhagavato, mārisā, suddhodano nāma rājā pitā ahosi. Māyā nāma devī mātā ahosi janetti. Kapilavatthu nāma nagaraṃ rājadhānī ahosi. Bhagavato, mārisā, evaṃ abhinikkhamanaṃ ahosi, evaṃ pabbajjā, evaṃ padhānaṃ, evaṃ abhisambodhi, evaṃ dhammacakkappavattanaṃ. Te mayaṃ, mārisā, bhagavati brahmacariyaṃ caritvā kāmesu kāmacchandaṃ virājetvā idhūpapannā’ti.

\paragraph{183.} ‘‘Iti kho, bhikkhave, tathāgatassevesā dhammadhātu suppaṭividdhā, yassā dhammadhātuyā suppaṭividdhattā tathāgato atīte buddhe parinibbute chinnapapañce chinnavaṭume pariyādinnavaṭṭe sabbadukkhavītivatte jātitopi anussarati, nāmatopi anussarati, gottatopi anussarati, āyuppamāṇatopi anussarati, sāvakayugatopi anussarati, sāvakasannipātatopi anussarati ‘evaṃjaccā te bhagavanto ahesuṃ’ itipi. ‘Evaṃnāmā evaṃgottā evaṃsīlā evaṃdhammā evaṃpaññā evaṃvihārī evaṃvimuttā te bhagavanto ahesuṃ’ itipīti.

\paragraph{184.} ‘‘Devatāpi tathāgatassa etamatthaṃ ārocesuṃ, yena tathāgato atīte buddhe parinibbute chinnapapañce chinnavaṭume pariyādinnavaṭṭe sabbadukkhavītivatte jātitopi anussarati, nāmatopi anussarati, gottatopi anussarati, āyuppamāṇatopi anussarati, sāvakayugatopi anussarati, sāvakasannipātatopi anussarati ‘evaṃjaccā te bhagavanto ahesuṃ’ itipi. ‘Evaṃnāmā evaṃgottā evaṃsīlā evaṃdhammā evaṃpaññā evaṃvihārī evaṃvimuttā te bhagavanto ahesuṃ’ itipī’’ti.

\paragraph{185.} Idamavoca bhagavā. Attamanā te bhikkhū bhagavato bhāsitaṃ abhinandunti.

\xsectionEnd{Mahāpadānasuttaṃ niṭṭhitaṃ paṭhamaṃ.}


\clearpage
\section{Mahānidānasuttaṃ}

\subsubsection{Paṭiccasamuppādo}

\paragraph{95.} Evaṃ me sutaṃ – ekaṃ samayaṃ bhagavā kurūsu viharati kammāsadhammaṃ nāma\footnote{kammāsadammaṃ nāma (syā.)} kurūnaṃ nigamo. Atha kho āyasmā ānando yena bhagavā tenupasaṅkami, upasaṅkamitvā bhagavantaṃ abhivādetvā ekamantaṃ nisīdi. Ekamantaṃ nisinno kho āyasmā ānando bhagavantaṃ etadavoca – ‘‘acchariyaṃ, bhante, abbhutaṃ, bhante! Yāva gambhīro cāyaṃ, bhante, paṭiccasamuppādo gambhīrāvabhāso ca, atha ca pana me uttānakuttānako viya khāyatī’’ti. ‘‘Mā hevaṃ, ānanda, avaca, mā hevaṃ, ānanda, avaca. Gambhīro cāyaṃ, ānanda, paṭiccasamuppādo gambhīrāvabhāso ca. Etassa, ānanda, dhammassa ananubodhā appaṭivedhā evamayaṃ pajā tantākulakajātā kulagaṇṭhikajātā\footnote{gulāguṇṭhikajātā (sī. pī.), guṇagaṇṭhikajātā (syā.)} muñjapabbajabhūtā apāyaṃ duggatiṃ vinipātaṃ saṃsāraṃ nātivattati.

\paragraph{96.} ‘‘‘Atthi idappaccayā jarāmaraṇa’nti iti puṭṭhena satā, ānanda, atthītissa vacanīyaṃ. ‘Kiṃpaccayā jarāmaraṇa’nti iti ce vadeyya, ‘jātipaccayā jarāmaraṇa’nti iccassa vacanīyaṃ.

‘‘‘Atthi idappaccayā jātī’ti iti puṭṭhena satā, ānanda, atthītissa vacanīyaṃ. ‘Kiṃpaccayā jātī’ti iti ce vadeyya, ‘bhavapaccayā jātī’ti iccassa vacanīyaṃ.

‘‘‘Atthi idappaccayā bhavo’ti iti puṭṭhena satā, ānanda, atthītissa vacanīyaṃ . ‘Kiṃpaccayā bhavo’ti iti ce vadeyya, ‘upādānapaccayā bhavo’ti iccassa vacanīyaṃ.

‘‘‘Atthi idappaccayā upādāna’nti iti puṭṭhena satā, ānanda, atthītissa vacanīyaṃ. ‘Kiṃpaccayā upādāna’nti iti ce vadeyya, ‘taṇhāpaccayā upādāna’nti iccassa vacanīyaṃ.

‘‘‘Atthi idappaccayā taṇhā’ti iti puṭṭhena satā, ānanda, atthītissa vacanīyaṃ. ‘Kiṃpaccayā taṇhā’ti iti ce vadeyya, ‘vedanāpaccayā taṇhā’ti iccassa vacanīyaṃ.

‘‘‘Atthi idappaccayā vedanā’ti iti puṭṭhena satā, ānanda, atthītissa vacanīyaṃ. ‘Kiṃpaccayā vedanā’ti iti ce vadeyya, ‘phassapaccayā vedanā’ti iccassa vacanīyaṃ.

‘‘‘Atthi idappaccayā phasso’ti iti puṭṭhena satā, ānanda, atthītissa vacanīyaṃ. ‘Kiṃpaccayā phasso’ti iti ce vadeyya, ‘nāmarūpapaccayā phasso’ti iccassa vacanīyaṃ.

‘‘‘Atthi idappaccayā nāmarūpa’nti iti puṭṭhena satā, ānanda, atthītissa vacanīyaṃ. ‘Kiṃpaccayā nāmarūpa’nti iti ce vadeyya, ‘viññāṇapaccayā nāmarūpa’nti iccassa vacanīyaṃ.

‘‘‘Atthi idappaccayā viññāṇa’nti iti puṭṭhena satā, ānanda, atthītissa vacanīyaṃ. ‘Kiṃpaccayā viññāṇa’nti iti ce vadeyya, ‘nāmarūpapaccayā viññāṇa’nti iccassa vacanīyaṃ.

\paragraph{97.} ‘‘Iti kho, ānanda, nāmarūpapaccayā viññāṇaṃ, viññāṇapaccayā nāmarūpaṃ, nāmarūpapaccayā phasso, phassapaccayā vedanā, vedanāpaccayā taṇhā, taṇhāpaccayā upādānaṃ, upādānapaccayā bhavo, bhavapaccayā jāti , jātipaccayā jarāmaraṇaṃ sokaparidevadukkhadomanassupāyāsā sambhavanti. Evametassa kevalassa dukkhakkhandhassa samudayo hoti.

\paragraph{98.} ‘‘‘Jātipaccayā jarāmaraṇa’nti iti kho panetaṃ vuttaṃ, tadānanda, imināpetaṃ pariyāyena veditabbaṃ, yathā jātipaccayā jarāmaraṇaṃ. Jāti ca hi, ānanda, nābhavissa sabbena sabbaṃ sabbathā sabbaṃ kassaci kimhici, seyyathidaṃ – devānaṃ vā devattāya, gandhabbānaṃ vā gandhabbattāya, yakkhānaṃ vā yakkhattāya, bhūtānaṃ vā bhūtattāya, manussānaṃ vā manussattāya, catuppadānaṃ vā catuppadattāya, pakkhīnaṃ vā pakkhittāya, sarīsapānaṃ vā sarīsapattāya\footnote{siriṃsapānaṃ siriṃsapattāya (sī. syā.)}, tesaṃ tesañca hi, ānanda, sattānaṃ tadattāya jāti nābhavissa. Sabbaso jātiyā asati jātinirodhā api nu kho jarāmaraṇaṃ paññāyethā’’ti? ‘‘No hetaṃ, bhante’’. ‘‘Tasmātihānanda, eseva hetu etaṃ nidānaṃ esa samudayo esa paccayo jarāmaraṇassa, yadidaṃ jāti’’.

\paragraph{99.} ‘‘‘Bhavapaccayā jātī’ti iti kho panetaṃ vuttaṃ, tadānanda, imināpetaṃ pariyāyena veditabbaṃ, yathā bhavapaccayā jāti. Bhavo ca hi, ānanda, nābhavissa sabbena sabbaṃ sabbathā sabbaṃ kassaci kimhici, seyyathidaṃ – kāmabhavo vā rūpabhavo vā arūpabhavo vā, sabbaso bhave asati bhavanirodhā api nu kho jāti paññāyethā’’ti? ‘‘No hetaṃ, bhante’’. ‘‘Tasmātihānanda, eseva hetu etaṃ nidānaṃ esa samudayo esa paccayo jātiyā, yadidaṃ bhavo’’.

\paragraph{100.} ‘‘‘Upādānapaccayā bhavo’ti iti kho panetaṃ vuttaṃ, tadānanda, imināpetaṃ pariyāyena veditabbaṃ, yathā upādānapaccayā bhavo. Upādānañca hi, ānanda, nābhavissa sabbena sabbaṃ sabbathā sabbaṃ kassaci kimhici , seyyathidaṃ – kāmupādānaṃ vā diṭṭhupādānaṃ vā sīlabbatupādānaṃ vā attavādupādānaṃ vā, sabbaso upādāne asati upādānanirodhā api nu kho bhavo paññāyethā’’ti? ‘‘No hetaṃ, bhante’’. ‘‘Tasmātihānanda, eseva hetu etaṃ nidānaṃ esa samudayo esa paccayo bhavassa, yadidaṃ upādānaṃ’’.

\paragraph{101.} ‘‘‘Taṇhāpaccayā upādāna’nti iti kho panetaṃ vuttaṃ tadānanda, imināpetaṃ pariyāyena veditabbaṃ, yathā taṇhāpaccayā upādānaṃ. Taṇhā ca hi, ānanda, nābhavissa sabbena sabbaṃ sabbathā sabbaṃ kassaci kimhici, seyyathidaṃ – rūpataṇhā saddataṇhā gandhataṇhā rasataṇhā phoṭṭhabbataṇhā dhammataṇhā, sabbaso taṇhāya asati taṇhānirodhā api nu kho upādānaṃ paññāyethā’’ti? ‘‘No hetaṃ, bhante’’. ‘‘Tasmātihānanda, eseva hetu etaṃ nidānaṃ esa samudayo esa paccayo upādānassa, yadidaṃ taṇhā’’.

\paragraph{102.} ‘‘‘Vedanāpaccayā taṇhā’ti iti kho panetaṃ vuttaṃ, tadānanda, imināpetaṃ pariyāyena veditabbaṃ, yathā vedanāpaccayā taṇhā. Vedanā ca hi, ānanda, nābhavissa sabbena sabbaṃ sabbathā sabbaṃ kassaci kimhici, seyyathidaṃ – cakkhusamphassajā vedanā sotasamphassajā vedanā ghānasamphassajā vedanā jivhāsamphassajā vedanā kāyasamphassajā vedanā manosamphassajā vedanā, sabbaso vedanāya asati vedanānirodhā api nu kho taṇhā paññāyethā’’ti ? ‘‘No hetaṃ, bhante’’. ‘‘Tasmātihānanda, eseva hetu etaṃ nidānaṃ esa samudayo esa paccayo taṇhāya, yadidaṃ vedanā’’.

\paragraph{103.} ‘‘Iti kho panetaṃ, ānanda, vedanaṃ paṭicca taṇhā, taṇhaṃ paṭicca pariyesanā, pariyesanaṃ paṭicca lābho, lābhaṃ paṭicca vinicchayo, vinicchayaṃ paṭicca chandarāgo, chandarāgaṃ paṭicca ajjhosānaṃ, ajjhosānaṃ paṭicca pariggaho, pariggahaṃ paṭicca macchariyaṃ, macchariyaṃ paṭicca ārakkho. Ārakkhādhikaraṇaṃ daṇḍādānasatthādānakalahaviggahavivādatuvaṃtuvaṃpesuññamusāvādā aneke pāpakā akusalā dhammā sambhavanti.

\paragraph{104.} ‘‘‘Ārakkhādhikaraṇaṃ\footnote{ārakkhaṃ paṭicca ārakkhādhikaraṇaṃ (syā.)} daṇḍādānasatthādānakalahaviggahavivādatuvaṃtuvaṃpesuññamusāvādā aneke pāpakā akusalā dhammā sambhavantī’ti iti kho panetaṃ vuttaṃ, tadānanda, imināpetaṃ pariyāyena veditabbaṃ, yathā ārakkhādhikaraṇaṃ daṇḍādānasatthādānakalahaviggahavivādatuvaṃtuvaṃpesuññamusāvādā aneke pāpakā akusalā dhammā sambhavanti. Ārakkho ca hi, ānanda, nābhavissa sabbena sabbaṃ sabbathā sabbaṃ kassaci kimhici, sabbaso ārakkhe asati ārakkhanirodhā api nu kho daṇḍādānasatthādānakalahaviggahavivādatuvaṃtuvaṃpesuññamusāvādā aneke pāpakā akusalā dhammā sambhaveyyu’’nti? ‘‘No hetaṃ, bhante’’. ‘‘Tasmātihānanda, eseva hetu etaṃ nidānaṃ esa samudayo esa paccayo daṇḍādānasatthādānakalahaviggahavivādatuvaṃtuvaṃpesuññamusāvādānaṃ anekesaṃ pāpakānaṃ akusalānaṃ dhammānaṃ sambhavāya yadidaṃ ārakkho.

\paragraph{105.} ‘‘‘Macchariyaṃ paṭicca ārakkho’ti iti kho panetaṃ vuttaṃ, tadānanda, imināpetaṃ pariyāyena veditabbaṃ, yathā macchariyaṃ paṭicca ārakkho. Macchariyañca hi, ānanda, nābhavissa sabbena sabbaṃ sabbathā sabbaṃ kassaci kimhici , sabbaso macchariye asati macchariyanirodhā api nu kho ārakkho paññāyethā’’ti? ‘‘No hetaṃ, bhante’’. ‘‘Tasmātihānanda, eseva hetu etaṃ nidānaṃ esa samudayo esa paccayo ārakkhassa, yadidaṃ macchariyaṃ’’.

\paragraph{106.} ‘‘‘Pariggahaṃ paṭicca macchariya’nti iti kho panetaṃ vuttaṃ, tadānanda, imināpetaṃ pariyāyena veditabbaṃ, yathā pariggahaṃ paṭicca macchariyaṃ. Pariggaho ca hi, ānanda, nābhavissa sabbena sabbaṃ sabbathā sabbaṃ kassaci kimhici, sabbaso pariggahe asati pariggahanirodhā api nu kho macchariyaṃ paññāyethā’’ti? ‘‘No hetaṃ, bhante’’. ‘‘Tasmātihānanda, eseva hetu etaṃ nidānaṃ esa samudayo esa paccayo macchariyassa, yadidaṃ pariggaho’’.

\paragraph{107.} ‘‘‘Ajjhosānaṃ paṭicca pariggaho’ti iti kho panetaṃ vuttaṃ, tadānanda, imināpetaṃ pariyāyena veditabbaṃ, yathā ajjhosānaṃ paṭicca pariggaho. Ajjhosānañca hi, ānanda, nābhavissa sabbena sabbaṃ sabbathā sabbaṃ kassaci kimhici, sabbaso ajjhosāne asati ajjhosānanirodhā api nu kho pariggaho paññāyethā’’ti ? ‘‘No hetaṃ, bhante’’. ‘‘Tasmātihānanda, eseva hetu etaṃ nidānaṃ esa samudayo esa paccayo pariggahassa – yadidaṃ ajjhosānaṃ’’.

\paragraph{108.} ‘‘‘Chandarāgaṃ paṭicca ajjhosāna’nti iti kho panetaṃ vuttaṃ, tadānanda, imināpetaṃ pariyāyena veditabbaṃ, yathā chandarāgaṃ paṭicca ajjhosānaṃ. Chandarāgo ca hi, ānanda, nābhavissa sabbena sabbaṃ sabbathā sabbaṃ kassaci kimhici, sabbaso chandarāge asati chandarāganirodhā api nu kho ajjhosānaṃ paññāyethā’’ti? ‘‘No hetaṃ, bhante’’. ‘‘Tasmātihānanda, eseva hetu etaṃ nidānaṃ esa samudayo esa paccayo ajjhosānassa, yadidaṃ chandarāgo’’.

\paragraph{109.} ‘‘‘Vinicchayaṃ paṭicca chandarāgo’ti iti kho panetaṃ vuttaṃ, tadānanda, imināpetaṃ pariyāyena veditabbaṃ, yathā vinicchayaṃ paṭicca chandarāgo. Vinicchayo ca hi, ānanda, nābhavissa sabbena sabbaṃ sabbathā sabbaṃ kassaci kimhici, sabbaso vinicchaye asati vinicchayanirodhā api nu kho chandarāgo paññāyethā’’ti? ‘‘No hetaṃ , bhante’’. ‘‘Tasmātihānanda, eseva hetu etaṃ nidānaṃ esa samudayo esa paccayo chandarāgassa, yadidaṃ vinicchayo’’.

\paragraph{110.} ‘‘‘Lābhaṃ paṭicca vinicchayo’ti iti kho panetaṃ vuttaṃ, tadānanda, imināpetaṃ pariyāyena veditabbaṃ, yathā lābhaṃ paṭicca vinicchayo. Lābho ca hi, ānanda, nābhavissa sabbena sabbaṃ sabbathā sabbaṃ kassaci kimhici, sabbaso lābhe asati lābhanirodhā api nu kho vinicchayo paññāyethā’’ti? ‘‘No hetaṃ, bhante’’. ‘‘Tasmātihānanda eseva hetu etaṃ nidānaṃ esa samudayo esa paccayo vinicchayassa, yadidaṃ lābho’’.

\paragraph{111.} ‘‘‘Pariyesanaṃ paṭicca lābho’ti iti kho panetaṃ vuttaṃ, tadānanda, imināpetaṃ pariyāyena veditabbaṃ, yathā pariyesanaṃ paṭicca lābho. Pariyesanā ca hi, ānanda, nābhavissa sabbena sabbaṃ sabbathā sabbaṃ kassaci kimhici, sabbaso pariyesanāya asati pariyesanānirodhā api nu kho lābho paññāyethā’’ti? ‘‘No hetaṃ, bhante’’. ‘‘Tasmātihānanda, eseva hetu etaṃ nidānaṃ esa samudayo esa paccayo lābhassa, yadidaṃ pariyesanā’’.

\paragraph{112.} ‘‘‘Taṇhaṃ paṭicca pariyesanā’ti iti kho panetaṃ vuttaṃ, tadānanda, imināpetaṃ pariyāyena veditabbaṃ, yathā taṇhaṃ paṭicca pariyesanā. Taṇhā ca hi, ānanda, nābhavissa sabbena sabbaṃ sabbathā sabbaṃ kassaci kimhici, seyyathidaṃ – kāmataṇhā bhavataṇhā vibhavataṇhā, sabbaso taṇhāya asati taṇhānirodhā api nu kho pariyesanā paññāyethā’’ti? ‘‘No hetaṃ, bhante’’. ‘‘Tasmātihānanda, eseva hetu etaṃ nidānaṃ esa samudayo esa paccayo pariyesanāya, yadidaṃ taṇhā. Iti kho, ānanda, ime dve dhammā\footnote{ime dhammā (ka.)} dvayena vedanāya ekasamosaraṇā bhavanti’’.

\paragraph{113.} ‘‘‘Phassapaccayā vedanā’ti iti kho panetaṃ vuttaṃ, tadānanda, imināpetaṃ pariyāyena veditabbaṃ, yathā ‘phassapaccayā vedanā. Phasso ca hi, ānanda, nābhavissa sabbena sabbaṃ sabbathā sabbaṃ kassaci kimhici, seyyathidaṃ – cakkhusamphasso sotasamphasso ghānasamphasso jivhāsamphasso kāyasamphasso manosamphasso, sabbaso phasse asati phassanirodhā api nu kho vedanā paññāyethā’’ti? ‘‘No hetaṃ, bhante’’. ‘‘Tasmātihānanda , eseva hetu etaṃ nidānaṃ esa samudayo esa paccayo vedanāya, yadidaṃ phasso’’.

\paragraph{114.} ‘‘‘Nāmarūpapaccayā phasso’ti iti kho panetaṃ vuttaṃ, tadānanda, imināpetaṃ pariyāyena veditabbaṃ, yathā nāmarūpapaccayā phasso. Yehi, ānanda, ākārehi yehi liṅgehi yehi nimittehi yehi uddesehi nāmakāyassa paññatti hoti, tesu ākāresu tesu liṅgesu tesu nimittesu tesu uddesesu asati api nu kho rūpakāye adhivacanasamphasso paññāyethā’’ti? ‘‘No hetaṃ, bhante’’. ‘‘Yehi, ānanda, ākārehi yehi liṅgehi yehi nimittehi yehi uddesehi rūpakāyassa paññatti hoti, tesu ākāresu…pe… tesu uddesesu asati api nu kho nāmakāye paṭighasamphasso paññāyethā’’ti? ‘‘No hetaṃ, bhante’’. ‘‘Yehi, ānanda, ākārehi…pe… yehi uddesehi nāmakāyassa ca rūpakāyassa ca paññatti hoti , tesu ākāresu…pe… tesu uddesesu asati api nu kho adhivacanasamphasso vā paṭighasamphasso vā paññāyethā’’ti? ‘‘No hetaṃ, bhante’’. ‘‘Yehi, ānanda, ākārehi…pe… yehi uddesehi nāmarūpassa paññatti hoti, tesu ākāresu …pe… tesu uddesesu asati api nu kho phasso paññāyethā’’ti? ‘‘No hetaṃ, bhante’’. ‘‘Tasmātihānanda, eseva hetu etaṃ nidānaṃ esa samudayo esa paccayo phassassa, yadidaṃ nāmarūpaṃ’’.

\paragraph{115.} ‘‘‘Viññāṇapaccayā nāmarūpa’nti iti kho panetaṃ vuttaṃ, tadānanda, imināpetaṃ pariyāyena veditabbaṃ, yathā viññāṇapaccayā nāmarūpaṃ. Viññāṇañca hi, ānanda, mātukucchismiṃ na okkamissatha, api nu kho nāmarūpaṃ mātukucchismiṃ samuccissathā’’ti? ‘‘No hetaṃ, bhante’’. ‘‘Viññāṇañca hi, ānanda, mātukucchismiṃ okkamitvā vokkamissatha, api nu kho nāmarūpaṃ itthattāya abhinibbattissathā’’ti? ‘‘No hetaṃ, bhante’’. ‘‘Viññāṇañca hi, ānanda, daharasseva sato vocchijjissatha kumārakassa vā kumārikāya vā, api nu kho nāmarūpaṃ vuddhiṃ virūḷhiṃ vepullaṃ āpajjissathā’’ti? ‘‘No hetaṃ, bhante’’. ‘‘Tasmātihānanda, eseva hetu etaṃ nidānaṃ esa samudayo esa paccayo nāmarūpassa – yadidaṃ viññāṇaṃ’’.

\paragraph{116.} ‘‘‘Nāmarūpapaccayā viññāṇa’nti iti kho panetaṃ vuttaṃ, tadānanda, imināpetaṃ pariyāyena veditabbaṃ, yathā nāmarūpapaccayā viññāṇaṃ. Viññāṇañca hi, ānanda, nāmarūpe patiṭṭhaṃ na labhissatha, api nu kho āyatiṃ jātijarāmaraṇaṃ dukkhasamudayasambhavo\footnote{jātijarāmaraṇadukkhasamudayasambhavo (sī. syā. pī.)} paññāyethā’’ti? ‘‘No hetaṃ, bhante’’. ‘‘Tasmātihānanda, eseva hetu etaṃ nidānaṃ esa samudayo esa paccayo viññāṇassa yadidaṃ nāmarūpaṃ. Ettāvatā kho, ānanda, jāyetha vā jīyetha\footnote{jiyyetha (ka.)} vā mīyetha\footnote{miyyetha (ka.)} vā cavetha vā upapajjetha vā. Ettāvatā adhivacanapatho, ettāvatā niruttipatho, ettāvatā paññattipatho, ettāvatā paññāvacaraṃ, ettāvatā vaṭṭaṃ vattati itthattaṃ paññāpanāya yadidaṃ nāmarūpaṃ saha viññāṇena aññamaññapaccayatā pavattati.

\subsubsection{Attapaññatti}

\paragraph{117.} ‘‘Kittāvatā ca, ānanda, attānaṃ paññapento paññapeti? Rūpiṃ vā hi, ānanda, parittaṃ attānaṃ paññapento paññapeti – ‘‘rūpī me paritto attā’’ti. Rūpiṃ vā hi , ānanda, anantaṃ attānaṃ paññapento paññapeti – ‘rūpī me ananto attā’ti. Arūpiṃ vā hi, ānanda, parittaṃ attānaṃ paññapento paññapeti – ‘arūpī me paritto attā’ti. Arūpiṃ vā hi, ānanda, anantaṃ attānaṃ paññapento paññapeti – ‘arūpī me ananto attā’ti.

\paragraph{118.} ‘‘Tatrānanda, yo so rūpiṃ parittaṃ attānaṃ paññapento paññapeti. Etarahi vā so rūpiṃ parittaṃ attānaṃ paññapento paññapeti, tattha bhāviṃ vā so rūpiṃ parittaṃ attānaṃ paññapento paññapeti, ‘atathaṃ vā pana santaṃ tathattāya upakappessāmī’ti iti vā panassa hoti. Evaṃ santaṃ kho, ānanda, rūpiṃ\footnote{rūpī (ka.)} parittattānudiṭṭhi anusetīti iccālaṃ vacanāya.

‘‘Tatrānanda, yo so rūpiṃ anantaṃ attānaṃ paññapento paññapeti. Etarahi vā so rūpiṃ anantaṃ attānaṃ paññapento paññapeti, tattha bhāviṃ vā so rūpiṃ anantaṃ attānaṃ paññapento paññapeti, ‘atathaṃ vā pana santaṃ tathattāya upakappessāmī’ti iti vā panassa hoti. Evaṃ santaṃ kho, ānanda, rūpiṃ\footnote{rūpī (ka.)} anantattānudiṭṭhi anusetīti iccālaṃ vacanāya.

‘‘Tatrānanda, yo so arūpiṃ parittaṃ attānaṃ paññapento paññapeti. Etarahi vā so arūpiṃ parittaṃ attānaṃ paññapento paññapeti, tattha bhāviṃ vā so arūpiṃ parittaṃ attānaṃ paññapento paññapeti, ‘atathaṃ vā pana santaṃ tathattāya upakappessāmī’ti iti vā panassa hoti. Evaṃ santaṃ kho, ānanda, arūpiṃ\footnote{arūpī (ka.)} parittattānudiṭṭhi anusetīti iccālaṃ vacanāya.

‘‘Tatrānanda, yo so arūpiṃ anantaṃ attānaṃ paññapento paññapeti. Etarahi vā so arūpiṃ anantaṃ attānaṃ paññapento paññapeti, tattha bhāviṃ vā so arūpiṃ anantaṃ attānaṃ paññapento paññapeti, ‘atathaṃ vā pana santaṃ tathattāya upakappessāmī’ti iti vā panassa hoti. Evaṃ santaṃ kho, ānanda, arūpiṃ\footnote{arūpī (ka.)} anantattānudiṭṭhi anusetīti iccālaṃ vacanāya. Ettāvatā kho, ānanda, attānaṃ paññapento paññapeti.

\subsubsection{Naattapaññatti}

\paragraph{119.} ‘‘Kittāvatā ca, ānanda, attānaṃ na paññapento na paññapeti? Rūpiṃ vā hi, ānanda, parittaṃ attānaṃ na paññapento na paññapeti – ‘rūpī me paritto attā’ti. Rūpiṃ vā hi, ānanda, anantaṃ attānaṃ na paññapento na paññapeti – ‘rūpī me ananto attā’ti. Arūpiṃ vā hi, ānanda, parittaṃ attānaṃ na paññapento na paññapeti – ‘arūpī me paritto attā’ti. Arūpiṃ vā hi, ānanda, anantaṃ attānaṃ na paññapento na paññapeti – ‘arūpī me ananto attā’ti.

\paragraph{120.} ‘‘Tatrānanda, yo so rūpiṃ parittaṃ attānaṃ na paññapento na paññapeti. Etarahi vā so rūpiṃ parittaṃ attānaṃ na paññapento na paññapeti, tattha bhāviṃ vā so rūpiṃ parittaṃ attānaṃ na paññapento na paññapeti, ‘atathaṃ vā pana santaṃ tathattāya upakappessāmī’ti iti vā panassa na hoti. Evaṃ santaṃ kho, ānanda, rūpiṃ parittattānudiṭṭhi nānusetīti iccālaṃ vacanāya.

‘‘Tatrānanda , yo so rūpiṃ anantaṃ attānaṃ na paññapento na paññapeti. Etarahi vā so rūpiṃ anantaṃ attānaṃ na paññapento na paññapeti, tattha bhāviṃ vā so rūpiṃ anantaṃ attānaṃ na paññapento na paññapeti, ‘atathaṃ vā pana santaṃ tathattāya upakappessāmī’ti iti vā panassa na hoti. Evaṃ santaṃ kho, ānanda, rūpiṃ anantattānudiṭṭhi nānusetīti iccālaṃ vacanāya.

‘‘Tatrānanda, yo so arūpiṃ parittaṃ attānaṃ na paññapento na paññapeti. Etarahi vā so arūpiṃ parittaṃ attānaṃ na paññapento na paññapeti, tattha bhāviṃ vā so arūpiṃ parittaṃ attānaṃ na paññapento na paññapeti, ‘atathaṃ vā pana santaṃ tathattāya upakappessāmī’ti iti vā panassa na hoti. Evaṃ santaṃ kho, ānanda, arūpiṃ parittattānudiṭṭhi nānusetīti iccālaṃ vacanāya.

‘‘Tatrānanda, yo so arūpiṃ anantaṃ attānaṃ na paññapento na paññapeti. Etarahi vā so arūpiṃ anantaṃ attānaṃ na paññapento na paññapeti, tattha bhāviṃ vā so arūpiṃ anantaṃ attānaṃ na paññapento na paññapeti, ‘atathaṃ vā pana santaṃ tathattāya upakappessāmī’ti iti vā panassa na hoti. Evaṃ santaṃ kho, ānanda, arūpiṃ anantattānudiṭṭhi nānusetīti iccālaṃ vacanāya. Ettāvatā kho, ānanda, attānaṃ na paññapento na paññapeti.

\subsubsection{Attasamanupassanā}

\paragraph{121.} ‘‘Kittāvatā ca, ānanda, attānaṃ samanupassamāno samanupassati? Vedanaṃ vā hi, ānanda, attānaṃ samanupassamāno samanupassati – ‘vedanā me attā’ti. ‘Na heva kho me vedanā attā, appaṭisaṃvedano me attā’ti iti vā hi, ānanda, attānaṃ samanupassamāno samanupassati. ‘Na heva kho me vedanā attā, nopi appaṭisaṃvedano me attā, attā me vediyati, vedanādhammo hi me attā’ti iti vā hi, ānanda, attānaṃ samanupassamāno samanupassati.

\paragraph{122.} ‘‘Tatrānanda, yo so evamāha – ‘vedanā me attā’ti, so evamassa vacanīyo – ‘tisso kho imā, āvuso, vedanā – sukhā vedanā dukkhā vedanā adukkhamasukhā vedanā. Imāsaṃ kho tvaṃ tissannaṃ vedanānaṃ katamaṃ attato samanupassasī’ti? Yasmiṃ, ānanda, samaye sukhaṃ vedanaṃ vedeti, neva tasmiṃ samaye dukkhaṃ vedanaṃ vedeti, na adukkhamasukhaṃ vedanaṃ vedeti; sukhaṃyeva tasmiṃ samaye vedanaṃ vedeti. Yasmiṃ, ānanda, samaye dukkhaṃ vedanaṃ vedeti, neva tasmiṃ samaye sukhaṃ vedanaṃ vedeti, na adukkhamasukhaṃ vedanaṃ vedeti; dukkhaṃyeva tasmiṃ samaye vedanaṃ vedeti. Yasmiṃ, ānanda, samaye adukkhamasukhaṃ vedanaṃ vedeti, neva tasmiṃ samaye sukhaṃ vedanaṃ vedeti, na dukkhaṃ vedanaṃ vedeti; adukkhamasukhaṃyeva tasmiṃ samaye vedanaṃ vedeti.

\paragraph{123.} ‘‘Sukhāpi kho, ānanda, vedanā aniccā saṅkhatā paṭiccasamuppannā khayadhammā vayadhammā virāgadhammā nirodhadhammā. Dukkhāpi kho, ānanda, vedanā aniccā saṅkhatā paṭiccasamuppannā khayadhammā vayadhammā virāgadhammā nirodhadhammā. Adukkhamasukhāpi kho, ānanda, vedanā aniccā saṅkhatā paṭiccasamuppannā khayadhammā vayadhammā virāgadhammā nirodhadhammā. Tassa sukhaṃ vedanaṃ vediyamānassa ‘eso me attā’ti hoti. Tassāyeva sukhāya vedanāya nirodhā ‘byagā\footnote{byaggā (sī. ka.)} me attā’ti hoti. Dukkhaṃ vedanaṃ vediyamānassa ‘eso me attā’ti hoti. Tassāyeva dukkhāya vedanāya nirodhā ‘byagā me attā’ti hoti. Adukkhamasukhaṃ vedanaṃ vediyamānassa ‘eso me attā’ti hoti. Tassāyeva adukkhamasukhāya vedanāya nirodhā ‘byagā me attā’ti hoti. Iti so diṭṭheva dhamme aniccasukhadukkhavokiṇṇaṃ uppādavayadhammaṃ attānaṃ samanupassamāno samanupassati, yo so evamāha – ‘vedanā me attā’ti. Tasmātihānanda, etena petaṃ nakkhamati – ‘vedanā me attā’ti samanupassituṃ.

\paragraph{124.} ‘‘Tatrānanda , yo so evamāha – ‘na heva kho me vedanā attā, appaṭisaṃvedano me attā’ti, so evamassa vacanīyo – ‘yattha panāvuso, sabbaso vedayitaṃ natthi api nu kho, tattha ‘‘ayamahamasmī’’ti siyā’’’ti ? ‘‘No hetaṃ, bhante’’. ‘‘Tasmātihānanda, etena petaṃ nakkhamati – ‘na heva kho me vedanā attā, appaṭisaṃvedano me attā’ti samanupassituṃ.

\paragraph{125.} ‘‘Tatrānanda , yo so evamāha – ‘na heva kho me vedanā attā, nopi appaṭisaṃvedano me attā, attā me vediyati, vedanādhammo hi me attā’ti. So evamassa vacanīyo – vedanā ca hi, āvuso, sabbena sabbaṃ sabbathā sabbaṃ aparisesā nirujjheyyuṃ. Sabbaso vedanāya asati vedanānirodhā api nu kho tattha ‘ayamahamasmī’ti siyā’’ti? ‘No hetaṃ, bhante’’. ‘‘Tasmātihānanda, etena petaṃ nakkhamati – ‘‘na heva kho me vedanā attā, nopi appaṭisaṃvedano me attā, attā me vediyati, vedanādhammo hi me attā’ti samanupassituṃ.

\paragraph{126.} ‘‘Yato kho, ānanda, bhikkhu neva vedanaṃ attānaṃ samanupassati, nopi appaṭisaṃvedanaṃ attānaṃ samanupassati, nopi ‘attā me vediyati, vedanādhammo hi me attā’ti samanupassati. So evaṃ na samanupassanto na ca kiñci loke upādiyati, anupādiyaṃ na paritassati, aparitassaṃ\footnote{aparitassanaṃ (ka.)} paccattaññeva parinibbāyati, ‘khīṇā jāti, vusitaṃ brahmacariyaṃ, kataṃ karaṇīyaṃ, nāparaṃ itthattāyā’ti pajānāti. Evaṃ vimuttacittaṃ kho, ānanda, bhikkhuṃ yo evaṃ vadeyya – ‘hoti tathāgato paraṃ maraṇā itissa\footnote{iti sā (aṭṭhakathāyaṃ pāṭhantaraṃ)} diṭṭhī’ti, tadakallaṃ. ‘Na hoti tathāgato paraṃ maraṇā itissa diṭṭhī’ti, tadakallaṃ. ‘Hoti ca na ca hoti tathāgato paraṃ maraṇā itissa diṭṭhī’ti, tadakallaṃ. ‘Neva hoti na na hoti tathāgato paraṃ maraṇā itissa diṭṭhī’ti, tadakallaṃ. Taṃ kissa hetu? Yāvatā, ānanda, adhivacanaṃ yāvatā adhivacanapatho, yāvatā nirutti yāvatā niruttipatho, yāvatā paññatti yāvatā paññattipatho, yāvatā paññā yāvatā paññāvacaraṃ, yāvatā vaṭṭaṃ\footnote{yāvatā vaṭṭaṃ vaṭṭati (ka. sī.)}, yāvatā vaṭṭati\footnote{yāvatā vaṭṭaṃ vaṭṭati (ka. sī.)}, tadabhiññāvimutto bhikkhu, tadabhiññāvimuttaṃ bhikkhuṃ ‘na jānāti na passati itissa diṭṭhī’ti, tadakallaṃ.

\subsubsection{Satta viññāṇaṭṭhiti}

\paragraph{127.} ‘‘Satta kho, ānanda\footnote{satta kho imā ānanda (ka. sī. syā.)}, viññāṇaṭṭhitiyo, dve āyatanāni. Katamā satta? Santānanda, sattā nānattakāyā nānattasaññino, seyyathāpi manussā , ekacce ca devā, ekacce ca vinipātikā. Ayaṃ paṭhamā viññāṇaṭṭhiti. Santānanda, sattā nānattakāyā ekattasaññino, seyyathāpi devā brahmakāyikā paṭhamābhinibbattā. Ayaṃ dutiyā viññāṇaṭṭhiti. Santānanda, sattā ekattakāyā nānattasaññino, seyyathāpi devā ābhassarā. Ayaṃ tatiyā viññāṇaṭṭhiti. Santānanda, sattā ekattakāyā ekattasaññino, seyyathāpi devā subhakiṇhā. Ayaṃ catutthī viññāṇaṭṭhiti. Santānanda, sattā sabbaso rūpasaññānaṃ samatikkamā paṭighasaññānaṃ atthaṅgamā nānattasaññānaṃ amanasikārā ‘ananto ākāso’ti ākāsānañcāyatanūpagā. Ayaṃ pañcamī viññāṇaṭṭhiti . Santānanda, sattā sabbaso ākāsānañcāyatanaṃ samatikkamma ‘anantaṃ viññāṇa’nti viññāṇañcāyatanūpagā. Ayaṃ chaṭṭhī viññāṇaṭṭhiti. Santānanda, sattā sabbaso viññāṇañcāyatanaṃ samatikkamma ‘natthi kiñcī’ti ākiñcaññāyatanūpagā. Ayaṃ sattamī viññāṇaṭṭhiti. Asaññasattāyatanaṃ nevasaññānāsaññāyatanameva dutiyaṃ.

\paragraph{128.} ‘‘Tatrānanda, yāyaṃ paṭhamā viññāṇaṭṭhiti nānattakāyā nānattasaññino, seyyathāpi manussā, ekacce ca devā, ekacce ca vinipātikā. Yo nu kho, ānanda, tañca pajānāti, tassā ca samudayaṃ pajānāti, tassā ca atthaṅgamaṃ pajānāti, tassā ca assādaṃ pajānāti, tassā ca ādīnavaṃ pajānāti, tassā ca nissaraṇaṃ pajānāti, kallaṃ nu tena tadabhinanditu’’nti? ‘‘No hetaṃ, bhante’’…pe… ‘‘tatrānanda, yamidaṃ asaññasattāyatanaṃ. Yo nu kho, ānanda, tañca pajānāti, tassa ca samudayaṃ pajānāti, tassa ca atthaṅgamaṃ pajānāti, tassa ca assādaṃ pajānāti, tassa ca ādīnavaṃ pajānāti, tassa ca nissaraṇaṃ pajānāti, kallaṃ nu tena tadabhinanditu’’nti? ‘‘No hetaṃ, bhante’’. ‘‘Tatrānanda, yamidaṃ nevasaññānāsaññāyatanaṃ. Yo nu kho, ānanda, tañca pajānāti, tassa ca samudayaṃ pajānāti, tassa ca atthaṅgamaṃ pajānāti, tassa ca assādaṃ pajānāti, tassa ca ādīnavaṃ pajānāti, tassa ca nissaraṇaṃ pajānāti, kallaṃ nu tena tadabhinanditu’’nti? ‘‘No hetaṃ, bhante’’. Yato kho, ānanda, bhikkhu imāsañca sattannaṃ viññāṇaṭṭhitīnaṃ imesañca dvinnaṃ āyatanānaṃ samudayañca atthaṅgamañca assādañca ādīnavañca nissaraṇañca yathābhūtaṃ viditvā anupādā vimutto hoti, ayaṃ vuccatānanda, bhikkhu paññāvimutto.

\subsubsection{Aṭṭha vimokkhā}

\paragraph{129.} ‘‘Aṭṭha kho ime, ānanda, vimokkhā. Katame aṭṭha? Rūpī rūpāni passati ayaṃ paṭhamo vimokkho. Ajjhattaṃ arūpasaññī bahiddhā rūpāni passati, ayaṃ dutiyo vimokkho. Subhanteva adhimutto hoti, ayaṃ tatiyo vimokkho. Sabbaso rūpasaññānaṃ samatikkamā paṭighasaññānaṃ atthaṅgamā nānattasaññānaṃ amanasikārā ‘ananto ākāso’ti ākāsānañcāyatanaṃ upasampajja viharati, ayaṃ catuttho vimokkho. Sabbaso ākāsānañcāyatanaṃ samatikkamma ‘anantaṃ viññāṇa’nti viññāṇañcāyatanaṃ upasampajja viharati, ayaṃ pañcamo vimokkho. Sabbaso viññāṇañcāyatanaṃ samatikkamma ‘natthi kiñcī’ti ākiñcaññāyatanaṃ upasampajja viharati, ayaṃ chaṭṭho vimokkho. Sabbaso ākiñcaññāyatanaṃ samatikkamma ‘nevasaññānāsaññā’yatanaṃ upasampajja viharati, ayaṃ sattamo vimokkho. Sabbaso nevasaññānāsaññāyatanaṃ samatikkamma saññāvedayitanirodhaṃ upasampajja viharati, ayaṃ aṭṭhamo vimokkho. Ime kho, ānanda, aṭṭha vimokkhā.

\paragraph{130.} ‘‘Yato kho, ānanda, bhikkhu ime aṭṭha vimokkhe anulomampi samāpajjati, paṭilomampi samāpajjati, anulomapaṭilomampi samāpajjati, yatthicchakaṃ yadicchakaṃ yāvaticchakaṃ samāpajjatipi vuṭṭhātipi. Āsavānañca khayā anāsavaṃ cetovimuttiṃ paññāvimuttiṃ diṭṭheva dhamme sayaṃ abhiññā sacchikatvā upasampajja viharati, ayaṃ vuccatānanda, bhikkhu ubhatobhāgavimutto. Imāya ca ānanda ubhatobhāgavimuttiyā aññā ubhatobhāgavimutti uttaritarā vā paṇītatarā vā natthī’’ti. Idamavoca bhagavā. Attamano āyasmā ānando bhagavato bhāsitaṃ abhinandīti.

\xsectionEnd{Mahānidānasuttaṃ niṭṭhitaṃ dutiyaṃ.}


\clearpage
\section{Mahāparinibbānasuttaṃ}

\paragraph{131.} Evaṃ me sutaṃ – ekaṃ samayaṃ bhagavā rājagahe viharati gijjhakūṭe pabbate. Tena kho pana samayena rājā māgadho ajātasattu vedehiputto vajjī abhiyātukāmo hoti. So evamāha – ‘‘ahaṃ hime vajjī evaṃmahiddhike evaṃmahānubhāve ucchecchāmi\footnote{ucchejjāmi (syā. pī.), ucchijjāmi (ka.)} vajjī, vināsessāmi vajjī, anayabyasanaṃ āpādessāmi vajjī’’ti\footnote{āpādessāmi vajjīti (sabbattha) a. ni. 7.22 passitabbaṃ}.

\paragraph{132.} Atha kho rājā māgadho ajātasattu vedehiputto vassakāraṃ brāhmaṇaṃ magadhamahāmattaṃ āmantesi – ‘‘ehi tvaṃ, brāhmaṇa, yena bhagavā tenupasaṅkama; upasaṅkamitvā mama vacanena bhagavato pāde sirasā vandāhi, appābādhaṃ appātaṅkaṃ lahuṭṭhānaṃ balaṃ phāsuvihāraṃ puccha – ‘rājā, bhante, māgadho ajātasattu vedehiputto bhagavato pāde sirasā vandati, appābādhaṃ appātaṅkaṃ lahuṭṭhānaṃ balaṃ phāsuvihāraṃ pucchatī’ti. Evañca vadehi – ‘rājā, bhante, māgadho ajātasattu vedehiputto vajjī abhiyātukāmo. So evamāha – ‘‘ahaṃ hime vajjī evaṃmahiddhike evaṃmahānubhāve ucchecchāmi vajjī, vināsessāmi vajjī, anayabyasanaṃ āpādessāmī’’’ti. Yathā te bhagavā byākaroti, taṃ sādhukaṃ uggahetvā mama āroceyyāsi. Na hi tathāgatā vitathaṃ bhaṇantī’’ti.

\subsubsection{Vassakārabrāhmaṇo}

\paragraph{133.} ‘‘Evaṃ, bho’’ti kho vassakāro brāhmaṇo magadhamahāmatto rañño māgadhassa ajātasattussa vedehiputtassa paṭissutvā bhaddāni bhaddāni yānāni yojetvā bhaddaṃ bhaddaṃ yānaṃ abhiruhitvā bhaddehi bhaddehi yānehi rājagahamhā niyyāsi, yena gijjhakūṭo pabbato tena pāyāsi. Yāvatikā yānassa bhūmi, yānena gantvā, yānā paccorohitvā pattikova yena bhagavā tenupasaṅkami; upasaṅkamitvā bhagavatā saddhiṃ sammodi. Sammodanīyaṃ kathaṃ sāraṇīyaṃ vītisāretvā ekamantaṃ nisīdi. Ekamantaṃ nisinno kho vassakāro brāhmaṇo magadhamahāmatto bhagavantaṃ etadavoca – ‘‘rājā, bho gotama, māgadho ajātasattu vedehiputto bhoto gotamassa pāde sirasā vandati, appābādhaṃ appātaṅkaṃ lahuṭṭhānaṃ balaṃ phāsuvihāraṃ pucchati. Rājā\footnote{evañca vadeti rājā (ka.)}, bho gotama, māgadho ajātasattu vedehiputto vajjī abhiyātukāmo. So evamāha – ‘ahaṃ hime vajjī evaṃmahiddhike evaṃmahānubhāve ucchecchāmi vajjī, vināsessāmi vajjī, anayabyasanaṃ āpādessāmī’’’ti.

\subsubsection{Rājaaparihāniyadhammā}

\paragraph{134.} Tena kho pana samayena āyasmā ānando bhagavato piṭṭhito ṭhito hoti bhagavantaṃ bījayamāno\footnote{vījayamāno (sī.), vījiyamāno (syā.)}. Atha kho bhagavā āyasmantaṃ ānandaṃ āmantesi – ‘‘kinti te, ānanda, sutaṃ, ‘vajjī abhiṇhaṃ sannipātā sannipātabahulā’ti? ‘‘Sutaṃ metaṃ, bhante – ‘vajjī abhiṇhaṃ sannipātā sannipātabahulā’’ti. ‘‘Yāvakīvañca, ānanda, vajjī abhiṇhaṃ sannipātā sannipātabahulā bhavissanti, vuddhiyeva, ānanda, vajjīnaṃ pāṭikaṅkhā, no parihāni.

‘‘Kinti te, ānanda, sutaṃ , ‘vajjī samaggā sannipatanti, samaggā vuṭṭhahanti, samaggā vajjikaraṇīyāni karontī’ti? ‘‘Sutaṃ metaṃ, bhante – ‘vajjī samaggā sannipatanti, samaggā vuṭṭhahanti, samaggā vajjikaraṇīyāni karontī’’ti. ‘‘Yāvakīvañca, ānanda, vajjī samaggā sannipatissanti, samaggā vuṭṭhahissanti, samaggā vajjikaraṇīyāni karissanti, vuddhiyeva, ānanda, vajjīnaṃ pāṭikaṅkhā, no parihāni.

‘‘Kinti te, ānanda, sutaṃ, ‘vajjī apaññattaṃ na paññapenti, paññattaṃ na samucchindanti, yathāpaññatte porāṇe vajjidhamme samādāya vattantī’’’ti? ‘‘Sutaṃ metaṃ, bhante – ‘vajjī apaññattaṃ na paññapenti, paññattaṃ na samucchindanti, yathāpaññatte porāṇe vajjidhamme samādāya vattantī’’’ti. ‘‘Yāvakīvañca, ānanda, ‘‘vajjī apaññattaṃ na paññapessanti, paññattaṃ na samucchindissanti, yathāpaññatte porāṇe vajjidhamme samādāya vattissanti, vuddhiyeva, ānanda, vajjīnaṃ pāṭikaṅkhā, no parihāni.

‘‘Kinti te, ānanda, sutaṃ, ‘vajjī ye te vajjīnaṃ vajjimahallakā, te sakkaronti garuṃ karonti\footnote{garukaronti (sī. syā. pī.)} mānenti pūjenti, tesañca sotabbaṃ maññantī’’’ti? ‘‘Sutaṃ metaṃ, bhante – ‘vajjī ye te vajjīnaṃ vajjimahallakā, te sakkaronti garuṃ karonti mānenti pūjenti, tesañca sotabbaṃ maññantī’’’ti. ‘‘Yāvakīvañca, ānanda, vajjī ye te vajjīnaṃ vajjimahallakā , te sakkarissanti garuṃ karissanti mānessanti pūjessanti, tesañca sotabbaṃ maññissanti, vuddhiyeva, ānanda, vajjīnaṃ pāṭikaṅkhā, no parihāni.

‘‘Kinti te, ānanda, sutaṃ, ‘vajjī yā tā kulitthiyo kulakumāriyo, tā na okkassa pasayha vāsentī’’’ti? ‘‘Sutaṃ metaṃ, bhante – ‘vajjī yā tā kulitthiyo kulakumāriyo tā na okkassa pasayha vāsentī’’’ti. ‘‘Yāvakīvañca, ānanda, vajjī yā tā kulitthiyo kulakumāriyo, tā na okkassa pasayha vāsessanti, vuddhiyeva, ānanda, vajjīnaṃ pāṭikaṅkhā, no parihāni.

‘‘Kinti te, ānanda, sutaṃ, ‘vajjī yāni tāni

Vajjīnaṃ vajjicetiyāni abbhantarāni ceva bāhirāni ca, tāni sakkaronti garuṃ karonti mānenti pūjenti, tesañca dinnapubbaṃ katapubbaṃ dhammikaṃ baliṃ no parihāpentī’’’ti? ‘‘Sutaṃ metaṃ, bhante – ‘vajjī yāni tāni vajjīnaṃ vajjicetiyāni abbhantarāni ceva bāhirāni ca, tāni sakkaronti garuṃ karonti mānenti pūjenti tesañca dinnapubbaṃ katapubbaṃ dhammikaṃ baliṃ no parihāpentī’’’ti. ‘‘Yāvakīvañca, ānanda, vajjī yāni tāni vajjīnaṃ vajjicetiyāni abbhantarāni ceva bāhirāni ca, tāni sakkarissanti garuṃ karissanti mānessanti pūjessanti, tesañca dinnapubbaṃ katapubbaṃ dhammikaṃ baliṃ no parihāpessanti, vuddhiyeva, ānanda, vajjīnaṃ pāṭikaṅkhā, no parihāni.

‘‘Kinti te, ānanda, sutaṃ, ‘vajjīnaṃ arahantesu dhammikā rakkhāvaraṇagutti susaṃvihitā, kinti anāgatā ca arahanto vijitaṃ āgaccheyyuṃ, āgatā ca arahanto vijite phāsu vihareyyu’’’nti? ‘‘Sutaṃ metaṃ, bhante ‘vajjīnaṃ arahantesu dhammikā rakkhāvaraṇagutti susaṃvihitā kinti anāgatā ca arahanto vijitaṃ āgaccheyyuṃ, āgatā ca arahanto vijite phāsu vihareyyu’’’nti. ‘‘Yāvakīvañca, ānanda, vajjīnaṃ arahantesu dhammikā rakkhāvaraṇagutti susaṃvihitā bhavissati, kinti anāgatā ca arahanto vijitaṃ āgaccheyyuṃ, āgatā ca arahanto vijite phāsu vihareyyunti. Vuddhiyeva, ānanda, vajjīnaṃ pāṭikaṅkhā, no parihānī’’ti.

\paragraph{135.} Atha kho bhagavā vassakāraṃ brāhmaṇaṃ magadhamahāmattaṃ āmantesi – ‘‘ekamidāhaṃ, brāhmaṇa, samayaṃ vesāliyaṃ viharāmi sārandade\footnote{sānandare (ka.)} cetiye. Tatrāhaṃ vajjīnaṃ ime satta aparihāniye dhamme desesiṃ. Yāvakīvañca, brāhmaṇa, ime satta aparihāniyā dhammā vajjīsu ṭhassanti, imesu ca sattasu aparihāniyesu dhammesu vajjī sandississanti, vuddhiyeva, brāhmaṇa, vajjīnaṃ pāṭikaṅkhā, no parihānī’’ti.

Evaṃ vutte, vassakāro brāhmaṇo magadhamahāmatto bhagavantaṃ etadavoca – ‘‘ekamekenapi, bho gotama, aparihāniyena dhammena samannāgatānaṃ vajjīnaṃ vuddhiyeva pāṭikaṅkhā, no parihāni . Ko pana vādo sattahi aparihāniyehi dhammehi. Akaraṇīyāva\footnote{akaraṇīyā ca (syā. ka.)}, bho gotama, vajjī\footnote{vajjīnaṃ (ka.)} raññā māgadhena ajātasattunā vedehiputtena yadidaṃ yuddhassa, aññatra upalāpanāya aññatra mithubhedā. Handa ca dāni mayaṃ, bho gotama, gacchāma , bahukiccā mayaṃ bahukaraṇīyā’’ti. ‘‘Yassadāni tvaṃ, brāhmaṇa, kālaṃ maññasī’’ti. Atha kho vassakāro brāhmaṇo magadhamahāmatto bhagavato bhāsitaṃ abhinanditvā anumoditvā uṭṭhāyāsanā pakkāmi.

\subsubsection{Bhikkhuaparihāniyadhammā}

\paragraph{136.} Atha kho bhagavā acirapakkante vassakāre brāhmaṇe magadhamahāmatte āyasmantaṃ ānandaṃ āmantesi – ‘‘gaccha tvaṃ, ānanda, yāvatikā bhikkhū rājagahaṃ upanissāya viharanti, te sabbe upaṭṭhānasālāyaṃ sannipātehī’’ti. ‘‘Evaṃ, bhante’’ti kho āyasmā ānando bhagavato paṭissutvā yāvatikā bhikkhū rājagahaṃ upanissāya viharanti, te sabbe upaṭṭhānasālāyaṃ sannipātetvā yena bhagavā tenupasaṅkami; upasaṅkamitvā bhagavantaṃ abhivādetvā ekamantaṃ aṭṭhāsi. Ekamantaṃ ṭhito kho āyasmā ānando bhagavantaṃ etadavoca – ‘‘sannipatito, bhante, bhikkhusaṅgho, yassadāni, bhante, bhagavā kālaṃ maññatī’’ti.

Atha kho bhagavā uṭṭhāyāsanā yena upaṭṭhānasālā tenupasaṅkami; upasaṅkamitvā paññatte āsane nisīdi. Nisajja kho bhagavā bhikkhū āmantesi – ‘‘satta vo, bhikkhave, aparihāniye dhamme desessāmi, taṃ suṇātha, sādhukaṃ manasikarotha, bhāsissāmī’’ti. ‘‘Evaṃ, bhante’’ti kho te bhikkhū bhagavato paccassosuṃ. Bhagavā etadavoca –

‘‘Yāvakīvañca , bhikkhave, bhikkhū abhiṇhaṃ sannipātā sannipātabahulā bhavissanti, vuddhiyeva, bhikkhave, bhikkhūnaṃ pāṭikaṅkhā, no parihāni.

‘‘Yāvakīvañca, bhikkhave, bhikkhū samaggā sannipatissanti, samaggā vuṭṭhahissanti, samaggā saṅghakaraṇīyāni karissanti , vuddhiyeva, bhikkhave, bhikkhūnaṃ pāṭikaṅkhā, no parihāni.

‘‘Yāvakīvañca, bhikkhave, bhikkhū apaññattaṃ na paññapessanti, paññattaṃ na samucchindissanti, yathāpaññattesu sikkhāpadesu samādāya vattissanti, vuddhiyeva, bhikkhave, bhikkhūnaṃ pāṭikaṅkhā, no parihāni.

‘‘Yāvakīvañca, bhikkhave, bhikkhū ye te bhikkhū therā rattaññū cirapabbajitā saṅghapitaro saṅghapariṇāyakā, te sakkarissanti garuṃ karissanti mānessanti pūjessanti, tesañca sotabbaṃ maññissanti, vuddhiyeva, bhikkhave, bhikkhūnaṃ pāṭikaṅkhā, no parihāni.

‘‘Yāvakīvañca, bhikkhave, bhikkhū uppannāya taṇhāya ponobbhavikāya na vasaṃ gacchissanti, vuddhiyeva, bhikkhave, bhikkhūnaṃ pāṭikaṅkhā, no parihāni.

‘‘Yāvakīvañca, bhikkhave, bhikkhū āraññakesu senāsanesu sāpekkhā bhavissanti, vuddhiyeva, bhikkhave, bhikkhūnaṃ pāṭikaṅkhā, no parihāni.

‘‘Yāvakīvañca, bhikkhave, bhikkhū paccattaññeva satiṃ upaṭṭhapessanti – ‘kinti anāgatā ca pesalā sabrahmacārī āgaccheyyuṃ, āgatā ca pesalā sabrahmacārī phāsu\footnote{phāsuṃ (sī. syā. pī.)} vihareyyu’nti. Vuddhiyeva, bhikkhave, bhikkhūnaṃ pāṭikaṅkhā, no parihāni.

‘‘Yāvakīvañca, bhikkhave, ime satta aparihāniyā dhammā bhikkhūsu ṭhassanti, imesu ca sattasu aparihāniyesu dhammesu bhikkhū sandississanti, vuddhiyeva, bhikkhave, bhikkhūnaṃ pāṭikaṅkhā, no parihāni.

\paragraph{137.} ‘‘Aparepi vo, bhikkhave, satta aparihāniye dhamme desessāmi, taṃ suṇātha, sādhukaṃ manasikarotha, bhāsissāmī’’ti. ‘‘Evaṃ, bhante’’ti kho te bhikkhū bhagavato paccassosuṃ. Bhagavā etadavoca –

‘‘Yāvakīvañca, bhikkhave, bhikkhū na kammārāmā bhavissanti na kammaratā na kammārāmatamanuyuttā, vuddhiyeva, bhikkhave, bhikkhūnaṃ pāṭikaṅkhā, no parihāni.

‘‘Yāvakīvañca, bhikkhave, bhikkhū na bhassārāmā bhavissanti na bhassaratā na bhassārāmatamanuyuttā, vuddhiyeva, bhikkhave, bhikkhūnaṃ pāṭikaṅkhā, no parihāni.

‘‘Yāvakīvañca, bhikkhave, bhikkhū na niddārāmā bhavissanti na niddāratā na niddārāmatamanuyuttā, vuddhiyeva, bhikkhave, bhikkhūnaṃ pāṭikaṅkhā, no parihāni.

‘‘Yāvakīvañca, bhikkhave, bhikkhū na saṅgaṇikārāmā bhavissanti na saṅgaṇikaratā na saṅgaṇikārāmatamanuyuttā, vuddhiyeva, bhikkhave, bhikkhūnaṃ pāṭikaṅkhā, no parihāni.

‘‘Yāvakīvañca, bhikkhave, bhikkhū na pāpicchā bhavissanti na pāpikānaṃ icchānaṃ vasaṃ gatā, vuddhiyeva, bhikkhave, bhikkhūnaṃ pāṭikaṅkhā, no parihāni.

‘‘Yāvakīvañca, bhikkhave, bhikkhū na pāpamittā bhavissanti na pāpasahāyā na pāpasampavaṅkā, vuddhiyeva, bhikkhave, bhikkhūnaṃ pāṭikaṅkhā, no parihāni.

‘‘Yāvakīvañca, bhikkhave, bhikkhū na oramattakena visesādhigamena antarāvosānaṃ āpajjissanti, vuddhiyeva, bhikkhave, bhikkhūnaṃ pāṭikaṅkhā, no parihāni.

‘‘Yāvakīvañca, bhikkhave, ime satta aparihāniyā dhammā bhikkhūsu ṭhassanti, imesu ca sattasu aparihāniyesu dhammesu bhikkhū sandississanti, vuddhiyeva, bhikkhave, bhikkhūnaṃ pāṭikaṅkhā, no parihāni.

\paragraph{138.} ‘‘Aparepi vo, bhikkhave, satta aparihāniye dhamme desessāmi…pe… ‘‘yāvakīvañca, bhikkhave, bhikkhū saddhā bhavissanti…pe… hirimanā bhavissanti… ottappī bhavissanti… bahussutā bhavissanti… āraddhavīriyā bhavissanti… upaṭṭhitassatī bhavissanti… paññavanto bhavissanti, vuddhiyeva, bhikkhave, bhikkhūnaṃ pāṭikaṅkhā, no parihāni. Yāvakīvañca, bhikkhave, ime satta aparihāniyā dhammā bhikkhūsu ṭhassanti, imesu ca sattasu aparihāniyesu dhammesu bhikkhū sandississanti, vuddhiyeva, bhikkhave, bhikkhūnaṃ pāṭikaṅkhā, no parihāni.

\paragraph{139.} ‘‘Aparepi vo, bhikkhave, satta aparihāniye dhamme desessāmi, taṃ suṇātha, sādhukaṃ manasikarotha, bhāsissāmī’’ti. ‘‘Evaṃ, bhante’’ti kho te bhikkhū bhagavato paccassosuṃ. Bhagavā etadavoca –

‘‘Yāvakīvañca, bhikkhave, bhikkhu satisambojjhaṅgaṃ bhāvessanti…pe… dhammavicayasambojjhaṅgaṃ bhāvessanti… vīriyasambojjhaṅgaṃ bhāvessanti… pītisambojjhaṅgaṃ bhāvessanti… passaddhisambojjhaṅgaṃ bhāvessanti… samādhisambojjhaṅgaṃ bhāvessanti… upekkhāsambojjhaṅgaṃ bhāvessanti, vuddhiyeva , bhikkhave, bhikkhūnaṃ pāṭikaṅkhā, no parihāni.

‘‘Yāvakīvañca, bhikkhave, ime satta aparihāniyā dhammā bhikkhūsu ṭhassanti, imesu ca sattasu aparihāniyesu dhammesu bhikkhū sandississanti, vuddhiyeva, bhikkhave, bhikkhūnaṃ pāṭikaṅkhā no parihāni.

\paragraph{140.} ‘‘Aparepi vo, bhikkhave, satta aparihāniye dhamme desessāmi, taṃ suṇātha, sādhukaṃ manasikarotha, bhāsissāmī’’ti. ‘‘Evaṃ, bhante’’ti kho te bhikkhū bhagavato paccassosuṃ. Bhagavā etadavoca –

‘‘Yāvakīvañca, bhikkhave, bhikkhū aniccasaññaṃ bhāvessanti…pe… anattasaññaṃ bhāvessanti… asubhasaññaṃ bhāvessanti… ādīnavasaññaṃ bhāvessanti… pahānasaññaṃ bhāvessanti… virāgasaññaṃ bhāvessanti… nirodhasaññaṃ bhāvessanti, vuddhiyeva, bhikkhave, bhikkhūnaṃ pāṭikaṅkhā, no parihāni.

‘‘Yāvakīvañca , bhikkhave, ime satta aparihāniyā dhammā bhikkhūsu ṭhassanti, imesu ca sattasu aparihāniyesu dhammesu bhikkhū sandississanti, vuddhiyeva, bhikkhave, bhikkhūnaṃ pāṭikaṅkhā, no parihāni.

\paragraph{141.} ‘‘Cha, vo bhikkhave, aparihāniye dhamme desessāmi, taṃ suṇātha, sādhukaṃ manasikarotha, bhāsissāmī’’ti. ‘‘Evaṃ, bhante’’ti kho te bhikkhū bhagavato paccassosuṃ. Bhagavā etadavoca –

‘‘Yāvakīvañca , bhikkhave, bhikkhū mettaṃ kāyakammaṃ paccupaṭṭhāpessanti sabrahmacārīsu āvi ceva raho ca, vuddhiyeva, bhikkhave, bhikkhūnaṃ pāṭikaṅkhā, no parihāni.

‘‘Yāvakīvañca, bhikkhave, bhikkhū mettaṃ vacīkammaṃ paccupaṭṭhāpessanti …pe… mettaṃ manokammaṃ paccupaṭṭhāpessanti sabrahmacārīsu āvi ceva raho ca, vuddhiyeva, bhikkhave, bhikkhūnaṃ pāṭikaṅkhā, no parihāni.

‘‘Yāvakīvañca, bhikkhave, bhikkhū, ye te lābhā dhammikā dhammaladdhā antamaso pattapariyāpannamattampi tathārūpehi lābhehi appaṭivibhattabhogī bhavissanti sīlavantehi sabrahmacārīhi sādhāraṇabhogī, vuddhiyeva, bhikkhave, bhikkhūnaṃ pāṭikaṅkhā, no parihāni.

‘‘Yāvakīvañca, bhikkhave, bhikkhū yāni kāni sīlāni akhaṇḍāni acchiddāni asabalāni akammāsāni bhujissāni viññūpasatthāni\footnote{viññuppasatthāni (sī.)} aparāmaṭṭhāni samādhisaṃvattanikāni tathārūpesu sīlesu sīlasāmaññagatā viharissanti sabrahmacārīhi āvi ceva raho ca, vuddhiyeva, bhikkhave, bhikkhūnaṃ pāṭikaṅkhā, no parihāni.

‘‘Yāvakīvañca, bhikkhave, bhikkhū yāyaṃ diṭṭhi ariyā niyyānikā, niyyāti takkarassa sammā dukkhakkhayāya, tathārūpāya diṭṭhiyā diṭṭhisāmaññagatā viharissanti sabrahmacārīhi āvi ceva raho ca, vuddhiyeva, bhikkhave, bhikkhūnaṃ pāṭikaṅkhā, no parihāni.

‘‘Yāvakīvañca , bhikkhave, ime cha aparihāniyā dhammā bhikkhūsu ṭhassanti, imesu ca chasu aparihāniyesu dhammesu bhikkhū sandississanti, vuddhiyeva, bhikkhave, bhikkhūnaṃ pāṭikaṅkhā, no parihānī’’ti.

\paragraph{142.} Tatra sudaṃ bhagavā rājagahe viharanto gijjhakūṭe pabbate etadeva bahulaṃ bhikkhūnaṃ dhammiṃ kathaṃ karoti – ‘‘iti sīlaṃ, iti samādhi, iti paññā. Sīlaparibhāvito samādhi mahapphalo hoti mahānisaṃso. Samādhiparibhāvitā paññā mahapphalā hoti mahānisaṃsā. Paññāparibhāvitaṃ cittaṃ sammadeva āsavehi vimuccati, seyyathidaṃ – kāmāsavā, bhavāsavā, avijjāsavā’’ti.

\paragraph{143.} Atha kho bhagavā rājagahe yathābhirantaṃ viharitvā āyasmantaṃ ānandaṃ āmantesi – ‘‘āyāmānanda, yena ambalaṭṭhikā tenupasaṅkamissāmā’’ti. ‘‘Evaṃ, bhante’’ti kho āyasmā ānando bhagavato paccassosi. Atha kho bhagavā mahatā bhikkhusaṅghena saddhiṃ yena ambalaṭṭhikā tadavasari. Tatra sudaṃ bhagavā ambalaṭṭhikāyaṃ viharati rājāgārake. Tatrāpi sudaṃ bhagavā ambalaṭṭhikāyaṃ viharanto rājāgārake etadeva bahulaṃ bhikkhūnaṃ dhammiṃ kathaṃ karoti – ‘‘iti sīlaṃ iti samādhi iti paññā. Sīlaparibhāvito samādhi mahapphalo hoti mahānisaṃso. Samādhiparibhāvitā paññā mahapphalā hoti mahānisaṃsā. Paññāparibhāvitaṃ cittaṃ sammadeva āsavehi vimuccati, seyyathidaṃ – kāmāsavā, bhavāsavā, avijjāsavā’’ti.

\paragraph{144.} Atha kho bhagavā ambalaṭṭhikāyaṃ yathābhirantaṃ viharitvā āyasmantaṃ ānandaṃ āmantesi – ‘‘āyāmānanda, yena nāḷandā tenupasaṅkamissāmā’’ti. ‘‘Evaṃ, bhante’’ti kho āyasmā ānando bhagavato paccassosi. Atha kho bhagavā mahatā bhikkhusaṅghena saddhiṃ yena nāḷandā tadavasari, tatra sudaṃ bhagavā nāḷandāyaṃ viharati pāvārikambavane .

\subsubsection{Sāriputtasīhanādo}

\paragraph{145.} Atha kho āyasmā sāriputto yena bhagavā tenupasaṅkami; upasaṅkamitvā bhagavantaṃ abhivādetvā ekamantaṃ nisīdi. Ekamantaṃ nisinno kho āyasmā sāriputto bhagavantaṃ etadavoca – ‘‘evaṃ pasanno ahaṃ, bhante, bhagavati; na cāhu na ca bhavissati na cetarahi vijjati añño samaṇo vā brāhmaṇo vā bhagavatā bhiyyobhiññataro yadidaṃ sambodhiya’’nti. ‘‘Uḷārā kho te ayaṃ, sāriputta, āsabhī vācā\footnote{āsabhivācā (syā.)} bhāsitā, ekaṃso gahito, sīhanādo nadito – ‘evaṃpasanno ahaṃ, bhante, bhagavati; na cāhu na ca bhavissati na cetarahi vijjati añño samaṇo vā brāhmaṇo vā bhagavatā bhiyyobhiññataro yadidaṃ sambodhiya’nti.

‘‘Kiṃ te\footnote{kiṃ nu (syā. pī. ka.)}, sāriputta, ye te ahesuṃ atītamaddhānaṃ arahanto sammāsambuddhā, sabbe te bhagavanto cetasā ceto paricca viditā – ‘evaṃsīlā te bhagavanto ahesuṃ itipi, evaṃdhammā evaṃpaññā evaṃvihārī evaṃvimuttā te bhagavanto ahesuṃ itipī’’’ti? ‘‘No hetaṃ, bhante’’.

‘‘Kiṃ pana te\footnote{kiṃ pana (syā. pī. ka.)}, sāriputta, ye te bhavissanti anāgatamaddhānaṃ arahanto sammāsambuddhā, sabbe te bhagavanto cetasā ceto paricca viditā – ‘evaṃsīlā te bhagavanto bhavissanti itipi, evaṃdhammā evaṃpaññā evaṃvihārī evaṃvimuttā te bhagavanto bhavissanti itipī’’’ti? ‘‘No hetaṃ, bhante’’.

‘‘Kiṃ pana te, sāriputta, ahaṃ etarahi arahaṃ sammāsambuddho cetasā ceto paricca vidito – ‘‘evaṃsīlo bhagavā itipi , evaṃdhammo evaṃpañño evaṃvihārī evaṃvimutto bhagavā itipī’’’ti? ‘‘No hetaṃ, bhante’’.

‘‘Ettha ca hi te, sāriputta, atītānāgatapaccuppannesu arahantesu sammāsambuddhesu cetopariyañāṇaṃ\footnote{cetopariññāyañāṇaṃ (syā.), cetasā cetopariyāyañāṇaṃ (ka.)} natthi. Atha kiñcarahi te ayaṃ, sāriputta, uḷārā āsabhī vācā bhāsitā, ekaṃso gahito, sīhanādo nadito – ‘evaṃpasanno ahaṃ, bhante, bhagavati; na cāhu na ca bhavissati na cetarahi vijjati añño samaṇo vā brāhmaṇo vā bhagavatā bhiyyobhiññataro yadidaṃ sambodhiya’’’nti?

\paragraph{146.} ‘‘Na kho me, bhante, atītānāgatapaccuppannesu arahantesu sammāsambuddhesu cetopariyañāṇaṃ atthi, api ca me dhammanvayo vidito. Seyyathāpi, bhante, rañño paccantimaṃ nagaraṃ daḷhuddhāpaṃ daḷhapākāratoraṇaṃ ekadvāraṃ, tatrassa dovāriko paṇḍito viyatto medhāvī aññātānaṃ nivāretā ñātānaṃ pavesetā. So tassa nagarassa samantā anupariyāyapathaṃ\footnote{anucariyāyapathaṃ (syā.)} anukkamamāno na passeyya pākārasandhiṃ vā pākāravivaraṃ vā, antamaso biḷāranikkhamanamattampi. Tassa evamassa\footnote{na passeyya tassa evamassa (syā.)} – ‘ye kho keci oḷārikā pāṇā imaṃ nagaraṃ pavisanti vā nikkhamanti vā, sabbe te imināva dvārena pavisanti vā nikkhamanti vā’ti. Evameva kho me, bhante, dhammanvayo vidito – ‘ye te, bhante, ahesuṃ atītamaddhānaṃ arahanto sammāsambuddhā , sabbe te bhagavanto pañca nīvaraṇe pahāya cetaso upakkilese paññāya dubbalīkaraṇe catūsu satipaṭṭhānesu supatiṭṭhitacittā sattabojjhaṅge yathābhūtaṃ bhāvetvā anuttaraṃ sammāsambodhiṃ abhisambujjhiṃsu. Yepi te, bhante, bhavissanti anāgatamaddhānaṃ arahanto sammāsambuddhā , sabbe te bhagavanto pañca nīvaraṇe pahāya cetaso upakkilese paññāya dubbalīkaraṇe catūsu satipaṭṭhānesu supatiṭṭhitacittā satta bojjhaṅge yathābhūtaṃ bhāvetvā anuttaraṃ sammāsambodhiṃ abhisambujjhissanti. Bhagavāpi, bhante, etarahi arahaṃ sammāsambuddho pañca nīvaraṇe pahāya cetaso upakkilese paññāya dubbalīkaraṇe catūsu satipaṭṭhānesu supatiṭṭhitacitto satta bojjhaṅge yathābhūtaṃ bhāvetvā anuttaraṃ sammāsambodhiṃ abhisambuddho’’’ti.

\paragraph{147.} Tatrapi sudaṃ bhagavā nāḷandāyaṃ viharanto pāvārikambavane etadeva bahulaṃ bhikkhūnaṃ dhammiṃ kathaṃ karoti – ‘‘iti sīlaṃ, iti samādhi, iti paññā. Sīlaparibhāvito samādhi mahapphalo hoti mahānisaṃso. Samādhiparibhāvitā paññā mahapphalā hoti mahānisaṃsā. Paññāparibhāvitaṃ cittaṃ sammadeva āsavehi vimuccati, seyyathidaṃ – kāmāsavā, bhavāsavā, avijjāsavā’’ti.

\subsubsection{Dussīlaādīnavā}

\paragraph{148.} Atha kho bhagavā nāḷandāyaṃ yathābhirantaṃ viharitvā āyasmantaṃ ānandaṃ āmantesi – ‘‘āyāmānanda, yena pāṭaligāmo tenupasaṅkamissāmā’’ti. ‘‘Evaṃ, bhante’’ti kho āyasmā ānando bhagavato paccassosi . Atha kho bhagavā mahatā bhikkhusaṅghena saddhiṃ yena pāṭaligāmo tadavasari. Assosuṃ kho pāṭaligāmikā upāsakā – ‘‘bhagavā kira pāṭaligāmaṃ anuppatto’’ti. Atha kho pāṭaligāmikā upāsakā yena bhagavā tenupasaṅkamiṃsu; upasaṅkamitvā bhagavantaṃ abhivādetvā ekamantaṃ nisīdiṃsu. Ekamantaṃ nisinnā kho pāṭaligāmikā upāsakā bhagavantaṃ etadavocuṃ – ‘‘adhivāsetu no, bhante, bhagavā āvasathāgāra’’nti. Adhivāsesi bhagavā tuṇhībhāvena. Atha kho pāṭaligāmikā upāsakā bhagavato adhivāsanaṃ viditvā uṭṭhāyāsanā bhagavantaṃ abhivādetvā padakkhiṇaṃ katvā yena āvasathāgāraṃ tenupasaṅkamiṃsu; upasaṅkamitvā sabbasanthariṃ\footnote{sabbasantharitaṃ satthataṃ (syā.), sabbasanthariṃ santhataṃ (ka.)} āvasathāgāraṃ santharitvā āsanāni paññapetvā udakamaṇikaṃ patiṭṭhāpetvā telapadīpaṃ āropetvā yena bhagavā tenupasaṅkamiṃsu, upasaṅkamitvā bhagavantaṃ abhivādetvā ekamantaṃ aṭṭhaṃsu. Ekamantaṃ ṭhitā kho pāṭaligāmikā upāsakā bhagavantaṃ etadavocuṃ – ‘‘sabbasantharisanthataṃ\footnote{sabbasanthariṃ santhataṃ (sī. syā. pī. ka.)}, bhante, āvasathāgāraṃ, āsanāni paññattāni, udakamaṇiko patiṭṭhāpito, telapadīpo āropito; yassadāni, bhante, bhagavā kālaṃ maññatī’’ti. Atha kho bhagavā sāyanhasamayaṃ\footnote{idaṃ padaṃ vinayamahāvagga na dissati}. Nivāsetvā pattacīvaramādāya saddhiṃ bhikkhusaṅghena yena āvasathāgāraṃ tenupasaṅkami; upasaṅkamitvā pāde pakkhāletvā āvasathāgāraṃ pavisitvā majjhimaṃ thambhaṃ nissāya puratthābhimukho\footnote{puratthimābhimukho (ka.)} nisīdi. Bhikkhusaṅghopi kho pāde pakkhāletvā āvasathāgāraṃ pavisitvā pacchimaṃ bhittiṃ nissāya puratthābhimukho nisīdi bhagavantameva purakkhatvā. Pāṭaligāmikāpi kho upāsakā pāde pakkhāletvā āvasathāgāraṃ pavisitvā puratthimaṃ bhittiṃ nissāya pacchimābhimukhā nisīdiṃsu bhagavantameva purakkhatvā.

\paragraph{149.} Atha kho bhagavā pāṭaligāmike upāsake āmantesi – ‘‘pañcime, gahapatayo, ādīnavā dussīlassa sīlavipattiyā. Katame pañca? Idha, gahapatayo, dussīlo sīlavipanno pamādādhikaraṇaṃ mahatiṃ bhogajāniṃ nigacchati. Ayaṃ paṭhamo ādīnavo dussīlassa sīlavipattiyā.

‘‘Puna caparaṃ, gahapatayo, dussīlassa sīlavipannassa pāpako kittisaddo abbhuggacchati. Ayaṃ dutiyo ādīnavo dussīlassa sīlavipattiyā.

‘‘Puna caparaṃ, gahapatayo, dussīlo sīlavipanno yaññadeva parisaṃ upasaṅkamati – yadi khattiyaparisaṃ yadi brāhmaṇaparisaṃ yadi gahapatiparisaṃ yadi samaṇaparisaṃ – avisārado upasaṅkamati maṅkubhūto. Ayaṃ tatiyo ādīnavo dussīlassa sīlavipattiyā.

‘‘Puna caparaṃ, gahapatayo, dussīlo sīlavipanno sammūḷho kālaṅkaroti. Ayaṃ catuttho ādīnavo dussīlassa sīlavipattiyā.

‘‘Puna caparaṃ, gahapatayo, dussīlo sīlavipanno kāyassa bhedā paraṃ maraṇā apāyaṃ duggatiṃ vinipātaṃ nirayaṃ upapajjati. Ayaṃ pañcamo ādīnavo dussīlassa sīlavipattiyā. Ime kho, gahapatayo, pañca ādīnavā dussīlassa sīlavipattiyā.

\subsubsection{Sīlavanttaānisaṃsā}

\paragraph{150.} ‘‘Pañcime , gahapatayo, ānisaṃsā sīlavato sīlasampadāya. Katame pañca? Idha, gahapatayo, sīlavā sīlasampanno appamādādhikaraṇaṃ mahantaṃ bhogakkhandhaṃ adhigacchati. Ayaṃ paṭhamo ānisaṃso sīlavato sīlasampadāya.

‘‘Puna caparaṃ, gahapatayo, sīlavato sīlasampannassa kalyāṇo kittisaddo abbhuggacchati. Ayaṃ dutiyo ānisaṃso sīlavato sīlasampadāya.

‘‘Puna caparaṃ, gahapatayo, sīlavā sīlasampanno yaññadeva parisaṃ upasaṅkamati – yadi khattiyaparisaṃ yadi brāhmaṇaparisaṃ yadi gahapatiparisaṃ yadi samaṇaparisaṃ visārado upasaṅkamati amaṅkubhūto. Ayaṃ tatiyo ānisaṃso sīlavato sīlasampadāya.

‘‘Puna caparaṃ, gahapatayo, sīlavā sīlasampanno asammūḷho kālaṅkaroti. Ayaṃ catuttho ānisaṃso sīlavato sīlasampadāya.

‘‘Puna caparaṃ, gahapatayo, sīlavā sīlasampanno kāyassa bhedā paraṃ maraṇā sugatiṃ saggaṃ lokaṃ upapajjati. Ayaṃ pañcamo ānisaṃso sīlavato sīlasampadāya. Ime kho, gahapatayo, pañca ānisaṃsā sīlavato sīlasampadāyā’’ti.

\paragraph{151.} Atha kho bhagavā pāṭaligāmike upāsake bahudeva rattiṃ dhammiyā kathāya sandassetvā samādapetvā samuttejetvā sampahaṃsetvā uyyojesi – ‘‘abhikkantā kho, gahapatayo, ratti, yassadāni tumhe kālaṃ maññathā’’ti. ‘‘Evaṃ, bhante’’ti kho pāṭaligāmikā upāsakā bhagavato paṭissutvā uṭṭhāyāsanā bhagavantaṃ abhivādetvā padakkhiṇaṃ katvā pakkamiṃsu. Atha kho bhagavā acirapakkantesu pāṭaligāmikesu upāsakesu suññāgāraṃ pāvisi.

\subsubsection{Pāṭaliputtanagaramāpanaṃ}

\paragraph{152.} Tena kho pana samayena sunidhavassakārā\footnote{sunīdhavassakārā (syā. ka.)} magadhamahāmattā pāṭaligāme nagaraṃ māpenti vajjīnaṃ paṭibāhāya. Tena samayena sambahulā devatāyo sahasseva\footnote{sahassasseva (sī. pī. ka.), sahassaseva (ṭīkāyaṃ pāṭhantaraṃ), sahassasahasseva (udānaṭṭhakathā)} pāṭaligāme vatthūni pariggaṇhanti. Yasmiṃ padese mahesakkhā devatā vatthūni pariggaṇhanti, mahesakkhānaṃ tattha raññaṃ rājamahāmattānaṃ cittāni namanti nivesanāni māpetuṃ. Yasmiṃ padese majjhimā devatā vatthūni pariggaṇhanti, majjhimānaṃ tattha raññaṃ rājamahāmattānaṃ cittāni namanti nivesanāni māpetuṃ. Yasmiṃ padese nīcā devatā vatthūni pariggaṇhanti, nīcānaṃ tattha raññaṃ rājamahāmattānaṃ cittāni namanti nivesanāni māpetuṃ. Addasā kho bhagavā dibbena cakkhunā visuddhena atikkantamānusakena tā devatāyo sahasseva pāṭaligāme vatthūni pariggaṇhantiyo. Atha kho bhagavā rattiyā paccūsasamayaṃ paccuṭṭhāya āyasmantaṃ ānandaṃ āmantesi – ‘‘ke nu kho\footnote{ko nu kho (sī. syā. pī. ka.)}, ānanda, pāṭaligāme nagaraṃ māpentī’’ti\footnote{māpetīti (sī. syā. pī. ka.)}? ‘‘Sunidhavassakārā, bhante, magadhamahāmattā pāṭaligāme nagaraṃ māpenti vajjīnaṃ paṭibāhāyā’’ti. ‘‘Seyyathāpi, ānanda, devehi tāvatiṃsehi saddhiṃ mantetvā, evameva kho, ānanda, sunidhavassakārā magadhamahāmattā pāṭaligāme nagaraṃ māpenti vajjīnaṃ paṭibāhāya. Idhāhaṃ, ānanda, addasaṃ dibbena cakkhunā visuddhena atikkantamānusakena sambahulā devatāyo sahasseva pāṭaligāme vatthūni pariggaṇhantiyo. Yasmiṃ , ānanda, padese mahesakkhā devatā vatthūni pariggaṇhanti, mahesakkhānaṃ tattha raññaṃ rājamahāmattānaṃ cittāni namanti nivesanāni māpetuṃ. Yasmiṃ padese majjhimā devatā vatthūni pariggaṇhanti, majjhimānaṃ tattha raññaṃ rājamahāmattānaṃ cittāni namanti nivesanāni māpetuṃ. Yasmiṃ padese nīcā devatā vatthūni pariggaṇhanti, nīcānaṃ tattha raññaṃ rājamahāmattānaṃ cittāni namanti nivesanāni māpetuṃ. Yāvatā, ānanda, ariyaṃ āyatanaṃ yāvatā vaṇippatho idaṃ agganagaraṃ bhavissati pāṭaliputtaṃ puṭabhedanaṃ . Pāṭaliputtassa kho, ānanda, tayo antarāyā bhavissanti – aggito vā udakato vā mithubhedā vā’’ti.

\paragraph{153.} Atha kho sunidhavassakārā magadhamahāmattā yena bhagavā tenupasaṅkamiṃsu; upasaṅkamitvā bhagavatā saddhiṃ sammodiṃsu, sammodanīyaṃ kathaṃ sāraṇīyaṃ vītisāretvā ekamantaṃ aṭṭhaṃsu, ekamantaṃ ṭhitā kho sunidhavassakārā magadhamahāmattā bhagavantaṃ etadavocuṃ – ‘‘adhivāsetu no bhavaṃ gotamo ajjatanāya bhattaṃ saddhiṃ bhikkhusaṅghenā’’ti. Adhivāsesi bhagavā tuṇhībhāvena. Atha kho sunidhavassakārā magadhamahāmattā bhagavato adhivāsanaṃ viditvā yena sako āvasatho tenupasaṅkamiṃsu; upasaṅkamitvā sake āvasathe paṇītaṃ khādanīyaṃ bhojanīyaṃ paṭiyādāpetvā bhagavato kālaṃ ārocāpesuṃ – ‘‘kālo, bho gotama, niṭṭhitaṃ bhatta’’nti.

Atha kho bhagavā pubbaṇhasamayaṃ nivāsetvā pattacīvaramādāya saddhiṃ bhikkhusaṅghena yena sunidhavassakārānaṃ magadhamahāmattānaṃ āvasatho tenupasaṅkami; upasaṅkamitvā paññatte āsane nisīdi. Atha kho sunidhavassakārā magadhamahāmattā buddhappamukhaṃ bhikkhusaṅghaṃ paṇītena khādanīyena bhojanīyena sahatthā santappesuṃ sampavāresuṃ. Atha kho sunidhavassakārā magadhamahāmattā bhagavantaṃ bhuttāviṃ onītapattapāṇiṃ aññataraṃ nīcaṃ āsanaṃ gahetvā ekamantaṃ nisīdiṃsu. Ekamantaṃ nisinne kho sunidhavassakāre magadhamahāmatte bhagavā imāhi gāthāhi anumodi –

‘‘Yasmiṃ padese kappeti, vāsaṃ paṇḍitajātiyo;

Sīlavantettha bhojetvā, saññate brahmacārayo\footnote{brahmacārino (syā.)}.

‘‘Yā tattha devatā āsuṃ, tāsaṃ dakkhiṇamādise;

Tā pūjitā pūjayanti\footnote{pūjitā pūjayanti naṃ (ka.)}, mānitā mānayanti naṃ.

‘‘Tato naṃ anukampanti, mātā puttaṃva orasaṃ;

Devatānukampito poso, sadā bhadrāni passatī’’ti.

Atha kho bhagavā sunidhavassakāre magadhamahāmatte imāhi gāthāhi anumoditvā uṭṭhāyāsanā pakkāmi.

\paragraph{154.} Tena kho pana samayena sunidhavassakārā magadhamahāmattā bhagavantaṃ piṭṭhito piṭṭhito anubandhā honti – ‘‘yenajja samaṇo gotamo dvārena nikkhamissati, taṃ gotamadvāraṃ nāma bhavissati. Yena titthena gaṅgaṃ nadiṃ tarissati, taṃ gotamatitthaṃ nāma bhavissatī’’ti. Atha kho bhagavā yena dvārena nikkhami , taṃ gotamadvāraṃ nāma ahosi. Atha kho bhagavā yena gaṅgā nadī tenupasaṅkami. Tena kho pana samayena gaṅgā nadī pūrā hoti samatittikā kākapeyyā. Appekacce manussā nāvaṃ pariyesanti, appekacce uḷumpaṃ pariyesanti, appekacce kullaṃ bandhanti apārā\footnote{pārā (sī. syā. ka.), orā (vi. mahāvagga)}, pāraṃ gantukāmā. Atha kho bhagavā – seyyathāpi nāma balavā puriso samiñjitaṃ vā bāhaṃ pasāreyya, pasāritaṃ vā bāhaṃ samiñjeyya, evameva – gaṅgāya nadiyā orimatīre antarahito pārimatīre paccuṭṭhāsi saddhiṃ bhikkhusaṅghena. Addasā kho bhagavā te manusse appekacce nāvaṃ pariyesante appekacce uḷumpaṃ pariyesante appekacce kullaṃ bandhante apārā pāraṃ gantukāme. Atha kho bhagavā etamatthaṃ viditvā tāyaṃ velāyaṃ imaṃ udānaṃ udānesi –

‘‘Ye taranti aṇṇavaṃ saraṃ, setuṃ katvāna visajja pallalāni;

Kullañhi jano bandhati\footnote{kullaṃ jano ca bandhati (syā.), kullaṃ hi jano pabandhati (sī. pī. ka.)}, tiṇṇā\footnote{nitiṇṇā, na tiṇṇā (ka.)} medhāvino janā’’ti.

\xsubsubsectionEnd{Paṭhamabhāṇavāro.}

\subsubsection{Ariyasaccakathā}

\paragraph{155.} Atha kho bhagavā āyasmantaṃ ānandaṃ āmantesi – ‘‘āyāmānanda, yena koṭigāmo tenupasaṅkamissāmā’’ti. ‘‘Evaṃ, bhante’’ti kho āyasmā ānando bhagavato paccassosi. Atha kho bhagavā mahatā bhikkhusaṅghena saddhiṃ yena koṭigāmo tadavasari. Tatra sudaṃ bhagavā koṭigāme viharati. Tatra kho bhagavā bhikkhū āmantesi –

‘‘Catunnaṃ , bhikkhave, ariyasaccānaṃ ananubodhā appaṭivedhā evamidaṃ dīghamaddhānaṃ sandhāvitaṃ saṃsaritaṃ mamañceva tumhākañca. Katamesaṃ catunnaṃ? Dukkhassa, bhikkhave, ariyasaccassa ananubodhā appaṭivedhā evamidaṃ dīghamaddhānaṃ sandhāvitaṃ saṃsaritaṃ mamañceva tumhākañca. Dukkhasamudayassa, bhikkhave, ariyasaccassa ananubodhā appaṭivedhā evamidaṃ dīghamaddhānaṃ sandhāvitaṃ saṃsaritaṃ mamañceva tumhākañca. Dukkhanirodhassa, bhikkhave, ariyasaccassa ananubodhā appaṭivedhā evamidaṃ dīghamaddhānaṃ sandhāvitaṃ saṃsaritaṃ mamañceva tumhākañca. Dukkhanirodhagāminiyā paṭipadāya, bhikkhave, ariyasaccassa ananubodhā appaṭivedhā evamidaṃ dīghamaddhānaṃ sandhāvitaṃ saṃsaritaṃ mamañceva tumhākañca. Tayidaṃ, bhikkhave, dukkhaṃ ariyasaccaṃ anubuddhaṃ paṭividdhaṃ, dukkhasamudayaṃ\footnote{dukkhasamudayo (syā.)} ariyasaccaṃ anubuddhaṃ paṭividdhaṃ, dukkhanirodhaṃ\footnote{dukkhanirodho (syā.)} ariyasaccaṃ anubuddhaṃ paṭividdhaṃ, dukkhanirodhagāminī paṭipadā ariyasaccaṃ anubuddhaṃ paṭividdhaṃ, ucchinnā bhavataṇhā, khīṇā bhavanetti, natthidāni punabbhavo’’ti. Idamavoca bhagavā. Idaṃ vatvāna sugato athāparaṃ etadavoca satthā –

‘‘Catunnaṃ ariyasaccānaṃ, yathābhūtaṃ adassanā;

Saṃsitaṃ dīghamaddhānaṃ, tāsu tāsveva jātisu.

Tāni etāni diṭṭhāni, bhavanetti samūhatā;

Ucchinnaṃ mūlaṃ dukkhassa, natthi dāni punabbhavo’’ti.

Tatrapi sudaṃ bhagavā koṭigāme viharanto etadeva bahulaṃ bhikkhūnaṃ dhammiṃ kathaṃ karoti – ‘‘iti sīlaṃ, iti samādhi, iti paññā. Sīlaparibhāvito samādhi mahapphalo hoti mahānisaṃso. Samādhiparibhāvitā paññā mahapphalā hoti mahānisaṃsā. Paññāparibhāvitaṃ cittaṃ sammadeva āsavehi vimuccati, seyyathidaṃ – kāmāsavā, bhavāsavā, avijjāsavā’’ti.

\subsubsection{Anāvattidhammasambodhiparāyaṇā}

\paragraph{156.} Atha kho bhagavā koṭigāme yathābhirantaṃ viharitvā āyasmantaṃ ānandaṃ āmantesi – ‘‘āyāmānanda, yena nātikā\footnote{nādikā (syā. pī.)} tenupaṅkamissāmā’’ti. ‘‘Evaṃ, bhante’’ti kho āyasmā ānando bhagavato paccassosi. Atha kho bhagavā mahatā bhikkhusaṅghena saddhiṃ yena nātikā tadavasari. Tatrapi sudaṃ bhagavā nātike viharati giñjakāvasathe. Atha kho āyasmā ānando yena bhagavā tenupasaṅkami; upasaṅkamitvā bhagavantaṃ abhivādetvā ekamantaṃ nisīdi. Ekamantaṃ nisinno kho āyasmā ānando bhagavantaṃ etadavoca – ‘‘sāḷho nāma, bhante, bhikkhu nātike kālaṅkato, tassa kā gati, ko abhisamparāyo? Nandā nāma, bhante, bhikkhunī nātike kālaṅkatā, tassā kā gati, ko abhisamparāyo? Sudatto nāma, bhante, upāsako nātike kālaṅkato, tassa kā gati, ko abhisamparāyo? Sujātā nāma, bhante, upāsikā nātike kālaṅkatā, tassā kā gati , ko abhisamparāyo? Kukkuṭo\footnote{kakudho (syā.)} nāma, bhante, upāsako nātike kālaṅkato, tassa kā gati, ko abhisamparāyo? Kāḷimbo\footnote{kāliṅgo (pī.), kāraḷimbo (syā.)} nāma, bhante, upāsako…pe… nikaṭo nāma, bhante, upāsako… kaṭissaho\footnote{kaṭissabho (sī. pī.)} nāma, bhante, upāsako… tuṭṭho nāma, bhante, upāsako… santuṭṭho nāma, bhante, upāsako… bhaddo\footnote{bhaṭo (syā.)} nāma, bhante, upāsako… subhaddo\footnote{subhaṭo (syā.)} nāma, bhante, upāsako nātike kālaṅkato, tassa kā gati, ko abhisamparāyo’’ti?

\paragraph{157.} ‘‘Sāḷho, ānanda, bhikkhu āsavānaṃ khayā anāsavaṃ cetovimuttiṃ paññāvimuttiṃ diṭṭheva dhamme sayaṃ abhiññā sacchikatvā upasampajja vihāsi. Nandā, ānanda, bhikkhunī pañcannaṃ orambhāgiyānaṃ saṃyojanānaṃ parikkhayā opapātikā tattha parinibbāyinī anāvattidhammā tasmā lokā. Sudatto, ānanda, upāsako tiṇṇaṃ saṃyojanānaṃ parikkhayā rāgadosamohānaṃ tanuttā sakadāgāmī sakideva imaṃ lokaṃ āgantvā dukkhassantaṃ karissati. Sujātā, ānanda, upāsikā tiṇṇaṃ saṃyojanānaṃ parikkhayā sotāpannā avinipātadhammā niyatā sambodhiparāyaṇā\footnote{parāyanā (sī. syā. pī. ka.)}. Kukkuṭo, ānanda, upāsako pañcannaṃ orambhāgiyānaṃ saṃyojanānaṃ parikkhayā opapātiko tattha parinibbāyī anāvattidhammo tasmā lokā. Kāḷimbo, ānanda, upāsako…pe… nikaṭo, ānanda, upāsako… kaṭissaho , ānanda, upāsako… tuṭṭho, ānanda, upāsako … santuṭṭho, ānanda, upāsako… bhaddo, ānanda, upāsako… subhaddo, ānanda, upāsako pañcannaṃ orambhāgiyānaṃ saṃyojanānaṃ parikkhayā opapātiko tattha parinibbāyī anāvattidhammo tasmā lokā . Paropaññāsaṃ, ānanda, nātike upāsakā kālaṅkatā, pañcannaṃ orambhāgiyānaṃ saṃyojanānaṃ parikkhayā opapātikā tattha parinibbāyino anāvattidhammā tasmā lokā. Sādhikā navuti\footnote{chādhikā navuti (syā.)}, ānanda, nātike upāsakā kālaṅkatā tiṇṇaṃ saṃyojanānaṃ parikkhayā rāgadosamohānaṃ tanuttā sakadāgāmino sakideva imaṃ lokaṃ āgantvā dukkhassantaṃ karissanti. Sātirekāni\footnote{dasātirekāni (syā.)}, ānanda, pañcasatāni nātike upāsakā kālaṅkatā, tiṇṇaṃ saṃyojanānaṃ parikkhayā sotāpannā avinipātadhammā niyatā sambodhiparāyaṇā.

\subsubsection{Dhammādāsadhammapariyāyā}

\paragraph{158.} ‘‘Anacchariyaṃ kho panetaṃ, ānanda, yaṃ manussabhūto kālaṅkareyya. Tasmiṃyeva\footnote{tasmiṃ tasmiṃ ce (sī. pī.), tasmiṃ tasmiṃ kho (syā.)} kālaṅkate tathāgataṃ upasaṅkamitvā etamatthaṃ pucchissatha, vihesā hesā, ānanda, tathāgatassa. Tasmātihānanda, dhammādāsaṃ nāma dhammapariyāyaṃ desessāmi, yena samannāgato ariyasāvako ākaṅkhamāno attanāva attānaṃ byākareyya – ‘khīṇanirayomhi khīṇatiracchānayoni khīṇapettivisayo khīṇāpāyaduggativinipāto, sotāpannohamasmi avinipātadhammo niyato sambodhiparāyaṇo’ti.

\paragraph{159.} ‘‘Katamo ca so, ānanda, dhammādāso dhammapariyāyo, yena samannāgato ariyasāvako ākaṅkhamāno attanāva attānaṃ byākareyya – ‘khīṇanirayomhi khīṇatiracchānayoni khīṇapettivisayo khīṇāpāyaduggativinipāto, sotāpannohamasmi avinipātadhammo niyato sambodhiparāyaṇo’ti?

‘‘Idhānanda , ariyasāvako buddhe aveccappasādena samannāgato hoti – ‘itipi so bhagavā arahaṃ sammāsambuddho vijjācaraṇasampanno sugato lokavidū anuttaro purisadammasārathi satthā devamanussānaṃ buddho bhagavā’ti.

‘‘Dhamme aveccappasādena samannāgato hoti – ‘svākkhāto bhagavatā dhammo sandiṭṭhiko akāliko ehipassiko opaneyyiko paccattaṃ veditabbo viññūhī’ti.

‘‘Saṅghe aveccappasādena samannāgato hoti – ‘suppaṭipanno bhagavato sāvakasaṅgho, ujuppaṭipanno bhagavato sāvakasaṅgho, ñāyappaṭipanno bhagavato sāvakasaṅgho, sāmīcippaṭipanno bhagavato sāvakasaṅgho yadidaṃ cattāri purisayugāni aṭṭha purisapuggalā, esa bhagavato sāvakasaṅgho āhuneyyo pāhuneyyo dakkhiṇeyyo añjalikaraṇīyo anuttaraṃ puññakkhettaṃ lokassā’ti.

‘‘Ariyakantehi sīlehi samannāgato hoti akhaṇḍehi acchiddehi asabalehi akammāsehi bhujissehi viññūpasatthehi aparāmaṭṭhehi samādhisaṃvattanikehi.

‘‘Ayaṃ kho so, ānanda, dhammādāso dhammapariyāyo, yena samannāgato ariyasāvako ākaṅkhamāno attanāva attānaṃ byākareyya – ‘khīṇanirayomhi khīṇatiracchānayoni khīṇapettivisayo khīṇāpāyaduggativinipāto, sotāpannohamasmi avinipātadhammo niyato sambodhiparāyaṇo’’’ti.

Tatrapi sudaṃ bhagavā nātike viharanto giñjakāvasathe etadeva bahulaṃ bhikkhūnaṃ dhammiṃ kathaṃ karoti –

‘‘Iti sīlaṃ iti samādhi iti paññā. Sīlaparibhāvito samādhi mahapphalo hoti mahānisaṃso. Samādhiparibhāvitā paññā mahapphalā hoti mahānisaṃsā. Paññāparibhāvitaṃ cittaṃ sammadeva āsavehi vimuccati, seyyathidaṃ – kāmāsavā, bhavāsavā, avijjāsavā’’ti.

\paragraph{160.} Atha kho bhagavā nātike yathābhirantaṃ viharitvā āyasmantaṃ ānandaṃ āmantesi – ‘‘āyāmānanda, yena vesālī tenupasaṅkamissāmā’’ti. ‘‘Evaṃ, bhante’’ti kho āyasmā ānando bhagavato paccassosi. Atha kho bhagavā mahatā bhikkhusaṅghena saddhiṃ yena vesālī tadavasari. Tatra sudaṃ bhagavā vesāliyaṃ viharati ambapālivane. Tatra kho bhagavā bhikkhū āmantesi –

‘‘Sato, bhikkhave, bhikkhu vihareyya sampajāno, ayaṃ vo amhākaṃ anusāsanī. Kathañca, bhikkhave, bhikkhu sato hoti? Idha, bhikkhave, bhikkhu kāye kāyānupassī viharati ātāpī sampajāno satimā vineyya loke abhijjhādomanassaṃ. Vedanāsu vedanānupassī…pe… citte cittānupassī…pe… dhammesu dhammānupassī viharati ātāpī sampajāno satimā vineyya loke abhijjhādomanassaṃ. Evaṃ kho, bhikkhave, bhikkhu sato hoti.

‘‘Kathañca , bhikkhave, bhikkhu sampajāno hoti? Idha, bhikkhave, bhikkhu abhikkante paṭikkante sampajānakārī hoti, ālokite vilokite sampajānakārī hoti, samiñjite pasārite sampajānakārī hoti, saṅghāṭipattacīvaradhāraṇe sampajānakārī hoti, asite pīte khāyite sāyite sampajānakārī hoti, uccārapassāvakamme sampajānakārī hoti, gate ṭhite nisinne sutte jāgarite bhāsite tuṇhībhāve sampajānakārī hoti. Evaṃ kho, bhikkhave, bhikkhu sampajāno hoti. Sato, bhikkhave, bhikkhu vihareyya sampajāno, ayaṃ vo amhākaṃ anusāsanī’’ti.

\subsubsection{Ambapālīgaṇikā}

\paragraph{161.} Assosi kho ambapālī gaṇikā – ‘‘bhagavā kira vesāliṃ anuppatto vesāliyaṃ viharati mayhaṃ ambavane’’ti. Atha kho ambapālī gaṇikā bhaddāni bhaddāni yānāni yojāpetvā bhaddaṃ bhaddaṃ yānaṃ abhiruhitvā bhaddehi bhaddehi yānehi vesāliyā niyyāsi. Yena sako ārāmo tena pāyāsi. Yāvatikā yānassa bhūmi, yānena gantvā, yānā paccorohitvā pattikāva yena bhagavā tenupasaṅkami; upasaṅkamitvā bhagavantaṃ abhivādetvā ekamantaṃ nisīdi. Ekamantaṃ nisinnaṃ kho ambapāliṃ gaṇikaṃ bhagavā dhammiyā kathāya sandassesi samādapesi samuttejesi sampahaṃsesi. Atha kho ambapālī gaṇikā bhagavatā dhammiyā kathāya sandassitā samādapitā samuttejitā sampahaṃsitā bhagavantaṃ etadavoca – ‘‘adhivāsetu me, bhante, bhagavā svātanāya bhattaṃ saddhiṃ bhikkhusaṅghenā’’ti. Adhivāsesi bhagavā tuṇhībhāvena. Atha kho ambapālī gaṇikā bhagavato adhivāsanaṃ viditvā uṭṭhāyāsanā bhagavantaṃ abhivādetvā padakkhiṇaṃ katvā pakkāmi.

Assosuṃ kho vesālikā licchavī – ‘‘bhagavā kira vesāliṃ anuppatto vesāliyaṃ viharati ambapālivane’’ti. Atha kho te licchavī bhaddāni bhaddāni yānāni yojāpetvā bhaddaṃ bhaddaṃ yānaṃ abhiruhitvā bhaddehi bhaddehi yānehi vesāliyā niyyiṃsu. Tatra ekacce licchavī nīlā honti nīlavaṇṇā nīlavatthā nīlālaṅkārā, ekacce licchavī pītā honti pītavaṇṇā pītavatthā pītālaṅkārā, ekacce licchavī lohitā honti lohitavaṇṇā lohitavatthā lohitālaṅkārā, ekacce licchavī odātā honti odātavaṇṇā odātavatthā odātālaṅkārā. Atha kho ambapālī gaṇikā daharānaṃ daharānaṃ licchavīnaṃ akkhena akkhaṃ cakkena cakkaṃ yugena yugaṃ paṭivaṭṭesi\footnote{parivattesi (vi. mahāvagga)}. Atha kho te licchavī ambapāliṃ gaṇikaṃ etadavocuṃ – ‘‘kiṃ, je ambapāli , daharānaṃ daharānaṃ licchavīnaṃ akkhena akkhaṃ cakkena cakkaṃ yugena yugaṃ paṭivaṭṭesī’’ti? ‘‘Tathā hi pana me, ayyaputtā, bhagavā nimantito svātanāya bhattaṃ saddhiṃ bhikkhusaṅghenā’’ti. ‘‘Dehi, je ambapāli, etaṃ\footnote{ekaṃ (ka.)} bhattaṃ satasahassenā’’ti. ‘‘Sacepi me, ayyaputtā, vesāliṃ sāhāraṃ dassatha\footnote{dajjeyyātha (vi. mahāvagga)}, evamahaṃ taṃ\footnote{evampi mahantaṃ (syā.), evaṃ mahantaṃ (sī. pī.)} bhattaṃ na dassāmī’’ti\footnote{neva dajjāhaṃ taṃ bhattanti (vi. mahāvagga)}. Atha kho te licchavī aṅguliṃ phoṭesuṃ – ‘‘jitamha\footnote{jitamhā (bahūsu)} vata bho ambakāya, jitamha vata bho ambakāyā’’ti\footnote{‘‘jitamhā vata bho ambapālikāya vañcitamhā vata bho ambapālikāyā’’ti (syā.)}.

Atha kho te licchavī yena ambapālivanaṃ tena pāyiṃsu. Addasā kho bhagavā te licchavī dūratova āgacchante. Disvāna bhikkhū āmantesi – ‘‘yesaṃ\footnote{yehi (vi. mahāvagga)}, bhikkhave, bhikkhūnaṃ devā tāvatiṃsā adiṭṭhapubbā, oloketha, bhikkhave, licchaviparisaṃ; apaloketha, bhikkhave , licchaviparisaṃ; upasaṃharatha, bhikkhave, licchaviparisaṃ – tāvatiṃsasadisa’’nti. Atha kho te licchavī yāvatikā yānassa bhūmi, yānena gantvā, yānā paccorohitvā pattikāva yena bhagavā tenupasaṅkamiṃsu; upasaṅkamitvā bhagavantaṃ abhivādetvā ekamantaṃ nisīdiṃsu. Ekamantaṃ nisinne kho te licchavī bhagavā dhammiyā kathāya sandassesi samādapesi samuttejesi sampahaṃsesi. Atha kho te licchavī bhagavatā dhammiyā kathāya sandassitā samādapitā samuttejitā sampahaṃsitā bhagavantaṃ etadavocuṃ – ‘‘adhivāsetu no, bhante, bhagavā svātanāya bhattaṃ saddhiṃ bhikkhusaṅghenā’’ti. Atha kho bhagavā te licchavī etadavoca – ‘‘adhivutthaṃ\footnote{adhivāsitaṃ (syā.)} kho me, licchavī, svātanāya ambapāliyā gaṇikāya bhatta’’nti. Atha kho te licchavī aṅguliṃ phoṭesuṃ – ‘‘jitamha vata bho ambakāya, jitamha vata bho ambakāyā’’ti. Atha kho te licchavī bhagavato bhāsitaṃ abhinanditvā anumoditvā uṭṭhāyāsanā bhagavantaṃ abhivādetvā padakkhiṇaṃ katvā pakkamiṃsu.

\paragraph{162.} Atha kho ambapālī gaṇikā tassā rattiyā accayena sake ārāme paṇītaṃ khādanīyaṃ bhojanīyaṃ paṭiyādāpetvā bhagavato kālaṃ ārocāpesi – ‘‘kālo, bhante, niṭṭhitaṃ bhatta’’nti. Atha kho bhagavā pubbaṇhasamayaṃ nivāsetvā pattacīvaramādāya saddhiṃ bhikkhusaṅghena yena ambapāliyā gaṇikāya nivesanaṃ tenupasaṅkami; upasaṅkamitvā paññatte āsane nisīdi. Atha kho ambapālī gaṇikā buddhappamukhaṃ bhikkhusaṅghaṃ paṇītena khādanīyena bhojanīyena sahatthā santappesi sampavāresi. Atha kho ambapālī gaṇikā bhagavantaṃ bhuttāviṃ onītapattapāṇiṃ aññataraṃ nīcaṃ āsanaṃ gahetvā ekamantaṃ nisīdi. Ekamantaṃ nisinnā kho ambapālī gaṇikā bhagavantaṃ etadavoca – ‘‘imāhaṃ, bhante, ārāmaṃ buddhappamukhassa bhikkhusaṅghassa dammī’’ti. Paṭiggahesi bhagavā ārāmaṃ. Atha kho bhagavā ambapāliṃ gaṇikaṃ dhammiyā kathāya sandassetvā samādapetvā samuttejetvā sampahaṃsetvā uṭṭhāyāsanā pakkāmi. Tatrapi sudaṃ bhagavā vesāliyaṃ viharanto ambapālivane etadeva bahulaṃ bhikkhūnaṃ dhammiṃ kathaṃ karoti – ‘‘iti sīlaṃ, iti samādhi, iti paññā. Sīlaparibhāvito samādhi mahapphalo hoti mahānisaṃso. Samādhiparibhāvitā paññā mahapphalā hoti mahānisaṃsā. Paññāparibhāvitaṃ cittaṃ sammadeva āsavehi vimuccati, seyyathidaṃ – kāmāsavā, bhavāsavā, avijjāsavā’’ti.

\subsubsection{Veḷuvagāmavassūpagamanaṃ}

\paragraph{163.} Atha kho bhagavā ambapālivane yathābhirantaṃ viharitvā āyasmantaṃ ānandaṃ āmantesi – ‘‘āyāmānanda, yena veḷuvagāmako\footnote{beḷuvagāmako (sī. pī.)} tenupasaṅkamissāmā’’ti. ‘‘Evaṃ, bhante’’ti kho āyasmā ānando bhagavato paccassosi. Atha kho bhagavā mahatā bhikkhusaṅghena saddhiṃ yena veḷuvagāmako tadavasari. Tatra sudaṃ bhagavā veḷuvagāmake viharati. Tatra kho bhagavā bhikkhū āmantesi – ‘‘etha tumhe, bhikkhave, samantā vesāliṃ yathāmittaṃ yathāsandiṭṭhaṃ yathāsambhattaṃ vassaṃ upetha\footnote{upagacchatha (syā.)}. Ahaṃ pana idheva veḷuvagāmake vassaṃ upagacchāmī’’ti. ‘‘Evaṃ, bhante’’ti kho te bhikkhū bhagavato paṭissutvā samantā vesāliṃ yathāmittaṃ yathāsandiṭṭhaṃ yathāsambhattaṃ vassaṃ upagacchiṃsu. Bhagavā pana tattheva veḷuvagāmake vassaṃ upagacchi.

\paragraph{164.} Atha kho bhagavato vassūpagatassa kharo ābādho uppajji, bāḷhā vedanā vattanti māraṇantikā. Tā sudaṃ bhagavā sato sampajāno adhivāsesi avihaññamāno. Atha kho bhagavato etadahosi – ‘‘na kho metaṃ patirūpaṃ, yvāhaṃ anāmantetvā upaṭṭhāke anapaloketvā bhikkhusaṅghaṃ parinibbāyeyyaṃ. Yaṃnūnāhaṃ imaṃ ābādhaṃ vīriyena paṭipaṇāmetvā jīvitasaṅkhāraṃ adhiṭṭhāya vihareyya’’nti. Atha kho bhagavā taṃ ābādhaṃ vīriyena paṭipaṇāmetvā jīvitasaṅkhāraṃ adhiṭṭhāya vihāsi. Atha kho bhagavato so ābādho paṭipassambhi. Atha kho bhagavā gilānā vuṭṭhito\footnote{gilānavuṭṭhito (saddanīti)} aciravuṭṭhito gelaññā vihārā nikkhamma vihārapacchāyāyaṃ paññatte āsane nisīdi. Atha kho āyasmā ānando yena bhagavā tenupasaṅkami; upasaṅkamitvā bhagavantaṃ abhivādetvā ekamantaṃ nisīdi. Ekamantaṃ nisinno kho āyasmā ānando bhagavantaṃ etadavoca – ‘‘diṭṭho me, bhante, bhagavato phāsu; diṭṭhaṃ me, bhante, bhagavato khamanīyaṃ, api ca me, bhante, madhurakajāto viya kāyo. Disāpi me na pakkhāyanti; dhammāpi maṃ na paṭibhanti bhagavato gelaññena, api ca me, bhante, ahosi kācideva assāsamattā – ‘na tāva bhagavā parinibbāyissati, na yāva bhagavā bhikkhusaṅghaṃ ārabbha kiñcideva udāharatī’’’ti.

\paragraph{165.} ‘‘Kiṃ panānanda, bhikkhusaṅgho mayi paccāsīsati\footnote{paccāsiṃsati (sī. syā.)}? Desito, ānanda, mayā dhammo anantaraṃ abāhiraṃ karitvā. Natthānanda, tathāgatassa dhammesu ācariyamuṭṭhi. Yassa nūna, ānanda, evamassa – ‘ahaṃ bhikkhusaṅghaṃ pariharissāmī’ti vā ‘mamuddesiko bhikkhusaṅgho’ti vā, so nūna, ānanda, bhikkhusaṅghaṃ ārabbha kiñcideva udāhareyya. Tathāgatassa kho, ānanda, na evaṃ hoti – ‘ahaṃ bhikkhusaṅghaṃ pariharissāmī’ti vā ‘mamuddesiko bhikkhusaṅgho’ti vā. Sakiṃ\footnote{kiṃ (sī. pī.)}, ānanda, tathāgato bhikkhusaṅghaṃ ārabbha kiñcideva udāharissati. Ahaṃ kho panānanda, etarahi jiṇṇo vuddho mahallako addhagato vayoanuppatto. Āsītiko me vayo vattati. Seyyathāpi, ānanda, jajjarasakaṭaṃ veṭhamissakena\footnote{veḷumissakena (syā.), veghamissakena (pī.), vedhamissakena, vekhamissakena (ka.)} yāpeti, evameva kho, ānanda, veṭhamissakena maññe tathāgatassa kāyo yāpeti. Yasmiṃ, ānanda, samaye tathāgato sabbanimittānaṃ amanasikārā ekaccānaṃ vedanānaṃ nirodhā animittaṃ cetosamādhiṃ upasampajja viharati, phāsutaro, ānanda, tasmiṃ samaye tathāgatassa kāyo hoti. Tasmātihānanda, attadīpā viharatha attasaraṇā anaññasaraṇā, dhammadīpā dhammasaraṇā anaññasaraṇā. Kathañcānanda, bhikkhu attadīpo viharati attasaraṇo anaññasaraṇo, dhammadīpo dhammasaraṇo anaññasaraṇo? Idhānanda, bhikkhu kāye kāyānupassī viharati atāpī sampajāno satimā, vineyya loke abhijjhādomanassaṃ. Vedanāsu…pe… citte…pe… dhammesu dhammānupassī viharati ātāpī sampajāno satimā, vineyya loke abhijjhādomanassaṃ. Evaṃ kho, ānanda, bhikkhu attadīpo viharati attasaraṇo anaññasaraṇo, dhammadīpo dhammasaraṇo anaññasaraṇo . Ye hi keci, ānanda, etarahi vā mama vā accayena attadīpā viharissanti attasaraṇā anaññasaraṇā, dhammadīpā dhammasaraṇā anaññasaraṇā, tamatagge me te, ānanda, bhikkhū bhavissanti ye keci sikkhākāmā’’ti.

\xsubsubsectionEnd{Dutiyabhāṇavāro.}

\subsubsection{Nimittobhāsakathā}

\paragraph{166.} Atha kho bhagavā pubbaṇhasamayaṃ nivāsetvā pattacīvaramādāya vesāliṃ piṇḍāya pāvisi. Vesāliyaṃ piṇḍāya caritvā pacchābhattaṃ piṇḍapātapaṭikkanto āyasmantaṃ ānandaṃ āmantesi – ‘‘gaṇhāhi, ānanda, nisīdanaṃ, yena cāpālaṃ cetiyaṃ\footnote{pāvālaṃ (cetiyaṃ (syā.)} tenupasaṅkamissāma divā vihārāyā’’ti. ‘‘Evaṃ, bhante’’ti kho āyasmā ānando bhagavato paṭissutvā nisīdanaṃ ādāya bhagavantaṃ piṭṭhito piṭṭhito anubandhi. Atha kho bhagavā yena cāpālaṃ cetiyaṃ tenupasaṅkami; upasaṅkamitvā paññatte āsane nisīdi. Āyasmāpi kho ānando bhagavantaṃ abhivādetvā ekamantaṃ nisīdi.

\paragraph{167.} Ekamantaṃ nisinnaṃ kho āyasmantaṃ ānandaṃ bhagavā etadavoca – ‘‘ramaṇīyā, ānanda, vesālī, ramaṇīyaṃ udenaṃ cetiyaṃ, ramaṇīyaṃ gotamakaṃ cetiyaṃ, ramaṇīyaṃ sattambaṃ\footnote{sattambakaṃ (pī.)} cetiyaṃ, ramaṇīyaṃ bahuputtaṃ cetiyaṃ, ramaṇīyaṃ sārandadaṃ cetiyaṃ, ramaṇīyaṃ cāpālaṃ cetiyaṃ. Yassa kassaci, ānanda, cattāro iddhipādā bhāvitā bahulīkatā yānīkatā vatthukatā anuṭṭhitā paricitā susamāraddhā, so ākaṅkhamāno kappaṃ vā tiṭṭheyya kappāvasesaṃ vā. Tathāgatassa kho, ānanda, cattāro iddhipādā bhāvitā bahulīkatā yānīkatā vatthukatā anuṭṭhitā paricitā susamāraddhā, so ākaṅkhamāno\footnote{ākaṅkhamāno (?)}, ānanda, tathāgato kappaṃ vā tiṭṭheyya kappāvasesaṃ vā’’ti. Evampi kho āyasmā ānando bhagavatā oḷārike nimitte kayiramāne oḷārike obhāse kayiramāne nāsakkhi paṭivijjhituṃ; na bhagavantaṃ yāci – ‘‘tiṭṭhatu, bhante, bhagavā kappaṃ, tiṭṭhatu sugato kappaṃ bahujanahitāya bahujanasukhāya lokānukampāya atthāya hitāya sukhāya devamanussāna’’nti, yathā taṃ mārena pariyuṭṭhitacitto. Dutiyampi kho bhagavā…pe… tatiyampi kho bhagavā āyasmantaṃ ānandaṃ āmantesi – ‘‘ramaṇīyā, ānanda, vesālī, ramaṇīyaṃ udenaṃ cetiyaṃ, ramaṇīyaṃ gotamakaṃ cetiyaṃ, ramaṇīyaṃ sattambaṃ cetiyaṃ, ramaṇīyaṃ bahuputtaṃ cetiyaṃ, ramaṇīyaṃ sārandadaṃ cetiyaṃ, ramaṇīyaṃ cāpālaṃ cetiyaṃ. Yassa kassaci, ānanda, cattāro iddhipādā bhāvitā bahulīkatā yānīkatā vatthukatā anuṭṭhitā paricitā susamāraddhā, so ākaṅkhamāno kappaṃ vā tiṭṭheyya kappāvasesaṃ vā. Tathāgatassa kho, ānanda, cattāro iddhipādā bhāvitā bahulīkatā yānīkatā vatthukatā anuṭṭhitā paricitā susamāraddhā, so ākaṅkhamāno, ānanda, tathāgato kappaṃ vā tiṭṭheyya kappāvasesaṃ vā’’ti. Evampi kho āyasmā ānando bhagavatā oḷārike nimitte kayiramāne oḷārike obhāse kayiramāne nāsakkhi paṭivijjhituṃ ; na bhagavantaṃ yāci – ‘‘tiṭṭhatu , bhante, bhagavā kappaṃ, tiṭṭhatu sugato kappaṃ bahujanahitāya bahujanasukhāya lokānukampāya atthāya hitāya sukhāya devamanussāna’’nti, yathā taṃ mārena pariyuṭṭhitacitto. Atha kho bhagavā āyasmantaṃ ānandaṃ āmantesi – ‘‘gaccha tvaṃ, ānanda, yassadāni kālaṃ maññasī’’ti. ‘‘Evaṃ, bhante’’ti kho āyasmā ānando bhagavato paṭissutvā uṭṭhāyāsanā bhagavantaṃ abhivādetvā padakkhiṇaṃ katvā avidūre aññatarasmiṃ rukkhamūle nisīdi.

\subsubsection{Mārayācanakathā}

\paragraph{168.} Atha kho māro pāpimā acirapakkante āyasmante ānande yena bhagavā tenupasaṅkami; upasaṅkamitvā ekamantaṃ aṭṭhāsi. Ekamantaṃ ṭhito kho māro pāpimā bhagavantaṃ etadavoca – ‘‘parinibbātudāni, bhante, bhagavā, parinibbātu sugato, parinibbānakālo dāni, bhante, bhagavato. Bhāsitā kho panesā, bhante, bhagavatā vācā – ‘na tāvāhaṃ, pāpima, parinibbāyissāmi, yāva me bhikkhū na sāvakā bhavissanti viyattā vinītā visāradā bahussutā dhammadharā dhammānudhammappaṭipannā sāmīcippaṭipannā anudhammacārino, sakaṃ ācariyakaṃ uggahetvā ācikkhissanti desessanti paññapessanti paṭṭhapessanti vivarissanti vibhajissanti uttānī\footnote{uttāniṃ (ka.), uttāni (sī. pī.)} karissanti, uppannaṃ parappavādaṃ sahadhammena suniggahitaṃ niggahetvā sappāṭihāriyaṃ dhammaṃ desessantī’ti . Etarahi kho pana, bhante, bhikkhū bhagavato sāvakā viyattā vinītā visāradā bahussutā dhammadharā dhammānudhammappaṭipannā sāmīcippaṭipannā anudhammacārino, sakaṃ ācariyakaṃ uggahetvā ācikkhanti desenti paññapenti paṭṭhapenti vivaranti vibhajanti uttānīkaronti, uppannaṃ parappavādaṃ sahadhammena suniggahitaṃ niggahetvā sappāṭihāriyaṃ dhammaṃ desenti. Parinibbātudāni, bhante, bhagavā, parinibbātu sugato, parinibbānakālodāni, bhante, bhagavato.

‘‘Bhāsitā kho panesā, bhante, bhagavatā vācā – ‘na tāvāhaṃ, pāpima, parinibbāyissāmi, yāva me bhikkhuniyo na sāvikā bhavissanti viyattā vinītā visāradā bahussutā dhammadharā dhammānudhammappaṭipannā sāmīcippaṭipannā anudhammacāriniyo, sakaṃ ācariyakaṃ uggahetvā ācikkhissanti desessanti paññapessanti paṭṭhapessanti vivarissanti vibhajissanti uttānīkarissanti, uppannaṃ parappavādaṃ sahadhammena suniggahitaṃ niggahetvā sappāṭihāriyaṃ dhammaṃ desessantī’ti . Etarahi kho pana, bhante, bhikkhuniyo bhagavato sāvikā viyattā vinītā visāradā bahussutā dhammadharā dhammānudhammappaṭipannā sāmīcippaṭipannā anudhammacāriniyo , sakaṃ ācariyakaṃ uggahetvā ācikkhanti desenti paññapenti paṭṭhapenti vivaranti vibhajanti uttānīkaronti, uppannaṃ parappavādaṃ sahadhammena suniggahitaṃ niggahetvā sappāṭihāriyaṃ dhammaṃ desenti. Parinibbātudāni, bhante, bhagavā, parinibbātu sugato, parinibbānakālodāni, bhante, bhagavato.

‘‘Bhāsitā kho panesā, bhante, bhagavatā vācā – ‘na tāvāhaṃ, pāpima, parinibbāyissāmi, yāva me upāsakā na sāvakā bhavissanti viyattā vinītā visāradā bahussutā dhammadharā dhammānudhammappaṭipannā sāmīcippaṭipannā anudhammacārino, sakaṃ ācariyakaṃ uggahetvā ācikkhissanti desessanti paññapessanti paṭṭhapessanti vivarissanti vibhajissanti uttānīkarissanti, uppannaṃ parappavādaṃ sahadhammena suniggahitaṃ niggahetvā sappāṭihāriyaṃ dhammaṃ desessantī’ti. Etarahi kho pana, bhante, upāsakā bhagavato sāvakā viyattā vinītā visāradā bahussutā dhammadharā dhammānudhammappaṭipannā sāmīcippaṭipannā anudhammacārino, sakaṃ ācariyakaṃ uggahetvā ācikkhanti desenti paññapenti paṭṭhapenti vivaranti vibhajanti uttānīkaronti, uppannaṃ parappavādaṃ sahadhammena suniggahitaṃ niggahetvā sappāṭihāriyaṃ dhammaṃ desenti. Parinibbātudāni , bhante, bhagavā, parinibbātu sugato, parinibbānakālodāni , bhante, bhagavato.

‘‘Bhāsitā kho panesā, bhante, bhagavatā vācā – ‘na tāvāhaṃ, pāpima parinibbāyissāmi, yāva me upāsikā na sāvikā bhavissanti viyattā vinītā visāradā bahussutā dhammadharā dhammānudhammappaṭipannā sāmīcippaṭipannā anudhammacāriniyo, sakaṃ ācariyakaṃ uggahetvā ācikkhissanti desessanti paññapessanti paṭṭhapessanti vivarissanti vibhajissanti uttānīkarissanti, uppannaṃ parappavādaṃ sahadhammena suniggahitaṃ niggahetvā sappāṭihāriyaṃ dhammaṃ desessantī’ti. Etarahi kho pana, bhante, upāsikā bhagavato sāvikā viyattā vinītā visāradā bahussutā dhammadharā dhammānudhammappaṭipannā sāmīcippaṭipannā anudhammacāriniyo, sakaṃ ācariyakaṃ uggahetvā ācikkhanti desenti paññapenti paṭṭhapenti vivaranti vibhajanti uttānīkaronti, uppannaṃ parappavādaṃ sahadhammena suniggahitaṃ niggahetvā sappāṭihāriyaṃ dhammaṃ desenti. Parinibbātudāni, bhante, bhagavā, parinibbātu sugato, parinibbānakālodāni, bhante, bhagavato.

‘‘Bhāsitā kho panesā, bhante, bhagavatā vācā – ‘na tāvāhaṃ, pāpima, parinibbāyissāmi , yāva me idaṃ brahmacariyaṃ na iddhaṃ ceva bhavissati phītañca vitthārikaṃ bāhujaññaṃ puthubhūtaṃ yāva devamanussehi suppakāsita’nti. Etarahi kho pana, bhante, bhagavato brahmacariyaṃ iddhaṃ ceva phītañca vitthārikaṃ bāhujaññaṃ puthubhūtaṃ, yāva devamanussehi suppakāsitaṃ. Parinibbātudāni, bhante, bhagavā, parinibbātu sugato, parinibbānakālodāni, bhante, bhagavato’’ti .

Evaṃ vutte bhagavā māraṃ pāpimantaṃ etadavoca – ‘‘appossukko tvaṃ, pāpima, hohi, na ciraṃ tathāgatassa parinibbānaṃ bhavissati. Ito tiṇṇaṃ māsānaṃ accayena tathāgato parinibbāyissatī’’ti.

\subsubsection{Āyusaṅkhāraossajjanaṃ}

\paragraph{169.} Atha kho bhagavā cāpāle cetiye sato sampajāno āyusaṅkhāraṃ ossaji. Ossaṭṭhe ca bhagavatā āyusaṅkhāre mahābhūmicālo ahosi bhiṃsanako salomahaṃso\footnote{lomahaṃso (syā.)}, devadundubhiyo\footnote{devadudrabhiyo (ka.)} ca phaliṃsu . Atha kho bhagavā etamatthaṃ viditvā tāyaṃ velāyaṃ imaṃ udānaṃ udānesi –

‘‘Tulamatulañca sambhavaṃ, bhavasaṅkhāramavassaji muni;

Ajjhattarato samāhito, abhindi kavacamivattasambhava’’nti.

\subsubsection{Mahābhūmicālahetu}

\paragraph{170.} Atha kho āyasmato ānandassa etadahosi – ‘‘acchariyaṃ vata bho, abbhutaṃ vata bho, mahā vatāyaṃ bhūmicālo; sumahā vatāyaṃ bhūmicālo bhiṃsanako salomahaṃso; devadundubhiyo ca phaliṃsu. Ko nu kho hetu ko paccayo mahato bhūmicālassa pātubhāvāyā’’ti?

Atha kho āyasmā ānando yena bhagavā tenupasaṅkami, upasaṅkamitvā bhagavantaṃ abhivādetvā ekamantaṃ nisīdi, ekamantaṃ nisinno kho āyasmā ānando bhagavantaṃ etadavoca – ‘‘acchariyaṃ, bhante, abbhutaṃ, bhante, mahā vatāyaṃ, bhante, bhūmicālo; sumahā vatāyaṃ , bhante, bhūmicālo bhiṃsanako salomahaṃso; devadundubhiyo ca phaliṃsu. Ko nu kho, bhante , hetu ko paccayo mahato bhūmicālassa pātubhāvāyā’’ti?

\paragraph{171.} ‘‘Aṭṭha kho ime, ānanda, hetū, aṭṭha paccayā mahato bhūmicālassa pātubhāvāya. Katame aṭṭha? Ayaṃ, ānanda, mahāpathavī udake patiṭṭhitā, udakaṃ vāte patiṭṭhitaṃ, vāto ākāsaṭṭho. Hoti kho so, ānanda, samayo, yaṃ mahāvātā vāyanti. Mahāvātā vāyantā udakaṃ kampenti. Udakaṃ kampitaṃ pathaviṃ kampeti. Ayaṃ paṭhamo hetu paṭhamo paccayo mahato bhūmicālassa pātubhāvāya.

‘‘Puna caparaṃ, ānanda, samaṇo vā hoti brāhmaṇo vā iddhimā cetovasippatto, devo vā mahiddhiko mahānubhāvo, tassa parittā pathavīsaññā bhāvitā hoti, appamāṇā āposaññā. So imaṃ pathaviṃ kampeti saṅkampeti sampakampeti sampavedheti. Ayaṃ dutiyo hetu dutiyo paccayo mahato bhūmicālassa pātubhāvāya.

‘‘Puna caparaṃ, ānanda, yadā bodhisatto tusitakāyā cavitvā sato sampajāno mātukucchiṃ okkamati, tadāyaṃ pathavī kampati saṅkampati sampakampati sampavedhati. Ayaṃ tatiyo hetu tatiyo paccayo mahato bhūmicālassa pātubhāvāya.

‘‘Puna caparaṃ, ānanda, yadā bodhisatto sato sampajāno mātukucchismā nikkhamati, tadāyaṃ pathavī kampati saṅkampati sampakampati sampavedhati. Ayaṃ catuttho hetu catuttho paccayo mahato bhūmicālassa pātubhāvāya.

‘‘Puna caparaṃ, ānanda, yadā tathāgato anuttaraṃ sammāsambodhiṃ abhisambujjhati, tadāyaṃ pathavī kampati saṅkampati sampakampati sampavedhati. Ayaṃ pañcamo hetu pañcamo paccayo mahato bhūmicālassa pātubhāvāya.

‘‘Puna caparaṃ, ānanda, yadā tathāgato anuttaraṃ dhammacakkaṃ pavatteti, tadāyaṃ pathavī kampati saṅkampati sampakampati sampavedhati. Ayaṃ chaṭṭho hetu chaṭṭho paccayo mahato bhūmicālassa pātubhāvāya.

‘‘Puna caparaṃ, ānanda, yadā tathāgato sato sampajāno āyusaṅkhāraṃ ossajjati, tadāyaṃ pathavī kampati saṅkampati sampakampati sampavedhati. Ayaṃ sattamo hetu sattamo paccayo mahato bhūmicālassa pātubhāvāya.

‘‘Puna caparaṃ, ānanda, yadā tathāgato anupādisesāya nibbānadhātuyā parinibbāyati, tadāyaṃ pathavī kampati saṅkampati sampakampati sampavedhati. Ayaṃ aṭṭhamo hetu aṭṭhamo paccayo mahato bhūmicālassa pātubhāvāya. Ime kho, ānanda, aṭṭha hetū, aṭṭha paccayā mahato bhūmicālassa pātubhāvāyā’’ti.

\subsubsection{Aṭṭha parisā}

\paragraph{172.} ‘‘Aṭṭha kho imā, ānanda, parisā. Katamā aṭṭha? Khattiyaparisā, brāhmaṇaparisā, gahapatiparisā, samaṇaparisā, cātumahārājikaparisā\footnote{cātummahārājikaparisā (sī. syā. kaṃ. pī.)}, tāvatiṃsaparisā, māraparisā, brahmaparisā. Abhijānāmi kho panāhaṃ, ānanda , anekasataṃ khattiyaparisaṃ upasaṅkamitā. Tatrapi mayā sannisinnapubbaṃ ceva sallapitapubbañca sākacchā ca samāpajjitapubbā . Tattha yādisako tesaṃ vaṇṇo hoti, tādisako mayhaṃ vaṇṇo hoti. Yādisako tesaṃ saro hoti, tādisako mayhaṃ saro hoti. Dhammiyā kathāya sandassemi samādapemi samuttejemi sampahaṃsemi. Bhāsamānañca maṃ na jānanti – ‘ko nu kho ayaṃ bhāsati devo vā manusso vā’ti? Dhammiyā kathāya sandassetvā samādapetvā samuttejetvā sampahaṃsetvā antaradhāyāmi. Antarahitañca maṃ na jānanti – ‘ko nu kho ayaṃ antarahito devo vā manusso vā’ti? Abhijānāmi kho panāhaṃ, ānanda, anekasataṃ brāhmaṇaparisaṃ…pe… gahapatiparisaṃ… samaṇaparisaṃ… cātumahārājikaparisaṃ… tāvatiṃsaparisaṃ… māraparisaṃ… brahmaparisaṃ upasaṅkamitā. Tatrapi mayā sannisinnapubbaṃ ceva sallapitapubbañca sākacchā ca samāpajjitapubbā. Tattha yādisako tesaṃ vaṇṇo hoti, tādisako mayhaṃ vaṇṇo hoti. Yādisako tesaṃ saro hoti, tādisako mayhaṃ saro hoti. Dhammiyā kathāya sandassemi samādapemi samuttejemi sampahaṃsemi. Bhāsamānañca maṃ na jānanti – ‘ko nu kho ayaṃ bhāsati devo vā manusso vā’ti? Dhammiyā kathāya sandassetvā samādapetvā samuttejetvā sampahaṃsetvā antaradhāyāmi. Antarahitañca maṃ na jānanti – ‘ko nu kho ayaṃ antarahito devo vā manusso vā’ti? Imā kho, ānanda, aṭṭha parisā.

\subsubsection{Aṭṭha abhibhāyatanāni}

\paragraph{173.} ‘‘Aṭṭha kho imāni, ānanda, abhibhāyatanāni. Katamāni aṭṭha ? Ajjhattaṃ rūpasaññī eko bahiddhā rūpāni passati parittāni suvaṇṇadubbaṇṇāni. ‘Tāni abhibhuyya jānāmi passāmī’ti evaṃsaññī hoti. Idaṃ paṭhamaṃ abhibhāyatanaṃ.

‘‘Ajjhattaṃ rūpasaññī eko bahiddhā rūpāni passati appamāṇāni suvaṇṇadubbaṇṇāni. ‘Tāni abhibhuyya jānāmi passāmī’ti evaṃsaññī hoti. Idaṃ dutiyaṃ abhibhāyatanaṃ.

‘‘Ajjhattaṃ arūpasaññī eko bahiddhā rūpāni passati parittāni suvaṇṇadubbaṇṇāni. ‘Tāni abhibhuyya jānāmi passāmī’ti evaṃsaññī hoti. Idaṃ tatiyaṃ abhibhāyatanaṃ.

‘‘Ajjhattaṃ arūpasaññī eko bahiddhā rūpāni passati appamāṇāni suvaṇṇadubbaṇṇāni. ‘Tāni abhibhuyya jānāmi passāmī’ti evaṃsaññī hoti. Idaṃ catutthaṃ abhibhāyatanaṃ.

‘‘Ajjhattaṃ arūpasaññī eko bahiddhā rūpāni passati nīlāni nīlavaṇṇāni nīlanidassanāni nīlanibhāsāni. Seyyathāpi nāma umāpupphaṃ nīlaṃ nīlavaṇṇaṃ nīlanidassanaṃ nīlanibhāsaṃ. Seyyathā vā pana taṃ vatthaṃ bārāṇaseyyakaṃ ubhatobhāgavimaṭṭhaṃ nīlaṃ nīlavaṇṇaṃ nīlanidassanaṃ nīlanibhāsaṃ. Evameva ajjhattaṃ arūpasaññī eko bahiddhā rūpāni passati nīlāni nīlavaṇṇāni nīlanidassanāni nīlanibhāsāni. ‘Tāni abhibhuyya jānāmi passāmī’ti evaṃsaññī hoti. Idaṃ pañcamaṃ abhibhāyatanaṃ.

‘‘Ajjhattaṃ arūpasaññī eko bahiddhā rūpāni passati pītāni pītavaṇṇāni pītanidassanāni pītanibhāsāni. Seyyathāpi nāma kaṇikārapupphaṃ pītaṃ pītavaṇṇaṃ pītanidassanaṃ pītanibhāsaṃ. Seyyathā vā pana taṃ vatthaṃ bārāṇaseyyakaṃ ubhatobhāgavimaṭṭhaṃ pītaṃ pītavaṇṇaṃ pītanidassanaṃ pītanibhāsaṃ. Evameva ajjhattaṃ arūpasaññī eko bahiddhā rūpāni passati pītāni pītavaṇṇāni pītanidassanāni pītanibhāsāni. ‘Tāni abhibhuyya jānāmi passāmī’ti evaṃsaññī hoti. Idaṃ chaṭṭhaṃ abhibhāyatanaṃ.

‘‘Ajjhattaṃ arūpasaññī eko bahiddhā rūpāni passati lohitakāni lohitakavaṇṇāni lohitakanidassanāni lohitakanibhāsāni. Seyyathāpi nāma bandhujīvakapupphaṃ lohitakaṃ lohitakavaṇṇaṃ lohitakanidassanaṃ lohitakanibhāsaṃ. Seyyathā vā pana taṃ vatthaṃ bārāṇaseyyakaṃ ubhatobhāgavimaṭṭhaṃ lohitakaṃ lohitakavaṇṇaṃ lohitakanidassanaṃ lohitakanibhāsaṃ. Evameva ajjhattaṃ arūpasaññī eko bahiddhā rūpāni passati lohitakāni lohitakavaṇṇāni lohitakanidassanāni lohitakanibhāsāni. ‘Tāni abhibhuyya jānāmi passāmī’ti evaṃsaññī hoti. Idaṃ sattamaṃ abhibhāyatanaṃ.

‘‘Ajjhattaṃ arūpasaññī eko bahiddhā rūpāni passati odātāni odātavaṇṇāni odātanidassanāni odātanibhāsāni. Seyyathāpi nāma osadhitārakā odātā odātavaṇṇā odātanidassanā odātanibhāsā. Seyyathā vā pana taṃ vatthaṃ bārāṇaseyyakaṃ ubhatobhāgavimaṭṭhaṃ odātaṃ odātavaṇṇaṃ odātanidassanaṃ odātanibhāsaṃ. Evameva ajjhattaṃ arūpasaññī eko bahiddhā rūpāni passati odātāni odātavaṇṇāni odātanidassanāni odātanibhāsāni. ‘Tāni abhibhuyya jānāmi passāmī’ti evaṃsaññī hoti. Idaṃ aṭṭhamaṃ abhibhāyatanaṃ . Imāni kho, ānanda, aṭṭha abhibhāyatanāni.

\subsubsection{Aṭṭha vimokkhā}

\paragraph{174.} ‘‘Aṭṭha kho ime, ānanda, vimokkhā. Katame aṭṭha? Rūpī rūpāni passati, ayaṃ paṭhamo vimokkho. Ajjhattaṃ arūpasaññī bahiddhā rūpāni passati, ayaṃ dutiyo vimokkho. Subhanteva adhimutto hoti, ayaṃ tatiyo vimokkho. Sabbaso rūpasaññānaṃ samatikkamā paṭighasaññānaṃ atthaṅgamā nānattasaññānaṃ amanasikārā ‘ananto ākāso’ti ākāsānañcāyatanaṃ upasampajja viharati, ayaṃ catuttho vimokkho. Sabbaso ākāsānañcāyatanaṃ samatikkamma ‘anantaṃ viññāṇa’nti viññāṇañcāyatanaṃ upasampajja viharati, ayaṃ pañcamo vimokkho. Sabbaso viññāṇañcāyatanaṃ samatikkamma ‘natthi kiñcī’ti ākiñcaññāyatanaṃ upasampajja viharati, ayaṃ chaṭṭho vimokkho. Sabbaso ākiñcaññāyatanaṃ samatikkamma nevasaññānāsaññāyatanaṃ upasampajja viharati. Ayaṃ sattamo vimokkho. Sabbaso nevasaññānāsaññāyatanaṃ samatikkamma saññāvedayitanirodhaṃ upasampajja viharati, ayaṃ aṭṭhamo vimokkho. Ime kho, ānanda, aṭṭha vimokkhā.

\paragraph{175.} ‘‘Ekamidāhaṃ , ānanda, samayaṃ uruvelāyaṃ viharāmi najjā nerañjarāya tīre ajapālanigrodhe paṭhamābhisambuddho. Atha kho, ānanda, māro pāpimā yenāhaṃ tenupasaṅkami; upasaṅkamitvā ekamantaṃ aṭṭhāsi. Ekamantaṃ ṭhito kho, ānanda, māro pāpimā maṃ etadavoca – ‘parinibbātudāni, bhante, bhagavā; parinibbātu sugato, parinibbānakālodāni, bhante, bhagavato’ti. Evaṃ vutte ahaṃ, ānanda, māraṃ pāpimantaṃ etadavocaṃ –

‘‘‘Na tāvāhaṃ, pāpima, parinibbāyissāmi, yāva me bhikkhū na sāvakā bhavissanti viyattā vinītā visāradā bahussutā dhammadharā dhammānudhammappaṭipannā sāmīcippaṭipannā anudhammacārino, sakaṃ ācariyakaṃ uggahetvā ācikkhissanti desessanti paññapessanti paṭṭhapessanti vivarissanti vibhajissanti uttānīkarissanti, uppannaṃ parappavādaṃ sahadhammena suniggahitaṃ niggahetvā sappāṭihāriyaṃ dhammaṃ desessanti.

‘‘‘Na tāvāhaṃ, pāpima, parinibbāyissāmi, yāva me bhikkhuniyo na sāvikā bhavissanti viyattā vinītā visāradā bahussutā dhammadharā dhammānudhammappaṭipannā sāmīcippaṭipannā anudhammacāriniyo, sakaṃ ācariyakaṃ uggahetvā ācikkhissanti desessanti paññapessanti paṭṭhapessanti vivarissanti vibhajissanti uttānīkarissanti, uppannaṃ parappavādaṃ sahadhammena suniggahitaṃ niggahetvā sappāṭihāriyaṃ dhammaṃ desessanti.

‘‘‘Na tāvāhaṃ, pāpima, parinibbāyissāmi, yāva me upāsakā na sāvakā bhavissanti viyattā vinītā visāradā bahussutā dhammadharā dhammānudhammappaṭipannā sāmīcippaṭipannā anudhammacārino, sakaṃ ācariyakaṃ uggahetvā ācikkhissanti desessanti paññapessanti paṭṭhapessanti vivarissanti vibhajissanti uttānīkarissanti, uppannaṃ parappavādaṃ sahadhammena suniggahitaṃ niggahetvā sappāṭihāriyaṃ dhammaṃ desessanti.

‘‘‘Na tāvāhaṃ, pāpima, parinibbāyissāmi, yāva me upāsikā na sāvikā bhavissanti viyattā vinītā visāradā bahussutā dhammadharā dhammānudhammappaṭipannā sāmīcippaṭipannā anudhammacāriniyo, sakaṃ ācariyakaṃ uggahetvā ācikkhissanti desessanti paññapessanti paṭṭhapessanti vivarissanti vibhajissanti uttānīkarissanti, uppannaṃ parappavādaṃ sahadhammena suniggahitaṃ niggahetvā sappāṭihāriyaṃ dhammaṃ desessanti.

‘‘‘Na tāvāhaṃ, pāpima, parinibbāyissāmi, yāva me idaṃ brahmacariyaṃ na iddhañceva bhavissati phītañca vitthārikaṃ bāhujaññaṃ puthubhūtaṃ yāva devamanussehi suppakāsita’nti.

\paragraph{176.} ‘‘Idāneva kho, ānanda, ajja cāpāle cetiye māro pāpimā yenāhaṃ tenupasaṅkami; upasaṅkamitvā ekamantaṃ aṭṭhāsi. Ekamantaṃ ṭhito kho, ānanda, māro pāpimā maṃ etadavoca – ‘parinibbātudāni, bhante, bhagavā, parinibbātu sugato, parinibbānakālodāni, bhante, bhagavato. Bhāsitā kho panesā, bhante, bhagavatā vācā – ‘‘na tāvāhaṃ, pāpima , parinibbāyissāmi , yāva me bhikkhū na sāvakā bhavissanti…pe… yāva me bhikkhuniyo na sāvikā bhavissanti…pe… yāva me upāsakā na sāvakā bhavissanti…pe… yāva me upāsikā na sāvikā bhavissanti…pe… yāva me idaṃ brahmacariyaṃ na iddhañceva bhavissati phītañca vitthārikaṃ bāhujaññaṃ puthubhūtaṃ, yāva devamanussehi suppakāsita’’nti. Etarahi kho pana, bhante, bhagavato brahmacariyaṃ iddhañceva phītañca vitthārikaṃ bāhujaññaṃ puthubhūtaṃ, yāva devamanussehi suppakāsitaṃ. Parinibbātudāni, bhante, bhagavā, parinibbātu sugato, parinibbānakālodāni, bhante, bhagavato’ti.

\paragraph{177.} ‘‘Evaṃ vutte, ahaṃ, ānanda, māraṃ pāpimantaṃ etadavocaṃ – ‘appossukko tvaṃ, pāpima, hohi, naciraṃ tathāgatassa parinibbānaṃ bhavissati. Ito tiṇṇaṃ māsānaṃ accayena tathāgato parinibbāyissatī’ti. Idāneva kho, ānanda, ajja cāpāle cetiye tathāgatena satena sampajānena āyusaṅkhāro ossaṭṭho’’ti.

\subsubsection{Ānandayācanakathā}

\paragraph{178.} Evaṃ vutte āyasmā ānando bhagavantaṃ etadavoca – ‘‘tiṭṭhatu, bhante, bhagavā kappaṃ, tiṭṭhatu sugato kappaṃ bahujanahitāya bahujanasukhāya lokānukampāya atthāya hitāya sukhāya devamanussāna’’nti.

‘‘Alaṃdāni, ānanda. Mā tathāgataṃ yāci, akālodāni, ānanda, tathāgataṃ yācanāyā’’ti. Dutiyampi kho āyasmā ānando…pe… tatiyampi kho āyasmā ānando bhagavantaṃ etadavoca – ‘‘tiṭṭhatu, bhante, bhagavā kappaṃ, tiṭṭhatu sugato kappaṃ bahujanahitāya bahujanasukhāya lokānukampāya atthāya hitāya sukhāya devamanussāna’’nti.

‘‘Saddahasi tvaṃ, ānanda, tathāgatassa bodhi’’nti? ‘‘Evaṃ, bhante’’. ‘‘Atha kiñcarahi tvaṃ, ānanda, tathāgataṃ yāvatatiyakaṃ abhinippīḷesī’’ti? ‘‘Sammukhā metaṃ, bhante, bhagavato sutaṃ sammukhā paṭiggahitaṃ – ‘yassa kassaci, ānanda, cattāro iddhipādā bhāvitā bahulīkatā yānīkatā vatthukatā anuṭṭhitā paricitā susamāraddhā, so ākaṅkhamāno kappaṃ vā tiṭṭheyya kappāvasesaṃ vā. Tathāgatassa kho, ānanda, cattāro iddhipādā bhāvitā bahulīkatā yānīkatā vatthukatā anuṭṭhitā paricitā susamāraddhā. So ākaṅkhamāno, ānanda, tathāgato kappaṃ vā tiṭṭheyya kappāvasesaṃ vā’’’ti. ‘‘Saddahasi tvaṃ, ānandā’’ti? ‘‘Evaṃ, bhante’’. ‘‘Tasmātihānanda, tuyhevetaṃ dukkaṭaṃ, tuyhevetaṃ aparaddhaṃ, yaṃ tvaṃ tathāgatena evaṃ oḷārike nimitte kayiramāne oḷārike obhāse kayiramāne nāsakkhi paṭivijjhituṃ, na tathāgataṃ yāci – ‘tiṭṭhatu, bhante, bhagavā kappaṃ, tiṭṭhatu sugato kappaṃ bahujanahitāya bahujanasukhāya lokānukampāya atthāya hitāya sukhāya devamanussāna’’nti. Sace tvaṃ, ānanda, tathāgataṃ yāceyyāsi, dveva te vācā tathāgato paṭikkhipeyya, atha tatiyakaṃ adhivāseyya. Tasmātihānanda, tuyhevetaṃ dukkaṭaṃ, tuyhevetaṃ aparaddhaṃ.

\paragraph{179.} ‘‘Ekamidāhaṃ, ānanda, samayaṃ rājagahe viharāmi gijjhakūṭe pabbate. Tatrāpi kho tāhaṃ, ānanda, āmantesiṃ – ‘ramaṇīyaṃ, ānanda, rājagahaṃ, ramaṇīyo, ānanda, gijjhakūṭo pabbato. Yassa kassaci, ānanda, cattāro iddhipādā bhāvitā bahulīkatā yānīkatā vatthukatā anuṭṭhitā paricitā susamāraddhā, so ākaṅkhamāno kappaṃ vā tiṭṭheyya kappāvasesaṃ vā. Tathāgatassa kho, ānanda, cattāro iddhipādā bhāvitā bahulīkatā yānīkatā vatthukatā anuṭṭhitā paricitā susamāraddhā, so ākaṅkhamāno, ānanda, tathāgato kappaṃ vā tiṭṭheyya kappāvasesaṃ vā’ti. Evampi kho tvaṃ, ānanda, tathāgatena oḷārike nimitte kayiramāne oḷārike obhāse kayiramāne nāsakkhi paṭivijjhituṃ, na tathāgataṃ yāci – ‘tiṭṭhatu, bhante, bhagavā kappaṃ, tiṭṭhatu sugato kappaṃ bahujanahitāya bahujanasukhāya lokānukampāya atthāya hitāya sukhāya devamanussāna’nti. Sace tvaṃ, ānanda, tathāgataṃ yāceyyāsi, dve te vācā tathāgato paṭikkhipeyya, atha tatiyakaṃ adhivāseyya. Tasmātihānanda, tuyhevetaṃ dukkaṭaṃ, tuyhevetaṃ aparaddhaṃ.

\paragraph{180.} ‘‘Ekamidāhaṃ, ānanda, samayaṃ tattheva rājagahe viharāmi gotamanigrodhe…pe… tattheva rājagahe viharāmi corapapāte… tattheva rājagahe viharāmi vebhārapasse sattapaṇṇiguhāyaṃ… tattheva rājagahe viharāmi isigilipasse kāḷasilāyaṃ… tattheva rājagahe viharāmi sītavane sappasoṇḍikapabbhāre… tattheva rājagahe viharāmi tapodārāme… tattheva rājagahe viharāmi veḷuvane kalandakanivāpe… tattheva rājagahe viharāmi jīvakambavane… tattheva rājagahe viharāmi maddakucchismiṃ migadāye tatrāpi kho tāhaṃ, ānanda, āmantesiṃ – ‘ramaṇīyaṃ, ānanda, rājagahaṃ, ramaṇīyo gijjhakūṭo pabbato, ramaṇīyo gotamanigrodho, ramaṇīyo corapapāto, ramaṇīyā vebhārapasse sattapaṇṇiguhā, ramaṇīyā isigilipasse kāḷasilā, ramaṇīyo sītavane sappasoṇḍikapabbhāro , ramaṇīyo tapodārāmo, ramaṇīyo veḷuvane kalandakanivāpo, ramaṇīyaṃ jīvakambavanaṃ, ramaṇīyo maddakucchismiṃ migadāyo. Yassa kassaci, ānanda, cattāro iddhipādā bhāvitā bahulīkatā yānīkatā vatthukatā anuṭṭhitā paricitā susamāraddhā…pe… ākaṅkhamāno, ānanda, tathāgato kappaṃ vā tiṭṭheyya kappāvasesaṃ vā’ti. Evampi kho tvaṃ, ānanda, tathāgatena oḷārike nimitte kayiramāne oḷārike obhāse kayiramāne nāsakkhi paṭivijjhituṃ, na tathāgataṃ yāci – ‘tiṭṭhatu, bhante, bhagavā kappaṃ, tiṭṭhatu sugato kappaṃ bahujanahitāya bahujanasukhāya lokānukampāya atthāya hitāya sukhāya devamanussāna’nti. Sace tvaṃ, ānanda, tathāgataṃ yāceyyāsi, dveva te vācā tathāgato paṭikkhipeyya, atha tatiyakaṃ adhivāseyya. Tasmātihānanda, tuyhevetaṃ dukkaṭaṃ, tuyhevetaṃ aparaddhaṃ.

\paragraph{181.} ‘‘Ekamidāhaṃ, ānanda, samayaṃ idheva vesāliyaṃ viharāmi udene cetiye. Tatrāpi kho tāhaṃ, ānanda, āmantesiṃ – ‘ramaṇīyā, ānanda, vesālī, ramaṇīyaṃ udenaṃ cetiyaṃ. Yassa kassaci, ānanda, cattāro iddhipādā bhāvitā bahulīkatā yānīkatā vatthukatā anuṭṭhitā paricitā susamāraddhā, so ākaṅkhamāno kappaṃ vā tiṭṭheyya kappāvasesaṃ vā. Tathāgatassa kho, ānanda, cattāro iddhipādā bhāvitā bahulīkatā yānīkatā vatthukatā anuṭṭhitā paricitā susamāraddhā, so ākaṅkhamāno, ānanda, tathāgato kappaṃ vā tiṭṭheyya kappāvasesaṃ vā’ti. Evampi kho tvaṃ, ānanda, tathāgatena oḷārike nimitte kayiramāne oḷārike obhāse kayiramāne nāsakkhi paṭivijjhituṃ, na tathāgataṃ yāci – ‘tiṭṭhatu, bhante, bhagavā kappaṃ, tiṭṭhatu sugato kappaṃ bahujanahitāya bahujanasukhāya lokānukampāya atthāya hitāya sukhāya devamanussāna’nti. Sace tvaṃ, ānanda, tathāgataṃ yāceyyāsi, dveva te vācā tathāgato paṭikkhipeyya, atha tatiyakaṃ adhivāseyya, tasmātihānanda, tuyhevetaṃ dukkaṭaṃ, tuyhevetaṃ aparaddhaṃ.

\paragraph{182.} ‘‘Ekamidāhaṃ , ānanda, samayaṃ idheva vesāliyaṃ viharāmi gotamake cetiye …pe… idheva vesāliyaṃ viharāmi sattambe cetiye… idheva vesāliyaṃ viharāmi bahuputte cetiye… idheva vesāliyaṃ viharāmi sārandade cetiye… idāneva kho tāhaṃ, ānanda, ajja cāpāle cetiye āmantesiṃ – ‘ramaṇīyā, ānanda, vesālī, ramaṇīyaṃ udenaṃ cetiyaṃ, ramaṇīyaṃ gotamakaṃ cetiyaṃ, ramaṇīyaṃ sattambaṃ cetiyaṃ, ramaṇīyaṃ bahuputtaṃ cetiyaṃ, ramaṇīyaṃ sārandadaṃ cetiyaṃ, ramaṇīyaṃ cāpālaṃ cetiyaṃ. Yassa kassaci, ānanda, cattāro iddhipādā bhāvitā bahulīkatā yānīkatā vatthukatā anuṭṭhitā paricitā susamāraddhā, so ākaṅkhamāno kappaṃ vā tiṭṭheyya kappāvasesaṃ vā. Tathāgatassa kho, ānanda, cattāro iddhipādā bhāvitā bahulīkatā yānīkatā vatthukatā anuṭṭhitā paricitā susamāraddhā, so ākaṅkhamāno, ānanda, tathāgato kappaṃ vā tiṭṭheyya kappāvasesaṃ vā’ti. Evampi kho tvaṃ, ānanda, tathāgatena oḷārike nimitte kayiramāne oḷārike obhāse kayiramāne nāsakkhi paṭivijjhituṃ, na tathāgataṃ yāci – ‘tiṭṭhatu bhagavā kappaṃ, tiṭṭhatu sugato kappaṃ bahujanahitāya bahujanasukhāya lokānukampāya atthāya hitāya sukhāya devamanussāna’nti. Sace tvaṃ, ānanda, tathāgataṃ yāceyyāsi, dveva te vācā tathāgato paṭikkhipeyya, atha tatiyakaṃ adhivāseyya. Tasmātihānanda, tuyhevetaṃ dukkaṭaṃ, tuyhevetaṃ aparaddhaṃ.

\paragraph{183.} ‘‘Nanu etaṃ\footnote{evaṃ (syā. pī.)}, ānanda, mayā paṭikacceva\footnote{paṭigacceva (sī. pī.)} akkhātaṃ – ‘sabbeheva piyehi manāpehi nānābhāvo vinābhāvo aññathābhāvo. Taṃ kutettha, ānanda, labbhā, yaṃ taṃ jātaṃ bhūtaṃ saṅkhataṃ palokadhammaṃ, taṃ vata mā palujjīti netaṃ ṭhānaṃ vijjati’. Yaṃ kho panetaṃ, ānanda, tathāgatena cattaṃ vantaṃ muttaṃ pahīnaṃ paṭinissaṭṭhaṃ ossaṭṭho āyusaṅkhāro, ekaṃsena vācā bhāsitā – ‘na ciraṃ tathāgatassa parinibbānaṃ bhavissati. Ito tiṇṇaṃ māsānaṃ accayena tathāgato parinibbāyissatī’ti. Tañca\footnote{taṃ vacanaṃ (sī.)} tathāgato jīvitahetu puna paccāvamissatīti\footnote{paccāgamissatīti (syā. ka.)} netaṃ ṭhānaṃ vijjati. Āyāmānanda, yena mahāvanaṃ kūṭāgārasālā tenupasaṅkamissāmā’’ti. ‘‘Evaṃ, bhante’’ti kho āyasmā ānando bhagavato paccassosi.

Atha kho bhagavā āyasmatā ānandena saddhiṃ yena mahāvanaṃ kūṭāgārasālā tenupasaṅkami; upasaṅkamitvā āyasmantaṃ ānandaṃ āmantesi – ‘‘gaccha tvaṃ, ānanda, yāvatikā bhikkhū vesāliṃ upanissāya viharanti, te sabbe upaṭṭhānasālāyaṃ sannipātehī’’ti. ‘‘Evaṃ, bhante’’ti kho āyasmā ānando bhagavato paṭissutvā yāvatikā bhikkhū vesāliṃ upanissāya viharanti, te sabbe upaṭṭhānasālāyaṃ sannipātetvā yena bhagavā tenupasaṅkami; upasaṅkamitvā bhagavantaṃ abhivādetvā ekamantaṃ aṭṭhāsi. Ekamantaṃ ṭhito kho āyasmā ānando bhagavantaṃ etadavoca – ‘‘sannipatito, bhante, bhikkhusaṅgho, yassadāni, bhante, bhagavā kālaṃ maññatī’’ti.

\paragraph{184.} Atha kho bhagavā yenupaṭṭhānasālā tenupasaṅkami; upasaṅkamitvā paññatte āsane nisīdi. Nisajja kho bhagavā bhikkhū āmantesi – ‘‘tasmātiha, bhikkhave, ye te mayā dhammā abhiññā desitā, te vo sādhukaṃ uggahetvā āsevitabbā bhāvetabbā bahulīkātabbā, yathayidaṃ brahmacariyaṃ addhaniyaṃ assa ciraṭṭhitikaṃ, tadassa bahujanahitāya bahujanasukhāya lokānukampāya atthāya hitāya sukhāya devamanussānaṃ. Katame ca te, bhikkhave, dhammā mayā abhiññā desitā, ye vo sādhukaṃ uggahetvā āsevitabbā bhāvetabbā bahulīkātabbā, yathayidaṃ brahmacariyaṃ addhaniyaṃ assa ciraṭṭhitikaṃ, tadassa bahujanahitāya bahujanasukhāya lokānukampāya atthāya hitāya sukhāya devamanussānaṃ. Seyyathidaṃ – cattāro satipaṭṭhānā cattāro sammappadhānā cattāro iddhipādā pañcindriyāni pañca balāni satta bojjhaṅgā ariyo aṭṭhaṅgiko maggo. Ime kho te, bhikkhave, dhammā mayā abhiññā desitā, ye vo sādhukaṃ uggahetvā āsevitabbā bhāvetabbā bahulīkātabbā, yathayidaṃ brahmacariyaṃ addhaniyaṃ assa ciraṭṭhitikaṃ, tadassa bahujanahitāya bahujanasukhāya lokānukampāya atthāya hitāya sukhāya devamanussāna’’nti.

\paragraph{185.} Atha kho bhagavā bhikkhū āmantesi – ‘‘handadāni, bhikkhave, āmantayāmi vo, vayadhammā saṅkhārā, appamādena sampādetha. Naciraṃ tathāgatassa parinibbānaṃ bhavissati. Ito tiṇṇaṃ māsānaṃ accayena tathāgato parinibbāyissatī’’ti. Idamavoca bhagavā, idaṃ vatvāna sugato athāparaṃ etadavoca satthā\footnote{ito paraṃ syāmapotthake evaṃpi pāṭho dissati –§daharāpi ca ye vuddhā, ye bālā ye ca paṇḍitā.§aḍḍhāceva daliddā ca, sabbe maccuparāyanā.§yathāpi kumbhakārassa, kataṃ mattikabhājanaṃ.§khuddakañca mahantañca, yañca pakkaṃ yañca āmakaṃ.§sabbaṃ bhedapariyantaṃ, evaṃ maccāna jīvitaṃ.§athāparaṃ etadavoca satthā}. –

‘‘Paripakko vayo mayhaṃ, parittaṃ mama jīvitaṃ;

Pahāya vo gamissāmi, kataṃ me saraṇamattano.

‘‘Appamattā satīmanto, susīlā hotha bhikkhavo;

Susamāhitasaṅkappā, sacittamanurakkhatha.

‘‘Yo imasmiṃ dhammavinaye, appamatto vihassati;

Pahāya jātisaṃsāraṃ, dukkhassantaṃ karissatī’’ti\footnote{viharissati (syā.), vihessati (sī.)}.

\xsubsubsectionEnd{Tatiyo bhāṇavāro.}

\subsubsection{Nāgāpalokitaṃ}

\paragraph{186.} Atha kho bhagavā pubbaṇhasamayaṃ nivāsetvā pattacīvaramādāya vesāliṃ piṇḍāya pāvisi. Vesāliyaṃ piṇḍāya caritvā pacchābhattaṃ piṇḍapātappaṭikkanto nāgāpalokitaṃ vesāliṃ apaloketvā āyasmantaṃ ānandaṃ āmantesi – ‘‘idaṃ pacchimakaṃ, ānanda, tathāgatassa vesāliyā dassanaṃ bhavissati. Āyāmānanda, yena bhaṇḍagāmo\footnote{bhaṇḍugāmo (ka.)} tenupasaṅkamissāmā’’ti. ‘‘Evaṃ, bhante’’ti kho āyasmā ānando bhagavato paccassosi.

Atha kho bhagavā mahatā bhikkhusaṅghena saddhiṃ yena bhaṇḍagāmo tadavasari. Tatra sudaṃ bhagavā bhaṇḍagāme viharati. Tatra kho bhagavā bhikkhū āmantesi – ‘‘catunnaṃ, bhikkhave, dhammānaṃ ananubodhā appaṭivedhā evamidaṃ dīghamaddhānaṃ sandhāvitaṃ saṃsaritaṃ mamañceva tumhākañca. Katamesaṃ catunnaṃ? Ariyassa, bhikkhave, sīlassa ananubodhā appaṭivedhā evamidaṃ dīghamaddhānaṃ sandhāvitaṃ saṃsaritaṃ mamaṃ ceva tumhākañca. Ariyassa, bhikkhave, samādhissa ananubodhā appaṭivedhā evamidaṃ dīghamaddhānaṃ sandhāvitaṃ saṃsaritaṃ mamaṃ ceva tumhākañca. Ariyāya, bhikkhave, paññāya ananubodhā appaṭivedhā evamidaṃ dīghamaddhānaṃ sandhāvitaṃ saṃsaritaṃ mamaṃ ceva tumhākañca. Ariyāya, bhikkhave, vimuttiyā ananubodhā appaṭivedhā evamidaṃ dīghamaddhānaṃ sandhāvitaṃ saṃsaritaṃ mamaṃ ceva tumhākañca. Tayidaṃ, bhikkhave, ariyaṃ sīlaṃ anubuddhaṃ paṭividdhaṃ, ariyo samādhi anubuddho paṭividdho, ariyā paññā anubuddhā paṭividdhā, ariyā vimutti anubuddhā paṭividdhā, ucchinnā bhavataṇhā, khīṇā bhavanetti, natthi dāni punabbhavo’’ti. Idamavoca bhagavā, idaṃ vatvāna sugato athāparaṃ etadavoca satthā –

‘‘Sīlaṃ samādhi paññā ca, vimutti ca anuttarā;

Anubuddhā ime dhammā, gotamena yasassinā.

‘‘Iti buddho abhiññāya, dhammamakkhāsi bhikkhunaṃ;

Dukkhassantakaro satthā, cakkhumā parinibbuto’’ti.

Tatrāpi sudaṃ bhagavā bhaṇḍagāme viharanto etadeva bahulaṃ bhikkhūnaṃ dhammiṃ kathaṃ karoti – ‘‘iti sīlaṃ, iti samādhi, iti paññā. Sīlaparibhāvito samādhi mahapphalo hoti mahānisaṃso. Samādhiparibhāvitā paññā mahapphalā hoti mahānisaṃsā. Paññāparibhāvitaṃ cittaṃ sammadeva āsavehi vimuccati, seyyathidaṃ – kāmāsavā, bhavāsavā, avijjāsavā’’ti.

\subsubsection{Catumahāpadesakathā}

\paragraph{187.} Atha kho bhagavā bhaṇḍagāme yathābhirantaṃ viharitvā āyasmantaṃ ānandaṃ āmantesi – ‘‘āyāmānanda, yena hatthigāmo, yena ambagāmo, yena jambugāmo, yena bhoganagaraṃ tenupasaṅkamissāmā’’ti. ‘‘Evaṃ, bhante’’ti kho āyasmā ānando bhagavato paccassosi. Atha kho bhagavā mahatā bhikkhusaṅghena saddhiṃ yena bhoganagaraṃ tadavasari. Tatra sudaṃ bhagavā bhoganagare viharati ānande\footnote{sānandare (ka.)} cetiye. Tatra kho bhagavā bhikkhū āmantesi – ‘‘cattārome, bhikkhave, mahāpadese desessāmi, taṃ suṇātha, sādhukaṃ manasikarotha, bhāsissāmī’’ti. ‘‘Evaṃ , bhante’’ti kho te bhikkhū bhagavato paccassosuṃ. Bhagavā etadavoca –

\paragraph{188.} ‘‘Idha, bhikkhave, bhikkhu evaṃ vadeyya – ‘sammukhā metaṃ, āvuso, bhagavato sutaṃ sammukhā paṭiggahitaṃ, ayaṃ dhammo ayaṃ vinayo idaṃ satthusāsana’nti. Tassa, bhikkhave, bhikkhuno bhāsitaṃ neva abhinanditabbaṃ nappaṭikkositabbaṃ. Anabhinanditvā appaṭikkositvā tāni padabyañjanāni sādhukaṃ uggahetvā sutte osāretabbāni\footnote{otāretabbāni}, vinaye sandassetabbāni. Tāni ce sutte osāriyamānāni\footnote{otāriyamānāni} vinaye sandassiyamānāni na ceva sutte osaranti\footnote{otaranti (sī. pī. a. ni. 4.180}, na ca vinaye sandissanti, niṭṭhamettha gantabbaṃ – ‘addhā, idaṃ na ceva tassa bhagavato vacanaṃ; imassa ca bhikkhuno duggahita’nti. Itihetaṃ, bhikkhave, chaḍḍeyyātha. Tāni ce sutte osāriyamānāni vinaye sandassiyamānāni sutte ceva osaranti, vinaye ca sandissanti, niṭṭhamettha gantabbaṃ – ‘addhā, idaṃ tassa bhagavato vacanaṃ; imassa ca bhikkhuno suggahita’nti. Idaṃ, bhikkhave, paṭhamaṃ mahāpadesaṃ dhāreyyātha.

‘‘Idha pana, bhikkhave, bhikkhu evaṃ vadeyya – ‘amukasmiṃ nāma āvāse saṅgho viharati sathero sapāmokkho. Tassa me saṅghassa sammukhā sutaṃ sammukhā paṭiggahitaṃ, ayaṃ dhammo ayaṃ vinayo idaṃ satthusāsana’nti. Tassa, bhikkhave, bhikkhuno bhāsitaṃ neva abhinanditabbaṃ nappaṭikkositabbaṃ. Anabhinanditvā appaṭikkositvā tāni padabyañjanāni sādhukaṃ uggahetvā sutte osāretabbāni, vinaye sandassetabbāni. Tāni ce sutte osāriyamānāni vinaye sandassiyamānāni na ceva sutte osaranti, na ca vinaye sandissanti, niṭṭhamettha gantabbaṃ – ‘addhā, idaṃ na ceva tassa bhagavato vacanaṃ; tassa ca saṅghassa duggahita’nti. Itihetaṃ, bhikkhave, chaḍḍeyyātha. Tāni ce sutte osāriyamānāni vinaye sandassiyamānāni sutte ceva osaranti vinaye ca sandissanti, niṭṭhamettha gantabbaṃ – ‘addhā , idaṃ tassa bhagavato vacanaṃ; tassa ca saṅghassa suggahita’nti. Idaṃ, bhikkhave, dutiyaṃ mahāpadesaṃ dhāreyyātha.

‘‘Idha pana, bhikkhave, bhikkhu evaṃ vadeyya – ‘amukasmiṃ nāma āvāse sambahulā therā bhikkhū viharanti bahussutā āgatāgamā dhammadharā vinayadharā mātikādharā. Tesaṃ me therānaṃ sammukhā sutaṃ sammukhā paṭiggahitaṃ – ayaṃ dhammo ayaṃ vinayo idaṃ satthusāsana’nti. Tassa, bhikkhave, bhikkhuno bhāsitaṃ neva abhinanditabbaṃ…pe… na ca vinaye sandissanti, niṭṭhamettha gantabbaṃ – ‘addhā, idaṃ na ceva tassa bhagavato vacanaṃ; tesañca therānaṃ duggahita’nti. Itihetaṃ, bhikkhave, chaḍḍeyyātha. Tāni ce sutte osāriyamānāni…pe… vinaye ca sandissanti, niṭṭhamettha gantabbaṃ – ‘addhā, idaṃ tassa bhagavato vacanaṃ; tesañca therānaṃ suggahita’nti. Idaṃ, bhikkhave, tatiyaṃ mahāpadesaṃ dhāreyyātha.

‘‘Idha pana, bhikkhave, bhikkhu evaṃ vadeyya – ‘amukasmiṃ nāma āvāse eko thero bhikkhu viharati bahussuto āgatāgamo dhammadharo vinayadharo mātikādharo. Tassa me therassa sammukhā sutaṃ sammukhā paṭiggahitaṃ – ayaṃ dhammo ayaṃ vinayo idaṃ satthusāsana’nti. Tassa, bhikkhave, bhikkhuno bhāsitaṃ neva abhinanditabbaṃ nappaṭikkositabbaṃ. Anabhinanditvā appaṭikkositvā tāni padabyañjanāni sādhukaṃ uggahetvā sutte osāritabbāni, vinaye sandassetabbāni. Tāni ce sutte osāriyamānāni vinaye sandassiyamānāni na ceva sutte osaranti, na ca vinaye sandissanti, niṭṭhamettha gantabbaṃ – ‘addhā, idaṃ na ceva tassa bhagavato vacanaṃ; tassa ca therassa duggahita’nti. Itihetaṃ, bhikkhave, chaḍḍeyyātha. Tāni ca sutte osāriyamānāni vinaye sandassiyamānāni sutte ceva osaranti, vinaye ca sandissanti , niṭṭhamettha gantabbaṃ – ‘addhā , idaṃ tassa bhagavato vacanaṃ; tassa ca therassa suggahita’nti. Idaṃ, bhikkhave, catutthaṃ mahāpadesaṃ dhāreyyātha. Ime kho, bhikkhave, cattāro mahāpadese dhāreyyāthā’’ti.

Tatrapi sudaṃ bhagavā bhoganagare viharanto ānande cetiye etadeva bahulaṃ bhikkhūnaṃ dhammiṃ kathaṃ karoti – ‘‘iti sīlaṃ, iti samādhi, iti paññā. Sīlaparibhāvito samādhi mahapphalo hoti mahānisaṃso . Samādhiparibhāvitā paññā mahapphalā hoti mahānisaṃsā. Paññāparibhāvitaṃ cittaṃ sammadeva āsavehi vimuccati, seyyathidaṃ – kāmāsavā, bhavāsavā, avijjāsavā’’ti.

\subsubsection{Kammāraputtacundavatthu}

\paragraph{189.} Atha kho bhagavā bhoganagare yathābhirantaṃ viharitvā āyasmantaṃ ānandaṃ āmantesi – ‘‘āyāmānanda, yena pāvā tenupasaṅkamissāmā’’ti. ‘‘Evaṃ, bhante’’ti kho āyasmā ānando bhagavato paccassosi. Atha kho bhagavā mahatā bhikkhusaṅghena saddhiṃ yena pāvā tadavasari. Tatra sudaṃ bhagavā pāvāyaṃ viharati cundassa kammāraputtassa ambavane. Assosi kho cundo kammāraputto – ‘‘bhagavā kira pāvaṃ anuppatto, pāvāyaṃ viharati mayhaṃ ambavane’’ti. Atha kho cundo kammāraputto yena bhagavā tenupasaṅkami; upasaṅkamitvā bhagavantaṃ abhivādetvā ekamantaṃ nisīdi. Ekamantaṃ nisinnaṃ kho cundaṃ kammāraputtaṃ bhagavā dhammiyā kathāya sandassesi samādapesi samuttejesi sampahaṃsesi. Atha kho cundo kammāraputto bhagavatā dhammiyā kathāya sandassito samādapito samuttejito sampahaṃsito bhagavantaṃ etadavoca – ‘‘adhivāsetu me, bhante, bhagavā svātanāya bhattaṃ saddhiṃ bhikkhusaṅghenā’’ti. Adhivāsesi bhagavā tuṇhībhāvena. Atha kho cundo kammāraputto bhagavato adhivāsanaṃ viditvā uṭṭhāyāsanā bhagavantaṃ abhivādetvā padakkhiṇaṃ katvā pakkāmi.

Atha kho cundo kammāraputto tassā rattiyā accayena sake nivesane paṇītaṃ khādanīyaṃ bhojanīyaṃ paṭiyādāpetvā pahūtañca sūkaramaddavaṃ bhagavato kālaṃ ārocāpesi – ‘‘kālo, bhante, niṭṭhitaṃ bhatta’’nti. Atha kho bhagavā pubbaṇhasamayaṃ nivāsetvā pattacīvaramādāya saddhiṃ bhikkhusaṅghena yena cundassa kammāraputtassa nivesanaṃ tenupasaṅkami; upasaṅkamitvā paññatte āsane nisīdi. Nisajja kho bhagavā cundaṃ kammāraputtaṃ āmantesi – ‘‘yaṃ te, cunda, sūkaramaddavaṃ paṭiyattaṃ, tena maṃ parivisa. Yaṃ panaññaṃ khādanīyaṃ bhojanīyaṃ paṭiyattaṃ, tena bhikkhusaṅghaṃ parivisā’’ti. ‘‘Evaṃ, bhante’’ti kho cundo kammāraputto bhagavato paṭissutvā yaṃ ahosi sūkaramaddavaṃ paṭiyattaṃ, tena bhagavantaṃ parivisi. Yaṃ panaññaṃ khādanīyaṃ bhojanīyaṃ paṭiyattaṃ , tena bhikkhusaṅghaṃ parivisi. Atha kho bhagavā cundaṃ kammāraputtaṃ āmantesi – ‘‘yaṃ te, cunda, sūkaramaddavaṃ avasiṭṭhaṃ, taṃ sobbhe nikhaṇāhi. Nāhaṃ taṃ, cunda, passāmi sadevake loke samārake sabrahmake sassamaṇabrāhmaṇiyā pajāya sadevamanussāya, yassa taṃ paribhuttaṃ sammā pariṇāmaṃ gaccheyya aññatra tathāgatassā’’ti. ‘‘Evaṃ, bhante’’ti kho cundo kammāraputto bhagavato paṭissutvā yaṃ ahosi sūkaramaddavaṃ avasiṭṭhaṃ, taṃ sobbhe nikhaṇitvā yena bhagavā tenupasaṅkami; upasaṅkamitvā bhagavantaṃ abhivādetvā ekamantaṃ nisīdi. Ekamantaṃ nisinnaṃ kho cundaṃ kammāraputtaṃ bhagavā dhammiyā kathāya sandassetvā samādapetvā samuttejetvā sampahaṃsetvā uṭṭhāyāsanā pakkāmi.

\paragraph{190.} Atha kho bhagavato cundassa kammāraputtassa bhattaṃ bhuttāvissa kharo ābādho uppajji, lohitapakkhandikā pabāḷhā vedanā vattanti māraṇantikā. Tā sudaṃ bhagavā sato sampajāno adhivāsesi avihaññamāno. Atha kho bhagavā āyasmantaṃ ānandaṃ āmantesi – ‘‘āyāmānanda, yena kusinārā tenupasaṅkamissāmā’’ti. ‘‘Evaṃ, bhante’’ti kho āyasmā ānando bhagavato paccassosi.

Cundassa bhattaṃ bhuñjitvā, kammārassāti me sutaṃ;

Ābādhaṃ samphusī dhīro, pabāḷhaṃ māraṇantikaṃ.

Bhuttassa ca sūkaramaddavena,

Byādhippabāḷho udapādi satthuno;

Virecamāno\footnote{viriccamāno (sī. syā. ka.), viriñcamāno (?)} bhagavā avoca,

Gacchāmahaṃ kusināraṃ nagaranti.

\subsubsection{Pānīyāharaṇaṃ}

\paragraph{191.} Atha kho bhagavā maggā okkamma yena aññataraṃ rukkhamūlaṃ tenupasaṅkami; upasaṅkamitvā āyasmantaṃ ānandaṃ āmantesi – ‘‘iṅgha me tvaṃ, ānanda, catugguṇaṃ saṅghāṭiṃ paññapehi, kilantosmi, ānanda, nisīdissāmī’’ti. ‘‘Evaṃ, bhante’’ti kho āyasmā ānando bhagavato paṭissutvā catugguṇaṃ saṅghāṭiṃ paññapesi. Nisīdi bhagavā paññatte āsane. Nisajja kho bhagavā āyasmantaṃ ānandaṃ āmantesi – ‘‘iṅgha me tvaṃ, ānanda, pānīyaṃ āhara, pipāsitosmi, ānanda, pivissāmī’’ti. Evaṃ vutte āyasmā ānando bhagavantaṃ etadavoca – ‘‘idāni, bhante, pañcamattāni sakaṭasatāni atikkantāni, taṃ cakkacchinnaṃ udakaṃ parittaṃ luḷitaṃ āvilaṃ sandati. Ayaṃ, bhante, kakudhā\footnote{kakuthā (sī. pī.)} nadī avidūre acchodakā sātodakā sītodakā setodakā\footnote{setakā (sī.)} suppatitthā ramaṇīyā. Ettha bhagavā pānīyañca pivissati, gattāni ca sītī\footnote{sītaṃ (sī. pī. ka.)} karissatī’’ti.

Dutiyampi kho bhagavā āyasmantaṃ ānandaṃ āmantesi – ‘‘iṅgha me tvaṃ, ānanda, pānīyaṃ āhara, pipāsitosmi, ānanda, pivissāmī’’ti. Dutiyampi kho āyasmā ānando bhagavantaṃ etadavoca – ‘‘idāni, bhante, pañcamattāni sakaṭasatāni atikkantāni, taṃ cakkacchinnaṃ udakaṃ parittaṃ luḷitaṃ āvilaṃ sandati. Ayaṃ, bhante, kakudhā nadī avidūre acchodakā sātodakā sītodakā setodakā suppatitthā ramaṇīyā. Ettha bhagavā pānīyañca pivissati, gattāni ca sītīkarissatī’’ti.

Tatiyampi kho bhagavā āyasmantaṃ ānandaṃ āmantesi – ‘‘iṅgha me tvaṃ, ānanda, pānīyaṃ āhara, pipāsitosmi, ānanda, pivissāmī’’ti. ‘‘Evaṃ, bhante’’ti kho āyasmā ānando bhagavato paṭissutvā pattaṃ gahetvā yena sā nadikā tenupasaṅkami. Atha kho sā nadikā cakkacchinnā parittā luḷitā āvilā sandamānā, āyasmante ānande upasaṅkamante acchā vippasannā anāvilā sandittha\footnote{sandati (syā.)}. Atha kho āyasmato ānandassa etadahosi – ‘‘acchariyaṃ vata, bho, abbhutaṃ vata, bho, tathāgatassa mahiddhikatā mahānubhāvatā. Ayañhi sā nadikā cakkacchinnā parittā luḷitā āvilā sandamānā mayi upasaṅkamante acchā vippasannā anāvilā sandatī’’ti. Pattena pānīyaṃ ādāya yena bhagavā tenupasaṅkami; upasaṅkamitvā bhagavantaṃ etadavoca – ‘‘acchariyaṃ, bhante, abbhutaṃ, bhante, tathāgatassa mahiddhikatā mahānubhāvatā. Idāni sā bhante nadikā cakkacchinnā parittā luḷitā āvilā sandamānā mayi upasaṅkamante acchā vippasannā anāvilā sandittha. Pivatu bhagavā pānīyaṃ pivatu sugato pānīya’’nti. Atha kho bhagavā pānīyaṃ apāyi.

\subsubsection{Pukkusamallaputtavatthu}

\paragraph{192.} Tena rokho pana samayena pukkuso mallaputto āḷārassa kālāmassa sāvako kusinārāya pāvaṃ addhānamaggappaṭippanno hoti. Addasā kho pukkuso mallaputto bhagavantaṃ aññatarasmiṃ rukkhamūle nisinnaṃ. Disvā yena bhagavā tenupasaṅkami; upasaṅkamitvā bhagavantaṃ abhivādetvā ekamantaṃ nisīdi. Ekamantaṃ nisinno kho pukkuso mallaputto bhagavantaṃ etadavoca – ‘‘acchariyaṃ, bhante, abbhutaṃ, bhante, santena vata, bhante, pabbajitā vihārena viharanti. Bhūtapubbaṃ, bhante , āḷāro kālāmo addhānamaggappaṭippanno maggā okkamma avidūre aññatarasmiṃ rukkhamūle divāvihāraṃ nisīdi. Atha kho, bhante, pañcamattāni sakaṭasatāni āḷāraṃ kālāmaṃ nissāya nissāya atikkamiṃsu. Atha kho, bhante, aññataro puriso tassa sakaṭasatthassa\footnote{sakaṭasatassa (ka.)} piṭṭhito piṭṭhito āgacchanto yena āḷāro kālāmo tenupasaṅkami; upasaṅkamitvā āḷāraṃ kālāmaṃ etadavoca – ‘api, bhante, pañcamattāni sakaṭasatāni atikkantāni addasā’ti? ‘Na kho ahaṃ, āvuso, addasa’nti. ‘Kiṃ pana, bhante, saddaṃ assosī’ti? ‘Na kho ahaṃ, āvuso, saddaṃ assosi’nti. ‘Kiṃ pana, bhante, sutto ahosī’ti? ‘Na kho ahaṃ, āvuso, sutto ahosi’nti. ‘Kiṃ pana, bhante, saññī ahosī’ti? ‘Evamāvuso’ti. ‘So tvaṃ, bhante, saññī samāno jāgaro pañcamattāni sakaṭasatāni nissāya nissāya atikkantāni neva addasa, na pana saddaṃ assosi; apisu\footnote{api hi (sī. syā. pī.)} te, bhante, saṅghāṭi rajena okiṇṇā’ti? ‘Evamāvuso’ti. Atha kho, bhante, tassa purisassa etadahosi – ‘acchariyaṃ vata bho, abbhutaṃ vata bho, santena vata bho pabbajitā vihārena viharanti. Yatra hi nāma saññī samāno jāgaro pañcamattāni sakaṭasatāni nissāya nissāya atikkantāni neva dakkhati, na pana saddaṃ sossatī’ti! Āḷāre kālāme uḷāraṃ pasādaṃ pavedetvā pakkāmī’’ti.

\paragraph{193.} ‘‘Taṃ kiṃ maññasi, pukkusa, katamaṃ nu kho dukkarataraṃ vā durabhisambhavataraṃ vā – yo vā saññī samāno jāgaro pañcamattāni sakaṭasatāni nissāya nissāya atikkantāni neva passeyya, na pana saddaṃ suṇeyya; yo vā saññī samāno jāgaro deve vassante deve gaḷagaḷāyante vijjullatāsu\footnote{vijjutāsu (sī. syā. pī.)} niccharantīsu asaniyā phalantiyā neva passeyya, na pana saddaṃ suṇeyyā’’ti? ‘‘Kiñhi, bhante, karissanti pañca vā sakaṭasatāni cha vā sakaṭasatāni satta vā sakaṭasatāni aṭṭha vā sakaṭasatāni nava vā sakaṭasatāni\footnote{nava vā sakaṭasatāni dasa vā sakaṭasatāni (sī.)}, sakaṭasahassaṃ vā sakaṭasatasahassaṃ vā. Atha kho etadeva dukkarataraṃ ceva durabhisambhavatarañca yo saññī samāno jāgaro deve vassante deve gaḷagaḷāyante vijjullatāsu niccharantīsu asaniyā phalantiyā neva passeyya, na pana saddaṃ suṇeyyā’’ti.

‘‘Ekamidāhaṃ, pukkusa, samayaṃ ātumāyaṃ viharāmi bhusāgāre. Tena kho pana samayena deve vassante deve gaḷagaḷāyante vijjullatāsu niccharantīsu asaniyā phalantiyā avidūre bhusāgārassa dve kassakā bhātaro hatā cattāro ca balibaddā\footnote{balibaddā (sī. pī.)}. Atha kho, pukkusa, ātumāya mahājanakāyo nikkhamitvā yena te dve kassakā bhātaro hatā cattāro ca balibaddā tenupasaṅkami. Tena kho panāhaṃ, pukkusa, samayena bhusāgārā nikkhamitvā bhusāgāradvāre abbhokāse caṅkamāmi. Atha kho, pukkusa, aññataro puriso tamhā mahājanakāyā yenāhaṃ tenupasaṅkami; upasaṅkamitvā maṃ abhivādetvā ekamantaṃ aṭṭhāsi. Ekamantaṃ ṭhitaṃ kho ahaṃ, pukkusa, taṃ purisaṃ etadavocaṃ – ‘kiṃ nu kho eso, āvuso, mahājanakāyo sannipatito’ti? ‘Idāni , bhante, deve vassante deve gaḷagaḷāyante vijjullatāsu niccharantīsu asaniyā phalantiyā dve kassakā bhātaro hatā cattāro ca balibaddā. Ettheso mahājanakāyo sannipatito. Tvaṃ pana, bhante, kva ahosī’ti? ‘Idheva kho ahaṃ, āvuso, ahosi’nti. ‘Kiṃ pana, bhante, addasā’ti? ‘Na kho ahaṃ, āvuso, addasa’nti. ‘Kiṃ pana, bhante, saddaṃ assosī’ti? ‘Na kho ahaṃ, āvuso, saddaṃ assosi’nti. ‘Kiṃ pana, bhante, sutto ahosī’ti? ‘Na kho ahaṃ, āvuso, sutto ahosi’nti. ‘Kiṃ pana, bhante, saññī ahosī’ti? ‘Evamāvuso’ti. ‘So tvaṃ, bhante, saññī samāno jāgaro deve vassante deve gaḷagaḷāyante vijjullatāsu niccharantīsu asaniyā phalantiyā neva addasa, na pana saddaṃ assosī’ti? ‘‘Evamāvuso’’ti?

‘‘Atha kho, pukkusa, purisassa etadahosi – ‘acchariyaṃ vata bho, abbhutaṃ vata bho, santena vata bho pabbajitā vihārena viharanti. Yatra hi nāma saññī samāno jāgaro deve vassante deve gaḷagaḷāyante vijjullatāsu niccharantīsu asaniyā phalantiyā neva dakkhati, na pana saddaṃ sossatī’ti\footnote{suṇissati (syā.)}. Mayi uḷāraṃ pasādaṃ pavedetvā maṃ abhivādetvā padakkhiṇaṃ katvā pakkāmī’’ti.

Evaṃ vutte pukkuso mallaputto bhagavantaṃ etadavoca – ‘‘esāhaṃ, bhante, yo me āḷāre kālāme pasādo taṃ mahāvāte vā ophuṇāmi sīghasotāya\footnote{siṅghasotāya (ka.)} vā nadiyā pavāhemi. Abhikkantaṃ, bhante, abhikkantaṃ, bhante! Seyyathāpi, bhante, nikkujjitaṃ vā ukkujjeyya, paṭicchannaṃ vā vivareyya, mūḷhassa vā maggaṃ ācikkheyya, andhakāre vā telapajjotaṃ dhāreyya ‘cakkhumanto rūpāni dakkhantī’ti; evamevaṃ bhagavatā anekapariyāyena dhammo pakāsito. Esāhaṃ, bhante, bhagavantaṃ saraṇaṃ gacchāmi dhammañca bhikkhusaṅghañca. Upāsakaṃ maṃ bhagavā dhāretu ajjatagge pāṇupetaṃ saraṇaṃ gata’’nti.

\paragraph{194.} Atha kho pukkuso mallaputto aññataraṃ purisaṃ āmantesi – ‘‘iṅgha me tvaṃ, bhaṇe, siṅgīvaṇṇaṃ yugamaṭṭhaṃ dhāraṇīyaṃ āharā’’ti. ‘‘Evaṃ, bhante’’ti kho so puriso pukkusassa mallaputtassa paṭissutvā taṃ siṅgīvaṇṇaṃ yugamaṭṭhaṃ dhāraṇīyaṃ āhari\footnote{āharasi (ka.)}. Atha kho pukkuso mallaputto taṃ siṅgīvaṇṇaṃ yugamaṭṭhaṃ dhāraṇīyaṃ bhagavato upanāmesi – ‘‘idaṃ, bhante, siṅgīvaṇṇaṃ yugamaṭṭhaṃ dhāraṇīyaṃ, taṃ me bhagavā paṭiggaṇhātu anukampaṃ upādāyā’’ti. ‘‘Tena hi, pukkusa, ekena maṃ acchādehi, ekena ānanda’’nti. ‘‘Evaṃ, bhante’’ti kho pukkuso mallaputto bhagavato paṭissutvā ekena bhagavantaṃ acchādeti, ekena āyasmantaṃ ānandaṃ. Atha kho bhagavā pukkusaṃ mallaputtaṃ dhammiyā kathāya sandassesi samādapesi samuttejesi sampahaṃsesi. Atha kho pukkuso mallaputto bhagavatā dhammiyā kathāya sandassito samādapito samuttejito sampahaṃsito uṭṭhāyāsanā bhagavantaṃ abhivādetvā padakkhiṇaṃ katvā pakkāmi.

\paragraph{195.} Atha kho āyasmā ānando acirapakkante pukkuse mallaputte taṃ siṅgīvaṇṇaṃ yugamaṭṭhaṃ dhāraṇīyaṃ bhagavato kāyaṃ upanāmesi. Taṃ bhagavato kāyaṃ upanāmitaṃ hataccikaṃ viya\footnote{vītaccikaṃviya (sī. pī.)} khāyati. Atha kho āyasmā ānando bhagavantaṃ etadavoca – ‘‘acchariyaṃ, bhante, abbhutaṃ, bhante, yāva parisuddho, bhante, tathāgatassa chavivaṇṇo pariyodāto. Idaṃ, bhante, siṅgīvaṇṇaṃ yugamaṭṭhaṃ dhāraṇīyaṃ bhagavato kāyaṃ upanāmitaṃ hataccikaṃ viya khāyatī’’ti. ‘‘Evametaṃ, ānanda, evametaṃ, ānanda dvīsu kālesu ativiya tathāgatassa kāyo parisuddho hoti chavivaṇṇo pariyodāto. Katamesu dvīsu? Yañca, ānanda, rattiṃ tathāgato anuttaraṃ sammāsambodhiṃ abhisambujjhati, yañca rattiṃ anupādisesāya nibbānadhātuyā parinibbāyati. Imesu kho, ānanda, dvīsu kālesu ativiya tathāgatassa kāyo parisuddho hoti chavivaṇṇo pariyodāto. ‘‘Ajja kho, panānanda, rattiyā pacchime yāme kusinārāyaṃ upavattane mallānaṃ sālavane antarena\footnote{antare (syā.)} yamakasālānaṃ tathāgatassa parinibbānaṃ bhavissati\footnote{bhavissatīti (ka.)}. Āyāmānanda, yena kakudhā nadī tenupasaṅkamissāmā’’ti. ‘‘Evaṃ, bhante’’ti kho āyasmā ānando bhagavato paccassosi.

Siṅgīvaṇṇaṃ yugamaṭṭhaṃ, pukkuso abhihārayi;

Tena acchādito satthā, hemavaṇṇo asobhathāti.

\paragraph{196.} Atha kho bhagavā mahatā bhikkhusaṅghena saddhiṃ yena kakudhā nadī tenupasaṅkami ; upasaṅkamitvā kakudhaṃ nadiṃ ajjhogāhetvā nhatvā ca pivitvā ca paccuttaritvā yena ambavanaṃ tenupasaṅkami. Upasaṅkamitvā āyasmantaṃ cundakaṃ āmantesi – ‘‘iṅgha me tvaṃ, cundaka, catugguṇaṃ saṅghāṭiṃ paññapehi, kilantosmi, cundaka, nipajjissāmī’’ti.

‘‘Evaṃ, bhante’’ti kho āyasmā cundako bhagavato paṭissutvā catugguṇaṃ saṅghāṭiṃ paññapesi. Atha kho bhagavā dakkhiṇena passena sīhaseyyaṃ kappesi pāde pādaṃ accādhāya sato sampajāno uṭṭhānasaññaṃ manasikaritvā. Āyasmā pana cundako tattheva bhagavato purato nisīdi.

Gantvāna buddho nadikaṃ kakudhaṃ,

Acchodakaṃ sātudakaṃ vippasannaṃ;

Ogāhi satthā akilantarūpo\footnote{sukilantarūpo (sī. pī.)},

Tathāgato appaṭimo ca\footnote{appaṭimodha (pī.)} loke.

Nhatvā ca pivitvā cudatāri satthā\footnote{pivitvā cundakena, pivitvā ca uttari (ka.)},

Purakkhato bhikkhugaṇassa majjhe;

Vattā\footnote{satthā (sī. syā. pī.)} pavattā bhagavā idha dhamme,

Upāgami ambavanaṃ mahesi.

Āmantayi cundakaṃ nāma bhikkhuṃ,

Catugguṇaṃ santhara me nipajjaṃ;

So codito bhāvitattena cundo,

Catugguṇaṃ santhari khippameva.

Nipajji satthā akilantarūpo,

Cundopi tattha pamukhe\footnote{samukhe (ka.)} nisīdīti.

\paragraph{197.} Atha kho bhagavā āyasmantaṃ ānandaṃ āmantesi – ‘‘siyā kho\footnote{yo kho (ka.)}, panānanda, cundassa kammāraputtassa koci vippaṭisāraṃ uppādeyya – ‘tassa te, āvuso cunda, alābhā tassa te dulladdhaṃ, yassa te tathāgato pacchimaṃ piṇḍapātaṃ paribhuñjitvā parinibbuto’ti. Cundassa, ānanda, kammāraputtassa evaṃ vippaṭisāro paṭivinetabbo – ‘tassa te, āvuso cunda, lābhā tassa te suladdhaṃ, yassa te tathāgato pacchimaṃ piṇḍapātaṃ paribhuñjitvā parinibbuto. Sammukhā metaṃ, āvuso cunda, bhagavato sutaṃ sammukhā paṭiggahitaṃ – dve me piṇḍapātā samasamaphalā\footnote{samā samaphalā (ka.)} samavipākā\footnote{samasamavipākā (sī. syā. pī.)}, ativiya aññehi piṇḍapātehi mahapphalatarā ca mahānisaṃsatarā ca. Katame dve? Yañca piṇḍapātaṃ paribhuñjitvā tathāgato anuttaraṃ sammāsambodhiṃ abhisambujjhati, yañca piṇḍapātaṃ paribhuñjitvā tathāgato anupādisesāya nibbānadhātuyā parinibbāyati. Ime dve piṇḍapātā samasamaphalā samavipākā , ativiya aññehi piṇḍapātehi mahapphalatarā ca mahānisaṃsatarā ca. Āyusaṃvattanikaṃ āyasmatā cundena kammāraputtena kammaṃ upacitaṃ, vaṇṇasaṃvattanikaṃ āyasmatā cundena kammāraputtena kammaṃ upacitaṃ, sukhasaṃvattanikaṃ āyasmatā cundena kammāraputtena kammaṃ upacitaṃ, yasasaṃvattanikaṃ āyasmatā cundena kammāraputtena kammaṃ upacitaṃ, saggasaṃvattanikaṃ āyasmatā cundena kammāraputtena kammaṃ upacitaṃ, ādhipateyyasaṃvattanikaṃ āyasmatā cundena kammāraputtena kammaṃ upacita’nti. Cundassa, ānanda, kammāraputtassa evaṃ vippaṭisāro paṭivinetabbo’’ti. Atha kho bhagavā etamatthaṃ viditvā tāyaṃ velāyaṃ imaṃ udānaṃ udānesi –

‘‘Dadato puññaṃ pavaḍḍhati,

Saṃyamato veraṃ na cīyati;

Kusalo ca jahāti pāpakaṃ,

Rāgadosamohakkhayā sanibbuto’’ti.

\xsubsubsectionEnd{Catuttho bhāṇavāro.}

\subsubsection{Yamakasālā}

\paragraph{198.} Atha kho bhagavā āyasmantaṃ ānandaṃ āmantesi – ‘‘āyāmānanda, yena hiraññavatiyā nadiyā pārimaṃ tīraṃ, yena kusinārā upavattanaṃ mallānaṃ sālavanaṃ tenupasaṅkamissāmā’’ti . ‘‘Evaṃ, bhante’’ti kho āyasmā ānando bhagavato paccassosi. Atha kho bhagavā mahatā bhikkhusaṅghena saddhiṃ yena hiraññavatiyā nadiyā pārimaṃ tīraṃ, yena kusinārā upavattanaṃ mallānaṃ sālavanaṃ tenupasaṅkami. Upasaṅkamitvā āyasmantaṃ ānandaṃ āmantesi – ‘‘iṅgha me tvaṃ, ānanda, antarena yamakasālānaṃ uttarasīsakaṃ mañcakaṃ paññapehi, kilantosmi, ānanda, nipajjissāmī’’ti. ‘‘Evaṃ, bhante’’ti kho āyasmā ānando bhagavato paṭissutvā antarena yamakasālānaṃ uttarasīsakaṃ mañcakaṃ paññapesi. Atha kho bhagavā dakkhiṇena passena sīhaseyyaṃ kappesi pāde pādaṃ accādhāya sato sampajāno.

Tena kho pana samayena yamakasālā sabbaphāliphullā honti akālapupphehi. Te tathāgatassa sarīraṃ okiranti ajjhokiranti abhippakiranti tathāgatassa pūjāya. Dibbānipi mandāravapupphāni antalikkhā papatanti, tāni tathāgatassa sarīraṃ okiranti ajjhokiranti abhippakiranti tathāgatassa pūjāya. Dibbānipi candanacuṇṇāni antalikkhā papatanti, tāni tathāgatassa sarīraṃ okiranti ajjhokiranti abhippakiranti tathāgatassa pūjāya. Dibbānipi tūriyāni antalikkhe vajjanti tathāgatassa pūjāya. Dibbānipi saṅgītāni antalikkhe vattanti tathāgatassa pūjāya.

\paragraph{199.} Atha kho bhagavā āyasmantaṃ ānandaṃ āmantesi – ‘‘sabbaphāliphullā kho, ānanda, yamakasālā akālapupphehi. Te tathāgatassa sarīraṃ okiranti ajjhokiranti abhippakiranti tathāgatassa pūjāya. Dibbānipi mandāravapupphāni antalikkhā papatanti, tāni tathāgatassa sarīraṃ okiranti ajjhokiranti abhippakiranti tathāgatassa pūjāya. Dibbānipi candanacuṇṇāni antalikkhā papatanti, tāni tathāgatassa sarīraṃ okiranti ajjhokiranti abhippakiranti tathāgatassa pūjāya. Dibbānipi tūriyāni antalikkhe vajjanti tathāgatassa pūjāya. Dibbānipi saṅgītāni antalikkhe vattanti tathāgatassa pūjāya. Na kho, ānanda, ettāvatā tathāgato sakkato vā hoti garukato vā mānito vā pūjito vā apacito vā. Yo kho, ānanda, bhikkhu vā bhikkhunī vā upāsako vā upāsikā vā dhammānudhammappaṭipanno viharati sāmīcippaṭipanno anudhammacārī, so tathāgataṃ sakkaroti garuṃ karoti māneti pūjeti apaciyati\footnote{idaṃ padaṃ sīsyāipotthakesu na dissati}, paramāya pūjāya. Tasmātihānanda, dhammānudhammappaṭipannā viharissāma sāmīcippaṭipannā anudhammacārinoti. Evañhi vo, ānanda, sikkhitabba’’nti.

\subsubsection{Upavāṇatthero}

\paragraph{200.} Tena kho pana samayena āyasmā upavāṇo bhagavato purato ṭhito hoti bhagavantaṃ bījayamāno. Atha kho bhagavā āyasmantaṃ upavāṇaṃ apasāresi – ‘‘apehi, bhikkhu, mā me purato aṭṭhāsī’’ti. Atha kho āyasmato ānandassa etadahosi – ‘‘ayaṃ kho āyasmā upavāṇo dīgharattaṃ bhagavato upaṭṭhāko santikāvacaro samīpacārī. Atha ca pana bhagavā pacchime kāle āyasmantaṃ upavāṇaṃ apasāreti – ‘apehi bhikkhu, mā me purato aṭṭhāsī’ti. Ko nu kho hetu, ko paccayo, yaṃ bhagavā āyasmantaṃ upavāṇaṃ apasāreti – ‘apehi, bhikkhu, mā me purato aṭṭhāsī’ti? Atha kho āyasmā ānando bhagavantaṃ etadavoca – ‘ayaṃ, bhante, āyasmā upavāṇo dīgharattaṃ bhagavato upaṭṭhāko santikāvacaro samīpacārī. Atha ca pana bhagavā pacchime kāle āyasmantaṃ upavāṇaṃ apasāreti – ‘‘apehi, bhikkhu, mā me purato aṭṭhāsī’’ti. Ko nu kho, bhante, hetu, ko paccayo, yaṃ bhagavā āyasmantaṃ upavāṇaṃ apasāreti – ‘‘apehi, bhikkhu, mā me purato aṭṭhāsī’’ti? ‘‘Yebhuyyena, ānanda, dasasu lokadhātūsu devatā sannipatitā tathāgataṃ dassanāya. Yāvatā, ānanda, kusinārā upavattanaṃ mallānaṃ sālavanaṃ samantato dvādasa yojanāni, natthi so padeso vālaggakoṭinitudanamattopi mahesakkhāhi devatāhi apphuṭo. Devatā, ānanda, ujjhāyanti – ‘dūrā ca vatamha āgatā tathāgataṃ dassanāya. Kadāci karahaci tathāgatā loke uppajjanti arahanto sammāsambuddhā. Ajjeva rattiyā pacchime yāme tathāgatassa parinibbānaṃ bhavissati. Ayañca mahesakkho bhikkhu bhagavato purato ṭhito ovārento, na mayaṃ labhāma pacchime kāle tathāgataṃ dassanāyā’’’ti.

\paragraph{201.} ‘‘Kathaṃbhūtā pana, bhante, bhagavā devatā manasikarotī’’ti\footnote{manasi karontīti (syā. ka.)}? ‘‘Santānanda, devatā ākāse pathavīsaññiniyo kese pakiriya kandanti, bāhā paggayha kandanti, chinnapātaṃ papatanti\footnote{chinnaṃpādaṃviya papatanti (syā.)}, āvaṭṭanti, vivaṭṭanti – ‘atikhippaṃ bhagavā parinibbāyissati, atikhippaṃ sugato parinibbāyissati, atikhippaṃ cakkhuṃ\footnote{cakkhumā (syā. ka.)} loke antaradhaṃāyissatī’ti.

‘‘Santānanda, devatā pathaviyaṃ pathavīsaññiniyo kese pakiriya kandanti, bāhā paggayha kandanti, chinnapātaṃ papatanti, āvaṭṭanti, vivaṭṭanti – ‘atikhippaṃ bhagavā parinibbāyissati, atikhippaṃ sugato parinibbāyissati, atikhippaṃ cakkhuṃ loke antaradhāyissatī’’’ti.

‘‘Yā pana tā devatā vītarāgā, tā satā sampajānā adhivāsenti – ‘aniccā saṅkhārā, taṃ kutettha labbhā’ti.

\subsubsection{Catusaṃvejanīyaṭṭhānāni}

\paragraph{202.} ‘‘Pubbe , bhante, disāsu vassaṃ vuṭṭhā\footnote{vassaṃvutthā (sī. syā. kaṃ. pī.)} bhikkhū āgacchanti tathāgataṃ dassanāya. Te mayaṃ labhāma manobhāvanīye bhikkhū dassanāya, labhāma payirupāsanāya. Bhagavato pana mayaṃ, bhante, accayena na labhissāma manobhāvanīye bhikkhū dassanāya, na labhissāma payirupāsanāyā’’ti.

‘‘Cattārimāni, ānanda, saddhassa kulaputtassa dassanīyāni saṃvejanīyāni ṭhānāni. Katamāni cattāri? ‘Idha tathāgato jāto’ti, ānanda, saddhassa kulaputtassa dassanīyaṃ saṃvejanīyaṃ ṭhānaṃ. ‘Idha tathāgato anuttaraṃ sammāsambodhiṃ abhisambuddho’ti, ānanda, saddhassa kulaputtassa dassanīyaṃ saṃvejanīyaṃ ṭhānaṃ. ‘Idha tathāgatena anuttaraṃ dhammacakkaṃ pavattita’nti, ānanda, saddhassa kulaputtassa dassanīyaṃ saṃvejanīyaṃ ṭhānaṃ. ‘Idha tathāgato anupādisesāya nibbānadhātuyā parinibbuto’ti, ānanda, saddhassa kulaputtassa dassanīyaṃ saṃvejanīyaṃ ṭhānaṃ. Imāni kho , ānanda, cattāri saddhassa kulaputtassa dassanīyāni saṃvejanīyāni ṭhānāni.

‘‘Āgamissanti kho, ānanda, saddhā bhikkhū bhikkhuniyo upāsakā upāsikāyo – ‘idha tathāgato jāto’tipi, ‘idha tathāgato anuttaraṃ sammāsambodhiṃ abhisambuddho’tipi, ‘idha tathāgatena anuttaraṃ dhammacakkaṃ pavattita’ntipi, ‘idha tathāgato anupādisesāya nibbānadhātuyā parinibbuto’tipi. Ye hi keci, ānanda, cetiyacārikaṃ āhiṇḍantā pasannacittā kālaṅkarissanti, sabbe te kāyassa bhedā paraṃ maraṇā sugatiṃ saggaṃ lokaṃ upapajjissantī’’ti.

\subsubsection{Ānandapucchākathā}

\paragraph{203.} ‘‘Kathaṃ mayaṃ, bhante, mātugāme paṭipajjāmā’’ti? ‘‘Adassanaṃ, ānandā’’ti. ‘‘Dassane, bhagavā, sati kathaṃ paṭipajjitabba’’nti? ‘‘Anālāpo, ānandā’’ti . ‘‘Ālapantena pana, bhante, kathaṃ paṭipajjitabba’’nti? ‘‘Sati, ānanda, upaṭṭhāpetabbā’’ti.

\paragraph{204.} ‘‘Kathaṃ mayaṃ, bhante, tathāgatassa sarīre paṭipajjāmā’’ti? ‘‘Abyāvaṭā tumhe, ānanda, hotha tathāgatassa sarīrapūjāya. Iṅgha tumhe, ānanda, sāratthe ghaṭatha anuyuñjatha\footnote{sadatthe anuyuñjatha (sī. syā.), sadatthaṃ anuyuñjatha (pī.), sāratthe anuyuñjatha (ka.)}, sāratthe appamattā ātāpino pahitattā viharatha. Santānanda, khattiyapaṇḍitāpi brāhmaṇapaṇḍitāpi gahapatipaṇḍitāpi tathāgate abhippasannā, te tathāgatassa sarīrapūjaṃ karissantī’’ti.

\paragraph{205.} ‘‘Kathaṃ pana, bhante, tathāgatassa sarīre paṭipajjitabba’’nti? ‘‘Yathā kho, ānanda, rañño cakkavattissa sarīre paṭipajjanti, evaṃ tathāgatassa sarīre paṭipajjitabba’’nti. ‘‘Kathaṃ pana, bhante, rañño cakkavattissa sarīre paṭipajjantī’’ti? ‘‘Rañño, ānanda, cakkavattissa sarīraṃ ahatena vatthena veṭhenti, ahatena vatthena veṭhetvā vihatena kappāsena veṭhenti, vihatena kappāsena veṭhetvā ahatena vatthena veṭhenti. Etenupāyena pañcahi yugasatehi rañño cakkavattissa sarīraṃ\footnote{sarīre (syā. ka.)} veṭhetvā āyasāya teladoṇiyā pakkhipitvā aññissā āyasāya doṇiyā paṭikujjitvā sabbagandhānaṃ citakaṃ karitvā rañño cakkavattissa sarīraṃ jhāpenti. Cātumahāpathe\footnote{cātummahāpathe (sī. syā. kaṃ. pī.)} rañño cakkavattissa thūpaṃ karonti . Evaṃ kho, ānanda, rañño cakkavattissa sarīre paṭipajjanti. Yathā kho, ānanda, rañño cakkavattissa sarīre paṭipajjanti, evaṃ tathāgatassa sarīre paṭipajjitabbaṃ. Cātumahāpathe tathāgatassa thūpo kātabbo. Tattha ye mālaṃ vā gandhaṃ vā cuṇṇakaṃ\footnote{vaṇṇakaṃ (sī. pī.)} vā āropessanti vā abhivādessanti vā cittaṃ vā pasādessanti tesaṃ taṃ bhavissati dīgharattaṃ hitāya sukhāya.

\subsubsection{Thūpārahapuggalo}

\paragraph{206.} ‘‘Cattārome, ānanda, thūpārahā. Katame cattāro? Tathāgato arahaṃ sammāsambuddho thūpāraho, paccekasambuddho thūpāraho, tathāgatassa sāvako thūpāraho, rājā cakkavattī\footnote{cakkavatti (syā. ka.)} thūpārahoti.

‘‘Kiñcānanda , atthavasaṃ paṭicca tathāgato arahaṃ sammāsambuddho thūpāraho? ‘Ayaṃ tassa bhagavato arahato sammāsambuddhassa thūpo’ti, ānanda, bahujanā cittaṃ pasādenti. Te tattha cittaṃ pasādetvā kāyassa bhedā paraṃ maraṇā sugatiṃ saggaṃ lokaṃ upapajjanti. Idaṃ kho, ānanda, atthavasaṃ paṭicca tathāgato arahaṃ sammāsambuddho thūpāraho.

‘‘Kiñcānanda, atthavasaṃ paṭicca paccekasambuddho thūpāraho? ‘Ayaṃ tassa bhagavato paccekasambuddhassa thūpo’ti, ānanda, bahujanā cittaṃ pasādenti. Te tattha cittaṃ pasādetvā kāyassa bhedā paraṃ maraṇā sugatiṃ saggaṃ lokaṃ upapajjanti. Idaṃ kho, ānanda, atthavasaṃ paṭicca paccekasambuddho thūpāraho.

‘‘Kiñcānanda, atthavasaṃ paṭicca tathāgatassa sāvako thūpāraho? ‘Ayaṃ tassa bhagavato arahato sammāsambuddhassa sāvakassa thūpo’ti ānanda, bahujanā cittaṃ pasādenti. Te tattha cittaṃ pasādetvā kāyassa bhedā paraṃ maraṇā sugatiṃ saggaṃ lokaṃ upapajjanti. Idaṃ kho, ānanda, atthavasaṃ paṭicca tathāgatassa sāvako thūpāraho.

‘‘Kiñcānanda, atthavasaṃ paṭicca rājā cakkavattī thūpāraho? ‘Ayaṃ tassa dhammikassa dhammarañño thūpo’ti, ānanda, bahujanā cittaṃ pasādenti. Te tattha cittaṃ pasādetvā kāyassa bhedā paraṃ maraṇā sugatiṃ saggaṃ lokaṃ upapajjanti. Idaṃ kho, ānanda, atthavasaṃ paṭicca rājā cakkavattī thūpāraho. Ime kho, ānanda cattāro thūpārahā’’ti.

\subsubsection{Ānandaacchariyadhammo}

\paragraph{207.} Atha kho āyasmā ānando vihāraṃ pavisitvā kapisīsaṃ ālambitvā rodamāno aṭṭhāsi – ‘‘ahañca vatamhi sekho sakaraṇīyo, satthu ca me parinibbānaṃ bhavissati, yo mama anukampako’’ti. Atha kho bhagavā bhikkhū āmantesi – ‘‘kahaṃ nu kho, bhikkhave, ānando’’ti? ‘‘Eso, bhante, āyasmā ānando vihāraṃ pavisitvā kapisīsaṃ ālambitvā rodamāno ṭhito – ‘ahañca vatamhi sekho sakaraṇīyo, satthu ca me parinibbānaṃ bhavissati, yo mama anukampako’’’ti. Atha kho bhagavā aññataraṃ bhikkhuṃ āmantesi – ‘‘ehi tvaṃ, bhikkhu, mama vacanena ānandaṃ āmantehi – ‘satthā taṃ, āvuso ānanda, āmantetī’’’ti. ‘‘Evaṃ , bhante’’ti kho so bhikkhu bhagavato paṭissutvā yenāyasmā ānando tenupasaṅkami; upasaṅkamitvā āyasmantaṃ ānandaṃ etadavoca – ‘‘satthā taṃ, āvuso ānanda, āmantetī’’ti. ‘‘Evamāvuso’’ti kho āyasmā ānando tassa bhikkhuno paṭissutvā yena bhagavā tenupasaṅkami; upasaṅkamitvā bhagavantaṃ abhivādetvā ekamantaṃ nisīdi. Ekamantaṃ nisinnaṃ kho āyasmantaṃ ānandaṃ bhagavā etadavoca – ‘‘alaṃ, ānanda, mā soci mā paridevi, nanu etaṃ, ānanda, mayā paṭikacceva akkhātaṃ – ‘sabbeheva piyehi manāpehi nānābhāvo vinābhāvo aññathābhāvo’; taṃ kutettha, ānanda, labbhā. Yaṃ taṃ jātaṃ bhūtaṃ saṅkhataṃ palokadhammaṃ, taṃ vata tathāgatassāpi sarīraṃ mā palujjī’ti netaṃ ṭhānaṃ vijjati. Dīgharattaṃ kho te, ānanda, tathāgato paccupaṭṭhito mettena kāyakammena hitena sukhena advayena appamāṇena, mettena vacīkammena hitena sukhena advayena appamāṇena, mettena manokammena hitena sukhena advayena appamāṇena. Katapuññosi tvaṃ, ānanda, padhānamanuyuñja, khippaṃ hohisi anāsavo’’ti.

\paragraph{208.} Atha kho bhagavā bhikkhū āmantesi – ‘‘yepi te, bhikkhave, ahesuṃ atītamaddhānaṃ arahanto sammāsambuddhā, tesampi bhagavantānaṃ etapparamāyeva upaṭṭhākā ahesuṃ, seyyathāpi mayhaṃ ānando. Yepi te, bhikkhave, bhavissanti anāgatamaddhānaṃ arahanto sammāsambuddhā, tesampi bhagavantānaṃ etapparamāyeva upaṭṭhākā bhavissanti, seyyathāpi mayhaṃ ānando. Paṇḍito, bhikkhave, ānando; medhāvī, bhikkhave, ānando. Jānāti ‘ayaṃ kālo tathāgataṃ dassanāya upasaṅkamituṃ bhikkhūnaṃ, ayaṃ kālo bhikkhunīnaṃ, ayaṃ kālo upāsakānaṃ , ayaṃ kālo upāsikānaṃ, ayaṃ kālo rañño rājamahāmattānaṃ titthiyānaṃ titthiyasāvakāna’nti.

\paragraph{209.} ‘‘Cattārome, bhikkhave, acchariyā abbhutā dhammā\footnote{abbhutadhammā (syā. ka.)} ānande. Katame cattāro? Sace, bhikkhave, bhikkhuparisā ānandaṃ dassanāya upasaṅkamati, dassanena sā attamanā hoti. Tatra ce ānando dhammaṃ bhāsati, bhāsitenapi sā attamanā hoti. Atittāva, bhikkhave, bhikkhuparisā hoti, atha kho ānando tuṇhī hoti. Sace, bhikkhave, bhikkhunīparisā ānandaṃ dassanāya upasaṅkamati, dassanena sā attamanā hoti. Tatra ce ānando dhammaṃ bhāsati, bhāsitenapi sā attamanā hoti. Atittāva, bhikkhave, bhikkhunīparisā hoti, atha kho ānando tuṇhī hoti. Sace, bhikkhave, upāsakaparisā ānandaṃ dassanāya upasaṅkamati, dassanena sā attamanā hoti. Tatra ce ānando dhammaṃ bhāsati, bhāsitenapi sā attamanā hoti. Atittāva, bhikkhave, upāsakaparisā hoti, atha kho ānando tuṇhī hoti. Sace, bhikkhave, upāsikāparisā ānandaṃ dassanāya upasaṅkamati, dassanena sā attamanā hoti. Tatra ce, ānando, dhammaṃ bhāsati, bhāsitenapi sā attamanā hoti. Atittāva, bhikkhave, upāsikāparisā hoti, atha kho ānando tuṇhī hoti. Ime kho, bhikkhave, cattāro acchariyā abbhutā dhammā ānande.

‘‘Cattārome, bhikkhave, acchariyā abbhutā dhammā raññe cakkavattimhi. Katame cattāro ? Sace, bhikkhave, khattiyaparisā rājānaṃ cakkavattiṃ dassanāya upasaṅkamati, dassanena sā attamanā hoti. Tatra ce rājā cakkavattī bhāsati, bhāsitenapi sā attamanā hoti. Atittāva, bhikkhave, khattiyaparisā hoti. Atha kho rājā cakkavattī tuṇhī hoti. Sace bhikkhave, brāhmaṇaparisā…pe… gahapatiparisā…pe… samaṇaparisā rājānaṃ cakkavattiṃ dassanāya upasaṅkamati, dassanena sā attamanā hoti. Tatra ce rājā cakkavattī bhāsati, bhāsitenapi sā attamanā hoti. Atittāva, bhikkhave, samaṇaparisā hoti, atha kho rājā cakkavattī tuṇhī hoti. Evameva kho, bhikkhave, cattārome acchariyā abbhutā dhammā ānande. Sace, bhikkhave, bhikkhuparisā ānandaṃ dassanāya upasaṅkamati, dassanena sā attamanā hoti. Tatra ce ānando dhammaṃ bhāsati, bhāsitenapi sā attamanā hoti. Atittāva, bhikkhave, bhikkhuparisā hoti. Atha kho ānando tuṇhī hoti. Sace, bhikkhave bhikkhunīparisā…pe… upāsakaparisā…pe… upāsikāparisā ānandaṃ dassanāya upasaṅkamati, dassanena sā attamanā hoti. Tatra ce ānando dhammaṃ bhāsati, bhāsitenapi sā attamanā hoti. Atittāva, bhikkhave, upāsikāparisā hoti. Atha kho ānando tuṇhī hoti. Ime kho, bhikkhave, cattāro acchariyā abbhutā dhammā ānande’’ti.

\subsubsection{Mahāsudassanasuttadesanā}

\paragraph{210.} Evaṃ vutte āyasmā ānando bhagavantaṃ etadavoca – ‘‘mā, bhante, bhagavā imasmiṃ khuddakanagarake ujjaṅgalanagarake sākhānagarake parinibbāyi. Santi, bhante, aññāni mahānagarāni, seyyathidaṃ – campā rājagahaṃ sāvatthī sāketaṃ kosambī bārāṇasī; ettha bhagavā parinibbāyatu. Ettha bahū khattiyamahāsālā, brāhmaṇamahāsālā gahapatimahāsālā tathāgate abhippasannā. Te tathāgatassa sarīrapūjaṃ karissantī’’ti ‘‘māhevaṃ, ānanda, avaca; māhevaṃ, ānanda, avaca – ‘khuddakanagarakaṃ ujjaṅgalanagarakaṃ sākhānagaraka’nti.

‘‘Bhūtapubbaṃ, ānanda, rājā mahāsudassano nāma ahosi cakkavattī dhammiko dhammarājā cāturanto vijitāvī janappadatthāvariyappatto sattaratanasamannāgato. Rañño, ānanda, mahāsudassanassa ayaṃ kusinārā kusāvatī nāma rājadhānī ahosi, puratthimena ca pacchimena ca dvādasayojanāni āyāmena; uttarena ca dakkhiṇena ca sattayojanāni vitthārena. Kusāvatī, ānanda, rājadhānī iddhā ceva ahosi phītā ca bahujanā ca ākiṇṇamanussā ca subhikkhā ca. Seyyathāpi, ānanda, devānaṃ āḷakamandā nāma rājadhānī iddhā ceva hoti phītā ca bahujanā ca ākiṇṇayakkhā ca subhikkhā ca; evameva kho, ānanda, kusāvatī rājadhānī iddhā ceva ahosi phītā ca bahujanā ca ākiṇṇamanussā ca subhikkhā ca. Kusāvatī, ānanda, rājadhānī dasahi saddehi avivittā ahosi divā ceva rattiñca, seyyathidaṃ – hatthisaddena assasaddena rathasaddena bherisaddena mudiṅgasaddena vīṇāsaddena gītasaddena saṅkhasaddena sammasaddena pāṇitāḷasaddena ‘asnātha pivatha khādathā’ti dasamena saddena.

‘‘Gaccha tvaṃ, ānanda, kusināraṃ pavisitvā kosinārakānaṃ mallānaṃ ārocehi – ‘ajja kho, vāseṭṭhā, rattiyā pacchime yāme tathāgatassa parinibbānaṃ bhavissati. Abhikkamatha vāseṭṭhā, abhikkamatha vāseṭṭhā. Mā pacchā vippaṭisārino ahuvattha – amhākañca no gāmakkhette tathāgatassa parinibbānaṃ ahosi, na mayaṃ labhimhā pacchime kāle tathāgataṃ dassanāyā’’’ti. ‘‘Evaṃ, bhante’’ti kho āyasmā ānando bhagavato paṭissutvā nivāsetvā pattacīvaramādāya attadutiyo kusināraṃ pāvisi.

\subsubsection{Mallānaṃ vandanā}

\paragraph{211.} Tena kho pana samayena kosinārakā mallā sandhāgāre\footnote{santhāgāre (sī. syā. pī.)} sannipatitā honti kenacideva karaṇīyena. Atha kho āyasmā ānando yena kosinārakānaṃ mallānaṃ sandhāgāraṃ tenupasaṅkami; upasaṅkamitvā kosinārakānaṃ mallānaṃ ārocesi – ‘‘ajja kho, vāseṭṭhā, rattiyā pacchime yāme tathāgatassa parinibbānaṃ bhavissati. Abhikkamatha vāseṭṭhā abhikkamatha vāseṭṭhā. Mā pacchā vippaṭisārino ahuvattha – ‘amhākañca no gāmakkhette tathāgatassa parinibbānaṃ ahosi, na mayaṃ labhimhā pacchime kāle tathāgataṃ dassanāyā’’’ti. Idamāyasmato ānandassa vacanaṃ sutvā mallā ca mallaputtā ca mallasuṇisā ca mallapajāpatiyo ca aghāvino dummanā cetodukkhasamappitā appekacce kese pakiriya kandanti, bāhā paggayha kandanti, chinnapātaṃ papatanti, āvaṭṭanti vivaṭṭanti – ‘atikhippaṃ bhagavā parinibbāyissati, atikhippaṃ sugato parinibbāyissati, atikhippaṃ cakkhuṃ loke antaradhāyissatī’ti. Atha kho mallā ca mallaputtā ca mallasuṇisā ca mallapajāpatiyo ca aghāvino dummanā cetodukkhasamappitā yena upavattanaṃ mallānaṃ sālavanaṃ yenāyasmā ānando tenupasaṅkamiṃsu. Atha kho āyasmato ānandassa etadahosi – ‘‘sace kho ahaṃ kosinārake malle ekamekaṃ bhagavantaṃ vandāpessāmi, avandito bhagavā kosinārakehi mallehi bhavissati, athāyaṃ ratti vibhāyissati. Yaṃnūnāhaṃ kosinārake malle kulaparivattaso kulaparivattaso ṭhapetvā bhagavantaṃ vandāpeyyaṃ – ‘itthannāmo, bhante, mallo saputto sabhariyo sapariso sāmacco bhagavato pāde sirasā vandatī’ti. Atha kho āyasmā ānando kosinārake malle kulaparivattaso kulaparivattaso ṭhapetvā bhagavantaṃ vandāpesi – ‘itthannāmo, bhante, mallo saputto sabhariyo sapariso sāmacco bhagavato pāde sirasā vandatī’’’ti. Atha kho āyasmā ānando etena upāyena paṭhameneva yāmena kosinārake malle bhagavantaṃ vandāpesi.

\subsubsection{Subhaddaparibbājakavatthu}

\paragraph{212.} Tena kho pana samayena subhaddo nāma paribbājako kusinārāyaṃ paṭivasati. Assosi kho subhaddo paribbājako – ‘‘ajja kira rattiyā pacchime yāme samaṇassa gotamassa parinibbānaṃ bhavissatī’’ti. Atha kho subhaddassa paribbājakassa etadahosi – ‘‘sutaṃ kho pana metaṃ paribbājakānaṃ vuḍḍhānaṃ mahallakānaṃ ācariyapācariyānaṃ bhāsamānānaṃ – ‘kadāci karahaci tathāgatā loke uppajjanti arahanto sammāsambuddhā’ti. Ajjeva rattiyā pacchime yāme samaṇassa gotamassa parinibbānaṃ bhavissati. Atthi ca me ayaṃ kaṅkhādhammo uppanno, evaṃ pasanno ahaṃ samaṇe gotame, ‘pahoti me samaṇo gotamo tathā dhammaṃ desetuṃ, yathāhaṃ imaṃ kaṅkhādhammaṃ pajaheyya’’’nti. Atha kho subhaddo paribbājako yena upavattanaṃ mallānaṃ sālavanaṃ, yenāyasmā ānando tenupasaṅkami; upasaṅkamitvā āyasmantaṃ ānandaṃ etadavoca – ‘‘sutaṃ metaṃ, bho ānanda, paribbājakānaṃ vuḍḍhānaṃ mahallakānaṃ ācariyapācariyānaṃ bhāsamānānaṃ – ‘kadāci karahaci tathāgatā loke uppajjanti arahanto sammāsambuddhā’ti. Ajjeva rattiyā pacchime yāme samaṇassa gotamassa parinibbānaṃ bhavissati. Atthi ca me ayaṃ kaṅkhādhammo uppanno – evaṃ pasanno ahaṃ samaṇe gotame ‘pahoti me samaṇo gotamo tathā dhammaṃ desetuṃ, yathāhaṃ imaṃ kaṅkhādhammaṃ pajaheyya’nti. Sādhāhaṃ, bho ānanda, labheyyaṃ samaṇaṃ gotamaṃ dassanāyā’’ti. Evaṃ vutte āyasmā ānando subhaddaṃ paribbājakaṃ etadavoca – ‘‘alaṃ, āvuso subhadda, mā tathāgataṃ viheṭhesi, kilanto bhagavā’’ti. Dutiyampi kho subhaddo paribbājako…pe… tatiyampi kho subhaddo paribbājako āyasmantaṃ ānandaṃ etadavoca – ‘‘sutaṃ metaṃ, bho ānanda, paribbājakānaṃ vuḍḍhānaṃ mahallakānaṃ ācariyapācariyānaṃ bhāsamānānaṃ – ‘kadāci karahaci tathāgatā loke uppajjanti arahanto sammāsambuddhā’ti. Ajjeva rattiyā pacchime yāme samaṇassa gotamassa parinibbānaṃ bhavissati. Atthi ca me ayaṃ kaṅkhādhammo uppanno – evaṃ pasanno ahaṃ samaṇe gotame, ‘pahoti me samaṇo gotamo tathā dhammaṃ desetuṃ, yathāhaṃ imaṃ kaṅkhādhammaṃ pajaheyya’nti. Sādhāhaṃ, bho ānanda, labheyyaṃ samaṇaṃ gotamaṃ dassanāyā’’ti. Tatiyampi kho āyasmā ānando subhaddaṃ paribbājakaṃ etadavoca – ‘‘alaṃ, āvuso subhadda, mā tathāgataṃ viheṭhesi, kilanto bhagavā’’ti.

\paragraph{213.} Assosi kho bhagavā āyasmato ānandassa subhaddena paribbājakena saddhiṃ imaṃ kathāsallāpaṃ. Atha kho bhagavā āyasmantaṃ ānandaṃ āmantesi – ‘‘alaṃ, ānanda, mā subhaddaṃ vāresi, labhataṃ, ānanda, subhaddo tathāgataṃ dassanāya. Yaṃ kiñci maṃ subhaddo pucchissati, sabbaṃ taṃ aññāpekkhova pucchissati, no vihesāpekkho. Yaṃ cassāhaṃ puṭṭho byākarissāmi, taṃ khippameva ājānissatī’’ti. Atha kho āyasmā ānando subhaddaṃ paribbājakaṃ etadavoca – ‘‘gacchāvuso subhadda, karoti te bhagavā okāsa’’nti. Atha kho subhaddo paribbājako yena bhagavā tenupasaṅkami; upasaṅkamitvā bhagavatā saddhiṃ sammodi, sammodanīyaṃ kathaṃ sāraṇīyaṃ vītisāretvā ekamantaṃ nisīdi. Ekamantaṃ nisinno kho subhaddo paribbājako bhagavantaṃ etadavoca – ‘‘yeme, bho gotama, samaṇabrāhmaṇā saṅghino gaṇino gaṇācariyā ñātā yasassino titthakarā sādhusammatā bahujanassa, seyyathidaṃ – pūraṇo kassapo, makkhali gosālo, ajito kesakambalo, pakudho kaccāyano, sañcayo belaṭṭhaputto, nigaṇṭho nāṭaputto, sabbete sakāya paṭiññāya abbhaññiṃsu, sabbeva na abbhaññiṃsu , udāhu ekacce abbhaññiṃsu, ekacce na abbhaññiṃsū’’ti? ‘‘Alaṃ, subhadda, tiṭṭhatetaṃ – ‘sabbete sakāya paṭiññāya abbhaññiṃsu, sabbeva na abbhaññiṃsu, udāhu ekacce abbhaññiṃsu, ekacce na abbhaññiṃsū’ti. Dhammaṃ te, subhadda, desessāmi; taṃ suṇāhi sādhukaṃ manasikarohi, bhāsissāmī’’ti. ‘‘Evaṃ, bhante’’ti kho subhaddo paribbājako bhagavato paccassosi. Bhagavā etadavoca –

\paragraph{214.} ‘‘Yasmiṃ kho, subhadda, dhammavinaye ariyo aṭṭhaṅgiko maggo na upalabbhati, samaṇopi tattha na upalabbhati. Dutiyopi tattha samaṇo na upalabbhati. Tatiyopi tattha samaṇo na upalabbhati. Catutthopi tattha samaṇo na upalabbhati. Yasmiñca kho, subhadda, dhammavinaye ariyo aṭṭhaṅgiko maggo upalabbhati, samaṇopi tattha upalabbhati, dutiyopi tattha samaṇo upalabbhati, tatiyopi tattha samaṇo upalabbhati, catutthopi tattha samaṇo upalabbhati. Imasmiṃ kho, subhadda, dhammavinaye ariyo aṭṭhaṅgiko maggo upalabbhati, idheva, subhadda, samaṇo, idha dutiyo samaṇo, idha tatiyo samaṇo, idha catuttho samaṇo, suññā parappavādā samaṇebhi aññehi\footnote{aññe (pī.)}. Ime ca\footnote{idheva (ka.)}, subhadda, bhikkhū sammā vihareyyuṃ, asuñño loko arahantehi assāti.

‘‘Ekūnatiṃso vayasā subhadda,

Yaṃ pabbajiṃ kiṃkusalānuesī;

Vassāni paññāsa samādhikāni,

Yato ahaṃ pabbajito subhadda.

Ñāyassa dhammassa padesavattī,

Ito bahiddhā samaṇopi natthi.

‘‘Dutiyopi samaṇo natthi. Tatiyopi samaṇo natthi. Catutthopi samaṇo natthi. Suññā parappavādā samaṇebhi aññehi. Ime ca, subhadda, bhikkhū sammā vihareyyuṃ, asuñño loko arahantehi assā’’ti.

\paragraph{215.} Evaṃ vutte subhaddo paribbājako bhagavantaṃ etadavoca – ‘‘abhikkantaṃ, bhante, abhikkantaṃ, bhante. Seyyathāpi, bhante, nikkujjitaṃ vā ukkujjeyya, paṭicchannaṃ vā vivareyya, mūḷhassa vā maggaṃ ācikkheyya, andhakāre vā telapajjotaṃ dhāreyya, ‘cakkhumanto rūpāni dakkhantī’ti, evamevaṃ bhagavatā anekapariyāyena dhammo pakāsito. Esāhaṃ, bhante, bhagavantaṃ saraṇaṃ gacchāmi dhammañca bhikkhusaṅghañca. Labheyyāhaṃ, bhante, bhagavato santike pabbajjaṃ, labheyyaṃ upasampada’’nti. ‘‘Yo kho, subhadda, aññatitthiyapubbo imasmiṃ dhammavinaye ākaṅkhati pabbajjaṃ, ākaṅkhati upasampadaṃ, so cattāro māse parivasati. Catunnaṃ māsānaṃ accayena āraddhacittā bhikkhū pabbājenti upasampādenti bhikkhubhāvāya. Api ca mettha puggalavemattatā viditā’’ti. ‘‘Sace, bhante, aññatitthiyapubbā imasmiṃ dhammavinaye ākaṅkhantā pabbajjaṃ ākaṅkhantā upasampadaṃ cattāro māse parivasanti, catunnaṃ māsānaṃ accayena āraddhacittā bhikkhū pabbājenti upasampādenti bhikkhubhāvāya. Ahaṃ cattāri vassāni parivasissāmi, catunnaṃ vassānaṃ accayena āraddhacittā bhikkhū pabbājentu upasampādentu bhikkhubhāvāyā’’ti.

Atha kho bhagavā āyasmantaṃ ānandaṃ āmantesi – ‘‘tenahānanda, subhaddaṃ pabbājehī’’ti. ‘‘Evaṃ, bhante’’ti kho āyasmā ānando bhagavato paccassosi. Atha kho subhaddo paribbājako āyasmantaṃ ānandaṃ etadavoca – ‘‘lābhā vo, āvuso ānanda; suladdhaṃ vo, āvuso ānanda, ye ettha satthu\footnote{satthārā (syā.)} sammukhā antevāsikābhisekena abhisittā’’ti. Alattha kho subhaddo paribbājako bhagavato santike pabbajjaṃ, alattha upasampadaṃ. Acirūpasampanno kho panāyasmā subhaddo eko vūpakaṭṭho appamatto ātāpī pahitatto viharanto nacirasseva – ‘yassatthāya kulaputtā sammadeva agārasmā anagāriyaṃ pabbajanti’ tadanuttaraṃ brahmacariyapariyosānaṃ diṭṭheva dhamme sayaṃ abhiññā sacchikatvā upasampajja vihāsi. ‘Khīṇā jāti, vusitaṃ brahmacariyaṃ, kataṃ karaṇīyaṃ, nāparaṃ itthattāyā’ti abbhaññāsi. Aññataro kho panāyasmā subhaddo arahataṃ ahosi. So bhagavato pacchimo sakkhisāvako ahosīti.

\xsubsubsectionEnd{Pañcamo bhāṇavāro.}

\subsubsection{Tathāgatapacchimavācā}

\paragraph{216.} Atha kho bhagavā āyasmantaṃ ānandaṃ āmantesi – ‘‘siyā kho panānanda, tumhākaṃ evamassa – ‘atītasatthukaṃ pāvacanaṃ, natthi no satthā’ti. Na kho panetaṃ, ānanda, evaṃ daṭṭhabbaṃ. Yo vo, ānanda, mayā dhammo ca vinayo ca desito paññatto, so vo mamaccayena satthā. Yathā kho panānanda, etarahi bhikkhū aññamaññaṃ āvusovādena samudācaranti, na kho mamaccayena evaṃ samudācaritabbaṃ. Theratarena, ānanda, bhikkhunā navakataro bhikkhu nāmena vā gottena vā āvusovādena vā samudācaritabbo. Navakatarena bhikkhunā therataro bhikkhu ‘bhante’ti vā ‘āyasmā’ti vā samudācaritabbo. Ākaṅkhamāno, ānanda, saṅgho mamaccayena khuddānukhuddakāni sikkhāpadāni samūhanatu. Channassa, ānanda, bhikkhuno mamaccayena brahmadaṇḍo dātabbo’’ti. ‘‘Katamo pana, bhante, brahmadaṇḍo’’ti? ‘‘Channo, ānanda, bhikkhu yaṃ iccheyya, taṃ vadeyya. So bhikkhūhi neva vattabbo, na ovaditabbo, na anusāsitabbo’’ti.

\paragraph{217.} Atha kho bhagavā bhikkhū āmantesi – ‘‘siyā kho pana, bhikkhave, ekabhikkhussāpi kaṅkhā vā vimati vā buddhe vā dhamme vā saṅghe vā magge vā paṭipadāya vā, pucchatha, bhikkhave, mā pacchā vippaṭisārino ahuvattha – ‘sammukhībhūto no satthā ahosi , na mayaṃ sakkhimhā bhagavantaṃ sammukhā paṭipucchitu’’’ nti. Evaṃ vutte te bhikkhū tuṇhī ahesuṃ. Dutiyampi kho bhagavā…pe… tatiyampi kho bhagavā bhikkhū āmantesi – ‘‘siyā kho pana, bhikkhave, ekabhikkhussāpi kaṅkhā vā vimati vā buddhe vā dhamme vā saṅghe vā magge vā paṭipadāya vā, pucchatha, bhikkhave, mā pacchā vippaṭisārino ahuvattha – ‘sammukhībhūto no satthā ahosi , na mayaṃ sakkhimhā bhagavantaṃ sammukhā paṭipucchitu’’’ nti. Tatiyampi kho te bhikkhū tuṇhī ahesuṃ. Atha kho bhagavā bhikkhū āmantesi – ‘‘siyā kho pana, bhikkhave, satthugāravenapi na puccheyyātha. Sahāyakopi, bhikkhave, sahāyakassa ārocetū’’ti. Evaṃ vutte te bhikkhū tuṇhī ahesuṃ. Atha kho āyasmā ānando bhagavantaṃ etadavoca – ‘‘acchariyaṃ, bhante, abbhutaṃ, bhante, evaṃ pasanno ahaṃ, bhante, imasmiṃ bhikkhusaṅghe, ‘natthi ekabhikkhussāpi kaṅkhā vā vimati vā buddhe vā dhamme vā saṅghe vā magge vā paṭipadāya vā’’’ti. ‘‘Pasādā kho tvaṃ, ānanda, vadesi, ñāṇameva hettha, ānanda, tathāgatassa. Natthi imasmiṃ bhikkhusaṅghe ekabhikkhussāpi kaṅkhā vā vimati vā buddhe vā dhamme vā saṅghe vā magge vā paṭipadāya vā. Imesañhi, ānanda, pañcannaṃ bhikkhusatānaṃ yo pacchimako bhikkhu, so sotāpanno avinipātadhammo niyato sambodhiparāyaṇo’’ti.

\paragraph{218.} Atha kho bhagavā bhikkhū āmantesi – ‘‘handa dāni, bhikkhave, āmantayāmi vo, vayadhammā saṅkhārā appamādena sampādethā’’ti. Ayaṃ tathāgatassa pacchimā vācā.

\subsubsection{Parinibbutakathā}

\paragraph{219.} Atha kho bhagavā paṭhamaṃ jhānaṃ samāpajji, paṭhamajjhānā vuṭṭhahitvā dutiyaṃ jhānaṃ samāpajji, dutiyajjhānā vuṭṭhahitvā tatiyaṃ jhānaṃ samāpajji, tatiyajjhānā vuṭṭhahitvā catutthaṃ jhānaṃ samāpajji. Catutthajjhānā vuṭṭhahitvā ākāsānañcāyatanaṃ samāpajji, ākāsānañcāyatanasamāpattiyā vuṭṭhahitvā viññāṇañcāyatanaṃ samāpajji, viññāṇañcāyatanasamāpattiyā vuṭṭhahitvā ākiñcaññāyatanaṃ samāpajji, ākiñcaññāyatanasamāpattiyā vuṭṭhahitvā nevasaññānāsaññāyatanaṃ samāpajji, nevasaññānāsaññāyatanasamāpattiyā vuṭṭhahitvā saññāvedayitanirodhaṃ samāpajji.

Atha kho āyasmā ānando āyasmantaṃ anuruddhaṃ etadavoca – ‘‘parinibbuto, bhante anuruddha , bhagavā’’ti. ‘‘Nāvuso ānanda, bhagavā parinibbuto, saññāvedayitanirodhaṃ samāpanno’’ti.

Atha kho bhagavā saññāvedayitanirodhasamāpattiyā vuṭṭhahitvā nevasaññānāsaññāyatanaṃ samāpajji, nevasaññānāsaññāyatanasamāpattiyā vuṭṭhahitvā ākiñcaññāyatanaṃ samāpajji, ākiñcaññāyatanasamāpattiyā vuṭṭhahitvā viññāṇañcāyatanaṃ samāpajji, viññāṇañcāyatanasamāpattiyā vuṭṭhahitvā ākāsānañcāyatanaṃ samāpajji, ākāsānañcāyatanasamāpattiyā vuṭṭhahitvā catutthaṃ jhānaṃ samāpajji, catutthajjhānā vuṭṭhahitvā tatiyaṃ jhānaṃ samāpajji, tatiyajjhānā vuṭṭhahitvā dutiyaṃ jhānaṃ samāpajji, dutiyajjhānā vuṭṭhahitvā paṭhamaṃ jhānaṃ samāpajji, paṭhamajjhānā vuṭṭhahitvā dutiyaṃ jhānaṃ samāpajji, dutiyajjhānā vuṭṭhahitvā tatiyaṃ jhānaṃ samāpajji, tatiyajjhānā vuṭṭhahitvā catutthaṃ jhānaṃ samāpajji, catutthajjhānā vuṭṭhahitvā samanantarā bhagavā parinibbāyi.

\paragraph{220.} Parinibbute bhagavati saha parinibbānā mahābhūmicālo ahosi bhiṃsanako salomahaṃso. Devadundubhiyo ca phaliṃsu. Parinibbute bhagavati saha parinibbānā brahmāsahampati imaṃ gāthaṃ abhāsi –

‘‘Sabbeva nikkhipissanti, bhūtā loke samussayaṃ;

Yattha etādiso satthā, loke appaṭipuggalo;

Tathāgato balappatto, sambuddho parinibbuto’’ti.

\paragraph{221.} Parinibbute bhagavati saha parinibbānā sakko devānamindo imaṃ gāthaṃ abhāsi –

‘‘Aniccā vata saṅkhārā, uppādavayadhammino;

Uppajjitvā nirujjhanti, tesaṃ vūpasamo sukho’’ti.

\paragraph{222.} Parinibbute bhagavati saha parinibbānā āyasmā anuruddho imā gāthāyo abhāsi –

‘‘Nāhu assāsapassāso, ṭhitacittassa tādino;

Anejo santimārabbha, yaṃ kālamakarī muni.

‘‘Asallīnena cittena, vedanaṃ ajjhavāsayi;

Pajjotasseva nibbānaṃ, vimokkho cetaso ahū’’ti.

\paragraph{223.} Parinibbute bhagavati saha parinibbānā āyasmā ānando imaṃ gāthaṃ abhāsi –

‘‘Tadāsi yaṃ bhiṃsanakaṃ, tadāsi lomahaṃsanaṃ;

Sabbākāravarūpete, sambuddhe parinibbute’’ti.

\paragraph{224.} Parinibbute bhagavati ye te tattha bhikkhū avītarāgā appekacce bāhā paggayha kandanti, chinnapātaṃ papatanti, āvaṭṭanti vivaṭṭanti, ‘‘atikhippaṃ bhagavā parinibbuto , atikhippaṃ sugato parinibbuto, atikhippaṃ cakkhuṃ loke antarahito’’ti. Ye pana te bhikkhū vītarāgā, te satā sampajānā adhivāsenti – ‘‘aniccā saṅkhārā, taṃ kutettha labbhā’’ti.

\paragraph{225.} Atha kho āyasmā anuruddho bhikkhū āmantesi – ‘‘alaṃ, āvuso, mā socittha mā paridevittha. Nanu etaṃ, āvuso, bhagavatā paṭikacceva akkhātaṃ – ‘sabbeheva piyehi manāpehi nānābhāvo vinābhāvo aññathābhāvo’. Taṃ kutettha, āvuso, labbhā. ‘Yaṃ taṃ jātaṃ bhūtaṃ saṅkhataṃ palokadhammaṃ, taṃ vata mā palujjī’ti, netaṃ ṭhānaṃ vijjati . Devatā, āvuso, ujjhāyantī’’ti. ‘‘Kathaṃbhūtā pana, bhante, āyasmā anuruddho devatā manasi karotī’’ti\footnote{bhante anuruddha devatā manasi karontīti (syā. ka.)}?

‘‘Santāvuso ānanda, devatā ākāse pathavīsaññiniyo kese pakiriya kandanti, bāhā paggayha kandanti, chinnapātaṃ papatanti, āvaṭṭanti, vivaṭṭanti – ‘atikhippaṃ bhagavā parinibbuto, atikhippaṃ sugato parinibbuto, atikhippaṃ cakkhuṃ loke antarahito’ti. Santāvuso ānanda, devatā pathaviyā pathavīsaññiniyo kese pakiriya kandanti, bāhā paggayha kandanti, chinnapātaṃ papatanti, āvaṭṭanti, vivaṭṭanti – ‘atikhippaṃ bhagavā parinibbuto , atikhippaṃ sugato parinibbuto, atikhippaṃ cakkhuṃ loke antarahito’ti. Yā pana tā devatā vītarāgā, tā satā sampajānā adhivāsenti – ‘aniccā saṅkhārā, taṃ kutettha labbhā’ti. Atha kho āyasmā ca anuruddho āyasmā ca ānando taṃ rattāvasesaṃ dhammiyā kathāya vītināmesuṃ.

\paragraph{226.} Atha kho āyasmā anuruddho āyasmantaṃ ānandaṃ āmantesi – ‘‘gacchāvuso ānanda, kusināraṃ pavisitvā kosinārakānaṃ mallānaṃ ārocehi – ‘parinibbuto, vāseṭṭhā, bhagavā, yassadāni kālaṃ maññathā’’’ti. ‘‘Evaṃ, bhante’’ti kho āyasmā ānando āyasmato anuruddhassa paṭissutvā pubbaṇhasamayaṃ nivāsetvā pattacīvaramādāya attadutiyo kusināraṃ pāvisi. Tena kho pana samayena kosinārakā mallā sandhāgāre sannipatitā honti teneva karaṇīyena. Atha kho āyasmā ānando yena kosinārakānaṃ mallānaṃ sandhāgāraṃ tenupasaṅkami; upasaṅkamitvā kosinārakānaṃ mallānaṃ ārocesi – ‘parinibbuto, vāseṭṭhā, bhagavā, yassadāni kālaṃ maññathā’ti. Idamāyasmato ānandassa vacanaṃ sutvā mallā ca mallaputtā ca mallasuṇisā ca mallapajāpatiyo ca aghāvino dummanā cetodukkhasamappitā appekacce kese pakiriya kandanti, bāhā paggayha kandanti, chinnapātaṃ papatanti, āvaṭṭanti, vivaṭṭanti – ‘‘atikhippaṃ bhagavā parinibbuto, atikhippaṃ sugato parinibbuto, atikhippaṃ cakkhuṃ loke antarahito’’ti.

\subsubsection{Buddhasarīrapūjā}

\paragraph{227.} Atha kho kosinārakā mallā purise āṇāpesuṃ – ‘‘tena hi, bhaṇe, kusinārāyaṃ gandhamālañca sabbañca tāḷāvacaraṃ sannipātethā’’ti. Atha kho kosinārakā mallā gandhamālañca sabbañca tāḷāvacaraṃ pañca ca dussayugasatāni ādāya yena upavattanaṃ mallānaṃ sālavanaṃ, yena bhagavato sarīraṃ tenupasaṅkamiṃsu; upasaṅkamitvā bhagavato sarīraṃ naccehi gītehi vāditehi mālehi gandhehi sakkarontā garuṃ karontā mānentā pūjentā celavitānāni karontā maṇḍalamāḷe paṭiyādentā ekadivasaṃ vītināmesuṃ.

Atha kho kosinārakānaṃ mallānaṃ etadahosi – ‘‘ativikālo kho ajja bhagavato sarīraṃ jhāpetuṃ, sve dāni mayaṃ bhagavato sarīraṃ jhāpessāmā’’ti. Atha kho kosinārakā mallā bhagavato sarīraṃ naccehi gītehi vāditehi mālehi gandhehi sakkarontā garuṃ karontā mānentā pūjentā celavitānāni karontā maṇḍalamāḷe paṭiyādentā dutiyampi divasaṃ vītināmesuṃ, tatiyampi divasaṃ vītināmesuṃ, catutthampi divasaṃ vītināmesuṃ, pañcamampi divasaṃ vītināmesuṃ, chaṭṭhampi divasaṃ vītināmesuṃ.

Atha kho sattamaṃ divasaṃ kosinārakānaṃ mallānaṃ etadahosi – ‘‘mayaṃ bhagavato sarīraṃ naccehi gītehi vāditehi mālehi gandhehi sakkarontā garuṃ karontā mānentā pūjentā dakkhiṇena dakkhiṇaṃ nagarassa haritvā bāhirena bāhiraṃ dakkhiṇato nagarassa bhagavato sarīraṃ jhāpessāmā’’ti.

\paragraph{228.} Tena kho pana samayena aṭṭha mallapāmokkhā sīsaṃnhātā ahatāni vatthāni nivatthā ‘‘mayaṃ bhagavato sarīraṃ uccāressāmā’’ti na sakkonti uccāretuṃ. Atha kho kosinārakā mallā āyasmantaṃ anuruddhaṃ etadavocuṃ – ‘‘ko nu kho, bhante anuruddha, hetu ko paccayo, yenime aṭṭha mallapāmokkhā sīsaṃnhātā ahatāni vatthāni nivatthā ‘mayaṃ bhagavato sarīraṃ uccāressāmā’ti na sakkonti uccāretu’’nti? ‘‘Aññathā kho, vāseṭṭhā, tumhākaṃ adhippāyo, aññathā devatānaṃ adhippāyo’’ti. ‘‘Kathaṃ pana, bhante, devatānaṃ adhippāyo’’ti? ‘‘Tumhākaṃ kho, vāseṭṭhā, adhippāyo – ‘mayaṃ bhagavato sarīraṃ naccehi gītehi vāditehi mālehi gandhehi sakkarontā garuṃ karontā mānentā pūjentā dakkhiṇena dakkhiṇaṃ nagarassa haritvā bāhirena bāhiraṃ dakkhiṇato nagarassa bhagavato sarīraṃ jhāpessāmā’ti; devatānaṃ kho, vāseṭṭhā, adhippāyo – ‘mayaṃ bhagavato sarīraṃ dibbehi naccehi gītehi vāditehi gandhehi sakkarontā garuṃ karontā mānentā pūjentā uttarena uttaraṃ nagarassa haritvā uttarena dvārena nagaraṃ pavesetvā majjhena majjhaṃ nagarassa haritvā puratthimena dvārena nikkhamitvā puratthimato nagarassa makuṭabandhanaṃ nāma mallānaṃ cetiyaṃ ettha bhagavato sarīraṃ jhāpessāmā’ti. ‘‘Yathā, bhante, devatānaṃ adhippāyo, tathā hotū’’ti.

\paragraph{229.} Tena kho pana samayena kusinārā yāva sandhisamalasaṃkaṭīrā jaṇṇumattena odhinā mandāravapupphehi santhatā\footnote{saṇṭhitā (syā.)} hoti. Atha kho devatā ca kosinārakā ca mallā bhagavato sarīraṃ dibbehi ca mānusakehi ca naccehi gītehi vāditehi mālehi gandhehi sakkarontā garuṃ karontā mānentā pūjentā uttarena uttaraṃ nagarassa haritvā uttarena dvārena nagaraṃ pavesetvā majjhena majjhaṃ nagarassa haritvā puratthimena dvārena nikkhamitvā puratthimato nagarassa makuṭabandhanaṃ nāma mallānaṃ cetiyaṃ ettha ca bhagavato sarīraṃ nikkhipiṃsu.

\paragraph{230.} Atha kho kosinārakā mallā āyasmantaṃ ānandaṃ etadavocuṃ – ‘‘kathaṃ mayaṃ, bhante ānanda, tathāgatassa sarīre paṭipajjāmā’’ti? ‘‘Yathā kho, vāseṭṭhā, rañño cakkavattissa sarīre paṭipajjanti, evaṃ tathāgatassa sarīre paṭipajjitabba’’nti. ‘‘Kathaṃ pana, bhante ānanda, rañño cakkavattissa sarīre paṭipajjantī’’ti? ‘‘Rañño, vāseṭṭhā, cakkavattissa sarīraṃ ahatena vatthena veṭhenti, ahatena vatthena veṭhetvā vihatena kappāsena veṭhenti, vihatena kappāsena veṭhetvā ahatena vatthena veṭhenti. Etena upāyena pañcahi yugasatehi rañño cakkavattissa sarīraṃ veṭhetvā āyasāya teladoṇiyā pakkhipitvā aññissā āyasāya doṇiyā paṭikujjitvā sabbagandhānaṃ citakaṃ karitvā rañño cakkavattissa sarīraṃ jhāpenti. Cātumahāpathe rañño cakkavattissa thūpaṃ karonti . Evaṃ kho, vāseṭṭhā, rañño cakkavattissa sarīre paṭipajjanti. Yathā kho, vāseṭṭhā, rañño cakkavattissa sarīre paṭipajjanti, evaṃ tathāgatassa sarīre paṭipajjitabbaṃ. Cātumahāpathe tathāgatassa thūpo kātabbo. Tattha ye mālaṃ vā gandhaṃ vā cuṇṇakaṃ vā āropessanti vā abhivādessanti vā cittaṃ vā pasādessanti, tesaṃ taṃ bhavissati dīgharattaṃ hitāya sukhāyā’’ti. Atha kho kosinārakā mallā purise āṇāpesuṃ – ‘‘tena hi, bhaṇe, mallānaṃ vihataṃ kappāsaṃ sannipātethā’’ti.

Atha kho kosinārakā mallā bhagavato sarīraṃ ahatena vatthena veṭhetvā vihatena kappāsena veṭhesuṃ, vihatena kappāsena veṭhetvā ahatena vatthena veṭhesuṃ. Etena upāyena pañcahi yugasatehi bhagavato sarīraṃ veṭhetvā āyasāya teladoṇiyā pakkhipitvā aññissā āyasāya doṇiyā paṭikujjitvā sabbagandhānaṃ citakaṃ karitvā bhagavato sarīraṃ citakaṃ āropesuṃ.

\subsubsection{Mahākassapattheravatthu}

\paragraph{231.} Tena kho pana samayena āyasmā mahākassapo pāvāya kusināraṃ addhānamaggappaṭippanno hoti mahatā bhikkhusaṅghena saddhiṃ pañcamattehi bhikkhusatehi. Atha kho āyasmā mahākassapo maggā okkamma aññatarasmiṃ rukkhamūle nisīdi. Tena kho pana samayena aññataro ājīvako kusinārāya mandāravapupphaṃ gahetvā pāvaṃ addhānamaggappaṭippanno hoti. Addasā kho āyasmā mahākassapo taṃ ājīvakaṃ dūratova āgacchantaṃ, disvā taṃ ājīvakaṃ etadavoca – ‘‘apāvuso, amhākaṃ satthāraṃ jānāsī’’ti? ‘‘Āmāvuso, jānāmi, ajja sattāhaparinibbuto samaṇo gotamo. Tato me idaṃ mandāravapupphaṃ gahita’’nti. Tattha ye te bhikkhū avītarāgā appekacce bāhā paggayha kandanti, chinnapātaṃ papatanti, āvaṭṭanti, vivaṭṭanti – ‘‘atikhippaṃ bhagavā parinibbuto, atikhippaṃ sugato parinibbuto, atikhippaṃ cakkhuṃ loke antarahito’’ti. Ye pana te bhikkhū vītarāgā, te satā sampajānā adhivāsenti – ‘‘aniccā saṅkhārā, taṃ kutettha labbhā’’ti.

\paragraph{232.} Tena kho pana samayena subhaddo nāma vuddhapabbajito tassaṃ parisāyaṃ nisinno hoti. Atha kho subhaddo vuddhapabbajito te bhikkhū etadavoca – ‘‘alaṃ, āvuso, mā socittha, mā paridevittha, sumuttā mayaṃ tena mahāsamaṇena. Upaddutā ca homa – ‘idaṃ vo kappati, idaṃ vo na kappatī’ti. Idāni pana mayaṃ yaṃ icchissāma, taṃ karissāma, yaṃ na icchissāma, na taṃ karissāmā’’ti. Atha kho āyasmā mahākassapo bhikkhū āmantesi – ‘‘alaṃ, āvuso, mā socittha, mā paridevittha. Nanu etaṃ , āvuso, bhagavatā paṭikacceva akkhātaṃ – ‘sabbeheva piyehi manāpehi nānābhāvo vinābhāvo aññathābhāvo’. Taṃ kutettha, āvuso, labbhā. ‘Yaṃ taṃ jātaṃ bhūtaṃ saṅkhataṃ palokadhammaṃ, taṃ tathāgatassāpi sarīraṃ mā palujjī’ti, netaṃ ṭhānaṃ vijjatī’’ti.

\paragraph{233.} Tena kho pana samayena cattāro mallapāmokkhā sīsaṃnhātā ahatāni vatthāni nivatthā – ‘‘mayaṃ bhagavato citakaṃ āḷimpessāmā’’ti na sakkonti āḷimpetuṃ. Atha kho kosinārakā mallā āyasmantaṃ anuruddhaṃ etadavocuṃ – ‘‘ko nu kho, bhante anuruddha, hetu ko paccayo, yenime cattāro mallapāmokkhā sīsaṃnhātā ahatāni vatthāni nivatthā – ‘mayaṃ bhagavato citakaṃ āḷimpessāmā’ti na sakkonti āḷimpetu’’nti? ‘‘Aññathā kho, vāseṭṭhā, devatānaṃ adhippāyo’’ti. ‘‘Kathaṃ pana, bhante, devatānaṃ adhippāyo’’ti? ‘‘Devatānaṃ kho, vāseṭṭhā, adhippāyo – ‘ayaṃ āyasmā mahākassapo pāvāya kusināraṃ addhānamaggappaṭippanno mahatā bhikkhusaṅghena saddhiṃ pañcamattehi bhikkhusatehi. Na tāva bhagavato citako pajjalissati, yāvāyasmā mahākassapo bhagavato pāde sirasā na vandissatī’’’ti. ‘‘Yathā, bhante, devatānaṃ adhippāyo, tathā hotū’’ti.

\paragraph{234.} Atha kho āyasmā mahākassapo yena kusinārā makuṭabandhanaṃ nāma mallānaṃ cetiyaṃ, yena bhagavato citako tenupasaṅkami; upasaṅkamitvā ekaṃsaṃ cīvaraṃ katvā añjaliṃ paṇāmetvā tikkhattuṃ citakaṃ padakkhiṇaṃ katvā bhagavato pāde sirasā vandi. Tānipi kho pañcabhikkhusatāni ekaṃsaṃ cīvaraṃ katvā añjaliṃ paṇāmetvā tikkhattuṃ citakaṃ padakkhiṇaṃ katvā bhagavato pāde sirasā vandiṃsu. Vandite ca panāyasmatā mahākassapena tehi ca pañcahi bhikkhusatehi sayameva bhagavato citako pajjali.

\paragraph{235.} Jhāyamānassa kho pana bhagavato sarīrassa yaṃ ahosi chavīti vā cammanti vā maṃsanti vā nhārūti vā lasikāti vā, tassa neva chārikā paññāyittha, na masi; sarīrāneva avasissiṃsu. Seyyathāpi nāma sappissa vā telassa vā jhāyamānassa neva chārikā paññāyati, na masi; evameva bhagavato sarīrassa jhāyamānassa yaṃ ahosi chavīti vā cammanti vā maṃsanti vā nhārūti vā lasikāti vā, tassa neva chārikā paññāyittha, na masi; sarīrāneva avasissiṃsu. Tesañca pañcannaṃ dussayugasatānaṃ dveva dussāni na ḍayhiṃsu yañca sabbaabbhantarimaṃ yañca bāhiraṃ. Daḍḍhe ca kho pana bhagavato sarīre antalikkhā udakadhārā pātubhavitvā bhagavato citakaṃ nibbāpesi. Udakasālatopi\footnote{udakaṃ sālatopi (sī. syā. kaṃ.)} abbhunnamitvā bhagavato citakaṃ nibbāpesi. Kosinārakāpi mallā sabbagandhodakena bhagavato citakaṃ nibbāpesuṃ. Atha kho kosinārakā mallā bhagavato sarīrāni sattāhaṃ sandhāgāre sattipañjaraṃ karitvā dhanupākāraṃ parikkhipāpetvā\footnote{parikkhipitvā (syā.)} naccehi gītehi vāditehi mālehi gandhehi sakkariṃsu garuṃ kariṃsu mānesuṃ pūjesuṃ.

\subsubsection{Sarīradhātuvibhājanaṃ}

\paragraph{236.} Assosi kho rājā māgadho ajātasattu vedehiputto – ‘‘bhagavā kira kusinārāyaṃ parinibbuto’’ti. Atha kho rājā māgadho ajātasattu vedehiputto kosinārakānaṃ mallānaṃ dūtaṃ pāhesi – ‘‘bhagavāpi khattiyo ahampi khattiyo, ahampi arahāmi bhagavato sarīrānaṃ bhāgaṃ, ahampi bhagavato sarīrānaṃ thūpañca mahañca karissāmī’’ti.

Assosuṃ kho vesālikā licchavī – ‘‘bhagavā kira kusinārāyaṃ parinibbuto’’ti. Atha kho vesālikā licchavī kosinārakānaṃ mallānaṃ dūtaṃ pāhesuṃ – ‘‘bhagavāpi khattiyo mayampi khattiyā, mayampi arahāma bhagavato sarīrānaṃ bhāgaṃ, mayampi bhagavato sarīrānaṃ thūpañca mahañca karissāmā’’ti.

Assosuṃ kho kapilavatthuvāsī sakyā – ‘‘bhagavā kira kusinārāyaṃ parinibbuto’’ti. Atha kho kapilavatthuvāsī sakyā kosinārakānaṃ mallānaṃ dūtaṃ pāhesuṃ – ‘‘bhagavā amhākaṃ ñātiseṭṭho , mayampi arahāma bhagavato sarīrānaṃ bhāgaṃ, mayampi bhagavato sarīrānaṃ thūpañca mahañca karissāmā’’ti.

Assosuṃ kho allakappakā bulayo\footnote{thūlayo (syā.)} – ‘‘bhagavā kira kusinārāyaṃ parinibbuto’’ti. Atha kho allakappakā bulayo kosinārakānaṃ mallānaṃ dūtaṃ pāhesuṃ – ‘‘bhagavāpi khattiyo mayampi khattiyā, mayampi arahāma bhagavato sarīrānaṃ bhāgaṃ, mayampi bhagavato sarīrānaṃ thūpañca mahañca karissāmā’’ti .

Assosuṃ kho rāmagāmakā koḷiyā – ‘‘bhagavā kira kusinārāyaṃ parinibbuto’’ti. Atha kho rāmagāmakā koḷiyā kosinārakānaṃ mallānaṃ dūtaṃ pāhesuṃ – ‘‘bhagavāpi khattiyo mayampi khattiyā, mayampi arahāma bhagavato sarīrānaṃ bhāgaṃ, mayampi bhagavato sarīrānaṃ thūpañca mahañca karissāmā’’ti.

Assosi kho veṭṭhadīpako brāhmaṇo – ‘‘bhagavā kira kusinārāyaṃ parinibbuto’’ti. Atha kho veṭṭhadīpako brāhmaṇo kosinārakānaṃ mallānaṃ dūtaṃ pāhesi – ‘‘bhagavāpi khattiyo ahaṃ pismi brāhmaṇo, ahampi arahāmi bhagavato sarīrānaṃ bhāgaṃ, ahampi bhagavato sarīrānaṃ thūpañca mahañca karissāmī’’ti.

Assosuṃ kho pāveyyakā mallā – ‘‘bhagavā kira kusinārāyaṃ parinibbuto’’ti. Atha kho pāveyyakā mallā kosinārakānaṃ mallānaṃ dūtaṃ pāhesuṃ – ‘‘bhagavāpi khattiyo mayampi khattiyā, mayampi arahāma bhagavato sarīrānaṃ bhāgaṃ, mayampi bhagavato sarīrānaṃ thūpañca mahañca karissāmā’’ti.

Evaṃ vutte kosinārakā mallā te saṅghe gaṇe etadavocuṃ – ‘‘bhagavā amhākaṃ gāmakkhette parinibbuto, na mayaṃ dassāma bhagavato sarīrānaṃ bhāga’’nti.

\paragraph{237.} Evaṃ vutte doṇo brāhmaṇo te saṅghe gaṇe etadavoca –

‘‘Suṇantu bhonto mama ekavācaṃ,

Amhāka\footnote{chandānurakkhaṇatthaṃ niggahītalopo}; Buddho ahu khantivādo;

Na hi sādhu yaṃ uttamapuggalassa,

Sarīrabhāge siyā sampahāro.

Sabbeva bhonto sahitā samaggā,

Sammodamānā karomaṭṭhabhāge;

Vitthārikā hontu disāsu thūpā,

Bahū janā cakkhumato pasannā’’ti.

\paragraph{238.} ‘‘Tena hi, brāhmaṇa, tvaññeva bhagavato sarīrāni aṭṭhadhā samaṃ savibhattaṃ vibhajāhī’’ti. ‘‘Evaṃ, bho’’ti kho doṇo brāhmaṇo tesaṃ saṅghānaṃ gaṇānaṃ paṭissutvā bhagavato sarīrāni aṭṭhadhā samaṃ suvibhattaṃ vibhajitvā te saṅghe gaṇe etadavoca – ‘‘imaṃ me bhonto tumbaṃ dadantu ahampi tumbassa thūpañca mahañca karissāmī’’ti. Adaṃsu kho te doṇassa brāhmaṇassa tumbaṃ.

Assosuṃ kho pippalivaniyā\footnote{pipphalivaniyā (syā.)} moriyā – ‘‘bhagavā kira kusinārāyaṃ parinibbuto’’ti. Atha kho pippalivaniyā moriyā kosinārakānaṃ mallānaṃ dūtaṃ pāhesuṃ – ‘‘bhagavāpi khattiyo mayampi khattiyā, mayampi arahāma bhagavato sarīrānaṃ bhāgaṃ, mayampi bhagavato sarīrānaṃ thūpañca mahañca karissāmā’’ti. ‘‘Natthi bhagavato sarīrānaṃ bhāgo, vibhattāni bhagavato sarīrāni. Ito aṅgāraṃ harathā’’ti. Te tato aṅgāraṃ hariṃsu\footnote{āhariṃsu (syā. ka.)}.

\subsubsection{Dhātuthūpapūjā}

\paragraph{239.} Atha kho rājā māgadho ajātasattu vedehiputto rājagahe bhagavato sarīrānaṃ thūpañca mahañca akāsi. Vesālikāpi licchavī vesāliyaṃ bhagavato sarīrānaṃ thūpañca mahañca akaṃsu. Kapilavatthuvāsīpi sakyā kapilavatthusmiṃ bhagavato sarīrānaṃ thūpañca mahañca akaṃsu. Allakappakāpi bulayo allakappe bhagavato sarīrānaṃ thūpañca mahañca akaṃsu. Rāmagāmakāpi koḷiyā rāmagāme bhagavato sarīrānaṃ thūpañca mahañca akaṃsu. Veṭṭhadīpakopi brāhmaṇo veṭṭhadīpe bhagavato sarīrānaṃ thūpañca mahañca akāsi. Pāveyyakāpi mallā pāvāyaṃ bhagavato sarīrānaṃ thūpañca mahañca akaṃsu. Kosinārakāpi mallā kusinārāyaṃ bhagavato sarīrānaṃ thūpañca mahañca akaṃsu. Doṇopi brāhmaṇo tumbassa thūpañca mahañca akāsi. Pippalivaniyāpi moriyā pippalivane aṅgārānaṃ thūpañca mahañca akaṃsu. Iti aṭṭha sarīrathūpā navamo tumbathūpo dasamo aṅgārathūpo. Evametaṃ bhūtapubbanti.

\paragraph{240.} Aṭṭhadoṇaṃ cakkhumato sarīraṃ, sattadoṇaṃ jambudīpe mahenti.

Ekañca doṇaṃ purisavaruttamassa, rāmagāme nāgarājā maheti.

Ekāhi dāṭhā tidivehi pūjitā, ekā pana gandhārapure mahīyati;

Kāliṅgarañño vijite punekaṃ, ekaṃ pana nāgarājā maheti.

Tasseva tejena ayaṃ vasundharā,

Āyāgaseṭṭhehi mahī alaṅkatā;

Evaṃ imaṃ cakkhumato sarīraṃ,

Susakkataṃ sakkatasakkatehi.

Devindanāgindanarindapūjito ,

Manussindaseṭṭhehi tatheva pūjito;

Taṃ vandatha\footnote{taṃ taṃ vandatha (syā.)} pañjalikā labhitvā,

Buddho have kappasatehi dullabhoti.

Cattālīsa samā dantā, kesā lomā ca sabbaso;

Devā hariṃsu ekekaṃ, cakkavāḷaparamparāti.

\xsectionEnd{Mahāparinibbānasuttaṃ niṭṭhitaṃ tatiyaṃ.}


\clearpage
\section{Mahāsudassanasuttaṃ}

\paragraph{241.} Evaṃ me sutaṃ – ekaṃ samayaṃ bhagavā kusinārāyaṃ viharati upavattane mallānaṃ sālavane antarena yamakasālānaṃ parinibbānasamaye. Atha kho āyasmā ānando yena bhagavā tenupasaṅkami; upasaṅkamitvā bhagavantaṃ abhivādetvā ekamantaṃ nisīdi. Ekamantaṃ nisinno kho āyasmā ānando bhagavantaṃ etadavoca – ‘‘mā, bhante, bhagavā imasmiṃ khuddakanagarake ujjaṅgalanagarake sākhānagarake parinibbāyi. Santi, bhante, aññāni mahānagarāni. Seyyathidaṃ – campā, rājagahaṃ, sāvatthi, sāketaṃ, kosambī, bārāṇasī; ettha bhagavā parinibbāyatu. Ettha bahū khattiyamahāsālā brāhmaṇamahāsālā gahapatimahāsālā tathāgate abhippasannā, te tathāgatassa sarīrapūjaṃ karissantī’’ti.

\paragraph{242.} ‘‘Mā hevaṃ, ānanda, avaca; mā hevaṃ, ānanda, avaca – khuddakanagarakaṃ ujjaṅgalanagarakaṃ sākhānagaraka’’nti.

\subsubsection{Kusāvatīrājadhānī}

‘‘Bhūtapubbaṃ, ānanda, rājā mahāsudassano nāma ahosi khattiyo muddhāvasitto\footnote{khattiyo muddhābhisitto (ka.), cakkavattīdhammiko dhammarājā (mahāparinibbānasutta)} cāturanto vijitāvī janapadatthāvariyappatto . Rañño, ānanda, mahāsudassanassa ayaṃ kusinārā kusāvatī nāma rājadhānī ahosi. Puratthimena ca pacchimena ca dvādasayojanāni āyāmena, uttarena ca dakkhiṇena ca sattayojanāni vitthārena. Kusāvatī, ānanda, rājadhānī iddhā ceva ahosi phītā ca bahujanā ca ākiṇṇamanussā ca subhikkhā ca. Seyyathāpi, ānanda , devānaṃ āḷakamandā nāma rājadhānī iddhā ceva hoti phītā ca\footnote{iddhā ceva ahosi phītā ca (syā.)} bahujanā ca ākiṇṇayakkhā ca subhikkhā ca; evameva kho, ānanda, kusāvatī rājadhānī iddhā ceva ahosi phītā ca bahujanā ca ākiṇṇamanussā ca subhikkhā ca. Kusāvatī, ānanda , rājadhānī dasahi saddehi avivittā ahosi divā ceva rattiñca, seyyathidaṃ – hatthisaddena assasaddena rathasaddena bherisaddena mudiṅgasaddena vīṇāsaddena gītasaddena saṅkhasaddena sammasaddena pāṇitāḷasaddena ‘asnātha pivatha khādathā’ti dasamena saddena.

‘‘Kusāvatī, ānanda, rājadhānī sattahi pākārehi parikkhittā ahosi. Eko pākāro sovaṇṇamayo, eko rūpiyamayo, eko veḷuriyamayo, eko phalikamayo, eko lohitaṅkamayo\footnote{lohitaṅgamayo (ka.), lohitakamayo (byākaraṇesu)}, eko masāragallamayo, eko sabbaratanamayo. Kusāvatiyā, ānanda, rājadhāniyā catunnaṃ vaṇṇānaṃ dvārāni ahesuṃ. Ekaṃ dvāraṃ sovaṇṇamayaṃ, ekaṃ rūpiyamayaṃ, ekaṃ veḷuriyamayaṃ, ekaṃ phalikamayaṃ . Ekekasmiṃ dvāre satta satta esikā nikhātā ahesuṃ tiporisaṅgā tiporisanikhātā dvādasaporisā ubbedhena. Ekā esikā sovaṇṇamayā, ekā rūpiyamayā, ekā veḷuriyamayā, ekā phalikamayā, ekā lohitaṅkamayā, ekā masāragallamayā, ekā sabbaratanamayā. Kusāvatī, ānanda, rājadhānī sattahi tālapantīhi parikkhittā ahosi. Ekā tālapanti sovaṇṇamayā, ekā rūpiyamayā, ekā veḷuriyamayā, ekā phalikamayā, ekā lohitaṅkamayā, ekā masāragallamayā, ekā sabbaratanamayā. Sovaṇṇamayassa tālassa sovaṇṇamayo khandho ahosi, rūpiyamayāni pattāni ca phalāni ca. Rūpiyamayassa tālassa rūpiyamayo khandho ahosi, sovaṇṇamayāni pattāni ca phalāni ca. Veḷuriyamayassa tālassa veḷuriyamayo khandho ahosi, phalikamayāni pattāni ca phalāni ca. Phalikamayassa tālassa phalikamayo khandho ahosi, veḷuriyamayāni pattāni ca phalāni ca. Lohitaṅkamayassa tālassa lohitaṅkamayo khandho ahosi, masāragallamayāni pattāni ca phalāni ca. Masāragallamayassa tālassa masāragallamayo khandho ahosi, lohitaṅkamayāni pattāni ca phalāni ca. Sabbaratanamayassa tālassa sabbaratanamayo khandho ahosi, sabbaratanamayāni pattāni ca phalāni ca. Tāsaṃ kho panānanda, tālapantīnaṃ vāteritānaṃ saddo ahosi vaggu ca rajanīyo ca khamanīyo\footnote{kamanīyo (sī. syā. pī.)} ca madanīyo ca. Seyyathāpi, ānanda, pañcaṅgikassa tūriyassa suvinītassa suppaṭitāḷitassa sukusalehi samannāhatassa saddo hoti vaggu ca rajanīyo ca khamanīyo ca madanīyo ca , evameva kho, ānanda, tāsaṃ tālapantīnaṃ vāteritānaṃ saddo ahosi vaggu ca rajanīyo ca khamanīyo ca madanīyo ca. Ye kho panānanda, tena samayena kusāvatiyā rājadhāniyā dhuttā ahesuṃ soṇḍā pipāsā, te tāsaṃ tālapantīnaṃ vāteritānaṃ saddena paricāresuṃ.

\subsubsection{Cakkaratanaṃ}

\paragraph{243.} ‘‘Rājā , ānanda, mahāsudassano sattahi ratanehi samannāgato ahosi catūhi ca iddhīhi. Katamehi sattahi? Idhānanda, rañño mahāsudassanassa tadahuposathe pannarase sīsaṃnhātassa uposathikassa uparipāsādavaragatassa dibbaṃ cakkaratanaṃ pāturahosi sahassāraṃ sanemikaṃ sanābhikaṃ sabbākāraparipūraṃ. Disvā rañño mahāsudassanassa etadahosi – ‘sutaṃ kho panetaṃ – ‘‘yassa rañño khattiyassa muddhāvasittassa tadahuposathe pannarase sīsaṃnhātassa uposathikassa uparipāsādavaragatassa dibbaṃ cakkaratanaṃ pātubhavati sahassāraṃ sanemikaṃ sanābhikaṃ sabbākāraparipūraṃ, so hoti rājā cakkavattī’’ti. Assaṃ nu kho ahaṃ rājā cakkavattī’ti.

\paragraph{244.} ‘‘Atha kho, ānanda, rājā mahāsudassano uṭṭhāyāsanā ekaṃsaṃ uttarāsaṅgaṃ karitvā vāmena hatthena suvaṇṇabhiṅkāraṃ gahetvā dakkhiṇena hatthena cakkaratanaṃ abbhukkiri – ‘pavattatu bhavaṃ cakkaratanaṃ, abhivijinātu bhavaṃ cakkaratana’nti. Atha kho taṃ, ānanda, cakkaratanaṃ puratthimaṃ disaṃ pavatti\footnote{pavattati (syā. ka.)}, anvadeva\footnote{anudeva (syā.)} rājā mahāsudassano saddhiṃ caturaṅginiyā senāya, yasmiṃ kho panānanda, padese cakkaratanaṃ patiṭṭhāsi, tattha rājā mahāsudassano vāsaṃ upagacchi saddhiṃ caturaṅginiyā senāya. Ye kho panānanda, puratthimāya disāya paṭirājāno, te rājānaṃ mahāsudassanaṃ upasaṅkamitvā evamāhaṃsu – ‘ehi kho mahārāja, svāgataṃ te mahārāja, sakaṃ te mahārāja, anusāsa mahārājā’ti. Rājā mahāsudassano evamāha – ‘pāṇo na hantabbo, adinnaṃ na ādātabbaṃ, kāmesu micchā na caritabbā, musā na bhaṇitabbā, majjaṃ na pātabbaṃ, yathābhuttañca bhuñjathā’ti . Ye kho panānanda, puratthimāya disāya paṭirājāno, te rañño mahāsudassanassa anuyantā ahesuṃ. Atha kho taṃ, ānanda, cakkaratanaṃ puratthimaṃ samuddaṃ ajjhogāhetvā paccuttaritvā dakkhiṇaṃ disaṃ pavatti…pe… dakkhiṇaṃ samuddaṃ ajjhogāhetvā paccuttaritvā pacchimaṃ disaṃ pavatti…pe… pacchimaṃ samuddaṃ ajjhogāhetvā paccuttaritvā uttaraṃ disaṃ pavatti, anvadeva rājā mahāsudassano saddhiṃ caturaṅginiyā senāya. Yasmiṃ kho panānanda, padese cakkaratanaṃ patiṭṭhāsi, tattha rājā mahāsudassano vāsaṃ upagacchi saddhiṃ caturaṅginiyā senāya. Ye kho panānanda, uttarāya disāya paṭirājāno, te rājānaṃ mahāsudassanaṃ upasaṅkamitvā evamāhaṃsu – ‘ehi kho mahārāja, svāgataṃ te mahārāja, sakaṃ te mahārāja, anusāsa mahārājā’ti. Rājā mahāsudassano evamāha – ‘pāṇo na hantabbo, adinnaṃ na ādātabbaṃ, kāmesu micchā na caritabbā, musā na bhaṇitabbā, majjaṃ na pātabbaṃ , yathābhuttañca bhuñjathā’ti. Ye kho panānanda, uttarāya disāya paṭirājāno , te rañño mahāsudassanassa anuyantā ahesuṃ.

\paragraph{245.} ‘‘Atha kho taṃ, ānanda, cakkaratanaṃ samuddapariyantaṃ pathaviṃ abhivijinitvā kusāvatiṃ rājadhāniṃ paccāgantvā rañño mahāsudassanassa antepuradvāre atthakaraṇapamukhe akkhāhataṃ maññe aṭṭhāsi rañño mahāsudassanassa antepuraṃ upasobhayamānaṃ. Rañño, ānanda, mahāsudassanassa evarūpaṃ cakkaratanaṃ pāturahosi.

\subsubsection{Hatthiratanaṃ}

\paragraph{246.} ‘‘Puna caparaṃ, ānanda, rañño mahāsudassanassa hatthiratanaṃ pāturahosi sabbaseto sattappatiṭṭho iddhimā vehāsaṅgamo uposatho nāma nāgarājā. Taṃ disvā rañño mahāsudassanassa cittaṃ pasīdi – ‘bhaddakaṃ vata bho hatthiyānaṃ, sace damathaṃ upeyyā’ti. Atha kho taṃ, ānanda, hatthiratanaṃ – seyyathāpi nāma gandhahatthājāniyo dīgharattaṃ suparidanto, evameva damathaṃ upagacchi. Bhūtapubbaṃ, ānanda, rājā mahāsudassano tameva hatthiratanaṃ vīmaṃsamāno pubbaṇhasamayaṃ abhiruhitvā samuddapariyantaṃ pathaviṃ anuyāyitvā kusāvatiṃ rājadhāniṃ paccāgantvā pātarāsamakāsi. Rañño, ānanda, mahāsudassanassa evarūpaṃ hatthiratanaṃ pāturahosi.

\subsubsection{Assaratanaṃ}

\paragraph{247.} ‘‘Puna caparaṃ, ānanda, rañño mahāsudassanassa assaratanaṃ pāturahosi sabbaseto kāḷasīso muñjakeso iddhimā vehāsaṅgamo valāhako nāma assarājā. Taṃ disvā rañño mahāsudassanassa cittaṃ pasīdi – ‘bhaddakaṃ vata bho assayānaṃ sace damathaṃ upeyyā’ti. Atha kho taṃ , ānanda, assaratanaṃ seyyathāpi nāma bhaddo assājāniyo dīgharattaṃ suparidanto, evameva damathaṃ upagacchi. Bhūtapubbaṃ, ānanda, rājā mahāsudassano tameva assaratanaṃ vīmaṃsamāno pubbaṇhasamayaṃ abhiruhitvā samuddapariyantaṃ pathaviṃ anuyāyitvā kusāvatiṃ rājadhāniṃ paccāgantvā pātarāsamakāsi. Rañño, ānanda, mahāsudassanassa evarūpaṃ assaratanaṃ pāturahosi.

\subsubsection{Maṇiratanaṃ}

\paragraph{248.} ‘‘Puna caparaṃ, ānanda, rañño mahāsudassanassa maṇiratanaṃ pāturahosi. So ahosi maṇi veḷuriyo subho jātimā aṭṭhaṃso suparikammakato accho vippasanno anāvilo sabbākārasampanno. Tassa kho panānanda, maṇiratanassa ābhā samantā yojanaṃ phuṭā ahosi. Bhūtapubbaṃ, ānanda, rājā mahāsudassano tameva maṇiratanaṃ vīmaṃsamāno caturaṅginiṃ senaṃ sannayhitvā maṇiṃ dhajaggaṃ āropetvā rattandhakāratimisāya pāyāsi. Ye kho panānanda, samantā gāmā ahesuṃ, te tenobhāsena kammante payojesuṃ divāti maññamānā. Rañño, ānanda, mahāsudassanassa evarūpaṃ maṇiratanaṃ pāturahosi.

\subsubsection{Itthiratanaṃ}

\paragraph{249.} ‘‘Puna caparaṃ, ānanda, rañño mahāsudassanassa itthiratanaṃ pāturahosi abhirūpā dassanīyā pāsādikā paramāya vaṇṇapokkharatāya samannāgatā nātidīghā nātirassā nātikisā nātithūlā nātikāḷikā nāccodātā atikkantā mānusivaṇṇaṃ\footnote{mānussivaṇṇaṃ (syā.)} appattā dibbavaṇṇaṃ. Tassa kho panānanda, itthiratanassa evarūpo kāyasamphasso hoti, seyyathāpi nāma tūlapicuno vā kappāsapicuno vā. Tassa kho panānanda, itthiratanassa sīte uṇhāni gattāni honti, uṇhe sītāni. Tassa kho panānanda, itthiratanassa kāyato candanagandho vāyati, mukhato uppalagandho. Taṃ kho panānanda, itthiratanaṃ rañño mahāsudassanassa pubbuṭṭhāyinī ahosi pacchānipātinī kiṅkārapaṭissāvinī manāpacārinī piyavādinī. Taṃ kho panānanda, itthiratanaṃ rājānaṃ mahāsudassanaṃ manasāpi no aticari\footnote{aticarī (ka.), aticārī (sī. syā. pī.)}, kuto pana kāyena. Rañño, ānanda, mahāsudassanassa evarūpaṃ itthiratanaṃ pāturahosi.

\subsubsection{Gahapatiratanaṃ}

\paragraph{250.} ‘‘Puna caparaṃ, ānanda, rañño mahāsudassanassa gahapatiratanaṃ pāturahosi. Tassa kammavipākajaṃ dibbacakkhu pāturahosi yena nidhiṃ passati sassāmikampi assāmikampi. So rājānaṃ mahāsudassanaṃ upasaṅkamitvā evamāha – ‘appossukko tvaṃ, deva, hohi, ahaṃ te dhanena dhanakaraṇīyaṃ karissāmī’ti. Bhūtapubbaṃ, ānanda, rājā mahāsudassano tameva gahapatiratanaṃ vīmaṃsamāno nāvaṃ abhiruhitvā majjhe gaṅgāya nadiyā sotaṃ ogāhitvā gahapatiratanaṃ etadavoca – ‘attho me, gahapati, hiraññasuvaṇṇenā’ti. ‘Tena hi, mahārāja, ekaṃ tīraṃ nāvā upetū’ti. ‘Idheva me, gahapati, attho hiraññasuvaṇṇenā’ti. Atha kho taṃ, ānanda, gahapatiratanaṃ ubhohi hatthehi udakaṃ omasitvā pūraṃ hiraññasuvaṇṇassa kumbhiṃ uddharitvā rājānaṃ mahāsudassanaṃ etadavoca – ‘alamettāvatā mahārāja, katamettāvatā mahārāja, pūjitamettāvatā mahārājā’ti? Rājā mahāsudassano evamāha – ‘alamettāvatā gahapati, katamettāvatā gahapati, pūjitamettāvatā gahapatī’ti. Rañño , ānanda, mahāsudassanassa evarūpaṃ gahapatiratanaṃ pāturahosi.

\subsubsection{Pariṇāyakaratanaṃ}

\paragraph{251.} ‘‘Puna caparaṃ, ānanda, rañño mahāsudassanassa pariṇāyakaratanaṃ pāturahosi paṇḍito viyatto medhāvī paṭibalo rājānaṃ mahāsudassanaṃ upayāpetabbaṃ upayāpetuṃ, apayāpetabbaṃ apayāpetuṃ, ṭhapetabbaṃ ṭhapetuṃ. So rājānaṃ mahāsudassanaṃ upasaṅkamitvā evamāha – ‘appossukko tvaṃ, deva, hohi, ahamanusāsissāmī’ti. Rañño, ānanda, mahāsudassanassa evarūpaṃ pariṇāyakaratanaṃ pāturahosi.

‘‘Rājā, ānanda, mahāsudassano imehi sattahi ratanehi samannāgato ahosi.

\subsubsection{Catuiddhisamannāgato}

\paragraph{252.} ‘‘Rājā, ānanda, mahāsudassano catūhi iddhīhi samannāgato ahosi. Katamāhi catūhi iddhīhi? Idhānanda, rājā mahāsudassano abhirūpo ahosi dassanīyo pāsādiko paramāya vaṇṇapokkharatāya samannāgato ativiya aññehi manussehi. Rājā, ānanda, mahāsudassano imāya paṭhamāya iddhiyā samannāgato ahosi.

‘‘Puna caparaṃ, ānanda, rājā mahāsudassano dīghāyuko ahosi ciraṭṭhitiko ativiya aññehi manussehi. Rājā, ānanda, mahāsudassano imāya dutiyāya iddhiyā samannāgato ahosi.

‘‘Puna caparaṃ, ānanda, rājā mahāsudassano appābādho ahosi appātaṅko samavepākiniyā gahaṇiyā samannāgato nātisītāya nāccuṇhāya ativiya aññehi manussehi. Rājā, ānanda, mahāsudassano imāya tatiyāya iddhiyā samannāgato ahosi.

‘‘Puna caparaṃ , ānanda, rājā mahāsudassano brāhmaṇagahapatikānaṃ piyo ahosi manāpo. Seyyathāpi, ānanda, pitā puttānaṃ piyo hoti manāpo, evameva kho, ānanda, rājā mahāsudassano brāhmaṇagahapatikānaṃ piyo ahosi manāpo. Raññopi, ānanda, mahāsudassanassa brāhmaṇagahapatikā piyā ahesuṃ manāpā. Seyyathāpi, ānanda, pitu puttā piyā honti manāpā, evameva kho, ānanda, raññopi mahāsudassanassa brāhmaṇagahapatikā piyā ahesuṃ manāpā.

‘‘Bhūtapubbaṃ, ānanda, rājā mahāsudassano caturaṅginiyā senāya uyyānabhūmiṃ niyyāsi. Atha kho, ānanda, brāhmaṇagahapatikā rājānaṃ mahāsudassanaṃ upasaṅkamitvā evamāhaṃsu – ‘ataramāno, deva, yāhi, yathā taṃ mayaṃ cirataraṃ passeyyāmā’ti. Rājāpi, ānanda, mahāsudassano sārathiṃ āmantesi – ‘ataramāno, sārathi, rathaṃ pesehi, yathā ahaṃ brāhmaṇagahapatike cirataraṃ passeyya’nti. Rājā, ānanda, mahāsudassano imāya catutthiyā\footnote{catutthāya (syā.)} iddhiyā samannāgato ahosi. Rājā, ānanda, mahāsudassano imāhi catūhi iddhīhi samannāgato ahosi.

\subsubsection{Dhammapāsādapokkharaṇī}

\paragraph{253.} ‘‘Atha kho, ānanda, rañño mahāsudassanassa etadahosi – ‘yaṃnūnāhaṃ imāsu tālantarikāsu dhanusate dhanusate pokkharaṇiyo māpeyya’nti.

‘‘Māpesi kho, ānanda, rājā mahāsudassano tāsu tālantarikāsu dhanusate dhanusate pokkharaṇiyo. Tā kho panānanda, pokkharaṇiyo catunnaṃ vaṇṇānaṃ iṭṭhakāhi citā ahesuṃ – ekā iṭṭhakā sovaṇṇamayā, ekā rūpiyamayā, ekā veḷuriyamayā, ekā phalikamayā.

‘‘Tāsu kho panānanda, pokkharaṇīsu cattāri cattāri sopānāni ahesuṃ catunnaṃ vaṇṇānaṃ, ekaṃ sopānaṃ sovaṇṇamayaṃ ekaṃ rūpiyamayaṃ ekaṃ veḷuriyamayaṃ ekaṃ phalikamayaṃ. Sovaṇṇamayassa sopānassa sovaṇṇamayā thambhā ahesuṃ, rūpiyamayā sūciyo ca uṇhīsañca. Rūpiyamayassa sopānassa rūpiyamayā thambhā ahesuṃ, sovaṇṇamayā sūciyo ca uṇhīsañca. Veḷuriyamayassa sopānassa veḷuriyamayā thambhā ahesuṃ, phalikamayā sūciyo ca uṇhīsañca. Phalikamayassa sopānassa phalikamayā thambhā ahesuṃ, veḷuriyamayā sūciyo ca uṇhīsañca. Tā kho panānanda, pokkharaṇiyo dvīhi vedikāhi parikkhittā ahesuṃ ekā vedikā sovaṇṇamayā, ekā rūpiyamayā. Sovaṇṇamayāya vedikāya sovaṇṇamayā thambhā ahesuṃ, rūpiyamayā sūciyo ca uṇhīsañca. Rūpiyamayāya vedikāya rūpiyamayā thambhā ahesuṃ, sovaṇṇamayā sūciyo ca uṇhīsañca. Atha kho, ānanda , rañño mahāsudassanassa etadahosi – ‘yaṃnūnāhaṃ imāsu pokkharaṇīsu evarūpaṃ mālaṃ ropāpeyyaṃ uppalaṃ padumaṃ kumudaṃ puṇḍarīkaṃ sabbotukaṃ sabbajanassa anāvaṭa’nti. Ropāpesi kho , ānanda, rājā mahāsudassano tāsu pokkharaṇīsu evarūpaṃ mālaṃ uppalaṃ padumaṃ kumudaṃ puṇḍarīkaṃ sabbotukaṃ sabbajanassa anāvaṭaṃ.

\paragraph{254.} ‘‘Atha kho, ānanda, rañño mahāsudassanassa etadahosi – ‘yaṃnūnāhaṃ imāsaṃ pokkharaṇīnaṃ tīre nhāpake purise ṭhapeyyaṃ, ye āgatāgataṃ janaṃ nhāpessantī’ti. Ṭhapesi kho, ānanda, rājā mahāsudassano tāsaṃ pokkharaṇīnaṃ tīre nhāpake purise, ye āgatāgataṃ janaṃ nhāpesuṃ.

‘‘Atha kho, ānanda, rañño mahāsudassanassa etadahosi – ‘yaṃnūnāhaṃ imāsaṃ pokkharaṇīnaṃ tīre evarūpaṃ dānaṃ paṭṭhapeyyaṃ – annaṃ annaṭṭhikassa\footnote{annatthitassa (sī. syā. kaṃ. pī.), evaṃ sabbattha pakabhirūpeneva dissati}, pānaṃ pānaṭṭhikassa, vatthaṃ vatthaṭṭhikassa, yānaṃ yānaṭṭhikassa, sayanaṃ sayanaṭṭhikassa, itthiṃ itthiṭṭhikassa, hiraññaṃ hiraññaṭṭhikassa, suvaṇṇaṃ suvaṇṇaṭṭhikassā’ti. Paṭṭhapesi kho, ānanda, rājā mahāsudassano tāsaṃ pokkharaṇīnaṃ tīre evarūpaṃ dānaṃ – annaṃ annaṭṭhikassa, pānaṃ pānaṭṭhikassa, vatthaṃ vatthaṭṭhikassa, yānaṃ yānaṭṭhikassa, sayanaṃ sayanaṭṭhikassa, itthiṃ itthiṭṭhikassa, hiraññaṃ hiraññaṭṭhikassa, suvaṇṇaṃ suvaṇṇaṭṭhikassa.

\paragraph{255.} ‘‘Atha kho, ānanda, brāhmaṇagahapatikā pahūtaṃ sāpateyyaṃ ādāya rājānaṃ mahāsudassanaṃ upasaṅkamitvā evamāhaṃsu – ‘idaṃ, deva, pahūtaṃ sāpateyyaṃ devaññeva uddissa ābhataṃ, taṃ devo paṭiggaṇhatū’ti. ‘Alaṃ bho, mamapidaṃ pahūtaṃ sāpateyyaṃ dhammikena balinā abhisaṅkhataṃ, tañca vo hotu, ito ca bhiyyo harathā’ti. Te raññā paṭikkhittā ekamantaṃ apakkamma evaṃ samacintesuṃ – ‘na kho etaṃ amhākaṃ patirūpaṃ, yaṃ mayaṃ imāni sāpateyyāni punadeva sakāni gharāni paṭihareyyāma. Yaṃnūna mayaṃ rañño mahāsudassanassa nivesanaṃ māpeyyāmā’ti. Te rājānaṃ mahāsudassanaṃ upasaṅkamitvā evamāhaṃsu – ‘nivesanaṃ te deva, māpessāmā’ti. Adhivāsesi kho, ānanda, rājā mahāsudassano tuṇhībhāvena.

\paragraph{256.} ‘‘Atha kho, ānanda, sakko devānamindo rañño mahāsudassanassa cetasā cetoparivitakkamaññāya vissakammaṃ\footnote{visukammaṃ (ka.)} devaputtaṃ āmantesi – ‘ehi tvaṃ, samma vissakamma, rañño mahāsudassanassa nivesanaṃ māpehi dhammaṃ nāma pāsāda’nti. ‘Evaṃ bhaddantavā’ti kho, ānanda, vissakammo devaputto sakkassa devānamindassa paṭissutvā seyyathāpi nāma balavā puriso samiñjitaṃ vā bāhaṃ pasāreyya pasāritaṃ vā bāhaṃ samiñjeyya, evameva devesu tāvatiṃsesu antarahito rañño mahāsudassanassa purato pāturahosi. Atha kho, ānanda, vissakammo devaputto rājānaṃ mahāsudassanaṃ etadavoca – ‘nivesanaṃ te deva, māpessāmi dhammaṃ nāma pāsāda’nti. Adhivāsesi kho, ānanda, rājā mahāsudassano tuṇhībhāvena.

‘‘Māpesi kho, ānanda, vissakammo devaputto rañño mahāsudassanassa nivesanaṃ dhammaṃ nāma pāsādaṃ. Dhammo, ānanda, pāsādo puratthimena pacchimena ca yojanaṃ āyāmena ahosi. Uttarena dakkhiṇena ca aḍḍhayojanaṃ vitthārena. Dhammassa, ānanda, pāsādassa tiporisaṃ uccatarena vatthu citaṃ ahosi catunnaṃ vaṇṇānaṃ iṭṭhakāhi – ekā iṭṭhakā sovaṇṇamayā, ekā rūpiyamayā, ekā veḷuriyamayā, ekā phalikamayā.

‘‘Dhammassa, ānanda, pāsādassa caturāsīti thambhasahassāni ahesuṃ catunnaṃ vaṇṇānaṃ – eko thambho sovaṇṇamayo, eko rūpiyamayo, eko veḷuriyamayo, eko phalikamayo. Dhammo, ānanda, pāsādo catunnaṃ vaṇṇānaṃ phalakehi santhato ahosi – ekaṃ phalakaṃ sovaṇṇamayaṃ, ekaṃ rūpiyamayaṃ, ekaṃ veḷuriyamayaṃ, ekaṃ phalikamayaṃ.

‘‘Dhammassa, ānanda, pāsādassa catuvīsati sopānāni ahesuṃ catunnaṃ vaṇṇānaṃ – ekaṃ sopānaṃ sovaṇṇamayaṃ, ekaṃ rūpiyamayaṃ, ekaṃ veḷuriyamayaṃ, ekaṃ phalikamayaṃ. Sovaṇṇamayassa sopānassa sovaṇṇamayā thambhā ahesuṃ rūpiyamayā sūciyo ca uṇhīsañca. Rūpiyamayassa sopānassa rūpiyamayā thambhā ahesuṃ sovaṇṇamayā sūciyo ca uṇhīsañca. Veḷuriyamayassa sopānassa veḷuriyamayā thambhā ahesuṃ phalikamayā sūciyo ca uṇhīsañca. Phalikamayassa sopānassa phalikamayā thambhā ahesuṃ veḷuriyamayā sūciyo ca uṇhīsañca.

‘‘Dhamme, ānanda, pāsāde caturāsīti kūṭāgārasahassāni ahesuṃ catunnaṃ vaṇṇānaṃ – ekaṃ kūṭāgāraṃ sovaṇṇamayaṃ, ekaṃ rūpiyamayaṃ, ekaṃ veḷuriyamayaṃ , ekaṃ phalikamayaṃ. Sovaṇṇamaye kūṭāgāre rūpiyamayo pallaṅko paññatto ahosi, rūpiyamaye kūṭāgāre sovaṇṇamayo pallaṅko paññatto ahosi, veḷuriyamaye kūṭāgāre dantamayo pallaṅko paññatto ahosi, phalikamaye kūṭāgāre sāramayo pallaṅko paññatto ahosi. Sovaṇṇamayassa kūṭāgārassa dvāre rūpiyamayo tālo ṭhito ahosi, tassa rūpiyamayo khandho sovaṇṇamayāni pattāni ca phalāni ca. Rūpiyamayassa kūṭāgārassa dvāre sovaṇṇamayo tālo ṭhito ahosi, tassa sovaṇṇamayo khandho, rūpiyamayāni pattāni ca phalāni ca. Veḷuriyamayassa kūṭāgārassa dvāre phalikamayo tālo ṭhito ahosi, tassa phalikamayo khandho, veḷuriyamayāni pattāni ca phalāni ca. Phalikamayassa kūṭāgārassa dvāre veḷuriyamayo tālo ṭhito ahosi, tassa veḷuriyamayo khandho, phalikamayāni pattāni ca phalāni ca.

\paragraph{257.} ‘‘Atha kho, ānanda, rañño mahāsudassanassa etadahosi – ‘yaṃnūnāhaṃ mahāviyūhassa kūṭāgārassa dvāre sabbasovaṇṇamayaṃ tālavanaṃ māpeyyaṃ, yattha divāvihāraṃ nisīdissāmī’ti. Māpesi kho, ānanda, rājā mahāsudassano mahāviyūhassa kūṭāgārassa dvāre sabbasovaṇṇamayaṃ tālavanaṃ, yattha divāvihāraṃ nisīdi. Dhammo, ānanda , pāsādo dvīhi vedikāhi parikkhitto ahosi, ekā vedikā sovaṇṇamayā, ekā rūpiyamayā. Sovaṇṇamayāya vedikāya sovaṇṇamayā thambhā ahesuṃ, rūpiyamayā sūciyo ca uṇhīsañca. Rūpiyamayāya vedikāya rūpiyamayā thambhā ahesuṃ, sovaṇṇamayā sūciyo ca uṇhīsañca.

\paragraph{258.} ‘‘Dhammo, ānanda, pāsādo dvīhi kiṅkiṇikajālehi\footnote{kiṅkaṇikajālehi (syā. ka.)} parikkhitto ahosi – ekaṃ jālaṃ sovaṇṇamayaṃ ekaṃ rūpiyamayaṃ. Sovaṇṇamayassa jālassa rūpiyamayā kiṅkiṇikā ahesuṃ, rūpiyamayassa jālassa sovaṇṇamayā kiṅkiṇikā ahesuṃ. Tesaṃ kho panānanda, kiṅkiṇikajālānaṃ vāteritānaṃ saddo ahosi vaggu ca rajanīyo ca khamanīyo ca madanīyo ca. Seyyathāpi, ānanda, pañcaṅgikassa tūriyassa suvinītassa suppaṭitāḷitassa sukusalehi\footnote{kusalehi (sī. syā. kaṃ. pī.)} samannāhatassa saddo hoti, vaggu ca rajanīyo ca khamanīyo ca madanīyo ca, evameva kho, ānanda, tesaṃ kiṅkiṇikajālānaṃ vāteritānaṃ saddo ahosi vaggu ca rajanīyo ca khamanīyo ca madanīyo ca. Ye kho panānanda, tena samayena kusāvatiyā rājadhāniyā dhuttā ahesuṃ soṇḍā pipāsā, te tesaṃ kiṅkiṇikajālānaṃ vāteritānaṃ saddena paricāresuṃ. Niṭṭhito kho panānanda, dhammo pāsādo duddikkho ahosi musati cakkhūni. Seyyathāpi, ānanda, vassānaṃ pacchime māse saradasamaye viddhe vigatavalāhake deve ādicco nabhaṃ abbhussakkamāno\footnote{abbhuggamamāno (sī. pī. ka.)} duddikkho\footnote{dudikkho (pī.)} hoti musati cakkhūni; evameva kho, ānanda, dhammo pāsādo duddikkho ahosi musati cakkhūni.

\paragraph{259.} ‘‘Atha kho, ānanda, rañño mahāsudassanassa etadahosi – ‘yaṃnūnāhaṃ dhammassa pāsādassa purato dhammaṃ nāma pokkharaṇiṃ māpeyya’nti. Māpesi kho, ānanda, rājā mahāsudassano dhammassa pāsādassa purato dhammaṃ nāma pokkharaṇiṃ. Dhammā, ānanda, pokkharaṇī puratthimena pacchimena ca yojanaṃ āyāmena ahosi, uttarena dakkhiṇena ca aḍḍhayojanaṃ vitthārena. Dhammā, ānanda, pokkharaṇī catunnaṃ vaṇṇānaṃ iṭṭhakāhi citā ahosi – ekā iṭṭhakā sovaṇṇamayā, ekā rūpiyamayā, ekā veḷuriyamayā, ekā phalikamayā.

‘‘Dhammāya, ānanda, pokkharaṇiyā catuvīsati sopānāni ahesuṃ catunnaṃ vaṇṇānaṃ – ekaṃ sopānaṃ sovaṇṇamayaṃ, ekaṃ rūpiyamayaṃ, ekaṃ veḷuriyamayaṃ, ekaṃ phalikamayaṃ. Sovaṇṇamayassa sopānassa sovaṇṇamayā thambhā ahesuṃ rūpiyamayā sūciyo ca uṇhīsañca. Rūpiyamayassa sopānassa rūpiyamayā thambhā ahesuṃ sovaṇṇamayā sūciyo ca uṇhīsañca. Veḷuriyamayassa sopānassa veḷuriyamayā thambhā ahesuṃ phalikamayā sūciyo ca uṇhīsañca. Phalikamayassa sopānassa phalikamayā thambhā ahesuṃ veḷuriyamayā sūciyo ca uṇhīsañca.

‘‘Dhammā, ānanda, pokkharaṇī dvīhi vedikāhi parikkhittā ahosi – ekā vedikā sovaṇṇamayā, ekā rūpiyamayā. Sovaṇṇamayāya vedikāya sovaṇṇamayā thambhā ahesuṃ rūpiyamayā sūciyo ca uṇhīsañca. Rūpiyamayāya vedikāya rūpiyamayā thambhā ahesuṃ sovaṇṇamayā sūciyo ca uṇhīsañca.

‘‘Dhammā, ānanda, pokkharaṇī sattahi tālapantīhi parikkhittā ahosi – ekā tālapanti sovaṇṇamayā, ekā rūpiyamayā, ekā veḷuriyamayā, ekā phalikamayā, ekā lohitaṅkamayā, ekā masāragallamayā, ekā sabbaratanamayā. Sovaṇṇamayassa tālassa sovaṇṇamayo khandho ahosi rūpiyamayāni pattāni ca phalāni ca. Rūpiyamayassa tālassa rūpiyamayo khandho ahosi sovaṇṇamayāni pattāni ca phalāni ca. Veḷuriyamayassa tālassa veḷuriyamayo khandho ahosi phalikamayāni pattāni ca phalāni ca. Phalikamayassa tālassa phalikamayo khandho ahosi veḷuriyamayāni pattāni ca phalāni ca. Lohitaṅkamayassa tālassa lohitaṅkamayo khandho ahosi masāragallamayāni pattāni ca phalāni ca. Masāragallamayassa tālassa masāragallamayo khandho ahosi lohitaṅkamayāni pattāni ca phalāni ca. Sabbaratanamayassa tālassa sabbaratanamayo khandho ahosi, sabbaratanamayāni pattāni ca phalāni ca. Tāsaṃ kho panānanda, tālapantīnaṃ vāteritānaṃ saddo ahosi, vaggu ca rajanīyo ca khamanīyo ca madanīyo ca. Seyyathāpi, ānanda, pañcaṅgikassa tūriyassa suvinītassa suppaṭitāḷitassa sukusalehi samannāhatassa saddo hoti vaggu ca rajanīyo ca khamanīyo ca madanīyo ca, evameva kho, ānanda, tāsaṃ tālapantīnaṃ vāteritānaṃ saddo ahosi vaggu ca rajanīyo ca khamanīyo ca madanīyo ca. Ye kho panānanda, tena samayena kusāvatiyā rājadhāniyā dhuttā ahesuṃ soṇḍā pipāsā, te tāsaṃ tālapantīnaṃ vāteritānaṃ saddena paricāresuṃ.

‘‘Niṭṭhite kho panānanda, dhamme pāsāde niṭṭhitāya dhammāya ca pokkharaṇiyā rājā mahāsudassano ‘ye\footnote{ye ko panānanda (syā. ka.)} tena samayena samaṇesu vā samaṇasammatā brāhmaṇesu vā brāhmaṇasammatā’, te sabbakāmehi santappetvā dhammaṃ pāsādaṃ abhiruhi.

\xsubsubsectionEnd{Paṭhamabhāṇavāro.}

\subsubsection{Jhānasampatti}

\paragraph{260.} ‘‘Atha kho, ānanda, rañño mahāsudassanassa etadahosi – ‘kissa nu kho me idaṃ kammassa phalaṃ kissa kammassa vipāko, yenāhaṃ etarahi evaṃmahiddhiko evaṃmahānubhāvo’ti? Atha kho, ānanda, rañño mahāsudassanassa etadahosi – ‘tiṇṇaṃ kho me idaṃ kammānaṃ phalaṃ tiṇṇaṃ kammānaṃ vipāko, yenāhaṃ etarahi evaṃmahiddhiko evaṃmahānubhāvo, seyyathidaṃ dānassa damassa saṃyamassā’ti.

‘‘Atha kho, ānanda, rājā mahāsudassano yena mahāviyūhaṃ kūṭāgāraṃ tenupasaṅkami; upasaṅkamitvā mahāviyūhassa kūṭāgārassa dvāre ṭhito udānaṃ udānesi – ‘tiṭṭha, kāmavitakka, tiṭṭha, byāpādavitakka, tiṭṭha, vihiṃsāvitakka. Ettāvatā kāmavitakka, ettāvatā byāpādavitakka, ettāvatā vihiṃsāvitakkā’ti.

\paragraph{261.} ‘‘Atha kho, ānanda, rājā mahāsudassano mahāviyūhaṃ kūṭāgāraṃ pavisitvā sovaṇṇamaye pallaṅke nisinno vivicceva kāmehi vivicca akusalehi dhammehi savitakkaṃ savicāraṃ vivekajaṃ pītisukhaṃ paṭhamaṃ jhānaṃ upasampajja vihāsi. Vitakkavicārānaṃ vūpasamā ajjhattaṃ sampasādanaṃ cetaso ekodibhāvaṃ avitakkaṃ avicāraṃ samādhijaṃ pītisukhaṃ dutiyaṃ jhānaṃ upasampajja vihāsi. Pītiyā ca virāgā upekkhako ca vihāsi, sato ca sampajāno sukhañca kāyena paṭisaṃvedesi, yaṃ taṃ ariyā ācikkhanti – ‘upekkhako satimā sukhavihārī’ti tatiyaṃ jhānaṃ upasampajja vihāsi. Sukhassa ca pahānā dukkhassa ca pahānā pubbeva somanassadomanassānaṃ atthaṅgamā adukkhamasukhaṃ upekkhāsatipārisuddhiṃ catutthaṃ jhānaṃ upasampajja vihāsi.

\paragraph{262.} ‘‘Atha kho, ānanda, rājā mahāsudassano mahāviyūhā kūṭāgārā nikkhamitvā sovaṇṇamayaṃ kūṭāgāraṃ pavisitvā rūpiyamaye pallaṅke nisinno mettāsahagatena cetasā ekaṃ disaṃ pharitvā vihāsi. Tathā dutiyaṃ tathā tatiyaṃ tathā catutthaṃ. Iti uddhamadho tiriyaṃ sabbadhi sabbattatāya sabbāvantaṃ lokaṃ mettāsahagatena cetasā vipulena mahaggatena appamāṇena averena abyāpajjena pharitvā vihāsi. Karuṇāsahagatena cetasā…pe… muditāsahagatena cetasā…pe… upekkhāsahagatena cetasā ekaṃ disaṃ pharitvā vihāsi tathā dutiyaṃ tathā tatiyaṃ tathā catutthaṃ. Iti uddhamadho tiriyaṃ sabbadhi sabbattatāya sabbāvantaṃ lokaṃ upekkhāsahagatena cetasā vipulena mahaggatena appamāṇena averena abyāpajjena pharitvā vihāsi.

\subsubsection{Caturāsīti nagarasahassādi}

\paragraph{263.} ‘‘Rañño, ānanda, mahāsudassanassa caturāsīti nagarasahassāni ahesuṃ kusāvatīrājadhānippamukhāni; caturāsīti pāsādasahassāni ahesuṃ dhammapāsādappamukhāni; caturāsīti kūṭāgārasahassāni ahesuṃ mahāviyūhakūṭāgārappamukhāni; caturāsīti pallaṅkasahassāni ahesuṃ sovaṇṇamayāni rūpiyamayāni dantamayāni sāramayāni gonakatthatāni paṭikatthatāni paṭalikatthatāni kadalimigapavarapaccattharaṇāni sauttaracchadāni ubhatolohitakūpadhānāni; caturāsīti nāgasahassāni ahesuṃ sovaṇṇālaṅkārāni sovaṇṇadhajāni hemajālapaṭicchannāni uposathanāgarājappamukhāni; caturāsīti assasahassāni ahesuṃ sovaṇṇālaṅkārāni sovaṇṇadhajāni hemajālapaṭicchannāni valāhakaassarājappamukhāni; caturāsīti rathasahassāni ahesuṃ sīhacammaparivārāni byagghacammaparivārāni dīpicammaparivārāni paṇḍukambalaparivārāni sovaṇṇālaṅkārāni sovaṇṇadhajāni hemajālapaṭicchannāni vejayantarathappamukhāni; caturāsīti maṇisahassāni ahesuṃ maṇiratanappamukhāni; caturāsīti itthisahassāni ahesuṃ subhaddādevippamukhāni; caturāsīti gahapatisahassāni ahesuṃ gahapatiratanappamukhāni; caturāsīti khattiyasahassāni ahesuṃ anuyantāni pariṇāyakaratanappamukhāni; caturāsīti dhenusahassāni ahesuṃ duhasandanāni\footnote{dukūlasandanāni(pī.)} dukūlasandānāni\footnote{dukūlasandanāni (pī.) dukūlasandānāni (saṃ. ni. 3.96)} kaṃsūpadhāraṇāni; caturāsīti vatthakoṭisahassāni ahesuṃ khomasukhumānaṃ kappāsikasukhumānaṃ koseyyasukhumānaṃ kambalasukhumānaṃ ; (rañño, ānanda, mahāsudassanassa)\footnote{( ) sī. ipotthakesu natthi} caturāsīti thālipākasahassāni ahesuṃ sāyaṃ pātaṃ bhattābhihāro abhihariyittha.

\paragraph{264.} ‘‘Tena kho panānanda, samayena rañño mahāsudassanassa caturāsīti nāgasahassāni sāyaṃ pātaṃ upaṭṭhānaṃ āgacchanti. Atha kho, ānanda, rañño mahāsudassanassa etadahosi – ‘imāni kho me caturāsīti nāgasahassāni sāyaṃ pātaṃ upaṭṭhānaṃ āgacchanti, yaṃnūna vassasatassa vassasatassa accayena dvecattālīsaṃ dvecattālīsaṃ nāgasahassāni sakiṃ sakiṃ upaṭṭhānaṃ āgaccheyyu’nti. Atha kho, ānanda, rājā mahāsudassano pariṇāyakaratanaṃ āmantesi – ‘imāni kho me, samma pariṇāyakaratana, caturāsīti nāgasahassāni sāyaṃ pātaṃ upaṭṭhānaṃ āgacchanti, tena hi, samma pariṇāyakaratana, vassasatassa vassasatassa accayena dvecattālīsaṃ dvecattālīsaṃ nāgasahassāni sakiṃ sakiṃ upaṭṭhānaṃ āgacchantū’ti. ‘Evaṃ, devā’ti kho, ānanda, pariṇāyakaratanaṃ rañño mahāsudassanassa paccassosi. Atha kho, ānanda, rañño mahāsudassanassa aparena samayena vassasatassa vassasatassa accayena dvecattālīsaṃ dvecattālīsaṃ nāgasahassāni sakiṃ sakiṃ upaṭṭhānaṃ āgamaṃsu.

\subsubsection{Subhaddādeviupasaṅkamanaṃ}

\paragraph{265.} ‘‘Atha kho, ānanda, subhaddāya deviyā bahunnaṃ vassānaṃ bahunnaṃ vassasatānaṃ bahunnaṃ vassasahassānaṃ accayena etadahosi – ‘ciraṃ diṭṭho kho me rājā mahāsudassano. Yaṃnūnāhaṃ rājānaṃ mahāsudassanaṃ dassanāya upasaṅkameyya’nti. Atha kho, ānanda, subhaddā devī itthāgāraṃ āmantesi – ‘etha tumhe sīsāni nhāyatha pītāni vatthāni pārupatha. Ciraṃ diṭṭho no rājā mahāsudassano, rājānaṃ mahāsudassanaṃ dassanāya upasaṅkamissāmā’ti. ‘Evaṃ, ayye’ti kho, ānanda, itthāgāraṃ subhaddāya deviyā paṭissutvā sīsāni nhāyitvā pītāni vatthāni pārupitvā yena subhaddā devī tenupasaṅkami. Atha kho, ānanda, subhaddā devī pariṇāyakaratanaṃ āmantesi – ‘kappehi, samma pariṇāyakaratana, caturaṅginiṃ senaṃ, ciraṃ diṭṭho no rājā mahāsudassano, rājānaṃ mahāsudassanaṃ dassanāya upasaṅkamissāmā’ti. ‘Evaṃ, devī’ti kho, ānanda, pariṇāyakaratanaṃ subhaddāya deviyā paṭissutvā caturaṅginiṃ senaṃ kappāpetvā subhaddāya deviyā paṭivedesi – ‘kappitā kho, devi, caturaṅginī senā, yassadāni kālaṃ maññasī’ti. Atha kho, ānanda, subhaddā devī caturaṅginiyā senāya saddhiṃ itthāgārena yena dhammo pāsādo tenupasaṅkami; upasaṅkamitvā dhammaṃ pāsādaṃ abhiruhitvā yena mahāviyūhaṃ kūṭāgāraṃ tenupasaṅkami. Upasaṅkamitvā mahāviyūhassa kūṭāgārassa dvārabāhaṃ ālambitvā aṭṭhāsi. Atha kho, ānanda, rājā mahāsudassano saddaṃ sutvā – ‘kiṃ nu kho mahato viya janakāyassa saddo’ti mahāviyūhā kūṭāgārā nikkhamanto addasa subhaddaṃ deviṃ dvārabāhaṃ ālambitvā ṭhitaṃ, disvāna subhaddaṃ deviṃ etadavoca – ‘ettheva, devi, tiṭṭha mā pāvisī’ti. Atha kho, ānanda, rājā mahāsudassano aññataraṃ purisaṃ āmantesi – ‘ehi tvaṃ, ambho purisa, mahāviyūhā kūṭāgārā sovaṇṇamayaṃ pallaṅkaṃ nīharitvā sabbasovaṇṇamaye tālavane paññapehī’ti. ‘Evaṃ, devā’ti kho, ānanda, so puriso rañño mahāsudassanassa paṭissutvā mahāviyūhā kūṭāgārā sovaṇṇamayaṃ pallaṅkaṃ nīharitvā sabbasovaṇṇamaye tālavane paññapesi. Atha kho, ānanda, rājā mahāsudassano dakkhiṇena passena sīhaseyyaṃ kappesi pāde pādaṃ accādhāya sato sampajāno.

\paragraph{266.} ‘‘Atha kho, ānanda, subhaddāya deviyā etadahosi – ‘vippasannāni kho rañño mahāsudassanassa indriyāni, parisuddho chavivaṇṇo pariyodāto, mā heva kho rājā mahāsudassano kālamakāsī’ti rājānaṃ mahāsudassanaṃ etadavoca –

‘Imāni te, deva, caturāsīti nagarasahassāni kusāvatīrājadhānippamukhāni. Ettha, deva, chandaṃ janehi jīvite apekkhaṃ karohi. Imāni te, deva, caturāsīti pāsādasahassāni dhammapāsādappamukhāni. Ettha, deva, chandaṃ janehi jīvite apekkhaṃ karohi. Imāni te, deva, caturāsīti kūṭāgārasahassāni mahāviyūhakūṭāgārappamukhāni. Ettha, deva, chandaṃ janehi jīvite apekkhaṃ karohi. Imāni te, deva, caturāsīti pallaṅkasahassāni sovaṇṇamayāni rūpiyamayāni dantamayāni sāramayāni gonakatthatāni paṭikatthatāni paṭalikatthatāni kadalimigapavarapaccattharaṇāni sauttaracchadāni ubhatolohitakūpadhānāni. Ettha, deva, chandaṃ janehi, jīvite apekkhaṃ karohi. Imāni te, deva, caturāsīti nāgasahassāni sovaṇṇālaṅkārāni sovaṇṇadhajāni hemajālapaṭicchannāni uposathanāgarājappamukhāni. Ettha, deva , chandaṃ janehi jīvite apekkhaṃ karohi. Imāni te, deva, caturāsīti assasahassāni sovaṇṇālaṅkārāni sovaṇṇadhajāni hemajālapaṭicchannāni valāhakaassarājappamukhāni. Ettha, deva, chandaṃ janehi jīvite apekkhaṃ karohi. Imāni te, deva caturāsīti rathasahassāni sīhacammaparivārāni byagghacammaparivārāni dīpicammaparivārāni paṇḍukambalaparivārāni sovaṇṇālaṅkārāni sovaṇṇadhajāni hemajālapaṭicchannāni vejayantarathappamukhāni. Ettha, deva, chandaṃ janehi jīvite apekkhaṃ karohi. Imāni te, deva, caturāsīti maṇisahassāni maṇiratanappamukhāni. Ettha, deva, chandaṃ janehi jīvite apekkhaṃ karohi. Imāni te, deva, caturāsīti itthisahassāni itthiratanappamukhāni. Ettha, deva, chandaṃ janehi jīvite apekkhaṃ karohi. Imāni te, deva, caturāsīti gahapatisahassāni gahapatiratanappamukhāni. Ettha, deva, chandaṃ janehi jīvite apekkhaṃ karohi. Imāni te, deva, caturāsīti khattiyasahassāni anuyantāni pariṇāyakaratanappamukhāni. Ettha, deva, chandaṃ janehi jīvite apekkhaṃ karohi. Imāni te, deva, caturāsīti dhenusahassāni duhasandanāni kaṃsūpadhāraṇāni. Ettha, deva, chandaṃ janehi jīvite apekkhaṃ karohi. Imāni te, deva, caturāsīti vatthakoṭisahassāni khomasukhumānaṃ kappāsikasukhumānaṃ koseyyasukhumānaṃ kambalasukhumānaṃ. Ettha, deva, chandaṃ janehi, jīvite apekkhaṃ karohi. Imāni te, deva, caturāsīti thālipākasahassāni sāyaṃ pātaṃ bhattābhihāro abhihariyati. Ettha, deva, chandaṃ janehi jīvite apekkhaṃ karohī’ti.

\paragraph{267.} ‘‘Evaṃ vutte, ānanda, rājā mahāsudassano subhaddaṃ deviṃ etadavoca –

‘Dīgharattaṃ kho maṃ tvaṃ, devi, iṭṭhehi kantehi piyehi manāpehi samudācarittha; atha ca pana maṃ tvaṃ pacchime kāle aniṭṭhehi akantehi appiyehi amanāpehi samudācarasī’ti. ‘Kathaṃ carahi taṃ, deva, samudācarāmī’ti? ‘Evaṃ kho maṃ tvaṃ, devi, samudācara – ‘‘sabbeheva, deva, piyehi manāpehi nānābhāvo vinābhāvo aññathābhāvo, mā kho tvaṃ, deva, sāpekkho kālamakāsi, dukkhā sāpekkhassa kālaṅkiriyā, garahitā ca sāpekkhassa kālaṅkiriyā. Imāni te, deva, caturāsīti nagarasahassāni kusāvatīrājadhānippamukhāni. Ettha, deva, chandaṃ pajaha jīvite apekkhaṃ mākāsi. Imāni te, deva, caturāsīti pāsādasahassāni dhammapāsādappamukhāni. Ettha, deva, chandaṃ pajaha jīvite apekkhaṃ mākāsi. Imāni te , deva, caturāsīti kūṭāgārasahassāni mahāviyūhakūṭāgārappamukhāni. Ettha, deva, chandaṃ pajaha jīvite apekkhaṃ mākāsi. Imāni te, deva, caturāsīti pallaṅkasahassāni sovaṇṇamayāni rūpiyamayāni dantamayāni sāramayāni gonakatthatāni paṭikatthatāni paṭalikatthatāni kadalimigapavarapaccattharaṇāni sauttaracchadāni ubhatolohitakūpadhānāni. Ettha, deva, chandaṃ pajaha jīvite apekkhaṃ mākāsi. Imāni te, deva, caturāsīti nāgasahassāni sovaṇṇālaṅkārāni sovaṇṇadhajāni hemajālapaṭicchannāni uposathanāgarājappamukhāni. Ettha, deva, chandaṃ pajaha jīvite apekkhaṃ mākāsi. Imāni te, deva, caturāsīti assasahassāni sovaṇṇālaṅkārāni sovaṇṇadhajāni hemajālapaṭicchannāni valāhakaassarājappamukhāni. Ettha, deva, chandaṃ pajaha jīvite apekkhaṃ mākāsi. Imāni te, deva, caturāsīti rathasahassāni sīhacammaparivārāni byagghacammaparivārāni dīpicammaparivārāni paṇḍukambalaparivārāni sovaṇṇālaṅkārāni sovaṇṇadhajāni hemajālapaṭicchannāni vejayantarathappamukhāni. Ettha, deva, chandaṃ pajaha jīvite apekkhaṃ mākāsi. Imāni te, deva, caturāsīti maṇisahassāni maṇiratanappamukhāni. Ettha, deva, chandaṃ pajaha jīvite apekkhaṃ mākāsi. Imāni te, deva, caturāsīti itthisahassāni subhaddādevippamukhāni. Ettha, deva, chandaṃ pajaha jīvite apekkhaṃ mākāsi. Imāni te, deva, caturāsīti gahapatisahassāni gahapatiratanappamukhāni. Ettha, deva, chandaṃ pajaha jīvite apekkhaṃ mākāsi. Imāni te, deva, caturāsīti khattiyasahassāni anuyantāni pariṇāyakaratanappamukhāni. Ettha, deva, chandaṃ pajaha jīvite apekkhaṃ mākāsi. Imāni te, deva, caturāsīti dhenusahassāni duhasandanāni kaṃsūpadhāraṇāni. Ettha deva, chandaṃ pajaha jīvite apekkhaṃ mākāsi. Imāni te, deva, caturāsīti vatthakoṭisahassāni khomasukhumānaṃ kappāsikasukhumānaṃ koseyyasukhumānaṃ kambalasukhumānaṃ. Ettha, deva, chandaṃ pajaha jīvite apekkhaṃ mākāsi. Imāni te deva caturāsīti thālipākasahassāni sāyaṃ pātaṃ bhattābhihāro abhihariyati. Ettha, deva, chandaṃ pajaha jīvite apekkhaṃ mākāsī’’’ti.

\paragraph{268.} ‘‘Evaṃ vutte, ānanda, subhaddā devī parodi assūni pavattesi. Atha kho, ānanda, subhaddā devī assūni puñchitvā\footnote{pamajjitvā (sī. syā. pī.), puñjitvā (ka.)} rājānaṃ mahāsudassanaṃ etadavoca –

‘Sabbeheva , deva, piyehi manāpehi nānābhāvo vinābhāvo aññathābhāvo, mā kho tvaṃ, deva, sāpekkho kālamakāsi, dukkhā sāpekkhassa kālaṅkiriyā, garahitā ca sāpekkhassa kālaṅkiriyā. Imāni te, deva, caturāsīti nagarasahassāni kusāvatīrājadhānippamukhāni. Ettha, deva, chandaṃ pajaha jīvite apekkhaṃ mākāsi. Imāni te, deva, caturāsīti pāsādasahassāni dhammapāsādappamukhāni. Ettha, deva, chandaṃ pajaha jīvite apekkhaṃ mākāsi. Imāni te, deva, caturāsīti kūṭāgārasahassāni mahāviyūhakūṭāgārappamukhāni. Ettha, deva, chandaṃ pajaha jīvite apekkhaṃ mākāsi. Imāni te, deva, caturāsīti pallaṅkasahassāni sovaṇṇamayāni rūpiyamayāni dantamayāni sāramayāni gonakatthatāni paṭikatthatāni paṭalikatthatāni kadalimigapavarapaccattharaṇāni sauttaracchadāni ubhatolohitakūpadhānāni. Ettha, deva, chandaṃ pajaha jīvite apekkhaṃ mākāsi. Imāni te, deva, caturāsīti nāgasahassāni sovaṇṇālaṅkārāni sovaṇṇadhajāni hemajālapaṭicchannāni uposathanāgarājappamukhāni. Ettha, deva, chandaṃ pajaha jīvite apekkhaṃ mākāsi. Imāni te , deva, caturāsīti assasahassāni sovaṇṇālaṅkārāni sovaṇṇadhajāni hemajālapaṭicchannāni valāhakaassarājappamukhāni . Ettha, deva, chandaṃ pajaha, jīvite apekkhaṃ mākāsi. Imāni te, deva, caturāsīti rathasahassāni sīhacammaparivārāni byagghacammaparivārāni dīpicammaparivārāni paṇḍukambalaparivārāni sovaṇṇālaṅkārāni sovaṇṇadhajāni hemajālapaṭicchannāni vejayantarathappamukhāni. Ettha, deva, chandaṃ pajaha jīvite apekkhaṃ mākāsi. Imāni te, deva, caturāsīti maṇisahassāni maṇiratanappamukhāni. Ettha, deva, chandaṃ pajaha jīvite apekkhaṃ mākāsi. Imāni te, deva, caturāsīti itthisahassāni itthiratanappamukhāni. Ettha, deva, chandaṃ pajaha, jīvite apekkhaṃ mākāsi. Imāni te , deva, caturāsīti gahapatisahassāni gahapatiratanappamukhāni. Ettha, deva, chandaṃ pajaha jīvite apekkhaṃ mākāsi. Imāni te, deva, caturāsīti khattiyasahassāni anuyantāni pariṇāyakaratanappamukhāni. Ettha, deva, chandaṃ pajaha jīvite apekkhaṃ mākāsi. Imāni te, deva, caturāsīti dhenusahassāni duhasandanāni kaṃsūpadhāraṇāni. Ettha, deva, chandaṃ pajaha jīvite apekkhaṃ mākāsi. Imāni te, deva, caturāsīti vatthakoṭisahassāni khomasukhumānaṃ kappāsikasukhumānaṃ koseyyasukhumānaṃ kambalasukhumānaṃ. Ettha, deva, chandaṃ pajaha jīvite apekkhaṃ mākāsi. Imāni te, deva, caturāsīti thālipākasahassāni sāyaṃ pātaṃ bhattābhihāro abhihariyati. Ettha , deva, chandaṃ pajaha jīvite apekkhaṃ mākāsī’ti.

\subsubsection{Brahmalokūpagamaṃ}

\paragraph{269.} ‘‘Atha kho, ānanda, rājā mahāsudassano nacirasseva kālamakāsi. Seyyathāpi, ānanda, gahapatissa vā gahapatiputtassa vā manuññaṃ bhojanaṃ bhuttāvissa bhattasammado hoti, evameva kho, ānanda, rañño mahāsudassanassa māraṇantikā vedanā ahosi. Kālaṅkato ca, ānanda, rājā mahāsudassano sugatiṃ brahmalokaṃ upapajji. Rājā, ānanda, mahāsudassano caturāsīti vassasahassāni kumārakīḷaṃ\footnote{kīḷitaṃ (ka.), kīḷikaṃ (sī. pī.)} kīḷi. Caturāsīti vassasahassāni oparajjaṃ kāresi. Caturāsīti vassasahassāni rajjaṃ kāresi. Caturāsīti vassasahassāni gihibhūto\footnote{gihībhūto (sī. pī.)} dhamme pāsāde brahmacariyaṃ cari\footnote{brahmacariyamacari (ka.)}. So cattāro brahmavihāre bhāvetvā kāyassa bhedā paraṃ maraṇā brahmalokūpago ahosi.

\paragraph{270.} ‘‘Siyā kho panānanda, evamassa – ‘añño nūna tena samayena rājā mahāsudassano ahosī’ti, na kho panetaṃ, ānanda, evaṃ daṭṭhabbaṃ. Ahaṃ tena samayena rājā mahāsudassano ahosiṃ. Mama tāni caturāsīti nagarasahassāni kusāvatīrājadhānippamukhāni, mama tāni caturāsīti pāsādasahassāni dhammapāsādappamukhāni, mama tāni caturāsīti kūṭāgārasahassāni mahāviyūhakūṭāgārappamukhāni, mama tāni caturāsīti pallaṅkasahassāni sovaṇṇamayāni rūpiyamayāni dantamayāni sāramayāni gonakatthatāni paṭikatthatāni paṭalikatthatāni kadalimigapavarapaccattharaṇāni sauttaracchadāni ubhatolohitakūpadhānāni, mama tāni caturāsīti nāgasahassāni sovaṇṇālaṅkārāni sovaṇṇadhajāni hemajālapaṭicchannāni uposathanāgarājappamukhāni, mama tāni caturāsīti assasahassāni sovaṇṇālaṅkārāni sovaṇṇadhajāni hemajālapaṭicchannāni valāhakaassarājappamukhāni, mama tāni caturāsīti rathasahassāni sīhacammaparivārāni byagghacammaparivārāni dīpicammaparivārāni paṇḍukambalaparivārāni sovaṇṇālaṅkārāni sovaṇṇadhajāni hemajālapaṭicchannāni vejayantarathappamukhāni, mama tāni caturāsīti maṇisahassāni maṇiratanappamukhāni, mama tāni caturāsīti itthisahassāni subhaddādevippamukhāni, mama tāni caturāsīti gahapatisahassāni gahapatiratanappamukhāni, mama tāni caturāsīti khattiyasahassāni anuyantāni pariṇāyakaratanappamukhāni, mama tāni caturāsīti dhenusahassāni duhasandanāni kaṃsūpadhāraṇāni, mama tāni caturāsīti vatthakoṭisahassāni khomasukhumānaṃ kappāsikasukhumānaṃ koseyyasukhumānaṃ kambalasukhumānaṃ, mama tāni caturāsīti thālipākasahassāni sāyaṃ pātaṃ bhattābhihāro abhihariyittha.

\paragraph{271.} ‘‘Tesaṃ kho panānanda, caturāsītinagarasahassānaṃ ekaññeva taṃ nagaraṃ hoti, yaṃ tena samayena ajjhāvasāmi yadidaṃ kusāvatī rājadhānī. Tesaṃ kho panānanda, caturāsītipāsādasahassānaṃ ekoyeva so pāsādo hoti, yaṃ tena samayena ajjhāvasāmi yadidaṃ dhammo pāsādo. Tesaṃ kho panānanda, caturāsītikūṭāgārasahassānaṃ ekaññeva taṃ kūṭāgāraṃ hoti, yaṃ tena samayena ajjhāvasāmi yadidaṃ mahāviyūhaṃ kūṭāgāraṃ. Tesaṃ kho panānanda, caturāsītipallaṅkasahassānaṃ ekoyeva so pallaṅko hoti, yaṃ tena samayena paribhuñjāmi yadidaṃ sovaṇṇamayo vā rūpiyamayo vā dantamayo vā sāramayo vā. Tesaṃ kho panānanda, caturāsītināgasahassānaṃ ekoyeva so nāgo hoti, yaṃ tena samayena abhiruhāmi yadidaṃ uposatho nāgarājā. Tesaṃ kho panānanda, caturāsītiassasahassānaṃ ekoyeva so asso hoti, yaṃ tena samayena abhiruhāmi yadidaṃ valāhako assarājā. Tesaṃ kho panānanda, caturāsītirathasahassānaṃ ekoyeva so ratho hoti, yaṃ tena samayena abhiruhāmi yadidaṃ vejayantaratho. Tesaṃ kho panānanda, caturāsītiitthisahassānaṃ ekāyeva sā itthī hoti, yā tena samayena paccupaṭṭhāti khattiyānī vā vessinī\footnote{vessāyinī (syā.), velāmikānī (ka. sī. pī.) velāmikā (saṃ. ni. 3.96)} vā. Tesaṃ kho panānanda, vā. Tesaṃ kho panānanda, caturāsītivatthakoṭisahassānaṃ ekaṃyeva taṃ dussayugaṃ hoti, yaṃ tena samayena paridahāmi khomasukhumaṃ vā kappāsikasukhumaṃ vā koseyyasukhumaṃ vā kambalasukhumaṃ vā. Tesaṃ kho panānanda, caturāsītithālipākasahassānaṃ ekoyeva so thālipāko hoti, yato nāḷikodanaparamaṃ bhuñjāmi tadupiyañca sūpeyyaṃ.

\paragraph{272.} ‘‘Passānanda, sabbete saṅkhārā atītā niruddhā vipariṇatā. Evaṃ aniccā kho, ānanda, saṅkhārā; evaṃ addhuvā kho, ānanda, saṅkhārā; evaṃ anassāsikā kho, ānanda, saṅkhārā! Yāvañcidaṃ, ānanda , alameva sabbasaṅkhāresu nibbindituṃ, alaṃ virajjituṃ, alaṃ vimuccituṃ.

‘‘Chakkhattuṃ kho panāhaṃ, ānanda, abhijānāmi imasmiṃ padese sarīraṃ nikkhipitaṃ, tañca kho rājāva samāno cakkavattī dhammiko dhammarājā cāturanto vijitāvī janapadatthāvariyapatto sattaratanasamannāgato, ayaṃ sattamo sarīranikkhepo. Na kho panāhaṃ, ānanda, taṃ padesaṃ samanupassāmi sadevake loke samārake sabrahmake sassamaṇabrāhmaṇiyā pajāya sadevamanussāya yattha tathāgato aṭṭhamaṃ sarīraṃ nikkhipeyyā’’ti. Idamavoca bhagavā, idaṃ vatvāna sugato athāparaṃ etadavoca satthā –

‘‘Aniccā vata saṅkhārā, uppādavayadhammino;

Uppajjitvā nirujjhanti, tesaṃ vūpasamo sukho’’ti.

\xsectionEnd{Mahāsudassanasuttaṃ niṭṭhitaṃ catutthaṃ.}


\clearpage
\section{Janavasabhasuttaṃ}

\subsubsection{Nātikiyādibyākaraṇaṃ}

\paragraph{273.} Evaṃ me sutaṃ – ekaṃ samayaṃ bhagavā nātike\footnote{nādike (sī. syā. pī.)} viharati giñjakāvasathe. Tena kho pana samayena bhagavā parito parito janapadesu paricārake abbhatīte kālaṅkate upapattīsu byākaroti kāsikosalesu vajjimallesu cetivaṃsesu\footnote{cetiyavaṃsesu (ka.)} kurupañcālesu majjhasūrasenesu\footnote{macchasurasenesu (syā.), macchasūrasenesu (sī. pī.)} – ‘‘asu amutra upapanno, asu amutra upapanno\footnote{upapannoti (ka.)}. Paropaññāsa nātikiyā paricārakā abbhatītā kālaṅkatā pañcannaṃ orambhāgiyānaṃ saṃyojanānaṃ parikkhayā opapātikā tattha parinibbāyino anāvattidhammā tasmā lokā. Sādhikā navuti nātikiyā paricārakā abbhatītā kālaṅkatā tiṇṇaṃ saṃyojanānaṃ parikkhayā rāgadosamohānaṃ tanuttā sakadāgāmino, sakideva\footnote{sakiṃdeva (ka.)} imaṃ lokaṃ āgantvā dukkhassantaṃ karissanti. Sātirekāni pañcasatāni nātikiyā paricārakā abbhatītā kālaṅkatā tiṇṇaṃ saṃyojanānaṃ parikkhayā sotāpannā avinipātadhammā niyatā sambodhiparāyaṇā’’ti.

\paragraph{274.} Assosuṃ kho nātikiyā paricārakā – ‘‘bhagavā kira parito parito janapadesu paricārake abbhatīte kālaṅkate upapattīsu byākaroti kāsikosalesu vajjimallesu cetivaṃsesu kurupañcālesu majjhasūrasenesu – ‘asu amutra upapanno, asu amutra upapanno. Paropaññāsa nātikiyā paricārakā abbhatītā kālaṅkatā pañcannaṃ orambhāgiyānaṃ saṃyojanānaṃ parikkhayā opapātikā tattha parinibbāyino anāvattidhammā tasmā lokā. Sādhikā navuti nātikiyā paricārakā abbhatītā kālaṅkatā tiṇṇaṃ saṃyojanānaṃ parikkhayā rāgadosamohānaṃ tanuttā sakadāgāmino sakideva imaṃ lokaṃ āgantvā dukkhassantaṃ karissanti. Sātirekāni pañcasatāni nātikiyā paricārakā abbhatītā kālaṅkatā tiṇṇaṃ saṃyojanānaṃ parikkhayā sotāpannā avinipātadhammā niyatā sambodhiparāyaṇā’ti. Tena ca nātikiyā paricārakā attamanā ahesuṃ pamuditā pītisomanassajātā bhagavato pañhaveyyākaraṇaṃ\footnote{pañhāveyyākaraṇaṃ (syā. ka.)} sutvā.

\paragraph{275.} Assosi kho āyasmā ānando – ‘‘bhagavā kira parito parito janapadesu paricārake abbhatīte kālaṅkate upapattīsu byākaroti kāsikosalesu vajjimallesu cetivaṃsesu kurupañcālesu majjhasūrasenesu – ‘asu amutra upapanno, asu amutra upapanno. Paropaññāsa nātikiyā paricārakā abbhatītā kālaṅkatā pañcannaṃ orambhāgiyānaṃ saṃyojanānaṃ parikkhayā opapātikā tattha parinibbāyino anāvattidhammā tasmā lokā. Sādhikā navuti nātikiyā paricārakā abbhatītā kālaṅkatā tiṇṇaṃ saṃyojanānaṃ parikkhayā rāgadosamohānaṃ tanuttā sakadāgāmino sakideva imaṃ lokaṃ āgantvā dukkhassantaṃ karissanti. Sātirekāni pañcasatāni nātikiyā paricārakā abbhatītā kālaṅkatā tiṇṇaṃ saṃyojanānaṃ parikkhayā sotāpannā avinipātadhammā niyatā sambodhiparāyaṇā’ti. Tena ca nātikiyā paricārakā attamanā ahesuṃ pamuditā pītisomanassajātā bhagavato pañhaveyyākaraṇaṃ sutvā’’ti.

\subsubsection{Ānandaparikathā}

\paragraph{276.} Atha kho āyasmato ānandassa etadahosi – ‘‘ime kho panāpi ahesuṃ māgadhakā paricārakā bahū ceva rattaññū ca abbhatītā kālaṅkatā. Suññā maññe aṅgamagadhā aṅgamāgadhakehi\footnote{aṅgamāgadhikehi (syā.)} paricārakehi abbhatītehi kālaṅkatehi. Te kho panāpi\footnote{tena kho panāpi (syā.)} ahesuṃ buddhe pasannā dhamme pasannā saṅghe pasannā sīlesu paripūrakārino. Te abbhatītā kālaṅkatā bhagavatā abyākatā; tesampissa sādhu veyyākaraṇaṃ, bahujano pasīdeyya, tato gaccheyya sugatiṃ. Ayaṃ kho panāpi ahosi rājā māgadho seniyo bimbisāro dhammiko dhammarājā hito brāhmaṇagahapatikānaṃ negamānañceva jānapadānañca. Apissudaṃ manussā kittayamānarūpā viharanti – ‘evaṃ no so dhammiko dhammarājā sukhāpetvā kālaṅkato, evaṃ mayaṃ tassa dhammikassa dhammarañño vijite phāsu\footnote{phāsukaṃ (syā.)} viharimhā’ti. So kho panāpi ahosi buddhe pasanno dhamme pasanno saṅghe pasanno sīlesu paripūrakārī. Apissudaṃ manussā evamāhaṃsu – ‘yāva maraṇakālāpi rājā māgadho seniyo bimbisāro bhagavantaṃ kittayamānarūpo kālaṅkato’ti. So abbhatīto kālaṅkato bhagavatā abyākato. Tassapissa sādhu veyyākaraṇaṃ bahujano pasīdeyya, tato gaccheyya sugatiṃ. Bhagavato kho pana sambodhi magadhesu. Yattha kho pana bhagavato sambodhi magadhesu, kathaṃ tatra bhagavā māgadhake paricārake abbhatīte kālaṅkate upapattīsu na byākareyya. Bhagavā ce kho pana māgadhake paricārake abbhatīte kālaṅkate upapattīsu na byākareyya, dīnamanā\footnote{ninnamanā (syā.), dīnamānā (sī. pī.)} tenassu māgadhakā paricārakā; yena kho panassu dīnamanā māgadhakā paricārakā kathaṃ te bhagavā na byākareyyā’’ti?

\paragraph{277.} Idamāyasmā ānando māgadhake paricārake ārabbha eko raho anuvicintetvā rattiyā paccūsasamayaṃ paccuṭṭhāya yena bhagavā tenupasaṅkami; upasaṅkamitvā bhagavantaṃ abhivādetvā ekamantaṃ nisīdi. Ekamantaṃ nisinno kho āyasmā ānando bhagavantaṃ etadavoca – ‘‘sutaṃ metaṃ, bhante – ‘bhagavā kira parito parito janapadesu paricārake abbhatīte kālaṅkate upapattīsu byākaroti kāsikosalesu vajjimallesu cetivaṃsesu kurupañcālesu majjhasūrasenesu – ‘‘asu amutra upapanno, asu amutra upapanno. Paropaññāsa nātikiyā paricārakā abbhatītā kālaṅkatā pañcannaṃ orambhāgiyānaṃ saṃyojanānaṃ parikkhayā opapātikā tattha parinibbāyino anāvattidhammā tasmā lokā. Sādhikā navuti nātikiyā paricārakā abbhatītā kālaṅkatā tiṇṇaṃ saṃyojanānaṃ parikkhayā rāgadosamohānaṃ tanuttā sakadāgāmino, sakideva imaṃ lokaṃ āgantvā dukkhassantaṃ karissanti. Sātirekāni pañcasatāni nātikiyā paricārakā abbhatītā kālaṅkatā tiṇṇaṃ saṃyojanānaṃ parikkhayā sotāpannā avinipātadhammā niyatā sambodhiparāyaṇāti. Tena ca nātikiyā paricārakā attamanā ahesuṃ pamuditā pītisomanassajātā bhagavato pañhaveyyākaraṇaṃ sutvā’’ti . Ime kho panāpi, bhante, ahesuṃ māgadhakā paricārakā bahū ceva rattaññū ca abbhatītā kālaṅkatā. Suññā maññe aṅgamagadhā aṅgamāgadhakehi paricārakehi abbhatītehi kālaṅkatehi. Te kho panāpi, bhante, ahesuṃ buddhe pasannā dhamme pasannā saṅghe pasannā sīlesu paripūrakārino, te abbhatītā kālaṅkatā bhagavatā abyākatā. Tesampissa sādhu veyyākaraṇaṃ, bahujano pasīdeyya, tato gaccheyya sugatiṃ. Ayaṃ kho panāpi, bhante, ahosi rājā māgadho seniyo bimbisāro dhammiko dhammarājā hito brāhmaṇagahapatikānaṃ negamānañceva jānapadānañca. Apissudaṃ manussā kittayamānarūpā viharanti – ‘evaṃ no so dhammiko dhammarājā sukhāpetvā kālaṅkato. Evaṃ mayaṃ tassa dhammikassa dhammarañño vijite phāsu viharimhā’ti. So kho panāpi, bhante, ahosi buddhe pasanno dhamme pasanno saṅghe pasanno sīlesu paripūrakārī. Apissudaṃ manussā evamāhaṃsu – ‘yāva maraṇakālāpi rājā māgadho seniyo bimbisāro bhagavantaṃ kittayamānarūpo kālaṅkato’ti. So abbhatīto kālaṅkato bhagavatā abyākato; tassapissa sādhu veyyākaraṇaṃ, bahujano pasīdeyya, tato gaccheyya sugatiṃ. Bhagavato kho pana, bhante, sambodhi magadhesu. Yattha kho pana , bhante, bhagavato sambodhi magadhesu, kathaṃ tatra bhagavā māgadhake paricārake abbhatīte kālaṅkate upapattīsu na byākareyya? Bhagavā ce kho pana, bhante, māgadhake paricārake abbhatīte kālaṅkate upapattīsu na byākareyya dīnamanā tenassu māgadhakā paricārakā; yena kho panassu dīnamanā māgadhakā paricārakā kathaṃ te bhagavā na byākareyyā’’ti. Idamāyasmā ānando māgadhake paricārake ārabbha bhagavato sammukhā parikathaṃ katvā uṭṭhāyāsanā bhagavantaṃ abhivādetvā padakkhiṇaṃ katvā pakkāmi.

\paragraph{278.} Atha kho bhagavā acirapakkante āyasmante ānande pubbaṇhasamayaṃ nivāsetvā pattacīvaramādāya nātikaṃ piṇḍāya pāvisi. Nātike piṇḍāya caritvā pacchābhattaṃ piṇḍapātapaṭikkanto pāde pakkhāletvā giñjakāvasathaṃ pavisitvā māgadhake paricārake ārabbha aṭṭhiṃ katvā\footnote{aṭṭhikatvā (sī. syā. pī.)} manasikatvā sabbaṃ cetasā\footnote{sabbacetasā (pī.)} samannāharitvā paññatte āsane nisīdi – ‘‘gatiṃ nesaṃ jānissāmi abhisamparāyaṃ, yaṃgatikā te bhavanto yaṃabhisamparāyā’’ti. Addasā kho bhagavā māgadhake paricārake ‘‘yaṃgatikā te bhavanto yaṃabhisamparāyā’’ti. Atha kho bhagavā sāyanhasamayaṃ paṭisallānā vuṭṭhito giñjakāvasathā nikkhamitvā vihārapacchāyāyaṃ paññatte āsane nisīdi.

\paragraph{279.} Atha kho āyasmā ānando yena bhagavā tenupasaṅkami; upasaṅkamitvā bhagavantaṃ abhivādetvā ekamantaṃ nisīdi. Ekamantaṃ nisinno kho āyasmā ānando bhagavantaṃ etadavoca – ‘‘upasantapadisso\footnote{upasantapatiso (ka.)} bhante bhagavā bhātiriva bhagavato mukhavaṇṇo vippasannattā indriyānaṃ. Santena nūnajja bhante bhagavā vihārena vihāsī’’ti? ‘‘Yadeva kho me tvaṃ, ānanda, māgadhake paricārake ārabbha sammukhā parikathaṃ katvā uṭṭhāyāsanā pakkanto, tadevāhaṃ nātike piṇḍāya caritvā pacchābhattaṃ piṇḍapātapaṭikkanto pāde pakkhāletvā giñjakāvasathaṃ pavisitvā māgadhake paricārake ārabbha aṭṭhiṃ katvā manasikatvā sabbaṃ cetasā samannāharitvā paññatte āsane nisīdiṃ – ‘gatiṃ nesaṃ jānissāmi abhisamparāyaṃ, yaṃgatikā te bhavanto yaṃabhisamparāyā’ti. Addasaṃ kho ahaṃ, ānanda, māgadhake paricārake ‘yaṃgatikā te bhavanto yaṃabhisamparāyā’’’ti.

\subsubsection{Janavasabhayakkho}

\paragraph{280.} ‘‘Atha kho, ānanda, antarahito yakkho saddamanussāvesi – ‘janavasabho ahaṃ bhagavā ; janavasabho ahaṃ sugatā’ti. Abhijānāsi no tvaṃ, ānanda, ito pubbe evarūpaṃ nāmadheyyaṃ sutaṃ\footnote{sutvā (pī.)} yadidaṃ janavasabho’’ti?

‘‘Na kho ahaṃ, bhante, abhijānāmi ito pubbe evarūpaṃ nāmadheyyaṃ sutaṃ yadidaṃ janavasabhoti, api ca me, bhante, lomāni haṭṭhāni ‘janavasabho’ti nāmadheyyaṃ sutvā. Tassa mayhaṃ, bhante, etadahosi – ‘na hi nūna so orako yakkho bhavissati yadidaṃ evarūpaṃ nāmadheyyaṃ supaññattaṃ yadidaṃ janavasabho’’ti. ‘‘Anantarā kho, ānanda, saddapātubhāvā uḷāravaṇṇo me yakkho sammukhe pāturahosi . Dutiyampi saddamanussāvesi – ‘bimbisāro ahaṃ bhagavā; bimbisāro ahaṃ sugatāti. Idaṃ sattamaṃ kho ahaṃ, bhante, vessavaṇassa mahārājassa sahabyataṃ upapajjāmi, so tato cuto manussarājā bhavituṃ pahomi\footnote{so tato cuto manussarājā, amanussarājā divi homi (sī. pī.)}.

Ito satta tato satta, saṃsārāni catuddasa;

Nivāsamabhijānāmi, yattha me vusitaṃ pure.

\paragraph{281.} ‘Dīgharattaṃ kho ahaṃ, bhante, avinipāto avinipātaṃ sañjānāmi, āsā ca pana me santiṭṭhati sakadāgāmitāyā’ti. ‘Acchariyamidaṃ āyasmato janavasabhassa yakkhassa, abbhutamidaṃ āyasmato janavasabhassa yakkhassa. ‘‘Dīgharattaṃ kho ahaṃ, bhante, avinipāto avinipātaṃ sañjānāmī’’ti ca vadesi, ‘‘āsā ca pana me santiṭṭhati sakadāgāmitāyā’’ti ca vadesi, kutonidānaṃ panāyasmā janavasabho yakkho evarūpaṃ uḷāraṃ visesādhigamaṃ sañjānātīti? Na aññatra, bhagavā, tava sāsanā, na aññatra\footnote{aññattha (sī. pī.)}, sugata, tava sāsanā; yadagge ahaṃ, bhante, bhagavati ekantikato\footnote{ekantato (syā.), ekantagato (pī.)} abhippasanno, tadagge ahaṃ, bhante, dīgharattaṃ avinipāto avinipātaṃ sañjānāmi, āsā ca pana me santiṭṭhati sakadāgāmitāya. Idhāhaṃ, bhante, vessavaṇena mahārājena pesito virūḷhakassa mahārājassa santike kenacideva karaṇīyena addasaṃ bhagavantaṃ antarāmagge giñjakāvasathaṃ pavisitvā māgadhake paricārake ārabbha aṭṭhiṃ katvā manasikatvā sabbaṃ cetasā samannāharitvā nisinnaṃ – ‘‘gatiṃ nesaṃ jānissāmi abhisamparāyaṃ, yaṃgatikā te bhavanto yaṃabhisamparāyā’’ti. Anacchariyaṃ kho panetaṃ, bhante, yaṃ vessavaṇassa mahārājassa tassaṃ parisāyaṃ bhāsato sammukhā sutaṃ sammukhā paṭiggahitaṃ – ‘‘yaṃgatikā te bhavanto yaṃabhisamparāyā’’ti. Tassa mayhaṃ, bhante, etadahosi – bhagavantañca dakkhāmi, idañca bhagavato ārocessāmīti. Ime kho me, bhante, dvepaccayā bhagavantaṃ dassanāya upasaṅkamituṃ’.

\subsubsection{Devasabhā}

\paragraph{282.} ‘Purimāni , bhante, divasāni purimatarāni tadahuposathe pannarase vassūpanāyikāya puṇṇāya puṇṇamāya rattiyā kevalakappā ca devā tāvatiṃsā sudhammāyaṃ sabhāyaṃ sannisinnā honti sannipatitā. Mahatī ca dibbaparisā\footnote{dibbā parisā (sī. pī.)} samantato nisinnā honti\footnote{nisinnā hoti (sī.), sannisinnā honti sannipatitā (ka.)}, cattāro ca mahārājāno catuddisā nisinnā honti. Puratthimāya disāya dhataraṭṭho mahārājā pacchimābhimukho\footnote{pacchābhimukho (ka.)} nisinno hoti deve purakkhatvā; dakkhiṇāya disāya virūḷhako mahārājā uttarābhimukho nisinno hoti deve purakkhatvā; pacchimāya disāya virūpakkho mahārājā puratthābhimukho nisinno hoti deve purakkhatvā; uttarāya disāya vessavaṇo mahārājā dakkhiṇābhimukho nisinno hoti deve purakkhatvā . Yadā, bhante, kevalakappā ca devā tāvatiṃsā sudhammāyaṃ sabhāyaṃ sannisinnā honti sannipatitā, mahatī ca dibbaparisā samantato nisinnā honti, cattāro ca mahārājāno catuddisā nisinnā honti. Idaṃ nesaṃ hoti āsanasmiṃ; atha pacchā amhākaṃ āsanaṃ hoti. Ye te, bhante, devā bhagavati brahmacariyaṃ caritvā adhunūpapannā tāvatiṃsakāyaṃ, te aññe deve atirocanti vaṇṇena ceva yasasā ca. Tena sudaṃ, bhante, devā tāvatiṃsā attamanā honti pamuditā pītisomanassajātā ‘‘dibbā vata bho kāyā paripūrenti, hāyanti asurakāyā’’ti. Atha kho, bhante, sakko devānamindo devānaṃ tāvatiṃsānaṃ sampasādaṃ viditvā imāhi gāthāhi anumodi –

‘‘Modanti vata bho devā, tāvatiṃsā sahindakā\footnote{saindakā (sī.)};

Tathāgataṃ namassantā, dhammassa ca sudhammataṃ.

Nave deve ca passantā, vaṇṇavante yasassine\footnote{yasassino (syā.)};

Sugatasmiṃ brahmacariyaṃ, caritvāna idhāgate.

Te aññe atirocanti, vaṇṇena yasasāyunā;

Sāvakā bhūripaññassa, visesūpagatā idha.

Idaṃ disvāna nandanti, tāvatiṃsā sahindakā;

Tathāgataṃ namassantā, dhammassa ca sudhammata’’nti.

‘Tena sudaṃ, bhante, devā tāvatiṃsā bhiyyosomattāya attamanā honti pamuditā pītisomanassajātā ‘‘dibbā vata, bho, kāyā paripūrenti, hāyanti asurakāyā’’ti. Atha kho, bhante, yenatthena devā tāvatiṃsā sudhammāyaṃ sabhāyaṃ sannisinnā honti sannipatitā, taṃ atthaṃ cintayitvā taṃ atthaṃ mantayitvā vuttavacanāpi taṃ\footnote{vuttavacanā nāmidaṃ (ka.)} cattāro mahārājāno tasmiṃ atthe honti. Paccānusiṭṭhavacanāpi taṃ\footnote{paccānusiṭṭhavacanā nāmidaṃ (ka.)} cattāro mahārājāno tasmiṃ atthe honti, sakesu sakesu āsanesu ṭhitā avipakkantā\footnote{adhipakkantā (ka.)}.

Te vuttavākyā rājāno, paṭiggayhānusāsaniṃ;

Vippasannamanā santā, aṭṭhaṃsu samhi āsaneti.

\paragraph{283.} ‘Atha kho, bhante, uttarāya disāya uḷāro āloko sañjāyi, obhāso pāturahosi atikkammeva devānaṃ devānubhāvaṃ. Atha kho, bhante, sakko devānamindo deve tāvatiṃse āmantesi – ‘‘yathā kho, mārisā, nimittāni dissanti, uḷāro āloko sañjāyati, obhāso pātubhavati, brahmā pātubhavissati. Brahmuno hetaṃ pubbanimittaṃ pātubhāvāya yadidaṃ āloko sañjāyati obhāso pātubhavatīti.

‘‘Yathā nimittā dissanti, brahmā pātubhavissati;

Brahmuno hetaṃ nimittaṃ, obhāso vipulo mahā’’ti.

\subsubsection{Sanaṅkumārakathā}

\paragraph{284.} ‘Atha kho, bhante, devā tāvatiṃsā yathāsakesu āsanesu nisīdiṃsu – ‘‘obhāsametaṃ ñassāma, yaṃvipāko bhavissati, sacchikatvāva naṃ gamissāmā’’ti. Cattāropi mahārājāno yathāsakesu āsanesu nisīdiṃsu – ‘‘obhāsametaṃ ñassāma yaṃvipāko bhavissati, sacchikatvāva naṃ gamissāmā’’ti. Idaṃ sutvā devā tāvatiṃsā ekaggā samāpajjiṃsu – ‘‘obhāsametaṃ ñassāma, yaṃvipāko bhavissati, sacchikatvāva naṃ gamissāmā’’ti.

‘Yadā, bhante, brahmā sanaṅkumāro devānaṃ tāvatiṃsānaṃ pātubhavati, oḷārikaṃ attabhāvaṃ abhinimminitvā pātubhavati. Yo kho pana, bhante, brahmuno pakativaṇṇo anabhisambhavanīyo so devānaṃ tāvatiṃsānaṃ cakkhupathasmiṃ. Yadā, bhante, brahmā sanaṅkumāro devānaṃ tāvatiṃsānaṃ pātubhavati , so aññe deve atirocati vaṇṇena ceva yasasā ca. Seyyathāpi, bhante, sovaṇṇo viggaho mānusaṃ viggahaṃ atirocati; evameva kho, bhante, yadā brahmā sanaṅkumāro devānaṃ tāvatiṃsānaṃ pātubhavati, so aññe deve atirocati vaṇṇena ceva yasasā ca. Yadā, bhante, brahmā sanaṅkumāro devānaṃ tāvatiṃsānaṃ pātubhavati, na tassaṃ parisāyaṃ koci devo abhivādeti vā paccuṭṭheti vā āsanena vā nimanteti. Sabbeva tuṇhībhūtā pañjalikā pallaṅkena nisīdanti – ‘‘yassadāni devassa pallaṅkaṃ icchissati brahmā sanaṅkumāro, tassa devassa pallaṅke nisīdissatī’’ti.

‘Yassa kho pana, bhante, devassa brahmā sanaṅkumāro pallaṅke nisīdati, uḷāraṃ so labhati devo vedapaṭilābhaṃ; uḷāraṃ so labhati devo somanassapaṭilābhaṃ. Seyyathāpi, bhante, rājā khattiyo muddhāvasitto adhunābhisitto rajjena, uḷāraṃ so labhati vedapaṭilābhaṃ, uḷāraṃ so labhati somanassapaṭilābhaṃ. Evameva kho, bhante, yassa devassa brahmā sanaṅkumāro pallaṅke nisīdati, uḷāraṃ so labhati devo vedapaṭilābhaṃ, uḷāraṃ so labhati devo somanassapaṭilābhaṃ. Atha , bhante, brahmā sanaṅkumāro oḷārikaṃ attabhāvaṃ abhinimminitvā kumāravaṇṇī\footnote{kumāravaṇṇo (syā. ka.)} hutvā pañcasikho devānaṃ tāvatiṃsānaṃ pāturahosi. So vehāsaṃ abbhuggantvā ākāse antalikkhe pallaṅkena nisīdi. Seyyathāpi, bhante, balavā puriso supaccatthate vā pallaṅke same vā bhūmibhāge pallaṅkena nisīdeyya; evameva kho, bhante, brahmā sanaṅkumāro vehāsaṃ abbhuggantvā ākāse antalikkhe pallaṅkena nisīditvā devānaṃ tāvatiṃsānaṃ sampasādaṃ viditvā imāhi gāthāhi anumodi –

‘‘Modanti vata bho devā, tāvatiṃsā sahindakā;

Tathāgataṃ namassantā, dhammassa ca sudhammataṃ.

‘‘Nave deve ca passantā, vaṇṇavante yasassine;

Sugatasmiṃ brahmacariyaṃ, caritvāna idhāgate.

‘‘Te aññe atirocanti, vaṇṇena yasasāyunā;

Sāvakā bhūripaññassa, visesūpagatā idha.

‘‘Idaṃ disvāna nandanti, tāvatiṃsā sahindakā;

Tathāgataṃ namassantā, dhammassa ca sudhammata’’nti.

\paragraph{285.} ‘Imamatthaṃ, bhante, brahmā sanaṅkumāro bhāsittha; imamatthaṃ, bhante, brahmuno sanaṅkumārassa bhāsato aṭṭhaṅgasamannāgato saro hoti vissaṭṭho ca viññeyyo ca mañju ca savanīyo ca bindu ca avisārī ca gambhīro ca ninnādī ca. Yathāparisaṃ kho pana, bhante, brahmā sanaṅkumāro sarena viññāpeti; na cassa bahiddhā parisāya ghoso niccharati. Yassa kho pana, bhante, evaṃ aṭṭhaṅgasamannāgato saro hoti, so vuccati ‘‘brahmassaro’’ti.

‘Atha kho, bhante, brahmā sanaṅkumāro tettiṃse attabhāve abhinimminitvā devānaṃ tāvatiṃsānaṃ paccekapallaṅkesu pallaṅkena\footnote{paccekapallaṅkena (ka.)} nisīditvā deve tāvatiṃse āmantesi – ‘‘taṃ kiṃ maññanti, bhonto devā tāvatiṃsā, yāvañca so bhagavā bahujanahitāya paṭipanno bahujanasukhāya lokānukampāya atthāya hitāya sukhāya devamanussānaṃ. Ye hi keci, bho, buddhaṃ saraṇaṃ gatā dhammaṃ saraṇaṃ gatā saṅghaṃ saraṇaṃ gatā sīlesu paripūrakārino te kāyassa bhedā paraṃ maraṇā appekacce paranimmitavasavattīnaṃ devānaṃ sahabyataṃ upapajjanti, appekacce nimmānaratīnaṃ devānaṃ sahabyataṃ upapajjanti, appekacce tusitānaṃ devānaṃ sahabyataṃ upapajjanti, appekacce yāmānaṃ devānaṃ sahabyataṃ upapajjanti, appekacce tāvatiṃsānaṃ devānaṃ sahabyataṃ upapajjanti, appekacce cātumahārājikānaṃ devānaṃ sahabyataṃ upapajjanti. Ye sabbanihīnaṃ kāyaṃ paripūrenti, te gandhabbakāyaṃ paripūrentī’’’ti.

\paragraph{286.} ‘Imamatthaṃ , bhante, brahmā sanaṅkumāro bhāsittha; imamatthaṃ, bhante, brahmuno sanaṅkumārassa bhāsato ghosoyeva devā maññanti – ‘‘yvāyaṃ mama pallaṅke svāyaṃ ekova bhāsatī’’ti.

Ekasmiṃ bhāsamānasmiṃ, sabbe bhāsanti nimmitā;

Ekasmiṃ tuṇhimāsīne, sabbe tuṇhī bhavanti te.

Tadāsu devā maññanti, tāvatiṃsā sahindakā;

Yvāyaṃ mama pallaṅkasmiṃ, svāyaṃ ekova bhāsatīti.

‘Atha kho, bhante, brahmā sanaṅkumāro ekattena attānaṃ upasaṃharati, ekattena attānaṃ upasaṃharitvā sakkassa devānamindassa pallaṅke pallaṅkena nisīditvā deve tāvatiṃse āmantesi –

\subsubsection{Bhāvitaiddhipādo}

\paragraph{287.} ‘‘‘Taṃ kiṃ maññanti, bhonto devā tāvatiṃsā, yāva supaññattā cime tena bhagavatā jānatā passatā arahatā sammāsambuddhena cattāro iddhipādā paññattā iddhipahutāya\footnote{iddhibahulīkatāya (syā.)} iddhivisavitāya\footnote{iddhivisevitāya (syā.)} iddhivikubbanatāya. Katame cattāro? Idha bho bhikkhu chandasamādhippadhānasaṅkhārasamannāgataṃ iddhipādaṃ bhāveti. Vīriyasamādhippadhānasaṅkhārasamannāgataṃ iddhipādaṃ bhāveti. Cittasamādhippadhānasaṅkhārasamannāgataṃ iddhipādaṃ bhāveti. Vīmaṃsāsamādhippadhānasaṅkhārasamannāgataṃ iddhipādaṃ bhāveti. Ime kho, bho, tena bhagavatā jānatā passatā arahatā sammāsambuddhena cattāro iddhipādā paññattā iddhipahutāya iddhivisavitāya iddhivikubbanatāya.

‘‘‘Ye hi keci bho atītamaddhānaṃ samaṇā vā brāhmaṇā vā anekavihitaṃ iddhividhaṃ paccanubhosuṃ, sabbe te imesaṃyeva catunnaṃ iddhipādānaṃ bhāvitattā bahulīkatattā. Yepi hi keci bho anāgatamaddhānaṃ samaṇā vā brāhmaṇā vā anekavihitaṃ iddhividhaṃ paccanubhossanti, sabbe te imesaṃyeva catunnaṃ iddhipādānaṃ bhāvitattā bahulīkatattā. Yepi hi keci bho etarahi samaṇā vā brāhmaṇā vā anekavihitaṃ iddhividhaṃ paccanubhonti, sabbe te imesaṃyeva catunnaṃ iddhipādānaṃ bhāvitattā bahulīkatattā. Passanti no bhonto devā tāvatiṃsā mamapimaṃ evarūpaṃ iddhānubhāva’’nti? ‘‘Evaṃ mahābrahme’’ti. ‘‘Ahampi kho bho imesaṃyeva catunnañca iddhipādānaṃ bhāvitattā bahulīkatattā evaṃ mahiddhiko evaṃmahānubhāvo’’ti. Imamatthaṃ, bhante, brahmā sanaṅkumāro bhāsittha. Imamatthaṃ, bhante, brahmā sanaṅkumāro bhāsitvā deve tāvatiṃse āmantesi –

\subsubsection{Tividho okāsādhigamo}

\paragraph{288.} ‘‘‘Taṃ kiṃ maññanti, bhonto devā tāvatiṃsā, yāvañcidaṃ tena bhagavatā jānatā passatā arahatā sammāsambuddhena tayo okāsādhigamā anubuddhā sukhassādhigamāya. Katame tayo? Idha bho ekacco saṃsaṭṭho viharati kāmehi saṃsaṭṭho akusalehi dhammehi. So aparena samayena ariyadhammaṃ suṇāti, yoniso manasi karoti, dhammānudhammaṃ paṭipajjati. So ariyadhammassavanaṃ āgamma yonisomanasikāraṃ dhammānudhammappaṭipattiṃ asaṃsaṭṭho viharati kāmehi asaṃsaṭṭho akusalehi dhammehi. Tassa asaṃsaṭṭhassa kāmehi asaṃsaṭṭhassa akusalehi dhammehi uppajjati sukhaṃ, sukhā bhiyyo somanassaṃ. Seyyathāpi, bho, pamudā pāmojjaṃ\footnote{pāmujjaṃ (pī. ka.)} jāyetha, evameva kho, bho, asaṃsaṭṭhassa kāmehi asaṃsaṭṭhassa akusalehi dhammehi uppajjati sukhaṃ, sukhā bhiyyo somanassaṃ. Ayaṃ kho, bho, tena bhagavatā jānatā passatā arahatā sammāsambuddhena paṭhamo okāsādhigamo anubuddho sukhassādhigamāya.

‘‘‘Puna caparaṃ, bho, idhekaccassa oḷārikā kāyasaṅkhārā appaṭippassaddhā honti, oḷārikā vacīsaṅkhārā appaṭippassaddhā honti, oḷārikā cittasaṅkhārā appaṭippassaddhā honti. So aparena samayena ariyadhammaṃ suṇāti, yoniso manasi karoti, dhammānudhammaṃ paṭipajjati. Tassa ariyadhammassavanaṃ āgamma yonisomanasikāraṃ dhammānudhammappaṭipattiṃ oḷārikā kāyasaṅkhārā paṭippassambhanti, oḷārikā vacīsaṅkhārā paṭippassambhanti, oḷārikā cittasaṅkhārā paṭippassambhanti. Tassa oḷārikānaṃ kāyasaṅkhārānaṃ paṭippassaddhiyā oḷārikānaṃ vacīsaṅkhārānaṃ paṭippassaddhiyā oḷārikānaṃ cittasaṅkhārānaṃ paṭippassaddhiyā uppajjati sukhaṃ, sukhā bhiyyo somanassaṃ. Seyyathāpi, bho, pamudā pāmojjaṃ jāyetha, evameva kho bho oḷārikānaṃ kāyasaṅkhārānaṃ paṭippassaddhiyā oḷārikānaṃ vacīsaṅkhārānaṃ paṭippassaddhiyā oḷārikānaṃ cittasaṅkhārānaṃ paṭippassaddhiyā uppajjati sukhaṃ, sukhā bhiyyo somanassaṃ. Ayaṃ kho, bho, tena bhagavatā jānatā passatā arahatā sammāsambuddhena dutiyo okāsādhigamo anubuddho sukhassādhigamāya.

‘‘‘Puna caparaṃ, bho, idhekacco ‘idaṃ kusala’nti yathābhūtaṃ nappajānāti, ‘idaṃ akusala’nti yathābhūtaṃ nappajānāti. ‘Idaṃ sāvajjaṃ idaṃ anavajjaṃ, idaṃ sevitabbaṃ idaṃ na sevitabbaṃ, idaṃ hīnaṃ idaṃ paṇītaṃ, idaṃ kaṇhasukkasappaṭibhāga’nti yathābhūtaṃ nappajānāti. So aparena samayena ariyadhammaṃ suṇāti, yoniso manasi karoti, dhammānudhammaṃ paṭipajjati. So ariyadhammassavanaṃ āgamma yonisomanasikāraṃ dhammānudhammappaṭipattiṃ, ‘idaṃ kusala’nti yathābhūtaṃ pajānāti, ‘idaṃ akusala’nti yathābhūtaṃ pajānāti. Idaṃ sāvajjaṃ idaṃ anavajjaṃ, idaṃ sevitabbaṃ idaṃ na sevitabbaṃ, idaṃ hīnaṃ idaṃ paṇītaṃ, idaṃ kaṇhasukkasappaṭibhāga’nti yathābhūtaṃ pajānāti. Tassa evaṃ jānato evaṃ passato avijjā pahīyati, vijjā uppajjati. Tassa avijjāvirāgā vijjuppādā uppajjati sukhaṃ, sukhā bhiyyo somanassaṃ. Seyyathāpi, bho, pamudā pāmojjaṃ jāyetha , evameva kho, bho, avijjāvirāgā vijjuppādā uppajjati sukhaṃ, sukhā bhiyyo somanassaṃ. Ayaṃ kho, bho, tena bhagavatā jānatā passatā arahatā sammāsambuddhena tatiyo okāsādhigamo anubuddho sukhassādhigamāya. Ime kho, bho, tena bhagavatā jānatā passatā arahatā sammāsambuddhena tayo okāsādhigamā anubuddhā sukhassādhigamāyā’’ti. Imamatthaṃ, bhante, brahmā sanaṅkumāro bhāsittha, imamatthaṃ, bhante, brahmā sanaṅkumāro bhāsitvā deve tāvatiṃse āmantesi –

\subsubsection{Catusatipaṭṭhānaṃ}

\paragraph{289.} ‘‘‘Taṃ kiṃ maññanti, bhonto devā tāvatiṃsā, yāva supaññattā cime tena bhagavatā jānatā passatā arahatā sammāsambuddhena cattāro satipaṭṭhānā paññattā kusalassādhigamāya. Katame cattāro? Idha , bho, bhikkhu ajjhattaṃ kāye kāyānupassī viharati ātāpī sampajāno satimā vineyya loke abhijjhādomanassaṃ. Ajjhattaṃ kāye kāyānupassī viharanto tattha sammā samādhiyati, sammā vippasīdati. So tattha sammā samāhito sammā vippasanno bahiddhā parakāye ñāṇadassanaṃ abhinibbatteti. Ajjhattaṃ vedanāsu vedanānupassī viharati…pe… bahiddhā paravedanāsu ñāṇadassanaṃ abhinibbatteti. Ajjhattaṃ citte cittānupassī viharati…pe… bahiddhā paracitte ñāṇadassanaṃ abhinibbatteti. Ajjhattaṃ dhammesu dhammānupassī viharati ātāpī sampajāno satimā vineyya loke abhijjhādomanassaṃ. Ajjhattaṃ dhammesu dhammānupassī viharanto tattha sammā samādhiyati, sammā vippasīdati. So tattha sammā samāhito sammā vippasanno bahiddhā paradhammesu ñāṇadassanaṃ abhinibbatteti. Ime kho, bho, tena bhagavatā jānatā passatā arahatā sammāsambuddhena cattāro satipaṭṭhānā paññattā kusalassādhigamāyā’’ti. Imamatthaṃ, bhante, brahmā sanaṅkumāro bhāsittha. Imamatthaṃ, bhante, brahmā sanaṅkumāro bhāsitvā deve tāvatiṃse āmantesi –

\subsubsection{Satta samādhiparikkhārā}

\paragraph{290.} ‘‘‘Taṃ kiṃ maññanti, bhonto devā tāvatiṃsā, yāva supaññattā cime tena bhagavatā jānatā passatā arahatā sammāsambuddhena satta samādhiparikkhārā sammāsamādhissa paribhāvanāya sammāsamādhissa pāripūriyā. Katame satta? Sammādiṭṭhi sammāsaṅkappo sammāvācā sammākammanto sammāājīvo sammāvāyāmo sammāsati. Yā kho, bho, imehi sattahaṅgehi cittassa ekaggatā parikkhatā, ayaṃ vuccati, bho, ariyo sammāsamādhi saupaniso itipi saparikkhāro itipi. Sammādiṭṭhissa bho, sammāsaṅkappo pahoti, sammāsaṅkappassa sammāvācā pahoti, sammāvācassa sammākammanto pahoti. Sammākammantassa sammāājīvo pahoti, sammāājīvassa sammāvāyāmo pahoti, sammāvāyāmassa sammāsati pahoti , sammāsatissa sammāsamādhi pahoti, sammāsamādhissa sammāñāṇaṃ pahoti, sammāñāṇassa sammāvimutti pahoti. Yañhi taṃ, bho, sammā vadamāno vadeyya – ‘svākkhāto bhagavatā dhammo sandiṭṭhiko akāliko ehipassiko opaneyyiko paccattaṃ veditabbo viññūhi apārutā amatassa dvārā’ti idameva taṃ sammā vadamāno vadeyya. Svākkhāto hi, bho, bhagavatā dhammo sandiṭṭhiko, akāliko ehipassiko opaneyyiko paccattaṃ veditabbo viññūhi apārutā amatassa dvārā\footnote{dvārāti (syā. ka.)}.

‘‘‘Ye hi keci, bho, buddhe aveccappasādena samannāgatā, dhamme aveccappasādena samannāgatā, saṅghe aveccappasādena samannāgatā, ariyakantehi sīlehi samannāgatā , ye cime opapātikā dhammavinītā sātirekāni catuvīsatisatasahassāni māgadhakā paricārakā abbhatītā kālaṅkatā tiṇṇaṃ saṃyojanānaṃ parikkhayā sotāpannā avinipātadhammā niyatā sambodhiparāyaṇā. Atthi cevettha sakadāgāmino.

‘‘Atthāyaṃ\footnote{athāyaṃ (sī. syā.)} itarā pajā, puññābhāgāti me mano;

Saṅkhātuṃ nopi sakkomi, musāvādassa ottappa’’nti.

\paragraph{291.} ‘Imamatthaṃ, bhante, brahmā sanaṅkumāro bhāsittha, imamatthaṃ, bhante, brahmuno sanaṅkumārassa bhāsato vessavaṇassa mahārājassa evaṃ cetaso parivitakko udapādi – ‘‘acchariyaṃ vata bho, abbhutaṃ vata bho, evarūpopi nāma uḷāro satthā bhavissati, evarūpaṃ uḷāraṃ dhammakkhānaṃ, evarūpā uḷārā visesādhigamā paññāyissantī’’ti. Atha, bhante, brahmā sanaṅkumāro vessavaṇassa mahārājassa cetasā cetoparivitakkamaññāya vessavaṇaṃ mahārājānaṃ etadavoca – ‘‘taṃ kiṃ maññati bhavaṃ vessavaṇo mahārājā atītampi addhānaṃ evarūpo uḷāro satthā ahosi, evarūpaṃ uḷāraṃ dhammakkhānaṃ, evarūpā uḷārā visesādhigamā paññāyiṃsu. Anāgatampi addhānaṃ evarūpo uḷāro satthā bhavissati, evarūpaṃ uḷāraṃ dhammakkhānaṃ, evarūpā uḷārā visesādhigamā paññāyissantī’’’ti.

\paragraph{292.} ‘‘‘Imamatthaṃ, bhante, brahmā sanaṅkumāro devānaṃ tāvatiṃsānaṃ abhāsi, imamatthaṃ vessavaṇo mahārājā brahmuno sanaṅkumārassa devānaṃ tāvatiṃsānaṃ bhāsato sammukhā sutaṃ\footnote{sutvā (sī. pī.)} sammukhā paṭiggahitaṃ sayaṃ parisāyaṃ ārocesi’’.

Imamatthaṃ janavasabho yakkho vessavaṇassa mahārājassa sayaṃ parisāyaṃ bhāsato sammukhā sutaṃ sammukhā paṭiggahitaṃ\footnote{paṭiggahetvā (sī. pī.)} bhagavato ārocesi. Imamatthaṃ bhagavā janavasabhassa yakkhassa sammukhā sutvā sammukhā paṭiggahetvā sāmañca abhiññāya āyasmato ānandassa ārocesi, imamatthamāyasmā ānando bhagavato sammukhā sutvā sammukhā paṭiggahetvā ārocesi bhikkhūnaṃ bhikkhunīnaṃ upāsakānaṃ upāsikānaṃ. Tayidaṃ brahmacariyaṃ iddhañceva phītañca vitthārikaṃ bāhujaññaṃ puthubhūtaṃ yāva devamanussehi suppakāsitanti.

\xsectionEnd{Janavasabhasuttaṃ niṭṭhitaṃ pañcamaṃ.}


\clearpage
\section{Mahāgovindasuttaṃ}

\paragraph{293.} Evaṃ me sutaṃ – ekaṃ samayaṃ bhagavā rājagahe viharati gijjhakūṭe pabbate. Atha kho pañcasikho gandhabbaputto abhikkantāya rattiyā abhikkantavaṇṇo kevalakappaṃ gijjhakūṭaṃ pabbataṃ obhāsetvā yena bhagavā tenupasaṅkami; upasaṅkamitvā bhagavantaṃ abhivādetvā ekamantaṃ aṭṭhāsi. Ekamantaṃ ṭhito kho pañcasikho gandhabbaputto bhagavantaṃ etadavoca – ‘‘yaṃ kho me, bhante, devānaṃ tāvatiṃsānaṃ sammukhā sutaṃ sammukhā paṭiggahitaṃ, ārocemi taṃ bhagavato’’ti. ‘‘Ārocehi me tvaṃ, pañcasikhā’’ti bhagavā avoca.

\subsubsection{Devasabhā}

\paragraph{294.} ‘‘Purimāni, bhante, divasāni purimatarāni tadahuposathe pannarase pavāraṇāya puṇṇāya puṇṇamāya rattiyā kevalakappā ca devā tāvatiṃsā sudhammāyaṃ sabhāyaṃ sannisinnā honti sannipatitā; mahatī ca dibbaparisā samantato nisinnā honti, cattāro ca mahārājāno catuddisā nisinnā honti; puratthimāya disāya dhataraṭṭho mahārājā pacchimābhimukho nisinno hoti deve purakkhatvā; dakkhiṇāya disāya virūḷhako mahārājā uttarābhimukho nisinno hoti deve purakkhatvā; pacchimāya disāya virūpakkho mahārājā puratthābhimukho nisinno hoti deve purakkhatvā; uttarāya disāya vessavaṇo mahārājā dakkhiṇābhimukho nisinno hoti deve purakkhatvā. Yadā bhante, kevalakappā ca devā tāvatiṃsā sudhammāyaṃ sabhāyaṃ sannisinnā honti sannipatitā, mahatī ca dibbaparisā samantato nisinnā honti, cattāro ca mahārājāno catuddisā nisinnā honti, idaṃ nesaṃ hoti āsanasmiṃ; atha pacchā amhākaṃ āsanaṃ hoti.

‘‘Ye te, bhante, devā bhagavati brahmacariyaṃ caritvā adhunūpapannā tāvatiṃsakāyaṃ, te aññe deve atirocanti vaṇṇena ceva yasasā ca. Tena sudaṃ, bhante, devā tāvatiṃsā attamanā honti pamuditā pītisomanassajātā; ‘dibbā vata, bho, kāyā paripūrenti, hāyanti asurakāyā’ti.

\paragraph{295.} ‘‘Atha kho, bhante, sakko devānamindo devānaṃ tāvatiṃsānaṃ sampasādaṃ viditvā imāhi gāthāhi anumodi –

‘Modanti vata bho devā, tāvatiṃsā sahindakā;

Tathāgataṃ namassantā, dhammassa ca sudhammataṃ.

Nave deve ca passantā, vaṇṇavante yasassine;

Sugatasmiṃ brahmacariyaṃ, caritvāna idhāgate.

Te aññe atirocanti, vaṇṇena yasasāyunā;

Sāvakā bhūripaññassa, visesūpagatā idha.

Idaṃ disvāna nandanti, tāvatiṃsā sahindakā;

Tathāgataṃ namassantā, dhammassa ca sudhammata’nti.

‘‘Tena sudaṃ , bhante, devā tāvatiṃsā bhiyyoso mattāya attamanā honti pamuditā pītisomanassajātā; ‘dibbā vata, bho, kāyā paripūrenti, hāyanti asurakāyā’’’ti.

\subsubsection{Aṭṭha yathābhuccavaṇṇā}

\paragraph{296.} ‘‘Atha kho, bhante, sakko devānamindo devānaṃ tāvatiṃsānaṃ sampasādaṃ viditvā deve tāvatiṃse āmantesi – ‘iccheyyātha no tumhe, mārisā, tassa bhagavato aṭṭha yathābhucce vaṇṇe sotu’nti? ‘Icchāma mayaṃ, mārisa, tassa bhagavato aṭṭha yathābhucce vaṇṇe sotu’nti. Atha kho, bhante, sakko devānamindo devānaṃ tāvatiṃsānaṃ bhagavato aṭṭha yathābhucce vaṇṇe payirudāhāsi – ‘taṃ kiṃ maññanti, bhonto devā tāvatiṃsā? Yāvañca so bhagavā bahujanahitāya paṭipanno bahujanasukhāya lokānukampāya atthāya hitāya sukhāya devamanussānaṃ. Evaṃ bahujanahitāya paṭipannaṃ bahujanasukhāya lokānukampāya atthāya hitāya sukhāya devamanussānaṃ imināpaṅgena samannāgataṃ satthāraṃ neva atītaṃse samanupassāma, na panetarahi, aññatra tena bhagavatā.

‘‘Svākkhāto kho pana tena bhagavatā dhammo sandiṭṭhiko akāliko ehipassiko opaneyyiko paccattaṃ veditabbo viññūhi. Evaṃ opaneyyikassa dhammassa desetāraṃ imināpaṅgena samannāgataṃ satthāraṃ neva atītaṃse samanupassāma, na panetarahi, aññatra tena bhagavatā.

‘‘Idaṃ kusalanti kho pana tena bhagavatā supaññattaṃ, idaṃ akusalanti supaññattaṃ. Idaṃ sāvajjaṃ idaṃ anavajjaṃ, idaṃ sevitabbaṃ idaṃ na sevitabbaṃ, idaṃ hīnaṃ idaṃ paṇītaṃ, idaṃ kaṇhasukkasappaṭibhāganti supaññattaṃ. Evaṃ kusalākusalasāvajjānavajjasevitabbāsevitabbahīna-paṇītakaṇhasukkasappaṭibhāgānaṃ dhammānaṃ paññapetāraṃ imināpaṅgena samannāgataṃ satthāraṃ neva atītaṃse samanupassāma, na panetarahi, aññatra tena bhagavatā.

‘‘Supaññattā kho pana tena bhagavatā sāvakānaṃ nibbānagāminī paṭipadā, saṃsandati nibbānañca paṭipadā ca. Seyyathāpi nāma gaṅgodakaṃ yamunodakena saṃsandati sameti, evameva supaññattā tena bhagavatā sāvakānaṃ nibbānagāminī paṭipadā, saṃsandati nibbānañca paṭipadā ca. Evaṃ nibbānagāminiyā paṭipadāya paññapetāraṃ imināpaṅgena samannāgataṃ satthāraṃ neva atītaṃse samanupassāma, na panetarahi, aññatra tena bhagavatā.

‘‘Abhinipphanno\footnote{abhinippanno (pī. ka.)} kho pana tassa bhagavato lābho abhinipphanno siloko, yāva maññe khattiyā sampiyāyamānarūpā viharanti, vigatamado kho pana so bhagavā āhāraṃ āhāreti. Evaṃ vigatamadaṃ āhāraṃ āharayamānaṃ imināpaṅgena samannāgataṃ satthāraṃ neva atītaṃse samanupassāma, na panetarahi, aññatra tena bhagavatā.

‘‘Laddhasahāyo kho pana so bhagavā sekhānañceva paṭipannānaṃ khīṇāsavānañca vusitavataṃ. Te bhagavā apanujja ekārāmataṃ anuyutto viharati. Evaṃ ekārāmataṃ anuyuttaṃ imināpaṅgena samannāgataṃ satthāraṃ neva atītaṃse samanupassāma, na panetarahi, aññatra tena bhagavatā.

‘‘Yathāvādī kho pana so bhagavā tathākārī, yathākārī tathāvādī, iti yathāvādī tathākārī, yathākārī tathāvādī. Evaṃ dhammānudhammappaṭipannaṃ imināpaṅgena samannāgataṃ satthāraṃ neva atītaṃse samanupassāma, na panetarahi, aññatra tena bhagavatā.

‘‘Tiṇṇavicikiccho kho pana so bhagavā vigatakathaṃkatho pariyositasaṅkappo ajjhāsayaṃ ādibrahmacariyaṃ. Evaṃ tiṇṇavicikicchaṃ vigatakathaṃkathaṃ pariyositasaṅkappaṃ ajjhāsayaṃ ādibrahmacariyaṃ imināpaṅgena samannāgataṃ satthāraṃ neva atītaṃse samanupassāma, na panetarahi, aññatra tena bhagavatā’ti.

\paragraph{297.} ‘‘Ime kho, bhante, sakko devānamindo devānaṃ tāvatiṃsānaṃ bhagavato aṭṭha yathābhucce vaṇṇe payirudāhāsi. Tena sudaṃ, bhante, devā tāvatiṃsā bhiyyoso mattāya attamanā honti pamuditā pītisomanassajātā bhagavato aṭṭha yathābhucce vaṇṇe sutvā. Tatra, bhante, ekacce devā evamāhaṃsu – ‘aho vata, mārisā, cattāro sammāsambuddhā loke uppajjeyyuṃ dhammañca deseyyuṃ yathariva bhagavā. Tadassa bahujanahitāya bahujanasukhāya lokānukampāya atthāya hitāya sukhāya devamanussāna’nti. Ekacce devā evamāhaṃsu – ‘tiṭṭhantu, mārisā, cattāro sammāsambuddhā, aho vata, mārisā, tayo sammāsambuddhā loke uppajjeyyuṃ dhammañca deseyyuṃ yathariva bhagavā. Tadassa bahujanahitāya bahujanasukhāya lokānukampāya atthāya hitāya sukhāya devamanussāna’nti. Ekacce devā evamāhaṃsu – ‘tiṭṭhantu, mārisā, tayo sammāsambuddhā, aho vata, mārisā, dve sammāsambuddhā loke uppajjeyyuṃ dhammañca deseyyuṃ yathariva bhagavā. Tadassa bahujanahitāya bahujanasukhāya lokānukampāya atthāya hitāya sukhāya devamanussāna’nti.

\paragraph{298.} ‘‘Evaṃ vutte , bhante, sakko devānamindo deve tāvatiṃse etadavoca – ‘aṭṭhānaṃ kho etaṃ, mārisā, anavakāso, yaṃ ekissā lokadhātuyā dve arahanto sammāsambuddhā apubbaṃ acarimaṃ uppajjeyyuṃ, netaṃ ṭhānaṃ vijjati. Aho vata, mārisā, so bhagavā appābādho appātaṅko ciraṃ dīghamaddhānaṃ tiṭṭheyya. Tadassa bahujanahitāya bahujanasukhāya lokānukampāya atthāya hitāya sukhāya devamanussāna’nti. Atha kho, bhante, yenatthena devā tāvatiṃsā sudhammāyaṃ sabhāyaṃ sannisinnā honti sannipatitā, taṃ atthaṃ cintayitvā taṃ atthaṃ mantayitvā vuttavacanāpi taṃ cattāro mahārājāno tasmiṃ atthe honti. Paccānusiṭṭhavacanāpi taṃ cattāro mahārājāno tasmiṃ atthe honti, sakesu sakesu āsanesu ṭhitā avipakkantā.

Te vuttavākyā rājāno, paṭiggayhānusāsaniṃ;

Vippasannamanā santā, aṭṭhaṃsu samhi āsaneti.

\paragraph{299.} ‘‘Atha kho, bhante, uttarāya disāya uḷāro āloko sañjāyi, obhāso pāturahosi atikkammeva devānaṃ devānubhāvaṃ. Atha kho, bhante, sakko devānamindo deve tāvatiṃse āmantesi – ‘yathā kho, mārisā, nimittāni dissanti, uḷāro āloko sañjāyati, obhāso pātubhavati , brahmā pātubhavissati; brahmuno hetaṃ pubbanimittaṃ pātubhāvāya, yadidaṃ āloko sañjāyati obhāso pātubhavatīti.

‘Yathā nimittā dissanti, brahmā pātubhavissati;

Brahmuno hetaṃ nimittaṃ, obhāso vipulo mahā’ti.

\subsubsection{Sanaṅkumārakathā}

\paragraph{300.} ‘‘Atha kho, bhante, devā tāvatiṃsā yathāsakesu āsanesu nisīdiṃsu – ‘obhāsametaṃ ñassāma, yaṃvipāko bhavissati, sacchikatvāva naṃ gamissāmā’ti. Cattāropi mahārājāno yathāsakesu āsanesu nisīdiṃsu – ‘obhāsametaṃ ñassāma, yaṃvipāko bhavissati, sacchikatvāva naṃ gamissāmā’ti. Idaṃ sutvā devā tāvatiṃsā ekaggā samāpajjiṃsu – ‘obhāsametaṃ ñassāma, yaṃvipāko bhavissati, sacchikatvāva naṃ gamissāmā’ti.

‘‘Yadā, bhante, brahmā sanaṅkumāro devānaṃ tāvatiṃsānaṃ pātubhavati, oḷārikaṃ attabhāvaṃ abhinimminitvā pātubhavati. Yo kho pana, bhante, brahmuno pakativaṇṇo, anabhisambhavanīyo so devānaṃ tāvatiṃsānaṃ cakkhupathasmiṃ. Yadā, bhante, brahmā sanaṅkumāro devānaṃ tāvatiṃsānaṃ pātubhavati, so aññe deve atirocati vaṇṇena ceva yasasā ca. Seyyathāpi, bhante, sovaṇṇo viggaho mānusaṃ viggahaṃ atirocati, evameva kho, bhante, yadā brahmā sanaṅkumāro devānaṃ tāvatiṃsānaṃ pātubhavati, so aññe deve atirocati vaṇṇena ceva yasasā ca. Yadā, bhante, brahmā sanaṅkumāro devānaṃ tāvatiṃsānaṃ pātubhavati, na tassaṃ parisāyaṃ koci devo abhivādeti vā paccuṭṭheti vā āsanena vā nimanteti. Sabbeva tuṇhībhūtā pañjalikā pallaṅkena nisīdanti – ‘yassadāni devassa pallaṅkaṃ icchissati brahmā sanaṅkumāro, tassa devassa pallaṅke nisīdissatī’ti. Yassa kho pana, bhante, devassa brahmā sanaṅkumāro pallaṅke nisīdati, uḷāraṃ so labhati devo vedapaṭilābhaṃ, uḷāraṃ so labhati devo somanassapaṭilābhaṃ . Seyyathāpi, bhante, rājā khattiyo muddhāvasitto adhunābhisitto rajjena, uḷāraṃ so labhati vedapaṭilābhaṃ, uḷāraṃ so labhati somanassapaṭilābhaṃ, evameva kho, bhante, yassa devassa brahmā sanaṅkumāro pallaṅke nisīdati, uḷāraṃ so labhati devo vedapaṭilābhaṃ, uḷāraṃ so labhati devo somanassapaṭilābhaṃ. Atha, bhante, brahmā sanaṅkumāro devānaṃ tāvatiṃsānaṃ sampasādaṃ viditvā antarahito imāhi gāthāhi anumodi –

‘Modanti vata bho devā, tāvatiṃsā sahindakā;

Tathāgataṃ namassantā, dhammassa ca sudhammataṃ.

‘Nave deve ca passantā, vaṇṇavante yasassine;

Sugatasmiṃ brahmacariyaṃ, caritvāna idhāgate.

‘Te aññe atirocanti, vaṇṇena yasasāyunā;

Sāvakā bhūripaññassa, visesūpagatā idha.

‘Idaṃ disvāna nandanti, tāvatiṃsā sahindakā;

Tathāgataṃ namassantā, dhammassa ca sudhammata’nti.

\paragraph{301.} ‘‘Imamatthaṃ, bhante, brahmā sanaṅkumāro abhāsittha. Imamatthaṃ, bhante , brahmuno sanaṅkumārassa bhāsato aṭṭhaṅgasamannāgato saro hoti vissaṭṭho ca viññeyyo ca mañju ca savanīyo ca bindu ca avisārī ca gambhīro ca ninnādī ca. Yathāparisaṃ kho pana, bhante, brahmā sanaṅkumāro sarena viññāpeti, na cassa bahiddhā parisāya ghoso niccharati. Yassa kho pana, bhante, evaṃ aṭṭhaṅgasamannāgato saro hoti, so vuccati ‘brahmassaro’ti. Atha kho, bhante, devā tāvatiṃsā brahmānaṃ sanaṅkumāraṃ etadavocuṃ – ‘sādhu, mahābrahme, etadeva mayaṃ saṅkhāya modāma; atthi ca sakkena devānamindena tassa bhagavato aṭṭha yathābhuccā vaṇṇā bhāsitā; te ca mayaṃ saṅkhāya modāmā’ti.

\subsubsection{Aṭṭha yathābhuccavaṇṇā}

\paragraph{302.} ‘‘Atha , bhante, brahmā sanaṅkumāro sakkaṃ devānamindaṃ etadavoca – ‘sādhu, devānaminda, mayampi tassa bhagavato aṭṭha yathābhucce vaṇṇe suṇeyyāmā’ti. ‘Evaṃ mahābrahme’ti kho, bhante, sakko devānamindo brahmuno sanaṅkumārassa bhagavato aṭṭha yathābhucce vaṇṇe payirudāhāsi.

‘‘Taṃ kiṃ maññati, bhavaṃ mahābrahmā? Yāvañca so bhagavā bahujanahitāya paṭipanno bahujanasukhāya lokānukampāya atthāya hitāya sukhāya devamanussānaṃ. Evaṃ bahujanahitāya paṭipannaṃ bahujanasukhāya lokānukampāya atthāya hitāya sukhāya devamanussānaṃ imināpaṅgena samannāgataṃ satthāraṃ neva atītaṃse samanupassāma, na panetarahi, aññatra tena bhagavatā.

‘‘Svākkhāto kho pana tena bhagavatā dhammo sandiṭṭhiko akāliko ehipassiko opaneyyiko paccattaṃ veditabbo viññūhi. Evaṃ opaneyyikassa dhammassa desetāraṃ imināpaṅgena samannāgataṃ satthāraṃ neva atītaṃse samanupassāma, na panetarahi, aññatra tena bhagavatā.

‘‘Idaṃ kusala’nti kho pana tena bhagavatā supaññattaṃ, ‘idaṃ akusala’nti supaññattaṃ, ‘idaṃ sāvajjaṃ idaṃ anavajjaṃ, idaṃ sevitabbaṃ idaṃ na sevitabbaṃ, idaṃ hīnaṃ idaṃ paṇītaṃ, idaṃ kaṇhasukkasappaṭibhāga’nti supaññattaṃ. Evaṃ kusalākusalasāvajjānavajjasevitabbāsevitabbahīnapaṇītakaṇhasukkasappaṭibhāgānaṃ dhammānaṃ paññāpetāraṃ. Imināpaṅgena samannāgataṃ satthāraṃ neva atītaṃse samanupassāma, na panetarahi, aññatra tena bhagavatā.

‘‘Supaññattā kho pana tena bhagavatā sāvakānaṃ nibbānagāminī paṭipadā saṃsandati nibbānañca paṭipadā ca. Seyyathāpi nāma gaṅgodakaṃ yamunodakena saṃsandati sameti, evameva supaññattā tena bhagavatā sāvakānaṃ nibbānagāminī paṭipadā saṃsandati nibbānañca paṭipadā ca. Evaṃ nibbānagāminiyā paṭipadāya paññāpetāraṃ imināpaṅgena samannāgataṃ satthāraṃ neva atītaṃse samanupassāma, na panetarahi, aññatra tena bhagavatā.

‘‘Abhinipphanno kho pana tassa bhagavato lābho abhinipphanno siloko, yāva maññe khattiyā sampiyāyamānarūpā viharanti. Vigatamado kho pana so bhagavā āhāraṃ āhāreti. Evaṃ vigatamadaṃ āhāraṃ āharayamānaṃ imināpaṅgena samannāgataṃ satthāraṃ neva atītaṃse samanupassāma, na panetarahi, aññatra tena bhagavatā.

‘‘Laddhasahāyo kho pana so bhagavā sekhānañceva paṭipannānaṃ khīṇāsavānañca vusitavataṃ, te bhagavā apanujja ekārāmataṃ anuyutto viharati. Evaṃ ekārāmataṃ anuyuttaṃ imināpaṅgena samannāgataṃ satthāraṃ neva atītaṃse samanupassāma, na panetarahi, aññatra tena bhagavatā.

‘‘Yathāvādī kho pana so bhagavā tathākārī, yathākārī tathāvādī; iti yathāvādī tathākārī, yathākārī tathāvādī. Evaṃ dhammānudhammappaṭippannaṃ imināpaṅgena samannāgataṃ satthāraṃ neva atītaṃse samanupassāma, na panetarahi, aññatra tena bhagavatā.

‘‘Tiṇṇavicikiccho kho pana so bhagavā vigatakathaṃkatho pariyositasaṅkappo ajjhāsayaṃ ādibrahmacariyaṃ . Evaṃ tiṇṇavicikicchaṃ vigatakathaṃkathaṃ pariyositasaṅkappaṃ ajjhāsayaṃ ādibrahmacariyaṃ. Imināpaṅgena samannāgataṃ satthāraṃ neva atītaṃse samanupassāma, na panetarahi, aññatra tena bhagavatā’ti.

\paragraph{303.} ‘‘Ime kho, bhante, sakko devānamindo brahmuno sanaṅkumārassa bhagavato aṭṭha yathābhucce vaṇṇe payirudāhāsi. Tena sudaṃ, bhante, brahmā sanaṅkumāro attamano hoti pamudito pītisomanassajāto bhagavato aṭṭha yathābhucce vaṇṇe sutvā. Atha, bhante, brahmā sanaṅkumāro oḷārikaṃ attabhāvaṃ abhinimminitvā kumāravaṇṇī hutvā pañcasikho devānaṃ tāvatiṃsānaṃ pāturahosi . So vehāsaṃ abbhuggantvā ākāse antalikkhe pallaṅkena nisīdi. Seyyathāpi, bhante, balavā puriso supaccatthate vā pallaṅke same vā bhūmibhāge pallaṅkena nisīdeyya, evameva kho, bhante, brahmā sanaṅkumāro vehāsaṃ abbhuggantvā ākāse antalikkhe pallaṅkena nisīditvā deve tāvatiṃse āmantesi –

\subsubsection{Govindabrāhmaṇavatthu}

\paragraph{304.} ‘‘Taṃ kiṃ maññanti, bhonto devā tāvatiṃsā, yāva dīgharattaṃ mahāpaññova so bhagavā ahosi. Bhūtapubbaṃ, bho, rājā disampati nāma ahosi. Disampatissa rañño govindo nāma brāhmaṇo purohito ahosi. Disampatissa rañño reṇu nāma kumāro putto ahosi. Govindassa brāhmaṇassa jotipālo nāma māṇavo putto ahosi. Iti reṇu ca rājaputto jotipālo ca māṇavo aññe ca cha khattiyā iccete aṭṭha sahāyā ahesuṃ. Atha kho, bho, ahorattānaṃ accayena govindo brāhmaṇo kālamakāsi. Govinde brāhmaṇe kālaṅkate rājā disampati paridevesi – ‘‘yasmiṃ vata, bho, mayaṃ samaye govinde brāhmaṇe sabbakiccāni sammā vossajjitvā pañcahi kāmaguṇehi samappitā samaṅgībhūtā paricārema, tasmiṃ no samaye govindo brāhmaṇo kālaṅkato’’ti. Evaṃ vutte bho reṇu rājaputto rājānaṃ disampatiṃ etadavoca – ‘‘mā kho tvaṃ, deva, govinde brāhmaṇe kālaṅkate atibāḷhaṃ paridevesi. Atthi, deva, govindassa brāhmaṇassa jotipālo nāma māṇavo putto paṇḍitataro ceva pitarā, alamatthadasataro ceva pitarā; yepissa pitā atthe anusāsi, tepi jotipālasseva māṇavassa anusāsaniyā’’ti. ‘‘Evaṃ kumārā’’ti? ‘‘Evaṃ devā’’ti.

\subsubsection{Mahāgovindavatthu}

\paragraph{305.} ‘‘Atha kho, bho, rājā disampati aññataraṃ purisaṃ āmantesi – ‘‘ehi tvaṃ, ambho purisa, yena jotipālo nāma māṇavo tenupasaṅkama; upasaṅkamitvā jotipālaṃ māṇavaṃ evaṃ vadehi – ‘bhavamatthu bhavantaṃ jotipālaṃ, rājā disampati bhavantaṃ jotipālaṃ māṇavaṃ āmantayati, rājā disampati bhoto jotipālassa māṇavassa dassanakāmo’’’ti. ‘‘Evaṃ, devā’’ti kho, bho, so puriso disampatissa rañño paṭissutvā yena jotipālo māṇavo tenupasaṅkami; upasaṅkamitvā jotipālaṃ māṇavaṃ etadavoca – ‘‘bhavamatthu bhavantaṃ jotipālaṃ, rājā disampati bhavantaṃ jotipālaṃ māṇavaṃ āmantayati , rājā disampati bhoto jotipālassa māṇavassa dassanakāmo’’ti. ‘‘Evaṃ, bho’’ti kho bho jotipālo māṇavo tassa purisassa paṭissutvā yena rājā disampati tenupasaṅkami; upasaṅkamitvā disampatinā raññā saddhiṃ sammodi; sammodanīyaṃ kathaṃ sāraṇīyaṃ vītisāretvā ekamantaṃ nisīdi. Ekamantaṃ nisinnaṃ kho, bho, jotipālaṃ māṇavaṃ rājā disampati etadavoca – ‘‘anusāsatu no bhavaṃ jotipālo, mā no bhavaṃ jotipālo anusāsaniyā paccabyāhāsi. Pettike taṃ ṭhāne ṭhapessāmi, govindiye abhisiñcissāmī’’ti. ‘‘Evaṃ, bho’’ti kho, bho, so jotipālo māṇavo disampatissa rañño paccassosi. Atha kho, bho, rājā disampati jotipālaṃ māṇavaṃ govindiye abhisiñci, taṃ pettike ṭhāne ṭhapesi. Abhisitto jotipālo māṇavo govindiye pettike ṭhāne ṭhapito yepissa pitā atthe anusāsi tepi atthe anusāsati, yepissa pitā atthe nānusāsi, tepi atthe anusāsati; yepissa pitā kammante abhisambhosi, tepi kammante abhisambhoti, yepissa pitā kammante nābhisambhosi, tepi kammante abhisambhoti. Tamenaṃ manussā evamāhaṃsu – ‘‘govindo vata, bho, brāhmaṇo, mahāgovindo vata, bho, brāhmaṇo’’ti. Iminā kho evaṃ, bho, pariyāyena jotipālassa māṇavassa govindo mahāgovindotveva samaññā udapādi.

\subsubsection{Rajjasaṃvibhajanaṃ}

\paragraph{306.} ‘‘Atha kho, bho, mahāgovindo brāhmaṇo yena te cha khattiyā tenupasaṅkami; upasaṅkamitvā te cha khattiye etadavoca – ‘‘disampati kho, bho, rājā jiṇṇo vuddho mahallako addhagato vayoanuppatto, ko nu kho pana, bho, jānāti jīvitaṃ? Ṭhānaṃ kho panetaṃ vijjati, yaṃ disampatimhi raññe kālaṅkate rājakattāro reṇuṃ rājaputtaṃ rajje abhisiñceyyuṃ. Āyantu, bhonto, yena reṇu rājaputto tenupasaṅkamatha; upasaṅkamitvā reṇuṃ rājaputtaṃ evaṃ vadetha – ‘‘mayaṃ kho bhoto reṇussa sahāyā piyā manāpā appaṭikūlā, yaṃsukho bhavaṃ taṃsukhā mayaṃ, yaṃdukkho bhavaṃ taṃdukkhā mayaṃ. Disampati kho, bho, rājā jiṇṇo vuddho mahallako addhagato vayoanuppatto, ko nu kho pana, bho, jānāti jīvitaṃ? Ṭhānaṃ kho panetaṃ vijjati, yaṃ disampatimhi raññe kālaṅkate rājakattāro bhavantaṃ reṇuṃ rajje abhisiñceyyuṃ. Sace bhavaṃ reṇu rajjaṃ labhetha, saṃvibhajetha no rajjenā’’ti. ‘‘Evaṃ bho’’ti kho, bho, te cha khattiyā mahāgovindassa brāhmaṇassa paṭissutvā yena reṇu rājaputto tenupasaṅkamiṃsu; upasaṅkamitvā reṇuṃ rājaputtaṃ etadavocuṃ – ‘‘mayaṃ kho bhoto reṇussa sahāyā piyā manāpā appaṭikūlā ; yaṃsukho bhavaṃ taṃsukhā mayaṃ, yaṃdukkho bhavaṃ taṃdukkhā mayaṃ. Disampati kho, bho, rājā jiṇṇo vuddho mahallako addhagato vayoanuppatto, ko nu kho pana bho jānāti jīvitaṃ? Ṭhānaṃ kho panetaṃ vijjati, yaṃ disampatimhi raññe kālaṅkate rājakattāro bhavantaṃ reṇuṃ rajje abhisiñceyyuṃ. Sace bhavaṃ reṇu rajjaṃ labhetha, saṃvibhajetha no rajjenā’’ti. ‘‘Ko nu kho, bho, añño mama vijite sukho bhavetha\footnote{sukhā bhaveyyātha (ka.), sukhaṃ bhaveyyātha, sukhamedheyyātha (sī. pī.),sukha medhetha (?)}, aññatra bhavantebhi? Sacāhaṃ, bho, rajjaṃ labhissāmi, saṃvibhajissāmi vo rajjenā’’’ti.

\paragraph{307.} ‘‘Atha kho, bho, ahorattānaṃ accayena rājā disampati kālamakāsi. Disampatimhi raññe kālaṅkate rājakattāro reṇuṃ rājaputtaṃ rajje abhisiñciṃsu. Abhisitto reṇu rajjena pañcahi kāmaguṇehi samappito samaṅgībhūto paricāreti. Atha kho, bho, mahāgovindo brāhmaṇo yena te cha khattiyā tenupasaṅkami; upasaṅkamitvā te cha khattiye etadavoca – ‘‘disampati kho, bho, rājā kālaṅkato. Abhisitto reṇu rajjena pañcahi kāmaguṇehi samappito samaṅgībhūto paricāreti. Ko nu kho pana, bho, jānāti, madanīyā kāmā? Āyantu, bhonto, yena reṇu rājā tenupasaṅkamatha; upasaṅkamitvā reṇuṃ rājānaṃ evaṃ vadetha – disampati kho, bho, rājā kālaṅkato, abhisitto bhavaṃ reṇu rajjena, sarati bhavaṃ taṃ vacana’’’nti?

\paragraph{308.} ‘‘‘Evaṃ , bho’’ti kho, bho, te cha khattiyā mahāgovindassa brāhmaṇassa paṭissutvā yena reṇu rājā tenupasaṅkamiṃsu; upasaṅkamitvā reṇuṃ rājānaṃ etadavocuṃ – ‘‘disampati kho, bho, rājā kālaṅkato, abhisitto bhavaṃ reṇu rajjena, sarati bhavaṃ taṃ vacana’’nti? ‘‘Sarāmahaṃ, bho, taṃ vacanaṃ\footnote{vacananti (syā. ka.)}. Ko nu kho, bho, pahoti imaṃ mahāpathaviṃ uttarena āyataṃ dakkhiṇena sakaṭamukhaṃ sattadhā samaṃ suvibhattaṃ vibhajitu’’nti? ‘‘Ko nu kho, bho, añño pahoti, aññatra mahāgovindena brāhmaṇenā’’ti? Atha kho, bho, reṇu rājā aññataraṃ purisaṃ āmantesi – ‘‘ehi tvaṃ, ambho purisa, yena mahāgovindo brāhmaṇo tenupasaṅkama; upasaṅkamitvā mahāgovindaṃ brāhmaṇaṃ evaṃ vadehi – ‘rājā taṃ, bhante, reṇu āmantetī’’’ti. ‘‘Evaṃ devā’’ti kho, bho, so puriso reṇussa rañño paṭissutvā yena mahāgovindo brāhmaṇo tenupasaṅkami; upasaṅkamitvā mahāgovindaṃ brāhmaṇaṃ etadavoca – ‘‘rājā taṃ, bhante, reṇu āmantetī’’ti. ‘‘Evaṃ, bho’’ti kho, bho, mahāgovindo brāhmaṇo tassa purisassa paṭissutvā yena reṇu rājā tenupasaṅkami; upasaṅkamitvā reṇunā raññā saddhiṃ sammodi. Sammodanīyaṃ kathaṃ sāraṇīyaṃ vītisāretvā ekamantaṃ nisīdi. Ekamantaṃ nisinnaṃ kho, bho, mahāgovindaṃ brāhmaṇaṃ reṇu rājā etadavoca – ‘‘etu, bhavaṃ govindo, imaṃ mahāpathaviṃ uttarena āyataṃ dakkhiṇena sakaṭamukhaṃ sattadhā samaṃ suvibhattaṃ vibhajatū’’ti. ‘‘Evaṃ, bho’’ti kho mahāgovindo brāhmaṇo reṇussa rañño paṭissutvā imaṃ mahāpathaviṃ uttarena āyataṃ dakkhiṇena sakaṭamukhaṃ sattadhā samaṃ suvibhattaṃ vibhaji. Sabbāni sakaṭamukhāni paṭṭhapesi\footnote{aṭṭhapesi (sī. pī.)}. Tatra sudaṃ majjhe reṇussa rañño janapado hoti.

\paragraph{309.} Dantapuraṃ kaliṅgānaṃ\footnote{kāliṅgānaṃ (syā. pī. ka.)}, assakānañca potanaṃ.

Mahesayaṃ\footnote{māhissati (sī. syā. pī.)} avantīnaṃ, sovīrānañca rorukaṃ.

Mithilā ca videhānaṃ, campā aṅgesu māpitā;

Bārāṇasī ca kāsīnaṃ, ete govindamāpitāti.

\paragraph{310.} ‘‘Atha kho, bho, te cha khattiyā yathāsakena lābhena attamanā ahesuṃ paripuṇṇasaṅkappā – ‘‘yaṃ vata no ahosi icchitaṃ, yaṃ ākaṅkhitaṃ, yaṃ adhippetaṃ, yaṃ abhipatthitaṃ, taṃ no laddha’’nti.

‘‘Sattabhū brahmadatto ca, vessabhū bharato saha;

Reṇu dve dhataraṭṭhā ca, tadāsuṃ satta bhāradhā’ti.

\xsubsubsectionEnd{Paṭhamabhāṇavāro niṭṭhito.}

\subsubsection{Kittisaddaabbhuggamanaṃ}

\paragraph{311.} ‘‘Atha kho, bho, te cha khattiyā yena mahāgovindo brāhmaṇo tenupasaṅkamiṃsu; upasaṅkamitvā mahāgovindaṃ brāhmaṇaṃ etadavocuṃ – ‘‘yathā kho bhavaṃ govindo reṇussa rañño sahāyo piyo manāpo appaṭikūlo. Evameva kho bhavaṃ govindo amhākampi sahāyo piyo manāpo appaṭikūlo, anusāsatu no bhavaṃ govindo; mā no bhavaṃ govindo anusāsaniyā paccabyāhāsī’’ti. ‘‘Evaṃ, bho’’ti kho mahāgovindo brāhmaṇo tesaṃ channaṃ khattiyānaṃ paccassosi. Atha kho, bho, mahāgovindo brāhmaṇo satta ca rājāno khattiye muddhāvasitte rajje\footnote{muddhābhisitte rajjena (syā.)} anusāsi, satta ca brāhmaṇamahāsāle satta ca nhātakasatāni mante vācesi.

\paragraph{312.} ‘‘Atha kho, bho, mahāgovindassa brāhmaṇassa aparena samayena evaṃ kalyāṇo kittisaddo abbhuggacchi\footnote{abbhuggañchi (sī. pī.)} – ‘‘sakkhi mahāgovindo brāhmaṇo brahmānaṃ passati, sakkhi mahāgovindo brāhmaṇo brahmunā sākaccheti sallapati mantetī’’ti. Atha kho, bho, mahāgovindassa brāhmaṇassa etadahosi – ‘‘mayhaṃ kho evaṃ kalyāṇo kittisaddo abbhuggato – ‘sakkhi mahāgovindo brāhmaṇo brahmānaṃ passati, sakkhi mahāgovindo brāhmaṇo brahmunā sākaccheti sallapati mantetī’ti. Na kho panāhaṃ brahmānaṃ passāmi, na brahmunā sākacchemi, na brahmunā sallapāmi , na brahmunā mantemi. Sutaṃ kho pana metaṃ brāhmaṇānaṃ vuddhānaṃ mahallakānaṃ ācariyapācariyānaṃ bhāsamānānaṃ – ‘yo vassike cattāro māse paṭisallīyati, karuṇaṃ jhānaṃ jhāyati, so brahmānaṃ passati brahmunā sākaccheti brahmunā sallapati brahmunā mantetī’ti. Yaṃnūnāhaṃ vassike cattāro māse paṭisallīyeyyaṃ, karuṇaṃ jhānaṃ jhāyeyya’’nti.

\paragraph{313.} ‘‘Atha kho, bho, mahāgovindo brāhmaṇo yena reṇu rājā tenupasaṅkami; upasaṅkamitvā reṇuṃ rājānaṃ etadavoca – ‘‘mayhaṃ kho, bho, evaṃ kalyāṇo kittisaddo abbhuggato – ‘sakkhi mahāgovindo brāhmaṇo brahmānaṃ passati, sakkhi mahāgovindo brāhmaṇo brahmunā sākaccheti sallapati mantetī’ti. Na kho panāhaṃ, bho, brahmānaṃ passāmi, na brahmunā sākacchemi, na brahmunā sallapāmi, na brahmunā mantemi. Sutaṃ kho pana metaṃ brāhmaṇānaṃ vuddhānaṃ mahallakānaṃ ācariyapācariyānaṃ bhāsamānānaṃ – ‘yo vassike cattāro māse paṭisallīyati, karuṇaṃ jhānaṃ jhāyati, so brahmānaṃ passati, brahmunā sākaccheti brahmunā sallapati brahmunā mantetī’ti. Icchāmahaṃ, bho, vassike cattāro māse paṭisallīyituṃ, karuṇaṃ jhānaṃ jhāyituṃ; namhi kenaci upasaṅkamitabbo aññatra ekena bhattābhihārenā’’ti. ‘‘Yassadāni bhavaṃ govindo kālaṃ maññatī’’ti.

\paragraph{314.} ‘‘Atha kho, bho, mahāgovindo brāhmaṇo yena te cha khattiyā tenupasaṅkami; upasaṅkamitvā te cha khattiye etadavoca – ‘‘mayhaṃ kho, bho, evaṃ kalyāṇo kittisaddo abbhuggato – ‘sakkhi mahāgovindo brāhmaṇo brahmānaṃ passati, sakkhi mahāgovindo brāhmaṇo brahmunā sākaccheti sallapati mantetī’ti. Na kho panāhaṃ, bho, brahmānaṃ passāmi, na brahmunā sākacchemi, na brahmunā sallapāmi, na brahmunā mantemi. Sutaṃ kho pana metaṃ brāhmaṇānaṃ vuddhānaṃ mahallakānaṃ ācariyapācariyānaṃ bhāsamānānaṃ, ‘yo vassike cattāro māse paṭisallīyati, karuṇaṃ jhānaṃ jhāyati, so brahmānaṃ passati brahmunā sākaccheti brahmunā sallapati brahmunā mantetī’ti. Icchāmahaṃ, bho, vassike cattāro māse paṭisallīyituṃ, karuṇaṃ jhānaṃ jhāyituṃ; namhi kenaci upasaṅkamitabbo aññatra ekena bhattābhihārenā’’ti. ‘‘Yassadāni bhavaṃ govindo kālaṃ maññatī’’’ti.

\paragraph{315.} ‘‘Atha kho, bho, mahāgovindo brāhmaṇo yena te satta ca brāhmaṇamahāsālā satta ca nhātakasatāni tenupasaṅkami; upasaṅkamitvā te satta ca brāhmaṇamahāsāle satta ca nhātakasatāni etadavoca – ‘‘mayhaṃ kho, bho, evaṃ kalyāṇo kittisaddo abbhuggato – ‘sakkhi mahāgovindo brāhmaṇo brahmānaṃ passati, sakkhi mahāgovindo brāhmaṇo brahmunā sākaccheti sallapati mantetī’ti. Na kho panāhaṃ, bho, brahmānaṃ passāmi, na brahmunā sākacchemi, na brahmunā sallapāmi, na brahmunā mantemi. Sutaṃ kho pana metaṃ brāhmaṇānaṃ vuddhānaṃ mahallakānaṃ ācariyapācariyānaṃ bhāsamānānaṃ – ‘yo vassike cattāro māse paṭisallīyati, karuṇaṃ jhānaṃ jhāyati, so brahmānaṃ passati, brahmunā sākaccheti, brahmunā sallapati, brahmunā mantetī’ti. Tena hi, bho, yathāsute yathāpariyatte mante vitthārena sajjhāyaṃ karotha, aññamaññañca mante vācetha; icchāmahaṃ, bho, vassike cattāro māse paṭisallīyituṃ, karuṇaṃ jhānaṃ jhāyituṃ; namhi kenaci upasaṅkamitabbo aññatra ekena bhattābhihārenā’’ti. ‘‘Yassa dāni bhavaṃ govindo kālaṃ maññatī’’ti.

\paragraph{316.} ‘‘Atha kho, bho, mahāgovindo brāhmaṇo yena cattārīsā bhariyā sādisiyo tenupasaṅkami; upasaṅkamitvā cattārīsā bhariyā sādisiyo etadavoca – ‘‘mayhaṃ kho, bhotī, evaṃ kalyāṇo kittisaddo abbhuggato – ‘sakkhi mahāgovindo brāhmaṇo brahmānaṃ passati, sakkhi mahāgovindo brāhmaṇo brahmunā sākaccheti sallapati mantetī’ti. Na kho panāhaṃ, bhotī, brahmānaṃ passāmi, na brahmunā sākacchemi, na brahmunā sallapāmi, na brahmunā mantemi. Sutaṃ kho pana metaṃ brāhmaṇānaṃ vuddhānaṃ mahallakānaṃ ācariyapācariyānaṃ bhāsamānānaṃ ‘yo vassike cattāro māse paṭisallīyati, karuṇaṃ jhānaṃ jhāyati, so brahmānaṃ passati, brahmunā sākaccheti, brahmunā sallapati, brahmunā mantetīti, icchāmahaṃ, bhotī, vassike cattāro māse paṭisallīyituṃ, karuṇaṃ jhānaṃ jhāyituṃ; namhi kenaci upasaṅkamitabbo aññatra ekena bhattābhihārenā’’ti. ‘‘Yassa dāni bhavaṃ govindo kālaṃ maññatī’’’ti.

\paragraph{317.} ‘‘Atha kho, bho, mahāgovindo brāhmaṇo puratthimena nagarassa navaṃ sandhāgāraṃ kārāpetvā vassike cattāro māse paṭisallīyi, karuṇaṃ jhānaṃ jhāyi; nāssudha koci upasaṅkamati\footnote{upasaṅkami (pī.)} aññatra ekena bhattābhihārena. Atha kho, bho, mahāgovindassa brāhmaṇassa catunnaṃ māsānaṃ accayena ahudeva ukkaṇṭhanā ahu paritassanā – ‘‘sutaṃ kho pana metaṃ brāhmaṇānaṃ vuddhānaṃ mahallakānaṃ ācariyapācariyānaṃ bhāsamānānaṃ – ‘yo vassike cattāro māse paṭisallīyati, karuṇaṃ jhānaṃ jhāyati, so brahmānaṃ passati, brahmunā sākaccheti brahmunā sallapati brahmunā mantetī’ti. Na kho panāhaṃ brahmānaṃ passāmi, na brahmunā sākacchemi na brahmunā sallapāmi na brahmunā mantemī’’’ti.

\subsubsection{Brahmunā sākacchā}

\paragraph{318.} ‘‘Atha kho, bho, brahmā sanaṅkumāro mahāgovindassa brāhmaṇassa cetasā cetoparivitakkamaññāya seyyathāpi nāma balavā puriso samiñjitaṃ vā bāhaṃ pasāreyya, pasāritaṃ vā bāhaṃ samiñjeyya, evameva, brahmaloke antarahito mahāgovindassa brāhmaṇassa sammukhe pāturahosi. Atha kho, bho, mahāgovindassa brāhmaṇassa ahudeva bhayaṃ ahu chambhitattaṃ ahu lomahaṃso yathā taṃ adiṭṭhapubbaṃ rūpaṃ disvā. Atha kho, bho, mahāgovindo brāhmaṇo bhīto saṃviggo lomahaṭṭhajāto brahmānaṃ sanaṅkumāraṃ gāthāya ajjhabhāsi –

‘‘‘Vaṇṇavā yasavā sirimā, ko nu tvamasi mārisa;

Ajānantā taṃ pucchāma, kathaṃ jānemu taṃ maya’’nti.

‘‘Maṃ ve kumāraṃ jānanti, brahmaloke sanantanaṃ\footnote{sanantica (ka.)};

Sabbe jānanti maṃ devā, evaṃ govinda jānahi’’.

‘‘‘Āsanaṃ udakaṃ pajjaṃ, madhusākañca\footnote{madhupākañca (sī. syā. pī.)} brahmuno;

Agghe bhavantaṃ pucchāma, agghaṃ kurutu no bhavaṃ’’.

‘‘Paṭiggaṇhāma te agghaṃ, yaṃ tvaṃ govinda bhāsasi;

Diṭṭhadhammahitatthāya, samparāya sukhāya ca;

Katāvakāso pucchassu, yaṃ kiñci abhipatthita’’nti.

\paragraph{319.} ‘‘Atha kho, bho, mahāgovindassa brāhmaṇassa etadahosi – ‘‘katāvakāso khomhi brahmunā sanaṅkumārena. Kiṃ nu kho ahaṃ brahmānaṃ sanaṅkumāraṃ puccheyyaṃ diṭṭhadhammikaṃ vā atthaṃ samparāyikaṃ vā’ti? Atha kho, bho, mahāgovindassa brāhmaṇassa etadahosi – ‘kusalo kho ahaṃ diṭṭhadhammikānaṃ atthānaṃ, aññepi maṃ diṭṭhadhammikaṃ atthaṃ pucchanti. Yaṃnūnāhaṃ brahmānaṃ sanaṅkumāraṃ samparāyikaññeva atthaṃ puccheyya’nti. Atha kho, bho, mahāgovindo brāhmaṇo brahmānaṃ sanaṅkumāraṃ gāthāya ajjhabhāsi –

‘‘Pucchāmi brahmānaṃ sanaṅkumāraṃ,

Kaṅkhī akaṅkhiṃ paravediyesu;

Katthaṭṭhito kimhi ca sikkhamāno,

Pappoti macco amataṃ brahmaloka’’nti.

‘‘Hitvā mamattaṃ manujesu brahme,

Ekodibhūto karuṇedhimutto\footnote{karuṇādhimutto (sī. syā. pī.)};

Nirāmagandho virato methunasmā,

Etthaṭṭhito ettha ca sikkhamāno;

Pappoti macco amataṃ brahmaloka’’nti.

\paragraph{320.} ‘‘Hitvā mamatta’nti ahaṃ bhoto ājānāmi. Idhekacco appaṃ vā bhogakkhandhaṃ pahāya mahantaṃ vā bhogakkhandhaṃ pahāya appaṃ vā ñātiparivaṭṭaṃ pahāya mahantaṃ vā ñātiparivaṭṭaṃ pahāya kesamassuṃ ohāretvā kāsāyāni vatthāni acchādetvā agārasmā anagāriyaṃ pabbajati, ‘iti hitvā mamatta’nti ahaṃ bhoto ājānāmi. ‘Ekodibhūto’ti ahaṃ bhoto ājānāmi. Idhekacco vivittaṃ senāsanaṃ bhajati araññaṃ rukkhamūlaṃ pabbataṃ kandaraṃ giriguhaṃ susānaṃ vanapatthaṃ abbhokāsaṃ palālapuñjaṃ, iti ekodibhūto’ti ahaṃ bhoto ājānāmi. ‘Karuṇedhimutto’ti ahaṃ bhoto ājānāmi. Idhekacco karuṇāsahagatena cetasā ekaṃ disaṃ pharitvā viharati, tathā dutiyaṃ, tathā tatiyaṃ, tathā catutthaṃ. Iti uddhamadhotiriyaṃ sabbadhi sabbattatāya sabbāvantaṃ lokaṃ karuṇāsahagatena cetasā vipulena mahaggatena appamāṇena averena abyāpajjena pharitvā viharati. Iti ‘karuṇedhimutto’ti ahaṃ bhoto ājānāmi. Āmagandhe ca kho ahaṃ bhoto bhāsamānassa na ājānāmi.

‘‘Ke āmagandhā manujesu brahme,

Ete avidvā idha brūhi dhīra;

Kenāvaṭā\footnote{kenāvuṭā (syā.)} vāti pajā kurutu\footnote{kururū (syā.), kuruṭṭharū (pī.), kurūru (?)},

Āpāyikā nivutabrahmalokā’’ti.

‘‘Kodho mosavajjaṃ nikati ca dubbho,

Kadariyatā atimāno usūyā;

Icchā vivicchā paraheṭhanā ca,

Lobho ca doso ca mado ca moho;

Etesu yuttā anirāmagandhā,

Āpāyikā nivutabrahmalokā’’ti.

‘‘Yathā kho ahaṃ bhoto āmagandhe bhāsamānassa ājānāmi. Te na sunimmadayā agāraṃ ajjhāvasatā. Pabbajissāmahaṃ, bho, agārasmā anagāriya’’nti. ‘‘Yassadāni bhavaṃ govindo kālaṃ maññatī’’ti.

\subsubsection{Reṇurājaāmantanā}

\paragraph{321.} ‘‘Atha kho, bho, mahāgovindo brāhmaṇo yena reṇu rājā tenupasaṅkami; upasaṅkamitvā reṇuṃ rājānaṃ etadavoca – ‘‘aññaṃ dāni bhavaṃ purohitaṃ pariyesatu, yo bhoto rajjaṃ anusāsissati. Icchāmahaṃ, bho , agārasmā anagāriyaṃ pabbajituṃ. Yathā kho pana me sutaṃ brahmuno āmagandhe bhāsamānassa, te na sunimmadayā agāraṃ ajjhāvasatā. Pabbajissāmahaṃ, bho, agārasmā anagāriya’’nti.

‘‘Āmantayāmi rājānaṃ, reṇuṃ bhūmipatiṃ ahaṃ;

Tvaṃ pajānassu rajjena, nāhaṃ porohicce rame’’.

‘‘Sace te ūnaṃ kāmehi, ahaṃ paripūrayāmi te;

Yo taṃ hiṃsati vāremi, bhūmisenāpati ahaṃ;

Tuvaṃ pitā ahaṃ putto, mā no govinda pājahi’’\footnote{pājehi (aṭṭhakathāyaṃ saṃvaṇṇitapāṭhantaraṃ)}.

‘‘Namatthi ūnaṃ kāmehi, hiṃsitā me na vijjati;

Amanussavaco sutvā, tasmāhaṃ na gahe rame’’.

‘‘Amanusso kathaṃvaṇṇo, kiṃ te atthaṃ abhāsatha;

Yañca sutvā jahāsi no, gehe amhe ca kevalī’’.

‘‘Upavutthassa me pubbe, yiṭṭhukāmassa me sato;

Aggi pajjalito āsi, kusapattaparitthato’’.

‘‘Tato me brahmā pāturahu, brahmalokā sanantano;

So me pañhaṃ viyākāsi, taṃ sutvā na gahe rame’’.

‘‘Saddahāmi ahaṃ bhoto, yaṃ tvaṃ govinda bhāsasi;

Amanussavaco sutvā, kathaṃ vattetha aññathā.

‘‘Te taṃ anuvattissāma, satthā govinda no bhavaṃ;

Maṇi yathā veḷuriyo, akāco vimalo subho;

Evaṃ suddhā carissāma, govindassānusāsane’’ti.

‘‘‘Sace bhavaṃ govindo agārasmā anagāriyaṃ pabbajissati, mayampi agārasmā anagāriyaṃ pabbajissāma. Atha yā te gati, sā no gati bhavissatī’’ti.

\subsubsection{Cha khattiyaāmantanā}

\paragraph{322.} ‘‘Atha kho, bho, mahāgovindo brāhmaṇo yena te cha khattiyā tenupasaṅkami; upasaṅkamitvā te cha khattiye etadavoca – ‘‘aññaṃ dāni bhavanto purohitaṃ pariyesantu, yo bhavantānaṃ rajje anusāsissati. Icchāmahaṃ, bho, agārasmā anagāriyaṃ pabbajituṃ. Yathā kho pana me sutaṃ brahmuno āmagandhe bhāsamānassa, te na sunimmadayā agāraṃ ajjhāvasatā. Pabbajissāmahaṃ, bho, agārasmā anagāriya’’nti. Atha kho, bho, te cha khattiyā ekamantaṃ apakkamma evaṃ samacintesuṃ – ‘‘ime kho brāhmaṇā nāma dhanaluddhā; yaṃnūna mayaṃ mahāgovindaṃ brāhmaṇaṃ dhanena sikkheyyāmā’’ti. Te mahāgovindaṃ brāhmaṇaṃ upasaṅkamitvā evamāhaṃsu – ‘‘saṃvijjati kho, bho, imesu sattasu rajjesu pahūtaṃ sāpateyyaṃ, tato bhoto yāvatakena attho, tāvatakaṃ āharīyata’’nti. ‘‘Alaṃ, bho, mamapidaṃ pahūtaṃ sāpateyyaṃ bhavantānaṃyeva vāhasā. Tamahaṃ sabbaṃ pahāya agārasmā anagāriyaṃ pabbajissāmi. Yathā kho pana me sutaṃ brahmuno āmagandhe bhāsamānassa, te na sunimmadayā agāraṃ ajjhāvasatā, pabbajissāmahaṃ, bho, agārasmā anagāriya’’nti. Atha kho, bho, te cha khattiyā ekamantaṃ apakkamma evaṃ samacintesuṃ – ‘‘ime kho brāhmaṇā nāma itthiluddhā; yaṃnūna mayaṃ mahāgovindaṃ brāhmaṇaṃ itthīhi sikkheyyāmā’’ti. Te mahāgovindaṃ brāhmaṇaṃ upasaṅkamitvā evamāhaṃsu – ‘‘saṃvijjanti kho, bho, imesu sattasu rajjesu pahūtā itthiyo, tato bhoto yāvatikāhi attho, tāvatikā ānīyata’’nti. ‘‘Alaṃ, bho, mamapimā\footnote{mamapitā (ka.), mamapi (sī.)} cattārīsā bhariyā sādisiyo. Tāpāhaṃ sabbā pahāya agārasmā anagāriyaṃ pabbajissāmi. Yathā kho pana me sutaṃ brahmuno āmagandhe bhāsamānassa, te na sunimmadayā agāraṃ ajjhāvasatā, pabbajissāmahaṃ, bho, agārasmā anagāriyanti’’.

\paragraph{323.} ‘‘Sace bhavaṃ govindo agārasmā anagāriyaṃ pabbajissati, mayampi agārasmā anagāriyaṃ pabbajissāma, atha yā te gati, sā no gati bhavissatīti.

‘‘Sace jahatha kāmāni, yattha satto puthujjano;

Ārambhavho daḷhā hotha, khantibalasamāhitā.

‘‘Esa maggo ujumaggo, esa maggo anuttaro;

Saddhammo sabbhi rakkhito, brahmalokūpapattiyāti.

‘‘Tena hi bhavaṃ govindo satta vassāni āgametu. Sattannaṃ vassānaṃ accayena mayampi agārasmā anagāriyaṃ pabbajissāma, atha yā te gati, sā no gati bhavissatī’’ti.

‘‘‘Aticiraṃ kho, bho, satta vassāni, nāhaṃ sakkomi, bhavante, satta vassāni āgametuṃ. Ko nu kho pana, bho, jānāti jīvitānaṃ! Gamanīyo samparāyo, mantāyaṃ\footnote{mantāya (bahūsu)} boddhabbaṃ, kattabbaṃ kusalaṃ, caritabbaṃ brahmacariyaṃ, natthi jātassa amaraṇaṃ. Yathā kho pana me sutaṃ brahmuno āmagandhe bhāsamānassa, te na sunimmadayā agāraṃ ajjhāvasatā, pabbajissāmahaṃ, bho, agārasmā anagāriya’’’nti. ‘‘Tena hi bhavaṃ govindo chabbassāni āgametu…pe… pañca vassāni āgametu… cattāri vassāni āgametu… tīṇi vassāni āgametu… dve vassāni āgametu… ekaṃ vassaṃ āgametu, ekassa vassassa accayena mayampi agārasmā anagāriyaṃ pabbajissāma, atha yā te gati, sā no gati bhavissatī’’ti.

‘‘‘Aticiraṃ kho, bho, ekaṃ vassaṃ, nāhaṃ sakkomi bhavante ekaṃ vassaṃ āgametuṃ. Ko nu kho pana, bho, jānāti jīvitānaṃ! Gamanīyo samparāyo, mantāyaṃ boddhabbaṃ, kattabbaṃ kusalaṃ , caritabbaṃ brahmacariyaṃ, natthi jātassa amaraṇaṃ. Yathā kho pana me sutaṃ brahmuno āmagandhe bhāsamānassa, te na sunimmadayā agāraṃ ajjhāvasatā, pabbajissāmahaṃ, bho, agārasmā anagāriya’’nti. ‘‘Tena hi bhavaṃ govindo satta māsāni āgametu, sattannaṃ māsānaṃ accayena mayampi agārasmā anagāriyaṃ pabbajissāma, atha yā te gati, sā no gati bhavissatī’’ti.

‘‘‘Aticiraṃ kho, bho, satta māsāni, nāhaṃ sakkomi bhavante satta māsāni āgametuṃ. Ko nu kho pana, bho, jānāti jīvitānaṃ. Gamanīyo samparāyo, mantāyaṃ boddhabbaṃ , kattabbaṃ kusalaṃ, caritabbaṃ brahmacariyaṃ, natthi jātassa amaraṇaṃ. Yathā kho pana me sutaṃ brahmuno āmagandhe bhāsamānassa, te na sunimmadayā agāraṃ ajjhāvasatā, pabbajissāmahaṃ, bho, agārasmā anagāriya’’nti.

‘‘‘Tena hi bhavaṃ govindo cha māsāni āgametu…pe… pañca māsāni āgametu… cattāri māsāni āgametu… tīṇi māsāni āgametu… dve māsāni āgametu… ekaṃ māsaṃ āgametu… addhamāsaṃ āgametu, addhamāsassa accayena mayampi agārasmā anagāriyaṃ pabbajissāma, atha yā te gati, sā no gati bhavissatī’’ti.

‘‘‘Aticiraṃ kho, bho, addhamāso, nāhaṃ sakkomi bhavante addhamāsaṃ āgametuṃ. Ko nu kho pana, bho, jānāti jīvitānaṃ! Gamanīyo samparāyo, mantāyaṃ boddhabbaṃ, kattabbaṃ kusalaṃ, caritabbaṃ brahmacariyaṃ, natthi jātassa amaraṇaṃ. Yathā kho pana me sutaṃ brahmuno āmagandhe bhāsamānassa, te na sunimmadayā agāraṃ ajjhāvasatā, pabbajissāmahaṃ, bho, agārasmā anagāriya’’nti. ‘‘Tena hi bhavaṃ govindo sattāhaṃ āgametu, yāva mayaṃ sake puttabhātaro rajjena\footnote{rajje (syā.)} anusāsissāma, sattāhassa accayena mayampi agārasmā anagāriyaṃ pabbajissāma, atha yā te gati, sā no gati bhavissatī’’ti. ‘‘Na ciraṃ kho, bho, sattāhaṃ, āgamessāmahaṃ bhavante sattāha’’nti.

\subsubsection{Brāhmaṇamahāsālādīnaṃ āmantanā}

\paragraph{324.} ‘‘Atha kho, bho, mahāgovindo brāhmaṇo yena te satta ca brāhmaṇamahāsālā satta ca nhātakasatāni tenupasaṅkami; upasaṅkamitvā te satta ca brāhmaṇamahāsāle satta ca nhātakasatāni etadavoca – ‘‘aññaṃ dāni bhavanto ācariyaṃ pariyesantu, yo bhavantānaṃ mante vācessati. Icchāmahaṃ, bho, agārasmā anagāriyaṃ pabbajituṃ. Yathā kho pana me sutaṃ brahmuno āmagandhe bhāsamānassa. Te na sunimmadayā agāraṃ ajjhāvasatā, pabbajissāmahaṃ, bho, agārasmā anagāriya’’nti. ‘‘Mā bhavaṃ govindo agārasmā anagāriyaṃ pabbaji. Pabbajjā, bho, appesakkhā ca appalābhā ca; brahmaññaṃ mahesakkhañca mahālābhañcā’’ti. ‘‘Mā bhavanto evaṃ avacuttha – ‘‘pabbajjā appesakkhā ca appalābhā ca, brahmaññaṃ mahesakkhañca mahālābhañcā’’ti. Ko nu kho, bho, aññatra mayā mahesakkhataro vā mahālābhataro vā! Ahañhi, bho, etarahi rājāva raññaṃ brahmāva brāhmaṇānaṃ\footnote{brahmānaṃ (sī. pī. ka.)} devatāva gahapatikānaṃ. Tamahaṃ sabbaṃ pahāya agārasmā anagāriyaṃ pabbajissāmi. Yathā kho pana me sutaṃ brahmuno āmagandhe bhāsamānassa, te na sunimmadayā agāraṃ ajjhāvasatā. Pabbajissāmahaṃ, bho, agārasmā anagāriya’’nti. ‘‘Sace bhavaṃ govindo agārasmā anagāriyaṃ pabbajissati, mayampi agārasmā anagāriyaṃ pabbajissāma, atha yā te gati, sā no gati bhavissatī’’ti.

\subsubsection{Bhariyānaṃ āmantanā}

\paragraph{325.} ‘‘Atha kho, bho, mahāgovindo brāhmaṇo yena cattārīsā bhariyā sādisiyo tenupasaṅkami; upasaṅkamitvā cattārīsā bhariyā sādisiyo etadavoca – ‘‘yā bhotīnaṃ icchati, sakāni vā ñātikulāni gacchatu aññaṃ vā bhattāraṃ pariyesatu. Icchāmahaṃ, bhotī, agārasmā anagāriyaṃ pabbajituṃ. Yathā kho pana me sutaṃ brahmuno āmagandhe bhāsamānassa, te na sunimmadayā agāraṃ ajjhāvasatā. Pabbajissāmahaṃ, bhotī, agārasmā anagāriya’’nti. ‘‘Tvaññeva no ñāti ñātikāmānaṃ, tvaṃ pana bhattā bhattukāmānaṃ. Sace bhavaṃ govindo agārasmā anagāriyaṃ pabbajissati, mayampi agārasmā anagāriyaṃ pabbajissāma, atha yā te gati, sā no gati bhavissatī’’ti.

\subsubsection{Mahāgovindapabbajjā}

\paragraph{326.} ‘‘Atha kho, bho, mahāgovindo brāhmaṇo tassa sattāhassa accayena kesamassuṃ ohāretvā kāsāyāni vatthāni acchādetvā agārasmā anagāriyaṃ pabbaji. Pabbajitaṃ pana mahāgovindaṃ brāhmaṇaṃ satta ca rājāno khattiyā muddhāvasittā satta ca brāhmaṇamahāsālā satta ca nhātakasatāni cattārīsā ca bhariyā sādisiyo anekāni ca khattiyasahassāni anekāni ca brāhmaṇasahassāni anekāni ca gahapatisahassāni anekehi ca itthāgārehi itthiyo kesamassuṃ ohāretvā kāsāyāni vatthāni acchādetvā mahāgovindaṃ brāhmaṇaṃ agārasmā anagāriyaṃ pabbajitaṃ anupabbajiṃsu. Tāya sudaṃ, bho, parisāya parivuto mahāgovindo brāhmaṇo gāmanigamarājadhānīsu cārikaṃ carati. Yaṃ kho pana, bho, tena samayena mahāgovindo brāhmaṇo gāmaṃ vā nigamaṃ vā upasaṅkamati, tattha rājāva hoti raññaṃ, brahmāva brāhmaṇānaṃ, devatāva gahapatikānaṃ. Tena kho pana samayena manussā khipanti vā upakkhalanti vā te evamāhaṃsu – ‘‘namatthu mahāgovindassa brāhmaṇassa, namatthu satta purohitassā’’’ti.

\paragraph{327.} ‘‘Mahāgovindo, bho, brāhmaṇo mettāsahagatena cetasā ekaṃ disaṃ pharitvā vihāsi, tathā dutiyaṃ, tathā tatiyaṃ, tathā catutthaṃ. Iti uddhamadho tiriyaṃ sabbadhi sabbattatāya sabbāvantaṃ lokaṃ mettāsahagatena cetasā vipulena mahaggatena appamāṇena averena abyāpajjena pharitvā vihāsi. Karuṇāsahagatena cetasā…pe… muditāsahagatena cetasā…pe… upekkhāsahagatena cetasā…pe… abyāpajjena pharitvā vihāsi sāvakānañca brahmalokasahabyatāya maggaṃ desesi.

\paragraph{328.} ‘‘Ye kho pana, bho, tena samayena mahāgovindassa brāhmaṇassa sāvakā sabbena sabbaṃ sāsanaṃ ājāniṃsu. Te kāyassa bhedā paraṃ maraṇā sugatiṃ brahmalokaṃ upapajjiṃsu. Ye na sabbena sabbaṃ sāsanaṃ ājāniṃsu, te kāyassa bhedā paraṃ maraṇā appekacce paranimmitavasavattīnaṃ devānaṃ sahabyataṃ upapajjiṃsu; appekacce nimmānaratīnaṃ devānaṃ sahabyataṃ upapajjiṃsu; appekacce tusitānaṃ devānaṃ sahabyataṃ upapajjiṃsu; appekacce yāmānaṃ devānaṃ sahabyataṃ upapajjiṃsu; appekacce tāvatiṃsānaṃ devānaṃ sahabyataṃ upapajjiṃsu; appekacce cātumahārājikānaṃ devānaṃ sahabyataṃ upapajjiṃsu; ye sabbanihīnaṃ kāyaṃ paripūresuṃ te gandhabbakāyaṃ paripūresuṃ. Iti kho, bho\footnote{pana (syā. ka.)}, sabbesaṃyeva tesaṃ kulaputtānaṃ amoghā pabbajjā ahosi avañjhā saphalā saudrayā’’’ti.

\paragraph{329.} ‘‘Sarati taṃ bhagavā’’ti? ‘‘Sarāmahaṃ, pañcasikha. Ahaṃ tena samayena mahāgovindo brāhmaṇo ahosiṃ. Ahaṃ tesaṃ sāvakānaṃ brahmalokasahabyatāya maggaṃ desesiṃ. Taṃ kho pana me, pañcasikha, brahmacariyaṃ na nibbidāya na virāgāya na nirodhāya na upasamāya na abhiññāya na sambodhāya na nibbānāya saṃvattati, yāvadeva brahmalokūpapattiyā.

Idaṃ kho pana me, pañcasikha, brahmacariyaṃ ekantanibbidāya virāgāya nirodhāya upasamāya abhiññāya sambodhāya nibbānāya saṃvattati. Katamañca taṃ, pañcasikha, brahmacariyaṃ ekantanibbidāya virāgāya nirodhāya upasamāya abhiññāya sambodhāya nibbānāya saṃvattati? Ayameva ariyo aṭṭhaṅgiko maggo. Seyyathidaṃ – sammādiṭṭhi sammāsaṅkappo sammāvācā sammākammanto sammāājīvo sammāvāyāmo sammāsati sammāsamādhi. Idaṃ kho taṃ, pañcasikha, brahmacariyaṃ ekantanibbidāya virāgāya nirodhāya upasamāya abhiññāya sambodhāya nibbānāya saṃvattati.

\paragraph{330.} ‘‘Ye kho pana me, pañcasikha, sāvakā sabbena sabbaṃ sāsanaṃ ājānanti, te āsavānaṃ khayā anāsavaṃ cetovimuttiṃ paññāvimuttiṃ diṭṭheva dhamme sayaṃ abhiññā sacchikatvā upasampajja viharanti; ye na sabbena sabbaṃ sāsanaṃ ājānanti, te pañcannaṃ orambhāgiyānaṃ saṃyojanānaṃ parikkhayā opapātikā honti tattha parinibbāyino anāvattidhammā tasmā lokā. Ye na sabbena sabbaṃ sāsanaṃ ājānanti, appekacce tiṇṇaṃ saṃyojanānaṃ parikkhayā rāgadosamohānaṃ tanuttā sakadāgāmino honti sakideva imaṃ lokaṃ āgantvā dukkhassantaṃ karissanti\footnote{karonti (sī. pī.)}. Ye na sabbena sabbaṃ sāsanaṃ ājānanti, appekacce tiṇṇaṃ saṃyojanānaṃ parikkhayā sotāpannā honti avinipātadhammā niyatā sambodhiparāyaṇā. Iti kho, pañcasikha, sabbesaṃyeva imesaṃ kulaputtānaṃ amoghā pabbajjā\footnote{pabbajā ahosi (ka.)} avañjhā saphalā saudrayā’’ti.

Idamavoca bhagavā. Attamano pañcasikho gandhabbaputto bhagavato bhāsitaṃ abhinanditvā anumoditvā bhagavantaṃ abhivādetvā padakkhiṇaṃ katvā tatthevantaradhāyīti.

\xsectionEnd{Mahāgovindasuttaṃ niṭṭhitaṃ chaṭṭhaṃ.}


\clearpage
\section{Mahāsamayasuttaṃ}

\paragraph{331.} Evaṃ me sutaṃ – ekaṃ samayaṃ bhagavā sakkesu viharati kapilavatthusmiṃ mahāvane mahatā bhikkhusaṅghena saddhiṃ pañcamattehi bhikkhusatehi sabbeheva arahantehi; dasahi ca lokadhātūhi devatā yebhuyyena sannipatitā honti bhagavantaṃ dassanāya bhikkhusaṅghañca. Atha kho catunnaṃ suddhāvāsakāyikānaṃ devatānaṃ\footnote{devānaṃ (sī. syā. pī.)} etadahosi – ‘‘ayaṃ kho bhagavā sakkesu viharati kapilavatthusmiṃ mahāvane mahatā bhikkhusaṅghena saddhiṃ pañcamattehi bhikkhusatehi sabbeheva arahantehi; dasahi ca lokadhātūhi devatā yebhuyyena sannipatitā honti bhagavantaṃ dassanāya bhikkhusaṅghañca. Yaṃnūna mayampi yena bhagavā tenupasaṅkameyyāma; upasaṅkamitvā bhagavato santike paccekaṃ gāthaṃ\footnote{paccekagāthaṃ (sī. syā. pī.), paccekagāthā (ka. sī.)} bhāseyyāmā’’ti.

\paragraph{332.} Atha kho tā devatā seyyathāpi nāma balavā puriso samiñjitaṃ vā bāhaṃ pasāreyya pasāritaṃ vā bāhaṃ samiñjeyya , evameva suddhāvāsesu devesu antarahitā bhagavato purato pāturahesuṃ. Atha kho tā devatā bhagavantaṃ abhivādetvā ekamantaṃ aṭṭhaṃsu. Ekamantaṃ ṭhitā kho ekā devatā bhagavato santike imaṃ gāthaṃ abhāsi –

‘‘Mahāsamayo pavanasmiṃ, devakāyā samāgatā;

Āgatamha imaṃ dhammasamayaṃ, dakkhitāye aparājitasaṅgha’’nti.

Atha kho aparā devatā bhagavato santike imaṃ gāthaṃ abhāsi –

‘‘Tatra bhikkhavo samādahaṃsu, cittamattano ujukaṃ akaṃsu\footnote{ujukamakaṃsu (sī. syā. pī.)};

Sārathīva nettāni gahetvā, indriyāni rakkhanti paṇḍitā’’ti.

Atha kho aparā devatā bhagavato santike imaṃ gāthaṃ abhāsi –

‘‘Chetvā khīlaṃ chetvā palighaṃ, indakhīlaṃ ūhacca\footnote{uhacca (ka.)} manejā;

Te caranti suddhā vimalā, cakkhumatā sudantā susunāgā’’ti.

Atha kho aparā devatā bhagavato santike imaṃ gāthaṃ abhāsi –

‘‘Yekeci buddhaṃ saraṇaṃ gatāse, na te gamissanti apāyabhūmiṃ;

Pahāya mānusaṃ dehaṃ, devakāyaṃ paripūressantī’’ti.

\subsubsection{Devatāsannipātā}

\paragraph{333.} Atha kho bhagavā bhikkhū āmantesi – ‘‘yebhuyyena, bhikkhave, dasasu lokadhātūsu devatā sannipatitā honti\footnote{( ) sī. ipotthakesu natthi}, tathāgataṃ dassanāya bhikkhusaṅghañca . Yepi te, bhikkhave, ahesuṃ atītamaddhānaṃ arahanto sammāsambuddhā, tesampi bhagavantānaṃ etaṃparamāyeva\footnote{etaparamāyeva (sī. syā. pī.)} devatā sannipatitā ahesuṃ seyyathāpi mayhaṃ etarahi. Yepi te, bhikkhave, bhavissanti anāgatamaddhānaṃ arahanto sammāsambuddhā, tesampi bhagavantānaṃ etaṃparamāyeva devatā sannipatitā bhavissanti seyyathāpi mayhaṃ etarahi. Ācikkhissāmi, bhikkhave, devakāyānaṃ nāmāni; kittayissāmi, bhikkhave, devakāyānaṃ nāmāni; desessāmi, bhikkhave, devakāyānaṃ nāmāni. Taṃ suṇātha, sādhukaṃ manasikarotha, bhāsissāmī’’ti. ‘‘Evaṃ, bhante’’ti kho te bhikkhū bhagavato paccassosuṃ.

\paragraph{334.} Bhagavā etadavoca –

‘‘Silokamanukassāmi, yattha bhummā tadassitā;

Ye sitā girigabbharaṃ, pahitattā samāhitā.

‘‘Puthūsīhāva sallīnā, lomahaṃsābhisambhuno;

Odātamanasā suddhā, vippasannamanāvilā’’\footnote{vippasannāmanāvilā (pī. ka.)}.

Bhiyyo pañcasate ñatvā, vane kāpilavatthave;

Tato āmantayī satthā, sāvake sāsane rate.

‘‘Devakāyā abhikkantā, te vijānātha bhikkhavo’’;

Te ca ātappamakaruṃ, sutvā buddhassa sāsanaṃ.

Tesaṃ pāturahu ñāṇaṃ, amanussānadassanaṃ;

Appeke satamaddakkhuṃ, sahassaṃ atha sattariṃ.

Sataṃ eke sahassānaṃ, amanussānamaddasuṃ;

Appekenantamaddakkhuṃ , disā sabbā phuṭā ahuṃ.

Tañca sabbaṃ abhiññāya, vavatthitvāna\footnote{vavakkhitvāna (sī. syā. pī.), avekkhitvāna (ṭīkā)} cakkhumā;

Tato āmantayī satthā, sāvake sāsane rate.

‘‘Devakāyā abhikkantā, te vijānātha bhikkhavo;

Ye vohaṃ kittayissāmi, girāhi anupubbaso.

\paragraph{335.}‘‘Sattasahassā te yakkhā, bhummā kāpilavatthavā.

Iddhimanto jutimanto, vaṇṇavanto yasassino;

Modamānā abhikkāmuṃ, bhikkhūnaṃ samitiṃ vanaṃ.

‘‘Chasahassā hemavatā, yakkhā nānattavaṇṇino;

Iddhimanto jutīmanto\footnote{jutīmanto (sī. pī.)}, vaṇṇavanto yasassino;

Modamānā abhikkāmuṃ, bhikkhūnaṃ samitiṃ vanaṃ.

‘‘Sātāgirā tisahassā, yakkhā nānattavaṇṇino;

Iddhimanto jutimanto, vaṇṇavanto yasassino;

Modamānā abhikkāmuṃ, bhikkhūnaṃ samitiṃ vanaṃ.

‘‘Iccete soḷasasahassā, yakkhā nānattavaṇṇino;

Iddhimanto jutimanto, vaṇṇavanto yasassino;

Modamānā abhikkāmuṃ, bhikkhūnaṃ samitiṃ vanaṃ.

‘‘Vessāmittā pañcasatā, yakkhā nānattavaṇṇino;

Iddhimanto jutimanto, vaṇṇavanto yasassino;

Modamānā abhikkāmuṃ, bhikkhūnaṃ samitiṃ vanaṃ.

‘‘Kumbhīro rājagahiko, vepullassa nivesanaṃ;

Bhiyyo naṃ satasahassaṃ, yakkhānaṃ payirupāsati;

Kumbhīro rājagahiko, sopāgā samitiṃ vanaṃ.

\paragraph{336.}‘‘Purimañca disaṃ rājā, dhataraṭṭho pasāsati.

Gandhabbānaṃ adhipati, mahārājā yasassiso.

‘‘Puttāpi tassa bahavo, indanāmā mahabbalā;

Iddhimanto jutimanto, vaṇṇavanto yasassino;

Modamānā abhikkāmuṃ, bhikkhūnaṃ samitiṃ vanaṃ.

‘‘Dakkhiṇañca disaṃ rājā, virūḷho taṃ pasāsati\footnote{tappasāsati (syā.)};

Kumbhaṇḍānaṃ adhipati, mahārājā yasassiso.

‘‘Puttāpi tassa bahavo, indanāmā mahabbalā;

Iddhimanto jutimanto, vaṇṇavanto yasassino;

Modamānā abhikkāmuṃ, bhikkhūnaṃ samitiṃ vanaṃ.

‘‘Pacchimañca disaṃ rājā, virūpakkho pasāsati;

Nāgānañca adhipati, mahārājā yasassiso.

‘‘Puttāpi tassa bahavo, indanāmā mahabbalā;

Iddhimanto jutimanto, vaṇṇavanto yasassino;

Modamānā abhikkāmuṃ, bhikkhūnaṃ samitiṃ vanaṃ.

‘‘Uttarañca disaṃ rājā, kuvero taṃ pasāsati;

Yakkhānañca adhipati, mahārājā yasassiso.

‘‘Puttāpi tassa bahavo, indanāmā mahabbalā;

Iddhimanto jutimanto, vaṇṇavanto yasassino;

Modamānā abhikkāmuṃ, bhikkhūnaṃ samitiṃ vanaṃ.

‘‘Purimaṃ disaṃ dhataraṭṭho, dakkhiṇena virūḷhako;

Pacchimena virūpakkho, kuvero uttaraṃ disaṃ.

‘‘Cattāro te mahārājā, samantā caturo disā;

Daddallamānā\footnote{daddaḷhamānā (ka.)} aṭṭhaṃsu, vane kāpilavatthave.

\paragraph{337.}‘‘Tesaṃ māyāvino dāsā, āguṃ\footnote{āgū (syā.), āgu (sī. pī.) evamuparipi} vañcanikā saṭhā.

Māyā kuṭeṇḍu viṭeṇḍu\footnote{veṭeṇḍu (sī. syā. pī.)}, viṭucca\footnote{viṭū ca (syā.)} viṭuṭo saha.

‘‘Candano kāmaseṭṭho ca, kinnighaṇḍu\footnote{kinnughaṇḍu (sī. syā. pī.)} nighaṇḍu ca;

Panādo opamañño ca, devasūto ca mātali.

‘‘Cittaseno ca gandhabbo, naḷorājā janesabho\footnote{janosabho (syā.)};

Āgā pañcasikho ceva, timbarū sūriyavaccasā\footnote{suriyavaccasā (sī. pī.)}.

‘‘Ete caññe ca rājāno, gandhabbā saha rājubhi;

Modamānā abhikkāmuṃ, bhikkhūnaṃ samitiṃ vanaṃ.

\paragraph{338.}‘‘Athāguṃ nāgasā nāgā, vesālā sahatacchakā.

Kambalassatarā āguṃ, pāyāgā saha ñātibhi.

‘‘Yāmunā dhataraṭṭhā ca, āgū nāgā yasassino;

Erāvaṇo mahānāgo, sopāgā samitiṃ vanaṃ.

‘‘Ye nāgarāje sahasā haranti, dibbā dijā pakkhi visuddhacakkhū;

Vehāyasā\footnote{vehāsayā (sī. pī.)} te vanamajjhapattā, citrā supaṇṇā iti tesa nāmaṃ.

‘‘Abhayaṃ tadā nāgarājānamāsi, supaṇṇato khemamakāsi buddho;

Saṇhāhi vācāhi upavhayantā, nāgā supaṇṇā saraṇamakaṃsu buddhaṃ.

\paragraph{339.}‘‘Jitā vajirahatthena, samuddaṃ asurāsitā.

Bhātaro vāsavassete, iddhimanto yasassino.

‘‘Kālakañcā mahābhismā\footnote{kālakañjā mahābhiṃsā (sī. pī.)}, asurā dānaveghasā;

Vepacitti sucitti ca, pahārādo namucī saha.

‘‘Satañca baliputtānaṃ, sabbe verocanāmakā;

Sannayhitvā balisenaṃ\footnote{balīsenaṃ (syā.)}, rāhubhaddamupāgamuṃ;

Samayodāni bhaddante, bhikkhūnaṃ samitiṃ vanaṃ.

\paragraph{340.}‘‘Āpo ca devā pathavī, tejo vāyo tadāgamuṃ.

Varuṇā vāraṇā\footnote{vāruṇā (syā.)} devā, somo ca yasasā saha.

‘‘Mettā karuṇā kāyikā, āguṃ devā yasassino;

Dasete dasadhā kāyā, sabbe nānattavaṇṇino.

‘‘Iddhimanto jutimanto, vaṇṇavanto yasassino;

Modamānā abhikkāmuṃ, bhikkhūnaṃ samitiṃ vanaṃ.

‘‘Veṇḍudevā sahali ca\footnote{veṇhūca devā sahalīca (sī. pī.)}, asamā ca duve yamā;

Candassūpanisā devā, candamāguṃ purakkhatvā.

‘‘Sūriyassūpanisā\footnote{suriyassūpanisā (sī. syā. pī.)} devā, sūriyamāguṃ purakkhatvā;

Nakkhattāni purakkhatvā, āguṃ mandavalāhakā.

‘‘Vasūnaṃ vāsavo seṭṭho, sakkopāgā purindado;

Dasete dasadhā kāyā, sabbe nānattavaṇṇino.

‘‘Iddhimanto jutimanto, vaṇṇavanto yasassino;

Modamānā abhikkāmuṃ, bhikkhūnaṃ samitiṃ vanaṃ.

‘‘Athāguṃ sahabhū devā, jalamaggisikhāriva;

Ariṭṭhakā ca rojā ca, umāpupphanibhāsino.

‘‘Varuṇā sahadhammā ca, accutā ca anejakā;

Sūleyyarucirā āguṃ, āguṃ vāsavanesino;

Dasete dasadhā kāyā, sabbe nānattavaṇṇino.

‘‘Iddhimanto jutimanto, vaṇṇavanto yasassino;

Modamānā abhikkāmuṃ, bhikkhūnaṃ samitiṃ vanaṃ.

‘‘Samānā mahāsamanā, mānusā mānusuttamā;

Khiḍḍāpadosikā āguṃ, āguṃ manopadosikā.

‘‘Athāguṃ harayo devā, ye ca lohitavāsino;

Pāragā mahāpāragā, āguṃ devā yasassino;

Dasete dasadhā kāyā, sabbe nānattavaṇṇino.

‘‘Iddhimanto jutimanto, vaṇṇavanto yasassino;

Modamānā abhikkāmuṃ, bhikkhūnaṃ samitiṃ vanaṃ.

‘‘Sukkā karambhā\footnote{karumhā (sī. syā. pī.)} aruṇā, āguṃ veghanasā saha;

Odātagayhā pāmokkhā, āguṃ devā vicakkhaṇā.

‘‘Sadāmattā hāragajā, missakā ca yasassino;

Thanayaṃ āga pajjunno, yo disā abhivassati.

‘‘Dasete dasadhā kāyā, sabbe nānattavaṇṇino;

Iddhimanto jutimanto, vaṇṇavanto yasassino;

Modamānā abhikkāmuṃ, bhikkhūnaṃ samitiṃ vanaṃ.

‘‘Khemiyā tusitā yāmā, kaṭṭhakā ca yasassino;

Lambītakā lāmaseṭṭhā, jotināmā ca āsavā;

Nimmānaratino āguṃ, athāguṃ paranimmitā.

‘‘Dasete dasadhā kāyā, sabbe nānattavaṇṇino;

Iddhimanto jutimanto, vaṇṇavanto yasassino;

Modamānā abhikkāmuṃ, bhikkhūnaṃ samitiṃ vanaṃ.

‘‘Saṭṭhete devanikāyā, sabbe nānattavaṇṇino;

Nāmanvayena āgacchuṃ\footnote{āgañchuṃ (sī. syā. pī.)}, ye caññe sadisā saha.

‘‘‘Pavuṭṭhajātimakhilaṃ\footnote{pavutthajātiṃ akhilaṃ (sī. pī.)}, oghatiṇṇamanāsavaṃ;

Dakkhemoghataraṃ nāgaṃ, candaṃva asitātigaṃ’.

\paragraph{341.}‘‘Subrahmā paramatto ca\footnote{paramattho ca (ka.)}, puttā iddhimato saha.

Sanaṅkumāro tisso ca, sopāga samitiṃ vanaṃ.

‘‘Sahassaṃ brahmalokānaṃ, mahābrahmābhitiṭṭhati;

Upapanno jutimanto, bhismākāyo yasassiso.

‘‘Dasettha issarā āguṃ, paccekavasavattino;

Tesañca majjhato āga, hārito parivārito.

\paragraph{342.}‘‘Te ca sabbe abhikkante, sainde\footnote{sinde (syā.)} deve sabrahmake.

Mārasenā abhikkāmi, passa kaṇhassa mandiyaṃ.

‘‘‘Etha gaṇhatha bandhatha, rāgena baddhamatthu vo;

Samantā parivāretha, mā vo muñcittha koci naṃ’.

‘‘Iti tattha mahāseno, kaṇho senaṃ apesayi;

Pāṇinā talamāhacca, saraṃ katvāna bheravaṃ.

‘‘Yathā pāvussako megho, thanayanto savijjuko; +

Tadā so paccudāvatti, saṅkuddho asayaṃvase\footnote{asayaṃvasī (sī. pī.)}.

\paragraph{343.} Tañca sabbaṃ abhiññāya, vavatthitvāna cakkhumā.

Tato āmantayī satthā, sāvake sāsane rate.

‘‘Mārasenā abhikkantā, te vijānātha bhikkhavo;

Te ca ātappamakaruṃ, sutvā buddhassa sāsanaṃ;

Vītarāgehi pakkāmuṃ, nesaṃ lomāpi iñjayuṃ.

‘‘‘Sabbe vijitasaṅgāmā, bhayātītā yasassino;

Modanti saha bhūtehi, sāvakā te janesutā’’ti.

\xsectionEnd{Mahāsamayasuttaṃ niṭṭhitaṃ sattamaṃ.}


\clearpage
\section{Sakkapañhasuttaṃ}

\paragraph{344.} Evaṃ me sutaṃ – ekaṃ samayaṃ bhagavā magadhesu viharati, pācīnato rājagahassa ambasaṇḍā nāma brāhmaṇagāmo, tassuttarato vediyake pabbate indasālaguhāyaṃ. Tena kho pana samayena sakkassa devānamindassa ussukkaṃ udapādi bhagavantaṃ dassanāya. Atha kho sakkassa devānamindassa etadahosi – ‘‘kahaṃ nu kho bhagavā etarahi viharati arahaṃ sammāsambuddho’’ti? Addasā kho sakko devānamindo bhagavantaṃ magadhesu viharantaṃ pācīnato rājagahassa ambasaṇḍā nāma brāhmaṇagāmo, tassuttarato vediyake pabbate indasālaguhāyaṃ. Disvāna deve tāvatiṃse āmantesi – ‘‘ayaṃ, mārisā, bhagavā magadhesu viharati, pācīnato rājagahassa ambasaṇḍā nāma brāhmaṇagāmo, tassuttarato vediyake pabbate indasālaguhāyaṃ. Yadi pana, mārisā, mayaṃ taṃ bhagavantaṃ dassanāya upasaṅkameyyāma arahantaṃ sammāsambuddha’’nti? ‘‘Evaṃ bhaddantavā’’ti kho devā tāvatiṃsā sakkassa devānamindassa paccassosuṃ.

\paragraph{345.} Atha kho sakko devānamindo pañcasikhaṃ gandhabbadevaputtaṃ\footnote{gandhabbaputtaṃ (syā.)} āmantesi – ‘‘ayaṃ, tāta pañcasikha, bhagavā magadhesu viharati pācīnato rājagahassa ambasaṇḍā nāma brāhmaṇagāmo, tassuttarato vediyake pabbate indasālaguhāyaṃ. Yadi pana , tāta pañcasikha, mayaṃ taṃ bhagavantaṃ dassanāya upasaṅkameyyāma arahantaṃ sammāsambuddha’’nti? ‘‘Evaṃ bhaddantavā’’ti kho pañcasikho gandhabbadevaputto sakkassa devānamindassa paṭissutvā beluvapaṇḍuvīṇaṃ ādāya sakkassa devānamindassa anucariyaṃ upāgami.

\paragraph{346.} Atha kho sakko devānamindo devehi tāvatiṃsehi parivuto pañcasikhena gandhabbadevaputtena purakkhato seyyathāpi nāma balavā puriso samiñjitaṃ vā bāhaṃ pasāreyya pasāritaṃ vā bāhaṃ samiñjeyya; evameva devesu tāvatiṃsesu antarahito magadhesu pācīnato rājagahassa ambasaṇḍā nāma brāhmaṇagāmo, tassuttarato vediyake pabbate paccuṭṭhāsi. Tena kho pana samayena vediyako pabbato atiriva obhāsajāto hoti ambasaṇḍā ca brāhmaṇagāmo yathā taṃ devānaṃ devānubhāvena. Apissudaṃ parito gāmesu manussā evamāhaṃsu – ‘‘ādittassu nāmajja vediyako pabbato jhāyatisu\footnote{jhāyatassu (syā.), pajjhāyitassu (sī. pī.)} nāmajja vediyako pabbato jalatisu\footnote{jalatassu (syā.), jalitassu (sī. pī.)} nāmajja vediyako pabbato kiṃsu nāmajja vediyako pabbato atiriva obhāsajāto ambasaṇḍā ca brāhmaṇagāmo’’ti saṃviggā lomahaṭṭhajātā ahesuṃ.

\paragraph{347.} Atha kho sakko devānamindo pañcasikhaṃ gandhabbadevaputtaṃ āmantesi – ‘‘durupasaṅkamā kho, tāta pañcasikha, tathāgatā mādisena, jhāyī jhānaratā, tadantaraṃ\footnote{tadanantaraṃ (sī. syā. pī. ka.)} paṭisallīnā. Yadi pana tvaṃ, tāta pañcasikha, bhagavantaṃ paṭhamaṃ pasādeyyāsi, tayā, tāta, paṭhamaṃ pasāditaṃ pacchā mayaṃ taṃ bhagavantaṃ dassanāya upasaṅkameyyāma arahantaṃ sammāsambuddha’’nti. ‘‘Evaṃ bhaddantavā’’ti kho pañcasikho gandhabbadevaputto sakkassa devānamindassa paṭissutvā beluvapaṇḍuvīṇaṃ ādāya yena indasālaguhā tenupasaṅkami; upasaṅkamitvā ‘‘ettāvatā me bhagavā neva atidūre bhavissati nāccāsanne, saddañca me sossatī’’ti ekamantaṃ aṭṭhāsi.

\subsubsection{Pañcasikhagītagāthā}

\paragraph{348.} Ekamantaṃ ṭhito kho pañcasikho gandhabbadevaputto beluvapaṇḍuvīṇaṃ\footnote{veḷuvapaṇḍuvīṇaṃ ādāya (syā.)} assāvesi, imā ca gāthā abhāsi buddhūpasañhitā dhammūpasañhitā saṅghūpasañhitā arahantūpasañhitā kāmūpasañhitā –

‘‘Vande te pitaraṃ bhadde, timbaruṃ sūriyavacchase;

Yena jātāsi kalyāṇī, ānandajananī mama.

‘‘Vātova sedataṃ kanto, pānīyaṃva pipāsato;

Aṅgīrasi piyāmesi, dhammo arahatāmiva.

‘‘Āturasseva bhesajjaṃ, bhojanaṃva jighacchato;

Parinibbāpaya maṃ bhadde, jalantamiva vārinā.

‘‘Sītodakaṃ pokkharaṇiṃ, yuttaṃ kiñjakkhareṇunā;

Nāgo ghammābhitattova, ogāhe te thanūdaraṃ.

‘‘Accaṅkusova nāgova, jitaṃ me tuttatomaraṃ;

Kāraṇaṃ nappajānāmi, sammatto lakkhaṇūruyā.

‘‘Tayi gedhitacittosmi, cittaṃ vipariṇāmitaṃ;

Paṭigantuṃ na sakkomi, vaṅkaghastova ambujo.

‘‘Vāmūru saja maṃ bhadde, saja maṃ mandalocane;

Palissaja maṃ kalyāṇi, etaṃ me abhipatthitaṃ.

‘‘Appako vata me santo, kāmo vellitakesiyā;

Anekabhāvo samuppādi, arahanteva dakkhiṇā.

‘‘Yaṃ me atthi kataṃ puññaṃ, arahantesu tādisu;

Taṃ me sabbaṅgakalyāṇi, tayā saddhiṃ vipaccataṃ.

‘‘Yaṃ me atthi kataṃ puññaṃ, asmiṃ pathavimaṇḍale;

Taṃ me sabbaṅgakalyāṇi, tayā saddhiṃ vipaccataṃ.

‘‘Sakyaputtova jhānena, ekodi nipako sato;

Amataṃ muni jigīsāno\footnote{jigiṃsāno (sī. syā. pī.)}, tamahaṃ sūriyavacchase.

‘‘Yathāpi muni nandeyya, patvā sambodhimuttamaṃ;

Evaṃ nandeyyaṃ kalyāṇi, missībhāvaṃ gato tayā.

‘‘Sakko ce me varaṃ dajjā, tāvatiṃsānamissaro;

Tāhaṃ bhadde vareyyāhe, evaṃ kāmo daḷho mama.

‘‘Sālaṃva na ciraṃ phullaṃ, pitaraṃ te sumedhase;

Vandamāno namassāmi, yassā setādisī pajā’’ti.

\paragraph{349.} Evaṃ vutte bhagavā pañcasikhaṃ gandhabbadevaputtaṃ etadavoca – ‘‘saṃsandati kho te, pañcasikha, tantissaro gītassarena, gītassaro ca tantissarena; na ca pana\footnote{neva pana (syā.)} te pañcasikha, tantissaro gītassaraṃ ativattati, gītassaro ca tantissaraṃ. Kadā saṃyūḷhā pana te, pañcasikha, imā gāthā buddhūpasañhitā dhammūpasañhitā saṅghūpasañhitā arahantūpasañhitā kāmūpasañhitā’’ti? ‘‘Ekamidaṃ, bhante, samayaṃ bhagavā uruvelāyaṃ viharati najjā nerañjarāya tīre ajapālanigrodhe paṭhamābhisambuddho . Tena kho panāhaṃ, bhante, samayena bhaddā nāma sūriyavacchasā timbaruno gandhabbarañño dhītā, tamabhikaṅkhāmi. Sā kho pana, bhante, bhaginī parakāminī hoti; sikhaṇḍī nāma mātalissa saṅgāhakassa putto, tamabhikaṅkhati. Yato kho ahaṃ, bhante, taṃ bhaginiṃ nālatthaṃ kenaci pariyāyena. Athāhaṃ beluvapaṇḍuvīṇaṃ ādāya yena timbaruno gandhabbarañño nivesanaṃ tenupasaṅkamiṃ; upasaṅkamitvā beluvapaṇḍuvīṇaṃ assāvesiṃ, imā ca gāthā abhāsiṃ buddhūpasañhitā dhammūpasañhitā saṅghūpasañhitā arahantūpasañhitā kāmūpasañhitā –

‘‘Vande te pitaraṃ bhadde, timbaruṃ sūriyavacchase;

Yena jātāsi kalyāṇī, ānandajananī mama. …pe…

Sālaṃva na ciraṃ phullaṃ, pitaraṃ te sumedhase;

Vandamāno namassāmi, yassā setādisī pajā’’ti.

‘‘Evaṃ vutte, bhante, bhaddā sūriyavacchasā maṃ etadavoca – ‘na kho me, mārisa, so bhagavā sammukhā diṭṭho api ca sutoyeva me so bhagavā devānaṃ tāvatiṃsānaṃ sudhammāyaṃ sabhāyaṃ upanaccantiyā. Yato kho tvaṃ, mārisa, taṃ bhagavantaṃ kittesi, hotu no ajja samāgamo’ti. Soyeva no, bhante, tassā bhaginiyā saddhiṃ samāgamo ahosi. Na ca dāni tato pacchā’’ti.

\subsubsection{Sakkūpasaṅkama}

\paragraph{350.} Atha kho sakkassa devānamindassa etadahosi – ‘‘paṭisammodati pañcasikho gandhabbadevaputto bhagavatā, bhagavā ca pañcasikhenā’’ti. Atha kho sakko devānamindo pañcasikhaṃ gandhabbadevaputtaṃ āmantesi – ‘‘abhivādehi me tvaṃ, tāta pañcasikha, bhagavantaṃ – ‘sakko, bhante, devānamindo sāmacco saparijano bhagavato pāde sirasā vandatī’ti’’. ‘‘Evaṃ bhaddantavā’’ti kho pañcasikho gandhabbadevaputto sakkassa devānamindassa paṭissutvā bhagavantaṃ abhivādeti – ‘‘sakko, bhante, devānamindo sāmacco saparijano bhagavato pāde sirasā vandatī’’ti. ‘‘Evaṃ sukhī hotu, pañcasikha, sakko devānamindo sāmacco saparijano; sukhakāmā hi devā manussā asurā nāgā gandhabbā ye caññe santi puthukāyā’’ti.

\paragraph{351.} Evañca pana tathāgatā evarūpe mahesakkhe yakkhe abhivadanti. Abhivadito sakko devānamindo bhagavato indasālaguhaṃ pavisitvā bhagavantaṃ abhivādetvā ekamantaṃ aṭṭhāsi. Devāpi tāvatiṃsā indasālaguhaṃ pavisitvā bhagavantaṃ abhivādetvā ekamantaṃ aṭṭhaṃsu. Pañcasikhopi gandhabbadevaputto indasālaguhaṃ pavisitvā bhagavantaṃ abhivādetvā ekamantaṃ aṭṭhāsi.

Tena kho pana samayena indasālaguhā visamā santī samā samapādi, sambādhā santī urundā\footnote{uruddā (ka.)} samapādi, andhakāro guhāyaṃ antaradhāyi, āloko udapādi yathā taṃ devānaṃ devānubhāvena.

\paragraph{352.} Atha kho bhagavā sakkaṃ devānamindaṃ etadavoca – ‘‘acchariyamidaṃ āyasmato kosiyassa, abbhutamidaṃ āyasmato kosiyassa tāva bahukiccassa bahukaraṇīyassa yadidaṃ idhāgamana’’nti. ‘‘Cirapaṭikāhaṃ, bhante, bhagavantaṃ dassanāya upasaṅkamitukāmo; api ca devānaṃ tāvatiṃsānaṃ kehici kehici\footnote{kehici (syā.)} kiccakaraṇīyehi byāvaṭo; evāhaṃ nāsakkhiṃ bhagavantaṃ dassanāya upasaṅkamituṃ. Ekamidaṃ, bhante, samayaṃ bhagavā sāvatthiyaṃ viharati salaḷāgārake. Atha khvāhaṃ, bhante, sāvatthiṃ agamāsiṃ bhagavantaṃ dassanāya. Tena kho pana, bhante, samayena bhagavā aññatarena samādhinā nisinno hoti, bhūjati\footnote{bhuñjatī ca (sī. pī.), bhujagī (syā.)} ca nāma vessavaṇassa mahārājassa paricārikā bhagavantaṃ paccupaṭṭhitā hoti, pañjalikā namassamānā tiṭṭhati. Atha khvāhaṃ, bhante, bhūjatiṃ etadavocaṃ – ‘abhivādehi me tvaṃ, bhagini, bhagavantaṃ – ‘‘sakko, bhante, devānamindo sāmacco saparijano bhagavato pāde sirasā vandatī’’ti. Evaṃ vutte, bhante, sā bhūjati maṃ etadavoca – ‘akālo kho, mārisa, bhagavantaṃ dassanāya; paṭisallīno bhagavā’ti. ‘Tena hī, bhagini, yadā bhagavā tamhā samādhimhā vuṭṭhito hoti, atha mama vacanena bhagavantaṃ abhivādehi – ‘‘sakko, bhante, devānamindo sāmacco saparijano bhagavato pāde sirasā vandatī’’ti. Kacci me sā, bhante, bhaginī bhagavantaṃ abhivādesi? Sarati bhagavā tassā bhaginiyā vacana’’nti? ‘‘Abhivādesi maṃ sā, devānaminda, bhaginī, sarāmahaṃ tassā bhaginiyā vacanaṃ. Api cāhaṃ āyasmato nemisaddena\footnote{cakkanemisaddena (syā.)} tamhā samādhimhā vuṭṭhito’’ti. ‘‘Ye te, bhante, devā amhehi paṭhamataraṃ tāvatiṃsakāyaṃ upapannā, tesaṃ me sammukhā sutaṃ sammukhā paṭiggahitaṃ – ‘yadā tathāgatā loke uppajjanti arahanto sammāsambuddhā, dibbā kāyā paripūrenti, hāyanti asurakāyā’ti. Taṃ me idaṃ, bhante, sakkhidiṭṭhaṃ yato tathāgato loke uppanno arahaṃ sammāsambuddho, dibbā kāyā paripūrenti, hāyanti asurakāyāti.

\subsubsection{Gopakavatthu}

\paragraph{353.} ‘‘Idheva, bhante, kapilavatthusmiṃ gopikā nāma sakyadhītā ahosi buddhe pasannā dhamme pasannā saṅghe pasannā sīlesu paripūrakārinī. Sā itthittaṃ\footnote{itthicittaṃ (syā.)} virājetvā purisattaṃ\footnote{purisacittaṃ (syā.)} bhāvetvā kāyassa bhedā paraṃ maraṇā sugatiṃ saggaṃ lokaṃ upapannā. Devānaṃ tāvatiṃsānaṃ sahabyataṃ amhākaṃ puttattaṃ ajjhupagatā. Tatrapi naṃ evaṃ jānanti – ‘gopako devaputto, gopako devaputto’ti. Aññepi, bhante, tayo bhikkhū bhagavati brahmacariyaṃ caritvā hīnaṃ gandhabbakāyaṃ upapannā. Te pañcahi kāmaguṇehi samappitā samaṅgībhūtā paricārayamānā amhākaṃ upaṭṭhānaṃ āgacchanti amhākaṃ pāricariyaṃ. Te amhākaṃ upaṭṭhānaṃ āgate amhākaṃ pāricariyaṃ gopako devaputto paṭicodesi – ‘kutomukhā nāma tumhe , mārisā, tassa bhagavato dhammaṃ assuttha\footnote{āyuhittha (syā.)} – ahañhi nāma itthikā samānā buddhe pasannā dhamme pasannā saṅghe pasannā sīlesu paripūrakārinī itthittaṃ virājetvā purisattaṃ bhāvetvā kāyassa bhedā paraṃ maraṇā sugatiṃ saggaṃ lokaṃ upapannā, devānaṃ tāvatiṃsānaṃ sahabyataṃ sakkassa devānamindassa puttattaṃ ajjhupagatā. Idhāpi maṃ evaṃ jānanti ‘‘gopako devaputto gopako devaputto’ti. Tumhe pana, mārisā, bhagavati brahmacariyaṃ caritvā hīnaṃ gandhabbakāyaṃ upapannā. Duddiṭṭharūpaṃ vata, bho, addasāma, ye mayaṃ addasāma sahadhammike hīnaṃ gandhabbakāyaṃ upapanne’ti. Tesaṃ, bhante, gopakena devaputtena paṭicoditānaṃ dve devā diṭṭheva dhamme satiṃ paṭilabhiṃsu kāyaṃ brahmapurohitaṃ, eko pana devo kāme ajjhāvasi.

\paragraph{354.}‘‘‘Upāsikā cakkhumato ahosiṃ,

Nāmampi mayhaṃ ahu ‘gopikā’ti;

Buddhe ca dhamme ca abhippasannā,

Saṅghañcupaṭṭhāsiṃ pasannacittā.

‘‘‘Tasseva buddhassa sudhammatāya,

Sakkassa puttomhi mahānubhāvo;

Mahājutīko tidivūpapanno,

Jānanti maṃ idhāpi ‘gopako’ti.

‘‘‘Athaddasaṃ bhikkhavo diṭṭhapubbe,

Gandhabbakāyūpagate vasīne;

Imehi te gotamasāvakāse,

Ye ca mayaṃ pubbe manussabhūtā.

‘‘‘Annena pānena upaṭṭhahimhā,

Pādūpasaṅgayha sake nivesane;

Kutomukhā nāma ime bhavanto,

Buddhassa dhammāni paṭiggahesuṃ\footnote{buddhassa dhammaṃ na paṭiggahesuṃ (syā.)}.

‘‘‘Paccattaṃ veditabbo hi dhammo,

Sudesito cakkhumatānubuddho;

Ahañhi tumheva upāsamāno,

Sutvāna ariyāna subhāsitāni.

‘‘‘Sakkassa puttomhi mahānubhāvo,

Mahājutīko tidivūpapanno;

Tumhe pana seṭṭhamupāsamānā,

Anuttaraṃ brahmacariyaṃ caritvā.

‘‘‘Hīnaṃ kāyaṃ upapannā bhavanto,

Anānulomā bhavatūpapatti;

Duddiṭṭharūpaṃ vata addasāma,

Sahadhammike hīnakāyūpapanne.

‘‘‘Gandhabbakāyūpagatā bhavanto,

Devānamāgacchatha pāricariyaṃ;

Agāre vasato mayhaṃ,

Imaṃ passa visesataṃ.

‘‘‘Itthī hutvā svajja pumomhi devo,

Dibbehi kāmehi samaṅgibhūto’;

Te coditā gotamasāvakena,

Saṃvegamāpādu samecca gopakaṃ.

‘‘‘Handa viyāyāma\footnote{vigāyāma (syā.), vitāyāma (pī.)} byāyāma\footnote{viyāyamāma (sī. pī.)},

Mā no mayaṃ parapessā ahumhā’;

Tesaṃ duve vīriyamārabhiṃsu,

Anussaraṃ gotamasāsanāni.

‘‘Idheva cittāni virājayitvā,

Kāmesu ādīnavamaddasaṃsu;

Te kāmasaṃyojanabandhanāni,

Pāpimayogāni duraccayāni.

‘‘Nāgova sannāni guṇāni\footnote{sandānaguṇāni (sī. pī.), santāni guṇāni (syā.)} chetvā,

Deve tāvatiṃse atikkamiṃsu;

Saindā devā sapajāpatikā,

Sabbe sudhammāya sabhāyupaviṭṭhā.

‘‘Tesaṃ nisinnānaṃ abhikkamiṃsu,

Vīrā virāgā virajaṃ karontā;

Te disvā saṃvegamakāsi vāsavo,

Devābhibhū devagaṇassa majjhe.

‘‘‘Imehi te hīnakāyūpapannā,

Deve tāvatiṃse abhikkamanti’;

Saṃvegajātassa vaco nisamma,

So gopako vāsavamajjhabhāsi.

‘‘‘Buddho janindatthi manussaloke,

Kāmābhibhū sakyamunīti ñāyati;

Tasseva te puttā satiyā vihīnā,

Coditā mayā te satimajjhalatthuṃ.

‘‘‘Tiṇṇaṃ tesaṃ āvasinettha\footnote{avasīnettha (pī.)} eko,

Gandhabbakāyūpagato vasīno;

Dve ca sambodhipathānusārino,

Devepi hīḷenti samāhitattā.

‘‘‘Etādisī dhammappakāsanettha,

Na tattha kiṃkaṅkhati koci sāvako;

Nitiṇṇaoghaṃ vicikicchachinnaṃ,

Buddhaṃ namassāma jinaṃ janindaṃ’.

‘‘Yaṃ te dhammaṃ idhaññāya,

Visesaṃ ajjhagaṃsu\footnote{ajjhagamaṃsu (syā.)} te;

Kāyaṃ brahmapurohitaṃ,

Duve tesaṃ visesagū.

‘‘Tassa dhammassa pattiyā,

Āgatamhāsi mārisa;

Katāvakāsā bhagavatā,

Pañhaṃ pucchemu mārisā’’ti.

\paragraph{355.} Atha kho bhagavato etadahosi – ‘‘dīgharattaṃ visuddho kho ayaṃ yakkho\footnote{sakko (sī. syā. pī.)}, yaṃ kiñci maṃ pañhaṃ pucchissati, sabbaṃ taṃ atthasañhitaṃyeva pucchissati, no anatthasañhitaṃ. Yañcassāhaṃ puṭṭho byākarissāmi, taṃ khippameva ājānissatī’’ti.

\paragraph{356.} Atha kho bhagavā sakkaṃ devānamindaṃ gāthāya ajjhabhāsi –

‘‘Puccha vāsava maṃ pañhaṃ, yaṃ kiñci manasicchasi;

Tassa tasseva pañhassa, ahaṃ antaṃ karomi te’’ti.

\subsubsection{Paṭhamabhāṇavāro niṭṭhito.}

\paragraph{357.} Katāvakāso sakko devānamindo bhagavatā imaṃ bhagavantaṃ\footnote{devānamindo bhagavantaṃ imaṃ (sī. pī.)} paṭhamaṃ pañhaṃ apucchi –

‘‘Kiṃ saṃyojanā nu kho, mārisa, devā manussā asurā nāgā gandhabbā ye caññe santi puthukāyā, te – ‘averā adaṇḍā asapattā abyāpajjā viharemu averino’ti iti ca nesaṃ hoti, atha ca pana saverā sadaṇḍā sasapattā sabyāpajjā viharanti saverino’’ti? Itthaṃ sakko devānamindo bhagavantaṃ pañhaṃ\footnote{imaṃ paṭhamaṃ pañhaṃ (sī. pī.)} apucchi. Tassa bhagavā pañhaṃ puṭṭho byākāsi –

‘‘Issāmacchariyasaṃyojanā kho, devānaminda, devā manussā asurā nāgā gandhabbā ye caññe santi puthukāyā, te – ‘averā adaṇḍā asapattā abyāpajjā viharemu averino’ti iti ca nesaṃ hoti, atha ca pana saverā sadaṇḍā sasapattā sabyāpajjā viharanti saverino’’ti. Itthaṃ bhagavā sakkassa devānamindassa pañhaṃ puṭṭho byākāsi. Attamano sakko devānamindo bhagavato bhāsitaṃ abhinandi anumodi – ‘‘evametaṃ, bhagavā, evametaṃ, sugata. Tiṇṇā mettha kaṅkhā vigatā kathaṃkathā bhagavato pañhaveyyākaraṇaṃ sutvā’’ti.

\paragraph{358.} Itiha sakko devānamindo bhagavato bhāsitaṃ abhinanditvā anumoditvā bhagavantaṃ uttariṃ\footnote{uttariṃ (sī. syā. pī.)} pañhaṃ apucchi –

‘‘Issāmacchariyaṃ pana, mārisa, kiṃnidānaṃ kiṃsamudayaṃ kiṃjātikaṃ kiṃpabhavaṃ; kismiṃ sati issāmacchariyaṃ hoti; kismiṃ asati issāmacchariyaṃ na hotī’’ti? ‘‘Issāmacchariyaṃ kho, devānaminda, piyāppiyanidānaṃ piyāppiyasamudayaṃ piyāppiyajātikaṃ piyāppiyapabhavaṃ; piyāppiye sati issāmacchariyaṃ hoti, piyāppiye asati issāmacchariyaṃ na hotī’’ti.

‘‘Piyāppiyaṃ kho pana, mārisa, kiṃnidānaṃ kiṃsamudayaṃ kiṃjātikaṃ kiṃpabhavaṃ; kismiṃ sati piyāppiyaṃ hoti; kismiṃ asati piyāppiyaṃ na hotī’’ti? ‘‘Piyāppiyaṃ kho, devānaminda, chandanidānaṃ chandasamudayaṃ chandajātikaṃ chandapabhavaṃ; chande sati piyāppiyaṃ hoti; chande asati piyāppiyaṃ na hotī’’ti.

‘‘Chando kho pana, mārisa, kiṃnidāno kiṃsamudayo kiṃjātiko kiṃpabhavo; kismiṃ sati chando hoti; kismiṃ asati chando na hotī’’ti? ‘‘Chando kho, devānaminda, vitakkanidāno vitakkasamudayo vitakkajātiko vitakkapabhavo; vitakke sati chando hoti; vitakke asati chando na hotī’’ti.

‘‘Vitakko kho pana, mārisa, kiṃnidāno kiṃsamudayo kiṃjātiko kiṃpabhavo; kismiṃ sati vitakko hoti; kismiṃ asati vitakko na hotī’’ti? ‘‘Vitakko kho, devānaminda, papañcasaññāsaṅkhānidāno papañcasaññāsaṅkhāsamudayo papañcasaññāsaṅkhājātiko papañcasaññāsaṅkhāpabhavo; papañcasaññāsaṅkhāya sati vitakko hoti; papañcasaññāsaṅkhāya asati vitakko na hotī’’ti.

‘‘Kathaṃ paṭipanno pana, mārisa, bhikkhu papañcasaññāsaṅkhānirodhasāruppagāminiṃ paṭipadaṃ paṭipanno hotī’’ti?

\subsubsection{Vedanākammaṭṭhānaṃ}

\paragraph{359.} ‘‘Somanassaṃpāhaṃ\footnote{pahaṃ (sī. pī.), cāhaṃ (syā. kaṃ.)}, devānaminda, duvidhena vadāmi – sevitabbampi, asevitabbampi. Domanassaṃpāhaṃ, devānaminda, duvidhena vadāmi – sevitabbampi, asevitabbampi. Upekkhaṃpāhaṃ, devānaminda, duvidhena vadāmi – sevitabbampi, asevitabbampi.

\paragraph{360.} ‘‘Somanassaṃpāhaṃ, devānaminda, duvidhena vadāmi sevitabbampi, asevitabbampīti iti kho panetaṃ vuttaṃ, kiñcetaṃ paṭicca vuttaṃ? Tattha yaṃ jaññā somanassaṃ ‘imaṃ kho me somanassaṃ sevato akusalā dhammā abhivaḍḍhanti, kusalā dhammā parihāyantī’ti, evarūpaṃ somanassaṃ na sevitabbaṃ. Tattha yaṃ jaññā somanassaṃ ‘imaṃ kho me somanassaṃ sevato akusalā dhammā parihāyanti, kusalā dhammā abhivaḍḍhantī’ti, evarūpaṃ somanassaṃ sevitabbaṃ. Tattha yaṃ ce savitakkaṃ savicāraṃ, yaṃ ce avitakkaṃ avicāraṃ, ye avitakke avicāre, te\footnote{se (sī. pī.)} paṇītatare. Somanassaṃpāhaṃ, devānaminda, duvidhena vadāmi sevitabbampi, asevitabbampīti. Iti yaṃ taṃ vuttaṃ, idametaṃ paṭicca vuttaṃ.

\paragraph{361.} ‘‘Domanassaṃpāhaṃ, devānaminda, duvidhena vadāmi sevitabbampi , asevitabbampīti. Iti kho panetaṃ vuttaṃ, kiñcetaṃ paṭicca vuttaṃ? Tattha yaṃ jaññā domanassaṃ ‘imaṃ kho me domanassaṃ sevato akusalā dhammā abhivaḍḍhanti, kusalā dhammā parihāyantī’ti, evarūpaṃ domanassaṃ na sevitabbaṃ. Tattha yaṃ jaññā domanassaṃ ‘imaṃ kho me domanassaṃ sevato akusalā dhammā parihāyanti, kusalā dhammā abhivaḍḍhantī’ti, evarūpaṃ domanassaṃ sevitabbaṃ. Tattha yaṃ ce savitakkaṃ savicāraṃ, yaṃ ce avitakkaṃ avicāraṃ, ye avitakke avicāre, te paṇītatare. Domanassaṃpāhaṃ, devānaminda, duvidhena vadāmi sevitabbampi, asevitabbampī’ti iti yaṃ taṃ vuttaṃ, idametaṃ paṭicca vuttaṃ.

\paragraph{362.} ‘‘Upekkhaṃpāhaṃ, devānaminda, duvidhena vadāmi sevitabbampi, asevitabbampīti iti kho panetaṃ vuttaṃ, kiñcetaṃ paṭicca vuttaṃ? Tattha yaṃ jaññā upekkhaṃ ‘imaṃ kho me upekkhaṃ sevato akusalā dhammā abhivaḍḍhanti, kusalā dhammā parihāyantī’ti, evarūpā upekkhā na sevitabbā. Tattha yaṃ jaññā upekkhaṃ ‘imaṃ kho me upekkhaṃ sevato akusalā dhammā parihāyanti, kusalā dhammā abhivaḍḍhantī’ti, evarūpā upekkhā sevitabbā. Tattha yaṃ ce savitakkaṃ savicāraṃ, yaṃ ce avitakkaṃ avicāraṃ, ye avitakke avicāre, te paṇītatare. Upekkhaṃpāhaṃ, devānaminda, duvidhena vadāmi sevitabbampi, asevitabbampīti iti yaṃ taṃ vuttaṃ, idametaṃ paṭicca vuttaṃ.

\paragraph{363.} ‘‘Evaṃ paṭipanno kho, devānaminda, bhikkhu papañcasaññāsaṅkhānirodhasāruppagāminiṃ paṭipadaṃ paṭipanno hotī’’ti. Itthaṃ bhagavā sakkassa devānamindassa pañhaṃ puṭṭho byākāsi. Attamano sakko devānamindo bhagavato bhāsitaṃ abhinandi anumodi – ‘‘evametaṃ, bhagavā, evametaṃ, sugata, tiṇṇā mettha kaṅkhā vigatā kathaṃkathā bhagavato pañhaveyyākaraṇaṃ sutvā’’ti.

\subsubsection{Pātimokkhasaṃvaro}

\paragraph{364.} Itiha sakko devānamindo bhagavato bhāsitaṃ abhinanditvā anumoditvā bhagavantaṃ uttariṃ pañhaṃ apucchi –

‘‘Kathaṃ paṭipanno pana, mārisa, bhikkhu pātimokkhasaṃvarāya paṭipanno hotī’’ti? ‘‘Kāyasamācāraṃpāhaṃ, devānaminda, duvidhena vadāmi – sevitabbampi, asevitabbampi. Vacīsamācāraṃpāhaṃ, devānaminda, duvidhena vadāmi – sevitabbampi, asevitabbampi. Pariyesanaṃpāhaṃ, devānaminda, duvidhena vadāmi – sevitabbampi, asevitabba’’mpi.

‘‘Kāyasamācāraṃpāhaṃ , devānaminda, duvidhena vadāmi sevitabbampi asevitabbampīti iti kho panetaṃ vuttaṃ, kiñcetaṃ paṭicca vuttaṃ? Tattha yaṃ jaññā kāyasamācāraṃ ‘imaṃ kho me kāyasamācāraṃ sevato akusalā dhammā abhivaḍḍhanti, kusalā dhammā parihāyantī’ti, evarūpo kāyasamācāro na sevitabbo. Tattha yaṃ jaññā kāyasamācāraṃ ‘imaṃ kho me kāyasamācāraṃ sevato akusalā dhammā parihāyanti, kusalā dhammā abhivaḍḍhantī’ti, evarūpo kāyasamācāro sevitabbo. Kāyasamācāraṃpāhaṃ, devānaminda, duvidhena vadāmi – sevitabbampi, asevitabbampīti iti yaṃ taṃ vuttaṃ, idametaṃ paṭicca vuttaṃ.

‘‘Vacīsamācāraṃpāhaṃ , devānaminda, duvidhena vadāmi – sevitabbampi, asevitabbampī’ti. Iti kho panetaṃ vuttaṃ, kiñcetaṃ paṭicca vuttaṃ? Tattha yaṃ jaññā vacīsamācāraṃ ‘imaṃ kho me vacīsamācāraṃ sevato akusalā dhammā abhivaḍḍhanti, kusalā dhammā parihāyantī’ti, evarūpo vacīsamācāro na sevitabbo. Tattha yaṃ jaññā vacīsamācāraṃ ‘imaṃ kho me vacīsamācāraṃ sevato akusalā dhammā parihāyanti, kusalā dhammā abhivaḍḍhantī’ti, evarūpo vacīsamācāro sevitabbo. Vacīsamācāraṃpāhaṃ, devānaminda, duvidhena vadāmi – sevitabbampi, asevitabbampīti iti yaṃ taṃ vuttaṃ, idametaṃ paṭicca vuttaṃ.

‘‘Pariyesanaṃpāhaṃ , devānaminda, duvidhena vadāmi – sevitabbampi, asevitabbampīti iti kho panetaṃ vuttaṃ, kiñcetaṃ paṭicca vuttaṃ? Tattha yaṃ jaññā pariyesanaṃ ‘imaṃ kho me pariyesanaṃ sevato akusalā dhammā abhivaḍḍhanti, kusalā dhammā parihāyantī’ti, evarūpā pariyesanā na sevitabbā. Tattha yaṃ jaññā pariyesanaṃ ‘imaṃ kho me pariyesanaṃ sevato akusalā dhammā parihāyanti, kusalā dhammā abhivaḍḍhantī’ti, evarūpā pariyesanā sevitabbā. Pariyesanaṃpāhaṃ, devānaminda, duvidhena vadāmi – sevitabbampi, asevitabbampīti iti yaṃ taṃ vuttaṃ, idametaṃ paṭicca vuttaṃ.

‘‘Evaṃ paṭipanno kho, devānaminda, bhikkhu pātimokkhasaṃvarāya paṭipanno hotī’’ti. Itthaṃ bhagavā sakkassa devānamindassa pañhaṃ puṭṭho byākāsi. Attamano sakko devānamindo bhagavato bhāsitaṃ abhinandi anumodi – ‘‘evametaṃ, bhagavā, evametaṃ, sugata. Tiṇṇā mettha kaṅkhā vigatā kathaṃkathā bhagavato pañhaveyyākaraṇaṃ sutvā’’ti.

\subsubsection{Indriyasaṃvaro}

\paragraph{365.} Itiha sakko devānamindo bhagavato bhāsitaṃ abhinanditvā anumoditvā bhagavantaṃ uttariṃ pañhaṃ apucchi –

‘‘Kathaṃ paṭipanno pana, mārisa, bhikkhu indriyasaṃvarāya paṭipanno hotī’’ti? ‘‘Cakkhuviññeyyaṃ rūpaṃpāhaṃ, devānaminda, duvidhena vadāmi – sevitabbampi, asevitabbampi. Sotaviññeyyaṃ saddaṃpāhaṃ, devānaminda, duvidhena vadāmi – sevitabbampi, asevitabbampi. Ghānaviññeyyaṃ gandhaṃpāhaṃ, devānaminda, duvidhena vadāmi – sevitabbampi, asevitabbampi. Jivhāviññeyyaṃ rasaṃpāhaṃ, devānaminda, duvidhena vadāmi – sevitabbampi, asevitabbampi. Kāyaviññeyyaṃ phoṭṭhabbaṃpāhaṃ, devānaminda, duvidhena vadāmi – sevitabbampi, asevitabbampi. Manoviññeyyaṃ dhammaṃpāhaṃ, devānaminda, duvidhena vadāmi – sevitabbampi, asevitabbampī’’ti.

Evaṃ vutte, sakko devānamindo bhagavantaṃ etadavoca –

‘‘Imassa kho ahaṃ, bhante, bhagavatā saṅkhittena bhāsitassa evaṃ vitthārena atthaṃ ājānāmi. Yathārūpaṃ, bhante, cakkhuviññeyyaṃ rūpaṃ sevato akusalā dhammā abhivaḍḍhanti, kusalā dhammā parihāyanti, evarūpaṃ cakkhuviññeyyaṃ rūpaṃ na sevitabbaṃ . Yathārūpañca kho, bhante, cakkhuviññeyyaṃ rūpaṃ sevato akusalā dhammā parihāyanti, kusalā dhammā abhivaḍḍhanti, evarūpaṃ cakkhuviññeyyaṃ rūpaṃ sevitabbaṃ. Yathārūpañca kho, bhante, sotaviññeyyaṃ saddaṃ sevato…pe… ghānaviññeyyaṃ gandhaṃ sevato… jivhāviññeyyaṃ rasaṃ sevato… kāyaviññeyyaṃ phoṭṭhabbaṃ sevato… manoviññeyyaṃ dhammaṃ sevato akusalā dhammā abhivaḍḍhanti, kusalā dhammā parihāyanti, evarūpo manoviññeyyo dhammo na sevitabbo. Yathārūpañca kho, bhante, manoviññeyyaṃ dhammaṃ sevato akusalā dhammā parihāyanti, kusalā dhammā abhivaḍḍhanti, evarūpo manoviññeyyo dhammo sevitabbo.

‘‘Imassa kho me, bhante, bhagavatā saṅkhittena bhāsitassa evaṃ vitthārena atthaṃ ājānato tiṇṇā mettha kaṅkhā vigatā kathaṃkathā bhagavato pañhaveyyākaraṇaṃ sutvā’’ti.

\paragraph{366.} Itiha sakko devānamindo bhagavato bhāsitaṃ abhinanditvā anumoditvā bhagavantaṃ uttariṃ pañhaṃ apucchi –

‘‘Sabbeva nu kho, mārisa, samaṇabrāhmaṇā ekantavādā ekantasīlā ekantachandā ekantaajjhosānā’’ti? ‘‘Na kho, devānaminda, sabbe samaṇabrāhmaṇā ekantavādā ekantasīlā ekantachandā ekantaajjhosānā’’ti.

‘‘Kasmā pana, mārisa, na sabbe samaṇabrāhmaṇā ekantavādā ekantasīlā ekantachandā ekantaajjhosānā’’ti? ‘‘Anekadhātu nānādhātu kho, devānaminda, loko. Tasmiṃ anekadhātunānādhātusmiṃ loke yaṃ yadeva sattā dhātuṃ abhinivisanti, taṃ tadeva thāmasā parāmāsā abhinivissa voharanti – ‘idameva saccaṃ moghamañña’nti. Tasmā na sabbe samaṇabrāhmaṇā ekantavādā ekantasīlā ekantachandā ekantaajjhosānā’’ti.

‘‘Sabbeva nu kho, mārisa, samaṇabrāhmaṇā accantaniṭṭhā accantayogakkhemī accantabrahmacārī accantapariyosānā’’ti? ‘‘Na kho, devānaminda, sabbe samaṇabrāhmaṇā accantaniṭṭhā accantayogakkhemī accantabrahmacārī accantapariyosānā’’ti.

‘‘Kasmā pana, mārisa, na sabbe samaṇabrāhmaṇā accantaniṭṭhā accantayogakkhemī accantabrahmacārī accantapariyosānā’’ti? ‘‘Ye kho, devānaminda, bhikkhū taṇhāsaṅkhayavimuttā te accantaniṭṭhā accantayogakkhemī accantabrahmacārī accantapariyosānā. Tasmā na sabbe samaṇabrāhmaṇā accantaniṭṭhā accantayogakkhemī accantabrahmacārī accantapariyosānā’’ti.

Itthaṃ bhagavā sakkassa devānamindassa pañhaṃ puṭṭho byākāsi. Attamano sakko devānamindo bhagavato bhāsitaṃ abhinandi anumodi – ‘‘evametaṃ, bhagavā, evametaṃ, sugata. Tiṇṇā mettha kaṅkhā vigatā kathaṃkathā bhagavato pañhaveyyākaraṇaṃ sutvā’’ti.

\paragraph{367.} Itiha sakko devānamindo bhagavato bhāsitaṃ abhinanditvā anumoditvā bhagavantaṃ etadavoca –

‘‘Ejā, bhante, rogo, ejā gaṇḍo, ejā sallaṃ, ejā imaṃ purisaṃ parikaḍḍhati tassa tasseva bhavassa abhinibbattiyā. Tasmā ayaṃ puriso uccāvacamāpajjati . Yesāhaṃ, bhante, pañhānaṃ ito bahiddhā aññesu samaṇabrāhmaṇesu okāsakammampi nālatthaṃ, te me bhagavatā byākatā. Dīgharattānusayitañca pana\footnote{dīgharattānupassatā, yañca pana (syā.), dīgharattānusayino, yañca pana (sī. pī.)} me vicikicchākathaṃkathāsallaṃ, tañca bhagavatā abbuḷha’’nti.

‘‘Abhijānāsi no tvaṃ, devānaminda, ime pañhe aññe samaṇabrāhmaṇe pucchitā’’ti? ‘‘Abhijānāmahaṃ, bhante, ime pañhe aññe samaṇabrāhmaṇe pucchitā’’ti. ‘‘Yathā kathaṃ pana te, devānaminda, byākaṃsu? Sace te agaru bhāsassū’’ti. ‘‘Na kho me, bhante, garu yatthassa bhagavā nisinno bhagavantarūpo vā’’ti. ‘‘Tena hi, devānaminda, bhāsassū’’ti. ‘‘Yesvāhaṃ\footnote{yesāhaṃ (sī. syā. pī.)}, bhante , maññāmi samaṇabrāhmaṇā āraññikā pantasenāsanāti, tyāhaṃ upasaṅkamitvā ime pañhe pucchāmi, te mayā puṭṭhā na sampāyanti, asampāyantā mamaṃyeva paṭipucchanti – ‘ko nāmo āyasmā’ti? Tesāhaṃ puṭṭho byākaromi – ‘ahaṃ kho, mārisa, sakko devānamindo’ti. Te mamaṃyeva uttari paṭipucchanti – ‘kiṃ panāyasmā, devānaminda\footnote{devānamindo (sī. pī.)}, kammaṃ katvā imaṃ ṭhānaṃ patto’ti? Tesāhaṃ yathāsutaṃ yathāpariyattaṃ dhammaṃ desemi. Te tāvatakeneva attamanā honti – ‘sakko ca no devānamindo diṭṭho, yañca no apucchimhā, tañca no byākāsī’ti. Te aññadatthu mamaṃyeva sāvakā sampajjanti, na cāhaṃ tesaṃ. Ahaṃ kho pana, bhante, bhagavato sāvako sotāpanno avinipātadhammo niyato sambodhiparāyaṇo’’ti .

\subsubsection{Somanassapaṭilābhakathā}

\paragraph{368.} ‘‘Abhijānāsi no tvaṃ, devānaminda, ito pubbe evarūpaṃ vedapaṭilābhaṃ somanassapaṭilābha’’nti? ‘‘Abhijānāmahaṃ , bhante, ito pubbe evarūpaṃ vedapaṭilābhaṃ somanassapaṭilābha’’nti. ‘‘Yathā kathaṃ pana tvaṃ, devānaminda, abhijānāsi ito pubbe evarūpaṃ vedapaṭilābhaṃ somanassapaṭilābha’’nti?

‘‘Bhūtapubbaṃ, bhante, devāsurasaṅgāmo samupabyūḷho\footnote{samūpabbuḷho (sī. pī.)} ahosi. Tasmiṃ kho pana, bhante, saṅgāme devā jiniṃsu, asurā parājayiṃsu\footnote{parājiṃsu (sī. pī.)}. Tassa mayhaṃ, bhante, taṃ saṅgāmaṃ abhivijinitvā vijitasaṅgāmassa etadahosi – ‘yā ceva dāni dibbā ojā yā ca asurā ojā, ubhayametaṃ\footnote{ubhayamettha (syā.)} devā paribhuñjissantī’ti. So kho pana me, bhante, vedapaṭilābho somanassapaṭilābho sadaṇḍāvacaro sasatthāvacaro na nibbidāya na virāgāya na nirodhāya na upasamāya na abhiññāya na sambodhāya na nibbānāya saṃvattati. Yo kho pana me ayaṃ, bhante, bhagavato dhammaṃ sutvā vedapaṭilābho somanassapaṭilābho, so adaṇḍāvacaro asatthāvacaro ekantanibbidāya virāgāya nirodhāya upasamāya abhiññāya sambodhāya nibbānāya saṃvattatī’’ti.

\paragraph{369.} ‘‘Kiṃ pana tvaṃ, devānaminda, atthavasaṃ sampassamāno evarūpaṃ vedapaṭilābhaṃ somanassapaṭilābhaṃ pavedesī’’ti? ‘‘Cha kho ahaṃ, bhante, atthavase sampassamāno evarūpaṃ vedapaṭilābhaṃ somanassapaṭilābhaṃ pavedemi.

‘‘Idheva tiṭṭhamānassa, devabhūtassa me sato;

Punarāyu ca me laddho, evaṃ jānāhi mārisa.

‘‘Imaṃ kho ahaṃ, bhante, paṭhamaṃ atthavasaṃ sampassamāno evarūpaṃ vedapaṭilābhaṃ somanassapaṭilābhaṃ pavedemi.

‘‘Cutāhaṃ diviyā kāyā, āyuṃ hitvā amānusaṃ;

Amūḷho gabbhamessāmi, yattha me ramatī mano.

‘‘Imaṃ kho ahaṃ, bhante, dutiyaṃ atthavasaṃ sampassamāno evarūpaṃ vedapaṭilābhaṃ somanassapaṭilābhaṃ pavedemi.

‘‘Svāhaṃ amūḷhapaññassa\footnote{amūḷhapañhassa (?)}, viharaṃ sāsane rato;

Ñāyena viharissāmi, sampajāno paṭissato.

‘‘Imaṃ kho ahaṃ, bhante, tatiyaṃ atthavasaṃ sampassamāno evarūpaṃ vedapaṭilābhaṃ somanassapaṭilābhaṃ pavedemi.

‘‘Ñāyena me carato ca, sambodhi ce bhavissati;

Aññātā viharissāmi, sveva anto bhavissati.

‘‘Imaṃ kho ahaṃ, bhante, catutthaṃ atthavasaṃ sampassamāno evarūpaṃ vedapaṭilābhaṃ somanassapaṭilābhaṃ pavedemi.

‘‘Cutāhaṃ mānusā kāyā, āyuṃ hitvāna mānusaṃ;

Puna devo bhavissāmi, devalokamhi uttamo.

‘‘Imaṃ kho ahaṃ, bhante, pañcamaṃ atthavasaṃ sampassamāno evarūpaṃ vedapaṭilābhaṃ somanassapaṭilābhaṃ pavedemi.

‘‘Te\footnote{ye (?)} paṇītatarā devā, akaniṭṭhā yasassino;

Antime vattamānamhi, so nivāso bhavissati.

‘‘Imaṃ kho ahaṃ, bhante, chaṭṭhaṃ atthavasaṃ sampassamāno evarūpaṃ vedapaṭilābhaṃ somanassapaṭilābhaṃ pavedemi.

‘‘Ime kho ahaṃ, bhante, cha atthavase sampassamāno evarūpaṃ vedapaṭilābhaṃ somanassapaṭilābhaṃ pavedemi.

\paragraph{370.}‘‘Apariyositasaṅkappo , vicikiccho kathaṃkathī.

Vicariṃ dīghamaddhānaṃ, anvesanto tathāgataṃ.

‘‘Yassu maññāmi samaṇe, pavivittavihārino;

Sambuddhā iti maññāno, gacchāmi te upāsituṃ.

‘‘‘Kathaṃ ārādhanā hoti, kathaṃ hoti virādhanā’;

Iti puṭṭhā na sampāyanti\footnote{sambhonti (syā.)}, magge paṭipadāsu ca.

‘‘Tyassu yadā maṃ jānanti, sakko devānamāgato;

Tyassu mameva pucchanti, ‘kiṃ katvā pāpuṇī idaṃ’.

‘‘Tesaṃ yathāsutaṃ dhammaṃ, desayāmi jane sutaṃ\footnote{janesuta (ka. sī.)};

Tena attamanā honti, ‘diṭṭho no vāsavoti ca’.

‘‘Yadā ca buddhamaddakkhiṃ, vicikicchāvitāraṇaṃ;

Somhi vītabhayo ajja, sambuddhaṃ payirupāsiya\footnote{payirupāsayiṃ (syā. ka.)}.

‘‘Taṇhāsallassa hantāraṃ, buddhaṃ appaṭipuggalaṃ;

Ahaṃ vande mahāvīraṃ, buddhamādiccabandhunaṃ.

‘‘Yaṃ karomasi brahmuno, samaṃ devehi mārisa;

Tadajja tuyhaṃ kassāma\footnote{dassāma (syā. ka.)}, handa sāmaṃ karoma te.

‘‘Tvameva asi\footnote{tuvamevasi (pī.)} sambuddho, tuvaṃ satthā anuttaro;

Sadevakasmiṃ lokasmiṃ, natthi te paṭipuggalo’’ti.

\paragraph{371.} Atha kho sakko devānamindo pañcasikhaṃ gandhabbaputtaṃ āmantesi – ‘‘bahūpakāro kho mesi tvaṃ, tāta pañcasikha, yaṃ tvaṃ bhagavantaṃ paṭhamaṃ pasādesi. Tayā, tāta, paṭhamaṃ pasāditaṃ pacchā mayaṃ taṃ bhagavantaṃ dassanāya upasaṅkamimhā arahantaṃ sammāsambuddhaṃ. Pettike vā ṭhāne ṭhapayissāmi , gandhabbarājā bhavissasi, bhaddañca te sūriyavacchasaṃ dammi, sā hi te abhipatthitā’’ti.

Atha kho sakko devānamindo pāṇinā pathaviṃ parāmasitvā tikkhattuṃ udānaṃ udānesi – ‘‘namo tassa bhagavato arahato sammāsambuddhassā’’ti.

Imasmiñca pana veyyākaraṇasmiṃ bhaññamāne sakkassa devānamindassa virajaṃ vītamalaṃ dhammacakkhuṃ udapādi – ‘‘yaṃ kiñci samudayadhammaṃ, sabbaṃ taṃ nirodhadhamma’’nti. Aññesañca asītiyā devatāsahassānaṃ , iti ye sakkena devānamindena ajjhiṭṭhapañhā puṭṭhā , te bhagavatā byākatā. Tasmā imassa veyyākaraṇassa sakkapañhātveva adhivacananti.

\xsectionEnd{Sakkapañhasuttaṃ niṭṭhitaṃ aṭṭhamaṃ.}


\clearpage
\section{Mahāsatipaṭṭhānasuttaṃ}

\paragraph{372.} Evaṃ me sutaṃ – ekaṃ samayaṃ bhagavā kurūsu viharati kammāsadhammaṃ nāma kurūnaṃ nigamo. Tatra kho bhagavā bhikkhū āmantesi – ‘‘bhikkhavo’’ti. ‘‘Bhaddante’’ti\footnote{bhadanteti (sī. syā. pī.)} te bhikkhū bhagavato paccassosuṃ. Bhagavā etadavoca –

\subsubsection{Uddeso}

\paragraph{373.} ‘‘Ekāyano ayaṃ, bhikkhave, maggo sattānaṃ visuddhiyā, sokaparidevānaṃ samatikkamāya dukkhadomanassānaṃ atthaṅgamāya ñāyassa adhigamāya nibbānassa sacchikiriyāya, yadidaṃ cattāro satipaṭṭhānā.

‘‘Katame cattāro? Idha, bhikkhave, bhikkhu kāye kāyānupassī viharati ātāpī sampajāno satimā vineyya loke abhijjhādomanassaṃ, vedanāsu vedanānupassī viharati ātāpī sampajāno satimā, vineyya loke abhijjhādomanassaṃ, citte cittānupassī viharati ātāpī sampajāno satimā vineyya loke abhijjhādomanassaṃ, dhammesu dhammānupassī viharati ātāpī sampajāno satimā vineyya loke abhijjhādomanassaṃ.

\xsubsubsectionEnd{Uddeso niṭṭhito.}

\subsubsection{Kāyānupassanā ānāpānapabbaṃ}

\paragraph{374.} ‘‘Kathañca pana, bhikkhave, bhikkhu kāye kāyānupassī viharati? Idha, bhikkhave, bhikkhu araññagato vā rukkhamūlagato vā suññāgāragato vā nisīdati pallaṅkaṃ ābhujitvā ujuṃ kāyaṃ paṇidhāya parimukhaṃ satiṃ upaṭṭhapetvā. So satova assasati, satova passasati. Dīghaṃ vā assasanto ‘dīghaṃ assasāmī’ti pajānāti, dīghaṃ vā passasanto ‘dīghaṃ passasāmī’ti pajānāti. Rassaṃ vā assasanto ‘rassaṃ assasāmī’ti pajānāti, rassaṃ vā passasanto ‘rassaṃ passasāmī’ti pajānāti. ‘Sabbakāyapaṭisaṃvedī assasissāmī’ti sikkhati , ‘sabbakāyapaṭisaṃvedī passasissāmī’ti sikkhati. ‘Passambhayaṃ kāyasaṅkhāraṃ assasissāmī’ti sikkhati, ‘passambhayaṃ kāyasaṅkhāraṃ passasissāmī’ti sikkhati.

‘‘Seyyathāpi, bhikkhave, dakkho bhamakāro vā bhamakārantevāsī vā dīghaṃ vā añchanto ‘dīghaṃ añchāmī’ti pajānāti, rassaṃ vā añchanto ‘rassaṃ añchāmī’ti pajānāti evameva kho, bhikkhave, bhikkhu dīghaṃ vā assasanto ‘dīghaṃ assasāmī’ti pajānāti, dīghaṃ vā passasanto ‘dīghaṃ passasāmī’ti pajānāti, rassaṃ vā assasanto ‘rassaṃ assasāmī’ti pajānāti, rassaṃ vā passasanto ‘rassaṃ passasāmī’ti pajānāti. ‘Sabbakāyapaṭisaṃvedī assasissāmī’ti sikkhati, ‘sabbakāyapaṭisaṃvedī passasissāmī’ti sikkhati, ‘passambhayaṃ kāyasaṅkhāraṃ assasissāmī’ti sikkhati, ‘passambhayaṃ kāyasaṅkhāraṃ passasissāmī’ti sikkhati. Iti ajjhattaṃ vā kāye kāyānupassī viharati, bahiddhā vā kāye kāyānupassī viharati, ajjhattabahiddhā vā kāye kāyānupassī viharati. Samudayadhammānupassī vā kāyasmiṃ viharati, vayadhammānupassī vā kāyasmiṃ viharati, samudayavayadhammānupassī vā kāyasmiṃ viharati. ‘Atthi kāyo’ti vā panassa sati paccupaṭṭhitā hoti yāvadeva ñāṇamattāya paṭissatimattāya anissito ca viharati, na ca kiñci loke upādiyati. Evampi kho\footnote{evampi (sī. syā. pī.)}, bhikkhave, bhikkhu kāye kāyānupassī viharati.

\xsubsubsectionEnd{Ānāpānapabbaṃ niṭṭhitaṃ.}

\subsubsection{Kāyānupassanā iriyāpathapabbaṃ}

\paragraph{375.} ‘‘Puna caparaṃ, bhikkhave, bhikkhu gacchanto vā ‘gacchāmī’ti pajānāti, ṭhito vā ‘ṭhitomhī’ti pajānāti, nisinno vā ‘nisinnomhī’ti pajānāti, sayāno vā ‘sayānomhī’ti pajānāti, yathā yathā vā panassa kāyo paṇihito hoti, tathā tathā naṃ pajānāti. Iti ajjhattaṃ vā kāye kāyānupassī viharati, bahiddhā vā kāye kāyānupassī viharati, ajjhattabahiddhā vā kāye kāyānupassī viharati. Samudayadhammānupassī vā kāyasmiṃ viharati, vayadhammānupassī vā kāyasmiṃ viharati, samudayavayadhammānupassī vā kāyasmiṃ viharati. ‘Atthi kāyo’ti vā panassa sati paccupaṭṭhitā hoti yāvadeva ñāṇamattāya paṭissatimattāya anissito ca viharati, na ca kiñci loke upādiyati. Evampi kho, bhikkhave, bhikkhu kāye kāyānupassī viharati.

\xsubsubsectionEnd{Iriyāpathapabbaṃ niṭṭhitaṃ.}

\subsubsection{Kāyānupassanā sampajānapabbaṃ}

\paragraph{376.} ‘‘Puna caparaṃ, bhikkhave, bhikkhu abhikkante paṭikkante sampajānakārī hoti, ālokite vilokite sampajānakārī hoti, samiñjite pasārite sampajānakārī hoti, saṅghāṭipattacīvaradhāraṇe sampajānakārī hoti, asite pīte khāyite sāyite sampajānakārī hoti, uccārapassāvakamme sampajānakārī hoti, gate ṭhite nisinne sutte jāgarite bhāsite tuṇhībhāve sampajānakārī hoti. Iti ajjhattaṃ vā…pe… evampi kho, bhikkhave, bhikkhu kāye kāyānupassī viharati.

\xsubsubsectionEnd{Sampajānapabbaṃ niṭṭhitaṃ.}

\subsubsection{Kāyānupassanā paṭikūlamanasikārapabbaṃ}

\paragraph{377.} ‘‘Puna caparaṃ, bhikkhave, bhikkhu imameva kāyaṃ uddhaṃ pādatalā adho kesamatthakā tacapariyantaṃ pūraṃ nānappakārassa asucino paccavekkhati – ‘atthi imasmiṃ kāye kesā lomā nakhā dantā taco, maṃsaṃ nhāru aṭṭhi aṭṭhimiñjaṃ vakkaṃ, hadayaṃ yakanaṃ kilomakaṃ pihakaṃ papphāsaṃ, antaṃ antaguṇaṃ udariyaṃ karīsaṃ\footnote{karīsaṃ matthaluṅgaṃ (ka.)}, pittaṃ semhaṃ pubbo lohitaṃ sedo medo, assu vasā kheḷo siṅghāṇikā lasikā mutta’nti.

‘‘Seyyathāpi, bhikkhave, ubhatomukhā putoḷi\footnote{mūtoḷī (syā.), mutoli (pī.)} pūrā nānāvihitassa dhaññassa, seyyathidaṃ sālīnaṃ vīhīnaṃ muggānaṃ māsānaṃ tilānaṃ taṇḍulānaṃ. Tamenaṃ cakkhumā puriso muñcitvā paccavekkheyya – ‘ime sālī, ime vīhī ime muggā ime māsā ime tilā ime taṇḍulā’ti. Evameva kho, bhikkhave, bhikkhu imameva kāyaṃ uddhaṃ pādatalā adho kesamatthakā tacapariyantaṃ pūraṃ nānappakārassa asucino paccavekkhati – ‘atthi imasmiṃ kāye kesā lomā…pe… mutta’nti.

Iti ajjhattaṃ vā…pe… evampi kho, bhikkhave, bhikkhu kāye kāyānupassī viharati.

\xsubsubsectionEnd{Paṭikūlamanasikārapabbaṃ niṭṭhitaṃ.}

\subsubsection{Kāyānupassanā dhātumanasikārapabbaṃ}

\paragraph{378.} ‘‘Puna caparaṃ, bhikkhave, bhikkhu imameva kāyaṃ yathāṭhitaṃ yathāpaṇihitaṃ dhātuso paccavekkhati – ‘atthi imasmiṃ kāye pathavīdhātu āpodhātu tejodhātu vāyodhātū’ti.

‘‘Seyyathāpi , bhikkhave, dakkho goghātako vā goghātakantevāsī vā gāviṃ vadhitvā catumahāpathe bilaso vibhajitvā nisinno assa, evameva kho, bhikkhave, bhikkhu imameva kāyaṃ yathāṭhitaṃ yathāpaṇihitaṃ dhātuso paccavekkhati – ‘atthi imasmiṃ kāye pathavīdhātu āpodhātu tejodhātu vāyodhātū’ti.

‘‘Iti ajjhattaṃ vā kāye kāyānupassī viharati…pe… evampi kho, bhikkhave, bhikkhu kāye kāyānupassī viharati.

\xsubsubsectionEnd{Dhātumanasikārapabbaṃ niṭṭhitaṃ.}

\subsubsection{Kāyānupassanā navasivathikapabbaṃ}

\paragraph{379.} ‘‘Puna caparaṃ, bhikkhave, bhikkhu seyyathāpi passeyya sarīraṃ sivathikāya chaḍḍitaṃ ekāhamataṃ vā dvīhamataṃ vā tīhamataṃ vā uddhumātakaṃ vinīlakaṃ vipubbakajātaṃ. So imameva kāyaṃ upasaṃharati – ‘ayampi kho kāyo evaṃdhammo evaṃbhāvī evaṃanatīto’ti.

‘‘Iti ajjhattaṃ vā …pe… evampi kho, bhikkhave, bhikkhu kāye kāyānupassī viharati.

‘‘Puna caparaṃ, bhikkhave, bhikkhu seyyathāpi passeyya sarīraṃ sivathikāya chaḍḍitaṃ kākehi vā khajjamānaṃ kulalehi vā khajjamānaṃ gijjhehi vā khajjamānaṃ kaṅkehi vā khajjamānaṃ sunakhehi vā khajjamānaṃ byagghehi vā khajjamānaṃ dīpīhi vā khajjamānaṃ siṅgālehi vā\footnote{gijjhehi vā khajjamānaṃ, suvānehi vā khajjamānaṃ, sigālehi vā khajjamānaṃ, (syā. pī.)} khajjamānaṃ vividhehi vā pāṇakajātehi khajjamānaṃ. So imameva kāyaṃ upasaṃharati – ‘ayampi kho kāyo evaṃdhammo evaṃbhāvī evaṃanatīto’ti.

‘‘Iti ajjhattaṃ vā…pe… evampi kho, bhikkhave, bhikkhu kāye kāyānupassī viharati.

‘‘Puna caparaṃ, bhikkhave, bhikkhu seyyathāpi passeyya sarīraṃ sivathikāya chaḍḍitaṃ aṭṭhikasaṅkhalikaṃ samaṃsalohitaṃ nhārusambandhaṃ…pe… aṭṭhikasaṅkhalikaṃ nimaṃsalohitamakkhitaṃ nhārusambandhaṃ…pe… aṭṭhikasaṅkhalikaṃ apagatamaṃsalohitaṃ nhārusambandhaṃ…pe… aṭṭhikāni apagatasambandhāni\footnote{apagatanhārusambandhāni (syā.)} disā vidisā vikkhittāni, aññena hatthaṭṭhikaṃ aññena pādaṭṭhikaṃ aññena gopphakaṭṭhikaṃ\footnote{‘‘aññena gopphakaṭṭhika’’nti idaṃ sī. syā. pī. potthakesu natthi} aññena jaṅghaṭṭhikaṃ aññena ūruṭṭhikaṃ aññena kaṭiṭṭhikaṃ\footnote{aññena kaṭaṭṭhikaṃ aññena piṭṭhaṭṭhikaṃ aññena kaṇḍakaṭṭhikaṃ aññena phāsukaṭṭhikaṃ aññena uraṭṭhikaṃ aññena aṃsaṭṭhikaṃ aññena bāhuṭṭhikaṃ (syā.)} aññena phāsukaṭṭhikaṃ aññena piṭṭhiṭṭhikaṃ aññena khandhaṭṭhikaṃ\footnote{aññena kaṭaṭṭhikaṃ aññena piṭṭhaṭṭhikaṃ aññena kaṇḍakaṭṭhikaṃ aññena phāsukaṭṭhikaṃ aññena uraṭṭhikaṃ aññena aṃsaṭṭhikaṃ aññena bāhuṭṭhikaṃ (syā.)} aññena gīvaṭṭhikaṃ aññena hanukaṭṭhikaṃ aññena dantaṭṭhikaṃ aññena sīsakaṭāhaṃ. So imameva kāyaṃ upasaṃharati – ‘ayampi kho kāyo evaṃdhammo evaṃbhāvī evaṃanatīto’ti.

‘‘Iti ajjhattaṃ vā …pe… viharati.

‘‘Puna caparaṃ, bhikkhave, bhikkhu seyyathāpi passeyya sarīraṃ sivathikāya chaḍḍitaṃ aṭṭhikāni setāni saṅkhavaṇṇapaṭibhāgāni…pe… aṭṭhikāni puñjakitāni terovassikāni …pe… aṭṭhikāni pūtīni cuṇṇakajātāni. So imameva kāyaṃ upasaṃharati – ‘ayampi kho kāyo evaṃdhammo evaṃbhāvī evaṃanatīto’ti. Iti ajjhattaṃ vā kāye kāyānupassī viharati, bahiddhā vā kāye kāyānupassī viharati, ajjhattabahiddhā vā kāye kāyānupassī viharati. Samudayadhammānupassī vā kāyasmiṃ viharati, vayadhammānupassī vā kāyasmiṃ viharati, samudayavayadhammānupassī vā kāyasmiṃ viharati. ‘Atthi kāyo’ti vā panassa sati paccupaṭṭhitā hoti yāvadeva ñāṇamattāya paṭissatimattāya anissito ca viharati, na ca kiñci loke upādiyati. Evampi kho, bhikkhave, bhikkhu kāye kāyānupassī viharati.

\xxsubsubsectionEnd{Navasivathikapabbaṃ niṭṭhitaṃ.\\ Cuddasa kāyānupassanā niṭṭhitā.}

\subsubsection{Vedanānupassanā}

\paragraph{380.} ‘‘Kathañca pana, bhikkhave, bhikkhu vedanāsu vedanānupassī viharati? Idha, bhikkhave, bhikkhu sukhaṃ vā vedanaṃ vedayamāno ‘sukhaṃ vedanaṃ vedayāmī’ti pajānāti . Dukkhaṃ vā vedanaṃ vedayamāno ‘dukkhaṃ vedanaṃ vedayāmī’ti pajānāti. Adukkhamasukhaṃ vā vedanaṃ vedayamāno ‘adukkhamasukhaṃ vedanaṃ vedayāmī’ti pajānāti. Sāmisaṃ vā sukhaṃ vedanaṃ vedayamāno ‘sāmisaṃ sukhaṃ vedanaṃ vedayāmī’ti pajānāti, nirāmisaṃ vā sukhaṃ vedanaṃ vedayamāno ‘nirāmisaṃ sukhaṃ vedanaṃ vedayāmī’ti pajānāti. Sāmisaṃ vā dukkhaṃ vedanaṃ vedayamāno ‘sāmisaṃ dukkhaṃ vedanaṃ vedayāmī’ti pajānāti, nirāmisaṃ vā dukkhaṃ vedanaṃ vedayamāno ‘nirāmisaṃ dukkhaṃ vedanaṃ vedayāmī’ti pajānāti. Sāmisaṃ vā adukkhamasukhaṃ vedanaṃ vedayamāno ‘sāmisaṃ adukkhamasukhaṃ vedanaṃ vedayāmī’ti pajānāti, nirāmisaṃ vā adukkhamasukhaṃ vedanaṃ vedayamāno ‘nirāmisaṃ adukkhamasukhaṃ vedanaṃ vedayāmī’ti pajānāti. Iti ajjhattaṃ vā vedanāsu vedanānupassī viharati, bahiddhā vā vedanāsu vedanānupassī viharati, ajjhattabahiddhā vā vedanāsu vedanānupassī viharati. Samudayadhammānupassī vā vedanāsu viharati, vayadhammānupassī vā vedanāsu viharati, samudayavayadhammānupassī vā vedanāsu viharati. ‘Atthi vedanā’ti vā panassa sati paccupaṭṭhitā hoti yāvadeva ñāṇamattāya paṭissatimattāya anissito ca viharati, na ca kiñci loke upādiyati. Evampi kho, bhikkhave, bhikkhu vedanāsu vedanānupassī viharati.

\xsubsubsectionEnd{Vedanānupassanā niṭṭhitā.}

\subsubsection{Cittānupassanā}

\paragraph{381.} ‘‘Kathañca pana, bhikkhave, bhikkhu citte cittānupassī viharati? Idha, bhikkhave, bhikkhu sarāgaṃ vā cittaṃ ‘sarāgaṃ citta’nti pajānāti, vītarāgaṃ vā cittaṃ ‘vītarāgaṃ citta’nti pajānāti. Sadosaṃ vā cittaṃ ‘sadosaṃ citta’nti pajānāti, vītadosaṃ vā cittaṃ ‘vītadosaṃ citta’nti pajānāti. Samohaṃ vā cittaṃ ‘samohaṃ citta’nti pajānāti, vītamohaṃ vā cittaṃ ‘vītamohaṃ citta’nti pajānāti. Saṅkhittaṃ vā cittaṃ ‘saṅkhittaṃ citta’nti pajānāti, vikkhittaṃ vā cittaṃ ‘vikkhittaṃ citta’nti pajānāti. Mahaggataṃ vā cittaṃ ‘mahaggataṃ citta’nti pajānāti, amahaggataṃ vā cittaṃ ‘amahaggataṃ citta’nti pajānāti. Sauttaraṃ vā cittaṃ ‘sauttaraṃ citta’nti pajānāti, anuttaraṃ vā cittaṃ ‘anuttaraṃ citta’nti pajānāti. Samāhitaṃ vā cittaṃ ‘samāhitaṃ citta’nti pajānāti, asamāhitaṃ vā cittaṃ ‘asamāhitaṃ citta’nti pajānāti. Vimuttaṃ vā cittaṃ ‘vimuttaṃ citta’nti pajānāti. Avimuttaṃ vā cittaṃ ‘avimuttaṃ citta’nti pajānāti. Iti ajjhattaṃ vā citte cittānupassī viharati, bahiddhā vā citte cittānupassī viharati, ajjhattabahiddhā vā citte cittānupassī viharati. Samudayadhammānupassī vā cittasmiṃ viharati, vayadhammānupassī vā cittasmiṃ viharati, samudayavayadhammānupassī vā cittasmiṃ viharati, ‘atthi citta’nti vā panassa sati paccupaṭṭhitā hoti yāvadeva ñāṇamattāya paṭissatimattāya anissito ca viharati, na ca kiñci loke upādiyati . Evampi kho, bhikkhave, bhikkhu citte cittānupassī viharati.

\xsubsubsectionEnd{Cittānupassanā niṭṭhitā.}

\subsubsection{Dhammānupassanā nīvaraṇapabbaṃ}

\paragraph{382.} ‘‘Kathañca pana, bhikkhave, bhikkhu dhammesu dhammānupassī viharati? Idha, bhikkhave, bhikkhu dhammesu dhammānupassī viharati pañcasu nīvaraṇesu. Kathañca pana, bhikkhave, bhikkhu dhammesu dhammānupassī viharati pañcasu nīvaraṇesu?

‘‘Idha, bhikkhave, bhikkhu santaṃ vā ajjhattaṃ kāmacchandaṃ ‘atthi me ajjhattaṃ kāmacchando’ti pajānāti, asantaṃ vā ajjhattaṃ kāmacchandaṃ ‘natthi me ajjhattaṃ kāmacchando’ti pajānāti, yathā ca anuppannassa kāmacchandassa uppādo hoti tañca pajānāti, yathā ca uppannassa kāmacchandassa pahānaṃ hoti tañca pajānāti, yathā ca pahīnassa kāmacchandassa āyatiṃ anuppādo hoti tañca pajānāti.

‘‘Santaṃ vā ajjhattaṃ byāpādaṃ ‘atthi me ajjhattaṃ byāpādo’ti pajānāti, asantaṃ vā ajjhattaṃ byāpādaṃ ‘natthi me ajjhattaṃ byāpādo’ti pajānāti, yathā ca anuppannassa byāpādassa uppādo hoti tañca pajānāti, yathā ca uppannassa byāpādassa pahānaṃ hoti tañca pajānāti, yathā ca pahīnassa byāpādassa āyatiṃ anuppādo hoti tañca pajānāti.

‘‘Santaṃ vā ajjhattaṃ thinamiddhaṃ ‘atthi me ajjhattaṃ thinamiddha’nti pajānāti, asantaṃ vā ajjhattaṃ thinamiddhaṃ ‘natthi me ajjhattaṃ thinamiddha’nti pajānāti, yathā ca anuppannassa thinamiddhassa uppādo hoti tañca pajānāti, yathā ca uppannassa thinamiddhassa pahānaṃ hoti tañca pajānāti, yathā ca pahīnassa thinamiddhassa āyatiṃ anuppādo hoti tañca pajānāti.

‘‘Santaṃ vā ajjhattaṃ uddhaccakukkuccaṃ ‘atthi me ajjhattaṃ uddhaccakukkucca’nti pajānāti, asantaṃ vā ajjhattaṃ uddhaccakukkuccaṃ ‘natthi me ajjhattaṃ uddhaccakukkucca’nti pajānāti, yathā ca anuppannassa uddhaccakukkuccassa uppādo hoti tañca pajānāti, yathā ca uppannassa uddhaccakukkuccassa pahānaṃ hoti tañca pajānāti, yathā ca pahīnassa uddhaccakukkuccassa āyatiṃ anuppādo hoti tañca pajānāti.

‘‘Santaṃ vā ajjhattaṃ vicikicchaṃ ‘atthi me ajjhattaṃ vicikicchā’ti pajānāti, asantaṃ vā ajjhattaṃ vicikicchaṃ ‘natthi me ajjhattaṃ vicikicchā’ti pajānāti, yathā ca anuppannāya vicikicchāya uppādo hoti tañca pajānāti, yathā ca uppannāya vicikicchāya pahānaṃ hoti tañca pajānāti, yathā ca pahīnāya vicikicchāya āyatiṃ anuppādo hoti tañca pajānāti.

‘‘Iti ajjhattaṃ vā dhammesu dhammānupassī viharati, bahiddhā vā dhammesu dhammānupassī viharati, ajjhattabahiddhā vā dhammesu dhammānupassī viharati samudayadhammānupassī vā dhammesu viharati, vayadhammānupassī vā dhammesu viharati, samudayavayadhammānupassī vā dhammesu viharati ‘atthi dhammā’ti vā panassa sati paccupaṭṭhitā hoti yāvadeva ñāṇamattāya paṭissatimattāya anissito ca viharati, na ca kiñci loke upādiyati. Evampi kho, bhikkhave, bhikkhu dhammesu dhammānupassī viharati pañcasu nīvaraṇesu.

\xsubsubsectionEnd{Nīvaraṇapabbaṃ niṭṭhitaṃ.}

\subsubsection{Dhammānupassanā khandhapabbaṃ}

\paragraph{383.} ‘‘Puna caparaṃ, bhikkhave, bhikkhu dhammesu dhammānupassī viharati pañcasu upādānakkhandhesu. Kathañca pana, bhikkhave, bhikkhu dhammesu dhammānupassī viharati pañcasu upādānakkhandhesu? Idha, bhikkhave, bhikkhu – ‘iti rūpaṃ, iti rūpassa samudayo, iti rūpassa atthaṅgamo; iti vedanā, iti vedanāya samudayo, iti vedanāya atthaṅgamo; iti saññā, iti saññāya samudayo, iti saññāya atthaṅgamo; iti saṅkhārā, iti saṅkhārānaṃ samudayo, iti saṅkhārānaṃ atthaṅgamo, iti viññāṇaṃ, iti viññāṇassa samudayo, iti viññāṇassa atthaṅgamo’ti, iti ajjhattaṃ vā dhammesu dhammānupassī viharati, bahiddhā vā dhammesu dhammānupassī viharati, ajjhattabahiddhā vā dhammesu dhammānupassī viharati. Samudayadhammānupassī vā dhammesu viharati, vayadhammānupassī vā dhammesu viharati, samudayavayadhammānupassī vā dhammesu viharati. ‘Atthi dhammā’ti vā panassa sati paccupaṭṭhitā hoti yāvadeva ñāṇamattāya paṭissatimattāya, anissito ca viharati, na ca kiñci loke upādiyati . Evampi kho, bhikkhave, bhikkhu dhammesu dhammānupassī viharati pañcasu upādānakkhandhesu.

\xsubsubsectionEnd{Khandhapabbaṃ niṭṭhitaṃ.}

\subsubsection{Dhammānupassanā āyatanapabbaṃ}

\paragraph{384.} ‘‘Puna caparaṃ, bhikkhave, bhikkhu dhammesu dhammānupassī viharati chasu ajjhattikabāhiresu āyatanesu. Kathañca pana, bhikkhave, bhikkhu dhammesu dhammānupassī viharati chasu ajjhattikabāhiresu āyatanesu?

‘‘Idha, bhikkhave, bhikkhu cakkhuñca pajānāti, rūpe ca pajānāti, yañca tadubhayaṃ paṭicca uppajjati saṃyojanaṃ tañca pajānāti, yathā ca anuppannassa saṃyojanassa uppādo hoti tañca pajānāti, yathā ca uppannassa saṃyojanassa pahānaṃ hoti tañca pajānāti, yathā ca pahīnassa saṃyojanassa āyatiṃ anuppādo hoti tañca pajānāti.

‘‘Sotañca pajānāti, sadde ca pajānāti, yañca tadubhayaṃ paṭicca uppajjati saṃyojanaṃ tañca pajānāti, yathā ca anuppannassa saṃyojanassa uppādo hoti tañca pajānāti, yathā ca uppannassa saṃyojanassa pahānaṃ hoti tañca pajānāti, yathā ca pahīnassa saṃyojanassa āyatiṃ anuppādo hoti tañca pajānāti.

‘‘Ghānañca pajānāti, gandhe ca pajānāti, yañca tadubhayaṃ paṭicca uppajjati saṃyojanaṃ tañca pajānāti, yathā ca anuppannassa saṃyojanassa uppādo hoti tañca pajānāti, yathā ca uppannassa saṃyojanassa pahānaṃ hoti tañca pajānāti, yathā ca pahīnassa saṃyojanassa āyatiṃ anuppādo hoti tañca pajānāti.

‘‘Jivhañca pajānāti, rase ca pajānāti, yañca tadubhayaṃ paṭicca uppajjati saṃyojanaṃ tañca pajānāti, yathā ca anuppannassa saṃyojanassa uppādo hoti tañca pajānāti , yathā ca uppannassa saṃyojanassa pahānaṃ hoti tañca pajānāti, yathā ca pahīnassa saṃyojanassa āyatiṃ anuppādo hoti tañca pajānāti.

‘‘Kāyañca pajānāti, phoṭṭhabbe ca pajānāti, yañca tadubhayaṃ paṭicca uppajjati saṃyojanaṃ tañca pajānāti, yathā ca anuppannassa saṃyojanassa uppādo hoti tañca pajānāti, yathā ca uppannassa saṃyojanassa pahānaṃ hoti tañca pajānāti, yathā ca pahīnassa saṃyojanassa āyatiṃ anuppādo hoti tañca pajānāti.

‘‘Manañca pajānāti, dhamme ca pajānāti, yañca tadubhayaṃ paṭicca uppajjati saṃyojanaṃ tañca pajānāti, yathā ca anuppannassa saṃyojanassa uppādo hoti tañca pajānāti, yathā ca uppannassa saṃyojanassa pahānaṃ hoti tañca pajānāti, yathā ca pahīnassa saṃyojanassa āyatiṃ anuppādo hoti tañca pajānāti.

‘‘Iti ajjhattaṃ vā dhammesu dhammānupassī viharati, bahiddhā vā dhammesu dhammānupassī viharati, ajjhattabahiddhā vā dhammesu dhammānupassī viharati. Samudayadhammānupassī vā dhammesu viharati, vayadhammānupassī vā dhammesu viharati, samudayavayadhammānupassī vā dhammesu viharati. ‘Atthi dhammā’ti vā panassa sati paccupaṭṭhitā hoti yāvadeva ñāṇamattāya paṭissatimattāya, anissito ca viharati, na ca kiñci loke upādiyati. Evampi kho, bhikkhave, bhikkhu dhammesu dhammānupassī viharati chasu ajjhattikabāhiresu āyatanesu.

\xsubsubsectionEnd{Āyatanapabbaṃ niṭṭhitaṃ.}

\subsubsection{Dhammānupassanā bojjhaṅgapabbaṃ}

\paragraph{385.} ‘‘Puna caparaṃ, bhikkhave, bhikkhu dhammesu dhammānupassī viharati sattasu bojjhaṅgesu. Kathañca pana, bhikkhave, bhikkhu dhammesu dhammānupassī viharati sattasu bojjhaṅgesu? Idha, bhikkhave, bhikkhu santaṃ vā ajjhattaṃ satisambojjhaṅgaṃ ‘atthi me ajjhattaṃ satisambojjhaṅgo’ti pajānāti, asantaṃ vā ajjhattaṃ satisambojjhaṅgaṃ ‘natthi me ajjhattaṃ satisambojjhaṅgo’ti pajānāti, yathā ca anuppannassa satisambojjhaṅgassa uppādo hoti tañca pajānāti, yathā ca uppannassa satisambojjhaṅgassa bhāvanāya pāripūrī hoti tañca pajānāti.

‘‘Santaṃ vā ajjhattaṃ dhammavicayasambojjhaṅgaṃ ‘atthi me ajjhattaṃ dhammavicayasambojjhaṅgo’ti pajānāti, asantaṃ vā ajjhattaṃ dhammavicayasambojjhaṅgaṃ ‘natthi me ajjhattaṃ dhammavicayasambojjhaṅgo’ti pajānāti, yathā ca anuppannassa dhammavicayasambojjhaṅgassa uppādo hoti tañca pajānāti, yathā ca uppannassa dhammavicayasambojjhaṅgassa bhāvanāya pāripūrī hoti tañca pajānāti.

‘‘Santaṃ vā ajjhattaṃ vīriyasambojjhaṅgaṃ ‘atthi me ajjhattaṃ vīriyasambojjhaṅgo’ti pajānāti, asantaṃ vā ajjhattaṃ vīriyasambojjhaṅgaṃ ‘natthi me ajjhattaṃ vīriyasambojjhaṅgo’ti pajānāti, yathā ca anuppannassa vīriyasambojjhaṅgassa uppādo hoti tañca pajānāti, yathā ca uppannassa vīriyasambojjhaṅgassa bhāvanāya pāripūrī hoti tañca pajānāti.

‘‘Santaṃ vā ajjhattaṃ pītisambojjhaṅgaṃ ‘atthi me ajjhattaṃ pītisambojjhaṅgo’ti pajānāti, asantaṃ vā ajjhattaṃ pītisambojjhaṅgaṃ ‘natthi me ajjhattaṃ pītisambojjhaṅgo’ti pajānāti, yathā ca anuppannassa pītisambojjhaṅgassa uppādo hoti tañca pajānāti, yathā ca uppannassa pītisambojjhaṅgassa bhāvanāya pāripūrī hoti tañca pajānāti.

‘‘Santaṃ vā ajjhattaṃ passaddhisambojjhaṅgaṃ ‘atthi me ajjhattaṃ passaddhisambojjhaṅgo’ti pajānāti, asantaṃ vā ajjhattaṃ passaddhisambojjhaṅgaṃ ‘natthi me ajjhattaṃ passaddhisambojjhaṅgo’ti pajānāti, yathā ca anuppannassa passaddhisambojjhaṅgassa uppādo hoti tañca pajānāti, yathā ca uppannassa passaddhisambojjhaṅgassa bhāvanāya pāripūrī hoti tañca pajānāti.

‘‘Santaṃ vā ajjhattaṃ samādhisambojjhaṅgaṃ ‘atthi me ajjhattaṃ samādhisambojjhaṅgo’ti pajānāti, asantaṃ vā ajjhattaṃ samādhisambojjhaṅgaṃ ‘natthi me ajjhattaṃ samādhisambojjhaṅgo’ti pajānāti, yathā ca anuppannassa samādhisambojjhaṅgassa uppādo hoti tañca pajānāti, yathā ca uppannassa samādhisambojjhaṅgassa bhāvanāya pāripūrī hoti tañca pajānāti.

‘‘Santaṃ vā ajjhattaṃ upekkhāsambojjhaṅgaṃ ‘atthi me ajjhattaṃ upekkhāsambojjhaṅgo’ti pajānāti , asantaṃ vā ajjhattaṃ upekkhāsambojjhaṅgaṃ ‘natthi me ajjhattaṃ upekkhāsambojjhaṅgo’ti pajānāti, yathā ca anuppannassa upekkhāsambojjhaṅgassa uppādo hoti tañca pajānāti, yathā ca uppannassa upekkhāsambojjhaṅgassa bhāvanāya pāripūrī hoti tañca pajānāti.

‘‘Iti ajjhattaṃ vā dhammesu dhammānupassī viharati, bahiddhā vā dhammesu dhammānupassī viharati, ajjhattabahiddhā vā dhammesu dhammānupassī viharati. Samudayadhammānupassī vā dhammesu viharati, vayadhammānupassī vā dhammesu viharati, samudayavayadhammānupassī vā dhammesu viharati ‘atthi dhammā’ti vā panassa sati paccupaṭṭhitā hoti yāvadeva ñāṇamattāya paṭissatimattāya anissito ca viharati, na ca kiñci loke upādiyati. Evampi kho, bhikkhave, bhikkhu dhammesu dhammānupassī viharati sattasu bojjhaṅgesu.

Bojjhaṅgapabbaṃ niṭṭhitaṃ.\footnote{bojjhaṅgapabbaṃ niṭṭhitaṃ, paṭhamabhāṇavāraṃ (syā.)}

\subsubsection{Dhammānupassanā saccapabbaṃ}

\paragraph{386.} ‘‘Puna caparaṃ, bhikkhave, bhikkhu dhammesu dhammānupassī viharati catūsu ariyasaccesu. Kathañca pana, bhikkhave, bhikkhu dhammesu dhammānupassī viharati catūsu ariyasaccesu? Idha, bhikkhave, bhikkhu ‘idaṃ dukkha’nti yathābhūtaṃ pajānāti, ‘ayaṃ dukkhasamudayo’ti yathābhūtaṃ pajānāti, ‘ayaṃ dukkhanirodho’ti yathābhūtaṃ pajānāti, ‘ayaṃ dukkhanirodhagāminī paṭipadā’ti yathābhūtaṃ pajānāti.

\xsubsubsectionEnd{Paṭhamabhāṇavāro niṭṭhito.}

\subsubsection{Dukkhasaccaniddeso}

\paragraph{387.} ‘‘Katamañca , bhikkhave, dukkhaṃ ariyasaccaṃ? Jātipi dukkhā, jarāpi dukkhā, maraṇampi dukkhaṃ, sokaparidevadukkhadomanassupāyāsāpi dukkhā, appiyehi sampayogopi dukkho, piyehi vippayogopi dukkho\footnote{appiyehi…pe… vippayogo dukkhotipāṭho ceva taṃniddeso ca katthaci na dissati, aṭṭhakathāyaṃpi taṃsaṃvaṇṇanā natthi}, yampicchaṃ na labhati tampi dukkhaṃ, saṅkhittena pañcupādānakkhandhā\footnote{pañcupādānakkhandhāpi (ka.)}dukkhā.

\paragraph{388.} ‘‘Katamā ca, bhikkhave, jāti? Yā tesaṃ tesaṃ sattānaṃ tamhi tamhi sattanikāye jāti sañjāti okkanti abhinibbatti khandhānaṃ pātubhāvo āyatanānaṃ paṭilābho, ayaṃ vuccati, bhikkhave, jāti.

\paragraph{389.} ‘‘Katamā ca, bhikkhave, jarā? Yā tesaṃ tesaṃ sattānaṃ tamhi tamhi sattanikāye jarā jīraṇatā khaṇḍiccaṃ pāliccaṃ valittacatā āyuno saṃhāni indriyānaṃ paripāko, ayaṃ vuccati, bhikkhave, jarā.

\paragraph{390.} ‘‘Katamañca, bhikkhave, maraṇaṃ? Yaṃ\footnote{aṭṭhakathā oloketabbā} tesaṃ tesaṃ sattānaṃ tamhā tamhā sattanikāyā cuti cavanatā bhedo antaradhānaṃ maccu maraṇaṃ kālakiriyā khandhānaṃ bhedo kaḷevarassa nikkhepo jīvitindriyassupacchedo, idaṃ vuccati, bhikkhave, maraṇaṃ.

\paragraph{391.} ‘‘Katamo ca, bhikkhave, soko? Yo kho, bhikkhave, aññataraññatarena byasanena samannāgatassa aññataraññatarena dukkhadhammena phuṭṭhassa soko socanā socitattaṃ antosoko antoparisoko, ayaṃ vuccati, bhikkhave, soko.

\paragraph{392.} ‘‘Katamo ca, bhikkhave, paridevo? Yo kho, bhikkhave, aññataraññatarena byasanena samannāgatassa aññataraññatarena dukkhadhammena phuṭṭhassa ādevo paridevo ādevanā paridevanā ādevitattaṃ paridevitattaṃ, ayaṃ vuccati, bhikkhave paridevo.

\paragraph{393.} ‘‘Katamañca , bhikkhave, dukkhaṃ? Yaṃ kho, bhikkhave, kāyikaṃ dukkhaṃ kāyikaṃ asātaṃ kāyasamphassajaṃ dukkhaṃ asātaṃ vedayitaṃ, idaṃ vuccati, bhikkhave, dukkhaṃ.

\paragraph{394.} ‘‘Katamañca , bhikkhave, domanassaṃ? Yaṃ kho, bhikkhave, cetasikaṃ dukkhaṃ cetasikaṃ asātaṃ manosamphassajaṃ dukkhaṃ asātaṃ vedayitaṃ, idaṃ vuccati, bhikkhave, domanassaṃ.

\paragraph{395.} ‘‘Katamo ca, bhikkhave, upāyāso? Yo kho, bhikkhave, aññataraññatarena byasanena samannāgatassa aññataraññatarena dukkhadhammena phuṭṭhassa āyāso upāyāso āyāsitattaṃ upāyāsitattaṃ, ayaṃ vuccati, bhikkhave, upāyāso.

\paragraph{396.} ‘‘Katamo ca, bhikkhave, appiyehi sampayogo dukkho? Idha yassa te honti aniṭṭhā akantā amanāpā rūpā saddā gandhā rasā phoṭṭhabbā dhammā, ye vā panassa te honti anatthakāmā ahitakāmā aphāsukakāmā ayogakkhemakāmā, yā tehi saddhiṃ saṅgati samāgamo samodhānaṃ missībhāvo, ayaṃ vuccati, bhikkhave, appiyehi sampayogo dukkho.

\paragraph{397.} ‘‘Katamo ca, bhikkhave, piyehi vippayogo dukkho? Idha yassa te honti iṭṭhā kantā manāpā rūpā saddā gandhā rasā phoṭṭhabbā dhammā, ye vā panassa te honti atthakāmā hitakāmā phāsukakāmā yogakkhemakāmā mātā vā pitā vā bhātā vā bhaginī vā mittā vā amaccā vā ñātisālohitā vā, yā tehi saddhiṃ asaṅgati asamāgamo asamodhānaṃ amissībhāvo, ayaṃ vuccati, bhikkhave, piyehi vippayogo dukkho.

\paragraph{398.} ‘‘Katamañca , bhikkhave, yampicchaṃ na labhati tampi dukkhaṃ? Jātidhammānaṃ, bhikkhave, sattānaṃ evaṃ icchā uppajjati – ‘aho vata mayaṃ na jātidhammā assāma, na ca vata no jāti āgaccheyyā’ti. Na kho panetaṃ icchāya pattabbaṃ, idampi yampicchaṃ na labhati tampi dukkhaṃ. Jarādhammānaṃ, bhikkhave, sattānaṃ evaṃ icchā uppajjati – ‘aho vata mayaṃ na jarādhammā assāma, na ca vata no jarā āgaccheyyā’ti. Na kho panetaṃ icchāya pattabbaṃ, idampi yampicchaṃ na labhati tampi dukkhaṃ. Byādhidhammānaṃ, bhikkhave, sattānaṃ evaṃ icchā uppajjati ‘aho vata mayaṃ na byādhidhammā assāma, na ca vata no byādhi āgaccheyyā’ti. Na kho panetaṃ icchāya pattabbaṃ, idampi yampicchaṃ na labhati tampi dukkhaṃ. Maraṇadhammānaṃ, bhikkhave, sattānaṃ evaṃ icchā uppajjati ‘aho vata mayaṃ na maraṇadhammā assāma, na ca vata no maraṇaṃ āgaccheyyā’ti. Na kho panetaṃ icchāya pattabbaṃ, idampi yampicchaṃ na labhati tampi dukkhaṃ. Sokaparidevadukkhadomanassupāyāsadhammānaṃ, bhikkhave, sattānaṃ evaṃ icchā uppajjati ‘aho vata mayaṃ na sokaparidevadukkhadomanassupāyāsadhammā assāma, na ca vata no sokaparidevadukkhadomanassupāyāsadhammā āgaccheyyu’nti. Na kho panetaṃ icchāya pattabbaṃ, idampi yampicchaṃ na labhati tampi dukkhaṃ.

\paragraph{399.} ‘‘Katame ca, bhikkhave, saṅkhittena pañcupādānakkhandhā dukkhā? Seyyathidaṃ – rūpupādānakkhandho, vedanupādānakkhandho, saññupādānakkhandho, saṅkhārupādānakkhandho, viññāṇupādānakkhandho. Ime vuccanti, bhikkhave, saṅkhittena pañcupādānakkhandhā dukkhā. Idaṃ vuccati, bhikkhave, dukkhaṃ ariyasaccaṃ.

\subsubsection{Samudayasaccaniddeso}

\paragraph{400.} ‘‘Katamañca , bhikkhave, dukkhasamudayaṃ\footnote{dukkhasamudayo (syā.)} ariyasaccaṃ? Yāyaṃ taṇhā ponobbhavikā\footnote{ponobhavikā (sī. pī.)} nandīrāgasahagatā\footnote{nandirāgasahagatā (sī. syā. pī.)} tatratatrābhinandinī, seyyathidaṃ – kāmataṇhā bhavataṇhā vibhavataṇhā.

‘‘Sā kho panesā, bhikkhave, taṇhā kattha uppajjamānā uppajjati, kattha nivisamānā nivisati? Yaṃ loke piyarūpaṃ sātarūpaṃ, etthesā taṇhā uppajjamānā uppajjati, ettha nivisamānā nivisati.

‘‘Kiñca loke piyarūpaṃ sātarūpaṃ? Cakkhu loke piyarūpaṃ sātarūpaṃ, etthesā taṇhā uppajjamānā uppajjati, ettha nivisamānā nivisati. Sotaṃ loke…pe… ghānaṃ loke… jivhā loke… kāyo loke… mano loke piyarūpaṃ sātarūpaṃ, etthesā taṇhā uppajjamānā uppajjati, ettha nivisamānā nivisati.

‘‘Rūpā loke… saddā loke… gandhā loke… rasā loke… phoṭṭhabbā loke… dhammā loke piyarūpaṃ sātarūpaṃ, etthesā taṇhā uppajjamānā uppajjati, ettha nivisamānā nivisati.

‘‘Cakkhuviññāṇaṃ loke… sotaviññāṇaṃ loke… ghānaviññāṇaṃ loke… jivhāviññāṇaṃ loke… kāyaviññāṇaṃ loke… manoviññāṇaṃ loke piyarūpaṃ sātarūpaṃ, etthesā taṇhā uppajjamānā uppajjati, ettha nivisamānā nivisati.

‘‘Cakkhusamphasso loke… sotasamphasso loke… ghānasamphasso loke… jivhāsamphasso loke… kāyasamphasso loke… manosamphasso loke piyarūpaṃ sātarūpaṃ, etthesā taṇhā uppajjamānā uppajjati, ettha nivisamānā nivisati.

‘‘Cakkhusamphassajā vedanā loke… sotasamphassajā vedanā loke… ghānasamphassajā vedanā loke… jivhāsamphassajā vedanā loke… kāyasamphassajā vedanā loke… manosamphassajā vedanā loke piyarūpaṃ sātarūpaṃ, etthesā taṇhā uppajjamānā uppajjati, ettha nivisamānā nivisati.

‘‘Rūpasaññā loke… saddasaññā loke… gandhasaññā loke… rasasaññā loke… phoṭṭhabbasaññā loke… dhammasaññā loke piyarūpaṃ sātarūpaṃ, etthesā taṇhā uppajjamānā uppajjati, ettha nivisamānā nivisati.

‘‘Rūpasañcetanā loke… saddasañcetanā loke… gandhasañcetanā loke… rasasañcetanā loke… phoṭṭhabbasañcetanā loke… dhammasañcetanā loke piyarūpaṃ sātarūpaṃ, etthesā taṇhā uppajjamānā uppajjati, ettha nivisamānā nivisati.

‘‘Rūpataṇhā loke… saddataṇhā loke… gandhataṇhā loke… rasataṇhā loke… phoṭṭhabbataṇhā loke… dhammataṇhā loke piyarūpaṃ sātarūpaṃ, etthesā taṇhā uppajjamānā uppajjati, ettha nivisamānā nivisati.

‘‘Rūpavitakko loke… saddavitakko loke… gandhavitakko loke… rasavitakko loke… phoṭṭhabbavitakko loke… dhammavitakko loke piyarūpaṃ sātarūpaṃ, etthesā taṇhā uppajjamānā uppajjati, ettha nivisamānā nivisati.

‘‘Rūpavicāro loke… saddavicāro loke… gandhavicāro loke… rasavicāro loke… phoṭṭhabbavicāro loke… dhammavicāro loke piyarūpaṃ sātarūpaṃ, etthesā taṇhā uppajjamānā uppajjati, ettha nivisamānā nivisati. Idaṃ vuccati, bhikkhave, dukkhasamudayaṃ ariyasaccaṃ.

\subsubsection{Nirodhasaccaniddeso}

\paragraph{401.} ‘‘Katamañca , bhikkhave, dukkhanirodhaṃ\footnote{dukkhanirodho (syā.)} ariyasaccaṃ? Yo tassāyeva taṇhāya asesavirāganirodho cāgo paṭinissaggo mutti anālayo.

‘‘Sā kho panesā, bhikkhave, taṇhā kattha pahīyamānā pahīyati, kattha nirujjhamānā nirujjhati? Yaṃ loke piyarūpaṃ sātarūpaṃ, etthesā taṇhā pahīyamānā pahīyati, ettha nirujjhamānā nirujjhati.

‘‘Kiñca loke piyarūpaṃ sātarūpaṃ? Cakkhu loke piyarūpaṃ sātarūpaṃ, etthesā taṇhā pahīyamānā pahīyati, ettha nirujjhamānā nirujjhati. Sotaṃ loke…pe… ghānaṃ loke… jivhā loke… kāyo loke… mano loke piyarūpaṃ sātarūpaṃ, etthesā taṇhā pahīyamānā pahīyati, ettha nirujjhamānā nirujjhati.

‘‘Rūpā loke… saddā loke… gandhā loke… rasā loke… phoṭṭhabbā loke… dhammā loke piyarūpaṃ sātarūpaṃ, etthesā taṇhā pahīyamānā pahīyati, ettha nirujjhamānā nirujjhati.

‘‘Cakkhuviññāṇaṃ loke… sotaviññāṇaṃ loke… ghānaviññāṇaṃ loke… jivhāviññāṇaṃ loke… kāyaviññāṇaṃ loke… manoviññāṇaṃ loke piyarūpaṃ sātarūpaṃ, etthesā taṇhā pahīyamānā pahīyati, ettha nirujjhamānā nirujjhati.

‘‘Cakkhusamphasso loke… sotasamphasso loke… ghānasamphasso loke… jivhāsamphasso loke… kāyasamphasso loke… manosamphasso loke piyarūpaṃ sātarūpaṃ, etthesā taṇhā pahīyamānā pahīyati, ettha nirujjhamānā nirujjhati.

‘‘Cakkhusamphassajā vedanā loke… sotasamphassajā vedanā loke … ghānasamphassajā vedanā loke… jivhāsamphassajā vedanā loke… kāyasamphassajā vedanā loke… manosamphassajā vedanā loke piyarūpaṃ sātarūpaṃ, etthesā taṇhā pahīyamānā pahīyati, ettha nirujjhamānā nirujjhati.

‘‘Rūpasaññā loke… saddasaññā loke… gandhasaññā loke… rasasaññā loke… phoṭṭhabbasaññā loke… dhammasaññā loke piyarūpaṃ sātarūpaṃ, etthesā taṇhā pahīyamānā pahīyati, ettha nirujjhamānā nirujjhati.

‘‘Rūpasañcetanā loke… saddasañcetanā loke… gandhasañcetanā loke… rasasañcetanā loke… phoṭṭhabbasañcetanā loke… dhammasañcetanā loke piyarūpaṃ sātarūpaṃ, etthesā taṇhā pahīyamānā pahīyati, ettha nirujjhamānā nirujjhati.

‘‘Rūpataṇhā loke… saddataṇhā loke… gandhataṇhā loke… rasataṇhā loke… phoṭṭhabbataṇhā loke… dhammataṇhā loke piyarūpaṃ sātarūpaṃ, etthesā taṇhā pahīyamānā pahīyati, ettha nirujjhamānā nirujjhati.

‘‘Rūpavitakko loke… saddavitakko loke… gandhavitakko loke… rasavitakko loke… phoṭṭhabbavitakko loke… dhammavitakko loke piyarūpaṃ sātarūpaṃ, etthesā taṇhā pahīyamānā pahīyati, ettha nirujjhamānā nirujjhati.

‘‘Rūpavicāro loke… saddavicāro loke… gandhavicāro loke… rasavicāro loke… phoṭṭhabbavicāro loke… dhammavicāro loke piyarūpaṃ sātarūpaṃ , etthesā taṇhā pahīyamānā pahīyati, ettha nirujjhamānā nirujjhati. Idaṃ vuccati, bhikkhave, dukkhanirodhaṃ ariyasaccaṃ.

\subsubsection{Maggasaccaniddeso}

\paragraph{402.} ‘‘Katamañca, bhikkhave, dukkhanirodhagāminī paṭipadā ariyasaccaṃ? Ayameva ariyo aṭṭhaṅgiko maggo seyyathidaṃ – sammādiṭṭhi sammāsaṅkappo sammāvācā sammākammanto sammāājīvo sammāvāyāmo sammāsati sammāsamādhi.

‘‘Katamā ca, bhikkhave, sammādiṭṭhi? Yaṃ kho, bhikkhave, dukkhe ñāṇaṃ, dukkhasamudaye ñāṇaṃ, dukkhanirodhe ñāṇaṃ, dukkhanirodhagāminiyā paṭipadāya ñāṇaṃ, ayaṃ vuccati, bhikkhave, sammādiṭṭhi.

‘‘Katamo ca, bhikkhave, sammāsaṅkappo? Nekkhammasaṅkappo abyāpādasaṅkappo avihiṃsāsaṅkappo, ayaṃ vuccati bhikkhave, sammāsaṅkappo.

‘‘Katamā ca, bhikkhave, sammāvācā? Musāvādā veramaṇī\footnote{veramaṇi (ka.)} pisuṇāya vācāya veramaṇī pharusāya vācāya veramaṇī samphappalāpā veramaṇī, ayaṃ vuccati, bhikkhave, sammāvācā.

‘‘Katamo ca, bhikkhave, sammākammanto? Pāṇātipātā veramaṇī adinnādānā veramaṇī kāmesumicchācārā veramaṇī, ayaṃ vuccati, bhikkhave, sammākammanto.

‘‘Katamo ca, bhikkhave, sammāājīvo? Idha, bhikkhave, ariyasāvako micchāājīvaṃ pahāya sammāājīvena jīvitaṃ kappeti, ayaṃ vuccati, bhikkhave, sammāājīvo.

‘‘Katamo ca, bhikkhave, sammāvāyāmo? Idha, bhikkhave, bhikkhu anuppannānaṃ pāpakānaṃ akusalānaṃ dhammānaṃ anuppādāya chandaṃ janeti vāyamati vīriyaṃ ārabhati cittaṃ paggaṇhāti padahati; uppannānaṃ pāpakānaṃ akusalānaṃ dhammānaṃ pahānāya chandaṃ janeti vāyamati vīriyaṃ ārabhati cittaṃ paggaṇhāti padahati; anuppannānaṃ kusalānaṃ dhammānaṃ uppādāya chandaṃ janeti vāyamati vīriyaṃ ārabhati cittaṃ paggaṇhāti padahati; uppannānaṃ kusalānaṃ dhammānaṃ ṭhitiyā asammosāya bhiyyobhāvāya vepullāya bhāvanāya pāripūriyā chandaṃ janeti vāyamati vīriyaṃ ārabhati cittaṃ paggaṇhāti padahati. Ayaṃ vuccati, bhikkhave, sammāvāyāmo.

‘‘Katamā ca, bhikkhave, sammāsati? Idha, bhikkhave, bhikkhu kāye kāyānupassī viharati ātāpī sampajāno satimā vineyya loke abhijjhādomanassaṃ; vedanāsu vedanānupassī viharati ātāpī sampajāno satimā vineyya loke abhijjhādomanassaṃ; citte cittānupassī viharati ātāpī sampajāno satimā vineyya loke abhijjhādomanassaṃ; dhammesu dhammānupassī viharati ātāpī sampajāno satimā vineyya loke abhijjhādomanassaṃ. Ayaṃ vuccati, bhikkhave, sammāsati.

‘‘Katamo ca, bhikkhave, sammāsamādhi? Idha, bhikkhave, bhikkhu vivicceva kāmehi vivicca akusalehi dhammehi savitakkaṃ savicāraṃ vivekajaṃ pītisukhaṃ paṭhamaṃ jhānaṃ upasampajja viharati. Vitakkavicārānaṃ vūpasamā ajjhattaṃ sampasādanaṃ cetaso ekodibhāvaṃ avitakkaṃ avicāraṃ samādhijaṃ pītisukhaṃ dutiyaṃ jhānaṃ upasampajja viharati. Pītiyā ca virāgā upekkhako ca viharati, sato ca sampajāno, sukhañca kāyena paṭisaṃvedeti, yaṃ taṃ ariyā ācikkhanti ‘upekkhako satimā sukhavihārī’ti tatiyaṃ jhānaṃ upasampajja viharati. Sukhassa ca pahānā dukkhassa ca pahānā pubbeva somanassadomanassānaṃ atthaṅgamā adukkhamasukhaṃ upekkhāsatipārisuddhiṃ catutthaṃ jhānaṃ upasampajja viharati. Ayaṃ vuccati, bhikkhave , sammāsamādhi. Idaṃ vuccati, bhikkhave, dukkhanirodhagāminī paṭipadā ariyasaccaṃ.

\paragraph{403.} ‘‘Iti ajjhattaṃ vā dhammesu dhammānupassī viharati, bahiddhā vā dhammesu dhammānupassī viharati, ajjhattabahiddhā vā dhammesu dhammānupassī viharati. Samudayadhammānupassī vā dhammesu viharati, vayadhammānupassī vā dhammesu viharati, samudayavayadhammānupassī vā dhammesu viharati. ‘Atthi dhammā’ti vā panassa sati paccupaṭṭhitā hoti yāvadeva ñāṇamattāya paṭissatimattāya anissito ca viharati, na ca kiñci loke upādiyati. Evampi kho, bhikkhave, bhikkhu dhammesu dhammānupassī viharati catūsu ariyasaccesu.

Saccapabbaṃ niṭṭhitaṃ.

\xsubsubsectionEnd{Dhammānupassanā niṭṭhitā.}

\paragraph{404.} ‘‘Yo hi koci, bhikkhave, ime cattāro satipaṭṭhāne evaṃ bhāveyya sattavassāni, tassa dvinnaṃ phalānaṃ aññataraṃ phalaṃ pāṭikaṅkhaṃ diṭṭheva dhamme aññā; sati vā upādisese anāgāmitā.

‘‘Tiṭṭhantu, bhikkhave, sattavassāni. Yo hi koci, bhikkhave, ime cattāro satipaṭṭhāne evaṃ bhāveyya cha vassāni…pe… pañca vassāni… cattāri vassāni… tīṇi vassāni… dve vassāni… ekaṃ vassaṃ… tiṭṭhatu, bhikkhave, ekaṃ vassaṃ. Yo hi koci, bhikkhave, ime cattāro satipaṭṭhāne evaṃ bhāveyya sattamāsāni, tassa dvinnaṃ phalānaṃ aññataraṃ phalaṃ pāṭikaṅkhaṃ diṭṭheva dhamme aññā; sati vā upādisese anāgāmitā.

‘‘Tiṭṭhantu , bhikkhave, satta māsāni. Yo hi koci, bhikkhave, ime cattāro satipaṭṭhāne evaṃ bhāveyya cha māsāni…pe… pañca māsāni… cattāri māsāni… tīṇi māsāni … dve māsāni… ekaṃ māsaṃ… aḍḍhamāsaṃ… tiṭṭhatu, bhikkhave, aḍḍhamāso. Yo hi koci, bhikkhave, ime cattāro satipaṭṭhāne evaṃ bhāveyya sattāhaṃ, tassa dvinnaṃ phalānaṃ aññataraṃ phalaṃ pāṭikaṅkhaṃ diṭṭheva dhamme aññā; sati vā upādisese anāgāmitāti.

\paragraph{405.} ‘‘Ekāyano ayaṃ, bhikkhave, maggo sattānaṃ visuddhiyā sokaparidevānaṃ samatikkamāya dukkhadomanassānaṃ atthaṅgamāya ñāyassa adhigamāya nibbānassa sacchikiriyāya yadidaṃ cattāro satipaṭṭhānāti. Iti yaṃ taṃ vuttaṃ, idametaṃ paṭicca vutta’’nti. Idamavoca bhagavā. Attamanā te bhikkhū bhagavato bhāsitaṃ abhinandunti.

\xsectionEnd{Mahāsatipaṭṭhānasuttaṃ niṭṭhitaṃ navamaṃ.}


\clearpage
\section{Pāyāsisuttaṃ}

\paragraph{406.} Evaṃ me sutaṃ – ekaṃ samayaṃ āyasmā kumārakassapo kosalesu cārikaṃ caramāno mahatā bhikkhusaṅghena saddhiṃ pañcamattehi bhikkhusatehi yena setabyā nāma kosalānaṃ nagaraṃ tadavasari. Tatra sudaṃ āyasmā kumārakassapo setabyāyaṃ viharati uttarena setabyaṃ siṃsapāvane\footnote{sīsapāvane (syā.)}. Tena kho pana samayena pāyāsi rājañño setabyaṃ ajjhāvasati sattussadaṃ satiṇakaṭṭhodakaṃ sadhaññaṃ rājabhoggaṃ raññā pasenadinā kosalena dinnaṃ rājadāyaṃ brahmadeyyaṃ.

\subsubsection{Pāyāsirājaññavatthu}

\paragraph{407.} Tena kho pana samayena pāyāsissa rājaññassa evarūpaṃ pāpakaṃ diṭṭhigataṃ uppannaṃ hoti – ‘‘itipi natthi paro loko, natthi sattā opapātikā, natthi sukatadukkaṭānaṃ\footnote{sukaṭakkaṭānaṃ (sī. pī.)} kammānaṃ phalaṃ vipāko’’ti. Assosuṃ kho setabyakā brāhmaṇagahapatikā – ‘‘samaṇo khalu bho kumārakassapo samaṇassa gotamassa sāvako kosalesu cārikaṃ caramāno mahatā bhikkhusaṅghena saddhiṃ pañcamattehi bhikkhusatehi setabyaṃ anuppatto setabyāyaṃ viharati uttarena setabyaṃ siṃsapāvane. Taṃ kho pana bhavantaṃ kumārakassapaṃ evaṃ kalyāṇo kittisaddo abbhuggato – ‘paṇḍito byatto medhāvī bahussuto cittakathī kalyāṇapaṭibhāno vuddho\footnote{buddho (syā. ka.)} ceva arahā ca. Sādhu kho pana tathārūpānaṃ arahataṃ dassanaṃ hotī’’’ti. Atha kho setabyakā brāhmaṇagahapatikā setabyāya nikkhamitvā saṅghasaṅghī gaṇībhūtā uttarenamukhā gacchanti yena siṃsapāvanaṃ\footnote{yena siṃsapāvanaṃ, tenupasaṅkamanti (sī. pī.)}.

\paragraph{408.} Tena kho pana samayena pāyāsi rājañño uparipāsāde divāseyyaṃ upagato hoti. Addasā kho pāyāsi rājañño setabyake brāhmaṇagahapatike setabyāya nikkhamitvā saṅghasaṅghī gaṇībhūte uttarenamukhe gacchante yena siṃsapāvanaṃ\footnote{yena siṃsapāvanaṃ, tenupasaṅkamante (sī. pī.)}, disvā khattaṃ āmantesi – ‘‘kiṃ nu kho, bho khatte, setabyakā brāhmaṇagahapatikā setabyāya nikkhamitvā saṅghasaṅghī gaṇībhūtā uttarenamukhā gacchanti yena siṃsapāvana’’nti\footnote{ettha pana sabbatthapi evameva dissati, natthi pāṭhantaraṃ}?

‘‘Atthi kho, bho, samaṇo kumārakassapo, samaṇassa gotamassa sāvako kosalesu cārikaṃ caramāno mahatā bhikkhusaṅghena saddhiṃ pañcamattehi bhikkhusatehi setabyaṃ anuppatto setabyāyaṃ viharati uttarena setabyaṃ siṃsapāvane. Taṃ kho pana bhavantaṃ kumārakassapaṃ evaṃ kalyāṇo kittisaddo abbhuggato – ‘paṇḍito byatto medhāvī bahussuto cittakathī kalyāṇapaṭibhāno vuddho ceva arahā cā’ti\footnote{arahā ca (syā. ka.)}. Tamete\footnote{tamenaṃ te (sī. ka.), tamenaṃ (pī.)} bhavantaṃ kumārakassapaṃ dassanāya upasaṅkamantī’’ti. ‘‘Tena hi, bho khatte, yena setabyakā brāhmaṇagahapatikā tenupasaṅkama; upasaṅkamitvā setabyake brāhmaṇagahapatike evaṃ vadehi – ‘pāyāsi, bho, rājañño evamāha – āgamentu kira bhavanto, pāyāsipi rājañño samaṇaṃ kumārakassapaṃ dassanāya upasaṅkamissatī’ti. Purā samaṇo kumārakassapo setabyake brāhmaṇagahapatike bāle abyatte saññāpeti – ‘itipi atthi paro loko, atthi sattā opapātikā, atthi sukatadukkaṭānaṃ kammānaṃ phalaṃ vipāko’ti. Natthi hi, bho khatte, paro loko, natthi sattā opapātikā, natthi sukatadukkaṭānaṃ kammānaṃ phalaṃ vipāko’’ti. ‘‘Evaṃ bho’’ti kho so khattā pāyāsissa rājaññassa paṭissutvā yena setabyakā brāhmaṇagahapatikā tenupasaṅkami; upasaṅkamitvā setabyake brāhmaṇagahapatike etadavoca – ‘‘pāyāsi, bho, rājañño evamāha, āgamentu kira bhavanto, pāyāsipi rājañño samaṇaṃ kumārakassapaṃ dassanāya upasaṅkamissatī’’ti.

\paragraph{409.} Atha kho pāyāsi rājañño setabyakehi brāhmaṇagahapatikehi parivuto yena siṃsapāvanaṃ yenāyasmā kumārakassapo tenupasaṅkami; upasaṅkamitvā āyasmatā kumārakassapena saddhiṃ sammodi, sammodanīyaṃ kathaṃ sāraṇīyaṃ vītisāretvā ekamantaṃ nisīdi. Setabyakāpi kho brāhmaṇagahapatikā appekacce āyasmantaṃ kumārakassapaṃ abhivādetvā ekamantaṃ nisīdiṃsu; appekacce āyasmatā kumārakassapena saddhiṃ sammodiṃsu; sammodanīyaṃ kathaṃ sāraṇīyaṃ vītisāretvā ekamantaṃ nisīdiṃsu. Appekacce yenāyasmā kumārakassapo tenañjaliṃ paṇāmetvā ekamantaṃ nisīdiṃsu. Appekacce nāmagottaṃ sāvetvā ekamantaṃ nisīdiṃsu. Appekacce tuṇhībhūtā ekamantaṃ nisīdiṃsu.

\subsubsection{Natthikavādo}

\paragraph{410.} Ekamantaṃ nisinno kho pāyāsi rājañño āyasmantaṃ kumārakassapaṃ etadavoca – ‘‘ahañhi, bho kassapa, evaṃvādī evaṃdiṭṭhī – ‘itipi natthi paro loko, natthi sattā opapātikā, natthi sukatadukkaṭānaṃ kammānaṃ phalaṃ vipāko’’’ti. ‘‘Nāhaṃ, rājañña, evaṃvādiṃ evaṃdiṭṭhiṃ addasaṃ vā assosiṃ vā. Kathañhi nāma evaṃ vadeyya – ‘itipi natthi paro loko, natthi sattā opapātikā, natthi sukatadukkaṭānaṃ kammānaṃ phalaṃ vipāko’ti?

\subsubsection{Candimasūriyaupamā}

\paragraph{411.} ‘‘Tena hi, rājañña, taññevettha paṭipucchissāmi, yathā te khameyya, tathā naṃ byākareyyāsi. Taṃ kiṃ maññasi, rājañña, ime candimasūriyā imasmiṃ vā loke parasmiṃ vā, devā vā te manussā vā’’ti? ‘‘Ime, bho kassapa, candimasūriyā parasmiṃ loke, na imasmiṃ; devā te na manussā’’ti. ‘‘Imināpi kho te, rājañña, pariyāyena evaṃ hotu – itipi atthi paro loko, atthi sattā opapātikā, atthi sukatadukkaṭānaṃ kammānaṃ phalaṃ vipāko’’ti.

\paragraph{412.} ‘‘Kiñcāpi bhavaṃ kassapo evamāha, atha kho evaṃ me ettha hoti – ‘itipi natthi paro loko, natthi sattā opapātikā, natthi sukatadukkaṭānaṃ kammānaṃ phalaṃ vipāko’’’ti. ‘‘Atthi pana, rājañña, pariyāyo, yena te pariyāyena evaṃ hoti – ‘itipi natthi paro loko, natthi sattā opapātikā, natthi sukatadukkaṭānaṃ kammānaṃ phalaṃ vipāko’’’ti? ‘‘Atthi , bho kassapa, pariyāyo, yena me pariyāyena evaṃ hoti – ‘itipi natthi paro loko, natthi sattā opapātikā, natthi sukatadukkaṭānaṃ kammānaṃ phalaṃ vipāko’’’ti. ‘‘Yathā kathaṃ viya, rājaññā’’ti? ‘‘Idha me, bho kassapa, mittāmaccā ñātisālohitā pāṇātipātī adinnādāyī kāmesumicchācārī musāvādī pisuṇavācā pharusavācā samphappalāpī abhijjhālū byāpannacittā micchādiṭṭhī. Te aparena samayena ābādhikā honti dukkhitā bāḷhagilānā. Yadāhaṃ jānāmi – ‘na dānime imamhā ābādhā vuṭṭhahissantī’ti tyāhaṃ upasaṅkamitvā evaṃ vadāmi – ‘santi kho, bho, eke samaṇabrāhmaṇā evaṃvādino evaṃdiṭṭhino – ye te pāṇātipātī adinnādāyī kāmesumicchācārī musāvādī pisuṇavācā pharusavācā samphappalāpī abhijjhālū byāpannacittā micchādiṭṭhī, te kāyassa bhedā paraṃ maraṇā apāyaṃ duggatiṃ vinipātaṃ nirayaṃ upapajjantī’ti. Bhavanto kho pāṇātipātī adinnādāyī kāmesumicchācārī musāvādī pisuṇavācā pharusavācā samphappalāpī abhijjhālū byāpannacittā micchādiṭṭhī. Sace tesaṃ bhavataṃ samaṇabrāhmaṇānaṃ saccaṃ vacanaṃ, bhavanto kāyassa bhedā paraṃ maraṇā apāyaṃ duggatiṃ vinipātaṃ nirayaṃ upapajjissanti. Sace, bho, kāyassa bhedā paraṃ maraṇā apāyaṃ duggatiṃ vinipātaṃ nirayaṃ upapajjeyyātha, yena me āgantvā āroceyyātha – ‘itipi atthi paro loko, atthi sattā opapātikā, atthi sukatadukkaṭānaṃ kammānaṃ phalaṃ vipāko’ti . Bhavanto kho pana me saddhāyikā paccayikā, yaṃ bhavantehi diṭṭhaṃ, yathā sāmaṃ diṭṭhaṃ evametaṃ bhavissatī’ti. Te me ‘sādhū’ti paṭissutvā neva āgantvā ārocenti, na pana dūtaṃ pahiṇanti. Ayampi kho, bho kassapa, pariyāyo, yena me pariyāyena evaṃ hoti – ‘itipi natthi paro loko, natthi sattā opapātikā, natthi sukatadukkaṭānaṃ kammānaṃ phalaṃ vipāko’’’ti.

\subsubsection{Coraupamā}

\paragraph{413.} ‘‘Tena hi, rājañña, taññevettha paṭipucchissāmi. Yathā te khameyya tathā naṃ byākareyyāsi. Taṃ kiṃ maññasi, rājañña, idha te purisā coraṃ āgucāriṃ gahetvā dasseyyuṃ – ‘ayaṃ te, bhante, coro āgucārī; imassa yaṃ icchasi, taṃ daṇḍaṃ paṇehī’ti. Te tvaṃ evaṃ vadeyyāsi – ‘tena hi, bho, imaṃ purisaṃ daḷhāya rajjuyā pacchābāhaṃ gāḷhabandhanaṃ bandhitvā khuramuṇḍaṃ karitvā\footnote{kāretvā (syā. ka.)} kharassarena paṇavena rathikāya rathikaṃ\footnote{rathiyāya rathiyaṃ (bahūsū)} siṅghāṭakena siṅghāṭakaṃ parinetvā dakkhiṇena dvārena nikkhamitvā dakkhiṇato nagarassa āghātane sīsaṃ chindathā’ti. Te ‘sādhū’ti paṭissutvā taṃ purisaṃ daḷhāya rajjuyā pacchābāhaṃ gāḷhabandhanaṃ bandhitvā khuramuṇḍaṃ karitvā kharassarena paṇavena rathikāya rathikaṃ siṅghāṭakena siṅghāṭakaṃ parinetvā dakkhiṇena dvārena nikkhamitvā dakkhiṇato nagarassa āghātane nisīdāpeyyuṃ. Labheyya nu kho so coro coraghātesu – ‘āgamentu tāva bhavanto coraghātā, amukasmiṃ me gāme vā nigame vā mittāmaccā ñātisālohitā, yāvāhaṃ tesaṃ uddisitvā āgacchāmī’ti , udāhu vippalapantasseva coraghātā sīsaṃ chindeyyu’’nti? ‘‘Na hi so, bho kassapa, coro labheyya coraghātesu – ‘āgamentu tāva bhavanto coraghātā amukasmiṃ me gāme vā nigame vā mittāmaccā ñātisālohitā, yāvāhaṃ tesaṃ uddisitvā āgacchāmī’ti. Atha kho naṃ vippalapantasseva coraghātā sīsaṃ chindeyyu’’nti. ‘‘So hi nāma, rājañña, coro manusso manussabhūtesu coraghātesu na labhissati – ‘āgamentu tāva bhavanto coraghātā, amukasmiṃ me gāme vā nigame vā mittāmaccā ñātisālohitā, yāvāhaṃ tesaṃ uddisitvā āgacchāmī’ti. Kiṃ pana te mittāmaccā ñātisālohitā pāṇātipātī adinnādāyī kāmesumicchācārī musāvādī pisuṇavācā pharusavācā samphappalāpī abhijjhālū byāpannacittā micchādiṭṭhī, te kāyassa bhedā paraṃ maraṇā apāyaṃ duggatiṃ vinipātaṃ nirayaṃ upapannā labhissanti nirayapālesu – ‘āgamentu tāva bhavanto nirayapālā, yāva mayaṃ pāyāsissa rājaññassa gantvā ārocema – ‘‘itipi atthi paro loko, atthi sattā opapātikā, atthi sukatadukkaṭānaṃ kammānaṃ phalaṃ vipāko’’’ti? Imināpi kho te, rājañña, pariyāyena evaṃ hotu – ‘itipi atthi paro loko, atthi sattā opapātikā, atthi sukatadukkaṭānaṃ kammānaṃ phalaṃ vipāko’’’ti.

\paragraph{414.} ‘‘Kiñcāpi bhavaṃ kassapo evamāha, atha kho evaṃ me ettha hoti – ‘itipi natthi paro loko, natthi sattā opapātikā, natthi sukatadukkaṭānaṃ kammānaṃ phalaṃ vipāko’’ti. ‘‘Atthi pana, rājañña, pariyāyo yena te pariyāyena evaṃ hoti – ‘itipi natthi paro loko, natthi sattā opapātikā, natthi sukatadukkaṭānaṃ kammānaṃ phalaṃ vipāko’’’ti? ‘‘Atthi, bho kassapa, pariyāyo, yena me pariyāyena evaṃ hoti – ‘itipi natthi paro loko, natthi sattā opapātikā, natthi sukatadukkaṭānaṃ kammānaṃ phalaṃ vipāko’’’ti. ‘‘Yathā kathaṃ viya, rājaññā’’ti? ‘‘Idha me, bho kassapa, mittāmaccā ñātisālohitā pāṇātipātā paṭiviratā adinnādānā paṭiviratā kāmesumicchācārā paṭiviratā musāvādā paṭiviratā pisuṇāya vācāya paṭiviratā pharusāya vācāya paṭiviratā samphappalāpā paṭiviratā anabhijjhālū abyāpannacittā sammādiṭṭhī. Te aparena samayena ābādhikā honti dukkhitā bāḷhagilānā. Yadāhaṃ jānāmi – ‘na dānime imamhā ābādhā vuṭṭhahissantī’ti tyāhaṃ upasaṅkamitvā evaṃ vadāmi – ‘santi kho, bho, eke samaṇabrāhmaṇā evaṃvādino evaṃdiṭṭhino – ye te pāṇātipātā paṭiviratā adinnādānā paṭiviratā kāmesumicchācārā paṭiviratā musāvādā paṭiviratā pisuṇāya vācāya paṭiviratā pharusāya vācāya paṭiviratā samphappalāpā paṭiviratā anabhijjhālū abyāpannacittā sammādiṭṭhī te kāyassa bhedā paraṃ maraṇā sugatiṃ saggaṃ lokaṃ upapajjantīti . Bhavanto kho pāṇātipātā paṭiviratā adinnādānā paṭiviratā kāmesumicchācārā paṭiviratā musāvādā paṭiviratā pisuṇāya vācāya paṭiviratā pharusāya vācāya paṭiviratā samphappalāpā paṭiviratā anabhijjhālū abyāpannacittā sammādiṭṭhī. Sace tesaṃ bhavataṃ samaṇabrāhmaṇānaṃ saccaṃ vacanaṃ, bhavanto kāyassa bhedā paraṃ maraṇā sugatiṃ saggaṃ lokaṃ upapajjissanti. Sace, bho, kāyassa bhedā paraṃ maraṇā sugatiṃ saggaṃ lokaṃ upapajjeyyātha, yena me āgantvā āroceyyātha – ‘itipi atthi paro loko, atthi sattā opapātikā, atthi sukatadukkaṭānaṃ kammānaṃ phalaṃ vipāko’ti. Bhavanto kho pana me saddhāyikā paccayikā, yaṃ bhavantehi diṭṭhaṃ, yathā sāmaṃ diṭṭhaṃ evametaṃ bhavissatī’ti. Te me ‘sādhū’ti paṭissutvā neva āgantvā ārocenti, na pana dūtaṃ pahiṇanti. Ayampi kho, bho kassapa, pariyāyo, yena me pariyāyena evaṃ hoti – ‘itipi natthi paro loko, natthi sattā opapātikā, natthi sukatadukkaṭānaṃ kammānaṃ phalaṃ vipāko’’’ti.

\subsubsection{Gūthakūpapurisaupamā}

\paragraph{415.} ‘‘Tena hi, rājañña, upamaṃ te karissāmi. Upamāya midhekacce\footnote{upamāyapidhekacce (sī. syā.), upamāyapiidhekacce (pī.)} viññū purisā bhāsitassa atthaṃ ājānanti. Seyyathāpi, rājañña, puriso gūthakūpe sasīsakaṃ\footnote{sasīsako (syā.)} nimuggo assa. Atha tvaṃ purise āṇāpeyyāsi – ‘tena hi, bho, taṃ purisaṃ tamhā gūthakūpā uddharathā’ti. Te ‘sādhū’ti paṭissutvā taṃ purisaṃ tamhā gūthakūpā uddhareyyuṃ. Te tvaṃ evaṃ vadeyyāsi – ‘tena hi, bho, tassa purisassa kāyā veḷupesikāhi gūthaṃ sunimmajjitaṃ nimmajjathā’ti. Te ‘sādhū’ti paṭissutvā tassa purisassa kāyā veḷupesikāhi gūthaṃ sunimmajjitaṃ nimmajjeyyuṃ. Te tvaṃ evaṃ vadeyyāsi – ‘tena hi, bho, tassa purisassa kāyaṃ paṇḍumattikāya tikkhattuṃ subbaṭṭitaṃ ubbaṭṭethā’ti\footnote{suppaṭṭitaṃ uppaṭṭethāti (ka.)}. Te tassa purisassa kāyaṃ paṇḍumattikāya tikkhattuṃ subbaṭṭitaṃ ubbaṭṭeyyuṃ. Te tvaṃ evaṃ vadeyyāsi – ‘tena hi, bho, taṃ purisaṃ telena abbhañjitvā sukhumena cuṇṇena tikkhattuṃ suppadhotaṃ karothā’ti. Te taṃ purisaṃ telena abbhañjitvā sukhumena cuṇṇena tikkhattuṃ suppadhotaṃ kareyyuṃ. Te tvaṃ evaṃ vadeyyāsi – ‘tena hi, bho, tassa purisassa kesamassuṃ kappethā’ti. Te tassa purisassa kesamassuṃ kappeyyuṃ. Te tvaṃ evaṃ vadeyyāsi – ‘tena hi, bho, tassa purisassa mahagghañca mālaṃ mahagghañca vilepanaṃ mahagghāni ca vatthāni upaharathā’ti. Te tassa purisassa mahagghañca mālaṃ mahagghañca vilepanaṃ mahagghāni ca vatthāni upahareyyuṃ. Te tvaṃ evaṃ vadeyyāsi – ‘tena hi, bho, taṃ purisaṃ pāsādaṃ āropetvā pañcakāmaguṇāni upaṭṭhāpethā’ti. Te taṃ purisaṃ pāsādaṃ āropetvā pañcakāmaguṇāni upaṭṭhāpeyyuṃ.

‘‘Taṃ kiṃ maññasi, rājañña, api nu tassa purisassa sunhātassa suvilittassa sukappitakesamassussa āmukkamālābharaṇassa odātavatthavasanassa uparipāsādavaragatassa pañcahi kāmaguṇehi samappitassa samaṅgībhūtassa paricārayamānassa punadeva tasmiṃ gūthakūpe nimujjitukāmatā\footnote{nimujjitukāmyatā (syā. ka.)} assā’’ti? ‘‘No hidaṃ, bho kassapa’’. ‘‘Taṃ kissa hetu’’? ‘‘Asuci, bho kassapa, gūthakūpo asuci ceva asucisaṅkhāto ca duggandho ca duggandhasaṅkhāto ca jeguccho ca jegucchasaṅkhāto ca paṭikūlo ca paṭikūlasaṅkhāto cā’’ti. ‘‘Evameva kho, rājañña, manussā devānaṃ asucī ceva asucisaṅkhātā ca, duggandhā ca duggandhasaṅkhātā ca, jegucchā ca jegucchasaṅkhātā ca, paṭikūlā ca paṭikūlasaṅkhātā ca. Yojanasataṃ kho, rājañña, manussagandho deve ubbādhati. Kiṃ pana te mittāmaccā ñātisālohitā pāṇātipātā paṭiviratā adinnādānā paṭiviratā kāmesumicchācārā paṭiviratā musāvādā paṭiviratā pisuṇāya vācāya paṭiviratā pharusāya vācāya paṭiviratā samphappalāpā paṭiviratā anabhijjhālū abyāpannacittā sammādiṭṭhī, kāyassa bhedā paraṃ maraṇā sugatiṃ saggaṃ lokaṃ upapannā te āgantvā ārocessanti – ‘itipi atthi paro loko, atthi sattā opapātikā, atthi sukatadukkaṭānaṃ kammānaṃ phalaṃ vipāko’ti? Imināpi kho te, rājañña, pariyāyena evaṃ hotu – ‘itipi atthi paro loko, atthi sattā opapātikā, atthi sukatadukkaṭānaṃ kammānaṃ phalaṃ vipāko’’’ti.

\paragraph{416.} ‘‘Kiñcāpi bhavaṃ kassapo evamāha, atha kho evaṃ me ettha hoti – ‘itipi natthi paro loko, natthi sattā opapātikā, natthi sukatadukkaṭānaṃ kammānaṃ phalaṃ vipāko’’’ti. ‘‘Atthi pana, rājañña, pariyāyo …pe… ‘‘atthi, bho kassapa, pariyāyo…pe… ``yathā kathaṃ viya, rājaññāti? ‘‘Idha me, bho kassapa, mittāmaccā ñātisālohitā pāṇātipātā paṭiviratā adinnādānā paṭiviratā kāmesumicchācārā paṭiviratā musāvādā paṭiviratā surāmerayamajjapamādaṭṭhānā paṭiviratā, te aparena samayena ābādhikā honti dukkhitā bāḷhagilānā. Yadāhaṃ jānāmi – ‘na dānime imamhā ābādhā vuṭṭhahissantī’ti tyāhaṃ upasaṅkamitvā evaṃ vadāmi – ‘santi kho, bho, eke samaṇabrāhmaṇā evaṃvādino evaṃdiṭṭhino – ye te pāṇātipātā paṭiviratā adinnādānā paṭiviratā kāmesumicchācārā paṭiviratā musāvādā paṭiviratā surāmerayamajjapamādaṭṭhānā paṭiviratā, te kāyassa bhedā paraṃ maraṇā sugatiṃ saggaṃ lokaṃ upapajjanti devānaṃ tāvatiṃsānaṃ sahabyatanti. Bhavanto kho pāṇātipātā paṭiviratā adinnādānā paṭiviratā kāmesumicchācārā paṭiviratā musāvādā paṭiviratā surāmerayamajjapamādaṭṭhānā paṭiviratā. Sace tesaṃ bhavataṃ samaṇabrāhmaṇānaṃ saccaṃ vacanaṃ, bhavanto kāyassa bhedā paraṃ maraṇā sugatiṃ saggaṃ lokaṃ upapajjissanti, devānaṃ tāvatiṃsānaṃ sahabyataṃ. Sace, bho, kāyassa bhedā paraṃ maraṇā sugatiṃ saggaṃ lokaṃ upapajjeyyātha devānaṃ tāvatiṃsānaṃ sahabyataṃ, yena me āgantvā āroceyyātha – `itipi atthi paro loko, atthi sattā opapātikā, atthi sukatadukkaṭānaṃ kammānaṃ phalaṃ vipākoti. Bhavanto kho pana me saddhāyikā paccayikā, yaṃ bhavantehi diṭṭhaṃ, yathā sāmaṃ diṭṭhaṃ evametaṃ bhavissatīti. Te me ‘sādhū’ti paṭissutvā neva āgantvā ārocenti, na pana dūtaṃ pahiṇanti. Ayampi kho, bho kassapa, pariyāyo, yena me pariyāyena evaṃ hoti – ‘itipi natthi paro loko, natthi sattā opapātikā, natthi sukatadukkaṭānaṃ kammānaṃ phalaṃ vipāko’’’ti.

\subsubsection{Tāvatiṃsadevaupamā}

\paragraph{417.} ‘‘Tena hi, rājañña, taññevettha paṭipucchissāmi; yathā te khameyya, tathā naṃ byākareyyāsi. Yaṃ kho pana, rājañña, mānussakaṃ vassasataṃ, devānaṃ tāvatiṃsānaṃ eso eko rattindivo\footnote{rattidivo (ka.)}, tāya rattiyā tiṃsarattiyo māso, tena māsena dvādasamāsiyo saṃvaccharo, tena saṃvaccharena dibbaṃ vassasahassaṃ devānaṃ tāvatiṃsānaṃ āyuppamāṇaṃ. Ye te mittāmaccā ñātisālohitā pāṇātipātā paṭiviratā adinnādānā paṭiviratā kāmesumicchācārā paṭiviratā musāvādā paṭiviratā surāmerayamajjapamādaṭṭhānā paṭiviratā, te kāyassa bhedā paraṃ maraṇā sugatiṃ saggaṃ lokaṃ upapannā devānaṃ tāvatiṃsānaṃ sahabyataṃ. Sace pana tesaṃ evaṃ bhavissati – ‘yāva mayaṃ dve vā tīṇi vā rattindivā dibbehi pañcahi kāmaguṇehi samappitā samaṅgībhūtā paricārema, atha mayaṃ pāyāsissa rājaññassa gantvā āroceyyāma – ‘‘itipi atthi paro loko, atthi sattā opapātikā, atthi sukatadukkaṭānaṃ kammānaṃ phalaṃ vipāko’’ti. Api nu te āgantvā āroceyyuṃ – ‘itipi atthi paro loko, atthi sattā opapātikā, atthi sukatadukkaṭānaṃ kammānaṃ phalaṃ vipāko’’’ti? ‘‘No hidaṃ, bho kassapa. Api hi mayaṃ, bho kassapa, ciraṃ kālaṅkatāpi bhaveyyāma. Ko panetaṃ bhoto kassapassa āroceti – ‘atthi devā tāvatiṃsā’ti vā ‘evaṃdīghāyukā devā tāvatiṃsā’ti vā. Na mayaṃ bhoto kassapassa saddahāma – ‘atthi devā tāvatiṃsā’ti vā ‘evaṃdīghāyukā devā tāvatiṃsā’ti vā’’ti.

\subsubsection{Jaccandhaupamā}

\paragraph{418.} ‘‘Seyyathāpi, rājañña, jaccandho puriso na passeyya kaṇha – sukkāni rūpāni , na passeyya nīlakāni rūpāni, na passeyya pītakāni\footnote{mañjeṭṭhakāni (syā.)} rūpāni, na passeyya lohitakāni rūpāni, na passeyya mañjiṭṭhakāni rūpāni, na passeyya samavisamaṃ, na passeyya tārakāni rūpāni, na passeyya candimasūriye. So evaṃ vadeyya – ‘natthi kaṇhasukkāni rūpāni, natthi kaṇhasukkānaṃ rūpānaṃ dassāvī. Natthi nīlakāni rūpāni, natthi nīlakānaṃ rūpānaṃ dassāvī. Natthi pītakāni rūpāni, natthi pītakānaṃ rūpānaṃ dassāvī. Natthi lohitakāni rūpāni, natthi lohitakānaṃ rūpānaṃ dassāvī. Natthi mañjiṭṭhakāni rūpāni, natthi mañjiṭṭhakānaṃ rūpānaṃ dassāvī. Natthi samavisamaṃ, natthi samavisamassa dassāvī. Natthi tārakāni rūpāni, natthi tārakānaṃ rūpānaṃ dassāvī. Natthi candimasūriyā, natthi candimasūriyānaṃ dassāvī. Ahametaṃ na jānāmi, ahametaṃ na passāmi, tasmā taṃ natthī’ti. Sammā nu kho so, rājañña, vadamāno vadeyyā’’ti? ‘‘No hidaṃ, bho kassapa. Atthi kaṇhasukkāni rūpāni, atthi kaṇhasukkānaṃ rūpānaṃ dassāvī. Atthi nīlakāni rūpāni, atthi nīlakānaṃ rūpānaṃ dassāvī…pe… atthi samavisamaṃ, atthi samavisamassa dassāvī. Atthi tārakāni rūpāni, atthi tārakānaṃ rūpānaṃ dassāvī. Atthi candimasūriyā, atthi candimasūriyānaṃ dassāvī. ‘Ahametaṃ na jānāmi, ahametaṃ na passāmi, tasmā taṃ natthī’ti. Na hi so, bho kassapa, sammā vadamāno vadeyyā’’ti. ‘‘Evameva kho tvaṃ, rājañña, jaccandhūpamo maññe paṭibhāsi yaṃ maṃ tvaṃ evaṃ vadesi’’.

‘‘Ko panetaṃ bhoto kassapassa āroceti – ‘atthi devā tāvatiṃsā’’ti vā, ‘evaṃdīghāyukā devā tāvatiṃsā’ti vā? Na mayaṃ bhoto kassapassa saddahāma – ‘atthi devā tāvatiṃsā’ti vā ‘evaṃdīghāyukā devā tāvatiṃsā’ti vā’’ti. ‘‘Na kho, rājañña, evaṃ paro loko daṭṭhabbo, yathā tvaṃ maññasi iminā maṃsacakkhunā. Ye kho te rājañña samaṇabrāhmaṇā araññavanapatthāni pantāni senāsanāni paṭisevanti , te tattha appamattā ātāpino pahitattā viharantā dibbacakkhuṃ visodhenti. Te dibbena cakkhunā visuddhena atikkantamānusakena imaṃ ceva lokaṃ passanti parañca satte ca opapātike. Evañca kho, rājañña, paro loko daṭṭhabbo; natveva yathā tvaṃ maññasi iminā maṃsacakkhunā. Imināpi kho te, rājañña, pariyāyena evaṃ hotu – ‘itipi atthi paro loko, atthi sattā opapātikā, atthi sukatadukkaṭānaṃ kammānaṃ phalaṃ vipāko’’’ti.

\paragraph{419.} ‘‘Kiñcāpi bhavaṃ kassapo evamāha, atha kho evaṃ me ettha hoti – ‘itipi natthi paro loko, natthi sattā opapātikā, natthi sukatadukkaṭānaṃ kammānaṃ phalaṃ vipāko’’ti . ‘‘Atthi pana, rājañña, pariyāyo…pe… atthi, bho kassapa, pariyāyo…pe… yathā kathaṃ viya, rājaññā’’ti? ‘‘Idhāhaṃ, bho kassapa, passāmi samaṇabrāhmaṇe sīlavante kalyāṇadhamme jīvitukāme amaritukāme sukhakāme dukkhapaṭikūle. Tassa mayhaṃ, bho kassapa, evaṃ hoti – sace kho ime bhonto samaṇabrāhmaṇā sīlavanto kalyāṇadhammā evaṃ jāneyyuṃ – ‘ito no matānaṃ seyyo bhavissatī’ti. Idānime bhonto samaṇabrāhmaṇā sīlavanto kalyāṇadhammā visaṃ vā khādeyyuṃ, satthaṃ vā āhareyyuṃ, ubbandhitvā vā kālaṅkareyyuṃ, papāte vā papateyyuṃ. Yasmā ca kho ime bhonto samaṇabrāhmaṇā sīlavanto kalyāṇadhammā na evaṃ jānanti – ‘ito no matānaṃ seyyo bhavissatī’ti, tasmā ime bhonto samaṇabrāhmaṇā sīlavanto kalyāṇadhammā jīvitukāmā amaritukāmā sukhakāmā dukkhapaṭikūlā attānaṃ na mārenti\footnote{( ) natthi (syā. pī.)}. Ayampi kho, bho kassapa, pariyāyo, yena me pariyāyena evaṃ hoti – ‘itipi natthi paro loko, natthi sattā opapātikā, natthi sukatadukkaṭānaṃ kammānaṃ phalaṃ vipāko’’’ti.

\subsubsection{Gabbhinīupamā}

\paragraph{420.} ‘‘Tena hi, rājañña, upamaṃ te karissāmi. Upamāya midhekacce viññū purisā bhāsitassa atthaṃ ājānanti. Bhūtapubbaṃ, rājañña, aññatarassa brāhmaṇassa dve pajāpatiyo ahesuṃ. Ekissā putto ahosi dasavassuddesiko vā dvādasavassuddesiko vā, ekā gabbhinī upavijaññā. Atha kho so brāhmaṇo kālamakāsi. Atha kho so māṇavako mātusapattiṃ\footnote{mātusapatiṃ (syā.)} etadavoca – ‘yamidaṃ, bhoti, dhanaṃ vā dhaññaṃ vā rajataṃ vā jātarūpaṃ vā, sabbaṃ taṃ mayhaṃ ; natthi tuyhettha kiñci. Pitu me\footnote{pitu me santako (syā.)} bhoti, dāyajjaṃ niyyādehī’ti\footnote{nīyyātehīti (sī. pī.)}. Evaṃ vutte sā brāhmaṇī taṃ māṇavakaṃ etadavoca – ‘āgamehi tāva, tāta, yāva vijāyāmi. Sace kumārako bhavissati, tassapi ekadeso bhavissati; sace kumārikā bhavissati, sāpi te opabhoggā\footnote{upabhoggā (syā.)} bhavissatī’ti. Dutiyampi kho so māṇavako mātusapattiṃ etadavoca – ‘yamidaṃ, bhoti, dhanaṃ vā dhaññaṃ vā rajataṃ vā jātarūpaṃ vā, sabbaṃ taṃ mayhaṃ; natthi tuyhettha kiñci. Pitu me, bhoti, dāyajjaṃ niyyādehī’ti. Dutiyampi kho sā brāhmaṇī taṃ māṇavakaṃ etadavoca – ‘āgamehi tāva, tāta, yāva vijāyāmi. Sace kumārako bhavissati, tassapi ekadeso bhavissati; sace kumārikā bhavissati sāpi te opabhoggā\footnote{upabhoggā (syā.)} bhavissatī’ti. Tatiyampi kho so māṇavako mātusapattiṃ etadavoca – ‘yamidaṃ, bhoti, dhanaṃ vā dhaññaṃ vā rajataṃ vā jātarūpaṃ vā , sabbaṃ taṃ mayhaṃ; natthi tuyhettha kiñci. Pitu me, bhoti, dāyajjaṃ niyyādehī’ti.

‘‘Atha kho sā brāhmaṇī satthaṃ gahetvā ovarakaṃ pavisitvā udaraṃ opādesi\footnote{uppātesi (syā.)} – ‘yāva vijāyāmi yadi vā kumārako yadi vā kumārikā’ti. Sā attānaṃ ceva jīvitañca gabbhañca sāpateyyañca vināsesi. Yathā taṃ bālā abyattā anayabyasanaṃ āpannā ayoniso dāyajjaṃ gavesantī, evameva kho tvaṃ, rājañña, bālo abyatto anayabyasanaṃ āpajjissasi ayoniso paralokaṃ gavesanto ; seyyathāpi sā brāhmaṇī bālā abyattā anayabyasanaṃ āpannā ayoniso dāyajjaṃ gavesantī. Na kho, rājañña, samaṇabrāhmaṇā sīlavanto kalyāṇadhammā apakkaṃ paripācenti; api ca paripākaṃ āgamenti. Paṇḍitānaṃ attho hi, rājañña, samaṇabrāhmaṇānaṃ sīlavantānaṃ kalyāṇadhammānaṃ jīvitena. Yathā yathā kho, rājañña, samaṇabrāhmaṇā sīlavanto kalyāṇadhammā ciraṃ dīghamaddhānaṃ tiṭṭhanti, tathā tathā bahuṃ puññaṃ pasavanti, bahujanahitāya ca paṭipajjanti bahujanasukhāya lokānukampāya atthāya hitāya sukhāya devamanussānaṃ. Imināpi kho te, rājañña, pariyāyena evaṃ hotu – ‘itipi atthi paro loko, atthi sattā opapātikā, atthi sukatadukkaṭānaṃ kammānaṃ phalaṃ vipāko’’’ti.

\paragraph{421.} ‘‘Kiñcāpi bhavaṃ kassapo evamāha, atha kho evaṃ me ettha hoti – ‘itipi natthi paro loko, natthi sattā opapātikā, natthi sukatadukkaṭānaṃ kammānaṃ phalaṃ vipāko’’’ti. ‘‘Atthi pana, rājañña, pariyāyo…pe… atthi, bho kassapa, pariyāyo…pe… yathā kathaṃ viya, rājaññā’’ti? ‘‘Idha me, bho kassapa, purisā coraṃ āgucāriṃ gahetvā dassenti – ‘ayaṃ te, bhante, coro āgucārī; imassa yaṃ icchasi, taṃ daṇḍaṃ paṇehī’ti. Tyāhaṃ evaṃ vadāmi – ‘tena hi, bho, imaṃ purisaṃ jīvantaṃyeva kumbhiyā pakkhipitvā mukhaṃ pidahitvā allena cammena onandhitvā allāya mattikāya bahalāvalepanaṃ\footnote{bahalavilepanaṃ (syā. ka.)} karitvā uddhanaṃ āropetvā aggiṃ dethā’ti. Te me ‘sādhū’ti paṭissutvā taṃ purisaṃ jīvantaṃyeva kumbhiyā pakkhipitvā mukhaṃ pidahitvā allena cammena onandhitvā allāya mattikāya bahalāvalepanaṃ karitvā uddhanaṃ āropetvā aggiṃ denti. Yadā mayaṃ jānāma ‘kālaṅkato so puriso’ti, atha naṃ kumbhiṃ oropetvā ubbhinditvā mukhaṃ vivaritvā saṇikaṃ nillokema\footnote{vilokema (syā.)} – ‘appeva nāmassa jīvaṃ nikkhamantaṃ passeyyāmā’ti. Nevassa mayaṃ jīvaṃ nikkhamantaṃ passāma. Ayampi kho, bho kassapa, pariyāyo, yena me pariyāyena evaṃ hoti – ‘itipi natthi paro loko, natthi sattā opapātikā, natthi sukatadukkaṭānaṃ kammānaṃ phalaṃ vipāko’’’ti.

\subsubsection{Supinakaupamā}

\paragraph{422.} ‘‘Tena hi, rājañña, taññevettha paṭipucchissāmi, yathā te khameyya, tathā naṃ byākareyyāsi. Abhijānāsi no tvaṃ, rājañña, divā seyyaṃ upagato supinakaṃ passitā ārāmarāmaṇeyyakaṃ vanarāmaṇeyyakaṃ bhūmirāmaṇeyyakaṃ pokkharaṇīrāmaṇeyyaka’’nti? ‘‘Abhijānāmahaṃ, bho kassapa, divāseyyaṃ upagato supinakaṃ passitā ārāmarāmaṇeyyakaṃ vanarāmaṇeyyakaṃ bhūmirāmaṇeyyakaṃ pokkharaṇīrāmaṇeyyaka’’nti. ‘‘Rakkhanti taṃ tamhi samaye khujjāpi vāmanakāpi velāsikāpi\footnote{celāvikāpi (syā.), keḷāyikāpi (sī.)} komārikāpī’’ti? ‘‘Evaṃ, bho kassapa, rakkhanti maṃ tamhi samaye khujjāpi vāmanakāpi velāsikāpi\footnote{celāvikāpi (syā.), keḷāyikāpi (sī.)} komārikāpī’’ti. ‘‘Api nu tā tuyhaṃ jīvaṃ passanti pavisantaṃ vā nikkhamantaṃ vā’’ti? ‘‘No hidaṃ, bho kassapa’’. ‘‘Tā hi nāma, rājañña, tuyhaṃ jīvantassa jīvantiyo jīvaṃ na passissanti pavisantaṃ vā nikkhamantaṃ vā. Kiṃ pana tvaṃ kālaṅkatassa jīvaṃ passissasi pavisantaṃ vā nikkhamantaṃ vā. Imināpi kho te, rājañña, pariyāyena evaṃ hotu – ‘‘itipi atthi paro loko, atthi sattā opapātikā, atthi sukatadukkaṭānaṃ kammānaṃ phalaṃ vipāko’’’ti.

\paragraph{423.} ‘‘Kiñcāpi bhavaṃ kassapo evamāha, atha kho evaṃ me ettha hoti – ‘itipi natthi paro loko, natthi sattā opapātikā, natthi sukatadukkaṭānaṃ kammānaṃ phalaṃ vipāko’’’ti. ‘‘Atthi pana, rājañña, pariyāyo…pe… ‘‘atthi, bho kassapa, pariyāyo…pe… yathā kathaṃ viya rājaññā’’ti? ‘‘Idha me, bho kassapa, purisā coraṃ āgucāriṃ gahetvā dassenti – ‘ayaṃ te, bhante, coro āgucārī; imassa yaṃ icchasi, taṃ daṇḍaṃ paṇehī’ti. Tyāhaṃ evaṃ vadāmi – ‘tena hi, bho, imaṃ purisaṃ jīvantaṃyeva tulāya tuletvā jiyāya anassāsakaṃ māretvā punadeva tulāya tulethā’ti. Te me ‘sādhū’ti paṭissutvā taṃ purisaṃ jīvantaṃyeva tulāya tuletvā jiyāya anassāsakaṃ māretvā punadeva tulāya tulenti. Yadā so jīvati, tadā lahutaro ca hoti mudutaro ca kammaññataro ca. Yadā pana so kālaṅkato hoti tadā garutaro ca hoti patthinnataro ca akammaññataro ca. Ayampi kho, bho kassapa, pariyāyo, yena me pariyāyena evaṃ hoti – ‘itipi natthi paro loko, natthi sattā opapātikā, natthi sukatadukkaṭānaṃ kammānaṃ phalaṃ vipāko’’’ti.

\subsubsection{Santattaayoguḷaupamā}

\paragraph{424.} ‘‘Tena hi, rājañña, upamaṃ te karissāmi. Upamāya midhekacce viññū purisā bhāsitassa atthaṃ ājānanti. Seyyathāpi, rājañña, puriso divasaṃ santattaṃ ayoguḷaṃ ādittaṃ sampajjalitaṃ sajotibhūtaṃ tulāya tuleyya. Tamenaṃ aparena samayena sītaṃ nibbutaṃ tulāya tuleyya. Kadā nu kho so ayoguḷo lahutaro vā hoti mudutaro vā kammaññataro vā, yadā vā āditto sampajjalito sajotibhūto, yadā vā sīto nibbuto’’ti? ‘‘Yadā so, bho kassapa, ayoguḷo tejosahagato ca hoti vāyosahagato ca āditto sampajjalito sajotibhūto, tadā lahutaro ca hoti mudutaro ca kammaññataro ca. Yadā pana so ayoguḷo neva tejosahagato hoti na vāyosahagato sīto nibbuto, tadā garutaro ca hoti patthinnataro ca akammaññataro cā’’ti. ‘‘Evameva kho, rājañña, yadāyaṃ kāyo āyusahagato ca hoti usmāsahagato ca viññāṇasahagato ca, tadā lahutaro ca hoti mudutaro ca kammaññataro ca. Yadā panāyaṃ kāyo neva āyusahagato hoti na usmāsahagato na viññāṇasahagato tadā garutaro ca hoti patthinnataro ca akammaññataro ca. Imināpi kho te, rājañña, pariyāyena evaṃ hotu – ‘itipi atthi paro loko, atthi sattā opapātikā, atthi sukatadukkaṭānaṃ kammānaṃ phalaṃ vipāko’’’ti.

\paragraph{425.} ‘‘Kiñcāpi bhavaṃ kassapo evamāha, atha kho evaṃ me ettha hoti – ‘itipi natthi paro loko, natthi sattā opapātikā, natthi sukatadukkaṭānaṃ kammānaṃ phalaṃ vipāko’’’ti. ‘‘Atthi pana, rājañña, pariyāyo…pe… atthi, bho kassapa, pariyāyo…pe… yathā kathaṃ viya rājaññā’’ti? ‘‘Idha me, bho kassapa, purisā coraṃ āgucāriṃ gahetvā dassenti – ‘ayaṃ te, bhante, coro āgucārī; imassa yaṃ icchasi , taṃ daṇḍaṃ paṇehī’ti. Tyāhaṃ evaṃ vadāmi – ‘tena hi, bho, imaṃ purisaṃ anupahacca chaviñca cammañca maṃsañca nhāruñca aṭṭhiñca aṭṭhimiñjañca jīvitā voropetha, appeva nāmassa jīvaṃ nikkhamantaṃ passeyyāmā’ti. Te me ‘sādhū’ti paṭissutvā taṃ purisaṃ anupahacca chaviñca…pe… jīvitā voropenti. Yadā so āmato hoti, tyāhaṃ evaṃ vadāmi – ‘tena hi, bho, imaṃ purisaṃ uttānaṃ nipātetha, appeva nāmassa jīvaṃ nikkhamantaṃ passeyyāmā’ti. Te taṃ purisaṃ uttānaṃ nipātenti. Nevassa mayaṃ jīvaṃ nikkhamantaṃ passāma. Tyāhaṃ evaṃ vadāmi – ‘tena hi, bho, imaṃ purisaṃ avakujjaṃ nipātetha… passena nipātetha… dutiyena passena nipātetha… uddhaṃ ṭhapetha… omuddhakaṃ ṭhapetha… pāṇinā ākoṭetha… leḍḍunā ākoṭetha… daṇḍena ākoṭetha… satthena ākoṭetha… odhunātha sandhunātha niddhunātha, appeva nāmassa jīvaṃ nikkhamantaṃ passeyyāmā’ti. Te taṃ purisaṃ odhunanti sandhunanti niddhunanti. Nevassa mayaṃ jīvaṃ nikkhamantaṃ passāma. Tassa tadeva cakkhu hoti te rūpā, tañcāyatanaṃ nappaṭisaṃvedeti. Tadeva sotaṃ hoti te saddā, tañcāyatanaṃ nappaṭisaṃvedeti. Tadeva ghānaṃ hoti te gandhā, tañcāyatanaṃ nappaṭisaṃvedeti . Sāva jivhā hoti te rasā, tañcāyatanaṃ nappaṭisaṃvedeti. Sveva kāyo hoti te phoṭṭhabbā, tañcāyatanaṃ nappaṭisaṃvedeti. Ayampi kho, bho kassapa, pariyāyo, yena me pariyāyena evaṃ hoti – ‘itipi natthi paro loko, natthi sattā opapātikā, natthi sukatadukkaṭānaṃ kammānaṃ phalaṃ vipāko’’’ti.

\subsubsection{Saṅkhadhamaupamā}

\paragraph{426.} ‘‘Tena hi, rājañña, upamaṃ te karissāmi. Upamāya midhekacce viññū purisā bhāsitassa atthaṃ ājānanti. Bhūtapubbaṃ, rājañña, aññataro saṅkhadhamo saṅkhaṃ ādāya paccantimaṃ janapadaṃ agamāsi. So yena aññataro gāmo tenupasaṅkami; upasaṅkamitvā majjhe gāmassa ṭhito tikkhattuṃ saṅkhaṃ upalāpetvā saṅkhaṃ bhūmiyaṃ nikkhipitvā ekamantaṃ nisīdi. Atha kho, rājañña, tesaṃ paccantajanapadānaṃ\footnote{paccantajānaṃ (sī.)} manussānaṃ etadahosi – ‘ambho kassa nu kho\footnote{etadahosi ‘‘kissa dukho (pī.)} eso saddo evaṃrajanīyo evaṃkamanīyo evaṃmadanīyo evaṃbandhanīyo evaṃmucchanīyo’ti. Sannipatitvā taṃ saṅkhadhamaṃ etadavocuṃ – ‘ambho, kassa nu kho eso saddo evaṃrajanīyo evaṃkamanīyo evaṃmadanīyo evaṃbandhanīyo evaṃmucchanīyo’ti. ‘Eso kho, bho, saṅkho nāma yasseso saddo evaṃrajanīyo evaṃkamanīyo evaṃmadanīyo evaṃbandhanīyo evaṃmucchanīyo’ti. Te taṃ saṅkhaṃ uttānaṃ nipātesuṃ – ‘vadehi, bho saṅkha, vadehi, bho saṅkhā’ti. Neva so saṅkho saddamakāsi. Te taṃ saṅkhaṃ avakujjaṃ nipātesuṃ, passena nipātesuṃ, dutiyena passena nipātesuṃ, uddhaṃ ṭhapesuṃ, omuddhakaṃ ṭhapesuṃ, pāṇinā ākoṭesuṃ, leḍḍunā ākoṭesuṃ, daṇḍena ākoṭesuṃ, satthena ākoṭesuṃ, odhuniṃsu sandhuniṃsu niddhuniṃsu – ‘vadehi, bho saṅkha, vadehi, bho saṅkhā’ti. Neva so saṅkho saddamakāsi.

‘‘Atha kho, rājañña, tassa saṅkhadhamassa etadahosi – ‘yāva bālā ime paccantajanapadāmanussā, kathañhi nāma ayoniso saṅkhasaddaṃ gavesissantī’ti. Tesaṃ pekkhamānānaṃ saṅkhaṃ gahetvā tikkhattuṃ saṅkhaṃ upalāpetvā saṅkhaṃ ādāya pakkāmi. Atha kho, rājañña, tesaṃ paccantajanapadānaṃ manussānaṃ etadahosi – ‘yadā kira, bho, ayaṃ saṅkho nāma purisasahagato ca hoti vāyāmasahagato\footnote{vāyosahagato (syā.)} ca vāyusahagato ca, tadāyaṃ saṅkho saddaṃ karoti, yadā panāyaṃ saṅkho neva purisasahagato hoti na vāyāmasahagato na vāyusahagato, nāyaṃ saṅkho saddaṃ karotī’ti . Evameva kho, rājañña, yadāyaṃ kāyo āyusahagato ca hoti usmāsahagato ca viññāṇasahagato ca, tadā abhikkamatipi paṭikkamatipi tiṭṭhatipi nisīdatipi seyyampi kappeti, cakkhunāpi rūpaṃ passati, sotenapi saddaṃ suṇāti, ghānenapi gandhaṃ ghāyati, jivhāyapi rasaṃ sāyati, kāyenapi phoṭṭhabbaṃ phusati, manasāpi dhammaṃ vijānāti. Yadā panāyaṃ kāyo neva āyusahagato hoti, na usmāsahagato, na viññāṇasahagato, tadā neva abhikkamati na paṭikkamati na tiṭṭhati na nisīdati na seyyaṃ kappeti, cakkhunāpi rūpaṃ na passati, sotenapi saddaṃ na suṇāti, ghānenapi gandhaṃ na ghāyati, jivhāyapi rasaṃ na sāyati, kāyenapi phoṭṭhabbaṃ na phusati, manasāpi dhammaṃ na vijānāti. Imināpi kho te, rājañña, pariyāyena evaṃ hotu – ‘itipi atthi paro loko, atthi sattā opapātikā, atthi sukatadukkaṭānaṃ kammānaṃ phalaṃ vipāko’ti\footnote{vipākoti, paṭhamabhāṇavāraṃ (syā.)}.

\paragraph{427.} ‘‘Kiñcāpi bhavaṃ kassapo evamāha, atha kho evaṃ me ettha hoti – ‘itipi natthi paro loko, natthi sattā opapātikā, natthi sukatadukkaṭānaṃ kammānaṃ phalaṃ vipāko’’’ti. ‘‘Atthi pana, rājañña, pariyāyo…pe… atthi, bho kassapa, pariyāyo…pe… yathā kathaṃ viya rājaññā’’ti? ‘‘Idha me, bho kassapa, purisā coraṃ āgucāriṃ gahetvā dassenti – ‘ayaṃ te, bhante, coro āgucārī, imassa yaṃ icchasi, taṃ daṇḍaṃ paṇehī’ti. Tyāhaṃ evaṃ vadāmi – ‘tena hi, bho, imassa purisassa chaviṃ chindatha , appeva nāmassa jīvaṃ passeyyāmā’ti. Te tassa purisassa chaviṃ chindanti. Nevassa mayaṃ jīvaṃ passāma. Tyāhaṃ evaṃ vadāmi – ‘tena hi, bho, imassa purisassa cammaṃ chindatha, maṃsaṃ chindatha, nhāruṃ chindatha, aṭṭhiṃ chindatha, aṭṭhimiñjaṃ chindatha, appeva nāmassa jīvaṃ passeyyāmā’ti. Te tassa purisassa aṭṭhimiñjaṃ chindanti, nevassa mayaṃ jīvaṃ passeyyāma. Ayampi kho, bho kassapa, pariyāyo, yena me pariyāyena evaṃ hoti – ‘itipi natthi paro loko, natthi sattā opapātikā, natthi sukatadukkaṭānaṃ kammānaṃ phalaṃ vipāko’’’ti.

\subsubsection{Aggikajaṭilaupamā}

\paragraph{428.} ‘‘Tena hi, rājañña, upamaṃ te karissāmi. Upamāya midhekacce viññū purisā bhāsitassa atthaṃ ājānanti. Bhūtapubbaṃ, rājañña, aññataro aggiko jaṭilo araññāyatane paṇṇakuṭiyā sammati\footnote{vasati (sī. pī.)}. Atha kho, rājañña, aññataro janapade sattho\footnote{sattho janapadapadesā (sī.), janapado satthavāso (syā.), janapadapadeso (pī.)} vuṭṭhāsi. Atha kho so sattho\footnote{satthavāso (syā.)} tassa aggikassa jaṭilassa assamassa sāmantā ekarattiṃ vasitvā pakkāmi. Atha kho, rājañña, tassa aggikassa jaṭilassa etadahosi – ‘yaṃnūnāhaṃ yena so satthavāso tenupasaṅkameyyaṃ, appeva nāmettha kiñci upakaraṇaṃ adhigaccheyya’nti. Atha kho so aggiko jaṭilo kālasseva vuṭṭhāya yena so satthavāso tenupasaṅkami; upasaṅkamitvā addasa tasmiṃ satthavāse daharaṃ kumāraṃ mandaṃ uttānaseyyakaṃ chaḍḍitaṃ. Disvānassa etadahosi – ‘na kho me taṃ patirūpaṃ yaṃ me pekkhamānassa manussabhūto kālaṅkareyya; yaṃnūnāhaṃ imaṃ dārakaṃ assamaṃ netvā āpādeyyaṃ poseyyaṃ vaḍḍheyya’nti. Atha kho so aggiko jaṭilo taṃ dārakaṃ assamaṃ netvā āpādesi posesi vaḍḍhesi. Yadā so dārako dasavassuddesiko vā hoti\footnote{ahosi (?)} dvādasavassuddesiko vā, atha kho tassa aggikassa jaṭilassa janapade kañcideva karaṇīyaṃ uppajji. Atha kho so aggiko jaṭilo taṃ dārakaṃ etadavoca – ‘icchāmahaṃ, tāta, janapadaṃ\footnote{nagaraṃ (ka.)} gantuṃ; aggiṃ, tāta, paricareyyāsi. Mā ca te aggi nibbāyi. Sace ca te aggi nibbāyeyya, ayaṃ vāsī imāni kaṭṭhāni idaṃ araṇisahitaṃ, aggiṃ nibbattetvā aggiṃ paricareyyāsī’ti. Atha kho so aggiko jaṭilo taṃ dārakaṃ evaṃ anusāsitvā janapadaṃ agamāsi. Tassa khiḍḍāpasutassa aggi nibbāyi.

‘‘Atha kho tassa dārakassa etadahosi – ‘pitā kho maṃ evaṃ avaca – ‘‘aggiṃ, tāta, paricareyyāsi. Mā ca te aggi nibbāyi. Sace ca te aggi nibbāyeyya, ayaṃ vāsī imāni kaṭṭhāni idaṃ araṇisahitaṃ, aggiṃ nibbattetvā aggiṃ paricareyyāsī’’ti. Yaṃnūnāhaṃ aggiṃ nibbattetvā aggiṃ paricareyya’nti. Atha kho so dārako araṇisahitaṃ vāsiyā tacchi – ‘appeva nāma aggiṃ adhigaccheyya’nti. Neva so aggiṃ adhigacchi. Araṇisahitaṃ dvidhā phālesi, tidhā phālesi, catudhā phālesi, pañcadhā phālesi, dasadhā phālesi, satadhā\footnote{vīsatidhā (syā.)} phālesi, sakalikaṃ sakalikaṃ akāsi, sakalikaṃ sakalikaṃ karitvā udukkhale koṭṭesi, udukkhale koṭṭetvā mahāvāte opuni\footnote{ophuni (syā. ka.)} – ‘appeva nāma aggiṃ adhigaccheyya’nti. Neva so aggiṃ adhigacchi.

‘‘Atha kho so aggiko jaṭilo janapade taṃ karaṇīyaṃ tīretvā yena sako assamo tenupasaṅkami; upasaṅkamitvā taṃ dārakaṃ etadavoca – ‘kacci te, tāta, aggi na nibbuto’ti? ‘Idha me, tāta, khiḍḍāpasutassa aggi nibbāyi. Tassa me etadahosi – ‘‘pitā kho maṃ evaṃ avaca aggiṃ, tāta, paricareyyāsi. Mā ca te, tāta, aggi nibbāyi. Sace ca te aggi nibbāyeyya, ayaṃ vāsī imāni kaṭṭhāni idaṃ araṇisahitaṃ, aggiṃ nibbattetvā aggiṃ paricareyyāsīti. Yaṃnūnāhaṃ aggiṃ nibbattetvā aggiṃ paricareyya’’nti. Atha khvāhaṃ, tāta, araṇisahitaṃ vāsiyā tacchiṃ – ‘‘appeva nāma aggiṃ adhigaccheyya’’nti. Nevāhaṃ aggiṃ adhigacchiṃ. Araṇisahitaṃ dvidhā phālesiṃ, tidhā phālesiṃ, catudhā phālesiṃ, pañcadhā phālesiṃ, dasadhā phālesiṃ , satadhā phālesiṃ, sakalikaṃ sakalikaṃ akāsiṃ, sakalikaṃ sakalikaṃ karitvā udukkhale koṭṭesiṃ, udukkhale koṭṭetvā mahāvāte opuniṃ – ‘‘appeva nāma aggiṃ adhigaccheyya’’nti. Nevāhaṃ aggiṃ adhigacchi’’’nti. Atha kho tassa aggikassa jaṭilassa etadahosi – ‘yāva bālo ayaṃ dārako abyatto, kathañhi nāma ayoniso aggiṃ gavesissatī’ti. Tassa pekkhamānassa araṇisahitaṃ gahetvā aggiṃ nibbattetvā taṃ dārakaṃ etadavoca – ‘evaṃ kho, tāta, aggi nibbattetabbo. Na tveva yathā tvaṃ bālo abyatto ayoniso aggiṃ gavesī’ti. Evameva kho tvaṃ, rājañña, bālo abyatto ayoniso paralokaṃ gavesissasi. Paṭinissajjetaṃ, rājañña, pāpakaṃ diṭṭhigataṃ, paṭinissajjetaṃ, rājañña, pāpakaṃ diṭṭhigataṃ, mā te ahosi dīgharattaṃ ahitāya dukkhāyā’’ti.

\paragraph{429.} ‘‘Kiñcāpi bhavaṃ kassapo evamāha, atha kho nevāhaṃ sakkomi idaṃ pāpakaṃ diṭṭhigataṃ paṭinissajjituṃ. Rājāpi maṃ pasenadi kosalo jānāti tirorājānopi – ‘pāyāsi rājañño evaṃvādī evaṃdiṭṭhī – ‘‘itipi natthi paro loko, natthi sattā opapātikā, natthi sukatadukkaṭānaṃ kammānaṃ phalaṃ vipāko’’ti. Sacāhaṃ, bho kassapa, idaṃ pāpakaṃ diṭṭhigataṃ paṭinissajjissāmi, bhavissanti me vattāro – ‘yāva bālo pāyāsi rājañño abyatto duggahitagāhī’ti. Kopenapi naṃ harissāmi, makkhenapi naṃ harissāmi, palāsenapi naṃ harissāmī’’ti.

\subsubsection{Dve satthavāhaupamā}

\paragraph{430.} ‘‘Tena hi, rājañña, upamaṃ te karissāmi. Upamāya midhekacce viññū purisā bhāsitassa atthaṃ ājānanti. Bhūtapubbaṃ, rājañña, mahāsakaṭasattho sakaṭasahassaṃ puratthimā janapadā pacchimaṃ janapadaṃ agamāsi. So yena yena gacchi, khippaṃyeva pariyādiyati tiṇakaṭṭhodakaṃ haritakapaṇṇaṃ. Tasmiṃ kho pana satthe dve satthavāhā ahesuṃ eko pañcannaṃ sakaṭasatānaṃ, eko pañcannaṃ sakaṭasatānaṃ. Atha kho tesaṃ satthavāhānaṃ etadahosi – ‘ayaṃ kho mahāsakaṭasattho sakaṭasahassaṃ; te mayaṃ yena yena gacchāma, khippameva pariyādiyati tiṇakaṭṭhodakaṃ haritakapaṇṇaṃ. Yaṃnūna mayaṃ imaṃ satthaṃ dvidhā vibhajeyyāma – ekato pañca sakaṭasatāni ekato pañca sakaṭasatānī’ti. Te taṃ satthaṃ dvidhā vibhajiṃsu\footnote{vibhajesuṃ (ka.)} ekato pañca sakaṭasatāni, ekato pañca sakaṭasatāni. Eko satthavāho bahuṃ tiṇañca kaṭṭhañca udakañca āropetvā satthaṃ payāpesi\footnote{pāyāpesi (sī. pī.)}. Dvīhatīhapayāto kho pana so sattho addasa purisaṃ kāḷaṃ lohitakkhaṃ\footnote{lohitakkhiṃ (syā.)} sannaddhakalāpaṃ\footnote{āsannaddhakalāpaṃ (syā.)} kumudamāliṃ allavatthaṃ allakesaṃ kaddamamakkhitehi cakkehi bhadrena rathena paṭipathaṃ āgacchantaṃ’, disvā etadavoca – ‘kuto, bho, āgacchasī’ti? ‘Amukamhā janapadā’ti. ‘Kuhiṃ gamissasī’ti? ‘Amukaṃ nāma janapada’nti. ‘Kacci, bho, purato kantāre mahāmegho abhippavuṭṭho’ti? ‘Evaṃ, bho, purato kantāre mahāmegho abhippavuṭṭho, āsittodakāni vaṭumāni, bahu tiṇañca kaṭṭhañca udakañca. Chaḍḍetha, bho, purāṇāni tiṇāni kaṭṭhāni udakāni, lahubhārehi sakaṭehi sīghaṃ sīghaṃ gacchatha, mā yoggāni kilamitthā’ti.

‘‘Atha kho so satthavāho satthike āmantesi – ‘ayaṃ, bho, puriso evamāha – ‘‘purato kantāre mahāmegho abhippavuṭṭho, āsittodakāni vaṭumāni, bahu tiṇañca kaṭṭhañca udakañca. Chaḍḍetha, bho, purāṇāni tiṇāni kaṭṭhāni udakāni, lahubhārehi sakaṭehi sīghaṃ sīghaṃ gacchatha, mā yoggāni kilamitthā’’ti. Chaḍḍetha, bho, purāṇāni tiṇāni kaṭṭhāni udakāni, lahubhārehi sakaṭehi satthaṃ payāpethā’ti. ‘Evaṃ, bho’ti kho te satthikā tassa satthavāhassa paṭissutvā chaḍḍetvā purāṇāni tiṇāni kaṭṭhāni udakāni lahubhārehi sakaṭehi satthaṃ payāpesuṃ. Te paṭhamepi satthavāse na addasaṃsu tiṇaṃ vā kaṭṭhaṃ vā udakaṃ vā. Dutiyepi satthavāse… tatiyepi satthavāse… catutthepi satthavāse… pañcamepi satthavāse… chaṭṭhepi satthavāse… sattamepi satthavāse na addasaṃsu tiṇaṃ vā kaṭṭhaṃ vā udakaṃ vā. Sabbeva anayabyasanaṃ āpajjiṃsu. Ye ca tasmiṃ satthe ahesuṃ manussā vā pasū vā, sabbe so yakkho amanusso bhakkhesi. Aṭṭhikāneva sesāni.

‘‘Yadā aññāsi dutiyo satthavāho – ‘bahunikkhanto kho, bho, dāni so sattho’ti bahuṃ tiṇañca kaṭṭhañca udakañca āropetvā satthaṃ payāpesi. Dvīhatīhapayāto kho pana so sattho addasa purisaṃ kāḷaṃ lohitakkhaṃ sannaddhakalāpaṃ kumudamāliṃ allavatthaṃ allakesaṃ kaddamamakkhitehi cakkehi bhadrena rathena paṭipathaṃ āgacchantaṃ, disvā etadavoca – ‘kuto, bho, āgacchasī’ti? ‘Amukamhā janapadā’ti. ‘Kuhiṃ gamissasī’ti? ‘Amukaṃ nāma janapada’nti. ‘Kacci, bho, purato kantāre mahāmegho abhippavuṭṭho’ti? ‘Evaṃ, bho, purato kantāre mahāmegho abhippavuṭṭho. Āsittodakāni vaṭumāni, bahu tiṇañca kaṭṭhañca udakañca. Chaḍḍetha , bho, purāṇāni tiṇāni kaṭṭhāni udakāni, lahubhārehi sakaṭehi sīghaṃ sīghaṃ gacchatha, mā yoggāni kilamitthā’ti.

‘‘Atha kho so satthavāho satthike āmantesi – ‘ayaṃ, bho, ‘‘puriso evamāha – purato kantāre mahāmegho abhippavuṭṭho, āsittodakāni vaṭumāni, bahu tiṇañca kaṭṭhañca udakañca. Chaḍḍetha, bho, purāṇāni tiṇāni kaṭṭhāni udakāni, lahubhārehi sakaṭehi sīghaṃ sīghaṃ gacchatha; mā yoggāni kilamitthā’’ti. Ayaṃ bho puriso neva amhākaṃ mitto, na ñātisālohito, kathaṃ mayaṃ imassa saddhāya gamissāma. Na vo chaḍḍetabbāni purāṇāni tiṇāni kaṭṭhāni udakāni, yathābhatena bhaṇḍena satthaṃ payāpetha. Na no purāṇaṃ chaḍḍessāmā’ti. ‘Evaṃ, bho’ti kho te satthikā tassa satthavāhassa paṭissutvā yathābhatena bhaṇḍena satthaṃ payāpesuṃ. Te paṭhamepi satthavāse na addasaṃsu tiṇaṃ vā kaṭṭhaṃ vā udakaṃ vā. Dutiyepi satthavāse… tatiyepi satthavāse… catutthepi satthavāse… pañcamepi satthavāse… chaṭṭhepi satthavāse… sattamepi satthavāse na addasaṃsu tiṇaṃ vā kaṭṭhaṃ vā udakaṃ vā. Tañca satthaṃ addasaṃsu anayabyasanaṃ āpannaṃ. Ye ca tasmiṃ satthepi ahesuṃ manussā vā pasū vā, tesañca aṭṭhikāneva addasaṃsu tena yakkhena amanussena bhakkhitānaṃ.

‘‘Atha kho so satthavāho satthike āmantesi – ‘ayaṃ kho, bho, sattho anayabyasanaṃ āpanno, yathā taṃ tena bālena satthavāhena pariṇāyakena. Tena hi, bho, yānamhākaṃ satthe appasārāni paṇiyāni, tāni chaḍḍetvā, yāni imasmiṃ satthe mahāsārāni paṇiyāni, tāni ādiyathā’ti. ‘Evaṃ, bho’ti kho te satthikā tassa satthavāhassa paṭissutvā yāni sakasmiṃ satthe appasārāni paṇiyāni, tāni chaḍḍetvā yāni tasmiṃ satthe mahāsārāni paṇiyāni, tāni ādiyitvā sotthinā taṃ kantāraṃ nitthariṃsu, yathā taṃ paṇḍitena satthavāhena pariṇāyakena. Evameva kho tvaṃ, rājañña, bālo abyatto anayabyasanaṃ āpajjissasi ayoniso paralokaṃ gavesanto seyyathāpi so purimo satthavāho. Yepi tava\footnote{te (ka.)} sotabbaṃ saddhātabbaṃ\footnote{saddahātabbaṃ (pī. ka.)} maññissanti, tepi anayabyasanaṃ āpajjissanti, seyyathāpi te satthikā. Paṭinissajjetaṃ, rājañña , pāpakaṃ diṭṭhigataṃ; paṭinissajjetaṃ, rājañña, pāpakaṃ diṭṭhigataṃ. Mā te ahosi dīgharattaṃ ahitāya dukkhāyā’’ti.

\paragraph{431.} ‘‘Kiñcāpi bhavaṃ kassapo evamāha, atha kho nevāhaṃ sakkomi idaṃ pāpakaṃ diṭṭhigataṃ paṭinissajjituṃ. Rājāpi maṃ pasenadi kosalo jānāti tirorājānopi – ‘pāyāsi rājañño evaṃvādī evaṃdiṭṭhī – ‘‘itipi natthi paro loko…pe… vipāko’’’ti. Sacāhaṃ, bho kassapa, idaṃ pāpakaṃ diṭṭhigataṃ paṭinissajjissāmi, bhavissanti me vattāro – ‘yāva bālo pāyāsi rājañño, abyatto duggahitagāhī’ti. Kopenapi naṃ harissāmi, makkhenapi naṃ harissāmi, palāsenapi naṃ harissāmī’’ti.

\subsubsection{Gūthabhārikaupamā}

\paragraph{432.} ‘‘Tena hi, rājañña, upamaṃ te karissāmi. Upamāya midhekacce viññū purisā bhāsitassa atthaṃ ājānanti. Bhūtapubbaṃ, rājañña, aññataro sūkaraposako puriso sakamhā gāmā aññaṃ gāmaṃ agamāsi. Tattha addasa pahūtaṃ sukkhagūthaṃ chaḍḍitaṃ. Disvānassa etadahosi – ‘ayaṃ kho pahuto sukkhagūtho chaḍḍito, mama ca sūkarabhattaṃ\footnote{sūkarānaṃ bhakkho (syā.)}; yaṃnūnāhaṃ ito sukkhagūthaṃ hareyya’nti. So uttarāsaṅgaṃ pattharitvā pahūtaṃ sukkhagūthaṃ ākiritvā bhaṇḍikaṃ bandhitvā sīse ubbāhetvā\footnote{uccāropetvā (ka. sī. ka.)} agamāsi. Tassa antarāmagge mahāakālamegho pāvassi. So uggharantaṃ paggharantaṃ yāva agganakhā gūthena makkhito gūthabhāraṃ ādāya agamāsi. Tamenaṃ manussā disvā evamāhaṃsu – ‘kacci no tvaṃ, bhaṇe, ummatto, kacci viceto, kathañhi nāma uggharantaṃ paggharantaṃ yāva agganakhā gūthena makkhito gūthabhāraṃ harissasī’ti. ‘Tumhe khvettha, bhaṇe, ummattā, tumhe vicetā, tathā hi pana me sūkarabhatta’nti. Evameva kho tvaṃ, rājañña, gūthabhārikūpamo\footnote{gūthahārikūpamo (sī. pī.)} maññe paṭibhāsi. Paṭinissajjetaṃ, rājañña, pāpakaṃ diṭṭhigataṃ. Paṭinissajjetaṃ, rājañña, pāpakaṃ diṭṭhigataṃ. Mā te ahosi dīgharattaṃ ahitāya dukkhāyā’’ti.

\paragraph{433.} ‘‘Kiñcāpi bhavaṃ kassapo evamāha, atha kho nevāhaṃ sakkomi idaṃ pāpakaṃ diṭṭhigataṃ paṭinissajjituṃ. Rājāpi maṃ pasenadi kosalo jānāti tirorājānopi – ‘pāyāsi rājañño evaṃvādī evaṃdiṭṭhī – ‘‘itipi natthi paro loko…pe… vipāko’’ti. Sacāhaṃ, bho kassapa, idaṃ pāpakaṃ diṭṭhigataṃ paṭinissajjissāmi, bhavissanti me vattāro – ‘yāva bālo pāyāsi rājañño abyatto duggahitagāhī’ti. Kopenapi naṃ harissāmi, makkhenapi naṃ harissāmi, palāsenapi naṃ harissāmī’’ti.

\subsubsection{Akkhadhuttakaupamā}

\paragraph{434.} ‘‘Tena hi, rājañña, upamaṃ te karissāmi, upamāya midhekacce viññū purisā bhāsitassa atthaṃ ājānanti. Bhūtapubbaṃ, rājañña, dve akkhadhuttā akkhehi dibbiṃsu. Eko akkhadhutto āgatāgataṃ kaliṃ gilati. Addasā kho dutiyo akkhadhutto taṃ akkhadhuttaṃ āgatāgataṃ kaliṃ gilantaṃ, disvā taṃ akkhadhuttaṃ etadavoca – ‘tvaṃ kho, samma, ekantikena jināsi, dehi me, samma, akkhe pajohissāmī’ti. ‘Evaṃ sammā’ti kho so akkhadhutto tassa akkhadhuttassa akkhe pādāsi. Atha kho so akkhadhutto akkhe visena paribhāvetvā taṃ akkhadhuttaṃ etadavoca – ‘ehi kho, samma, akkhehi dibbissāmā’ti. ‘Evaṃ sammā’ti kho so akkhadhutto tassa akkhadhuttassa paccassosi. Dutiyampi kho te akkhadhuttā akkhehi dibbiṃsu. Dutiyampi kho so akkhadhutto āgatāgataṃ kaliṃ gilati. Addasā kho dutiyo akkhadhutto taṃ akkhadhuttaṃ dutiyampi āgatāgataṃ kaliṃ gilantaṃ, disvā taṃ akkhadhuttaṃ etadavoca –

‘‘Littaṃ paramena tejasā, gilamakkhaṃ puriso na bujjhati;

Gila re gila pāpadhuttaka\footnote{gili re pāpadhuttaka (ka.)}, pacchā te kaṭukaṃ bhavissatīti.

‘‘Evameva kho tvaṃ, rājañña, akkhadhuttakūpamo maññe paṭibhāsi. Paṭinissajjetaṃ, rājañña, pāpakaṃ diṭṭhigataṃ; paṭinissajjetaṃ, rājañña, pāpakaṃ diṭṭhigataṃ. Mā te ahosi dīgharattaṃ ahitāya dukkhāyā’’ti.

\paragraph{435.} ‘‘Kiñcāpi bhavaṃ kassapo evamāha, atha kho nevāhaṃ sakkomi idaṃ pāpakaṃ diṭṭhigataṃ paṭinissajjituṃ. Rājāpi maṃ pasenadi kosalo jānāti tirorājānopi – ‘pāyāsi rājañño evaṃvādī evaṃdiṭṭhī – ‘‘itipi natthi paro loko…pe… vipāko’’ti. Sacāhaṃ, bho kassapa, idaṃ pāpakaṃ diṭṭhigataṃ paṭinissajjissāmi, bhavissanti me vattāro – ‘yāva bālo pāyāsi rājañño abyatto duggahitagāhī’ti. Kopenapi naṃ harissāmi, makkhenapi naṃ harissāmi, palāsenapi naṃ harissāmī’’ti.

\subsubsection{Sāṇabhārikaupamā}

\paragraph{436.} ‘‘Tena hi, rājañña, upamaṃ te karissāmi, upamāya midhekacce viññū purisā bhāsitassa atthaṃ ājānanti. Bhūtapubbaṃ, rājañña, aññataro janapado vuṭṭhāsi. Atha kho sahāyako sahāyakaṃ āmantesi – ‘āyāma, samma, yena so janapado tenupasaṅkamissāma, appeva nāmettha kiñci dhanaṃ adhigaccheyyāmā’ti. ‘Evaṃ sammā’ti kho sahāyako sahāyakassa paccassosi. Te yena so janapado, yena aññataraṃ gāmapaṭṭaṃ\footnote{gāmapajjaṃ (syā.), gāmapattaṃ (sī.)} tenupasaṅkamiṃsu , tattha addasaṃsu pahūtaṃ sāṇaṃ chaḍḍitaṃ, disvā sahāyako sahāyakaṃ āmantesi – ‘idaṃ kho, samma, pahūtaṃ sāṇaṃ chaḍḍitaṃ, tena hi, samma, tvañca sāṇabhāraṃ bandha, ahañca sāṇabhāraṃ bandhissāmi, ubho sāṇabhāraṃ ādāya gamissāmā’ti. ‘Evaṃ sammā’ti kho sahāyako sahāyakassa paṭissutvā sāṇabhāraṃ bandhitvā te ubho sāṇabhāraṃ ādāya yena aññataraṃ gāmapaṭṭaṃ tenupasaṅkamiṃsu. Tattha addasaṃsu pahūtaṃ sāṇasuttaṃ chaḍḍitaṃ, disvā sahāyako sahāyakaṃ āmantesi – ‘yassa kho, samma, atthāya iccheyyāma sāṇaṃ, idaṃ pahūtaṃ sāṇasuttaṃ chaḍḍitaṃ. Tena hi, samma, tvañca sāṇabhāraṃ chaḍḍehi, ahañca sāṇabhāraṃ chaḍḍessāmi, ubho sāṇasuttabhāraṃ ādāya gamissāmā’ti. ‘Ayaṃ kho me, samma, sāṇabhāro dūrābhato ca susannaddho ca, alaṃ me tvaṃ pajānāhī’ti. Atha kho so sahāyako sāṇabhāraṃ chaḍḍetvā sāṇasuttabhāraṃ ādiyi.

‘‘Te yena aññataraṃ gāmapaṭṭaṃ tenupasaṅkamiṃsu. Tattha addasaṃsu pahūtā sāṇiyo chaḍḍitā, disvā sahāyako sahāyakaṃ āmantesi – ‘yassa kho , samma, atthāya iccheyyāma sāṇaṃ vā sāṇasuttaṃ vā, imā pahūtā sāṇiyo chaḍḍitā. Tena hi, samma, tvañca sāṇabhāraṃ chaḍḍehi, ahañca sāṇasuttabhāraṃ chaḍḍessāmi, ubho sāṇibhāraṃ ādāya gamissāmā’ti . ‘Ayaṃ kho me, samma, sāṇabhāro dūrābhato ca susannaddho ca, alaṃ me, tvaṃ pajānāhī’ti. Atha kho so sahāyako sāṇasuttabhāraṃ chaḍḍetvā sāṇibhāraṃ ādiyi.

‘‘Te yena aññataraṃ gāmapaṭṭaṃ tenupasaṅkamiṃsu. Tattha addasaṃsu pahūtaṃ khomaṃ chaḍḍitaṃ, disvā…pe… pahūtaṃ khomasuttaṃ chaḍḍitaṃ, disvā… pahūtaṃ khomadussaṃ chaḍḍitaṃ, disvā… pahūtaṃ kappāsaṃ chaḍḍitaṃ, disvā… pahūtaṃ kappāsikasuttaṃ chaḍḍitaṃ, disvā… pahūtaṃ kappāsikadussaṃ chaḍḍitaṃ, disvā… pahūtaṃ ayaṃ\footnote{ayasaṃ (syā.)} chaḍḍitaṃ, disvā… pahūtaṃ lohaṃ chaḍḍitaṃ, disvā… pahūtaṃ tipuṃ chaḍḍitaṃ, disvā… pahūtaṃ sīsaṃ chaḍḍitaṃ, disvā… pahūtaṃ sajjhaṃ\footnote{sajjhuṃ (sī. syā. pī.)} chaḍḍitaṃ, disvā… pahūtaṃ suvaṇṇaṃ chaḍḍitaṃ, disvā sahāyako sahāyakaṃ āmantesi – ‘yassa kho, samma, atthāya iccheyyāma sāṇaṃ vā sāṇasuttaṃ vā sāṇiyo vā khomaṃ vā khomasuttaṃ vā khomadussaṃ vā kappāsaṃ vā kappāsikasuttaṃ vā kappāsikadussaṃ vā ayaṃ vā lohaṃ vā tipuṃ vā sīsaṃ vā sajjhaṃ vā, idaṃ pahūtaṃ suvaṇṇaṃ chaḍḍitaṃ. Tena hi, samma, tvañca sāṇabhāraṃ chaḍḍehi, ahañca sajjhabhāraṃ\footnote{sajjhubhāraṃ (sī. syā. pī.)} chaḍḍessāmi, ubho suvaṇṇabhāraṃ ādāya gamissāmā’ti. ‘Ayaṃ kho me, samma, sāṇabhāro dūrābhato ca susannaddho ca, alaṃ me tvaṃ pajānāhī’ti. Atha kho so sahāyako sajjhabhāraṃ chaḍḍetvā suvaṇṇabhāraṃ ādiyi.

‘‘Te yena sako gāmo tenupasaṅkamiṃsu. Tattha yo so sahāyako sāṇabhāraṃ ādāya agamāsi, tassa neva mātāpitaro abhinandiṃsu, na puttadārā abhinandiṃsu, na mittāmaccā abhinandiṃsu, na ca tatonidānaṃ sukhaṃ somanassaṃ adhigacchi. Yo pana so sahāyako suvaṇṇabhāraṃ ādāya agamāsi, tassa mātāpitaropi abhinandiṃsu, puttadārāpi abhinandiṃsu, mittāmaccāpi abhinandiṃsu, tatonidānañca sukhaṃ somanassaṃ adhigacchi. ‘‘Evameva kho tvaṃ, rājañña, sāṇabhārikūpamo maññe paṭibhāsi. Paṭinissajjetaṃ, rājañña, pāpakaṃ diṭṭhigataṃ; paṭinissajjetaṃ, rājañña, pāpakaṃ diṭṭhigataṃ. Mā te ahosi dīgharattaṃ ahitāya dukkhāyā’’ti.

\subsubsection{Saraṇagamanaṃ}

\paragraph{437.} ‘‘Purimeneva ahaṃ opammena bhoto kassapassa attamano abhiraddho. Api cāhaṃ imāni vicitrāni pañhāpaṭibhānāni sotukāmo evāhaṃ bhavantaṃ kassapaṃ paccanīkaṃ kātabbaṃ amaññissaṃ. Abhikkantaṃ, bho kassapa, abhikkantaṃ, bho kassapa. Seyyathāpi, bho kassapa, nikkujjitaṃ vā ukkujjeyya, paṭicchannaṃ vā vivareyya, mūḷhassa vā maggaṃ ācikkheyya, andhakāre vā telapajjotaṃ dhāreyya ‘cakkhumanto rūpāni dakkhantī’ti evamevaṃ bhotā kassapena anekapariyāyena dhammo pakāsito. Esāhaṃ, bho kassapa, taṃ bhavantaṃ gotamaṃ saraṇaṃ gacchāmi, dhammañca, bhikkhusaṅghañca. Upāsakaṃ maṃ bhavaṃ kassapo dhāretu ajjatagge pāṇupetaṃ saraṇaṃ gataṃ.

‘‘Icchāmi cāhaṃ, bho kassapa, mahāyaññaṃ yajituṃ, anusāsatu maṃ bhavaṃ kassapo, yaṃ mamassa dīgharattaṃ hitāya sukhāyā’’ti.

\subsubsection{Yaññakathā}

\paragraph{438.} ‘‘Yathārūpe kho, rājañña, yaññe gāvo vā haññanti ajeḷakā vā haññanti, kukkuṭasūkarā vā haññanti, vividhā vā pāṇā saṃghātaṃ āpajjanti, paṭiggāhakā ca honti micchādiṭṭhī micchāsaṅkappā micchāvācā micchākammantā micchāājīvā micchāvāyāmā micchāsatī micchāsamādhī, evarūpo kho, rājañña, yañño na mahapphalo hoti na mahānisaṃso na mahājutiko na mahāvipphāro. Seyyathāpi, rājañña, kassako bījanaṅgalaṃ ādāya vanaṃ paviseyya. So tattha dukkhette dubbhūme avihatakhāṇukaṇṭake bījāni patiṭṭhāpeyya khaṇḍāni pūtīni vātātapahatāni asāradāni asukhasayitāni. Devo ca na kālena kālaṃ sammādhāraṃ anuppaveccheyya. Api nu tāni bījāni vuddhiṃ virūḷhiṃ\footnote{viruḷhiṃ (moggalāne)} vepullaṃ āpajjeyyuṃ, kassako vā vipulaṃ phalaṃ adhigaccheyyā’’ti? ‘‘No hidaṃ\footnote{na evaṃ (syā. ka.)} bho kassapa’’. ‘‘Evameva kho, rājañña, yathārūpe yaññe gāvo vā haññanti, ajeḷakā vā haññanti, kukkuṭasūkarā vā haññanti, vividhā vā pāṇā saṃghātaṃ āpajjanti, paṭiggāhakā ca honti micchādiṭṭhī micchāsaṅkappā micchāvācā micchākammantā micchāājīvā micchāvāyāmā micchāsatī micchāsamādhī, evarūpo kho , rājañña, yañño na mahapphalo hoti na mahānisaṃso na mahājutiko na mahāvipphāro.

‘‘Yathārūpe ca kho, rājañña, yaññe neva gāvo haññanti, na ajeḷakā haññanti, na kukkuṭasūkarā haññanti, na vividhā vā pāṇā saṃghātaṃ āpajjanti, paṭiggāhakā ca honti sammādiṭṭhī sammāsaṅkappā sammāvācā sammākammantā sammāājīvā sammāvāyāmā sammāsatī sammāsamādhī, evarūpo kho, rājañña, yañño mahapphalo hoti mahānisaṃso mahājutiko mahāvipphāro. Seyyathāpi, rājañña, kassako bījanaṅgalaṃ ādāya vanaṃ paviseyya. So tattha sukhette subhūme suvihatakhāṇukaṇṭake bījāni patiṭṭhapeyya akhaṇḍāni apūtīni avātātapahatāni sāradāni sukhasayitāni. Devo ca kālena kālaṃ sammādhāraṃ anuppaveccheyya. Api nu tāni bījāni vuddhiṃ virūḷhiṃ vepullaṃ āpajjeyyuṃ, kassako vā vipulaṃ phalaṃ adhigaccheyyā’’ti? ‘‘Evaṃ, bho kassapa’’. ‘‘Evameva kho, rājañña, yathārūpe yaññe neva gāvo haññanti, na ajeḷakā haññanti, na kukkuṭasūkarā haññanti, na vividhā vā pāṇā saṃghātaṃ āpajjanti, paṭiggāhakā ca honti sammādiṭṭhī sammāsaṅkappā sammāvācā sammākammantā sammāājīvā sammāvāyāmā sammāsatī sammāsamādhī, evarūpo kho, rājañña, yañño mahapphalo hoti mahānisaṃso mahājutiko mahāvipphāro’’ti.

\subsubsection{Uttaramāṇavavatthu}

\paragraph{439.} Atha kho pāyāsi rājañño dānaṃ paṭṭhapesi samaṇabrāhmaṇakapaṇaddhikavaṇibbakayācakānaṃ . Tasmiṃ kho pana dāne evarūpaṃ bhojanaṃ dīyati kaṇājakaṃ bilaṅgadutiyaṃ, dhorakāni\footnote{thorakāni (sī. pī.), corakāni (syā.)} ca vatthāni guḷavālakāni\footnote{guḷagāḷakāni (ka.)}. Tasmiṃ kho pana dāne uttaro nāma māṇavo vāvaṭo\footnote{byāvaṭo (sī. pī.)} ahosi. So dānaṃ datvā evaṃ anuddisati – ‘‘imināhaṃ dānena pāyāsiṃ rājaññameva imasmiṃ loke samāgacchiṃ, mā parasmi’’nti. Assosi kho pāyāsi rājañño – ‘‘uttaro kira māṇavo dānaṃ datvā evaṃ anuddisati – ‘imināhaṃ dānena pāyāsiṃ rājaññameva imasmiṃ loke samāgacchiṃ, mā parasmi’’’nti. Atha kho pāyāsi rājañño uttaraṃ māṇavaṃ āmantāpetvā etadavoca – ‘‘saccaṃ kira tvaṃ, tāta uttara, dānaṃ datvā evaṃ anuddisasi – ‘imināhaṃ dānena pāyāsiṃ rājaññameva imasmiṃ loke samāgacchiṃ, mā parasmi’’’nti? ‘‘Evaṃ, bho’’. ‘‘Kissa pana tvaṃ, tāta uttara, dānaṃ datvā evaṃ anuddisasi – ‘imināhaṃ dānena pāyāsiṃ rājaññameva imasmiṃ loke samāgacchiṃ, mā parasmi’nti? Nanu mayaṃ, tāta uttara, puññatthikā dānasseva phalaṃ pāṭikaṅkhino’’ti? ‘‘Bhoto kho dāne evarūpaṃ bhojanaṃ dīyati kaṇājakaṃ bilaṅgadutiyaṃ, yaṃ bhavaṃ pādāpi\footnote{pādāsi (ka.)} na iccheyya samphusituṃ\footnote{chupituṃ (pī. ka.)}, kuto bhuñjituṃ, dhorakāni ca vatthāni guḷavālakāni, yāni bhavaṃ pādāpi\footnote{acittikataṃ (ka.)} na iccheyya samphusituṃ, kuto paridahituṃ. Bhavaṃ kho panamhākaṃ piyo manāpo, kathaṃ mayaṃ manāpaṃ amanāpena saṃyojemā’’ti? ‘‘Tena hi tvaṃ, tāta uttara, yādisāhaṃ bhojanaṃ bhuñjāmi, tādisaṃ bhojanaṃ paṭṭhapehi. Yādisāni cāhaṃ vatthāni paridahāmi, tādisāni ca vatthāni paṭṭhapehī’’ti. ‘‘Evaṃ, bho’’ti kho uttaro māṇavo pāyāsissa rājaññassa paṭissutvā yādisaṃ bhojanaṃ pāyāsi rājañño bhuñjati, tādisaṃ bhojanaṃ paṭṭhapesi. Yādisāni ca vatthāni pāyāsi rājañño paridahati, tādisāni ca vatthāni paṭṭhapesi.

\paragraph{440.} Atha kho pāyāsi rājañño asakkaccaṃ dānaṃ datvā asahatthā dānaṃ datvā acittīkataṃ dānaṃ datvā apaviddhaṃ dānaṃ datvā kāyassa bhedā paraṃ maraṇā cātumahārājikānaṃ devānaṃ sahabyataṃ upapajji suññaṃ serīsakaṃ vimānaṃ. Yo pana tassa dāne vāvaṭo ahosi uttaro nāma māṇavo. So sakkaccaṃ dānaṃ datvā sahatthā dānaṃ datvā cittīkataṃ dānaṃ datvā anapaviddhaṃ dānaṃ datvā kāyassa bhedā paraṃ maraṇā sugatiṃ saggaṃ lokaṃ upapajji devānaṃ tāvatiṃsānaṃ sahabyataṃ.

\subsubsection{Pāyāsidevaputto}

\paragraph{441.} Tena kho pana samayena āyasmā gavampati abhikkhaṇaṃ suññaṃ serīsakaṃ vimānaṃ divāvihāraṃ gacchati. Atha kho pāyāsi devaputto yenāyasmā gavampati tenupasaṅkami; upasaṅkamitvā āyasmantaṃ gavampatiṃ abhivādetvā ekamantaṃ aṭṭhāsi. Ekamantaṃ ṭhitaṃ kho pāyāsiṃ devaputtaṃ āyasmā gavampati etadavoca – ‘‘kosi tvaṃ, āvuso’’ti? ‘‘Ahaṃ, bhante, pāyāsi rājañño’’ti. ‘‘Nanu tvaṃ, āvuso, evaṃdiṭṭhiko ahosi – ‘itipi natthi paro loko, natthi sattā opapātikā, natthi sukatadukkaṭānaṃ kammānaṃ phalaṃ vipāko’’’ti? ‘‘Saccāhaṃ, bhante, evaṃdiṭṭhiko ahosiṃ – ‘itipi natthi paro loko, natthi sattā opapātikā, natthi sukatadukkaṭānaṃ kammānaṃ phalaṃ vipāko’ti. Api cāhaṃ ayyena kumārakassapena etasmā pāpakā diṭṭhigatā vivecito’’ti. ‘‘Yo pana te, āvuso, dāne vāvaṭo ahosi uttaro nāma māṇavo, so kuhiṃ upapanno’’ti? ‘‘Yo me, bhante, dāne vāvaṭo ahosi uttaro nāma māṇavo, so sakkaccaṃ dānaṃ datvā sahatthā dānaṃ datvā cittīkataṃ dānaṃ datvā anapaviddhaṃ dānaṃ datvā kāyassa bhedā paraṃ maraṇā sugatiṃ saggaṃ lokaṃ upapanno devānaṃ tāvatiṃsānaṃ sahabyataṃ. Ahaṃ pana, bhante, asakkaccaṃ dānaṃ datvā asahatthā dānaṃ datvā acittīkataṃ dānaṃ datvā apaviddhaṃ dānaṃ datvā kāyassa bhedā paraṃ maraṇā cātumahārājikānaṃ devānaṃ sahabyataṃ upapanno suññaṃ serīsakaṃ vimānaṃ. Tena hi, bhante gavampati, manussalokaṃ gantvā evamārocehi – ‘sakkaccaṃ dānaṃ detha, sahatthā dānaṃ detha, cittīkataṃ dānaṃ detha, anapaviddhaṃ dānaṃ detha. Pāyāsi rājañño asakkaccaṃ dānaṃ datvā asahatthā dānaṃ datvā acittīkataṃ dānaṃ datvā apaviddhaṃ dānaṃ datvā kāyassa bhedā paraṃ maraṇā cātumahārājikānaṃ devānaṃ sahabyataṃ upapanno suññaṃ serīsakaṃ vimānaṃ. Yo pana tassa dāne vāvaṭo ahosi uttaro nāma māṇavo, so sakkaccaṃ dānaṃ datvā sahatthā dānaṃ datvā cittīkataṃ dānaṃ datvā anapaviddhaṃ dānaṃ datvā kāyassa bhedā paraṃ maraṇā sugatiṃ saggaṃ lokaṃ upapanno devānaṃ tāvatiṃsānaṃ sahabyata’’’nti.

Atha kho āyasmā gavampati manussalokaṃ āgantvā evamārocesi – ‘‘sakkaccaṃ dānaṃ detha, sahatthā dānaṃ detha, cittīkataṃ dānaṃ detha, anapaviddhaṃ dānaṃ detha. Pāyāsi rājañño asakkaccaṃ dānaṃ datvā asahatthā dānaṃ datvā acittīkataṃ dānaṃ datvā apaviddhaṃ dānaṃ datvā kāyassa bhedā paraṃ maraṇā cātumahārājikānaṃ devānaṃ sahabyataṃ upapanno suññaṃ serīsakaṃ vimānaṃ. Yo pana tassa dāne vāvaṭo ahosi uttaro nāma māṇavo, so sakkaccaṃ dānaṃ datvā sahatthā dānaṃ datvā cittīkataṃ dānaṃ datvā anapaviddhaṃ dānaṃ datvā kāyassa bhedā paraṃ maraṇā sugatiṃ saggaṃ lokaṃ upapanno devānaṃ tāvatiṃsānaṃ sahabyata’’nti.

\xsectionEnd{Pāyāsisuttaṃ niṭṭhitaṃ dasamaṃ.\\ Mahāvaggo niṭṭhito.}

\paragraph{}
Tassuddānaṃ –

Mahāpadāna nidānaṃ, nibbānañca sudassanaṃ;

Janavasabha govindaṃ, samayaṃ sakkapañhakaṃ;

Mahāsatipaṭṭhānañca, pāyāsi dasamaṃ bhave\footnote{satipaṭṭhānapāyāsi, mahāvaggassa saṅgaho (sī. pī.) satipaṭṭhānapāyāsi, mahāvaggoti vuccatīti (syā.)}.

\xsectionEnd{Mahāvaggapāḷi niṭṭhitā.}





\xchapter{3}{Pāthikavaggapāḷi}


\section{Pāthikasuttaṃ}

\subsubsection{Sunakkhattavatthu}

\paragraph{1.} Evaṃ me sutaṃ – ekaṃ samayaṃ bhagavā mallesu viharati anupiyaṃ nāma\footnote{anuppiyaṃ nāma (syā.)} mallānaṃ nigamo. Atha kho bhagavā pubbaṇhasamayaṃ nivāsetvā pattacīvaramādāya anupiyaṃ piṇḍāya pāvisi. Atha kho bhagavato etadahosi – ‘‘atippago kho tāva anupiyāyaṃ\footnote{anupiyaṃ (ka.)} piṇḍāya carituṃ. Yaṃnūnāhaṃ yena bhaggavagottassa paribbājakassa ārāmo, yena bhaggavagotto paribbājako tenupasaṅkameyya’’nti.

\paragraph{2.} Atha kho bhagavā yena bhaggavagottassa paribbājakassa ārāmo, yena bhaggavagotto paribbājako tenupasaṅkami. Atha kho bhaggavagotto paribbājako bhagavantaṃ etadavoca – ‘‘etu kho, bhante, bhagavā. Svāgataṃ, bhante, bhagavato. Cirassaṃ kho, bhante, bhagavā imaṃ pariyāyamakāsi yadidaṃ idhāgamanāya. Nisīdatu, bhante, bhagavā, idamāsanaṃ paññatta’’nti. Nisīdi bhagavā paññatte āsane. Bhaggavagottopi kho paribbājako aññataraṃ nīcaṃ āsanaṃ gahetvā ekamantaṃ nisīdi. Ekamantaṃ nisinno kho bhaggavagotto paribbājako bhagavantaṃ etadavoca – ‘‘purimāni, bhante, divasāni purimatarāni sunakkhatto licchaviputto yenāhaṃ tenupasaṅkami; upasaṅkamitvā maṃ etadavoca – ‘paccakkhāto dāni mayā, bhaggava, bhagavā. Na dānāhaṃ bhagavantaṃ uddissa viharāmī’ti. Kaccetaṃ, bhante, tatheva, yathā sunakkhatto licchaviputto avacā’’ti? ‘‘Tatheva kho etaṃ, bhaggava, yathā sunakkhatto licchaviputto avaca’’.

\paragraph{3.} Purimāni, bhaggava, divasāni purimatarāni sunakkhatto licchaviputto yenāhaṃ tenupasaṅkami; upasaṅkamitvā maṃ abhivādetvā ekamantaṃ nisīdi. Ekamantaṃ nisinno kho, bhaggava, sunakkhatto licchaviputto maṃ etadavoca – ‘paccakkhāmi dānāhaṃ, bhante, bhagavantaṃ. Na dānāhaṃ, bhante, bhagavantaṃ uddissa viharissāmī’ti. ‘Evaṃ vutte, ahaṃ, bhaggava, sunakkhattaṃ licchaviputtaṃ etadavocaṃ – ‘api nu tāhaṃ, sunakkhatta, evaṃ avacaṃ, ehi tvaṃ, sunakkhatta, mamaṃ uddissa viharāhī’ti? ‘No hetaṃ, bhante’. ‘Tvaṃ vā pana maṃ evaṃ avaca – ahaṃ, bhante, bhagavantaṃ uddissa viharissāmī’ti? ‘No hetaṃ, bhante’. ‘Iti kira, sunakkhatta, nevāhaṃ taṃ vadāmi – ehi tvaṃ, sunakkhatta, mamaṃ uddissa viharāhīti. Napi kira maṃ tvaṃ vadesi – ahaṃ, bhante, bhagavantaṃ uddissa viharissāmīti. Evaṃ sante, moghapurisa, ko santo kaṃ paccācikkhasi? Passa, moghapurisa, yāvañca\footnote{yāva ca (ka.)} te idaṃ aparaddha’nti.

\paragraph{4.} ‘Na hi pana me, bhante, bhagavā uttarimanussadhammā iddhipāṭihāriyaṃ karotī’ti. ‘Api nu tāhaṃ, sunakkhatta, evaṃ avacaṃ – ehi tvaṃ, sunakkhatta, mamaṃ uddissa viharāhi, ahaṃ te uttarimanussadhammā iddhipāṭihāriyaṃ karissāmī’ti? ‘No hetaṃ, bhante’. ‘Tvaṃ vā pana maṃ evaṃ avaca – ahaṃ, bhante, bhagavantaṃ uddissa viharissāmi, bhagavā me uttarimanussadhammā iddhipāṭihāriyaṃ karissatī’ti? ‘No hetaṃ, bhante’. ‘Iti kira, sunakkhatta, nevāhaṃ taṃ vadāmi – ehi tvaṃ, sunakkhatta, mamaṃ uddissa viharāhi, ahaṃ te uttarimanussadhammā iddhipāṭihāriyaṃ karissāmī’ti; napi kira maṃ tvaṃ vadesi – ahaṃ, bhante, bhagavantaṃ uddissa viharissāmi, bhagavā me uttarimanussadhammā iddhipāṭihāriyaṃ karissatī’ti. Evaṃ sante, moghapurisa , ko santo kaṃ paccācikkhasi? Taṃ kiṃ maññasi, sunakkhatta, kate vā uttarimanussadhammā iddhipāṭihāriye akate vā uttarimanussadhammā iddhipāṭihāriye yassatthāya mayā dhammo desito so niyyāti takkarassa sammā dukkhakkhayāyā’ti? ‘Kate vā, bhante, uttarimanussadhammā iddhipāṭihāriye akate vā uttarimanussadhammā iddhipāṭihāriye yassatthāya bhagavatā dhammo desito so niyyāti takkarassa sammā dukkhakkhayāyā’ti. ‘Iti kira, sunakkhatta, kate vā uttarimanussadhammā iddhipāṭihāriye, akate vā uttarimanussadhammā iddhipāṭihāriye, yassatthāya mayā dhammo desito, so niyyāti takkarassa sammā dukkhakkhayāya. Tatra, sunakkhatta, kiṃ uttarimanussadhammā iddhipāṭihāriyaṃ kataṃ karissati? Passa, moghapurisa, yāvañca te idaṃ aparaddha’nti.

\paragraph{5.} ‘Na hi pana me, bhante, bhagavā aggaññaṃ paññapetī’ti\footnote{paññāpetīti (pī.)}? ‘Api nu tāhaṃ, sunakkhatta, evaṃ avacaṃ – ehi tvaṃ, sunakkhatta, mamaṃ uddissa viharāhi, ahaṃ te aggaññaṃ paññapessāmī’ti? ‘No hetaṃ, bhante’. ‘Tvaṃ vā pana maṃ evaṃ avaca – ahaṃ, bhante, bhagavantaṃ uddissa viharissāmi, bhagavā me aggaññaṃ paññapessatī’ti? ‘No hetaṃ, bhante’. ‘Iti kira, sunakkhatta, nevāhaṃ taṃ vadāmi – ehi tvaṃ, sunakkhatta, mamaṃ uddissa viharāhi, ahaṃ te aggaññaṃ paññapessāmīti. Napi kira maṃ tvaṃ vadesi – ahaṃ, bhante, bhagavantaṃ uddissa viharissāmi, bhagavā me aggaññaṃ paññapessatī’ti. Evaṃ sante, moghapurisa, ko santo kaṃ paccācikkhasi? Taṃ kiṃ maññasi, sunakkhatta, paññatte vā aggaññe, apaññatte vā aggaññe, yassatthāya mayā dhammo desito, so niyyāti takkarassa sammā dukkhakkhayāyā’ti? ‘Paññatte vā, bhante, aggaññe, apaññatte vā aggaññe, yassatthāya bhagavatā dhammo desito, so niyyāti takkarassa sammā dukkhakkhayāyā’ti. ‘Iti kira, sunakkhatta, paññatte vā aggaññe, apaññatte vā aggaññe, yassatthāya mayā dhammo desito, so niyyāti takkarassa sammā dukkhakkhayāya. Tatra, sunakkhatta, kiṃ aggaññaṃ paññattaṃ karissati? Passa, moghapurisa, yāvañca te idaṃ aparaddhaṃ’.

\paragraph{6.} ‘Anekapariyāyena kho te, sunakkhatta, mama vaṇṇo bhāsito vajjigāme – itipi so bhagavā arahaṃ sammāsambuddho vijjācaraṇasampanno sugato lokavidū anuttaro purisadammasārathi satthā devamanussānaṃ buddho bhagavāti. Iti kho te, sunakkhatta, anekapariyāyena mama vaṇṇo bhāsito vajjigāme.

‘Anekapariyāyena kho te, sunakkhatta, dhammassa vaṇṇo bhāsito vajjigāme – svākkhāto bhagavatā dhammo sandiṭṭhiko akāliko ehipassiko opaneyyiko paccattaṃ veditabbo viññūhīti. Iti kho te, sunakkhatta, anekapariyāyena dhammassa vaṇṇo bhāsito vajjigāme.

‘Anekapariyāyena kho te, sunakkhatta, saṅghassa vaṇṇo bhāsito vajjigāme – suppaṭipanno bhagavato sāvakasaṅgho, ujuppaṭipanno bhagavato sāvakasaṅgho, ñāyappaṭipanno bhagavato sāvakasaṅgho, sāmīcippaṭipanno bhagavato sāvakasaṅgho, yadidaṃ cattāri purisayugāni aṭṭha purisapuggalā, esa bhagavato sāvakasaṅgho, āhuneyyo pāhuneyyo dakkhiṇeyyo añjalikaraṇīyo anuttaraṃ puññakkhettaṃ lokassāti. Iti kho te, sunakkhatta, anekapariyāyena saṅghassa vaṇṇo bhāsito vajjigāme.

‘Ārocayāmi kho te, sunakkhatta, paṭivedayāmi kho te, sunakkhatta. Bhavissanti kho te, sunakkhatta, vattāro, no visahi sunakkhatto licchaviputto samaṇe gotame brahmacariyaṃ carituṃ, so avisahanto sikkhaṃ paccakkhāya hīnāyāvattoti. Iti kho te, sunakkhatta, bhavissanti vattāro’ti.

Evaṃ pi kho, bhaggava, sunakkhatto licchaviputto mayā vuccamāno apakkameva imasmā dhammavinayā, yathā taṃ āpāyiko nerayiko.

\subsubsection{Korakkhattiyavatthu}

\paragraph{7.} ‘‘Ekamidāhaṃ, bhaggava, samayaṃ thūlūsu\footnote{bumūsu (sī. pī.)} viharāmi uttarakā nāma thūlūnaṃ nigamo. Atha khvāhaṃ, bhaggava, pubbaṇhasamayaṃ nivāsetvā pattacīvaramādāya sunakkhattena licchaviputtena pacchāsamaṇena uttarakaṃ piṇḍāya pāvisiṃ. Tena kho pana samayena acelo korakkhattiyo kukkuravatiko catukkuṇḍiko\footnote{catukuṇḍiko (sī. pī.) catukoṇḍiko (syā. ka.)} chamānikiṇṇaṃ bhakkhasaṃ mukheneva khādati, mukheneva bhuñjati. Addasā kho, bhaggava, sunakkhatto licchaviputto acelaṃ korakkhattiyaṃ kukkuravatikaṃ catukkuṇḍikaṃ chamānikiṇṇaṃ bhakkhasaṃ mukheneva khādantaṃ mukheneva bhuñjantaṃ. Disvānassa etadahosi – ‘sādhurūpo vata, bho, ayaṃ\footnote{arahaṃ (sī. syā. pī.)} samaṇo catukkuṇḍiko chamānikiṇṇaṃ bhakkhasaṃ mukheneva khādati, mukheneva bhuñjatī’ti.

‘‘Atha khvāhaṃ, bhaggava, sunakkhattassa licchaviputtassa cetasā cetoparivitakkamaññāya sunakkhattaṃ licchaviputtaṃ etadavocaṃ – ‘tvampi nāma, moghapurisa, samaṇo sakyaputtiyo\footnote{moghapurisa sakyaputtiyo (sī. syā. pī.)} paṭijānissasī’ti! ‘Kiṃ pana maṃ, bhante, bhagavā evamāha – ‘tvampi nāma, moghapurisa, samaṇo sakyaputtiyo\footnote{moghapurisa sakyaputtiyo (sī. syā. pī.)} paṭijānissasī’ti? ‘Nanu te, sunakkhatta, imaṃ acelaṃ korakkhattiyaṃ kukkuravatikaṃ catukkuṇḍikaṃ chamānikiṇṇaṃ bhakkhasaṃ mukheneva khādantaṃ mukheneva bhuñjantaṃ disvāna etadahosi – sādhurūpo vata, bho, ayaṃ samaṇo catukkuṇḍiko chamānikiṇṇaṃ bhakkhasaṃ mukheneva khādati, mukheneva bhuñjatī’ti? ‘Evaṃ, bhante. Kiṃ pana, bhante, bhagavā arahattassa maccharāyatī’ti? ‘Na kho ahaṃ, moghapurisa, arahattassa maccharāyāmi. Api ca, tuyhevetaṃ pāpakaṃ diṭṭhigataṃ uppannaṃ, taṃ pajaha. Mā te ahosi dīgharattaṃ ahitāya dukkhāya. Yaṃ kho panetaṃ, sunakkhatta, maññasi acelaṃ korakkhattiyaṃ – sādhurūpo ayaṃ samaṇoti\footnote{maññasi ‘‘acelo korakhattiyo sādhurūpo arahaṃ samaṇoti’’ (syā.)}. So sattamaṃ divasaṃ alasakena kālaṅkarissati. Kālaṅkato\footnote{kālakato (sī. syā. pī.)} ca kālakañcikā\footnote{kālakañjā (sī. pī.), kālakañjikā (syā.)} nāma asurā sabbanihīno asurakāyo, tatra upapajjissati. Kālaṅkatañca naṃ bīraṇatthambake susāne chaḍḍessanti. Ākaṅkhamāno ca tvaṃ, sunakkhatta, acelaṃ korakkhattiyaṃ upasaṅkamitvā puccheyyāsi – jānāsi, āvuso korakkhattiya\footnote{acela korakhattiya (ka.)}, attano gatinti? Ṭhānaṃ kho panetaṃ, sunakkhatta, vijjati yaṃ te acelo korakkhattiyo byākarissati – jānāmi, āvuso sunakkhatta, attano gatiṃ; kālakañcikā nāma asurā sabbanihīno asurakāyo, tatrāmhi upapannoti.

‘‘Atha kho, bhaggava, sunakkhatto licchaviputto yena acelo korakkhattiyo tenupasaṅkami; upasaṅkamitvā acelaṃ korakkhattiyaṃ etadavoca – ‘byākato khosi, āvuso korakkhattiya, samaṇena gotamena – acelo korakkhattiyo sattamaṃ divasaṃ alasakena kālaṅkarissati. Kālaṅkato ca kālakañcikā nāma asurā sabbanihīno asurakāyo , tatra upapajjissati. Kālaṅkatañca naṃ bīraṇatthambake susāne chaḍḍessantī’ti. Yena tvaṃ, āvuso korakkhattiya, mattaṃ mattañca bhattaṃ bhuñjeyyāsi, mattaṃ mattañca pānīyaṃ piveyyāsi. Yathā samaṇassa gotamassa micchā assa vacana’nti.

\paragraph{8.} ‘‘Atha kho, bhaggava, sunakkhatto licchaviputto ekadvīhikāya sattarattindivāni gaṇesi, yathā taṃ tathāgatassa asaddahamāno. Atha kho, bhaggava, acelo korakkhattiyo sattamaṃ divasaṃ alasakena kālamakāsi. Kālaṅkato ca kālakañcikā nāma asurā sabbanihīno asurakāyo, tatra upapajji. Kālaṅkatañca naṃ bīraṇatthambake susāne chaḍḍesuṃ.

\paragraph{9.} ‘‘Assosi kho, bhaggava, sunakkhatto licchaviputto – ‘acelo kira korakkhattiyo alasakena kālaṅkato bīraṇatthambake susāne chaḍḍito’ti. Atha kho, bhaggava, sunakkhatto licchaviputto yena bīraṇatthambakaṃ susānaṃ, yena acelo korakkhattiyo tenupasaṅkami; upasaṅkamitvā acelaṃ korakkhattiyaṃ tikkhattuṃ pāṇinā ākoṭesi – ‘jānāsi, āvuso korakkhattiya, attano gati’nti? Atha kho, bhaggava, acelo korakkhattiyo pāṇinā piṭṭhiṃ paripuñchanto vuṭṭhāsi. ‘Jānāmi, āvuso sunakkhatta, attano gatiṃ. Kālakañcikā nāma asurā sabbanihīno asurakāyo, tatrāmhi upapanno’ti vatvā tattheva uttāno papati\footnote{paripati (syā. ka.)}.

\paragraph{10.} ‘‘Atha kho, bhaggava, sunakkhatto licchaviputto yenāhaṃ tenupasaṅkami; upasaṅkamitvā maṃ abhivādetvā ekamantaṃ nisīdi. Ekamantaṃ nisinnaṃ kho ahaṃ, bhaggava , sunakkhattaṃ licchaviputtaṃ etadavocaṃ – ‘taṃ kiṃ maññasi, sunakkhatta, yatheva te ahaṃ acelaṃ korakkhattiyaṃ ārabbha byākāsiṃ, tatheva taṃ vipākaṃ, aññathā vā’ti? ‘Yatheva me, bhante, bhagavā acelaṃ korakkhattiyaṃ ārabbha byākāsi, tatheva taṃ vipākaṃ, no aññathā’ti. ‘Taṃ kiṃ maññasi, sunakkhatta, yadi evaṃ sante kataṃ vā hoti uttarimanussadhammā iddhipāṭihāriyaṃ, akataṃ vāti? ‘Addhā kho, bhante, evaṃ sante kataṃ hoti uttarimanussadhammā iddhipāṭihāriyaṃ, no akata’nti. ‘Evampi kho maṃ tvaṃ, moghapurisa, uttarimanussadhammā iddhipāṭihāriyaṃ karontaṃ evaṃ vadesi – na hi pana me, bhante, bhagavā uttarimanussadhammā iddhipāṭihāriyaṃ karotīti. Passa, moghapurisa, yāvañca te idaṃ aparaddha’nti. ‘‘Evampi kho, bhaggava, sunakkhatto licchaviputto mayā vuccamāno apakkameva imasmā dhammavinayā, yathā taṃ āpāyiko nerayiko.

\subsubsection{Acelakaḷāramaṭṭakavatthu}

\paragraph{11.} ‘‘Ekamidāhaṃ, bhaggava, samayaṃ vesāliyaṃ viharāmi mahāvane kūṭāgārasālāyaṃ. Tena kho pana samayena acelo kaḷāramaṭṭako vesāliyaṃ paṭivasati lābhaggappatto ceva yasaggappatto ca vajjigāme. Tassa sattavatapadāni\footnote{sattavattapadāni (syā. pī.)} samattāni samādinnāni honti – ‘yāvajīvaṃ acelako assaṃ, na vatthaṃ paridaheyyaṃ, yāvajīvaṃ brahmacārī assaṃ, na methunaṃ dhammaṃ paṭiseveyyaṃ, yāvajīvaṃ surāmaṃseneva yāpeyyaṃ, na odanakummāsaṃ bhuñjeyyaṃ. Puratthimena vesāliṃ udenaṃ nāma cetiyaṃ, taṃ nātikkameyyaṃ, dakkhiṇena vesāliṃ gotamakaṃ nāma cetiyaṃ, taṃ nātikkameyyaṃ, pacchimena vesāliṃ sattambaṃ nāma cetiyaṃ, taṃ nātikkameyyaṃ, uttarena vesāliṃ bahuputtaṃ nāma\footnote{bahuputtakaṃ nāma (syā.)} cetiyaṃ taṃ nātikkameyya’nti. So imesaṃ sattannaṃ vatapadānaṃ samādānahetu lābhaggappatto ceva yasaggappatto ca vajjigāme.

\paragraph{12.} ‘‘Atha kho, bhaggava, sunakkhatto licchaviputto yena acelo kaḷāramaṭṭako tenupasaṅkami; upasaṅkamitvā acelaṃ kaḷāramaṭṭakaṃ pañhaṃ apucchi. Tassa acelo kaḷāramaṭṭako pañhaṃ puṭṭho na sampāyāsi. Asampāyanto kopañca dosañca appaccayañca pātvākāsi. Atha kho, bhaggava, sunakkhattassa licchaviputtassa etadahosi – ‘sādhurūpaṃ vata bho arahantaṃ samaṇaṃ āsādimhase\footnote{asādiyimhase (syā.)}. Mā vata no ahosi dīgharattaṃ ahitāya dukkhāyā’ti.

\paragraph{13.} ‘‘Atha kho, bhaggava, sunakkhatto licchaviputto yenāhaṃ tenupasaṅkami; upasaṅkamitvā maṃ abhivādetvā ekamantaṃ nisīdi. Ekamantaṃ nisinnaṃ kho ahaṃ, bhaggava, sunakkhattaṃ licchaviputtaṃ etadavocaṃ – ‘tvampi nāma, moghapurisa, samaṇo sakyaputtiyo paṭijānissasī’ti! ‘Kiṃ pana maṃ, bhante, bhagavā evamāha – tvampi nāma, moghapurisa, samaṇo sakyaputtiyo paṭijānissasī’ti? ‘Nanu tvaṃ, sunakkhatta, acelaṃ kaḷāramaṭṭakaṃ upasaṅkamitvā pañhaṃ apucchi. Tassa te acelo kaḷāramaṭṭako pañhaṃ puṭṭho na sampāyāsi. Asampāyanto kopañca dosañca appaccayañca pātvākāsi. Tassa te etadahosi – ‘‘sādhurūpaṃ vata, bho, arahantaṃ samaṇaṃ āsādimhase. Mā vata no ahosi dīgharattaṃ ahitāya dukkhāyā’ti. ‘Evaṃ, bhante. Kiṃ pana, bhante, bhagavā arahattassa maccharāyatī’ti? ‘Na kho ahaṃ, moghapurisa, arahattassa maccharāyāmi, api ca tuyhevetaṃ pāpakaṃ diṭṭhigataṃ uppannaṃ, taṃ pajaha. Mā te ahosi dīgharattaṃ ahitāya dukkhāya. Yaṃ kho panetaṃ, sunakkhatta, maññasi acelaṃ kaḷāramaṭṭakaṃ – sādhurūpo ayaṃ\footnote{arahaṃ (syā.)} samaṇoti, so nacirasseva parihito sānucāriko vicaranto odanakummāsaṃ bhuñjamāno sabbāneva vesāliyāni cetiyāni samatikkamitvā yasā nihīno\footnote{yasānikiṇṇo (ka.)} kālaṃ karissatī’ti.

‘‘‘Atha kho, bhaggava, acelo kaḷāramaṭṭako nacirasseva parihito sānucāriko vicaranto odanakummāsaṃ bhuñjamāno sabbāneva vesāliyāni cetiyāni samatikkamitvā yasā nihīno kālamakāsi.

\paragraph{14.} ‘‘Assosi kho, bhaggava, sunakkhatto licchaviputto – ‘acelo kira kaḷāramaṭṭako parihito sānucāriko vicaranto odanakummāsaṃ bhuñjamāno sabbāneva vesāliyāni cetiyāni samatikkamitvā yasā nihīno kālaṅkato’ti. Atha kho, bhaggava, sunakkhatto licchaviputto yenāhaṃ tenupasaṅkami; upasaṅkamitvā maṃ abhivādetvā ekamantaṃ nisīdi. Ekamantaṃ nisinnaṃ kho ahaṃ, bhaggava, sunakkhattaṃ licchaviputtaṃ etadavocaṃ – ‘taṃ kiṃ maññasi, sunakkhatta, yatheva te ahaṃ acelaṃ kaḷāramaṭṭakaṃ ārabbha byākāsiṃ, tatheva taṃ vipākaṃ, aññathā vā’ti? ‘Yatheva me, bhante, bhagavā acelaṃ kaḷāramaṭṭakaṃ ārabbha byākāsi, tatheva taṃ vipākaṃ, no aññathā’ti. ‘Taṃ kiṃ maññasi, sunakkhatta, yadi evaṃ sante kataṃ vā hoti uttarimanussadhammā iddhipāṭihāriyaṃ akataṃ vā’ti? ‘Addhā kho, bhante, evaṃ sante kataṃ hoti uttarimanussadhammā iddhipāṭihāriyaṃ, no akata’nti. ‘Evampi kho maṃ tvaṃ, moghapurisa, uttarimanussadhammā iddhipāṭihāriyaṃ karontaṃ evaṃ vadesi – na hi pana me, bhante, bhagavā uttarimanussadhammā iddhipāṭihāriyaṃ karotī’’ti. Passa, moghapurisa, yāvañca te idaṃ aparaddha’nti. ‘‘Eva’mpi kho, bhaggava, sunakkhatto licchaviputto mayā vuccamāno apakkameva imasmā dhammavinayā, yathā taṃ āpāyiko nerayiko.

\subsubsection{Acelapāthikaputtavatthu}

\paragraph{15.} ‘‘Ekamidāhaṃ, bhaggava, samayaṃ tattheva vesāliyaṃ viharāmi mahāvane kūṭāgārasālāyaṃ. Tena kho pana samayena acelo pāthikaputto\footnote{pāṭikaputto (sī. syā. pī.)} vesāliyaṃ paṭivasati lābhaggappatto ceva yasaggappatto ca vajjigāme. So vesāliyaṃ parisati evaṃ vācaṃ bhāsati – ‘samaṇopi gotamo ñāṇavādo, ahampi ñāṇavādo. Ñāṇavādo kho pana ñāṇavādena arahati uttarimanussadhammā iddhipāṭihāriyaṃ dassetuṃ. Samaṇo gotamo upaḍḍhapathaṃ āgaccheyya, ahampi upaḍḍhapathaṃ gaccheyyaṃ. Te tattha ubhopi uttarimanussadhammā iddhipāṭihāriyaṃ kareyyāma. Ekaṃ ce samaṇo gotamo uttarimanussadhammā iddhipāṭihāriyaṃ karissati, dvāhaṃ karissāmi. Dve ce samaṇo gotamo uttarimanussadhammā iddhipāṭihāriyāni karissati, cattārāhaṃ karissāmi . Cattāri ce samaṇo gotamo uttarimanussadhammā iddhipāṭihāriyāni karissati, aṭṭhāhaṃ karissāmi. Iti yāvatakaṃ yāvatakaṃ samaṇo gotamo uttarimanussadhammā iddhipāṭihāriyaṃ karissati, taddiguṇaṃ taddiguṇāhaṃ karissāmī’ti.

\paragraph{16.} ‘‘Atha kho, bhaggava, sunakkhatto licchaviputto yenāhaṃ tenupasaṅkami; upasaṅkamitvā maṃ abhivādetvā ekamantaṃ nisīdi. Ekamantaṃ nisinno kho, bhaggava, sunakkhatto licchaviputto maṃ etadavoca – ‘acelo, bhante, pāthikaputto vesāliyaṃ paṭivasati lābhaggappatto ceva yasaggappatto ca vajjigāme. So vesāliyaṃ parisati evaṃ vācaṃ bhāsati – samaṇopi gotamo ñāṇavādo, ahampi ñāṇavādo. Ñāṇavādo kho pana ñāṇavādena arahati uttarimanussadhammā iddhipāṭihāriyaṃ dassetuṃ. Samaṇo gotamo upaḍḍhapathaṃ āgaccheyya, ahampi upaḍḍhapathaṃ gaccheyyaṃ. Te tattha ubhopi uttarimanussadhammā iddhipāṭihāriyaṃ kareyyāma. Ekaṃ ce samaṇo gotamo uttarimanussadhammā iddhipāṭihāriyaṃ karissati, dvāhaṃ karissāmi. Dve ce samaṇo gotamo uttarimanussadhammā iddhipāṭihāriyāni karissati, cattārāhaṃ karissāmi. Cattāri ce samaṇo gotamo uttarimanussadhammā iddhipāṭihāriyāni karissati, aṭṭhāhaṃ karissāmi. Iti yāvatakaṃ yāvatakaṃ samaṇo gotamo uttari manussadhammā iddhipāṭihāriyaṃ karissati, taddiguṇaṃ taddiguṇāhaṃ karissāmī’’ti.

‘‘Evaṃ vutte, ahaṃ, bhaggava, sunakkhattaṃ licchaviputtaṃ etadavocaṃ – ‘abhabbo kho, sunakkhatta, acelo pāthikaputto taṃ vācaṃ appahāya taṃ cittaṃ appahāya taṃ diṭṭhiṃ appaṭinissajjitvā mama sammukhībhāvaṃ āgantuṃ. Sacepissa evamassa – ahaṃ taṃ vācaṃ appahāya taṃ cittaṃ appahāya taṃ diṭṭhiṃ appaṭinissajjitvā samaṇassa gotamassa sammukhībhāvaṃ gaccheyyanti, muddhāpi tassa vipateyyā’ti.

\paragraph{17.} ‘Rakkhatetaṃ, bhante, bhagavā vācaṃ, rakkhatetaṃ sugato vāca’nti. ‘Kiṃ pana maṃ tvaṃ, sunakkhatta, evaṃ vadesi – rakkhatetaṃ, bhante, bhagavā vācaṃ, rakkhatetaṃ sugato vāca’nti? ‘Bhagavatā cassa, bhante, esā vācā ekaṃsena odhāritā\footnote{ovāditā (ka.)} – abhabbo acelo pāthikaputto taṃ vācaṃ appahāya taṃ cittaṃ appahāya taṃ diṭṭhiṃ appaṭinissajjitvā mama sammukhībhāvaṃ āgantuṃ. Sacepissa evamassa – ahaṃ taṃ vācaṃ appahāya taṃ cittaṃ appahāya taṃ diṭṭhiṃ appaṭinissajjitvā samaṇassa gotamassa sammukhībhāvaṃ gaccheyyanti, muddhāpi tassa vipateyyāti. Acelo ca, bhante, pāthikaputto virūparūpena bhagavato sammukhībhāvaṃ āgaccheyya, tadassa bhagavato musā’ti.

\paragraph{18.} ‘Api nu, sunakkhatta, tathāgato taṃ vācaṃ bhāseyya yā sā vācā dvayagāminī’ti? ‘Kiṃ pana, bhante, bhagavatā acelo pāthikaputto cetasā ceto paricca vidito – abhabbo acelo pāthikaputto taṃ vācaṃ appahāya taṃ cittaṃ appahāya taṃ diṭṭhiṃ appaṭinissajjitvā mama sammukhībhāvaṃ āgantuṃ. Sacepissa evamassa – ahaṃ taṃ vācaṃ appahāya taṃ cittaṃ appahāya taṃ diṭṭhiṃ appaṭinissajjitvā samaṇassa gotamassa sammukhībhāvaṃ gaccheyyanti, muddhāpi tassa vipateyyā’ti?

‘Udāhu , devatā bhagavato etamatthaṃ ārocesuṃ – abhabbo, bhante, acelo pāthikaputto taṃ vācaṃ appahāya taṃ cittaṃ appahāya taṃ diṭṭhiṃ appaṭinissajjitvā bhagavato sammukhībhāvaṃ āgantuṃ. Sacepissa evamassa – ahaṃ taṃ vācaṃ appahāya taṃ cittaṃ appahāya taṃ diṭṭhiṃ appaṭinissajjitvā samaṇassa gotamassa sammukhībhāvaṃ gaccheyyanti, muddhāpi tassa vipateyyā’ti?

\paragraph{19.} ‘Cetasā ceto paricca vidito ceva me, sunakkhatta , acelo pāthikaputto abhabbo acelo pāthikaputto taṃ vācaṃ appahāya taṃ cittaṃ appahāya taṃ diṭṭhiṃ appaṭinissajjitvā mama sammukhībhāvaṃ āgantuṃ. Sacepissa evamassa – ahaṃ taṃ vācaṃ appahāya taṃ cittaṃ appahāya taṃ diṭṭhiṃ appaṭinissajjitvā samaṇassa gotamassa sammukhībhāvaṃ gaccheyyanti, muddhāpi tassa vipateyyā’ti.

‘Devatāpi me etamatthaṃ ārocesuṃ – abhabbo , bhante, acelo pāthikaputto taṃ vācaṃ appahāya taṃ cittaṃ appahāya taṃ diṭṭhiṃ appaṭinissajjitvā bhagavato sammukhībhāvaṃ āgantuṃ. Sacepissa evamassa – ahaṃ taṃ vācaṃ appahāya taṃ cittaṃ appahāya taṃ diṭṭhiṃ appaṭinissajjitvā samaṇassa gotamassa sammukhībhāvaṃ gaccheyyanti, muddhāpi tassa vipateyyā’ti.

‘Ajitopi nāma licchavīnaṃ senāpati adhunā kālaṅkato tāvatiṃsakāyaṃ upapanno. Sopi maṃ upasaṅkamitvā evamārocesi – alajjī, bhante, acelo pāthikaputto; musāvādī, bhante, acelo pāthikaputto. Mampi, bhante, acelo pāthikaputto byākāsi vajjigāme – ajito licchavīnaṃ senāpati mahānirayaṃ upapannoti. Na kho panāhaṃ, bhante, mahānirayaṃ upapanno; tāvatiṃsakāyamhi upapanno. Alajjī, bhante, acelo pāthikaputto; musāvādī, bhante, acelo pāthikaputto; abhabbo ca, bhante, acelo pāthikaputto taṃ vācaṃ appahāya taṃ cittaṃ appahāya taṃ diṭṭhiṃ appaṭinissajjitvā bhagavato sammukhībhāvaṃ āgantuṃ. Sacepissa evamassa – ahaṃ taṃ vācaṃ appahāya taṃ cittaṃ appahāya taṃ diṭṭhiṃ appaṭinissajjitvā samaṇassa gotamassa sammukhībhāvaṃ gaccheyyanti, muddhāpi tassa vipateyyā’ti.

‘Iti kho, sunakkhatta, cetasā ceto paricca vidito ceva me acelo pāthikaputto abhabbo acelo pāthikaputto taṃ vācaṃ appahāya taṃ cittaṃ appahāya taṃ diṭṭhiṃ appaṭinissajjitvā mama sammukhībhāvaṃ āgantuṃ. Sacepissa evamassa – ahaṃ taṃ vācaṃ appahāya taṃ cittaṃ appahāya taṃ diṭṭhiṃ appaṭinissajjitvā samaṇassa gotamassa sammukhībhāvaṃ gaccheyyanti, muddhāpi tassa vipateyyāti. Devatāpi me etamatthaṃ ārocesuṃ – abhabbo, bhante , acelo pāthikaputto taṃ vācaṃ appahāya taṃ cittaṃ appahāya taṃ diṭṭhiṃ appaṭinissajjitvā bhagavato sammukhībhāvaṃ āgantuṃ. Sacepissa evamassa – ahaṃ taṃ vācaṃ appahāya taṃ cittaṃ appahāya taṃ diṭṭhiṃ appaṭinissajjitvā samaṇassa gotamassa sammukhībhāvaṃ gaccheyyanti, muddhāpi tassa vipateyyā’ti.

‘So kho panāhaṃ, sunakkhatta, vesāliyaṃ piṇḍāya caritvā pacchābhattaṃ piṇḍapātappaṭikkanto yena acelassa pāthikaputtassa ārāmo tenupasaṅkamissāmi divāvihārāya. Yassadāni tvaṃ, sunakkhatta, icchasi, tassa ārocehī’ti.

\subsubsection{Iddhipāṭihāriyakathā}

\paragraph{20.} ‘‘Atha khvāhaṃ\footnote{atha kho svāhaṃ (syā.)}, bhaggava, pubbaṇhasamayaṃ nivāsetvā pattacīvaramādāya vesāliṃ piṇḍāya pāvisiṃ. Vesāliyaṃ piṇḍāya caritvā pacchābhattaṃ piṇḍapātappaṭikkanto yena acelassa pāthikaputtassa ārāmo tenupasaṅkamiṃ divāvihārāya. Atha kho, bhaggava, sunakkhatto licchaviputto taramānarūpo vesāliṃ pavisitvā yena abhiññātā abhiññātā licchavī tenupasaṅkami; upasaṅkamitvā abhiññāte abhiññāte licchavī etadavoca – ‘esāvuso, bhagavā vesāliyaṃ piṇḍāya caritvā pacchābhattaṃ piṇḍapātappaṭikkanto yena acelassa pāthikaputtassa ārāmo tenupasaṅkami divāvihārāya. Abhikkamathāyasmanto abhikkamathāyasmanto, sādhurūpānaṃ samaṇānaṃ uttarimanussadhammā iddhipāṭihāriyaṃ bhavissatī’ti . Atha kho, bhaggava, abhiññātānaṃ abhiññātānaṃ licchavīnaṃ etadahosi – ‘sādhurūpānaṃ kira, bho, samaṇānaṃ uttarimanussadhammā iddhipāṭihāriyaṃ bhavissati; handa vata, bho, gacchāmā’ti. Yena ca abhiññātā abhiññātā brāhmaṇamahāsālā gahapatinecayikā nānātitthiyā\footnote{nānātitthiya (syā.)} samaṇabrāhmaṇā tenupasaṅkami. Upasaṅkamitvā abhiññāte abhiññāte nānātitthiye\footnote{nānātitthiya (syā.)} samaṇabrāhmaṇe etadavoca – ‘esāvuso, bhagavā vesāliyaṃ piṇḍāya caritvā pacchābhattaṃ piṇḍapātappaṭikkanto yena acelassa pāthikaputtassa ārāmo tenupasaṅkami divāvihārāya. Abhikkamathāyasmanto abhikkamathāyasmanto, sādhurūpānaṃ samaṇānaṃ uttarimanussadhammā iddhipāṭihāriyaṃ bhavissatī’ti. Atha kho, bhaggava, abhiññātānaṃ abhiññātānaṃ nānātitthiyānaṃ samaṇabrāhmaṇānaṃ etadahosi – ‘sādhurūpānaṃ kira, bho, samaṇānaṃ uttarimanussadhammā iddhipāṭihāriyaṃ bhavissati; handa vata, bho, gacchāmā’ti.

‘‘Atha kho, bhaggava, abhiññātā abhiññātā licchavī, abhiññātā abhiññātā ca brāhmaṇamahāsālā gahapatinecayikā nānātitthiyā samaṇabrāhmaṇā yena acelassa pāthikaputtassa ārāmo tenupasaṅkamiṃsu. Sā esā, bhaggava, parisā mahā hoti\footnote{parisā hoti (sī. syā. pī.)} anekasatā anekasahassā.

\paragraph{21.} ‘‘Assosi kho, bhaggava, acelo pāthikaputto – ‘abhikkantā kira abhiññātā abhiññātā licchavī, abhikkantā abhiññātā abhiññātā ca brāhmaṇamahāsālā gahapatinecayikā nānātitthiyā samaṇabrāhmaṇā. Samaṇopi gotamo mayhaṃ ārāme divāvihāraṃ nisinno’ti. Sutvānassa bhayaṃ chambhitattaṃ lomahaṃso udapādi. Atha kho, bhaggava, acelo pāthikaputto bhīto saṃviggo lomahaṭṭhajāto yena tindukakhāṇuparibbājakārāmo tenupasaṅkami.

‘‘Assosi kho, bhaggava, sā parisā – ‘acelo kira pāthikaputto bhīto saṃviggo lomahaṭṭhajāto yena tindukakhāṇuparibbājakārāmo tenupasaṅkanto’ti\footnote{tenupasaṅkamanto (sī. pī. ka.)}. Atha kho, bhaggava, sā parisā aññataraṃ purisaṃ āmantesi –

‘Ehi tvaṃ, bho purisa, yena tindukakhāṇuparibbājakārāmo, yena acelo pāthikaputto tenupasaṅkama. Upasaṅkamitvā acelaṃ pāthikaputtaṃ evaṃ vadehi – abhikkamāvuso, pāthikaputta, abhikkantā abhiññātā abhiññātā licchavī, abhikkantā abhiññātā abhiññātā ca brāhmaṇamahāsālā gahapatinecayikā nānātitthiyā samaṇabrāhmaṇā, samaṇopi gotamo āyasmato ārāme divāvihāraṃ nisinno; bhāsitā kho pana te esā, āvuso pāthikaputta, vesāliyaṃ parisati vācā samaṇopi gotamo ñāṇavādo, ahampi ñāṇavādo. Ñāṇavādo kho pana ñāṇavādena arahati uttarimanussadhammā iddhipāṭihāriyaṃ dassetuṃ. Samaṇo gotamo upaḍḍhapathaṃ āgaccheyya ahampi upaḍḍhapathaṃ gaccheyyaṃ. Te tattha ubhopi uttarimanussadhammā iddhipāṭihāriyaṃ kareyyāma. Ekaṃ ce samaṇo gotamo uttarimanussadhammā iddhipāṭihāriyaṃ karissati, dvāhaṃ karissāmi. Dve ce samaṇo gotamo uttarimanussadhammā iddhipāṭihāriyāni karissati, cattārāhaṃ karissāmi. Cattāri ce samaṇo gotamo uttarimanussadhammā iddhipāṭihāriyāni karissati , aṭṭhāhaṃ karissāmi. Iti yāvatakaṃ yāvatakaṃ samaṇo gotamo uttarimanussadhammā iddhipāṭihāriyaṃ karissati, taddiguṇaṃ taddiguṇāhaṃ karissāmī’ti abhikkamasseva\footnote{abhikkamayeva (sī. syā. pī.)} kho; āvuso pāthikaputta, upaḍḍhapathaṃ. Sabbapaṭhamaṃyeva āgantvā samaṇo gotamo āyasmato ārāme divāvihāraṃ nisinno’ti.

\paragraph{22.} ‘‘Evaṃ, bhoti kho, bhaggava, so puriso tassā parisāya paṭissutvā yena tindukakhāṇuparibbājakārāmo, yena acelo pāthikaputto tenupasaṅkami. Upasaṅkamitvā acelaṃ pāthikaputtaṃ etadavoca – ‘abhikkamāvuso pāthikaputta, abhikkantā abhiññātā abhiññātā licchavī, abhikkantā abhiññātā abhiññātā ca brāhmaṇamahāsālā gahapatinecayikā nānātitthiyā samaṇabrāhmaṇā. Samaṇopi gotamo āyasmato ārāme divāvihāraṃ nisinno. Bhāsitā kho pana te esā, āvuso pāthikaputta, vesāliyaṃ parisati vācā – samaṇopi gotamo ñāṇavādo; ahampi ñāṇavādo. Ñāṇavādo kho pana ñāṇavādena arahati uttarimanussadhammā iddhipāṭihāriyaṃ dassetuṃ…pe… taddiguṇaṃ taddiguṇāhaṃ karissāmīti. Abhikkamasseva kho, āvuso pāthikaputta, upaḍḍhapathaṃ. Sabbapaṭhamaṃyeva āgantvā samaṇo gotamo āyasmato ārāme divāvihāraṃ nisinno’ti.

‘‘Evaṃ vutte, bhaggava, acelo pāthikaputto ‘āyāmi āvuso, āyāmi āvuso’ti vatvā tattheva saṃsappati\footnote{saṃsabbati (ka.)}, na sakkoti āsanāpi vuṭṭhātuṃ. Atha kho so, bhaggava, puriso acelaṃ pāthikaputtaṃ etadavoca – ‘kiṃ su nāma te, āvuso pāthikaputta, pāvaḷā su nāma te pīṭhakasmiṃ allīnā, pīṭhakaṃ su nāma te pāvaḷāsu allīnaṃ? Āyāmi āvuso, āyāmi āvusoti vatvā tattheva saṃsappasi, na sakkosi āsanāpi vuṭṭhātu’nti. Evampi kho, bhaggava, vuccamāno acelo pāthikaputto ‘āyāmi āvuso, āyāmi āvuso’ti vatvā tattheva saṃsappati , na sakkoti āsanāpi vuṭṭhātuṃ.

\paragraph{23.} ‘‘Yadā kho so, bhaggava, puriso aññāsi – ‘parābhūtarūpo ayaṃ acelo pāthikaputto. Āyāmi āvuso, āyāmi āvusoti vatvā tattheva saṃsappati, na sakkoti āsanāpi vuṭṭhātu’nti. Atha taṃ parisaṃ āgantvā evamārocesi – ‘parābhūtarūpo, bho\footnote{parābhūtarūpo bho ayaṃ (syā. ka.), parābhūtarūpo (sī. pī.)}, acelo pāthikaputto. Āyāmi āvuso, āyāmi āvusoti vatvā tattheva saṃsappati, na sakkoti āsanāpi vuṭṭhātu’nti. Evaṃ vutte, ahaṃ, bhaggava, taṃ parisaṃ etadavocaṃ – ‘abhabbo kho, āvuso, acelo pāthikaputto taṃ vācaṃ appahāya taṃ cittaṃ appahāya taṃ diṭṭhiṃ appaṭinissajjitvā mama sammukhībhāvaṃ āgantuṃ. Sacepissa evamassa – ‘ahaṃ taṃ vācaṃ appahāya taṃ cittaṃ appahāya taṃ diṭṭhiṃ appaṭinissajjitvā samaṇassa gotamassa sammukhībhāvaṃ gaccheyya’nti, muddhāpi tassa vipateyyāti.

\xsubsubsectionEnd{Paṭhamabhāṇavāro niṭṭhito.}

\paragraph{24.} ‘‘Atha kho, bhaggava, aññataro licchavimahāmatto uṭṭhāyāsanā taṃ parisaṃ etadavoca – ‘tena hi, bho, muhuttaṃ tāva āgametha, yāvāhaṃ gacchāmi\footnote{paccāgacchāmi (?)}. Appeva nāma ahampi sakkuṇeyyaṃ acelaṃ pāthikaputtaṃ imaṃ parisaṃ ānetu’nti.

‘‘Atha kho so, bhaggava, licchavimahāmatto yena tindukakhāṇuparibbājakārāmo, yena acelo pāthikaputto tenupasaṅkami. Upasaṅkamitvā acelaṃ pāthikaputtaṃ etadavoca – ‘abhikkamāvuso pāthikaputta, abhikkantaṃ te seyyo, abhikkantā abhiññātā abhiññātā licchavī, abhikkantā abhiññātā abhiññātā ca brāhmaṇamahāsālā gahapatinecayikā nānātitthiyā samaṇabrāhmaṇā. Samaṇopi gotamo āyasmato ārāme divāvihāraṃ nisinno. Bhāsitā kho pana te esā, āvuso pāthikaputta, vesāliyaṃ parisati vācā – samaṇopi gotamo ñāṇavādo…pe… taddiguṇaṃ taddiguṇāhaṃ karissāmīti. Abhikkamasseva kho, āvuso pāthikaputta, upaḍḍhapathaṃ. Sabbapaṭhamaṃyeva āgantvā samaṇo gotamo āyasmato ārāme divāvihāraṃ nisinno. Bhāsitā kho panesā, āvuso pāthikaputta, samaṇena gotamena parisati vācā – abhabbo kho acelo pāthikaputto taṃ vācaṃ appahāya taṃ cittaṃ appahāya taṃ diṭṭhiṃ appaṭinissajjitvā mama sammukhībhāvaṃ āgantuṃ. Sacepissa evamassa – ahaṃ taṃ vācaṃ appahāya taṃ cittaṃ appahāya taṃ diṭṭhiṃ appaṭinissajjitvā samaṇassa gotamassa sammukhībhāvaṃ gaccheyyanti, muddhāpi tassa vipateyyāti. Abhikkamāvuso pāthikaputta, abhikkamaneneva te jayaṃ karissāma, samaṇassa gotamassa parājaya’nti.

‘‘Evaṃ vutte, bhaggava, acelo pāthikaputto ‘āyāmi āvuso, āyāmi āvuso’ti vatvā tattheva saṃsappati, na sakkoti āsanāpi vuṭṭhātuṃ. Atha kho so, bhaggava, licchavimahāmatto acelaṃ pāthikaputtaṃ etadavoca – ‘kiṃ su nāma te, āvuso pāthikaputta, pāvaḷā su nāma te pīṭhakasmiṃ allīnā, pīṭhakaṃ su nāma te pāvaḷāsu allīnaṃ ? Āyāmi āvuso, āyāmi āvusoti vatvā tattheva saṃsappasi, na sakkosi āsanāpi vuṭṭhātu’nti . Evampi kho, bhaggava, vuccamāno acelo pāthikaputto ‘āyāmi āvuso, āyāmi āvuso’ti vatvā tattheva saṃsappati, na sakkoti āsanāpi vuṭṭhātuṃ.

\paragraph{25.} ‘‘Yadā kho so, bhaggava, licchavimahāmatto aññāsi – ‘parābhūtarūpo ayaṃ acelo pāthikaputto āyāmi āvuso, āyāmi āvusoti vatvā tattheva saṃsappati, na sakkoti āsanāpi vuṭṭhātu’nti. Atha taṃ parisaṃ āgantvā evamārocesi – ‘parābhūtarūpo, bho\footnote{parābhūtarūpo (sī. pī.), parābhūtarūpo ayaṃ (syā.)}, acelo pāthikaputto āyāmi āvuso, āyāmi āvusoti vatvā tattheva saṃsappati, na sakkoti āsanāpi vuṭṭhātu’nti. Evaṃ vutte, ahaṃ, bhaggava, taṃ parisaṃ etadavocaṃ – ‘abhabbo kho, āvuso, acelo pāthikaputto taṃ vācaṃ appahāya taṃ cittaṃ appahāya taṃ diṭṭhiṃ appaṭinissajjitvā mama sammukhībhāvaṃ āgantuṃ. Sacepissa evamassa – ahaṃ taṃ vācaṃ appahāya taṃ cittaṃ appahāya taṃ diṭṭhiṃ appaṭinissajjitvā samaṇassa gotamassa sammukhībhāvaṃ gaccheyyanti, muddhāpi tassa vipateyya. Sace pāyasmantānaṃ licchavīnaṃ evamassa – mayaṃ acelaṃ pāthikaputtaṃ varattāhi\footnote{yāhi varattāhi (syā. ka.)} bandhitvā goyugehi āviñcheyyāmāti\footnote{āviñjeyyāmāti (syā.), āvijjheyyāmāti (sī. pī.)}, tā varattā chijjeyyuṃ pāthikaputto vā. Abhabbo pana acelo pāthikaputto taṃ vācaṃ appahāya taṃ cittaṃ appahāya taṃ diṭṭhiṃ appaṭinissajjitvā mama sammukhībhāvaṃ āgantuṃ. Sacepissa evamassa – ahaṃ taṃ vācaṃ appahāya taṃ cittaṃ appahāya taṃ diṭṭhiṃ appaṭinissajjitvā samaṇassa gotamassa sammukhībhāvaṃ gaccheyyanti, muddhāpi tassa vipateyyā’ti.

\paragraph{26.} ‘‘Atha kho, bhaggava, jāliyo dārupattikantevāsī uṭṭhāyāsanā taṃ parisaṃ etadavoca – ‘tena hi, bho, muhuttaṃ tāva āgametha, yāvāhaṃ gacchāmi; appeva nāma ahampi sakkuṇeyyaṃ acelaṃ pāthikaputtaṃ imaṃ parisaṃ ānetu’’nti.

‘‘Atha kho, bhaggava, jāliyo dārupattikantevāsī yena tindukakhāṇuparibbājakārāmo, yena acelo pāthikaputto tenupasaṅkami. Upasaṅkamitvā acelaṃ pāthikaputtaṃ etadavoca – ‘abhikkamāvuso pāthikaputta, abhikkantaṃ te seyyo. Abhikkantā abhiññātā abhiññātā licchavī, abhikkantā abhiññātā abhiññātā ca brāhmaṇamahāsālā gahapatinecayikā nānātitthiyā samaṇabrāhmaṇā. Samaṇopi gotamo āyasmato ārāme divāvihāraṃ nisinno. Bhāsitā kho pana te esā, āvuso pāthikaputta, vesāliyaṃ parisati vācā – samaṇopi gotamo ñāṇavādo…pe… taddiguṇaṃ taddiguṇāhaṃ karissāmīti. Abhikkamasseva, kho āvuso pāthikaputta, upaḍḍhapathaṃ. Sabbapaṭhamaṃyeva āgantvā samaṇo gotamo āyasmato ārāme divāvihāraṃ nisinno. Bhāsitā kho panesā, āvuso pāthikaputta, samaṇena gotamena parisati vācā – abhabbo acelo pāthikaputto taṃ vācaṃ appahāya taṃ cittaṃ appahāya taṃ diṭṭhiṃ appaṭinissajjitvā mama sammukhībhāvaṃ āgantuṃ. Sacepissa evamassa – ahaṃ taṃ vācaṃ appahāya taṃ cittaṃ appahāya taṃ diṭṭhiṃ appaṭinissajjitvā samaṇassa gotamassa sammukhībhāvaṃ gaccheyyanti, muddhāpi tassa vipateyya. Sace pāyasmantānaṃ licchavīnaṃ evamassa – mayaṃ acelaṃ pāthikaputtaṃ varattāhi bandhitvā goyugehi āviñcheyyāmāti. Tā varattā chijjeyyuṃ pāthikaputto vā. Abhabbo pana acelo pāthikaputto taṃ vācaṃ appahāya taṃ cittaṃ appahāya taṃ diṭṭhiṃ appaṭinissajjitvā mama sammukhībhāvaṃ āgantuṃ. Sacepissa evamassa – ahaṃ taṃ vācaṃ appahāya taṃ cittaṃ appahāya taṃ diṭṭhiṃ appaṭinissajjitvā samaṇassa gotamassa sammukhībhāvaṃ āgaccheyyanti, muddhāpi tassa vipateyyāti. Abhikkamāvuso pāthikaputta, abhikkamaneneva te jayaṃ karissāma, samaṇassa gotamassa parājaya’nti.

‘‘Evaṃ vutte, bhaggava, acelo pāthikaputto ‘āyāmi āvuso, āyāmi āvuso’ti vatvā tattheva saṃsappati, na sakkoti āsanāpi vuṭṭhātuṃ. Atha kho, bhaggava, jāliyo dārupattikantevāsī acelaṃ pāthikaputtaṃ etadavoca – ‘kiṃ su nāma te, āvuso pāthikaputta, pāvaḷā su nāma te pīṭhakasmiṃ allīnā, pīṭhakaṃ su nāma te pāvaḷāsu allīnaṃ? Āyāmi āvuso, āyāmi āvusoti vatvā tattheva saṃsappasi, na sakkosi āsanāpi vuṭṭhātu’nti. Evampi kho, bhaggava, vuccamāno acelo pāthikaputto ‘‘āyāmi āvuso, āyāmi āvuso’’ti vatvā tattheva saṃsappati, na sakkoti āsanāpi vuṭṭhātunti.

\paragraph{27.} ‘‘Yadā kho, bhaggava, jāliyo dārupattikantevāsī aññāsi – ‘parābhūtarūpo ayaṃ acelo pāthikaputto ‘āyāmi āvuso, āyāmi āvusoti vatvā tattheva saṃsappati, na sakkoti āsanāpi vuṭṭhātu’nti, atha naṃ etadavoca –

‘Bhūtapubbaṃ, āvuso pāthikaputta, sīhassa migarañño etadahosi – yaṃnūnāhaṃ aññataraṃ vanasaṇḍaṃ nissāya āsayaṃ kappeyyaṃ. Tatrāsayaṃ kappetvā sāyanhasamayaṃ āsayā nikkhameyyaṃ, āsayā nikkhamitvā vijambheyyaṃ, vijambhitvā samantā catuddisā anuvilokeyyaṃ, samantā catuddisā anuviloketvā tikkhattuṃ sīhanādaṃ nadeyyaṃ, tikkhattuṃ sīhanādaṃ naditvā gocarāya pakkameyyaṃ. So varaṃ varaṃ migasaṃghe\footnote{migasaṃghaṃ (syā. ka.)} vadhitvā mudumaṃsāni mudumaṃsāni bhakkhayitvā tameva āsayaṃ ajjhupeyya’nti.

‘Atha kho, āvuso, so sīho migarājā aññataraṃ vanasaṇḍaṃ nissāya āsayaṃ kappesi. Tatrāsayaṃ kappetvā sāyanhasamayaṃ āsayā nikkhami, āsayā nikkhamitvā vijambhi, vijambhitvā samantā catuddisā anuvilokesi, samantā catuddisā anuviloketvā tikkhattuṃ sīhanādaṃ nadi, tikkhattuṃ sīhanādaṃ naditvā gocarāya pakkāmi. So varaṃ varaṃ migasaṅghe vadhitvā mudumaṃsāni mudumaṃsāni bhakkhayitvā tameva āsayaṃ ajjhupesi.

\paragraph{28.} ‘Tasseva kho, āvuso pāthikaputta, sīhassa migarañño vighāsasaṃvaḍḍho jarasiṅgālo\footnote{jarasigālo (sī. syā. pī.)} ditto ceva balavā ca. Atha kho, āvuso, tassa jarasiṅgālassa etadahosi – ko cāhaṃ, ko sīho migarājā. Yaṃnūnāhampi aññataraṃ vanasaṇḍaṃ nissāya āsayaṃ kappeyyaṃ. Tatrāsayaṃ kappetvā sāyanhasamayaṃ āsayā nikkhameyyaṃ, āsayā nikkhamitvā vijambheyyaṃ, vijambhitvā samantā catuddisā anuvilokeyyaṃ, samantā catuddisā anuviloketvā tikkhattuṃ sīhanādaṃ nadeyyaṃ, tikkhattuṃ sīhanādaṃ naditvā gocarāya pakkameyyaṃ. So varaṃ varaṃ migasaṅghe vadhitvā mudumaṃsāni mudumaṃsāni bhakkhayitvā tameva āsayaṃ ajjhupeyya’nti.

‘Atha kho so, āvuso, jarasiṅgālo aññataraṃ vanasaṇḍaṃ nissāya āsayaṃ kappesi. Tatrāsayaṃ kappetvā sāyanhasamayaṃ āsayā nikkhami, āsayā nikkhamitvā vijambhi, vijambhitvā samantā catuddisā anuvilokesi, samantā catuddisā anuviloketvā tikkhattuṃ sīhanādaṃ nadissāmīti siṅgālakaṃyeva anadi bheraṇḍakaṃyeva\footnote{bhedaṇḍakaṃyeva (ka.)} anadi, ke ca chave siṅgāle, ke pana sīhanādeti\footnote{sīhanāde (?)}.

‘Evameva kho tvaṃ, āvuso pāthikaputta, sugatāpadānesu jīvamāno sugatātirittāni bhuñjamāno tathāgate arahante sammāsambuddhe āsādetabbaṃ maññasi. Ke ca chave pāthikaputte, kā ca tathāgatānaṃ arahantānaṃ sammāsambuddhānaṃ āsādanā’ti.

\paragraph{29.} ‘‘Yato kho, bhaggava, jāliyo dārupattikantevāsī iminā opammena neva asakkhi acelaṃ pāthikaputtaṃ tamhā āsanā cāvetuṃ. Atha naṃ etadavoca –

‘Sīhoti attānaṃ samekkhiyāna,

Amaññi kotthu migarājāhamasmi;

Tatheva\footnote{tameva (syā.)} so siṅgālakaṃ anadi,

Ke ca chave siṅgāle ke pana sīhanāde’ti.

‘Evameva kho tvaṃ, āvuso pāthikaputta, sugatāpadānesu jīvamāno sugatātirittāni bhuñjamāno tathāgate arahante sammāsambuddhe āsādetabbaṃ maññasi. Ke ca chave pāthikaputte, kā ca tathāgatānaṃ arahantānaṃ sammāsambuddhānaṃ āsādanā’ti.

\paragraph{30.} ‘‘Yato kho, bhaggava, jāliyo dārupattikantevāsī imināpi opammena neva asakkhi acelaṃ pāthikaputtaṃ tamhā āsanā cāvetuṃ. Atha naṃ etadavoca –

‘Aññaṃ anucaṅkamanaṃ, attānaṃ vighāse samekkhiya;

Yāva attānaṃ na passati, kotthu tāva byagghoti maññati.

Tatheva so siṅgālakaṃ anadi;

Ke ca chave siṅgāle ke pana sīhanāde’ti.

‘Evameva kho tvaṃ, āvuso pāthikaputta, sugatāpadānesu jīvamāno sugatātirittāni bhuñjamāno tathāgate arahante sammāsambuddhe āsādetabbaṃ maññasi. Ke ca chave pāthikaputte, kā ca tathāgatānaṃ arahantānaṃ sammāsambuddhānaṃ āsādanā’ti.

\paragraph{31.} ‘‘Yato kho, bhaggava, jāliyo dārupattikantevāsī imināpi opammena neva asakkhi acelaṃ pāthikaputtaṃ tamhā āsanā cāvetuṃ. Atha naṃ etadavoca –

‘Bhutvāna bheke\footnote{bhiṅge (ka.)} khalamūsikāyo,

Kaṭasīsu khittāni ca koṇapāni\footnote{kūṇapāni (syā.)};

Mahāvane suññavane vivaḍḍho,

Amaññi kotthu migarājāhamasmi.

Tatheva so siṅgālakaṃ anadi;

Ke ca chave siṅgāle ke pana sīhanāde’ti.

‘Evameva kho tvaṃ, āvuso pāthikaputta, sugatāpadānesu jīvamāno sugatātirittāni bhuñjamāno tathāgate arahante sammāsambuddhe āsādetabbaṃ maññasi. Ke ca chave pāthikaputte, kā ca tathāgatānaṃ arahantānaṃ sammāsambuddhānaṃ āsādanā’ti.

\paragraph{32.} ‘‘Yato kho, bhaggava, jāliyo dārupattikantevāsī imināpi opammena neva asakkhi acelaṃ pāthikaputtaṃ tamhā āsanā cāvetuṃ. Atha taṃ parisaṃ āgantvā evamārocesi – ‘parābhūtarūpo, bho, acelo pāthikaputto āyāmi āvuso, āyāmi āvusoti vatvā tattheva saṃsappati, na sakkoti āsanāpi vuṭṭhātu’nti.

\paragraph{33.} ‘‘Evaṃ vutte, ahaṃ, bhaggava, taṃ parisaṃ etadavocaṃ – ‘abhabbo kho, āvuso, acelo pāthikaputto taṃ vācaṃ appahāya taṃ cittaṃ appahāya taṃ diṭṭhiṃ appaṭinissajjitvā mama sammukhībhāvaṃ āgantuṃ. Sacepissa evamassa – ahaṃ taṃ vācaṃ appahāya taṃ cittaṃ appahāya taṃ diṭṭhiṃ appaṭinissajjitvā samaṇassa gotamassa sammukhībhāvaṃ gaccheyyanti, muddhāpi tassa vipateyya. Sacepāyasmantānaṃ licchavīnaṃ evamassa – mayaṃ acelaṃ pāthikaputtaṃ varattāhi bandhitvā nāgehi\footnote{goyugehi (sabbattha) aṭṭhakathā passitabbā} āviñcheyyāmāti . Tā varattā chijjeyyuṃ pāthikaputto vā. Abhabbo pana acelo pāthikaputto taṃ vācaṃ appahāya taṃ cittaṃ appahāya taṃ diṭṭhiṃ appaṭinissajjitvā mama sammukhībhāvaṃ āgantuṃ. Sacepissa evamassa – ahaṃ taṃ vācaṃ appahāya taṃ cittaṃ appahāya taṃ diṭṭhiṃ appaṭinissajjitvā samaṇassa gotamassa sammukhībhāvaṃ gaccheyyanti, muddhāpi tassa vipateyyā’ti.

\paragraph{34.} ‘‘Atha khvāhaṃ, bhaggava, taṃ parisaṃ dhammiyā kathāya sandassesiṃ samādapesiṃ samuttejesiṃ sampahaṃsesiṃ, taṃ parisaṃ dhammiyā kathāya sandassetvā samādapetvā samuttejetvā sampahaṃsetvā mahābandhanā mokkhaṃ karitvā caturāsītipāṇasahassāni mahāviduggā uddharitvā tejodhātuṃ samāpajjitvā sattatālaṃ vehāsaṃ abbhuggantvā aññaṃ sattatālampi acciṃ\footnote{aggiṃ (syā.)} abhinimminitvā pajjalitvā dhūmāyitvā\footnote{dhūpāyitvā (sī. pī.)} mahāvane kūṭāgārasālāyaṃ paccuṭṭhāsiṃ.

\paragraph{35.} ‘‘Atha kho, bhaggava, sunakkhatto licchaviputto yenāhaṃ tenupasaṅkami; upasaṅkamitvā maṃ abhivādetvā ekamantaṃ nisīdi. Ekamantaṃ nisinnaṃ kho ahaṃ, bhaggava, sunakkhattaṃ licchaviputtaṃ etadavocaṃ – ‘taṃ kiṃ maññasi, sunakkhatta, yatheva te ahaṃ acelaṃ pāthikaputtaṃ ārabbha byākāsiṃ, tatheva taṃ vipākaṃ aññathā vā’ti? ‘Yatheva me, bhante, bhagavā acelaṃ pāthikaputtaṃ ārabbha byākāsi, tatheva taṃ vipākaṃ, no aññathā’ti.

‘Taṃ kiṃ maññasi, sunakkhatta, yadi evaṃ sante kataṃ vā hoti uttarimanussadhammā iddhipāṭihāriyaṃ, akataṃ vā’ti? ‘Addhā kho, bhante, evaṃ sante kataṃ hoti uttarimanussadhammā iddhipāṭihāriyaṃ, no akata’nti. ‘Evampi kho maṃ tvaṃ, moghapurisa, uttarimanussadhammā iddhipāṭihāriyaṃ karontaṃ evaṃ vadesi – na hi pana me, bhante, bhagavā uttarimanussadhammā iddhipāṭihāriyaṃ karotīti. Passa, moghapurisa, yāvañca te idaṃ aparaddhaṃ’ti.

‘‘Evampi kho, bhaggava, sunakkhatto licchaviputto mayā vuccamāno apakkameva imasmā dhammavinayā, yathā taṃ āpāyiko nerayiko.

\subsubsection{Aggaññapaññattikathā}

\paragraph{36.} ‘‘Aggaññañcāhaṃ, bhaggava, pajānāmi. Tañca pajānāmi\footnote{‘‘tañcapajānāmī’’ti idaṃ syāpotthakenatthi}, tato ca uttaritaraṃ pajānāmi, tañca pajānaṃ\footnote{pajānanaṃ (syā. ka.) aṭṭhakathāsaṃvaṇṇanā passitabbā} na parāmasāmi, aparāmasato ca me paccattaññeva nibbuti viditā, yadabhijānaṃ tathāgato no anayaṃ āpajjati .

\paragraph{37.} ‘‘Santi, bhaggava, eke samaṇabrāhmaṇā issarakuttaṃ brahmakuttaṃ ācariyakaṃ aggaññaṃ paññapenti. Tyāhaṃ upasaṅkamitvā evaṃ vadāmi – ‘saccaṃ kira tumhe āyasmanto issarakuttaṃ brahmakuttaṃ ācariyakaṃ aggaññaṃ paññapethā’ti? Te ca me evaṃ puṭṭhā, ‘āmo’ti\footnote{āmāti (syā.)} paṭijānanti. Tyāhaṃ evaṃ vadāmi – ‘kathaṃvihitakaṃ pana\footnote{kathaṃ vihitakaṃno pana (ka.)} tumhe āyasmanto issarakuttaṃ brahmakuttaṃ ācariyakaṃ aggaññaṃ paññapethā’ti? Te mayā puṭṭhā na sampāyanti, asampāyantā mamaññeva paṭipucchanti. Tesāhaṃ puṭṭho byākaromi –

\paragraph{38.} ‘Hoti kho so, āvuso, samayo yaṃ kadāci karahaci dīghassa addhuno accayena ayaṃ loko saṃvaṭṭati. Saṃvaṭṭamāne loke yebhuyyena sattā ābhassarasaṃvattanikā honti. Te tattha honti manomayā pītibhakkhā sayaṃpabhā antalikkhacarā subhaṭṭhāyino ciraṃ dīghamaddhānaṃ tiṭṭhanti.

‘Hoti kho so, āvuso, samayo yaṃ kadāci karahaci dīghassa addhuno accayena ayaṃ loko vivaṭṭati. Vivaṭṭamāne loke suññaṃ brahmavimānaṃ pātubhavati. Atha kho\footnote{atha (sī. syā. pī.)} aññataro satto āyukkhayā vā puññakkhayā vā ābhassarakāyā cavitvā suññaṃ brahmavimānaṃ upapajjati . So tattha hoti manomayo pītibhakkho sayaṃpabho antalikkhacaro subhaṭṭhāyī, ciraṃ dīghamaddhānaṃ tiṭṭhati.

‘Tassa tattha ekakassa dīgharattaṃ nivusitattā anabhirati paritassanā uppajjati – aho vata aññepi sattā itthattaṃ āgaccheyyunti. Atha aññepi sattā āyukkhayā vā puññakkhayā vā ābhassarakāyā cavitvā brahmavimānaṃ upapajjanti tassa sattassa sahabyataṃ. Tepi tattha honti manomayā pītibhakkhā sayaṃpabhā antalikkhacarā subhaṭṭhāyino, ciraṃ dīghamaddhānaṃ tiṭṭhanti.

\paragraph{39.} ‘Tatrāvuso, yo so satto paṭhamaṃ upapanno, tassa evaṃ hoti – ahamasmi brahmā mahābrahmā abhibhū anabhibhūto aññadatthudaso vasavattī issaro kattā nimmātā seṭṭho sajitā\footnote{sañjitā (sī. pī.), sajjitā (syā. kaṃ.)} vasī pitā bhūtabhabyānaṃ, mayā ime sattā nimmitā. Taṃ kissa hetu? Mamañhi pubbe etadahosi – aho vata aññepi sattā itthattaṃ āgaccheyyunti; iti mama ca manopaṇidhi. Ime ca sattā itthattaṃ āgatāti.

‘Yepi te sattā pacchā upapannā, tesampi evaṃ hoti – ayaṃ kho bhavaṃ brahmā mahābrahmā abhibhū anabhibhūto aññadatthudaso vasavattī issaro kattā nimmātā seṭṭho sajitā vasī pitā bhūtabhabyānaṃ; iminā mayaṃ bhotā brahmunā nimmitā. Taṃ kissa hetu? Imañhi mayaṃ addasāma idha paṭhamaṃ upapannaṃ; mayaṃ panāmha pacchā upapannāti.

\paragraph{40.} ‘Tatrāvuso , yo so satto paṭhamaṃ upapanno, so dīghāyukataro ca hoti vaṇṇavantataro ca mahesakkhataro ca. Ye pana te sattā pacchā upapannā, te appāyukatarā ca honti dubbaṇṇatarā ca appesakkhatarā ca.

‘Ṭhānaṃ kho panetaṃ, āvuso, vijjati, yaṃ aññataro satto tamhā kāyā cavitvā itthattaṃ āgacchati. Itthattaṃ āgato samāno agārasmā anagāriyaṃ pabbajati. Agārasmā anagāriyaṃ pabbajito samāno ātappamanvāya padhānamanvāya anuyogamanvāya appamādamanvāya sammāmanasikāramanvāya tathārūpaṃ cetosamādhiṃ phusati, yathāsamāhite citte taṃ pubbenivāsaṃ anussarati; tato paraṃ nānussarati.

‘So evamāha – yo kho so bhavaṃ brahmā mahābrahmā abhibhū anabhibhūto aññadatthudaso vasavattī issaro kattā nimmātā seṭṭho sajitā vasī pitā bhūtabhabyānaṃ, yena mayaṃ bhotā brahmunā nimmitā. So nicco dhuvo\footnote{sassato dīghāyuko (syā. ka.)} sassato avipariṇāmadhammo sassatisamaṃ tatheva ṭhassati. Ye pana mayaṃ ahumhā tena bhotā brahmunā nimmitā, te mayaṃ aniccā addhuvā\footnote{addhuvā asassatā (syā. ka.)} appāyukā cavanadhammā itthattaṃ āgatā’ti. Evaṃvihitakaṃ no tumhe āyasmanto issarakuttaṃ brahmakuttaṃ ācariyakaṃ aggaññaṃ paññapethāti. ‘Te evamāhaṃsu – evaṃ kho no, āvuso gotama, sutaṃ, yathevāyasmā gotamo āhā’ti. ‘‘Aggaññañcāhaṃ, bhaggava, pajānāmi. Tañca pajānāmi, tato ca uttaritaraṃ pajānāmi, tañca pajānaṃ na parāmasāmi, aparāmasato ca me paccattaññeva nibbuti viditā. Yadabhijānaṃ tathāgato no anayaṃ āpajjati.

\paragraph{41.} ‘‘Santi, bhaggava, eke samaṇabrāhmaṇā khiḍḍāpadosikaṃ ācariyakaṃ aggaññaṃ paññapenti. Tyāhaṃ upasaṅkamitvā evaṃ vadāmi – ‘saccaṃ kira tumhe āyasmanto khiḍḍāpadosikaṃ ācariyakaṃ aggaññaṃ paññapethā’ti? Te ca me evaṃ puṭṭhā ‘āmo’ti paṭijānanti. Tyāhaṃ evaṃ vadāmi – ‘kathaṃvihitakaṃ pana tumhe āyasmanto khiḍḍāpadosikaṃ ācariyakaṃ aggaññaṃ paññapethā’ti? Te mayā puṭṭhā na sampāyanti, asampāyantā mamaññeva paṭipucchanti, tesāhaṃ puṭṭho byākaromi –

\paragraph{42.} ‘Santāvuso, khiḍḍāpadosikā nāma devā. Te ativelaṃ hassakhiḍḍāratidhammasamāpannā\footnote{hasakhiḍḍāratidhammasamāpannā (ka.)} viharanti. Tesaṃ ativelaṃ hassakhiḍḍāratidhammasamāpannānaṃ viharataṃ sati sammussati, satiyā sammosā\footnote{satiyā sammosāya (syā.)} te devā tamhā kāyā cavanti.

‘Ṭhānaṃ kho panetaṃ, āvuso, vijjati, yaṃ aññataro satto tamhā kāyā cavitvā itthattaṃ āgacchati, itthattaṃ āgato samāno agārasmā anagāriyaṃ pabbajati, agārasmā anagāriyaṃ pabbajito samāno ātappamanvāya padhānamanvāya anuyogamanvāya appamādamanvāya sammāmanasikāramanvāya tathārūpaṃ cetosamādhiṃ phusati, yathāsamāhite citte taṃ pubbenivāsaṃ anussarati; tato paraṃ nānussarati.

‘So evamāha – ye kho te bhonto devā na khiḍḍāpadosikā te na ativelaṃ hassakhiḍḍāratidhammasamāpannā viharanti. Tesaṃ nātivelaṃ hassakhiḍḍāratidhammasamāpannānaṃ viharataṃ sati na sammussati, satiyā asammosā te devā tamhā kāyā na cavanti, niccā dhuvā sassatā avipariṇāmadhammā sassatisamaṃ tatheva ṭhassanti. Ye pana mayaṃ ahumhā khiḍḍāpadosikā te mayaṃ ativelaṃ hassakhiḍḍāratidhammasamāpannā viharimhā, tesaṃ no ativelaṃ hassakhiḍḍāratidhammasamāpannānaṃ viharataṃ sati sammussati, satiyā sammosā evaṃ\footnote{sammosā eva (sī. pī.) sammosā te (syā. ka.)} mayaṃ tamhā kāyā cutā, aniccā addhuvā appāyukā cavanadhammā itthattaṃ āgatāti. Evaṃvihitakaṃ no tumhe āyasmanto khiḍḍāpadosikaṃ ācariyakaṃ aggaññaṃ paññapethā’ti. ‘Te evamāhaṃsu – evaṃ kho no, āvuso gotama, sutaṃ, yathevāyasmā gotamo āhā’ti. ‘‘Aggaññañcāhaṃ, bhaggava, pajānāmi…pe… yadabhijānaṃ tathāgato no anayaṃ āpajjati.

\paragraph{43.} ‘‘Santi, bhaggava, eke samaṇabrāhmaṇā manopadosikaṃ ācariyakaṃ aggaññaṃ paññapenti. Tyāhaṃ upasaṅkamitvā evaṃ vadāmi – ‘saccaṃ kira tumhe āyasmanto manopadosikaṃ ācariyakaṃ aggaññaṃ paññapethā’ti? Te ca me evaṃ puṭṭhā ‘āmo’ti paṭijānanti. Tyāhaṃ evaṃ vadāmi – ‘kathaṃvihitakaṃ pana tumhe āyasmanto manopadosikaṃ ācariyakaṃ aggaññaṃ paññapethā’ti? Te mayā puṭṭhā na sampāyanti, asampāyantā mamaññeva paṭipucchanti. Tesāhaṃ puṭṭho byākaromi –

\paragraph{44.} ‘Santāvuso, manopadosikā nāma devā. Te ativelaṃ aññamaññaṃ upanijjhāyanti. Te ativelaṃ aññamaññaṃ upanijjhāyantā aññamaññamhi cittāni padūsenti. Te aññamaññaṃ paduṭṭhacittā kilantakāyā kilantacittā. Te devā tamhā kāyā cavanti.

‘Ṭhānaṃ kho panetaṃ, āvuso, vijjati, yaṃ aññataro satto tamhā kāyā cavitvā itthattaṃ āgacchati. Itthattaṃ āgato samāno agārasmā anagāriyaṃ pabbajati. Agārasmā anagāriyaṃ pabbajito samāno ātappamanvāya padhānamanvāya anuyogamanvāya appamādamanvāya sammāmanasikāramanvāya tathārūpaṃ cetosamādhiṃ phusati, yathāsamāhite citte taṃ pubbenivāsaṃ anussarati, tato paraṃ nānussarati.

‘So evamāha – ye kho te bhonto devā na manopadosikā te nātivelaṃ aññamaññaṃ upanijjhāyanti. Te nātivelaṃ aññamaññaṃ upanijjhāyantā aññamaññamhi cittāni nappadūsenti. Te aññamaññaṃ appaduṭṭhacittā akilantakāyā akilantacittā. Te devā tamhā\footnote{akilantacittā tamhā (ka.)} kāyā na cavanti, niccā dhuvā sassatā avipariṇāmadhammā sassatisamaṃ tatheva ṭhassanti. Ye pana mayaṃ ahumhā manopadosikā, te mayaṃ ativelaṃ aññamaññaṃ upanijjhāyimhā. Te mayaṃ ativelaṃ aññamaññaṃ upanijjhāyantā aññamaññamhi cittāni padūsimhā\footnote{padosiyimhā (syā.), padūsayimhā (?)}. Te mayaṃ aññamaññaṃ paduṭṭhacittā kilantakāyā kilantacittā. Evaṃ mayaṃ\footnote{kilantacittāeva mayaṃ (sī. pī.), kilantacittā (ka.)} tamhā kāyā cutā, aniccā addhuvā appāyukā cavanadhammā itthattaṃ āgatāti. Evaṃvihitakaṃ no tumhe āyasmanto manopadosikaṃ ācariyakaṃ aggaññaṃ paññapethā’ti. ‘Te evamāhaṃsu – evaṃ kho no, āvuso gotama, sutaṃ, yathevāyasmā gotamo āhā’ti. ‘‘Aggaññañcāhaṃ, bhaggava, pajānāmi…pe… yadabhijānaṃ tathāgato no anayaṃ āpajjati.

\paragraph{45.} ‘‘Santi, bhaggava, eke samaṇabrāhmaṇā adhiccasamuppannaṃ ācariyakaṃ aggaññaṃ paññapenti. Tyāhaṃ upasaṅkamitvā evaṃ vadāmi – ‘saccaṃ kira tumhe āyasmanto adhiccasamuppannaṃ ācariyakaṃ aggaññaṃ paññapethā’ti? Te ca me evaṃ puṭṭhā ‘āmo’ti paṭijānanti. Tyāhaṃ evaṃ vadāmi – ‘kathaṃvihitakaṃ pana tumhe āyasmanto adhiccasamuppannaṃ ācariyakaṃ aggaññaṃ paññapethā’ti? Te mayā puṭṭhā na sampāyanti, asampāyantā mamaññeva paṭipucchanti. Tesāhaṃ puṭṭho byākaromi –

\paragraph{46.} ‘Santāvuso, asaññasattā nāma devā. Saññuppādā ca pana te devā tamhā kāyā cavanti.

‘Ṭhānaṃ kho panetaṃ, āvuso, vijjati. Yaṃ aññataro satto tamhā kāyā cavitvā itthattaṃ āgacchati. Itthattaṃ āgato samāno agārasmā anagāriyaṃ pabbajati. Agārasmā anagāriyaṃ pabbajito samāno ātappamanvāya padhānamanvāya anuyogamanvāya appamādamanvāya sammāmanasikāramanvāya tathārūpaṃ cetosamādhiṃ phusati, yathāsamāhite citte taṃ\footnote{idaṃ padaṃ brahmajālasutte na dissati. evaṃ (pī. ka.)} saññuppādaṃ anussarati, tato paraṃ nānussarati.

‘So evamāha – adhiccasamuppanno attā ca loko ca. Taṃ kissa hetu? Ahañhi pubbe nāhosiṃ, somhi etarahi ahutvā santatāya\footnote{sattakāya (sī. pī.), sattāya (ka. sī.)} pariṇatoti. Evaṃvihitakaṃ no tumhe āyasmanto adhiccasamuppannaṃ ācariyakaṃ aggaññaṃ paññapethā’ti? ‘Te evamāhaṃsu – evaṃ kho no, āvuso gotama, sutaṃ yathevāyasmā gotamo āhā’ti. ‘‘Aggaññañcāhaṃ, bhaggava, pajānāmi tañca pajānāmi, tato ca uttaritaraṃ pajānāmi, tañca pajānaṃ na parāmasāmi, aparāmasato ca me paccattaññeva nibbuti viditā. Yadabhijānaṃ tathāgato no anayaṃ āpajjati.

\paragraph{47.} ‘‘Evaṃvādiṃ kho maṃ, bhaggava, evamakkhāyiṃ eke samaṇabrāhmaṇā asatā tucchā musā abhūtena abbhācikkhanti – ‘viparīto samaṇo gotamo bhikkhavo ca. Samaṇo gotamo evamāha – yasmiṃ samaye subhaṃ vimokkhaṃ upasampajja viharati, sabbaṃ tasmiṃ samaye asubhantveva\footnote{asubhanteva (sī. syā. pī.)} pajānātī’ti\footnote{sañjānātīti (sī. pī.)}. Na kho panāhaṃ, bhaggava, evaṃ vadāmi – ‘yasmiṃ samaye subhaṃ vimokkhaṃ upasampajja viharati, sabbaṃ tasmiṃ samaye asubhantveva pajānātī’ti. Evañca khvāhaṃ, bhaggava, vadāmi – ‘yasmiṃ samaye subhaṃ vimokkhaṃ upasampajja viharati, subhantveva tasmiṃ samaye pajānātī’ti.

‘‘Te ca, bhante, viparītā, ye bhagavantaṃ viparītato dahanti bhikkhavo ca. Evaṃpasanno ahaṃ, bhante, bhagavati. Pahoti me bhagavā tathā dhammaṃ desetuṃ, yathā ahaṃ subhaṃ vimokkhaṃ upasampajja vihareyya’’nti.

\paragraph{48.} ‘‘Dukkaraṃ kho etaṃ, bhaggava, tayā aññadiṭṭhikena aññakhantikena aññarucikena aññatrāyogena aññatrācariyakena subhaṃ vimokkhaṃ upasampajja viharituṃ. Iṅgha tvaṃ, bhaggava, yo ca te ayaṃ mayi pasādo, tameva tvaṃ sādhukamanurakkhā’’ti. ‘‘Sace taṃ, bhante, mayā dukkaraṃ aññadiṭṭhikena aññakhantikena aññarucikena aññatrāyogena aññatrācariyakena subhaṃ vimokkhaṃ upasampajja viharituṃ. Yo ca me ayaṃ, bhante, bhagavati pasādo, tamevāhaṃ sādhukamanurakkhissāmī’’ti. Idamavoca bhagavā. Attamano bhaggavagotto paribbājako bhagavato bhāsitaṃ abhinandīti.

\xsectionEnd{Pāthikasuttaṃ\footnote{pāṭikasuttantaṃ (sī. syā. kaṃ. pī.)} niṭṭhitaṃ paṭhamaṃ.}


\section{Udumbarikasuttaṃ}

\subsubsection{Nigrodhaparibbājakavatthu}

\paragraph{49.} Evaṃ me sutaṃ – ekaṃ samayaṃ bhagavā rājagahe viharati gijjhakūṭe pabbate. Tena kho pana samayena nigrodho paribbājako udumbarikāya paribbājakārāme paṭivasati mahatiyā paribbājakaparisāya saddhiṃ tiṃsamattehi paribbājakasatehi. Atha kho sandhāno gahapati divā divassa\footnote{divādivasseva (sī. syā. pī.)} rājagahā nikkhami bhagavantaṃ dassanāya. Atha kho sandhānassa gahapatissa etadahosi – ‘‘akālo kho bhagavantaṃ dassanāya. Paṭisallīno bhagavā. Manobhāvanīyānampi bhikkhūnaṃ asamayo dassanāya. Paṭisallīnā manobhāvanīyā bhikkhū. Yaṃnūnāhaṃ yena udumbarikāya paribbājakārāmo, yena nigrodho paribbājako tenupasaṅkameyya’’nti. Atha kho sandhāno gahapati yena udumbarikāya paribbājakārāmo, tenupasaṅkami.

\paragraph{50.} Tena kho pana samayena nigrodho paribbājako mahatiyā paribbājakaparisāya saddhiṃ nisinno hoti unnādiniyā uccāsaddamahāsaddāya anekavihitaṃ tiracchānakathaṃ kathentiyā. Seyyathidaṃ – rājakathaṃ corakathaṃ mahāmattakathaṃ senākathaṃ bhayakathaṃ yuddhakathaṃ annakathaṃ pānakathaṃ vatthakathaṃ sayanakathaṃ mālākathaṃ gandhakathaṃ ñātikathaṃ yānakathaṃ gāmakathaṃ nigamakathaṃ nagarakathaṃ janapadakathaṃ itthikathaṃ sūrakathaṃ visikhākathaṃ kumbhaṭṭhānakathaṃ pubbapetakathaṃ nānattakathaṃ lokakkhāyikaṃ samuddakkhāyikaṃ itibhavābhavakathaṃ iti vā.

\paragraph{51.} Addasā kho nigrodho paribbājako sandhānaṃ gahapatiṃ dūratova āgacchantaṃ. Disvā sakaṃ parisaṃ saṇṭhāpesi – ‘‘appasaddā bhonto hontu, mā bhonto saddamakattha. Ayaṃ samaṇassa gotamassa sāvako āgacchati sandhāno gahapati. Yāvatā kho pana samaṇassa gotamassa sāvakā gihī odātavasanā rājagahe paṭivasanti, ayaṃ tesaṃ aññataro sandhāno gahapati. Appasaddakāmā kho panete āyasmanto appasaddavinītā , appasaddassa vaṇṇavādino. Appeva nāma appasaddaṃ parisaṃ viditvā upasaṅkamitabbaṃ maññeyyā’’ti. Evaṃ vutte te paribbājakā tuṇhī ahesuṃ.

\paragraph{52.} Atha kho sandhāno gahapati yena nigrodho paribbājako tenupasaṅkami, upasaṅkamitvā nigrodhena paribbājakena saddhiṃ sammodi. Sammodanīyaṃ kathaṃ sāraṇīyaṃ vītisāretvā ekamantaṃ nisīdi. Ekamantaṃ nisinno kho sandhāno gahapati nigrodhaṃ paribbājakaṃ etadavoca – ‘‘aññathā kho ime bhonto aññatitthiyā paribbājakā saṅgamma samāgamma unnādino uccāsaddamahāsaddā anekavihitaṃ tiracchānakathaṃ anuyuttā viharanti. Seyyathidaṃ – rājakathaṃ…pe… itibhavābhavakathaṃ iti vā. Aññathā kho\footnote{ca (sī. pī.)} pana so bhagavā araññavanapatthāni pantāni senāsanāni paṭisevati appasaddāni appanigghosāni vijanavātāni manussarāhasseyyakāni paṭisallānasāruppānī’’ti.

\paragraph{53.} Evaṃ vutte nigrodho paribbājako sandhānaṃ gahapatiṃ etadavoca – ‘‘yagghe gahapati, jāneyyāsi, kena samaṇo gotamo saddhiṃ sallapati, kena sākacchaṃ samāpajjati, kena paññāveyyattiyaṃ samāpajjati? Suññāgārahatā samaṇassa gotamassa paññā aparisāvacaro samaṇo gotamo nālaṃ sallāpāya. So antamantāneva sevati\footnote{antapantāneva (syā.)}. Seyyathāpi nāma gokāṇā pariyantacārinī antamantāneva sevati. Evameva suññāgārahatā samaṇassa gotamassa paññā; aparisāvacaro samaṇo gotamo; nālaṃ sallāpāya. So antamantāneva sevati. Iṅgha, gahapati, samaṇo gotamo imaṃ parisaṃ āgaccheyya, ekapañheneva naṃ saṃsādeyyāma\footnote{saṃhareyyāma (ka.)}, tucchakumbhīva naṃ maññe orodheyyāmā’’ti.

\paragraph{54.} Assosi kho bhagavā dibbāya sotadhātuyā visuddhāya atikkantamānusikāya sandhānassa gahapatissa nigrodhena paribbājakena saddhiṃ imaṃ kathāsallāpaṃ. Atha kho bhagavā gijjhakūṭā pabbatā orohitvā yena sumāgadhāya tīre moranivāpo tenupasaṅkami; upasaṅkamitvā sumāgadhāya tīre moranivāpe abbhokāse caṅkami. Addasā kho nigrodho paribbājako bhagavantaṃ sumāgadhāya tīre moranivāpe abbhokāse caṅkamantaṃ. Disvāna sakaṃ parisaṃ saṇṭhāpesi – ‘‘appasaddā bhonto hontu, mā bhonto saddamakattha, ayaṃ samaṇo gotamo sumāgadhāya tīre moranivāpe abbhokāse caṅkamati. Appasaddakāmo kho pana so āyasmā, appasaddassa vaṇṇavādī. Appeva nāma appasaddaṃ parisaṃ viditvā upasaṅkamitabbaṃ maññeyya. Sace samaṇo gotamo imaṃ parisaṃ āgaccheyya, imaṃ taṃ pañhaṃ puccheyyāma – ‘ko nāma so, bhante, bhagavato dhammo, yena bhagavā sāvake vineti, yena bhagavatā sāvakā vinītā assāsappattā paṭijānanti ajjhāsayaṃ ādibrahmacariya’nti? Evaṃ vutte te paribbājakā tuṇhī ahesuṃ.

\subsubsection{Tapojigucchāvādo}

\paragraph{55.} Atha kho bhagavā yena nigrodho paribbājako tenupasaṅkami. Atha kho nigrodho paribbājako bhagavantaṃ etadavoca – ‘‘etu kho, bhante, bhagavā, svāgataṃ, bhante, bhagavato. Cirassaṃ kho, bhante, bhagavā imaṃ pariyāyamakāsi yadidaṃ idhāgamanāya. Nisīdatu, bhante, bhagavā, idamāsanaṃ paññatta’’nti. Nisīdi bhagavā paññatte āsane. Nigrodhopi kho paribbājako aññataraṃ nīcāsanaṃ gahetvā ekamantaṃ nisīdi. Ekamantaṃ nisinnaṃ kho nigrodhaṃ paribbājakaṃ bhagavā etadavoca – ‘‘kāya nuttha, nigrodha, etarahi kathāya sannisinnā, kā ca pana vo antarākathā vippakatā’’ti? Evaṃ vutte, nigrodho paribbājako bhagavantaṃ etadavoca, ‘‘idha mayaṃ, bhante, addasāma bhagavantaṃ sumāgadhāya tīre moranivāpe abbhokāse caṅkamantaṃ, disvāna evaṃ avocumhā – ‘sace samaṇo gotamo imaṃ parisaṃ āgaccheyya, imaṃ taṃ pañhaṃ puccheyyāma – ko nāma so, bhante, bhagavato dhammo, yena bhagavā sāvake vineti, yena bhagavatā sāvakā vinītā assāsappattā paṭijānanti ajjhāsayaṃ ādibrahmacariya’nti? Ayaṃ kho no, bhante, antarākathā vippakatā; atha bhagavā anuppatto’’ti.

\paragraph{56.} ‘‘Dujjānaṃ kho etaṃ, nigrodha, tayā aññadiṭṭhikena aññakhantikena aññarucikena aññatrāyogena aññatrācariyakena, yenāhaṃ sāvake vinemi , yena mayā sāvakā vinītā assāsappattā paṭijānanti ajjhāsayaṃ ādibrahmacariyaṃ. Iṅgha tvaṃ maṃ, nigrodha, sake ācariyake adhijegucche pañhaṃ puccha – ‘kathaṃ santā nu kho, bhante, tapojigucchā paripuṇṇā hoti, kathaṃ aparipuṇṇā’ti? Evaṃ vutte te paribbājakā unnādino uccāsaddamahāsaddā ahesuṃ – ‘‘acchariyaṃ vata bho, abbhutaṃ vata bho, samaṇassa gotamassa mahiddhikatā mahānubhāvatā, yatra hi nāma sakavādaṃ ṭhapessati, paravādena pavāressatī’’ti.

\paragraph{57.} Atha kho nigrodho paribbājako te paribbājake appasadde katvā bhagavantaṃ etadavoca – ‘‘mayaṃ kho, bhante, tapojigucchāvādā\footnote{tarojigucchaṃsārodā (ka.)} tapojigucchāsārā tapojigucchāallīnā viharāma . Kathaṃ santā nu kho, bhante, tapojigucchā paripuṇṇā hoti, kathaṃ aparipuṇṇā’’ti?

‘‘Idha, nigrodha, tapassī acelako hoti muttācāro, hatthāpalekhano\footnote{hatthāvalekhano (syā.)}, na ehibhaddantiko, na tiṭṭhabhaddantiko, nābhihaṭaṃ , na uddissakataṃ, na nimantanaṃ sādiyati, so na kumbhimukhā paṭiggaṇhāti, na kaḷopimukhā paṭiggaṇhāti, na eḷakamantaraṃ, na daṇḍamantaraṃ, na musalamantaraṃ, na dvinnaṃ bhuñjamānānaṃ, na gabbhiniyā, na pāyamānāya, na purisantaragatāya, na saṅkittīsu, na yattha sā upaṭṭhito hoti, na yattha makkhikā saṇḍasaṇḍacārinī, na macchaṃ, na maṃsaṃ, na suraṃ, na merayaṃ, na thusodakaṃ pivati, so ekāgāriko vā hoti ekālopiko, dvāgāriko vā hoti dvālopiko, sattāgāriko vā hoti sattālopiko, ekissāpi dattiyā yāpeti, dvīhipi dattīhi yāpeti, sattahipi dattīhi yāpeti; ekāhikampi āhāraṃ āhāreti, dvīhikampi\footnote{dvāhikaṃpi (sī. syā.)} āhāraṃ āhāreti, sattāhikampi āhāraṃ āhāreti, iti evarūpaṃ addhamāsikampi pariyāyabhattabhojanānuyogamanuyutto viharati. So sākabhakkho vā hoti, sāmākabhakkho vā hoti, nīvārabhakkho vā hoti, daddulabhakkho vā hoti, haṭabhakkho vā hoti, kaṇabhakkho vā hoti, ācāmabhakkho vā hoti, piññākabhakkho vā hoti, tiṇabhakkho vā hoti, gomayabhakkho vā hoti; vanamūlaphalāhāro yāpeti pavattaphalabhojī. So sāṇānipi dhāreti , masāṇānipi dhāreti, chavadussānipi dhāreti, paṃsukūlānipi dhāreti, tirīṭānipi dhāreti, ajinampi dhāreti, ajinakkhipampi dhāreti, kusacīrampi dhāreti, vākacīrampi dhāreti, phalakacīrampi dhāreti, kesakambalampi dhāreti, vāḷakambalampi dhāreti, ulūkapakkhampi dhāreti, kesamassulocakopi hoti kesamassulocanānuyogamanuyutto , ubbhaṭṭhakopi\footnote{ubhaṭṭhakopi (syā.), ubbhaṭṭhikopi (ka.)} hoti āsanapaṭikkhitto, ukkuṭikopi hoti ukkuṭikappadhānamanuyutto, kaṇṭakāpassayikopi hoti kaṇṭakāpassaye seyyaṃ kappeti, phalakaseyyampi kappeti, thaṇḍilaseyyampi kappeti, ekapassayikopi hoti rajojalladharo, abbhokāsikopi hoti yathāsanthatiko, vekaṭikopi hoti vikaṭabhojanānuyogamanuyutto, apānakopi hoti apānakattamanuyutto, sāyatatiyakampi udakorohanānuyogamanuyutto viharati. Taṃ kiṃ maññasi, nigrodha, yadi evaṃ sante tapojigucchā paripuṇṇā vā hoti aparipuṇṇā vā’’ti? ‘‘Addhā kho, bhante, evaṃ sante tapojigucchā paripuṇṇā hoti, no aparipuṇṇā’’ti. ‘‘Evaṃ paripuṇṇāyapi kho ahaṃ, nigrodha, tapojigucchāya anekavihite upakkilese vadāmī’’ti.

\subsubsection{Upakkileso}

\paragraph{58.} ‘‘Yathā kathaṃ pana, bhante, bhagavā evaṃ paripuṇṇāya tapojigucchāya anekavihite upakkilese vadatī’’ti? ‘‘Idha, nigrodha, tapassī tapaṃ samādiyati, so tena tapasā attamano hoti paripuṇṇasaṅkappo. Yampi, nigrodha, tapassī tapaṃ samādiyati, so tena tapasā attamano hoti paripuṇṇasaṅkappo. Ayampi kho, nigrodha, tapassino upakkileso hoti.

‘‘Puna caparaṃ, nigrodha, tapassī tapaṃ samādiyati, so tena tapasā attānukkaṃseti paraṃ vambheti. Yampi, nigrodha, tapassī tapaṃ samādiyati, so tena tapasā attānukkaṃseti paraṃ vambheti. Ayampi kho, nigrodha, tapassino upakkileso hoti.

‘‘Puna caparaṃ, nigrodha, tapassī tapaṃ samādiyati, so tena tapasā majjati mucchati pamādamāpajjati\footnote{madamāpajjati (syā.)}. Yampi, nigrodha, tapassī tapaṃ samādiyati, so tena tapasā majjati mucchati pamādamāpajjati. Ayampi kho, nigrodha, tapassino upakkileso hoti.

\paragraph{59.} ‘‘Puna caparaṃ, nigrodha, tapassī tapaṃ samādiyati, so tena tapasā lābhasakkārasilokaṃ abhinibbatteti, so tena lābhasakkārasilokena attamano hoti paripuṇṇasaṅkappo. Yampi, nigrodha, tapassī tapaṃ samādiyati, so tena tapasā lābhasakkārasilokaṃ abhinibbatteti, so tena lābhasakkārasilokena attamano hoti paripuṇṇasaṅkappo. Ayampi kho, nigrodha, tapassino upakkileso hoti.

‘‘Puna caparaṃ, nigrodha, tapassī tapaṃ samādiyati, so tena tapasā lābhasakkārasilokaṃ abhinibbatteti, so tena lābhasakkārasilokena attānukkaṃseti paraṃ vambheti. Yampi, nigrodha, tapassī tapaṃ samādiyati, so tena tapasā lābhasakkārasilokaṃ abhinibbatteti, so tena lābhasakkārasilokena attānukkaṃseti paraṃ vambheti. Ayampi kho, nigrodha, tapassino upakkileso hoti.

‘‘Puna caparaṃ, nigrodha, tapassī tapaṃ samādiyati, so tena tapasā lābhasakkārasilokaṃ abhinibbatteti, so tena lābhasakkārasilokena majjati mucchati pamādamāpajjati. Yampi, nigrodha, tapassī tapaṃ samādiyati, so tena tapasā lābhasakkārasilokaṃ abhinibbatteti, so tena lābhasakkārasilokena majjati mucchati pamādamāpajjati. Ayampi kho, nigrodha, tapassino upakkileso hoti.

\paragraph{60.} ‘‘Puna caparaṃ, nigrodha, tapassī bhojanesu vodāsaṃ āpajjati – ‘idaṃ me khamati, idaṃ me nakkhamatī’ti. So yañca\footnote{yaṃ hi (sī. pī.)} khvassa nakkhamati, taṃ sāpekkho pajahati. Yaṃ panassa khamati, taṃ gadhito\footnote{gathito (sī. pī.)} mucchito ajjhāpanno anādīnavadassāvī anissaraṇapañño paribhuñjati…pe… ayampi kho, nigrodha, tapassino upakkileso hoti.

‘‘Puna caparaṃ, nigrodha, tapassī tapaṃ samādiyati lābhasakkārasilokanikantihetu – ‘sakkarissanti maṃ rājāno rājamahāmattā khattiyā brāhmaṇā gahapatikā titthiyā’ti…pe… ayampi kho, nigrodha, tapassino upakkileso hoti.

\paragraph{61.} ‘‘Puna caparaṃ, nigrodha, tapassī aññataraṃ samaṇaṃ vā brāhmaṇaṃ vā apasādetā\footnote{apasāretā (ka.)} hoti – ‘kiṃ panāyaṃ sambahulājīvo\footnote{bahulājīvo (sī. pī.)} sabbaṃ saṃbhakkheti. Seyyathidaṃ – mūlabījaṃ khandhabījaṃ phaḷubījaṃ aggabījaṃ bījabījameva pañcamaṃ, asanivicakkaṃ dantakūṭaṃ, samaṇappavādenā’ti…pe… ayampi kho, nigrodha, tapassino upakkileso hoti.

‘‘Puna caparaṃ, nigrodha, tapassī passati aññataraṃ samaṇaṃ vā brāhmaṇaṃ vā kulesu sakkariyamānaṃ garukariyamānaṃ māniyamānaṃ pūjiyamānaṃ. Disvā tassa evaṃ hoti – ‘imañhi nāma sambahulājīvaṃ kulesu sakkaronti garuṃ karonti mānenti pūjenti. Maṃ pana tapassiṃ lūkhājīviṃ kulesu na sakkaronti na garuṃ karonti na mānenti na pūjentī’ti, iti so issāmacchariyaṃ kulesu uppādetā hoti…pe… ayampi kho, nigrodha, tapassino upakkileso hoti.

\paragraph{62.} ‘‘Puna caparaṃ, nigrodha, tapassī āpāthakanisādī hoti…pe… ayampi kho, nigrodha, tapassino upakkileso hoti.

‘‘Puna caparaṃ, nigrodha, tapassī attānaṃ adassayamāno kulesu carati – ‘idampi me tapasmiṃ idampi me tapasmi’nti…pe… ayampi kho, nigrodha, tapassino upakkileso hoti.

‘‘Puna caparaṃ, nigrodha, tapassī kiñcideva paṭicchannaṃ sevati. So ‘khamati te ida’nti puṭṭho samāno akkhamamānaṃ āha – ‘khamatī’ti. Khamamānaṃ āha – ‘nakkhamatī’ti. Iti so sampajānamusā bhāsitā hoti…pe… ayampi kho, nigrodha, tapassino upakkileso hoti.

‘‘Puna caparaṃ, nigrodha, tapassī tathāgatassa vā tathāgatasāvakassa vā dhammaṃ desentassa santaṃyeva pariyāyaṃ anuññeyyaṃ nānujānāti…pe… ayampi kho, nigrodha, tapassino upakkileso hoti.

\paragraph{63.} ‘‘Puna caparaṃ, nigrodha, tapassī kodhano hoti upanāhī. Yampi, nigrodha, tapassī kodhano hoti upanāhī. Ayampi kho, nigrodha, tapassino upakkileso hoti.

‘‘Puna caparaṃ, nigrodha, tapassī makkhī hoti paḷāsī\footnote{palāsī (sī. syā. pī.)} …pe… issukī hoti maccharī… saṭho hoti māyāvī… thaddho hoti atimānī… pāpiccho hoti pāpikānaṃ icchānaṃ vasaṃ gato… micchādiṭṭhiko hoti antaggāhikāya diṭṭhiyā samannāgato… sandiṭṭhiparāmāsī hoti ādhānaggāhī duppaṭinissaggī. Yampi, nigrodha, tapassī sandiṭṭhiparāmāsī hoti ādhānaggāhī duppaṭinissaggī. Ayampi kho, nigrodha, tapassino upakkileso hoti.

‘‘Taṃ kiṃ maññasi, nigrodha, yadime tapojigucchā\footnote{tapojigucchāya (?)} upakkilesā vā anupakkilesā vā’’ti? ‘‘Addhā kho ime, bhante, tapojigucchā\footnote{tapojigucchāya (?)} upakkilesā\footnote{upakkilesā hoti (ka.)}, no anupakkilesā. Ṭhānaṃ kho panetaṃ, bhante, vijjati yaṃ idhekacco tapassī sabbeheva imehi upakkilesehi samannāgato assa; ko pana vādo aññataraññatarenā’’ti.

\subsubsection{Parisuddhapapaṭikappattakathā}

\paragraph{64.} ‘‘Idha, nigrodha, tapassī tapaṃ samādiyati, so tena tapasā na attamano hoti na paripuṇṇasaṅkappo. Yampi, nigrodha, tapassī tapaṃ samādiyati, so tena tapasā na attamano hoti na paripuṇṇasaṅkappo. Evaṃ so tasmiṃ ṭhāne parisuddho hoti.

‘‘Puna caparaṃ, nigrodha, tapassī tapaṃ samādiyati, so tena tapasā na attānukkaṃseti na paraṃ vambheti…pe… evaṃ so tasmiṃ ṭhāne parisuddho hoti.

‘‘Puna caparaṃ, nigrodha, tapassī tapaṃ samādiyati, so tena tapasā na majjati na mucchati na pamādamāpajjati…pe… evaṃ so tasmiṃ ṭhāne parisuddho hoti.

\paragraph{65.} ‘‘Puna caparaṃ, nigrodha, tapassī tapaṃ samādiyati, so tena tapasā lābhasakkārasilokaṃ abhinibbatteti, so tena lābhasakkārasilokena na attamano hoti na paripuṇṇasaṅkappo…pe… evaṃ so tasmiṃ ṭhāne parisuddho hoti.

‘‘Puna caparaṃ, nigrodha, tapassī tapaṃ samādiyati, so tena tapasā lābhasakkārasilokaṃ abhinibbatteti, so tena lābhasakkārasilokena na attānukkaṃseti na paraṃ vambheti…pe… evaṃ so tasmiṃ ṭhāne parisuddho hoti.

‘‘Puna caparaṃ, nigrodha, tapassī tapaṃ samādiyati, so tena tapasā lābhasakkārasilokaṃ abhinibbatteti, so tena lābhasakkārasilokena na majjati na mucchati na pamādamāpajjati…pe… evaṃ so tasmiṃ ṭhāne parisuddho hoti.

\paragraph{66.} ‘‘Puna caparaṃ, nigrodha, tapassī bhojanesu na vodāsaṃ āpajjati – ‘idaṃ me khamati, idaṃ me nakkhamatī’ti. So yañca khvassa nakkhamati, taṃ anapekkho pajahati. Yaṃ panassa khamati , taṃ agadhito amucchito anajjhāpanno ādīnavadassāvī nissaraṇapañño paribhuñjati…pe… evaṃ so tasmiṃ ṭhāne parisuddho hoti.

‘‘Puna caparaṃ, nigrodha, tapassī na tapaṃ samādiyati lābhasakkārasilokanikantihetu – ‘sakkarissanti maṃ rājāno rājamahāmattā khattiyā brāhmaṇā gahapatikā titthiyā’ti…pe… evaṃ so tasmiṃ ṭhāne parisuddho hoti.

\paragraph{67.} ‘‘Puna caparaṃ, nigrodha, tapassī aññataraṃ samaṇaṃ vā brāhmaṇaṃ vā nāpasādetā hoti – ‘kiṃ panāyaṃ sambahulājīvo sabbaṃ saṃbhakkheti. Seyyathidaṃ – mūlabījaṃ khandhabījaṃ phaḷubījaṃ aggabījaṃ bījabījameva pañcamaṃ, asanivicakkaṃ dantakūṭaṃ, samaṇappavādenā’ti…pe… evaṃ so tasmiṃ ṭhāne parisuddho hoti.

‘‘Puna caparaṃ, nigrodha, tapassī passati aññataraṃ samaṇaṃ vā brāhmaṇaṃ vā kulesu sakkariyamānaṃ garu kariyamānaṃ māniyamānaṃ pūjiyamānaṃ. Disvā tassa na evaṃ hoti – ‘imañhi nāma sambahulājīvaṃ kulesu sakkaronti garuṃ karonti mānenti pūjenti. Maṃ pana tapassiṃ lūkhājīviṃ kulesu na sakkaronti na garuṃ karonti na mānenti na pūjentī’ti, iti so issāmacchariyaṃ kulesu nuppādetā hoti…pe… evaṃ so tasmiṃ ṭhāne parisuddho hoti.

\paragraph{68.} ‘‘Puna caparaṃ, nigrodha, tapassī na āpāthakanisādī hoti…pe… evaṃ so tasmiṃ ṭhāne parisuddho hoti.

‘‘Puna caparaṃ, nigrodha, tapassī na attānaṃ adassayamāno kulesu carati – ‘idampi me tapasmiṃ, idampi me tapasmi’nti…pe… evaṃ so tasmiṃ ṭhāne parisuddho hoti.

‘‘Puna caparaṃ, nigrodha, tapassī na kañcideva paṭicchannaṃ sevati, so – ‘khamati te ida’nti puṭṭho samāno akkhamamānaṃ āha – ‘nakkhamatī’ti. Khamamānaṃ āha – ‘khamatī’ti. Iti so sampajānamusā na bhāsitā hoti…pe… evaṃ so tasmiṃ ṭhāne parisuddho hoti.

‘‘Puna caparaṃ, nigrodha, tapassī tathāgatassa vā tathāgatasāvakassa vā dhammaṃ desentassa santaṃyeva pariyāyaṃ anuññeyyaṃ anujānāti…pe… evaṃ so tasmiṃ ṭhāne parisuddho hoti.

\paragraph{69.} ‘‘Puna caparaṃ, nigrodha, tapassī akkodhano hoti anupanāhī. Yampi, nigrodha, tapassī akkodhano hoti anupanāhī evaṃ so tasmiṃ ṭhāne parisuddho hoti.

‘‘Puna caparaṃ, nigrodha, tapassī amakkhī hoti apaḷāsī…pe… anissukī hoti amaccharī… asaṭho hoti amāyāvī… atthaddho hoti anatimānī… na pāpiccho hoti na pāpikānaṃ icchānaṃ vasaṃ gato… na micchādiṭṭhiko hoti na antaggāhikāya diṭṭhiyā samannāgato… na sandiṭṭhiparāmāsī hoti na ādhānaggāhī suppaṭinissaggī. Yampi, nigrodha, tapassī na sandiṭṭhiparāmāsī hoti na ādhānaggāhī suppaṭinissaggī. Evaṃ so tasmiṃ ṭhāne parisuddho hoti.

‘‘Taṃ kiṃ maññasi, nigrodha, yadi evaṃ sante tapojigucchā parisuddhā vā hoti aparisuddhā vā’’ti? ‘‘Addhā kho, bhante, evaṃ sante tapojigucchā parisuddhā hoti no aparisuddhā, aggappattā ca sārappattā cā’’ti. ‘‘Na kho, nigrodha, ettāvatā tapojigucchā aggappattā ca hoti sārappattā ca; api ca kho papaṭikappattā\footnote{pappaṭikapattā (ka.)} hotī’’ti.

\subsubsection{Parisuddhatacappattakathā}

\paragraph{70.} ‘‘Kittāvatā pana, bhante, tapojigucchā aggappattā ca hoti sārappattā ca? Sādhu me, bhante, bhagavā tapojigucchāya aggaññeva pāpetu, sāraññeva pāpetū’’ti. ‘‘Idha, nigrodha, tapassī cātuyāmasaṃvarasaṃvuto hoti. Kathañca, nigrodha, tapassī cātuyāmasaṃvarasaṃvuto hoti? Idha, nigrodha, tapassī na pāṇaṃ atipāteti\footnote{atipāpeti (ka. sī. pī. ka.)}, na pāṇaṃ atipātayati, na pāṇamatipātayato samanuñño hoti . Na adinnaṃ ādiyati, na adinnaṃ ādiyāpeti, na adinnaṃ ādiyato samanuñño hoti. Na musā bhaṇati, na musā bhaṇāpeti, na musā bhaṇato samanuñño hoti. Na bhāvitamāsīsati\footnote{na bhāvitamāsiṃ sati (sī. syā. pī.)}, na bhāvitamāsīsāpeti, na bhāvitamāsīsato samanuñño hoti. Evaṃ kho, nigrodha, tapassī cātuyāmasaṃvarasaṃvuto hoti.

‘‘Yato kho, nigrodha, tapassī cātuyāmasaṃvarasaṃvuto hoti, aduṃ cassa hoti tapassitāya. So abhiharati no hīnāyāvattati. So vivittaṃ senāsanaṃ bhajati araññaṃ rukkhamūlaṃ pabbataṃ kandaraṃ giriguhaṃ susānaṃ vanapatthaṃ abbhokāsaṃ palālapuñjaṃ. So pacchābhattaṃ piṇḍapātappaṭikkanto nisīdati pallaṅkaṃ ābhujitvā ujuṃ kāyaṃ paṇidhāya parimukhaṃ satiṃ upaṭṭhapetvā. So abhijjhaṃ loke pahāya vigatābhijjhena cetasā viharati, abhijjhāya cittaṃ parisodheti. Byāpādappadosaṃ pahāya abyāpannacitto viharati sabbapāṇabhūtahitānukampī, byāpādappadosā cittaṃ parisodheti. Thinamiddhaṃ\footnote{thīnamiddhaṃ (sī. syā. pī.)} pahāya vigatathinamiddho viharati ālokasaññī sato sampajāno, thinamiddhā cittaṃ parisodheti. Uddhaccakukkuccaṃ pahāya anuddhato viharati ajjhattaṃ vūpasantacitto, uddhaccakukkuccā cittaṃ parisodheti. Vicikicchaṃ pahāya tiṇṇavicikiccho viharati akathaṃkathī kusalesu dhammesu, vicikicchāya cittaṃ parisodheti.

\paragraph{71.} ‘‘So ime pañca nīvaraṇe pahāya cetaso upakkilese paññāya dubbalīkaraṇe mettāsahagatena cetasā ekaṃ disaṃ pharitvā viharati. Tathā dutiyaṃ. Tathā tatiyaṃ. Tathā catutthaṃ. Iti uddhamadho tiriyaṃ sabbadhi sabbattatāya sabbāvantaṃ lokaṃ mettāsahagatena cetasā vipulena mahaggatena appamāṇena averena abyāpajjena pharitvā viharati. Karuṇāsahagatena cetasā…pe… muditāsahagatena cetasā…pe… upekkhāsahagatena cetasā ekaṃ disaṃ pharitvā viharati. Tathā dutiyaṃ. Tathā tatiyaṃ. Tathā catutthaṃ. Iti uddhamadho tiriyaṃ sabbadhi sabbattatāya sabbāvantaṃ lokaṃ upekkhāsahagatena cetasā vipulena mahaggatena appamāṇena averena abyāpajjena pharitvā viharati.

‘‘Taṃ kiṃ maññasi, nigrodha. Yadi evaṃ sante tapojigucchā parisuddhā vā hoti aparisuddhā vā’’ti? ‘‘Addhā kho, bhante, evaṃ sante tapojigucchā parisuddhā hoti no aparisuddhā, aggappattā ca sārappattā cā’’ti. ‘‘Na kho, nigrodha, ettāvatā tapojigucchā aggappattā ca hoti sārappattā ca; api ca kho tacappattā hotī’’ti.

\subsubsection{Parisuddhaphegguppattakathā}

\paragraph{72.} ‘‘Kittāvatā pana, bhante, tapojigucchā aggappattā ca hoti sārappattā ca? Sādhu me, bhante, bhagavā tapojigucchāya aggaññeva pāpetu, sāraññeva pāpetū’’ti. ‘‘Idha, nigrodha, tapassī cātuyāmasaṃvarasaṃvuto hoti. Kathañca, nigrodha, tapassī cātuyāmasaṃvarasaṃvuto hoti…pe… yato kho, nigrodha, tapassī cātuyāmasaṃvarasaṃvuto hoti, aduṃ cassa hoti tapassitāya. So abhiharati no hīnāyāvattati. So vivittaṃ senāsanaṃ bhajati…pe… so ime pañca nīvaraṇe pahāya cetaso upakkilese paññāya dubbalīkaraṇe mettāsahagatena cetasā…pe… karuṇāsahagatena cetasā…pe… muditāsahagatena cetasā…pe… upekkhāsahagatena cetasā vipulena mahaggatena appamāṇena averena abyāpajjena pharitvā viharati. So anekavihitaṃ pubbenivāsaṃ anussarati seyyathidaṃ – ekampi jātiṃ dvepi jātiyo tissopi jātiyo catassopi jātiyo pañcapi jātiyo dasapi jātiyo vīsampi jātiyo tiṃsampi jātiyo cattālīsampi jātiyo paññāsampi jātiyo jātisatampi jātisahassampi jātisatasahassampi anekepi saṃvaṭṭakappe anekepi vivaṭṭakappe anekepi saṃvaṭṭavivaṭṭakappe – ‘amutrāsiṃ evaṃnāmo evaṃgotto evaṃvaṇṇo evamāhāro evaṃsukhadukkhappaṭisaṃvedī evamāyupariyanto, so tato cuto amutra udapādiṃ, tatrāpāsiṃ evaṃnāmo evaṃgotto evaṃvaṇṇo evamāhāro evaṃsukhadukkhappaṭisaṃvedī evamāyupariyanto, so tato cuto idhūpapanno’ti. Iti sākāraṃ sauddesaṃ anekavihitaṃ pubbenivāsaṃ anussarati.

‘‘Taṃ kiṃ maññasi, nigrodha, yadi evaṃ sante tapojigucchā parisuddhā vā hoti aparisuddhā vā’’ti? ‘‘Addhā kho, bhante, evaṃ sante tapojigucchā parisuddhā hoti, no aparisuddhā, aggappattā ca sārappattā cā’’ti. ‘‘Na kho, nigrodha, ettāvatā tapojigucchā aggappattā ca hoti sārappattā ca; api ca kho phegguppattā hotī’’ti.

\subsubsection{Parisuddhaaggappattasārappattakathā}

\paragraph{73.} ‘‘Kittāvatā pana, bhante, tapojigucchā aggappattā ca hoti sārappattā ca? Sādhu me, bhante, bhagavā tapojigucchāya aggaññeva pāpetu, sāraññeva pāpetū’’ti. ‘‘Idha, nigrodha, tapassī cātuyāmasaṃvarasaṃvuto hoti. Kathañca, nigrodha, tapassī cātuyāmasaṃvarasaṃvuto hoti…pe… yato kho, nigrodha, tapassī cātuyāmasaṃvarasaṃvuto hoti, aduṃ cassa hoti tapassitāya. So abhiharati no hīnāyāvattati. So vivittaṃ senāsanaṃ bhajati…pe… so ime pañca nīvaraṇe pahāya cetaso upakkilese paññāya dubbalīkaraṇe mettāsahagatena cetasā…pe… upekkhāsahagatena cetasā vipulena mahaggatena appamāṇena averena abyāpajjena pharitvā viharati. So anekavihitaṃ pubbenivāsaṃ anussarati. Seyyathidaṃ – ekampi jātiṃ dvepi jātiyo tissopi jātiyo catassopi jātiyo pañcapi jātiyo…pe… iti sākāraṃ sauddesaṃ anekavihitaṃ pubbenivāsaṃ anussarati. So dibbena cakkhunā visuddhena atikkantamānusakena satte passati cavamāne upapajjamāne hīne paṇīte suvaṇṇe dubbaṇṇe sugate duggate, yathākammūpage satte pajānāti – ‘ime vata bhonto sattā kāyaduccaritena samannāgatā vacīduccaritena samannāgatā manoduccaritena samannāgatā ariyānaṃ upavādakā micchādiṭṭhikā micchādiṭṭhikammasamādānā. Te kāyassa bhedā paraṃ maraṇā apāyaṃ duggatiṃ vinipātaṃ nirayaṃ upapannā. Ime vā pana bhonto sattā kāyasucaritena samannāgatā vacīsucaritena samannāgatā manosucaritena samannāgatā ariyānaṃ anupavādakā sammādiṭṭhikā sammādiṭṭhikammasamādānā. Te kāyassa bhedā paraṃ maraṇā sugatiṃ saggaṃ lokaṃ upapannā’ti. Iti dibbena cakkhunā visuddhena atikkantamānusakena satte passati cavamāne upapajjamāne hīne paṇīte suvaṇṇe dubbaṇṇe sugate duggate, yathākammūpage satte pajānāti.

‘‘Taṃ kiṃ maññasi, nigrodha, yadi evaṃ sante tapojigucchā parisuddhā vā hoti aparisuddhā vā’’ti? ‘‘Addhā kho, bhante, evaṃ sante tapojigucchā parisuddhā hoti no aparisuddhā, aggappattā ca sārappattā cā’’ti.

\paragraph{74.} ‘‘Ettāvatā kho, nigrodha, tapojigucchā aggappattā ca hoti sārappattā ca. Iti kho, nigrodha\footnote{iti nigrodha (syā.)}, yaṃ maṃ tvaṃ avacāsi – ‘ko nāma so, bhante, bhagavato dhammo, yena bhagavā sāvake vineti, yena bhagavatā sāvakā vinītā assāsappattā paṭijānanti ajjhāsayaṃ ādibrahmacariya’nti. Iti kho taṃ, nigrodha, ṭhānaṃ uttaritarañca paṇītatarañca, yenāhaṃ sāvake vinemi, yena mayā sāvakā vinītā assāsappattā paṭijānanti ajjhāsayaṃ ādibrahmacariya’’nti.

Evaṃ vutte, te paribbājakā unnādino uccāsaddamahāsaddā ahesuṃ – ‘‘ettha mayaṃ anassāma sācariyakā, na mayaṃ ito bhiyyo uttaritaraṃ pajānāmā’’ti.

\subsubsection{Nigrodhassa pajjhāyanaṃ}

\paragraph{75.} Yadā aññāsi sandhāno gahapati – ‘‘aññadatthu kho dānime aññatitthiyā paribbājakā bhagavato bhāsitaṃ sussūsanti, sotaṃ odahanti, aññācittaṃ upaṭṭhāpentī’’ti. Atha\footnote{atha naṃ (ka.)} nigrodhaṃ paribbājakaṃ etadavoca – ‘‘iti kho, bhante nigrodha, yaṃ maṃ tvaṃ avacāsi – ‘yagghe, gahapati, jāneyyāsi, kena samaṇo gotamo saddhiṃ sallapati, kena sākacchaṃ samāpajjati, kena paññāveyyattiyaṃ samāpajjati, suññāgārahatā samaṇassa gotamassa paññā, aparisāvacaro samaṇo gotamo nālaṃ sallāpāya, so antamantāneva sevati; seyyathāpi nāma gokāṇā pariyantacārinī antamantāneva sevati. Evameva suññāgārahatā samaṇassa gotamassa paññā, aparisāvacaro samaṇo gotamo nālaṃ sallāpāya; so antamantāneva sevati; iṅgha, gahapati, samaṇo gotamo imaṃ parisaṃ āgaccheyya, ekapañheneva naṃ saṃsādeyyāma, tucchakumbhīva naṃ maññe orodheyyāmā’ti. Ayaṃ kho so, bhante, bhagavā arahaṃ sammāsambuddho idhānuppatto, aparisāvacaraṃ pana naṃ karotha, gokāṇaṃ pariyantacāriniṃ karotha, ekapañheneva naṃ saṃsādetha, tucchakumbhīva naṃ orodhethā’’ti. Evaṃ vutte, nigrodho paribbājako tuṇhībhūto maṅkubhūto pattakkhandho adhomukho pajjhāyanto appaṭibhāno nisīdi.

\paragraph{76.} Atha kho bhagavā nigrodhaṃ paribbājakaṃ tuṇhībhūtaṃ maṅkubhūtaṃ pattakkhandhaṃ adhomukhaṃ pajjhāyantaṃ appaṭibhānaṃ viditvā nigrodhaṃ paribbājakaṃ etadavoca – ‘‘saccaṃ kira, nigrodha, bhāsitā te esā vācā’’ti? ‘‘Saccaṃ , bhante, bhāsitā me esā vācā, yathābālena yathāmūḷhena yathāakusalenā’’ti. ‘‘Taṃ kiṃ maññasi, nigrodha. Kinti te sutaṃ paribbājakānaṃ vuḍḍhānaṃ mahallakānaṃ ācariyapācariyānaṃ bhāsamānānaṃ – ‘ye te ahesuṃ atītamaddhānaṃ arahanto sammāsambuddhā, evaṃ su te bhagavanto saṃgamma samāgamma unnādino uccāsaddamahāsaddā anekavihitaṃ tiracchānakathaṃ anuyuttā viharanti. Seyyathidaṃ – rājakathaṃ corakathaṃ…pe… itibhavābhavakathaṃ iti vā. Seyyathāpi tvaṃ etarahi sācariyako. Udāhu, evaṃ su te bhagavanto araññavanapatthāni pantāni senāsanāni paṭisevanti appasaddāni appanigghosāni vijanavātāni manussarāhasseyyakāni paṭisallānasāruppāni, seyyathāpāhaṃ etarahī’ti.

‘‘Sutaṃ metaṃ, bhante. Paribbājakānaṃ vuḍḍhānaṃ mahallakānaṃ ācariyapācariyānaṃ bhāsamānānaṃ – ‘ye te ahesuṃ atītamaddhānaṃ arahanto sammāsambuddhā , na evaṃ su\footnote{nāssu (sī. pī.)} te bhagavanto saṃgamma samāgamma unnādino uccāsaddamahāsaddā anekavihitaṃ tiracchānakathaṃ anuyuttā viharanti. Seyyathidaṃ – rājakathaṃ corakathaṃ…pe… itibhavābhavakathaṃ iti vā, seyyathāpāhaṃ etarahi sācariyako. Evaṃ su te bhagavanto araññavanapatthāni pantāni senāsanāni paṭisevanti appasaddāni appanigghosāni vijanavātāni manussarāhasseyyakāni paṭisallānasāruppāni, seyyathāpi bhagavā etarahī’’’ti.

‘‘Tassa te, nigrodha, viññussa sato mahallakassa na etadahosi – ‘buddho so bhagavā bodhāya dhammaṃ deseti, danto so bhagavā damathāya dhammaṃ deseti, santo so bhagavā samathāya dhammaṃ deseti, tiṇṇo so bhagavā taraṇāya dhammaṃ deseti, parinibbuto so bhagavā parinibbānāya dhammaṃ desetī’’’ti?

\subsubsection{Brahmacariyapariyosānasacchikiriyā}

\paragraph{77.} Evaṃ vutte, nigrodho paribbājako bhagavantaṃ etadavoca – ‘‘accayo maṃ, bhante, accagamā yathābālaṃ yathāmūḷhaṃ yathāakusalaṃ, yvāhaṃ evaṃ bhagavantaṃ avacāsiṃ. Tassa me, bhante, bhagavā accayaṃ accayato paṭiggaṇhātu āyatiṃ saṃvarāyā’’ti. ‘‘Taggha tvaṃ\footnote{taṃ (sī. syā. pī.)}, nigrodha, accayo accagamā yathābālaṃ yathāmūḷhaṃ yathāakusalaṃ, yo maṃ tvaṃ evaṃ avacāsi. Yato ca kho tvaṃ, nigrodha, accayaṃ accayato disvā yathādhammaṃ paṭikarosi, taṃ te mayaṃ paṭiggaṇhāma. Vuddhi hesā, nigrodha, ariyassa vinaye, yo accayaṃ accayato disvā yathādhammaṃ paṭikaroti āyatiṃ saṃvaraṃ āpajjati. Ahaṃ kho pana, nigrodha, evaṃ vadāmi –

‘Etu viññū puriso asaṭho amāyāvī ujujātiko, ahamanusāsāmi ahaṃ dhammaṃ desemi. Yathānusiṭṭhaṃ tathā\footnote{yathānusiṭṭhaṃ (?)} paṭipajjamāno, yassatthāya kulaputtā sammadeva agārasmā anagāriyaṃ pabbajanti, tadanuttaraṃ brahmacariyapariyosānaṃ diṭṭheva dhamme sayaṃ abhiññā sacchikatvā upasampajja viharissati sattavassāni. Tiṭṭhantu, nigrodha, satta vassāni. Etu viññū puriso asaṭho amāyāvī ujujātiko, ahamanusāsāmi ahaṃ dhammaṃ desemi. Yathānusiṭṭhaṃ tathā paṭipajjamāno, yassatthāya kulaputtā sammadeva agārasmā anagāriyaṃ pabbajanti, tadanuttaraṃ brahmacariyapariyosānaṃ diṭṭheva dhamme sayaṃ abhiññā sacchikatvā upasampajja viharissati cha vassāni. Pañca vassāni… cattāri vassāni… tīṇi vassāni… dve vassāni… ekaṃ vassaṃ. Tiṭṭhatu, nigrodha, ekaṃ vassaṃ. Etu viññū puriso asaṭho amāyāvī ujujātiko ahamanusāsāmi ahaṃ dhammaṃ desemi. Yathānusiṭṭhaṃ tathā paṭipajjamāno, yassatthāya kulaputtā sammadeva agārasmā anagāriyaṃ pabbajanti, tadanuttaraṃ brahmacariyapariyosānaṃ diṭṭheva dhamme sayaṃ abhiññā sacchikatvā upasampajja viharissati satta māsāni. Tiṭṭhantu, nigrodha, satta māsāni… cha māsāni… pañca māsāni … cattāri māsāni… tīṇi māsāni… dve māsāni… ekaṃ māsaṃ… aḍḍhamāsaṃ. Tiṭṭhatu, nigrodha, aḍḍhamāso. Etu viññū puriso asaṭho amāyāvī ujujātiko, ahamanusāsāmi ahaṃ dhammaṃ desemi. Yathānusiṭṭhaṃ tathā paṭipajjamāno, yassatthāya kulaputtā sammadeva agārasmā anagāriyaṃ pabbajanti, tadanuttaraṃ brahmacariyapariyosānaṃ diṭṭheva dhamme sayaṃ abhiññā sacchikatvā upasampajja viharissati sattāhaṃ’.

\subsubsection{Paribbājakānaṃ pajjhāyanaṃ}

\paragraph{78.} ‘‘Siyā kho pana te, nigrodha, evamassa – ‘antevāsikamyatā no samaṇo gotamo evamāhā’ti. Na kho panetaṃ, nigrodha, evaṃ daṭṭhabbaṃ. Yo eva vo\footnote{te (sī. syā.)} ācariyo, so eva vo ācariyo hotu. Siyā kho pana te, nigrodha, evamassa – ‘uddesā no cāvetukāmo samaṇo gotamo evamāhā’ti. Na kho panetaṃ, nigrodha , evaṃ daṭṭhabbaṃ. Yo eva vo uddeso so eva vo uddeso hotu. Siyā kho pana te, nigrodha, evamassa – ‘ājīvā no cāvetukāmo samaṇo gotamo evamāhā’ti. Na kho panetaṃ, nigrodha, evaṃ daṭṭhabbaṃ. Yo eva vo ājīvo, so eva vo ājīvo hotu. Siyā kho pana te, nigrodha, evamassa – ‘ye no dhammā akusalā akusalasaṅkhātā sācariyakānaṃ, tesu patiṭṭhāpetukāmo samaṇo gotamo evamāhā’ti. Na kho panetaṃ, nigrodha, evaṃ daṭṭhabbaṃ. Akusalā ceva vo te dhammā\footnote{vodhammā (ka.), te dhammā (syā.)} hontu akusalasaṅkhātā ca sācariyakānaṃ. Siyā kho pana te , nigrodha, evamassa – ‘ye no dhammā kusalā kusalasaṅkhātā sācariyakānaṃ, tehi vivecetukāmo samaṇo gotamo evamāhā’ti. Na kho panetaṃ, nigrodha, evaṃ daṭṭhabbaṃ. Kusalā ceva vo te dhammā hontu kusalasaṅkhātā ca sācariyakānaṃ. Iti khvāhaṃ, nigrodha, neva antevāsikamyatā evaṃ vadāmi, napi uddesā cāvetukāmo evaṃ vadāmi, napi ājīvā cāvetukāmo evaṃ vadāmi, napi ye vo dhammā\footnote{napi ye kho dhammā (sī.), napi ye te dhammā (syā.), napi ye ca vo dhammā (ka.)} akusalā akusalasaṅkhātā sācariyakānaṃ, tesu patiṭṭhāpetukāmo evaṃ vadāmi, napi ye vo dhammā\footnote{napi ye kho dhammā (sī.), napi ye te dhammā (syā.), napi ye ca vo dhammā (ka.)} kusalā kusalasaṅkhātā sācariyakānaṃ, tehi vivecetukāmo evaṃ vadāmi. Santi ca kho, nigrodha, akusalā dhammā appahīnā saṃkilesikā ponobbhavikā\footnote{ponobhavikā (ka.)} sadarā\footnote{saddarā (pī. ka.), sadarathā (syā. ka.)} dukkhavipākā āyatiṃ jātijarāmaraṇiyā, yesāhaṃ pahānāya dhammaṃ desemi. Yathāpaṭipannānaṃ vo saṃkilesikā dhammā pahīyissanti, vodānīyā dhammā abhivaḍḍhissanti, paññāpāripūriṃ vepullattañca diṭṭheva dhamme sayaṃ abhiññā sacchikatvā upasampajja viharissathā’’ti.

\paragraph{79.} Evaṃ vutte, te paribbājakā tuṇhībhūtā maṅkubhūtā pattakkhandhā adhomukhā pajjhāyantā appaṭibhānā nisīdiṃsu yathā taṃ mārena pariyuṭṭhitacittā. Atha kho bhagavato etadahosi – ‘‘sabbe pime moghapurisā phuṭṭhā pāpimatā. Yatra hi nāma ekassapi na evaṃ bhavissati – ‘handa mayaṃ aññāṇatthampi samaṇe gotame brahmacariyaṃ carāma, kiṃ karissati sattāho’’’ti? Atha kho bhagavā udumbarikāya paribbājakārāme sīhanādaṃ naditvā vehāsaṃ abbhuggantvā gijjhakūṭe pabbate paccupaṭṭhāsi\footnote{paccuṭṭhāsi (sī. syā. pī.)}. Sandhāno pana gahapati tāvadeva rājagahaṃ pāvisīti.

\xsectionEnd{Udumbarikasuttaṃ niṭṭhitaṃ dutiyaṃ.}


\section{Cakkavattisuttaṃ}

\subsubsection{Attadīpasaraṇatā}

\paragraph{80.} Evaṃ me sutaṃ – ekaṃ samayaṃ bhagavā magadhesu viharati mātulāyaṃ. Tatra kho bhagavā bhikkhū āmantesi – ‘‘bhikkhavo’’ti. ‘‘Bhaddante’’ti te bhikkhū bhagavato paccassosuṃ. Bhagavā etadavoca – ‘‘attadīpā, bhikkhave, viharatha attasaraṇā anaññasaraṇā, dhammadīpā dhammasaraṇā anaññasaraṇā. Kathañca pana, bhikkhave, bhikkhu attadīpo viharati attasaraṇo anaññasaraṇo, dhammadīpo dhammasaraṇo anaññasaraṇo? Idha, bhikkhave, bhikkhu kāye kāyānupassī viharati ātāpī sampajāno satimā vineyya loke abhijjhādomanassaṃ. Vedanāsu vedanānupassī…pe… citte cittānupassī…pe… dhammesu dhammānupassī viharati ātāpī sampajāno satimā vineyya loke abhijjhādomanassaṃ. Evaṃ kho, bhikkhave, bhikkhu attadīpo viharati attasaraṇo anaññasaraṇo, dhammadīpo dhammasaraṇo anaññasaraṇo.

‘‘Gocare, bhikkhave, caratha sake pettike visaye. Gocare, bhikkhave, carataṃ sake pettike visaye na lacchati māro otāraṃ, na lacchati māro ārammaṇaṃ\footnote{āramaṇaṃ (?)}. Kusalānaṃ, bhikkhave, dhammānaṃ samādānahetu evamidaṃ puññaṃ pavaḍḍhati.

\subsubsection{Daḷhanemicakkavattirājā}

\paragraph{81.} ‘‘Bhūtapubbaṃ , bhikkhave, rājā daḷhanemi nāma ahosi cakkavattī\footnote{cakkavatti (syā. pī.)} dhammiko dhammarājā cāturanto vijitāvī janapadatthāvariyappatto sattaratanasamannāgato. Tassimāni satta ratanāni ahesuṃ seyyathidaṃ – cakkaratanaṃu hatthiratanaṃ assaratanaṃ maṇiratanaṃ itthiratanaṃ gahapatiratanaṃ pariṇāyakaratanameva sattamaṃ. Parosahassaṃ kho panassa puttā ahesuṃ sūrā vīraṅgarūpā parasenappamaddanā. So imaṃ pathaviṃ sāgarapariyantaṃ adaṇḍena asatthena dhammena\footnote{dhammena samena (syā. ka.)} abhivijiya ajjhāvasi.

\paragraph{82.} ‘‘Atha kho, bhikkhave, rājā daḷhanemi bahunnaṃ vassānaṃ bahunnaṃ vassasatānaṃ bahunnaṃ vassasahassānaṃ accayena aññataraṃ purisaṃ āmantesi – ‘yadā tvaṃ, ambho purisa, passeyyāsi dibbaṃ cakkaratanaṃ osakkitaṃ ṭhānā cutaṃ, atha me āroceyyāsī’ti. ‘Evaṃ, devā’ti kho, bhikkhave, so puriso rañño daḷhanemissa paccassosi. Addasā kho, bhikkhave, so puriso bahunnaṃ vassānaṃ bahunnaṃ vassasatānaṃ bahunnaṃ vassasahassānaṃ accayena dibbaṃ cakkaratanaṃ osakkitaṃ ṭhānā cutaṃ, disvāna yena rājā daḷhanemi tenupasaṅkami; upasaṅkamitvā rājānaṃ daḷhanemiṃ etadavoca – ‘yagghe, deva, jāneyyāsi, dibbaṃ te cakkaratanaṃ osakkitaṃ ṭhānā cuta’nti. Atha kho, bhikkhave, rājā daḷhanemi jeṭṭhaputtaṃ kumāraṃ āmantāpetvā\footnote{āmantetvā (syā. ka.)} etadavoca – ‘dibbaṃ kira me, tāta kumāra, cakkaratanaṃ osakkitaṃ ṭhānā cutaṃ. Sutaṃ kho pana metaṃ – yassa rañño cakkavattissa dibbaṃ cakkaratanaṃ osakkati ṭhānā cavati, na dāni tena raññā ciraṃ jīvitabbaṃ hotīti. Bhuttā kho pana me mānusakā kāmā, samayo dāni me dibbe kāme pariyesituṃ. Ehi tvaṃ, tāta kumāra, imaṃ samuddapariyantaṃ pathaviṃ paṭipajja. Ahaṃ pana kesamassuṃ ohāretvā kāsāyāni vatthāni acchādetvā agārasmā anagāriyaṃ pabbajissāmī’ti.

\paragraph{83.} ‘‘Atha kho, bhikkhave, rājā daḷhanemi jeṭṭhaputtaṃ kumāraṃ sādhukaṃ rajje samanusāsitvā kesamassuṃ ohāretvā kāsāyāni vatthāni acchādetvā agārasmā anagāriyaṃ pabbaji. Sattāhapabbajite kho pana, bhikkhave, rājisimhi dibbaṃ cakkaratanaṃ antaradhāyi.

‘‘Atha kho, bhikkhave, aññataro puriso yena rājā khattiyo muddhābhisitto\footnote{muddhāvasitto (sī. syā. pī.) evamuparipi} tenupasaṅkami; upasaṅkamitvā rājānaṃ khattiyaṃ muddhābhisittaṃ etadavoca – ‘yagghe, deva, jāneyyāsi, dibbaṃ cakkaratanaṃ antarahita’nti. Atha kho, bhikkhave, rājā khattiyo muddhābhisitto dibbe cakkaratane antarahite anattamano ahosi, anattamanatañca paṭisaṃvedesi. So yena rājisi tenupasaṅkami; upasaṅkamitvā rājisiṃ etadavoca – ‘yagghe, deva, jāneyyāsi, dibbaṃ cakkaratanaṃ antarahita’nti. Evaṃ vutte, bhikkhave, rājisi rājānaṃ khattiyaṃ muddhābhisittaṃ etadavoca – ‘mā kho tvaṃ, tāta, dibbe cakkaratane antarahite anattamano ahosi, mā anattamanatañca paṭisaṃvedesi, na hi te, tāta, dibbaṃ cakkaratanaṃ pettikaṃ dāyajjaṃ. Iṅgha tvaṃ, tāta, ariye cakkavattivatte vattāhi. Ṭhānaṃ kho panetaṃ vijjati, yaṃ te ariye cakkavattivatte vattamānassa tadahuposathe pannarase sīsaṃnhātassa\footnote{sīsaṃ nahātassa (sī. pī.), sīsanhātassa (syā.)} uposathikassa uparipāsādavaragatassa dibbaṃ cakkaratanaṃ pātubhavissati sahassāraṃ sanemikaṃ sanābhikaṃ sabbākāraparipūra’nti.

\subsubsection{Cakkavattiariyavattaṃ}

\paragraph{84.} ‘‘‘Katamaṃ pana taṃ, deva, ariyaṃ cakkavattivatta’nti ? ‘Tena hi tvaṃ, tāta, dhammaṃyeva nissāya dhammaṃ sakkaronto dhammaṃ garuṃ karonto\footnote{garukaronto (sī. syā. pī.)} dhammaṃ mānento dhammaṃ pūjento dhammaṃ apacāyamāno dhammaddhajo dhammaketu dhammādhipateyyo dhammikaṃ rakkhāvaraṇaguttiṃ saṃvidahassu antojanasmiṃ balakāyasmiṃ khattiyesu anuyantesu\footnote{anuyuttesu (sī. pī.)} brāhmaṇagahapatikesu negamajānapadesu samaṇabrāhmaṇesu migapakkhīsu. Mā ca te, tāta, vijite adhammakāro pavattittha. Ye ca te, tāta, vijite adhanā assu, tesañca dhanamanuppadeyyāsi\footnote{dhanamanuppadajjeyyāsi (sī. syā. pī.)}. Ye ca te, tāta, vijite samaṇabrāhmaṇā madappamādā paṭiviratā khantisoracce niviṭṭhā ekamattānaṃ damenti, ekamattānaṃ samenti, ekamattānaṃ parinibbāpenti, te kālena kālaṃ upasaṅkamitvā paripuccheyyāsi pariggaṇheyyāsi – ‘‘kiṃ, bhante, kusalaṃ, kiṃ akusalaṃ, kiṃ sāvajjaṃ, kiṃ anavajjaṃ, kiṃ sevitabbaṃ, kiṃ na sevitabbaṃ, kiṃ me karīyamānaṃ dīgharattaṃ ahitāya dukkhāya assa, kiṃ vā pana me karīyamānaṃ dīgharattaṃ hitāya sukhāya assā’’ti? Tesaṃ sutvā yaṃ akusalaṃ taṃ abhinivajjeyyāsi, yaṃ kusalaṃ taṃ samādāya vatteyyāsi. Idaṃ kho, tāta, taṃ ariyaṃ cakkavattivatta’nti.

\subsubsection{Cakkaratanapātubhāvo}

\paragraph{85.} ‘‘‘Evaṃ, devā’ti kho, bhikkhave, rājā khattiyo muddhābhisitto rājisissa paṭissutvā ariye cakkavattivatte\footnote{ariyaṃ cakkavattivattaṃ (ka.)} vatti. Tassa ariye cakkavattivatte vattamānassa tadahuposathe pannarase sīsaṃnhātassa uposathikassa uparipāsādavaragatassa dibbaṃ cakkaratanaṃ pāturahosi sahassāraṃ sanemikaṃ sanābhikaṃ sabbākāraparipūraṃ. Disvāna rañño khattiyassa muddhābhisittassa etadahosi – ‘sutaṃ kho pana metaṃ – yassa rañño khattiyassa muddhābhisittassa tadahuposathe pannarase sīsaṃnhātassa uposathikassa uparipāsādavaragatassa dibbaṃ cakkaratanaṃ pātubhavati sahassāraṃ sanemikaṃ sanābhikaṃ sabbākāraparipūraṃ , so hoti rājā cakkavattī’ti. Assaṃ nu kho ahaṃ rājā cakkavattīti.

‘‘Atha kho, bhikkhave, rājā khattiyo muddhābhisitto uṭṭhāyāsanā ekaṃsaṃ utarāsaṅgaṃ karitvā vāmena hatthena bhiṅkāraṃ gahetvā dakkhiṇena hatthena cakkaratanaṃ abbhukkiri – ‘pavattatu bhavaṃ cakkaratanaṃ, abhivijinātu bhavaṃ cakkaratana’nti.

‘‘Atha kho taṃ, bhikkhave, cakkaratanaṃ puratthimaṃ disaṃ pavatti, anvadeva rājā cakkavattī saddhiṃ caturaṅginiyā senāya. Yasmiṃ kho pana, bhikkhave, padese cakkaratanaṃ patiṭṭhāsi, tattha rājā cakkavattī vāsaṃ upagacchi saddhiṃ caturaṅginiyā senāya. Ye kho pana, bhikkhave, puratthimāya disāya paṭirājāno, te rājānaṃ cakkavattiṃ upasaṅkamitvā evamāhaṃsu – ‘ehi kho, mahārāja, svāgataṃ te\footnote{sāgataṃ (sī. pī.)} mahārāja, sakaṃ te, mahārāja, anusāsa, mahārājā’ti. Rājā cakkavattī evamāha – ‘pāṇo na hantabbo, adinnaṃ nādātabbaṃ, kāmesumicchā na caritabbā, musā na bhāsitabbā, majjaṃ na pātabbaṃ, yathābhuttañca bhuñjathā’ti. Ye kho pana, bhikkhave, puratthimāya disāya paṭirājāno, te rañño cakkavattissa anuyantā\footnote{anuyuttā (sī. pī.)} ahesuṃ.

\paragraph{86.} ‘‘Atha kho taṃ, bhikkhave, cakkaratanaṃ puratthimaṃ samuddaṃ ajjhogāhetvā\footnote{ajjhogahetvā (sī. syā. pī.)} paccuttaritvā dakkhiṇaṃ disaṃ pavatti…pe… dakkhiṇaṃ samuddaṃ ajjhogāhetvā paccuttaritvā pacchimaṃ disaṃ pavatti, anvadeva rājā cakkavattī saddhiṃ caturaṅginiyā senāya. Yasmiṃ kho pana, bhikkhave, padese cakkaratanaṃ patiṭṭhāsi, tattha rājā cakkavattī vāsaṃ upagacchi saddhiṃ caturaṅginiyā senāya. Ye kho pana, bhikkhave, pacchimāya disāya paṭirājāno, te rājānaṃ cakkavattiṃ upasaṅkamitvā evamāhaṃsu – ‘ehi kho, mahārāja, svāgataṃ te, mahārāja, sakaṃ te, mahārāja, anusāsa, mahārājā’ti. Rājā cakkavattī evamāha – ‘pāṇo na hantabbo, adinnaṃ nādātabbaṃ, kāmesumicchā na caritabbā, musā na bhāsitabbā, majjaṃ na pātabbaṃ, yathābhuttañca bhuñjathā’ti. Ye kho pana, bhikkhave, pacchimāya disāya paṭirājāno, te rañño cakkavattissa anuyantā ahesuṃ.

\paragraph{87.} ‘‘Atha kho taṃ, bhikkhave, cakkaratanaṃ pacchimaṃ samuddaṃ ajjhogāhetvā paccuttaritvā uttaraṃ disaṃ pavatti, anvadeva rājā cakkavattī saddhiṃ caturaṅginiyā senāya. Yasmiṃ kho pana, bhikkhave, padese cakkaratanaṃ patiṭṭhāsi, tattha rājā cakkavattī vāsaṃ upagacchi saddhiṃ caturaṅginiyā senāya. Ye kho pana, bhikkhave, uttarāya disāya paṭirājāno, te rājānaṃ cakkavattiṃ upasaṅkamitvā evamāhaṃsu – ‘ehi kho, mahārāja, svāgataṃ te, mahārāja , sakaṃ te, mahārāja, anusāsa, mahārājā’ti. Rājā cakkavattī evamāha – ‘pāṇo na hantabbo, adinnaṃ nādātabbaṃ, kāmesumicchā na caritabbā, musā na bhāsitabbā, majjaṃ na pātabbaṃ, yathābhuttañca bhuñjathā’ti. Ye kho pana, bhikkhave, uttarāya disāya paṭirājāno, te rañño cakkavattissa anuyantā ahesuṃ.

‘‘Atha kho taṃ, bhikkhave, cakkaratanaṃ samuddapariyantaṃ pathaviṃ abhivijinitvā tameva rājadhāniṃ paccāgantvā rañño cakkavattissa antepuradvāre atthakaraṇapamukhe\footnote{aḍḍakaraṇapamukhe (ka.)} akkhāhataṃ maññe aṭṭhāsi rañño cakkavattissa antepuraṃ upasobhayamānaṃ.

\subsubsection{Dutiyādicakkavattikathā}

\paragraph{88.} ‘‘Dutiyopi kho, bhikkhave, rājā cakkavattī…pe… tatiyopi kho, bhikkhave, rājā cakkavattī… catutthopi kho, bhikkhave, rājā cakkavattī… pañcamopi kho, bhikkhave, rājā cakkavattī… chaṭṭhopi kho, bhikkhave, rājā cakkavattī… sattamopi kho, bhikkhave, rājā cakkavattī bahunnaṃ vassānaṃ bahunnaṃ vassasatānaṃ bahunnaṃ vassasahassānaṃ accayena aññataraṃ purisaṃ āmantesi – ‘yadā tvaṃ, ambho purisa, passeyyāsi dibbaṃ cakkaratanaṃ osakkitaṃ ṭhānā cutaṃ, atha me āroceyyāsī’ti. ‘Evaṃ, devā’ti kho, bhikkhave, so puriso rañño cakkavattissa paccassosi. Addasā kho , bhikkhave, so puriso bahunnaṃ vassānaṃ bahunnaṃ vassasatānaṃ bahunnaṃ vassasahassānaṃ accayena dibbaṃ cakkaratanaṃ osakkitaṃ ṭhānā cutaṃ. Disvāna yena rājā cakkavattī tenupasaṅkami; upasaṅkamitvā rājānaṃ cakkavattiṃ etadavoca – ‘yagghe , deva, jāneyyāsi, dibbaṃ te cakkaratanaṃ osakkitaṃ ṭhānā cuta’nti?

\paragraph{89.} ‘‘Atha kho, bhikkhave, rājā cakkavattī jeṭṭhaputtaṃ kumāraṃ āmantāpetvā etadavoca – ‘dibbaṃ kira me, tāta kumāra, cakkaratanaṃ osakkitaṃ, ṭhānā cutaṃ, sutaṃ kho pana metaṃ – yassa rañño cakkavattissa dibbaṃ cakkaratanaṃ osakkati, ṭhānā cavati, na dāni tena raññā ciraṃ jīvitabbaṃ hotīti. Bhuttā kho pana me mānusakā kāmā, samayo dāni me dibbe kāme pariyesituṃ, ehi tvaṃ, tāta kumāra, imaṃ samuddapariyantaṃ pathaviṃ paṭipajja . Ahaṃ pana kesamassuṃ ohāretvā kāsāyāni vatthāni acchādetvā agārasmā anagāriyaṃ pabbajissāmī’ti.

‘‘Atha kho, bhikkhave, rājā cakkavattī jeṭṭhaputtaṃ kumāraṃ sādhukaṃ rajje samanusāsitvā kesamassuṃ ohāretvā kāsāyāni vatthāni acchādetvā agārasmā anagāriyaṃ pabbaji. Sattāhapabbajite kho pana, bhikkhave, rājisimhi dibbaṃ cakkaratanaṃ antaradhāyi.

\paragraph{90.} ‘‘Atha kho, bhikkhave, aññataro puriso yena rājā khattiyo muddhābhisitto tenupasaṅkami; upasaṅkamitvā rājānaṃ khattiyaṃ muddhābhisittaṃ etadavoca – ‘yagghe, deva, jāneyyāsi, dibbaṃ cakkaratanaṃ antarahita’nti? Atha kho, bhikkhave, rājā khattiyo muddhābhisitto dibbe cakkaratane antarahite anattamano ahosi. Anattamanatañca paṭisaṃvedesi; no ca kho rājisiṃ upasaṅkamitvā ariyaṃ cakkavattivattaṃ pucchi. So samateneva sudaṃ janapadaṃ pasāsati. Tassa samatena janapadaṃ pasāsato pubbenāparaṃ janapadā na pabbanti, yathā taṃ pubbakānaṃ rājūnaṃ ariye cakkavattivatte vattamānānaṃ.

‘‘Atha kho, bhikkhave, amaccā pārisajjā gaṇakamahāmattā anīkaṭṭhā dovārikā mantassājīvino sannipatitvā rājānaṃ khattiyaṃ muddhābhisittaṃ etadavocuṃ – ‘na kho te, deva, samatena (sudaṃ) janapadaṃ pasāsato pubbenāparaṃ janapadā pabbanti, yathā taṃ pubbakānaṃ rājūnaṃ ariye cakkavattivatte vattamānānaṃ. Saṃvijjanti kho te, deva, vijite amaccā pārisajjā gaṇakamahāmattā anīkaṭṭhā dovārikā mantassājīvino mayañceva aññe ca\footnote{aññe ca paṇḍite samaṇabrāhmaṇe puccheyyāsi (ka.)} ye mayaṃ ariyaṃ cakkavattivattaṃ dhārema. Iṅgha tvaṃ, deva, amhe ariyaṃ cakkavattivattaṃ puccha. Tassa te mayaṃ ariyaṃ cakkavattivattaṃ puṭṭhā byākarissāmā’ti.

\subsubsection{Āyuvaṇṇādipariyānikathā}

\paragraph{91.} ‘‘Atha kho, bhikkhave, rājā khattiyo muddhābhisitto amacce pārisajje gaṇakamahāmatte anīkaṭṭhe dovārike mantassājīvino sannipātetvā ariyaṃ cakkavattivattaṃ pucchi. Tassa te ariyaṃ cakkavattivattaṃ puṭṭhā byākariṃsu. Tesaṃ sutvā dhammikañhi kho rakkhāvaraṇaguttiṃ saṃvidahi, no ca kho adhanānaṃ dhanamanuppadāsi. Adhanānaṃ dhane ananuppadiyamāne dāliddiyaṃ vepullamagamāsi. Dāliddiye vepullaṃ gate aññataro puriso paresaṃ adinnaṃ theyyasaṅkhātaṃ ādiyi. Tamenaṃ aggahesuṃ. Gahetvā rañño khattiyassa muddhābhisittassa dassesuṃ – ‘ayaṃ, deva, puriso paresaṃ adinnaṃ theyyasaṅkhātaṃ ādiyī’ti. Evaṃ vutte, bhikkhave, rājā khattiyo muddhābhisitto taṃ purisaṃ etadavoca – ‘saccaṃ kira tvaṃ, ambho purisa, paresaṃ adinnaṃ theyyasaṅkhātaṃ ādiyī’ti\footnote{ādiyasīti (syā.)}? ‘Saccaṃ, devā’ti. ‘Kiṃ kāraṇā’ti? ‘Na hi, deva, jīvāmī’ti. Atha kho, bhikkhave, rājā khattiyo muddhābhisitto tassa purisassa dhanamanuppadāsi – ‘iminā tvaṃ, ambho purisa, dhanena attanā ca jīvāhi, mātāpitaro ca posehi, puttadārañca posehi, kammante ca payojehi, samaṇabrāhmaṇesu\footnote{samaṇesu brāhmaṇesu (bahūsu)} uddhaggikaṃ dakkhiṇaṃ patiṭṭhāpehi sovaggikaṃ sukhavipākaṃ saggasaṃvattanika’nti. ‘Evaṃ, devā’ti kho, bhikkhave, so puriso rañño khattiyassa muddhābhisittassa paccassosi.

‘‘Aññataropi kho, bhikkhave, puriso paresaṃ adinnaṃ theyyasaṅkhātaṃ ādiyi. Tamenaṃ aggahesuṃ. Gahetvā rañño khattiyassa muddhābhisittassa dassesuṃ – ‘ayaṃ, deva, puriso paresaṃ adinnaṃ theyyasaṅkhātaṃ ādiyī’ti. Evaṃ vutte, bhikkhave, rājā khattiyo muddhābhisitto taṃ purisaṃ etadavoca – ‘saccaṃ kira tvaṃ, ambho purisa, paresaṃ adinnaṃ theyyasaṅkhātaṃ ādiyī’ti? ‘Saccaṃ, devā’ti. ‘Kiṃ kāraṇā’ti? ‘Na hi, deva, jīvāmī’ti. Atha kho, bhikkhave, rājā khattiyo muddhābhisitto tassa purisassa dhanamanuppadāsi – ‘iminā tvaṃ, ambho purisa, dhanena attanā ca jīvāhi, mātāpitaro ca posehi, puttadārañca posehi, kammante ca payojehi, samaṇabrāhmaṇesu uddhaggikaṃ dakkhiṇaṃ patiṭṭhāpehi sovaggikaṃ sukhavipākaṃ saggasaṃvattanika’nti. ‘Evaṃ, devā’ti kho, bhikkhave, so puriso rañño khattiyassa muddhābhisittassa paccassosi .

\paragraph{92.} ‘‘Assosuṃ kho, bhikkhave, manussā – ‘ye kira, bho, paresaṃ adinnaṃ theyyasaṅkhātaṃ ādiyanti, tesaṃ rājā dhanamanuppadetī’ti. Sutvāna tesaṃ etadahosi – ‘yaṃnūna mayampi paresaṃ adinnaṃ theyyasaṅkhātaṃ ādiyeyyāmā’ti. Atha kho, bhikkhave, aññataro puriso paresaṃ adinnaṃ theyyasaṅkhātaṃ ādiyi. Tamenaṃ aggahesuṃ. Gahetvā rañño khattiyassa muddhābhisittassa dassesuṃ – ‘ayaṃ, deva, puriso paresaṃ adinnaṃ theyyasaṅkhātaṃ ādiyī’ti. Evaṃ vutte, bhikkhave, rājā khattiyo muddhābhisitto taṃ purisaṃ etadavoca – ‘saccaṃ kira tvaṃ, ambho purisa, paresaṃ adinnaṃ theyyasaṅkhātaṃ ādiyī’ti? ‘Saccaṃ, devā’ti. ‘Kiṃ kāraṇā’ti? ‘Na hi, deva, jīvāmī’ti. Atha kho, bhikkhave, rañño khattiyassa muddhābhisittassa etadahosi – ‘sace kho ahaṃ yo yo paresaṃ adinnaṃ theyyasaṅkhātaṃ ādiyissati, tassa tassa dhanamanuppadassāmi, evamidaṃ adinnādānaṃ pavaḍḍhissati. Yaṃnūnāhaṃ imaṃ purisaṃ sunisedhaṃ nisedheyyaṃ, mūlaghaccaṃ\footnote{mūlaghacchaṃ (syā.), mūlachejja (ka.)} kareyyaṃ, sīsamassa chindeyya’nti. Atha kho, bhikkhave, rājā khattiyo muddhābhisitto purise āṇāpesi – ‘tena hi, bhaṇe, imaṃ purisaṃ daḷhāya rajjuyā pacchābāhaṃ\footnote{pacchābāhuṃ (syā.)} gāḷhabandhanaṃ bandhitvā khuramuṇḍaṃ karitvā kharassarena paṇavena rathikāya rathikaṃ siṅghāṭakena siṅghāṭakaṃ parinetvā dakkhiṇena dvārena nikkhamitvā dakkhiṇato nagarassa sunisedhaṃ nisedhetha, mūlaghaccaṃ karotha, sīsamassa chindathā’ti. ‘Evaṃ, devā’ti kho, bhikkhave, te purisā rañño khattiyassa muddhābhisittassa paṭissutvā taṃ purisaṃ daḷhāya rajjuyā pacchābāhaṃ gāḷhabandhanaṃ bandhitvā khuramuṇḍaṃ karitvā kharassarena paṇavena rathikāya rathikaṃ siṅghāṭakena siṅghāṭakaṃ parinetvā dakkhiṇena dvārena nikkhamitvā dakkhiṇato nagarassa sunisedhaṃ nisedhesuṃ, mūlaghaccaṃ akaṃsu, sīsamassa chindiṃsu.

\paragraph{93.} ‘‘Assosuṃ kho, bhikkhave, manussā – ‘ye kira, bho, paresaṃ adinnaṃ theyyasaṅkhātaṃ ādiyanti, te rājā sunisedhaṃ nisedheti, mūlaghaccaṃ karoti, sīsāni tesaṃ chindatī’ti. Sutvāna tesaṃ etadahosi – ‘yaṃnūna mayampi tiṇhāni satthāni kārāpessāma\footnote{kārāpeyyāma (syā. pī.) kārāpeyyāmāti (ka. sī.)}, tiṇhāni satthāni kārāpetvā yesaṃ adinnaṃ theyyasaṅkhātaṃ ādiyissāma, te sunisedhaṃ nisedhessāma, mūlaghaccaṃ karissāma, sīsāni tesaṃ chindissāmā’ti. Te tiṇhāni satthāni kārāpesuṃ, tiṇhāni satthāni kārāpetvā gāmaghātampi upakkamiṃsu kātuṃ, nigamaghātampi upakkamiṃsu kātuṃ, nagaraghātampi upakkamiṃsu kātuṃ, panthaduhanampi\footnote{panthadūhanaṃpi (sī. syā. pī.)} upakkamiṃsu kātuṃ. Yesaṃ te adinnaṃ theyyasaṅkhātaṃ ādiyanti, te sunisedhaṃ nisedhenti, mūlaghaccaṃ karonti, sīsāni tesaṃ chindanti.

\paragraph{94.} ‘‘Iti kho, bhikkhave, adhanānaṃ dhane ananuppadiyamāne dāliddiyaṃ vepullamagamāsi, dāliddiye vepullaṃ gate adinnādānaṃ vepullamagamāsi, adinnādāne vepullaṃ gate satthaṃ vepullamagamāsi, satthe vepullaṃ gate pāṇātipāto vepullamagamāsi, pāṇātipāte vepullaṃ gate tesaṃ sattānaṃ āyupi parihāyi, vaṇṇopi parihāyi. Tesaṃ āyunāpi parihāyamānānaṃ vaṇṇenapi parihāyamānānaṃ asītivassasahassāyukānaṃ manussānaṃ cattārīsavassasahassāyukā puttā ahesuṃ.

‘‘Cattārīsavassasahassāyukesu, bhikkhave, manussesu aññataro puriso paresaṃ adinnaṃ theyyasaṅkhātaṃ ādiyi. Tamenaṃ aggahesuṃ. Gahetvā rañño khattiyassa muddhābhisittassa dassesuṃ – ‘ayaṃ, deva, puriso paresaṃ adinnaṃ theyyasaṅkhātaṃ ādiyī’ti. Evaṃ vutte, bhikkhave, rājā khattiyo muddhābhisitto taṃ purisaṃ etadavoca – ‘saccaṃ kira tvaṃ, ambho purisa, paresaṃ adinnaṃ theyyasaṅkhātaṃ ādiyī’ti? ‘Na hi, devā’ti sampajānamusā abhāsi.

\paragraph{95.} ‘‘Iti kho, bhikkhave, adhanānaṃ dhane ananuppadiyamāne dāliddiyaṃ vepullamagamāsi. Dāliddiye vepullaṃ gate adinnādānaṃ vepullamagamāsi, adinnādāne vepullaṃ gate satthaṃ vepullamagamāsi. Satthe vepullaṃ gate pāṇātipāto vepullamagamāsi, pāṇātipāte vepullaṃ gate musāvādo vepullamagamāsi , musāvāde vepullaṃ gate tesaṃ sattānaṃ āyupi parihāyi, vaṇṇopi parihāyi. Tesaṃ āyunāpi parihāyamānānaṃ vaṇṇenapi parihāyamānānaṃ cattārīsavassasahassāyukānaṃ manussānaṃ vīsativassasahassāyukā puttā ahesuṃ.

‘‘Vīsativassasahassāyukesu , bhikkhave, manussesu aññataro puriso paresaṃ adinnaṃ theyyasaṅkhātaṃ ādiyi. Tamenaṃ aññataro puriso rañño khattiyassa muddhābhisittassa ārocesi – ‘itthannāmo, deva, puriso paresaṃ adinnaṃ theyyasaṅkhātaṃ ādiyī’ti pesuññamakāsi.

\paragraph{96.} ‘‘Iti kho, bhikkhave, adhanānaṃ dhane ananuppadiyamāne dāliddiyaṃ vepullamagamāsi. Dāliddiye vepullaṃ gate adinnādānaṃ vepullamagamāsi, adinnādāne vepullaṃ gate satthaṃ vepullamagamāsi, satthe vepullaṃ gate pāṇātipāto vepullamagamāsi, pāṇātipāte vepullaṃ gate musāvādo vepullamagamāsi, musāvāde vepullaṃ gate pisuṇā vācā vepullamagamāsi, pisuṇāya vācāya vepullaṃ gatāya tesaṃ sattānaṃ āyupi parihāyi, vaṇṇopi parihāyi. Tesaṃ āyunāpi parihāyamānānaṃ vaṇṇenapi parihāyamānānaṃ vīsativassasahassāyukānaṃ manussānaṃ dasavassasahassāyukā puttā ahesuṃ.

‘‘Dasavassasahassāyukesu, bhikkhave, manussesu ekidaṃ sattā vaṇṇavanto honti, ekidaṃ sattā dubbaṇṇā. Tattha ye te sattā dubbaṇṇā, te vaṇṇavante satte abhijjhāyantā paresaṃ dāresu cārittaṃ āpajjiṃsu.

\paragraph{97.} ‘‘Iti kho, bhikkhave, adhanānaṃ dhane ananuppadiyamāne dāliddiyaṃ vepullamagamāsi. Dāliddiye vepullaṃ gate…pe… kāmesumicchācāro vepullamagamāsi, kāmesumicchācāre vepullaṃ gate tesaṃ sattānaṃ āyupi parihāyi, vaṇṇopi parihāyi. Tesaṃ āyunāpi parihāyamānānaṃ vaṇṇenapi parihāyamānānaṃ dasavassasahassāyukānaṃ manussānaṃ pañcavassasahassāyukā puttā ahesuṃ.

\paragraph{98.} ‘‘Pañcavassasahassāyukesu, bhikkhave , manussesu dve dhammā vepullamagamaṃsu – pharusāvācā samphappalāpo ca. Dvīsu dhammesu vepullaṃ gatesu tesaṃ sattānaṃ āyupi parihāyi, vaṇṇopi parihāyi. Tesaṃ āyunāpi parihāyamānānaṃ vaṇṇenapi parihāyamānānaṃ pañcavassasahassāyukānaṃ manussānaṃ appekacce aḍḍhateyyavassasahassāyukā, appekacce dvevassasahassāyukā puttā ahesuṃ.

\paragraph{99.} ‘‘Aḍḍhateyyavassasahassāyukesu, bhikkhave, manussesu abhijjhābyāpādā vepullamagamaṃsu. Abhijjhābyāpādesu vepullaṃ gatesu tesaṃ sattānaṃ āyupi parihāyi, vaṇṇopi parihāyi. Tesaṃ āyunāpi parihāyamānānaṃ vaṇṇenapi parihāyamānānaṃ aḍḍhateyyavassasahassāyukānaṃ manussānaṃ vassasahassāyukā puttā ahesuṃ.

\paragraph{100.} ‘‘Vassasahassāyukesu, bhikkhave, manussesu micchādiṭṭhi vepullamagamāsi. Micchādiṭṭhiyā vepullaṃ gatāya tesaṃ sattānaṃ āyupi parihāyi, vaṇṇopi parihāyi. Tesaṃ āyunāpi parihāyamānānaṃ vaṇṇenapi parihāyamānānaṃ vassasahassāyukānaṃ manussānaṃ pañcavassasatāyukā puttā ahesuṃ.

\paragraph{101.} ‘‘Pañcavassasatāyukesu, bhikkhave, manussesu tayo dhammā vepullamagamaṃsu. Adhammarāgo visamalobho micchādhammo. Tīsu dhammesu vepullaṃ gatesu tesaṃ sattānaṃ āyupi parihāyi, vaṇṇopi parihāyi. Tesaṃ āyunāpi parihāyamānānaṃ vaṇṇenapi parihāyamānānaṃ pañcavassasatāyukānaṃ manussānaṃ appekacce aḍḍhateyyavassasatāyukā, appekacce dvevassasatāyukā puttā ahesuṃ.

‘‘Aḍḍhateyyavassasatāyukesu, bhikkhave , manussesu ime dhammā vepullamagamaṃsu. Amatteyyatā apetteyyatā asāmaññatā abrahmaññatā na kule jeṭṭhāpacāyitā.

\paragraph{102.} ‘‘Iti kho, bhikkhave, adhanānaṃ dhane ananuppadiyamāne dāliddiyaṃ vepullamagamāsi. Dāliddiye vepullaṃ gate adinnādānaṃ vepullamagamāsi. Adinnādāne vepullaṃ gate satthaṃ vepullamagamāsi. Satthe vepullaṃ gate pāṇātipāto vepullamagamāsi. Pāṇātipāte vepullaṃ gate musāvādo vepullamagamāsi. Musāvāde vepullaṃ gate pisuṇā vācā vepullamagamāsi. Pisuṇāya vācāya vepullaṃ gatāya kāmesumicchācāro vepullamagamāsi. Kāmesumicchācāre vepullaṃ gate dve dhammā vepullamagamaṃsu, pharusā vācā samphappalāpo ca. Dvīsu dhammesu vepullaṃ gatesu abhijjhābyāpādā vepullamagamaṃsu. Abhijjhābyāpādesu vepullaṃ gatesu micchādiṭṭhi vepullamagamāsi. Micchādiṭṭhiyā vepullaṃ gatāya tayo dhammā vepullamagamaṃsu, adhammarāgo visamalobho micchādhammo. Tīsu dhammesu vepullaṃ gatesu ime dhammā vepullamagamaṃsu, amatteyyatā apetteyyatā asāmaññatā abrahmaññatā na kule jeṭṭhāpacāyitā. Imesu dhammesu vepullaṃ gatesu tesaṃ sattānaṃ āyupi parihāyi, vaṇṇopi parihāyi. Tesaṃ āyunāpi parihāyamānānaṃ vaṇṇenapi parihāyamānānaṃ aḍḍhateyyavassasatāyukānaṃ manussānaṃ vassasatāyukā puttā ahesuṃ.

\subsubsection{Dasavassāyukasamayo}

\paragraph{103.} ‘‘Bhavissati , bhikkhave, so samayo, yaṃ imesaṃ manussānaṃ dasavassāyukā puttā bhavissanti. Dasavassāyukesu, bhikkhave, manussesu pañcavassikā\footnote{pañcamāsikā (ka. sī.)} kumārikā alaṃpateyyā bhavissanti. Dasavassāyukesu, bhikkhave, manussesu imāni rasāni antaradhāyissanti, seyyathidaṃ, sappi navanītaṃ telaṃ madhu phāṇitaṃ loṇaṃ. Dasavassāyukesu, bhikkhave, manussesu kudrūsako aggaṃ bhojanānaṃ\footnote{aggabhojanaṃ (syā.)} bhavissati. Seyyathāpi, bhikkhave, etarahi sālimaṃsodano aggaṃ bhojanānaṃ; evameva kho, bhikkhave, dasavassāyukesu manussesu kudrūsako aggaṃ bhojanānaṃ bhavissati.

‘‘Dasavassāyukesu, bhikkhave, manussesu dasa kusalakammapathā sabbena sabbaṃ antaradhāyissanti, dasa akusalakammapathā atibyādippissanti\footnote{ativiya dippissanti (syā. pī.), ativyādippissanti (sī.)}. Dasavassāyukesu , bhikkhave, manussesu kusalantipi na bhavissati, kuto pana kusalassa kārako. Dasavassāyukesu, bhikkhave, manussesu ye te bhavissanti amatteyyā apetteyyā asāmaññā abrahmaññā na kule jeṭṭhāpacāyino, te pujjā ca bhavissanti pāsaṃsā ca. Seyyathāpi, bhikkhave, etarahi matteyyā petteyyā sāmaññā brahmaññā kule jeṭṭhāpacāyino pujjā ca pāsaṃsā ca; evameva kho, bhikkhave, dasavassāyukesu manussesu ye te bhavissanti amatteyyā apetteyyā asāmaññā abrahmaññā na kule jeṭṭhāpacāyino, te pujjā ca bhavissanti pāsaṃsā ca.

‘‘Dasavassāyukesu , bhikkhave, manussesu na bhavissati mātāti vā mātucchāti vā mātulānīti vā ācariyabhariyāti vā garūnaṃ dārāti vā. Sambhedaṃ loko gamissati yathā ajeḷakā kukkuṭasūkarā soṇasiṅgālā\footnote{soṇasigālā (sī. pī.)}.

‘‘Dasavassāyukesu, bhikkhave, manussesu tesaṃ sattānaṃ aññamaññamhi tibbo āghāto paccupaṭṭhito bhavissati tibbo byāpādo tibbo manopadoso tibbaṃ vadhakacittaṃ. Mātupi puttamhi puttassapi mātari; pitupi puttamhi puttassapi pitari; bhātupi bhaginiyā bhaginiyāpi bhātari tibbo āghāto paccupaṭṭhito bhavissati tibbo byāpādo tibbo manopadoso tibbaṃ vadhakacittaṃ. Seyyathāpi, bhikkhave, māgavikassa migaṃ disvā tibbo āghāto paccupaṭṭhito hoti tibbo byāpādo tibbo manopadoso tibbaṃ vadhakacittaṃ; evameva kho, bhikkhave, dasavassāyukesu manussesu tesaṃ sattānaṃ aññamaññamhi tibbo āghāto paccupaṭṭhito bhavissati tibbo byāpādo tibbo manopadoso tibbaṃ vadhakacittaṃ. Mātupi puttamhi puttassapi mātari; pitupi puttamhi puttassapi pitari; bhātupi bhaginiyā bhaginiyāpi bhātari tibbo āghāto paccupaṭṭhito bhavissati tibbo byāpādo tibbo manopadoso tibbaṃ vadhakacittaṃ.

\paragraph{104.} ‘‘Dasavassāyukesu, bhikkhave, manussesu sattāhaṃ satthantarakappo bhavissati. Te aññamaññamhi migasaññaṃ paṭilabhissanti. Tesaṃ tiṇhāni satthāni hatthesu pātubhavissanti. Te tiṇhena satthena ‘esa migo esa migo’ti aññamaññaṃ jīvitā voropessanti.

‘‘Atha kho tesaṃ, bhikkhave, sattānaṃ ekaccānaṃ evaṃ bhavissati – ‘mā ca mayaṃ kañci\footnote{kiñci (ka.)}, mā ca amhe koci, yaṃnūna mayaṃ tiṇagahanaṃ vā vanagahanaṃ vā rukkhagahanaṃ vā nadīviduggaṃ vā pabbatavisamaṃ vā pavisitvā vanamūlaphalāhārā yāpeyyāmā’ti. Te tiṇagahanaṃ vā vanagahanaṃ vā rukkhagahanaṃ vā nadīviduggaṃ vā pabbatavisamaṃ vā\footnote{te tiṇagahanaṃ vanagahanaṃ rukkhagahanaṃ nadīviduggaṃ pabbatavisamaṃ (sī. pī.)} pavisitvā sattāhaṃ vanamūlaphalāhārā yāpessanti. Te tassa sattāhassa accayena tiṇagahanā vanagahanā rukkhagahanā nadīviduggā pabbatavisamā nikkhamitvā aññamaññaṃ āliṅgitvā sabhāgāyissanti samassāsissanti – ‘diṭṭhā, bho, sattā jīvasi, diṭṭhā, bho, sattā jīvasī’ti.

\subsubsection{Āyuvaṇṇādivaḍḍhanakathā}

\paragraph{105.} ‘‘Atha kho tesaṃ, bhikkhave, sattānaṃ evaṃ bhavissati – ‘mayaṃ kho akusalānaṃ dhammānaṃ samādānahetu evarūpaṃ āyataṃ ñātikkhayaṃ pattā. Yaṃnūna mayaṃ kusalaṃ kareyyāma. Kiṃ kusalaṃ kareyyāma? Yaṃnūna mayaṃ pāṇātipātā virameyyāma, idaṃ kusalaṃ dhammaṃ samādāya vatteyyāmā’ti. Te pāṇātipātā viramissanti, idaṃ kusalaṃ dhammaṃ samādāya vattissanti. Te kusalānaṃ dhammānaṃ samādānahetu āyunāpi vaḍḍhissanti, vaṇṇenapi vaḍḍhissanti . Tesaṃ āyunāpi vaḍḍhamānānaṃ vaṇṇenapi vaḍḍhamānānaṃ dasavassāyukānaṃ manussānaṃ vīsativassāyukā puttā bhavissanti.

‘‘Atha kho tesaṃ, bhikkhave, sattānaṃ evaṃ bhavissati – ‘mayaṃ kho kusalānaṃ dhammānaṃ samādānahetu āyunāpi vaḍḍhāma, vaṇṇenapi vaḍḍhāma. Yaṃnūna mayaṃ bhiyyosomattāya kusalaṃ kareyyāma. Kiṃ kusalaṃ kareyyāma? Yaṃnūna mayaṃ adinnādānā virameyyāma… kāmesumicchācārā virameyyāma… musāvādā virameyyāma… pisuṇāya vācāya virameyyāma… pharusāya vācāya virameyyāma… samphappalāpā virameyyāma… abhijjhaṃ pajaheyyāma… byāpādaṃ pajaheyyāma… micchādiṭṭhiṃ pajaheyyāma… tayo dhamme pajaheyyāma – adhammarāgaṃ visamalobhaṃ micchādhammaṃ… yaṃnūna mayaṃ matteyyā assāma petteyyā sāmaññā brahmaññā kule jeṭṭhāpacāyino, idaṃ kusalaṃ dhammaṃ samādāya vatteyyāmā’ti. Te matteyyā bhavissanti petteyyā sāmaññā brahmaññā kule jeṭṭhāpacāyino, idaṃ kusalaṃ dhammaṃ samādāya vattissanti.

‘‘Te kusalānaṃ dhammānaṃ samādānahetu āyunāpi vaḍḍhissanti, vaṇṇenapi vaḍḍhissanti. Tesaṃ āyunāpi vaḍḍhamānānaṃ vaṇṇenapi vaḍḍhamānānaṃ vīsativassāyukānaṃ manussānaṃ cattārīsavassāyukā puttā bhavissanti… cattārīsavassāyukānaṃ manussānaṃ asītivassāyukā puttā bhavissanti… asītivassāyukānaṃ manussānaṃ saṭṭhivassasatāyukā puttā bhavissanti… saṭṭhivassasatāyukānaṃ manussānaṃ vīsatitivassasatāyukā puttā bhavissanti… vīsatitivassasatāyukānaṃ manussānaṃ cattārīsachabbassasatāyukā puttā bhavissanti. Cattārīsachabbassasatāyukānaṃ manussānaṃ dvevassasahassāyukā puttā bhavissanti… dvevassasahassāyukānaṃ manussānaṃ cattārivassasahassāyukā puttā bhavissanti… cattārivassasahassāyukānaṃ manussānaṃ aṭṭhavassasahassāyukā puttā bhavissanti… aṭṭhavassasahassāyukānaṃ manussānaṃ vīsativassasahassāyukā puttā bhavissanti… vīsativassasahassāyukānaṃ manussānaṃ cattārīsavassasahassāyukā puttā bhavissanti… cattārīsavassasahassāyukānaṃ manussānaṃ asītivassasahassāyukā puttā bhavissanti… asītivassasahassāyukesu, bhikkhave, manussesu pañcavassasatikā kumārikā alaṃpateyyā bhavissanti.

\subsubsection{Saṅkharājauppatti}

\paragraph{106.} ‘‘Asītivassasahassāyukesu, bhikkhave, manussesu tayo ābādhā bhavissanti, icchā, anasanaṃ, jarā. Asītivassasahassāyukesu, bhikkhave, manussesu ayaṃ jambudīpo iddho ceva bhavissati phīto ca, kukkuṭasampātikā gāmanigamarājadhāniyo\footnote{gāmanigamajanapadā rājadhāniyo (ka.)}. Asītivassasahassāyukesu, bhikkhave, manussesu ayaṃ jambudīpo avīci maññe phuṭo bhavissati manussehi, seyyathāpi naḷavanaṃ vā saravanaṃ\footnote{sāravanaṃ (syā.)} vā. Asītivassasahassāyukesu, bhikkhave, manussesu ayaṃ bārāṇasī ketumatī nāma rājadhānī bhavissati iddhā ceva phītā ca bahujanā ca ākiṇṇamanussā ca subhikkhā ca. Asītivassasahassāyukesu, bhikkhave, manussesu imasmiṃ jambudīpe caturāsītinagarasahassāni bhavissanti ketumatīrājadhānīpamukhāni. Asītivassasahassāyukesu, bhikkhave, manussesu ketumatiyā rājadhāniyā saṅkho nāma rājā uppajjissati cakkavattī dhammiko dhammarājā cāturanto vijitāvī janapadatthāvariyappatto sattaratanasamannāgato. Tassimāni satta ratanāni bhavissanti, seyyathidaṃ, cakkaratanaṃ hatthiratanaṃ assaratanaṃ maṇiratanaṃ itthiratanaṃ gahapatiratanaṃ pariṇāyakaratanameva sattamaṃ. Parosahassaṃ kho panassa puttā bhavissanti sūrā vīraṅgarūpā parasenappamaddanā. So imaṃ pathaviṃ sāgarapariyantaṃ adaṇḍena asatthena dhammena abhivijiya ajjhāvasissati.

\subsubsection{Metteyyabuddhuppādo}

\paragraph{107.} ‘‘Asītivassasahassāyukesu, bhikkhave, manussesu metteyyo nāma bhagavā loke uppajjissati arahaṃ sammāsambuddho vijjācaraṇasampanno sugato lokavidū anuttaro purisadammasārathi satthā devamanussānaṃ buddho bhagavā. Seyyathāpāhametarahi loke uppanno arahaṃ sammāsambuddho vijjācaraṇasampanno sugato lokavidū anuttaro purisadammasārathi satthā devamanussānaṃ buddho bhagavā. So imaṃ lokaṃ sadevakaṃ samārakaṃ sabrahmakaṃ sassamaṇabrāhmaṇiṃ pajaṃ sadevamanussaṃ sayaṃ abhiññā sacchikatvā pavedessati, seyyathāpāhametarahi imaṃ lokaṃ sadevakaṃ samārakaṃ sabrahmakaṃ sassamaṇabrāhmaṇiṃ pajaṃ sadevamanussaṃ sayaṃ abhiññā sacchikatvā pavedemi. So dhammaṃ desessati ādikalyāṇaṃ majjhekalyāṇaṃ pariyosānakalyāṇaṃ sātthaṃ sabyañjanaṃ kevalaparipuṇṇaṃ parisuddhaṃ brahmacariyaṃ pakāsessati; seyyathāpāhametarahi dhammaṃ desemi ādikalyāṇaṃ majjhekalyāṇaṃ pariyosānakalyāṇaṃ sātthaṃ sabyañjanaṃ kevalaparipuṇṇaṃ parisuddhaṃ brahmacariyaṃ pakāsemi. So anekasahassaṃ\footnote{anekasatasahassaṃ (ka.)} bhikkhusaṃghaṃ pariharissati, seyyathāpāhametarahi anekasataṃ bhikkhusaṃghaṃ pariharāmi.

\paragraph{108.} ‘‘Atha kho, bhikkhave, saṅkho nāma rājā yo so yūpo raññā mahāpanādena kārāpito. Taṃ yūpaṃ ussāpetvā ajjhāvasitvā taṃ datvā vissajjitvā samaṇabrāhmaṇakapaṇaddhikavaṇibbakayācakānaṃ dānaṃ datvā metteyyassa bhagavato arahato sammāsambuddhassa santike kesamassuṃ ohāretvā kāsāyāni vatthāni acchādetvā agārasmā anagāriyaṃ pabbajissati. So evaṃ pabbajito samāno eko vūpakaṭṭho appamatto ātāpī pahitatto viharanto nacirasseva yassatthāya kulaputtā sammadeva agārasmā anagāriyaṃ pabbajanti, tadanuttaraṃ brahmacariyapariyosānaṃ diṭṭheva dhamme sayaṃ abhiññā sacchikatvā upasampajja viharissati.

\paragraph{109.} ‘‘Attadīpā, bhikkhave, viharatha attasaraṇā anaññasaraṇā, dhammadīpā dhammasaraṇā anaññasaraṇā. Kathañca, bhikkhave, bhikkhu attadīpo viharati attasaraṇo anaññasaraṇo dhammadīpo dhammasaraṇo anaññasaraṇo? Idha, bhikkhave, bhikkhu kāye kāyānupassī viharati ātāpī sampajāno satimā vineyya loke abhijjhādomanassaṃ. Vedanāsu vedanānupassī…pe… citte cittānupassī…pe… dhammesu dhammānupassī viharati ātāpī sampajāno satimā vineyya loke abhijjhādomanassaṃ. Evaṃ kho, bhikkhave, bhikkhu attadīpo viharati attasaraṇo anaññasaraṇo dhammadīpo dhammasaraṇo anaññasaraṇo.

\subsubsection{Bhikkhunoāyuvaṇṇādivaḍḍhanakathā}

\paragraph{110.} ‘‘Gocare, bhikkhave, caratha sake pettike visaye. Gocare, bhikkhave, carantā sake pettike visaye āyunāpi vaḍḍhissatha, vaṇṇenapi vaḍḍhissatha, sukhenapi vaḍḍhissatha, bhogenapi vaḍḍhissatha, balenapi vaḍḍhissatha.

‘‘Kiñca , bhikkhave, bhikkhuno āyusmiṃ? Idha, bhikkhave, bhikkhu chandasamādhipadhānasaṅkhārasamannāgataṃ iddhipādaṃ bhāveti, vīriyasamādhipadhānasaṅkhārasamannāgataṃ iddhipādaṃ bhāveti, cittasamādhipadhānasaṅkhārasamannāgataṃ iddhipādaṃ bhāveti, vīmaṃsāsamādhipadhānasaṅkhārasamannāgataṃ iddhipādaṃ bhāveti. So imesaṃ catunnaṃ iddhipādānaṃ bhāvitattā bahulīkatattā ākaṅkhamāno kappaṃ vā tiṭṭheyya kappāvasesaṃ vā. Idaṃ kho, bhikkhave, bhikkhuno āyusmiṃ.

‘‘Kiñca, bhikkhave, bhikkhuno vaṇṇasmiṃ? Idha, bhikkhave, bhikkhu sīlavā hoti, pātimokkhasaṃvarasaṃvuto viharati ācāragocarasampanno, aṇumattesu vajjesu bhayadassāvī, samādāya sikkhati sikkhāpadesu. Idaṃ kho, bhikkhave, bhikkhuno vaṇṇasmiṃ.

‘‘Kiñca, bhikkhave, bhikkhuno sukhasmiṃ? Idha, bhikkhave, bhikkhu vivicceva kāmehi vivicca akusalehi dhammehi savitakkaṃ savicāraṃ vivekajaṃ pītisukhaṃ paṭhamaṃ jhānaṃ upasampajja viharati. Vitakkavicārānaṃ vūpasamā…pe… dutiyaṃ jhānaṃ…pe… tatiyaṃ jhānaṃ…pe… catutthaṃ jhānaṃ upasampajja viharati. Idaṃ kho, bhikkhave, bhikkhuno, sukhasmiṃ.

‘‘Kiñca, bhikkhave, bhikkhuno bhogasmiṃ? Idha, bhikkhave, bhikkhu mettāsahagatena cetasā ekaṃ disaṃ pharitvā viharati tathā dutiyaṃ. Tathā tatiyaṃ. Tathā catutthaṃ. Iti uddhamadho tiriyaṃ sabbadhi sabbattatāya sabbāvantaṃ lokaṃ mettāsahagatena cetasā vipulena mahaggatena appamāṇena averena abyāpajjena pharitvā viharati. Karuṇāsahagatena cetasā…pe… muditāsahagatena cetasā…pe… upekkhāsahagatena cetasā ekaṃ disaṃ pharitvā viharati. Tathā dutiyaṃ. Tathā tatiyaṃ. Tathā catutthaṃ. Iti uddhamadho tiriyaṃ sabbadhi sabbattatāya sabbāvantaṃ lokaṃ upekkhāsahagatena cetasā vipulena mahaggatena appamāṇena averena abyāpajjena pharitvā viharati. Idaṃ kho, bhikkhave, bhikkhuno bhogasmiṃ.

‘‘Kiñca , bhikkhave, bhikkhuno balasmiṃ? Idha, bhikkhave, bhikkhu āsavānaṃ khayā anāsavaṃ cetovimuttiṃ paññāvimuttiṃ diṭṭheva dhamme sayaṃ abhiññā sacchikatvā upasampajja viharati. Idaṃ kho, bhikkhave, bhikkhuno balasmiṃ.

‘‘Nāhaṃ, bhikkhave, aññaṃ ekabalampi samanupassāmi yaṃ evaṃ duppasahaṃ, yathayidaṃ, bhikkhave, mārabalaṃ. Kusalānaṃ , bhikkhave, dhammānaṃ samādānahetu evamidaṃ puññaṃ pavaḍḍhatī’’ti. Idamavoca bhagavā. Attamanā te bhikkhū bhagavato bhāsitaṃ abhinandunti.

\xsectionEnd{Cakkavattisuttaṃ niṭṭhitaṃ tatiyaṃ.}


\section{Aggaññasuttaṃ}

\subsubsection{Vāseṭṭhabhāradvājā}

\paragraph{111.} Evaṃ me sutaṃ – ekaṃ samayaṃ bhagavā sāvatthiyaṃ viharati pubbārāme migāramātupāsāde. Tena kho pana samayena vāseṭṭhabhāradvājā bhikkhūsu parivasanti bhikkhubhāvaṃ ākaṅkhamānā. Atha kho bhagavā sāyanhasamayaṃ paṭisallānā vuṭṭhito pāsādā orohitvā pāsādapacchāyāyaṃ\footnote{pāsādacchāyāyaṃ (ka.)} abbhokāse caṅkamati.

\paragraph{112.} Addasā kho vāseṭṭho bhagavantaṃ sāyanhasamayaṃ paṭisallānā vuṭṭhitaṃ pāsādā orohitvā pāsādapacchāyāyaṃ abbhokāse caṅkamantaṃ. Disvāna bhāradvājaṃ āmantesi – ‘‘ayaṃ, āvuso bhāradvāja, bhagavā sāyanhasamayaṃ paṭisallānā vuṭṭhito pāsādā orohitvā pāsādapacchāyāyaṃ abbhokāse caṅkamati. Āyāmāvuso bhāradvāja, yena bhagavā tenupasaṅkamissāma; appeva nāma labheyyāma bhagavato santikā\footnote{sammukhā (syā. ka.)} dhammiṃ kathaṃ savanāyā’’ti. ‘‘Evamāvuso’’ti kho bhāradvājo vāseṭṭhassa paccassosi.

\paragraph{113.} Atha kho vāseṭṭhabhāradvājā yena bhagavā tenupasaṅkamiṃsu; upasaṅkamitvā bhagavantaṃ abhivādetvā bhagavantaṃ caṅkamantaṃ anucaṅkamiṃsu. Atha kho bhagavā vāseṭṭhaṃ āmantesi – ‘‘tumhe khvattha, vāseṭṭha, brāhmaṇajaccā brāhmaṇakulīnā brāhmaṇakulā agārasmā anagāriyaṃ pabbajitā, kacci vo, vāseṭṭha, brāhmaṇā na akkosanti na paribhāsantī’’ti? ‘‘Taggha no, bhante, brāhmaṇā akkosanti paribhāsanti attarūpāya paribhāsāya paripuṇṇāya, no aparipuṇṇāyā’’ti. ‘‘Yathā kathaṃ pana vo, vāseṭṭha, brāhmaṇā akkosanti paribhāsanti attarūpāya paribhāsāya paripuṇṇāya, no aparipuṇṇāyā’’ti? ‘‘Brāhmaṇā, bhante, evamāhaṃsu – ‘brāhmaṇova seṭṭho vaṇṇo, hīnā aññe vaṇṇā\footnote{hīno añño vaṇṇo (sī. pī. ma. ni. 2 madhurasutta)}. Brāhmaṇova sukko vaṇṇo , kaṇhā aññe vaṇṇā\footnote{kaṇho añño vaṇṇo (sī. pī. ma. ni. 2 madhurasutta)}. Brāhmaṇāva sujjhanti, no abrāhmaṇā. Brāhmaṇāva\footnote{brāhmaṇā (syā.)} brahmuno puttā orasā mukhato jātā brahmajā brahmanimmitā brahmadāyādā. Te tumhe seṭṭhaṃ vaṇṇaṃ hitvā hīnamattha vaṇṇaṃ ajjhupagatā, yadidaṃ muṇḍake samaṇake ibbhe kaṇhe bandhupādāpacce. Tayidaṃ na sādhu, tayidaṃ nappatirūpaṃ, yaṃ tumhe seṭṭhaṃ vaṇṇaṃ hitvā hīnamattha vaṇṇaṃ ajjhupagatā yadidaṃ muṇḍake samaṇake ibbhe kaṇhe bandhupādāpacce’ti. Evaṃ kho no, bhante, brāhmaṇā akkosanti paribhāsanti attarūpāya paribhāsāya paripuṇṇāya, no aparipuṇṇāyā’’ti.

\paragraph{114.} ‘‘Taggha vo, vāseṭṭha, brāhmaṇā porāṇaṃ assarantā evamāhaṃsu – ‘brāhmaṇova seṭṭho vaṇṇo, hīnā aññe vaṇṇā; brāhmaṇova sukko vaṇṇo, kaṇhā aññe vaṇṇā; brāhmaṇāva sujjhanti, no abrāhmaṇā; brāhmaṇāva brahmuno puttā orasā mukhato jātā brahmajā brahmanimmitā brahmadāyādā’ti. Dissanti kho pana, vāseṭṭha, brāhmaṇānaṃ brāhmaṇiyo utuniyopi gabbhiniyopi vijāyamānāpi pāyamānāpi. Te ca brāhmaṇā yonijāva samānā evamāhaṃsu – ‘brāhmaṇova seṭṭho vaṇṇo, hīnā aññe vaṇṇā; brāhmaṇova sukko vaṇṇo, kaṇhā aññe vaṇṇā; brāhmaṇāva sujjhanti, no abrāhmaṇā; brāhmaṇāva brahmuno puttā orasā mukhato jātā brahmajā brahmanimmitā brahmadāyādā’ti. Te\footnote{te ca (syā. ka.)} brahmānañceva abbhācikkhanti, musā ca bhāsanti, bahuñca apuññaṃ pasavanti.

\subsubsection{Catuvaṇṇasuddhi}

\paragraph{115.} ‘‘Cattārome, vāseṭṭha, vaṇṇā – khattiyā, brāhmaṇā, vessā, suddā. Khattiyopi kho, vāseṭṭha, idhekacco pāṇātipātī hoti adinnādāyī kāmesumicchācārī musāvādī pisuṇavāco pharusavāco samphappalāpī abhijjhālu byāpannacitto micchādiṭṭhī. Iti kho, vāseṭṭha, yeme dhammā akusalā akusalasaṅkhātā sāvajjā sāvajjasaṅkhātā asevitabbā asevitabbasaṅkhātā naalamariyā naalamariyasaṅkhātā kaṇhā kaṇhavipākā viññugarahitā, khattiyepi te\footnote{kho vāseṭṭha (ka.)} idhekacce sandissanti. Brāhmaṇopi kho, vāseṭṭha…pe… vessopi kho, vāseṭṭha…pe… suddopi kho, vāseṭṭha, idhekacco pāṇātipātī hoti adinnādāyī kāmesumicchācārī musāvādī pisuṇavāco pharusavāco samphappalāpī abhijjhālu byāpannacitto micchādiṭṭhī. Iti kho, vāseṭṭha, yeme dhammā akusalā akusalasaṅkhātā…pe… kaṇhā kaṇhavipākā viññugarahitā; suddepi te idhekacce sandissanti.

‘‘Khattiyopi kho, vāseṭṭha, idhekacco pāṇātipātā paṭivirato hoti, adinnādānā paṭivirato, kāmesumicchācārā paṭivirato, musāvādā paṭivirato, pisuṇāya vācāya paṭivirato, pharusāya vācāya paṭivirato, samphappalāpā paṭivirato, anabhijjhālu abyāpannacitto, sammādiṭṭhī. Iti kho, vāseṭṭha, yeme dhammā kusalā kusalasaṅkhātā anavajjā anavajjasaṅkhātā sevitabbā sevitabbasaṅkhātā alamariyā alamariyasaṅkhātā sukkā sukkavipākā viññuppasatthā, khattiyepi te idhekacce sandissanti. Brāhmaṇopi kho, vāseṭṭha…pe… vessopi kho, vāseṭṭha…pe… suddopi kho, vāseṭṭha, idhekacco pāṇātipātā paṭivirato hoti…pe… anabhijjhālu , abyāpannacitto, sammādiṭṭhī. Iti kho, vāseṭṭha, yeme dhammā kusalā kusalasaṅkhātā anavajjā anavajjasaṅkhātā sevitabbā sevitabbasaṅkhātā alamariyā alamariyasaṅkhātā sukkā sukkavipākā viññuppasatthā; suddepi te idhekacce sandissanti.

\paragraph{116.} ‘‘Imesu kho, vāseṭṭha, catūsu vaṇṇesu evaṃ ubhayavokiṇṇesu vattamānesu kaṇhasukkesu dhammesu viññugarahitesu ceva viññuppasatthesu ca yadettha brāhmaṇā evamāhaṃsu – ‘brāhmaṇova seṭṭho vaṇṇo, hīnā aññe vaṇṇā; brāhmaṇova sukko vaṇṇo, kaṇhā aññe vaṇṇā; brāhmaṇāva sujjhanti, no abrāhmaṇā; brāhmaṇāva brahmuno puttā orasā mukhato jātā brahmajā brahmanimmitā brahmadāyādā’ti. Taṃ tesaṃ viññū nānujānanti. Taṃ kissa hetu? Imesañhi, vāseṭṭha, catunnaṃ vaṇṇānaṃ yo hoti bhikkhu arahaṃ khīṇāsavo vusitavā katakaraṇīyo ohitabhāro anuppattasadattho parikkhīṇabhavasaṃyojano sammadaññāvimutto, so nesaṃ aggamakkhāyati dhammeneva, no adhammena. Dhammo hi, vāseṭṭha, seṭṭho janetasmiṃ, diṭṭhe ceva dhamme abhisamparāyañca .

\paragraph{117.} ‘‘Tadamināpetaṃ, vāseṭṭha, pariyāyena veditabbaṃ, yathā dhammova seṭṭho janetasmiṃ, diṭṭhe ceva dhamme abhisamparāyañca.

‘‘Jānāti kho\footnote{kho pana (ka.)}, vāseṭṭha, rājā pasenadi kosalo – ‘samaṇo gotamo anantarā\footnote{anuttaro (bahūsu)} sakyakulā pabbajito’ti. Sakyā kho pana, vāseṭṭha, rañño pasenadissa kosalassa anuyuttā\footnote{anantarā anuyantā (syā.), anantarā anuyuttā (ka.)} bhavanti. Karonti kho, vāseṭṭha, sakyā raññe pasenadimhi kosale nipaccakāraṃ abhivādanaṃ paccuṭṭhānaṃ añjalikammaṃ sāmīcikammaṃ. Iti kho, vāseṭṭha, yaṃ karonti sakyā raññe pasenadimhi kosale nipaccakāraṃ abhivādanaṃ paccuṭṭhānaṃ añjalikammaṃ sāmīcikammaṃ, karoti taṃ rājā pasenadi kosalo tathāgate nipaccakāraṃ abhivādanaṃ paccuṭṭhānaṃ añjalikammaṃ sāmīcikammaṃ, na naṃ\footnote{nanu (bahūsu)} ‘sujāto samaṇo gotamo, dujjātohamasmi. Balavā samaṇo gotamo, dubbalohamasmi. Pāsādiko samaṇo gotamo, dubbaṇṇohamasmi. Mahesakkho samaṇo gotamo, appesakkhohamasmī’ti. Atha kho naṃ dhammaṃyeva sakkaronto dhammaṃ garuṃ karonto dhammaṃ mānento dhammaṃ pūjento dhammaṃ apacāyamāno evaṃ rājā pasenadi kosalo tathāgate nipaccakāraṃ karoti, abhivādanaṃ paccuṭṭhānaṃ añjalikammaṃ sāmīcikammaṃ. Imināpi kho etaṃ, vāseṭṭha, pariyāyena veditabbaṃ, yathā dhammova seṭṭho janetasmiṃ, diṭṭhe ceva dhamme abhisamparāyañca.

\paragraph{118.} ‘‘Tumhe khvattha, vāseṭṭha, nānājaccā nānānāmā nānāgottā nānākulā agārasmā anagāriyaṃ pabbajitā. ‘Ke tumhe’ti – puṭṭhā samānā ‘samaṇā sakyaputtiyāmhā’ti – paṭijānātha. Yassa kho panassa, vāseṭṭha, tathāgate saddhā niviṭṭhā mūlajātā patiṭṭhitā daḷhā asaṃhāriyā samaṇena vā brāhmaṇena vā devena vā mārena vā brahmunā vā kenaci vā lokasmiṃ, tassetaṃ kallaṃ vacanāya – ‘bhagavatomhi putto oraso mukhato jāto dhammajo dhammanimmito dhammadāyādo’ti. Taṃ kissa hetu? Tathāgatassa hetaṃ, vāseṭṭha, adhivacanaṃ ‘dhammakāyo’ itipi, ‘brahmakāyo’ itipi, ‘dhammabhūto’ itipi, ‘brahmabhūto’ itipi.

\paragraph{119.} ‘‘Hoti kho so, vāseṭṭha, samayo yaṃ kadāci karahaci dīghassa addhuno accayena ayaṃ loko saṃvaṭṭati. Saṃvaṭṭamāne loke yebhuyyena sattā ābhassarasaṃvattanikā honti. Te tattha honti manomayā pītibhakkhā sayaṃpabhā antalikkhacarā subhaṭṭhāyino ciraṃ dīghamaddhānaṃ tiṭṭhanti.

‘‘Hoti kho so, vāseṭṭha, samayo yaṃ kadāci karahaci dīghassa addhuno accayena ayaṃ loko vivaṭṭati. Vivaṭṭamāne loke yebhuyyena sattā ābhassarakāyā cavitvā itthattaṃ āgacchanti. Tedha honti manomayā pītibhakkhā sayaṃpabhā antalikkhacarā subhaṭṭhāyino ciraṃ dīghamaddhānaṃ tiṭṭhanti.

\subsubsection{Rasapathavipātubhāvo}

\paragraph{120.} ‘‘Ekodakībhūtaṃ kho pana, vāseṭṭha, tena samayena hoti andhakāro andhakāratimisā . Na candimasūriyā paññāyanti, na nakkhattāni tārakarūpāni paññāyanti, na rattindivā paññāyanti, na māsaḍḍhamāsā paññāyanti, na utusaṃvaccharā paññāyanti , na itthipumā paññāyanti, sattā sattātveva saṅkhyaṃ gacchanti. Atha kho tesaṃ, vāseṭṭha, sattānaṃ kadāci karahaci dīghassa addhuno accayena rasapathavī udakasmiṃ samatani\footnote{samatāni (bahūsu)}; seyyathāpi nāma payaso tattassa\footnote{payatattassa (syā.)} nibbāyamānassa upari santānakaṃ hoti, evameva pāturahosi. Sā ahosi vaṇṇasampannā gandhasampannā rasasampannā, seyyathāpi nāma sampannaṃ vā sappi sampannaṃ vā navanītaṃ evaṃvaṇṇā ahosi. Seyyathāpi nāma khuddamadhuṃ\footnote{khuddaṃ madhuṃ (ka. sī.)} aneḷakaṃ\footnote{anelakaṃ (sī. pī.)}, evamassādā ahosi. Atha kho, vāseṭṭha, aññataro satto lolajātiko – ‘ambho, kimevidaṃ bhavissatī’ti rasapathaviṃ aṅguliyā sāyi. Tassa rasapathaviṃ aṅguliyā sāyato acchādesi, taṇhā cassa okkami. Aññepi kho, vāseṭṭha, sattā tassa sattassa diṭṭhānugatiṃ āpajjamānā rasapathaviṃ aṅguliyā sāyiṃsu. Tesaṃ rasapathaviṃ aṅguliyā sāyataṃ acchādesi, taṇhā ca tesaṃ okkami.

\subsubsection{Candimasūriyādipātubhāvo}

\paragraph{121.} ‘‘Atha kho te, vāseṭṭha, sattā rasapathaviṃ hatthehi āluppakārakaṃ upakkamiṃsu paribhuñjituṃ. Yato kho te\footnote{yato kho (sī. syā. pī.)}, vāseṭṭha, sattā rasapathaviṃ hatthehi āluppakārakaṃ upakkamiṃsu paribhuñjituṃ. Atha tesaṃ sattānaṃ sayaṃpabhā antaradhāyi. Sayaṃpabhāya antarahitāya candimasūriyā pāturahesuṃ. Candimasūriyesu pātubhūtesu nakkhattāni tārakarūpāni pāturahesuṃ. Nakkhattesu tārakarūpesu pātubhūtesu rattindivā paññāyiṃsu. Rattindivesu paññāyamānesu māsaḍḍhamāsā paññāyiṃsu. Māsaḍḍhamāsesu paññāyamānesu utusaṃvaccharā paññāyiṃsu. Ettāvatā kho , vāseṭṭha, ayaṃ loko puna vivaṭṭo hoti.

\paragraph{122.} ‘‘Atha kho te, vāseṭṭha, sattā rasapathaviṃ paribhuñjantā taṃbhakkhā\footnote{tabbhakkhā (syā.)} tadāhārā ciraṃ dīghamaddhānaṃ aṭṭhaṃsu. Yathā yathā kho te, vāseṭṭha, sattā rasapathaviṃ paribhuñjantā taṃbhakkhā tadāhārā ciraṃ dīghamaddhānaṃ aṭṭhaṃsu, tathā tathā tesaṃ sattānaṃ (rasapathaviṃ paribhuñjantānaṃ)\footnote{( ) sī. syā. pī. potthakesu natthi} kharattañceva kāyasmiṃ okkami, vaṇṇavevaṇṇatā\footnote{vaṇṇavevajjatā (ṭīkā)} ca paññāyittha. Ekidaṃ sattā vaṇṇavanto honti, ekidaṃ sattā dubbaṇṇā. Tattha ye te sattā vaṇṇavanto, te dubbaṇṇe satte atimaññanti – ‘mayametehi vaṇṇavantatarā, amhehete dubbaṇṇatarā’ti. Tesaṃ vaṇṇātimānapaccayā mānātimānajātikānaṃ rasapathavī antaradhāyi. Rasāya pathaviyā antarahitāya sannipatiṃsu. Sannipatitvā anutthuniṃsu – ‘aho rasaṃ, aho rasa’nti! Tadetarahipi manussā kañcideva surasaṃ\footnote{sādhurasaṃ (sī. syā. pī.)} labhitvā evamāhaṃsu – ‘aho rasaṃ, aho rasa’nti! Tadeva porāṇaṃ aggaññaṃ akkharaṃ anusaranti, na tvevassa atthaṃ ājānanti.

\subsubsection{Bhūmipappaṭakapātubhāvo}

\paragraph{123.} ‘‘Atha kho tesaṃ, vāseṭṭha, sattānaṃ rasāya pathaviyā antarahitāya bhūmipappaṭako pāturahosi. Seyyathāpi nāma ahicchattako, evameva pāturahosi. So ahosi vaṇṇasampanno gandhasampanno rasasampanno, seyyathāpi nāma sampannaṃ vā sappi sampannaṃ vā navanītaṃ evaṃvaṇṇo ahosi. Seyyathāpi nāma khuddamadhuṃ aneḷakaṃ, evamassādo ahosi.

‘‘Atha kho te, vāseṭṭha, sattā bhūmipappaṭakaṃ upakkamiṃsu paribhuñjituṃ. Te taṃ paribhuñjantā taṃbhakkhā tadāhārā ciraṃ dīghamaddhānaṃ aṭṭhaṃsu. Yathā yathā kho te, vāseṭṭha, sattā bhūmipappaṭakaṃ paribhuñjantā taṃbhakkhā tadāhārā ciraṃ dīghamaddhānaṃ aṭṭhaṃsu, tathā tathā tesaṃ sattānaṃ bhiyyoso mattāya kharattañceva kāyasmiṃ okkami, vaṇṇavevaṇṇatā ca paññāyittha. Ekidaṃ sattā vaṇṇavanto honti, ekidaṃ sattā dubbaṇṇā. Tattha ye te sattā vaṇṇavanto, te dubbaṇṇe satte atimaññanti – ‘mayametehi vaṇṇavantatarā, amhehete dubbaṇṇatarā’ti. Tesaṃ vaṇṇātimānapaccayā mānātimānajātikānaṃ bhūmipappaṭako antaradhāyi.

\subsubsection{Padālatāpātubhāvo}

\paragraph{124.} ‘‘Bhūmipappaṭake antarahite padālatā\footnote{saddālatā (sī.)} pāturahosi, seyyathāpi nāma kalambukā\footnote{kalambakā (syā.)}, evameva pāturahosi. Sā ahosi vaṇṇasampannā gandhasampannā rasasampannā, seyyathāpi nāma sampannaṃ vā sappi sampannaṃ vā navanītaṃ evaṃvaṇṇā ahosi. Seyyathāpi nāma khuddamadhuṃ aneḷakaṃ, evamassādā ahosi.

‘‘Atha kho te, vāseṭṭha, sattā padālataṃ upakkamiṃsu paribhuñjituṃ. Te taṃ paribhuñjantā taṃbhakkhā tadāhārā ciraṃ dīghamaddhānaṃ aṭṭhaṃsu. Yathā yathā kho te, vāseṭṭha, sattā padālataṃ paribhuñjantā taṃbhakkhā tadāhārā ciraṃ dīghamaddhānaṃ aṭṭhaṃsu, tathā tathā tesaṃ sattānaṃ bhiyyosomattāya kharattañceva kāyasmiṃ okkami, vaṇṇavevaṇṇatā ca paññāyittha. Ekidaṃ sattā vaṇṇavanto honti, ekidaṃ sattā dubbaṇṇā. Tattha ye te sattā vaṇṇavanto, te dubbaṇṇe satte atimaññanti – ‘mayametehi vaṇṇavantatarā, amhehete dubbaṇṇatarā’ti. Tesaṃ vaṇṇātimānapaccayā mānātimānajātikānaṃ padālatā antaradhāyi.

‘‘Padālatāya antarahitāya sannipatiṃsu. Sannipatitvā anutthuniṃsu – ‘ahu vata no, ahāyi vata no padālatā’ti! Tadetarahipi manussā kenaci\footnote{kenacideva (sī. syā. pī.)} dukkhadhammena phuṭṭhā evamāhaṃsu – ‘ahu vata no, ahāyi vata no’ti! Tadeva porāṇaṃ aggaññaṃ akkharaṃ anusaranti, na tvevassa atthaṃ ājānanti.

\subsubsection{Akaṭṭhapākasālipātubhāvo}

\paragraph{125.} ‘‘Atha kho tesaṃ, vāseṭṭha, sattānaṃ padālatāya antarahitāya akaṭṭhapāko sāli pāturahosi akaṇo athuso suddho sugandho taṇḍulapphalo. Yaṃ taṃ sāyaṃ sāyamāsāya āharanti, pāto taṃ hoti pakkaṃ paṭivirūḷhaṃ. Yaṃ taṃ pāto pātarāsāya āharanti, sāyaṃ taṃ hoti pakkaṃ paṭivirūḷhaṃ; nāpadānaṃ paññāyati. Atha kho te, vāseṭṭha, sattā akaṭṭhapākaṃ sāliṃ paribhuñjantā taṃbhakkhā tadāhārā ciraṃ dīghamaddhānaṃ aṭṭhaṃsu.

\subsubsection{Itthipurisaliṅgapātubhāvo}

\paragraph{126.} ‘‘Yathā yathā kho te, vāseṭṭha, sattā akaṭṭhapākaṃ sāliṃ paribhuñjantā taṃbhakkhā tadāhārā ciraṃ dīghamaddhānaṃ aṭṭhaṃsu, tathā tathā tesaṃ sattānaṃ bhiyyosomattāya kharattañceva kāyasmiṃ okkami, vaṇṇavevaṇṇatā ca paññāyittha, itthiyā ca itthiliṅgaṃ pāturahosi purisassa ca purisaliṅgaṃ. Itthī ca purisaṃ ativelaṃ upanijjhāyati puriso ca itthiṃ. Tesaṃ ativelaṃ aññamaññaṃ upanijjhāyataṃ sārāgo udapādi, pariḷāho kāyasmiṃ okkami. Te pariḷāhapaccayā methunaṃ dhammaṃ paṭiseviṃsu.

‘‘Ye kho pana te, vāseṭṭha, tena samayena sattā passanti methunaṃ dhammaṃ paṭisevante, aññe paṃsuṃ khipanti, aññe seṭṭhiṃ khipanti , aññe gomayaṃ khipanti – ‘nassa asuci\footnote{vasali (syā.), vasalī (ka.)}, nassa asucī’ti. ‘Kathañhi nāma satto sattassa evarūpaṃ karissatī’ti! Tadetarahipi manussā ekaccesu janapadesu vadhuyā nibbuyhamānāya\footnote{nivayhamānāya, niggayhamānāya (ka.)} aññe paṃsuṃ khipanti, aññe seṭṭhiṃ khipanti, aññe gomayaṃ khipanti. Tadeva porāṇaṃ aggaññaṃ akkharaṃ anusaranti, na tvevassa atthaṃ ājānanti.

\subsubsection{Methunadhammasamācāro}

\paragraph{127.} ‘‘Adhammasammataṃ kho pana\footnote{adhammasammataṃ taṃ kho pana (syā.), adhammasammataṃ kho pana taṃ (?)}, vāseṭṭha, tena samayena hoti, tadetarahi dhammasammataṃ. Ye kho pana, vāseṭṭha, tena samayena sattā methunaṃ dhammaṃ paṭisevanti, te māsampi dvemāsampi na labhanti gāmaṃ vā nigamaṃ vā pavisituṃ. Yato kho te, vāseṭṭha, sattā tasmiṃ asaddhamme ativelaṃ pātabyataṃ āpajjiṃsu. Atha agārāni upakkamiṃsu kātuṃ tasseva asaddhammassa paṭicchādanatthaṃ. Atha kho, vāseṭṭha, aññatarassa sattassa alasajātikassa etadahosi – ‘ambho, kimevāhaṃ vihaññāmi sāliṃ āharanto sāyaṃ sāyamāsāya pāto pātarāsāya! Yaṃnūnāhaṃ sāliṃ āhareyyaṃ sakiṃdeva\footnote{sakiṃdeva (ka.)} sāyapātarāsāyā’ti !

‘‘Atha kho so, vāseṭṭha, satto sāliṃ āhāsi sakiṃdeva sāyapātarāsāya. Atha kho, vāseṭṭha, aññataro satto yena so satto tenupasaṅkami; upasaṅkamitvā taṃ sattaṃ etadavoca – ‘ehi, bho satta, sālāhāraṃ gamissāmā’ti. ‘Alaṃ, bho satta, āhato\footnote{āhaṭo (syā.)} me sāli sakiṃdeva sāyapātarāsāyā’ti. Atha kho so, vāseṭṭha, satto tassa sattassa diṭṭhānugatiṃ āpajjamāno sāliṃ āhāsi sakiṃdeva dvīhāya. ‘Evampi kira, bho, sādhū’ti.

‘‘Atha kho, vāseṭṭha, aññataro satto yena so satto tenupasaṅkami; upasaṅkamitvā taṃ sattaṃ etadavoca – ‘ehi, bho satta, sālāhāraṃ gamissāmā’ti. ‘Alaṃ, bho satta, āhato me sāli sakiṃdeva dvīhāyā’ti. Atha kho so, vāseṭṭha, satto tassa sattassa diṭṭhānugatiṃ āpajjamāno sāliṃ āhāsi sakiṃdeva catūhāya, ‘evampi kira, bho, sādhū’ti.

‘‘Atha kho, vāseṭṭha, aññataro satto yena so satto tenupasaṅkami; upasaṅkamitvā taṃ sattaṃ etadavoca – ‘ehi, bho satta, sālāhāraṃ gamissāmā’ti. ‘Alaṃ, bho satta, āhato me sāli sakideva catūhāyā’ti. Atha kho so, vāseṭṭha, satto tassa sattassa diṭṭhānugatiṃ āpajjamāno sāliṃ āhāsi sakideva aṭṭhāhāya, ‘evampi kira, bho, sādhū’ti.

‘‘Yato kho te, vāseṭṭha, sattā sannidhikārakaṃ sāliṃ upakkamiṃsu paribhuñjituṃ. Atha kaṇopi taṇḍulaṃ pariyonandhi, thusopi taṇḍulaṃ pariyonandhi; lūnampi nappaṭivirūḷhaṃ , apadānaṃ paññāyittha, saṇḍasaṇḍā sālayo aṭṭhaṃsu.

\subsubsection{Sālivibhāgo}

\paragraph{128.} ‘‘Atha kho te, vāseṭṭha, sattā sannipatiṃsu, sannipatitvā anutthuniṃsu – ‘pāpakā vata, bho, dhammā sattesu pātubhūtā. Mayañhi pubbe manomayā ahumhā pītibhakkhā sayaṃpabhā antalikkhacarā subhaṭṭhāyino, ciraṃ dīghamaddhānaṃ aṭṭhamhā. Tesaṃ no amhākaṃ kadāci karahaci dīghassa addhuno accayena rasapathavī udakasmiṃ samatani. Sā ahosi vaṇṇasampannā gandhasampannā rasasampannā. Te mayaṃ rasapathaviṃ hatthehi āluppakārakaṃ upakkamimha paribhuñjituṃ, tesaṃ no rasapathaviṃ hatthehi āluppakārakaṃ upakkamataṃ paribhuñjituṃ sayaṃpabhā antaradhāyi. Sayaṃpabhāya antarahitāya candimasūriyā pāturahesuṃ, candimasūriyesu pātubhūtesu nakkhattāni tārakarūpāni pāturahesuṃ, nakkhattesu tārakarūpesu pātubhūtesu rattindivā paññāyiṃsu, rattindivesu paññāyamānesu māsaḍḍhamāsā paññāyiṃsu. Māsaḍḍhamāsesu paññāyamānesu utusaṃvaccharā paññāyiṃsu. Te mayaṃ rasapathaviṃ paribhuñjantā taṃbhakkhā tadāhārā ciraṃ dīghamaddhānaṃ aṭṭhamhā. Tesaṃ no pāpakānaṃyeva akusalānaṃ dhammānaṃ pātubhāvā rasapathavī antaradhāyi. Rasapathaviyā antarahitāya bhūmipappaṭako pāturahosi. So ahosi vaṇṇasampanno gandhasampanno rasasampanno. Te mayaṃ bhūmipappaṭakaṃ upakkamimha paribhuñjituṃ. Te mayaṃ taṃ paribhuñjantā taṃbhakkhā tadāhārā ciraṃ dīghamaddhānaṃ aṭṭhamhā. Tesaṃ no pāpakānaṃyeva akusalānaṃ dhammānaṃ pātubhāvā bhūmipappaṭako antaradhāyi. Bhūmipappaṭake antarahite padālatā pāturahosi. Sā ahosi vaṇṇasampannā gandhasampannā rasasampannā. Te mayaṃ padālataṃ upakkamimha paribhuñjituṃ. Te mayaṃ taṃ paribhuñjantā taṃbhakkhā tadāhārā ciraṃ dīghamaddhānaṃ aṭṭhamhā. Tesaṃ no pāpakānaṃyeva akusalānaṃ dhammānaṃ pātubhāvā padālatā antaradhāyi. Padālatāya antarahitāya akaṭṭhapāko sāli pāturahosi akaṇo athuso suddho sugandho taṇḍulapphalo. Yaṃ taṃ sāyaṃ sāyamāsāya āharāma, pāto taṃ hoti pakkaṃ paṭivirūḷhaṃ. Yaṃ taṃ pāto pātarāsāya āharāma, sāyaṃ taṃ hoti pakkaṃ paṭivirūḷhaṃ. Nāpadānaṃ paññāyittha. Te mayaṃ akaṭṭhapākaṃ sāliṃ paribhuñjantā taṃbhakkhā tadāhārā ciraṃ dīghamaddhānaṃ aṭṭhamhā. Tesaṃ no pāpakānaṃyeva akusalānaṃ dhammānaṃ pātubhāvā kaṇopi taṇḍulaṃ pariyonandhi, thusopi taṇḍulaṃ pariyonandhi, lūnampi nappaṭivirūḷhaṃ, apadānaṃ paññāyittha, saṇḍasaṇḍā sālayo ṭhitā. Yaṃnūna mayaṃ sāliṃ vibhajeyyāma, mariyādaṃ ṭhapeyyāmā’ti! Atha kho te, vāseṭṭha, sattā sāliṃ vibhajiṃsu, mariyādaṃ ṭhapesuṃ.

\paragraph{129.} ‘‘Atha kho, vāseṭṭha, aññataro satto lolajātiko sakaṃ bhāgaṃ parirakkhanto aññataraṃ\footnote{aññassa (?)} bhāgaṃ adinnaṃ ādiyitvā paribhuñji. Tamenaṃ aggahesuṃ, gahetvā etadavocuṃ – ‘pāpakaṃ vata, bho satta, karosi, yatra hi nāma sakaṃ bhāgaṃ parirakkhanto aññataraṃ bhāgaṃ adinnaṃ ādiyitvā paribhuñjasi. Māssu, bho satta, punapi evarūpamakāsī’ti. ‘Evaṃ, bho’ti kho, vāseṭṭha, so satto tesaṃ sattānaṃ paccassosi. Dutiyampi kho, vāseṭṭha, so satto…pe… tatiyampi kho, vāseṭṭha, so satto sakaṃ bhāgaṃ parirakkhanto aññataraṃ bhāgaṃ adinnaṃ ādiyitvā paribhuñji. Tamenaṃ aggahesuṃ, gahetvā etadavocuṃ – ‘pāpakaṃ vata, bho satta, karosi, yatra hi nāma sakaṃ bhāgaṃ parirakkhanto aññataraṃ bhāgaṃ adinnaṃ ādiyitvā paribhuñjasi. Māssu, bho satta, punapi evarūpamakāsī’ti. Aññe pāṇinā pahariṃsu, aññe leḍḍunā pahariṃsu, aññe daṇḍena pahariṃsu. Tadagge kho, vāseṭṭha, adinnādānaṃ paññāyati, garahā paññāyati, musāvādo paññāyati, daṇḍādānaṃ paññāyati.

\subsubsection{Mahāsammatarājā}

\paragraph{130.} ‘‘Atha kho te, vāseṭṭha, sattā sannipatiṃsu, sannipatitvā anutthuniṃsu – ‘pāpakā vata bho dhammā sattesu pātubhūtā, yatra hi nāma adinnādānaṃ paññāyissati, garahā paññāyissati, musāvādo paññāyissati, daṇḍādānaṃ paññāyissati. Yaṃnūna mayaṃ ekaṃ sattaṃ sammanneyyāma, yo no sammā khīyitabbaṃ khīyeyya, sammā garahitabbaṃ garaheyya, sammā pabbājetabbaṃ pabbājeyya. Mayaṃ panassa sālīnaṃ bhāgaṃ anuppadassāmā’ti.

‘‘Atha kho te, vāseṭṭha, sattā yo nesaṃ satto abhirūpataro ca dassanīyataro ca pāsādikataro ca mahesakkhataro ca taṃ sattaṃ upasaṅkamitvā etadavocuṃ – ‘ehi, bho satta, sammā khīyitabbaṃ khīya, sammā garahitabbaṃ garaha, sammā pabbājetabbaṃ pabbājehi. Mayaṃ pana te sālīnaṃ bhāgaṃ anuppadassāmā’ti. ‘Evaṃ, bho’ti kho, vāseṭṭha, so satto tesaṃ sattānaṃ paṭissuṇitvā sammā khīyitabbaṃ khīyi, sammā garahitabbaṃ garahi, sammā pabbājetabbaṃ pabbājesi. Te panassa sālīnaṃ bhāgaṃ anuppadaṃsu.

\paragraph{131.} ‘‘Mahājanasammatoti kho, vāseṭṭha, ‘mahāsammato, mahāsammato’ tveva paṭhamaṃ akkharaṃ upanibbattaṃ. Khettānaṃ adhipatīti kho, vāseṭṭha, ‘khattiyo, khattiyo’ tveva dutiyaṃ akkharaṃ upanibbattaṃ. Dhammena pare rañjetīti kho, vāseṭṭha, ‘rājā, rājā’ tveva tatiyaṃ akkharaṃ upanibbattaṃ. Iti kho, vāseṭṭha, evametassa khattiyamaṇḍalassa porāṇena aggaññena akkharena abhinibbatti ahosi tesaṃyeva sattānaṃ, anaññesaṃ. Sadisānaṃyeva, no asadisānaṃ. Dhammeneva, no adhammena. Dhammo hi, vāseṭṭha, seṭṭho janetasmiṃ diṭṭhe ceva dhamme abhisamparāyañca.

\subsubsection{Brāhmaṇamaṇḍalaṃ}

\paragraph{132.} ‘‘Atha kho tesaṃ, vāseṭṭha, sattānaṃyeva\footnote{tesaṃ yeva kho vāseṭṭha sattānaṃ (sī. pī.)} ekaccānaṃ etadahosi – ‘pāpakā vata, bho, dhammā sattesu pātubhūtā, yatra hi nāma adinnādānaṃ paññāyissati, garahā paññāyissati, musāvādo paññāyissati, daṇḍādānaṃ paññāyissati, pabbājanaṃ paññāyissati. Yaṃnūna mayaṃ pāpake akusale dhamme vāheyyāmā’ti. Te pāpake akusale dhamme vāhesuṃ . Pāpake akusale dhamme vāhentīti kho, vāseṭṭha, ‘brāhmaṇā, brāhmaṇā’ tveva paṭhamaṃ akkharaṃ upanibbattaṃ. Te araññāyatane paṇṇakuṭiyo karitvā paṇṇakuṭīsu jhāyanti vītaṅgārā vītadhūmā pannamusalā sāyaṃ sāyamāsāya pāto pātarāsāya gāmanigamarājadhāniyo osaranti ghāsamesamānā\footnote{ghāsamesanā (sī. syā. pī.)}. Te ghāsaṃ paṭilabhitvā punadeva araññāyatane paṇṇakuṭīsu jhāyanti. Tamenaṃ manussā disvā evamāhaṃsu – ‘ime kho, bho, sattā araññāyatane paṇṇakuṭiyo karitvā paṇṇakuṭīsu jhāyanti, vītaṅgārā vītadhūmā pannamusalā sāyaṃ sāyamāsāya pāto pātarāsāya gāmanigamarājadhāniyo osaranti ghāsamesamānā. Te ghāsaṃ paṭilabhitvā punadeva araññāyatane paṇṇakuṭīsu jhāyantī’ti, jhāyantīti kho\footnote{paṇṇakuṭīsu jhāyanti jhāyantīti kho (sī. pī.), paṇṇakuṭīsu jhāyantīti kho (ka.)}, vāseṭṭha, ‘jhāyakā, jhāyakā’ tveva dutiyaṃ akkharaṃ upanibbattaṃ. Tesaṃyeva kho, vāseṭṭha, sattānaṃ ekacce sattā araññāyatane paṇṇakuṭīsu taṃ jhānaṃ anabhisambhuṇamānā\footnote{anabhisaṃbhūnamānā (katthaci)} gāmasāmantaṃ nigamasāmantaṃ osaritvā ganthe karontā acchanti. Tamenaṃ manussā disvā evamāhaṃsu – ‘ime kho, bho, sattā araññāyatane paṇṇakuṭīsu taṃ jhānaṃ anabhisambhuṇamānā gāmasāmantaṃ nigamasāmantaṃ osaritvā ganthe karontā acchanti, na dānime jhāyantī’ti. Na dānime\footnote{na dānime jhāyantī na dānime (sī. pī. ka.)} jhāyantīti kho, vāseṭṭha, ‘ajjhāyakā ajjhāyakā’ tveva tatiyaṃ akkharaṃ upanibbattaṃ. Hīnasammataṃ kho pana, vāseṭṭha, tena samayena hoti, tadetarahi seṭṭhasammataṃ. Iti kho, vāseṭṭha, evametassa brāhmaṇamaṇḍalassa porāṇena aggaññena akkharena abhinibbatti ahosi tesaṃyeva sattānaṃ , anaññesaṃ sadisānaṃyeva no asadisānaṃ dhammeneva , no adhammena. Dhammo hi, vāseṭṭha, seṭṭho janetasmiṃ diṭṭhe ceva dhamme abhisamparāyañca.

\subsubsection{Vessamaṇḍalaṃ}

\paragraph{133.} ‘‘Tesaṃyeva kho, vāseṭṭha, sattānaṃ ekacce sattā methunaṃ dhammaṃ samādāya visukammante\footnote{vissutakammante (sī. pī.), vissukammante (ka. sī.), visuṃ kammante (syā. ka.)} payojesuṃ. Methunaṃ dhammaṃ samādāya visukammante payojentīti kho, vāseṭṭha, ‘vessā, vessā’ tveva akkharaṃ upanibbattaṃ. Iti kho, vāseṭṭha, evametassa vessamaṇḍalassa porāṇena aggaññena akkharena abhinibbatti ahosi tesaññeva sattānaṃ anaññesaṃ sadisānaṃyeva , no asadisānaṃ, dhammeneva no adhammena. Dhammo hi, vāseṭṭha, seṭṭho janetasmiṃ diṭṭhe ceva dhamme abhisamparāyañca.

\subsubsection{Suddamaṇḍalaṃ}

\paragraph{134.} ‘‘Tesaññeva kho, vāseṭṭha, sattānaṃ ye te sattā avasesā te luddācārā khuddācārā ahesuṃ. Luddācārā khuddācārāti kho, vāseṭṭha, ‘suddā, suddā’ tveva akkharaṃ upanibbattaṃ. Iti kho, vāseṭṭha, evametassa suddamaṇḍalassa porāṇena aggaññena akkharena abhinibbatti ahosi tesaṃyeva sattānaṃ anaññesaṃ, sadisānaṃyeva no asadisānaṃ, dhammeneva, no adhammena. Dhammo hi, vāseṭṭha, seṭṭho janetasmiṃ diṭṭhe ceva dhamme abhisamparāyañca.

\paragraph{135.} ‘‘Ahu kho so, vāseṭṭha, samayo, yaṃ khattiyopi sakaṃ dhammaṃ garahamāno agārasmā anagāriyaṃ pabbajati – ‘samaṇo bhavissāmī’ti. Brāhmaṇopi kho, vāseṭṭha…pe… vessopi kho, vāseṭṭha…pe… suddopi kho, vāseṭṭha, sakaṃ dhammaṃ garahamāno agārasmā anagāriyaṃ pabbajati – ‘samaṇo bhavissāmī’ti. Imehi kho, vāseṭṭha, catūhi maṇḍalehi samaṇamaṇḍalassa abhinibbatti ahosi, tesaṃyeva sattānaṃ anaññesaṃ, sadisānaṃyeva no asadisānaṃ, dhammeneva no adhammena. Dhammo hi, vāseṭṭha, seṭṭho janetasmiṃ diṭṭhe ceva dhamme abhisamparāyañca.

\subsubsection{Duccaritādikathā}

\paragraph{136.} ‘‘Khattiyopi kho, vāseṭṭha, kāyena duccaritaṃ caritvā vācāya duccaritaṃ caritvā manasā duccaritaṃ caritvā micchādiṭṭhiko micchādiṭṭhikammasamādāno\footnote{idaṃ padaṃ sī. ipotthakesu natthi} micchādiṭṭhikammasamādānahetu kāyassa bhedā paraṃ maraṇā apāyaṃ duggatiṃ vinipātaṃ nirayaṃ upapajjati. Brāhmaṇopi kho, vāseṭṭha…pe… vessopi kho, vāseṭṭha… suddopi kho, vāseṭṭha… samaṇopi kho, vāseṭṭha, kāyena duccaritaṃ caritvā vācāya duccaritaṃ caritvā manasā duccaritaṃ caritvā micchādiṭṭhiko micchādiṭṭhikammasamādāno micchādiṭṭhikammasamādānahetu kāyassa bhedā paraṃ maraṇā apāyaṃ duggatiṃ vinipātaṃ nirayaṃ upapajjati.

‘‘Khattiyopi kho, vāseṭṭha, kāyena sucaritaṃ caritvā vācāya sucaritaṃ caritvā manasā sucaritaṃ caritvā sammādiṭṭhiko sammādiṭṭhikammasamādāno\footnote{idaṃ padaṃ sī. pī. potthakesu natthi} sammādiṭṭhikammasamādānahetu kāyassa bhedā paraṃ maraṇā sugatiṃ saggaṃ lokaṃ upapajjati. Brāhmaṇopi kho, vāseṭṭha…pe… vessopi kho, vāseṭṭha… suddopi kho, vāseṭṭha… samaṇopi kho, vāseṭṭha, kāyena sucaritaṃ caritvā vācāya sucaritaṃ caritvā manasā sucaritaṃ caritvā sammādiṭṭhiko sammādiṭṭhikammasamādāno sammādiṭṭhikammasamādānahetu kāyassa bhedā paraṃ maraṇā sugatiṃ saggaṃ lokaṃ upapajjati.

\paragraph{137.} ‘‘Khattiyopi kho, vāseṭṭha, kāyena dvayakārī, vācāya dvayakārī, manasā dvayakārī, vimissadiṭṭhiko vimissadiṭṭhikammasamādāno vimissadiṭṭhikammasamādānahetu\footnote{vimissadiṭṭhiko vimissakammasamādāno vimissakammasamādānahetu (syā.), vītimissadiṭṭhiko vītimissadiṭṭhikammasamādānahetu (sī. pī.)} kāyassa bhedā paraṃ maraṇā sukhadukkhappaṭisaṃvedī hoti. Brāhmaṇopi kho, vāseṭṭha …pe… vessopi kho, vāseṭṭha… suddopi kho, vāseṭṭha… samaṇopi kho, vāseṭṭha, kāyena dvayakārī , vācāya dvayakārī, manasā dvayakārī, vimissadiṭṭhiko vimissadiṭṭhikammasamādāno vimissadiṭṭhikammasamādānahetu kāyassa bhedā paraṃ maraṇā sukhadukkhappaṭisaṃvedī hoti.

\subsubsection{Bodhipakkhiyabhāvanā}

\paragraph{138.} ‘‘Khattiyopi kho, vāseṭṭha, kāyena saṃvuto vācāya saṃvuto manasā saṃvuto sattannaṃ bodhipakkhiyānaṃ dhammānaṃ bhāvanamanvāya diṭṭheva dhamme parinibbāyati\footnote{parinibbāti (ka.)}. Brāhmaṇopi kho, vāseṭṭha…pe… vessopi kho vāseṭṭha… suddopi kho, vāseṭṭha … samaṇopi kho, vāseṭṭha, kāyena saṃvuto vācāya saṃvuto manasā saṃvuto sattannaṃ bodhipakkhiyānaṃ dhammānaṃ bhāvanamanvāya diṭṭheva dhamme parinibbāyati.

\paragraph{139.} ‘‘Imesañhi, vāseṭṭha, catunnaṃ vaṇṇānaṃ yo hoti bhikkhu arahaṃ khīṇāsavo vusitavā katakaraṇīyo ohitabhāro anuppattasadattho parikkhīṇabhavasaṃyojano sammadaññā vimutto so nesaṃ aggamakkhāyati dhammeneva. No adhammena. Dhammo hi, vāseṭṭha, seṭṭho janetasmiṃ diṭṭhe ceva dhamme abhisamparāyañca.

\paragraph{140.} ‘‘Brahmunā pesā, vāseṭṭha, sanaṅkumārena gāthā bhāsitā –

‘Khattiyo seṭṭho janetasmiṃ, ye gottapaṭisārino;

Vijjācaraṇasampanno, so seṭṭho devamānuse’ti.

‘‘Sā kho panesā, vāseṭṭha, brahmunā sanaṅkumārena gāthā sugītā, no duggītā. Subhāsitā, no dubbhāsitā. Atthasaṃhitā, no anatthasaṃhitā. Anumatā mayā. Ahampi, vāseṭṭha, evaṃ vadāmi –

‘Khattiyo seṭṭho janetasmiṃ, ye gottapaṭisārino;

Vijjācaraṇasampanno, so seṭṭho devamānuse’ti.

Idamavoca bhagavā. Attamanā vāseṭṭhabhāradvājā bhagavato bhāsitaṃ abhinandunti.

\xsectionEnd{Aggaññasuttaṃ niṭṭhitaṃ catutthaṃ.}


\section{Sampasādanīyasuttaṃ}

\subsubsection{Sāriputtasīhanādo}

\paragraph{141.} Evaṃ me sutaṃ – ekaṃ samayaṃ bhagavā nāḷandāyaṃ viharati pāvārikambavane. Atha kho āyasmā sāriputto yena bhagavā tenupasaṅkami; upasaṅkamitvā bhagavantaṃ abhivādetvā ekamantaṃ nisīdi. Ekamantaṃ nisinno kho āyasmā sāriputto bhagavantaṃ etadavoca – ‘‘evaṃpasanno ahaṃ, bhante, bhagavati, na cāhu na ca bhavissati na cetarahi vijjati añño samaṇo vā brāhmaṇo vā bhagavatā bhiyyobhiññataro yadidaṃ sambodhiya’’nti.

\paragraph{142.} ‘‘Uḷārā kho te ayaṃ, sāriputta, āsabhī vācā bhāsitā, ekaṃso gahito, sīhanādo nadito – ‘evaṃpasanno ahaṃ, bhante, bhagavati; na cāhu na ca bhavissati na cetarahi vijjati añño samaṇo vā brāhmaṇo vā bhagavatā bhiyyobhiññataro yadidaṃ sambodhiya’nti. Kiṃ te\footnote{kiṃ nu (sī. pī.), kiṃ nu kho te (syā.)}, sāriputta, ye te ahesuṃ atītamaddhānaṃ arahanto sammāsambuddhā, sabbe te bhagavanto cetasā ceto paricca viditā – ‘evaṃsīlā te bhagavanto ahesuṃ itipi, evaṃdhammā te bhagavanto ahesuṃ itipi , evaṃpaññā te bhagavanto ahesuṃ itipi, evaṃvihārī te bhagavanto ahesuṃ itipi, evaṃvimuttā te bhagavanto ahesuṃ itipī’’’ti? ‘‘No hetaṃ, bhante’’.

‘‘Kiṃ pana te\footnote{kiṃ pana (sī. pī.)}, sāriputta, ye te bhavissanti anāgatamaddhānaṃ arahanto sammāsambuddhā , sabbe te bhagavanto cetasā ceto paricca viditā, `evaṃsīlā te bhagavanto bhavissanti itipi, evaṃdhammā…pe… evaṃpaññā… evaṃvihārī… evaṃvimuttā te bhagavanto bhavissanti itipī’’’ti? ‘‘No hetaṃ, bhante’’.

‘‘Kiṃ pana te\footnote{kiṃ pana (sī. pī.)}, sāriputta, ahaṃ etarahi arahaṃ sammāsambuddho cetasā ceto paricca vidito – ‘evaṃsīlo bhagavā itipi, evaṃdhammo…pe… evaṃpañño … evaṃvihārī… evaṃvimutto bhagavā itipī’’’ti? ‘‘No hetaṃ, bhante’’.

‘‘Ettha ca hi te, sāriputta, atītānāgatapaccuppannesu arahantesu sammāsambuddhesu cetopariyañāṇaṃ natthi. Atha kiṃ carahi te ayaṃ, sāriputta, uḷārā āsabhī vācā bhāsitā, ekaṃso gahito, sīhanādo nadito – ‘evaṃpasanno ahaṃ, bhante, bhagavati, na cāhu na ca bhavissati na cetarahi vijjati añño samaṇo vā brāhmaṇo vā bhagavatā bhiyyobhiññataro yadidaṃ sambodhiya’’’nti?

\paragraph{143.} ‘‘Na kho me\footnote{na kho panetaṃ (syā. ka.)}, bhante, atītānāgatapaccuppannesu arahantesu sammāsambuddhesu cetopariyañāṇaṃ atthi. Api ca, me\footnote{me bhante (sī. pī. ka.)} dhammanvayo vidito. Seyyathāpi, bhante , rañño paccantimaṃ nagaraṃ daḷhuddhāpaṃ\footnote{daḷhuddāpaṃ (sī. pī. ka.)} daḷhapākāratoraṇaṃ ekadvāraṃ. Tatrassa dovāriko paṇḍito byatto medhāvī aññātānaṃ nivāretā, ñātānaṃ pavesetā. So tassa nagarassa samantā anupariyāyapathaṃ anukkamamāno na passeyya pākārasandhiṃ vā pākāravivaraṃ vā antamaso biḷāranikkhamanamattampi. Tassa evamassa – ‘ye kho keci oḷārikā pāṇā imaṃ nagaraṃ pavisanti vā nikkhamanti vā, sabbe te imināva dvārena pavisanti vā nikkhamanti vā’ti. Evameva kho me, bhante, dhammanvayo vidito. Ye te, bhante, ahesuṃ atītamaddhānaṃ arahanto sammāsambuddhā, sabbe te bhagavanto pañca nīvaraṇe pahāya cetaso upakkilese paññāya dubbalīkaraṇe catūsu satipaṭṭhānesu suppatiṭṭhitacittā, satta sambojjhaṅge yathābhūtaṃ bhāvetvā anuttaraṃ sammāsambodhiṃ abhisambujjhiṃsu. Yepi te, bhante, bhavissanti anāgatamaddhānaṃ arahanto sammāsambuddhā, sabbe te bhagavanto pañca nīvaraṇe pahāya cetaso upakkilese paññāya dubbalīkaraṇe catūsu satipaṭṭhānesu suppatiṭṭhitacittā, satta sambojjhaṅge yathābhūtaṃ bhāvetvā anuttaraṃ sammāsambodhiṃ abhisambujjhissanti. Bhagavāpi, bhante, etarahi arahaṃ sammāsambuddho pañca nīvaraṇe pahāya cetaso upakkilese paññāya dubbalīkaraṇe catūsu satipaṭṭhānesu suppatiṭṭhitacitto satta sambojjhaṅge yathābhūtaṃ bhāvetvā anuttaraṃ sammāsambodhiṃ abhisambuddho.

\paragraph{144.} ‘‘Idhāhaṃ, bhante, yena bhagavā tenupasaṅkamiṃ dhammassavanāya. Tassa me, bhante, bhagavā dhammaṃ deseti uttaruttaraṃ paṇītapaṇītaṃ kaṇhasukkasappaṭibhāgaṃ. Yathā yathā me, bhante, bhagavā dhammaṃ desesi uttaruttaraṃ paṇītapaṇītaṃ kaṇhasukkasappaṭibhāgaṃ, tathā tathāhaṃ tasmiṃ dhamme abhiññā idhekaccaṃ dhammaṃ dhammesu niṭṭhamagamaṃ; satthari pasīdiṃ – ‘sammāsambuddho bhagavā, svākkhāto bhagavatā dhammo, suppaṭipanno sāvakasaṅgho’ti.

\subsubsection{Kusaladhammadesanā}

\paragraph{145.} ‘‘Aparaṃ pana, bhante, etadānuttariyaṃ, yathā bhagavā dhammaṃ deseti kusalesu dhammesu. Tatrime kusalā dhammā seyyathidaṃ, cattāro satipaṭṭhānā, cattāro sammappadhānā, cattāro iddhipādā, pañcindriyāni, pañca balāni, satta bojjhaṅgā, ariyo aṭṭhaṅgiko maggo. Idha, bhante, bhikkhu āsavānaṃ khayā anāsavaṃ cetovimuttiṃ paññāvimuttiṃ diṭṭheva dhamme sayaṃ abhiññā sacchikatvā upasampajja viharati. Etadānuttariyaṃ, bhante, kusalesu dhammesu. Taṃ bhagavā asesamabhijānāti, taṃ bhagavato asesamabhijānato uttari abhiññeyyaṃ natthi, yadabhijānaṃ añño samaṇo vā brāhmaṇo vā bhagavatā bhiyyobhiññataro assa, yadidaṃ kusalesu dhammesu.

\subsubsection{Āyatanapaṇṇattidesanā}

\paragraph{146.} ‘‘Aparaṃ pana, bhante, etadānuttariyaṃ, yathā bhagavā dhammaṃ deseti āyatanapaṇṇattīsu. Chayimāni, bhante, ajjhattikabāhirāni āyatanāni. Cakkhuñceva rūpā\footnote{rūpāni (ka.)} ca, sotañceva saddā ca, ghānañceva gandhā ca, jivhā ceva rasā ca, kāyo ceva phoṭṭhabbā ca, mano ceva dhammā ca. Etadānuttariyaṃ, bhante, āyatanapaṇṇattīsu. Taṃ bhagavā asesamabhijānāti, taṃ bhagavato asesamabhijānato uttari abhiññeyyaṃ natthi, yadabhijānaṃ añño samaṇo vā brāhmaṇo vā bhagavatā bhiyyobhiññataro assa yadidaṃ āyatanapaṇṇattīsu.

\subsubsection{Gabbhāvakkantidesanā}

\paragraph{147.} ‘‘Aparaṃ pana, bhante, etadānuttariyaṃ, yathā bhagavā dhammaṃ deseti gabbhāvakkantīsu. Catasso imā, bhante, gabbhāvakkantiyo. Idha , bhante, ekacco asampajāno mātukucchiṃ okkamati; asampajāno mātukucchismiṃ ṭhāti; asampajāno mātukucchimhā nikkhamati. Ayaṃ paṭhamā gabbhāvakkanti.

‘‘Puna caparaṃ, bhante, idhekacco sampajāno mātukucchiṃ okkamati; asampajāno mātukucchismiṃ ṭhāti; asampajāno mātukucchimhā nikkhamati. Ayaṃ dutiyā gabbhāvakkanti.

‘‘Puna caparaṃ, bhante, idhekacco sampajāno mātukucchiṃ okkamati; sampajāno mātukucchismiṃ ṭhāti; asampajāno mātukucchimhā nikkhamati. Ayaṃ tatiyā gabbhāvakkanti.

‘‘Puna caparaṃ, bhante, idhekacco sampajāno mātukucchiṃ okkamati; sampajāno mātukucchismiṃ ṭhāti; sampajāno mātukucchimhā nikkhamati. Ayaṃ catutthā gabbhāvakkanti. Etadānuttariyaṃ, bhante, gabbhāvakkantīsu.

\subsubsection{Ādesanavidhādesanā}

\paragraph{148.} ‘‘Aparaṃ pana, bhante, etadānuttariyaṃ, yathā bhagavā dhammaṃ deseti ādesanavidhāsu. Catasso imā, bhante, ādesanavidhā. Idha, bhante, ekacco nimittena ādisati – ‘evampi te mano, itthampi te mano, itipi te citta’nti. So bahuṃ cepi ādisati, tatheva taṃ hoti, no aññathā. Ayaṃ paṭhamā ādesanavidhā.

‘‘Puna caparaṃ, bhante, idhekacco na heva kho nimittena ādisati. Api ca kho manussānaṃ vā amanussānaṃ vā devatānaṃ vā saddaṃ sutvā ādisati – ‘evampi te mano, itthampi te mano, itipi te citta’nti. So bahuṃ cepi ādisati, tatheva taṃ hoti, no aññathā. Ayaṃ dutiyā ādesanavidhā.

‘‘Puna caparaṃ, bhante, idhekacco na heva kho nimittena ādisati, nāpi manussānaṃ vā amanussānaṃ vā devatānaṃ vā saddaṃ sutvā ādisati. Api ca kho vitakkayato vicārayato vitakkavipphārasaddaṃ sutvā ādisati – ‘evampi te mano, itthampi te mano, itipi te citta’nti. So bahuṃ cepi ādisati, tatheva taṃ hoti, no aññathā. Ayaṃ tatiyā ādesanavidhā.

‘‘Puna caparaṃ, bhante, idhekacco na heva kho nimittena ādisati, nāpi manussānaṃ vā amanussānaṃ vā devatānaṃ vā saddaṃ sutvā ādisati, nāpi vitakkayato vicārayato vitakkavipphārasaddaṃ sutvā ādisati. Api ca kho avitakkaṃ avicāraṃ samādhiṃ samāpannassa\footnote{vitakkavicārasamādhisamāpannassa (syā. ka.) a. ni. 3.61 passitabbaṃ} cetasā ceto paricca pajānāti – ‘yathā imassa bhoto manosaṅkhārā paṇihitā. Tathā imassa cittassa anantarā imaṃ nāma vitakkaṃ vitakkessatī’ti. So bahuṃ cepi ādisati, tatheva taṃ hoti, no aññathā. Ayaṃ catutthā ādesanavidhā. Etadānuttariyaṃ, bhante, ādesanavidhāsu.

\subsubsection{Dassanasamāpattidesanā}

\paragraph{149.} ‘‘Aparaṃ pana, bhante, etadānuttariyaṃ, yathā bhagavā dhammaṃ deseti dassanasamāpattīsu. Catasso imā, bhante, dassanasamāpattiyo. Idha, bhante, ekacco samaṇo vā brāhmaṇo vā ātappamanvāya padhānamanvāya anuyogamanvāya appamādamanvāya sammāmanasikāramanvāya tathārūpaṃ cetosamādhiṃ phusati, yathāsamāhite citte imameva kāyaṃ uddhaṃ pādatalā adho kesamatthakā tacapariyantaṃ pūraṃ nānappakārassa asucino paccavekkhati – ‘atthi imasmiṃ kāye kesā lomā nakhā dantā taco maṃsaṃ nhāru aṭṭhi aṭṭhimiñjaṃ vakkaṃ hadayaṃ yakanaṃ kilomakaṃ pihakaṃ papphāsaṃ antaṃ antaguṇaṃ udariyaṃ karīsaṃ pittaṃ semhaṃ pubbo lohitaṃ sedo medo assu vasā kheḷo siṅghānikā lasikā mutta’nti. Ayaṃ paṭhamā dassanasamāpatti.

‘‘Puna caparaṃ , bhante, idhekacco samaṇo vā brāhmaṇo vā ātappamanvāya…pe… tathārūpaṃ cetosamādhiṃ phusati, yathāsamāhite citte imameva kāyaṃ uddhaṃ pādatalā adho kesamatthakā tacapariyantaṃ pūraṃ nānappakārassa asucino paccavekkhati – ‘atthi imasmiṃ kāye kesā lomā…pe… lasikā mutta’nti. Atikkamma ca purisassa chavimaṃsalohitaṃ aṭṭhiṃ paccavekkhati. Ayaṃ dutiyā dassanasamāpatti.

‘‘Puna caparaṃ, bhante, idhekacco samaṇo vā brāhmaṇo vā ātappamanvāya…pe… tathārūpaṃ cetosamādhiṃ phusati, yathāsamāhite citte imameva kāyaṃ uddhaṃ pādatalā adho kesamatthakā tacapariyantaṃ pūraṃ nānappakārassa asucino paccavekkhati – ‘atthi imasmiṃ kāye kesā lomā…pe… lasikā mutta’nti. Atikkamma ca purisassa chavimaṃsalohitaṃ aṭṭhiṃ paccavekkhati. Purisassa ca viññāṇasotaṃ pajānāti, ubhayato abbocchinnaṃ idha loke patiṭṭhitañca paraloke patiṭṭhitañca. Ayaṃ tatiyā dassanasamāpatti.

‘‘Puna caparaṃ, bhante, idhekacco samaṇo vā brāhmaṇo vā ātappamanvāya…pe… tathārūpaṃ cetosamādhiṃ phusati, yathāsamāhite citte imameva kāyaṃ uddhaṃ pādatalā adho kesamatthakā tacapariyantaṃ pūraṃ nānappakārassa asucino paccavekkhati – ‘atthi imasmiṃ kāye kesā lomā…pe… lasikā mutta’nti. Atikkamma ca purisassa chavimaṃsalohitaṃ aṭṭhiṃ paccavekkhati. Purisassa ca viññāṇasotaṃ pajānāti, ubhayato abbocchinnaṃ idha loke appatiṭṭhitañca paraloke appatiṭṭhitañca. Ayaṃ catutthā dassanasamāpatti. Etadānuttariyaṃ, bhante, dassanasamāpattīsu.

\subsubsection{Puggalapaṇṇattidesanā}

\paragraph{150.} ‘‘Aparaṃ pana, bhante, etadānuttariyaṃ, yathā bhagavā dhammaṃ deseti puggalapaṇṇattīsu. Sattime, bhante, puggalā. Ubhatobhāgavimutto paññāvimutto kāyasakkhī diṭṭhippatto saddhāvimutto dhammānusārī saddhānusārī. Etadānuttariyaṃ, bhante, puggalapaṇṇattīsu.

\subsubsection{Padhānadesanā}

\paragraph{151.} ‘‘Aparaṃ pana, bhante, etadānuttariyaṃ, yathā bhagavā dhammaṃ deseti padhānesu. Sattime, bhante sambojjhaṅgā satisambojjhaṅgo dhammavicayasambojjhaṅgo vīriyasambojjhaṅgo pītisambojjhaṅgo passaddhisambojjhaṅgo samādhisambojjhaṅgo upekkhāsambojjhaṅgo. Etadānuttariyaṃ, bhante, padhānesu.

\subsubsection{Paṭipadādesanā}

\paragraph{152.} ‘‘Aparaṃ pana, bhante, etadānuttariyaṃ, yathā bhagavā dhammaṃ deseti paṭipadāsu. Catasso imā, bhante, paṭipadā dukkhā paṭipadā dandhābhiññā, dukkhā paṭipadā khippābhiññā, sukhā paṭipadā dandhābhiññā, sukhā paṭipadā khippābhiññāti. Tatra, bhante, yāyaṃ paṭipadā dukkhā dandhābhiññā, ayaṃ, bhante, paṭipadā ubhayeneva hīnā akkhāyati dukkhattā ca dandhattā ca. Tatra, bhante, yāyaṃ paṭipadā dukkhā khippābhiññā, ayaṃ pana, bhante, paṭipadā dukkhattā hīnā akkhāyati . Tatra, bhante, yāyaṃ paṭipadā sukhā dandhābhiññā, ayaṃ pana, bhante, paṭipadā dandhattā hīnā akkhāyati. Tatra, bhante, yāyaṃ paṭipadā sukhā khippābhiññā, ayaṃ pana, bhante, paṭipadā ubhayeneva paṇītā akkhāyati sukhattā ca khippattā ca. Etadānuttariyaṃ, bhante, paṭipadāsu.

\subsubsection{Bhassasamācārādidesanā}

\paragraph{153.} ‘‘Aparaṃ pana, bhante, etadānuttariyaṃ, yathā bhagavā dhammaṃ deseti bhassasamācāre. Idha, bhante, ekacco na ceva musāvādupasañhitaṃ vācaṃ bhāsati na ca vebhūtiyaṃ na ca pesuṇiyaṃ na ca sārambhajaṃ jayāpekkho; mantā mantā ca vācaṃ bhāsati nidhānavatiṃ kālena. Etadānuttariyaṃ, bhante, bhassasamācāre.

‘‘Aparaṃ pana, bhante, etadānuttariyaṃ, yathā bhagavā dhammaṃ deseti purisasīlasamācāre. Idha, bhante, ekacco sacco cassa saddho ca, na ca kuhako, na ca lapako, na ca nemittiko, na ca nippesiko, na ca lābhena lābhaṃ nijigīsanako\footnote{jijigiṃsanako (syā.), nijigiṃsitā (sī. pī.)}, indriyesu guttadvāro, bhojane mattaññū, samakārī, jāgariyānuyogamanuyutto, atandito, āraddhavīriyo, jhāyī, satimā, kalyāṇapaṭibhāno, gatimā, dhitimā, matimā, na ca kāmesu giddho, sato ca nipako ca. Etadānuttariyaṃ, bhante, purisasīlasamācāre.

\subsubsection{Anusāsanavidhādesanā}

\paragraph{154.} ‘‘Aparaṃ pana, bhante, etadānuttariyaṃ, yathā bhagavā dhammaṃ deseti anusāsanavidhāsu. Catasso imā bhante anusāsanavidhā – jānāti , bhante, bhagavā aparaṃ puggalaṃ paccattaṃ yonisomanasikārā ‘ayaṃ puggalo yathānusiṭṭhaṃ tathā paṭipajjamāno tiṇṇaṃ saṃyojanānaṃ parikkhayā sotāpanno bhavissati avinipātadhammo niyato sambodhiparāyaṇo’ti. Jānāti, bhante, bhagavā paraṃ puggalaṃ paccattaṃ yonisomanasikārā – ‘ayaṃ puggalo yathānusiṭṭhaṃ tathā paṭipajjamāno tiṇṇaṃ saṃyojanānaṃ parikkhayā rāgadosamohānaṃ tanuttā sakadāgāmī bhavissati, sakideva imaṃ lokaṃ āgantvā dukkhassantaṃ karissatī’ti. Jānāti, bhante, bhagavā paraṃ puggalaṃ paccattaṃ yonisomanasikārā – ‘ayaṃ puggalo yathānusiṭṭhaṃ tathā paṭipajjamāno pañcannaṃ orambhāgiyānaṃ saṃyojanānaṃ parikkhayā opapātiko bhavissati tattha parinibbāyī anāvattidhammo tasmā lokā’ti. Jānāti, bhante, bhagavā paraṃ puggalaṃ paccattaṃ yonisomanasikārā – ‘ayaṃ puggalo yathānusiṭṭhaṃ tathā paṭipajjamāno āsavānaṃ khayā anāsavaṃ cetovimuttiṃ paññāvimuttiṃ diṭṭheva dhamme sayaṃ abhiññā sacchikatvā upasampajja viharissatī’ti. Etadānuttariyaṃ, bhante, anusāsanavidhāsu.

\subsubsection{Parapuggalavimuttiñāṇadesanā}

\paragraph{155.} ‘‘Aparaṃ pana, bhante, etadānuttariyaṃ, yathā bhagavā dhammaṃ deseti parapuggalavimuttiñāṇe. Jānāti, bhante, bhagavā paraṃ puggalaṃ paccattaṃ yonisomanasikārā – ‘ayaṃ puggalo tiṇṇaṃ saṃyojanānaṃ parikkhayā sotāpanno bhavissati avinipātadhammo niyato sambodhiparāyaṇo’ti. Jānāti, bhante, bhagavā paraṃ puggalaṃ paccattaṃ yonisomanasikārā – ‘ayaṃ puggalo tiṇṇaṃ saṃyojanānaṃ parikkhayā rāgadosamohānaṃ tanuttā sakadāgāmī bhavissati, sakideva imaṃ lokaṃ āgantvā dukkhassantaṃ karissatī’ti. Jānāti, bhante, bhagavā paraṃ puggalaṃ paccattaṃ yonisomanasikārā – ‘ayaṃ puggalo pañcannaṃ orambhāgiyānaṃ saṃyojanānaṃ parikkhayā opapātiko bhavissati tattha parinibbāyī anāvattidhammo tasmā lokā’ti. Jānāti, bhante, bhagavā paraṃ puggalaṃ paccattaṃ yonisomanasikārā – ‘ayaṃ puggalo āsavānaṃ khayā anāsavaṃ cetovimuttiṃ paññāvimuttiṃ diṭṭheva dhamme sayaṃ abhiññā sacchikatvā upasampajja viharissatī’ti. Etadānuttariyaṃ, bhante, parapuggalavimuttiñāṇe.

\subsubsection{Sassatavādadesanā}

\paragraph{156.} ‘‘Aparaṃ pana, bhante, etadānuttariyaṃ, yathā bhagavā dhammaṃ deseti sassatavādesu. Tayome, bhante, sassatavādā. Idha, bhante, ekacco samaṇo vā brāhmaṇo vā ātappamanvāya…pe… tathārūpaṃ cetosamādhiṃ phusati, yathāsamāhite citte anekavihitaṃ pubbenivāsaṃ anussarati. Seyyathidaṃ, ekampi jātiṃ dvepi jātiyo tissopi jātiyo catassopi jātiyo pañcapi jātiyo dasapi jātiyo vīsampi jātiyo tiṃsampi jātiyo cattālīsampi jātiyo paññāsampi jātiyo jātisatampi jātisahassampi jātisatasahassampi anekānipi jātisatāni anekānipi jātisahassāni anekānipi jātisatasahassāni, ‘amutrāsiṃ evaṃnāmo evaṃgotto evaṃvaṇṇo evamāhāro evaṃsukhadukkhappaṭisaṃvedī evamāyupariyanto, so tato cuto amutra udapādiṃ; tatrāpāsiṃ evaṃnāmo evaṃgotto evaṃvaṇṇo evamāhāro evaṃsukhadukkhappaṭisaṃvedī evamāyupariyanto, so tato cuto idhūpapanno’ti. Iti sākāraṃ sauddesaṃ anekavihitaṃ pubbenivāsaṃ anussarati. So evamāha – ‘atītaṃpāhaṃ addhānaṃ jānāmi – saṃvaṭṭi vā loko vivaṭṭi vāti. Anāgataṃpāhaṃ addhānaṃ jānāmi – saṃvaṭṭissati vā loko vivaṭṭissati vāti. Sassato attā ca loko ca vañjho kūṭaṭṭho esikaṭṭhāyiṭṭhito. Te ca sattā sandhāvanti saṃsaranti cavanti upapajjanti, atthitveva sassatisama’nti. Ayaṃ paṭhamo sassatavādo.

‘‘Puna caparaṃ, bhante, idhekacco samaṇo vā brāhmaṇo vā ātappamanvāya…pe… tathārūpaṃ cetosamādhiṃ phusati, yathāsamāhite citte anekavihitaṃ pubbenivāsaṃ anussarati. Seyyathidaṃ, ekampi saṃvaṭṭavivaṭṭaṃ dvepi saṃvaṭṭavivaṭṭāni tīṇipi saṃvaṭṭavivaṭṭāni cattāripi saṃvaṭṭavivaṭṭāni pañcapi saṃvaṭṭavivaṭṭāni dasapi saṃvaṭṭavivaṭṭāni, ‘amutrāsiṃ evaṃnāmo evaṃgotto evaṃvaṇṇo evamāhāro evaṃsukhadukkhappaṭisaṃvedī evamāyupariyanto, so tato cuto amutra udapādiṃ; tatrāpāsiṃ evaṃnāmo evaṃgotto evaṃvaṇṇo evamāhāro evaṃsukhadukkhappaṭisaṃvedī evamāyupariyanto, so tato cuto idhūpapanno’ti. Iti sākāraṃ sauddesaṃ anekavihitaṃ pubbenivāsaṃ anussarati. So evamāha – ‘atītaṃpāhaṃ addhānaṃ jānāmi saṃvaṭṭi vā loko vivaṭṭi vāti . Anāgataṃpāhaṃ addhānaṃ jānāmi saṃvaṭṭissati vā loko vivaṭṭissati vāti. Sassato attā ca loko ca vañjho kūṭaṭṭho esikaṭṭhāyiṭṭhito. Te ca sattā sandhāvanti saṃsaranti cavanti upapajjanti, atthitveva sassatisama’nti. Ayaṃ dutiyo sassatavādo.

‘‘Puna caparaṃ, bhante, idhekacco samaṇo vā brāhmaṇo vā ātappamanvāya…pe… tathārūpaṃ cetosamādhiṃ phusati, yathāsamāhite citte anekavihitaṃ pubbenivāsaṃ anussarati. Seyyathidaṃ, dasapi saṃvaṭṭavivaṭṭāni vīsampi saṃvaṭṭavivaṭṭāni tiṃsampi saṃvaṭṭavivaṭṭāni cattālīsampi saṃvaṭṭavivaṭṭāni, ‘amutrāsiṃ evaṃnāmo evaṃgotto evaṃvaṇṇo evamāhāro evaṃsukhadukkhappaṭisaṃvedī evamāyupariyanto, so tato cuto amutra udapādiṃ; tatrāpāsiṃ evaṃnāmo evaṃgotto evaṃvaṇṇo evamāhāro evaṃsukhadukkhappaṭisaṃvedī evamāyupariyanto, so tato cuto idhūpapanno’ti. Iti sākāraṃ sauddesaṃ anekavihitaṃ pubbenivāsaṃ anussarati. So evamāha – ‘atītaṃpāhaṃ addhānaṃ jānāmi saṃvaṭṭipi loko vivaṭṭipīti; anāgataṃpāhaṃ addhānaṃ jānāmi saṃvaṭṭissatipi loko vivaṭṭissatipīti. Sassato attā ca loko ca vañjho kūṭaṭṭho esikaṭṭhāyiṭṭhito. Te ca sattā sandhāvanti saṃsaranti cavanti upapajjanti, atthitveva sassatisama’nti. Ayaṃ tatiyo sassatavādo, etadānuttariyaṃ, bhante, sassatavādesu.

\subsubsection{Pubbenivāsānussatiñāṇadesanā}

\paragraph{157.} ‘‘Aparaṃ pana, bhante, etadānuttariyaṃ, yathā bhagavā dhammaṃ deseti pubbenivāsānussatiñāṇe. Idha, bhante, ekacco samaṇo vā brāhmaṇo vā ātappamanvāya…pe… tathārūpaṃ cetosamādhiṃ phusati, yathāsamāhite citte anekavihitaṃ pubbenivāsaṃ anussarati. Seyyathidaṃ, ekampi jātiṃ dvepi jātiyo tissopi jātiyo catassopi jātiyo pañcapi jātiyo dasapi jātiyo vīsampi jātiyo tiṃsampi jātiyo cattālīsampi jātiyo paññāsampi jātiyo jātisatampi jātisahassampi jātisatasahassampi anekepi saṃvaṭṭakappe anekepi vivaṭṭakappe anekepi saṃvaṭṭavivaṭṭakappe, ‘amutrāsiṃ evaṃnāmo evaṃgotto evaṃvaṇṇo evamāhāro evaṃsukhadukkhappaṭisaṃvedī evamāyupariyanto, so tato cuto amutra udapādiṃ; tatrāpāsiṃ evaṃnāmo evaṃgotto evaṃvaṇṇo evamāhāro evaṃsukhadukkhappaṭisaṃvedī evamāyupariyanto, so tato cuto idhūpapanno’ti. Iti sākāraṃ sauddesaṃ anekavihitaṃ pubbenivāsaṃ anussarati. Santi, bhante, devā\footnote{sattā (syā.)}, yesaṃ na sakkā gaṇanāya vā saṅkhānena vā āyu saṅkhātuṃ. Api ca, yasmiṃ yasmiṃ attabhāve abhinivuṭṭhapubbo\footnote{abhinivutthapubbo (sī. syā. pī.)} hoti yadi vā rūpīsu yadi vā arūpīsu yadi vā saññīsu yadi vā asaññīsu yadi vā nevasaññīnāsaññīsu. Iti sākāraṃ sauddesaṃ anekavihitaṃ pubbenivāsaṃ anussarati. Etadānuttariyaṃ, bhante, pubbenivāsānussatiñāṇe.

\subsubsection{Cutūpapātañāṇadesanā}

\paragraph{158.} ‘‘Aparaṃ pana, bhante, etadānuttariyaṃ, yathā bhagavā dhammaṃ deseti sattānaṃ cutūpapātañāṇe. Idha, bhante, ekacco samaṇo vā brāhmaṇo vā ātappamanvāya…pe… tathārūpaṃ cetosamādhiṃ phusati, yathāsamāhite citte dibbena cakkhunā visuddhena atikkantamānusakena satte passati cavamāne upapajjamāne hīne paṇīte suvaṇṇe dubbaṇṇe sugate duggate yathākammūpage satte pajānāti – ‘ime vata bhonto sattā kāyaduccaritena samannāgatā vacīduccaritena samannāgatā manoduccaritena samannāgatā ariyānaṃ upavādakā micchādiṭṭhikā micchādiṭṭhikammasamādānā. Te kāyassa bhedā paraṃ maraṇā apāyaṃ duggatiṃ vinipātaṃ nirayaṃ upapannā. Ime vā pana bhonto sattā kāyasucaritena samannāgatā vacīsucaritena samannāgatā manosucaritena samannāgatā ariyānaṃ anupavādakā sammādiṭṭhikā sammādiṭṭhikammasamādānā. Te kāyassa bhedā paraṃ maraṇā sugatiṃ saggaṃ lokaṃ upapannā’ti. Iti dibbena cakkhunā visuddhena atikkantamānusakena satte passati cavamāne upapajjamāne hīne paṇīte suvaṇṇe dubbaṇṇe sugate duggate yathākammūpage satte pajānāti. Etadānuttariyaṃ, bhante, sattānaṃ cutūpapātañāṇe.

\subsubsection{Iddhividhadesanā}

\paragraph{159.} ‘‘Aparaṃ pana, bhante, etadānuttariyaṃ, yathā bhagavā dhammaṃ deseti iddhividhāsu. Dvemā, bhante, iddhividhāyo – atthi, bhante, iddhi sāsavā saupadhikā, ‘no ariyā’ti vuccati. Atthi, bhante, iddhi anāsavā anupadhikā ‘ariyā’ti vuccati. ‘‘Katamā ca, bhante, iddhi sāsavā saupadhikā, ‘no ariyā’ti vuccati? Idha, bhante, ekacco samaṇo vā brāhmaṇo vā ātappamanvāya…pe… tathārūpaṃ cetosamādhiṃ phusati, yathāsamāhite citte anekavihitaṃ iddhividhaṃ paccanubhoti. Ekopi hutvā bahudhā hoti, bahudhāpi hutvā eko hoti; āvibhāvaṃ tirobhāvaṃ tirokuṭṭaṃ tiropākāraṃ tiropabbataṃ asajjamāno gacchati seyyathāpi ākāse. Pathaviyāpi ummujjanimujjaṃ karoti, seyyathāpi udake. Udakepi abhijjamāne gacchati, seyyathāpi pathaviyaṃ. Ākāsepi pallaṅkena kamati, seyyathāpi pakkhī sakuṇo. Imepi candimasūriye evaṃmahiddhike evaṃmahānubhāve pāṇinā parāmasati parimajjati. Yāva brahmalokāpi kāyena vasaṃ vatteti. Ayaṃ, bhante, iddhi sāsavā saupadhikā, ‘no ariyā’ti vuccati.

‘‘Katamā pana, bhante, iddhi anāsavā anupadhikā, ‘ariyā’ti vuccati? Idha, bhante, bhikkhu sace ākaṅkhati – ‘paṭikūle appaṭikūlasaññī vihareyya’nti, appaṭikūlasaññī tattha viharati. Sace ākaṅkhati – ‘appaṭikūle paṭikūlasaññī vihareyya’nti, paṭikūlasaññī tattha viharati. Sace ākaṅkhati – ‘paṭikūle ca appaṭikūle ca appaṭikūlasaññī vihareyya’nti, appaṭikūlasaññī tattha viharati. Sace ākaṅkhati – ‘paṭikūle ca appaṭikūle ca paṭikūlasaññī vihareyya’nti, paṭikūlasaññī tattha viharati. Sace ākaṅkhati – ‘paṭikūlañca appaṭikūlañca tadubhayaṃ abhinivajjetvā upekkhako vihareyyaṃ sato sampajāno’ti, upekkhako tattha viharati sato sampajāno. Ayaṃ, bhante, iddhi anāsavā anupadhikā ‘ariyā’ti vuccati. Etadānuttariyaṃ, bhante, iddhividhāsu . Taṃ bhagavā asesamabhijānāti, taṃ bhagavato asesamabhijānato uttari abhiññeyyaṃ natthi, yadabhijānaṃ añño samaṇo vā brāhmaṇo vā bhagavatā bhiyyobhiññataro assa yadidaṃ iddhividhāsu.

\subsubsection{Aññathāsatthuguṇadassanaṃ}

\paragraph{160.} ‘‘Yaṃ taṃ, bhante, saddhena kulaputtena pattabbaṃ āraddhavīriyena thāmavatā purisathāmena purisavīriyena purisaparakkamena purisadhorayhena, anuppattaṃ taṃ bhagavatā. Na ca, bhante, bhagavā kāmesu kāmasukhallikānuyogamanuyutto hīnaṃ gammaṃ pothujjanikaṃ anariyaṃ anatthasaṃhitaṃ, na ca attakilamathānuyogamanuyutto dukkhaṃ anariyaṃ anatthasaṃhitaṃ. Catunnañca bhagavā jhānānaṃ ābhicetasikānaṃ diṭṭhadhammasukhavihārānaṃ nikāmalābhī akicchalābhī akasiralābhī.

\subsubsection{Anuyogadānappakāro}

\paragraph{161.} ‘‘Sace maṃ, bhante, evaṃ puccheyya – ‘kiṃ nu kho, āvuso sāriputta, ahesuṃ atītamaddhānaṃ aññe samaṇā vā brāhmaṇā vā bhagavatā bhiyyobhiññatarā sambodhiya’nti, evaṃ puṭṭho ahaṃ, bhante, ‘no’ti vadeyyaṃ. ‘Kiṃ panāvuso sāriputta, bhavissanti anāgatamaddhānaṃ aññe samaṇā vā brāhmaṇā vā bhagavatā bhiyyobhiññatarā sambodhiya’nti, evaṃ puṭṭho ahaṃ, bhante, ‘no’ti vadeyyaṃ . ‘Kiṃ panāvuso sāriputta, atthetarahi añño samaṇo vā brāhmaṇo vā bhagavatā bhiyyobhiññataro sambodhiya’nti, evaṃ puṭṭho ahaṃ, bhante, ‘no’ti vadeyyaṃ.

‘‘Sace pana maṃ, bhante, evaṃ puccheyya – ‘kiṃ nu kho, āvuso sāriputta, ahesuṃ atītamaddhānaṃ aññe samaṇā vā brāhmaṇā vā bhagavatā samasamā sambodhiya’nti, evaṃ puṭṭho ahaṃ, bhante, ‘eva’nti vadeyyaṃ. ‘Kiṃ panāvuso sāriputta, bhavissanti anāgatamaddhānaṃ aññe samaṇā vā brāhmaṇā vā bhagavatā samasamā sambodhiya’nti, evaṃ puṭṭho ahaṃ, bhante, ‘‘eva’’nti vadeyyaṃ . ‘Kiṃ panāvuso sāriputta, atthetarahi aññe samaṇā vā brāhmaṇā vā bhagavatā samasamā sambodhiya’nti, evaṃ puṭṭho ahaṃ bhante ‘no’ti vadeyyaṃ.

‘‘Sace pana maṃ, bhante, evaṃ puccheyya – ‘kiṃ panāyasmā sāriputto ekaccaṃ abbhanujānāti , ekaccaṃ na abbhanujānātī’ti, evaṃ puṭṭho ahaṃ, bhante, evaṃ byākareyyaṃ – ‘sammukhā metaṃ, āvuso, bhagavato sutaṃ, sammukhā paṭiggahitaṃ – ‘‘ahesuṃ atītamaddhānaṃ arahanto sammāsambuddhā mayā samasamā sambodhiya’’nti. Sammukhā metaṃ, āvuso, bhagavato sutaṃ, sammukhā paṭiggahitaṃ – ‘‘bhavissanti anāgatamaddhānaṃ arahanto sammāsambuddhā mayā samasamā sambodhiya’’nti. Sammukhā metaṃ, āvuso, bhagavato sutaṃ sammukhā paṭiggahitaṃ – ‘‘aṭṭhānametaṃ anavakāso yaṃ ekissā lokadhātuyā dve arahanto sammāsambuddhā apubbaṃ acarimaṃ uppajjeyyuṃ, netaṃ ṭhānaṃ vijjatī’’’ti.

‘‘Kaccāhaṃ, bhante, evaṃ puṭṭho evaṃ byākaramāno vuttavādī ceva bhagavato homi, na ca bhagavantaṃ abhūtena abbhācikkhāmi, dhammassa cānudhammaṃ byākaromi, na ca koci sahadhammiko vādānuvādo\footnote{vādānupāto (sī.)} gārayhaṃ ṭhānaṃ āgacchatī’’ti? ‘‘Taggha tvaṃ, sāriputta, evaṃ puṭṭho evaṃ byākaramāno vuttavādī ceva me hosi, na ca maṃ abhūtena abbhācikkhasi, dhammassa cānudhammaṃ byākarosi, na ca koci sahadhammiko vādānuvādo gārayhaṃ ṭhānaṃ āgacchatī’’ti.

\subsubsection{Acchariyaabbhutaṃ}

\paragraph{162.} Evaṃ vutte, āyasmā udāyī bhagavantaṃ etadavoca – ‘‘acchariyaṃ, bhante, abbhutaṃ, bhante, tathāgatassa appicchatā santuṭṭhitā sallekhatā. Yatra hi nāma tathāgato evaṃmahiddhiko evaṃmahānubhāvo, atha ca pana nevattānaṃ pātukarissati! Ekamekañcepi ito, bhante, dhammaṃ aññatitthiyā paribbājakā attani samanupasseyyuṃ, te tāvatakeneva paṭākaṃ parihareyyuṃ. Acchariyaṃ, bhante, abbhutaṃ, bhante, tathāgatassa appicchatā santuṭṭhitā sallekhatā. Yatra hi nāma tathāgato evaṃ mahiddhiko evaṃmahānubhāvo. Atha ca pana nevattānaṃ pātukarissatī’’ti!

‘‘Passa kho tvaṃ, udāyi, ‘tathāgatassa appicchatā santuṭṭhitā sallekhatā. Yatra hi nāma tathāgato evaṃmahiddhiko evaṃmahānubhāvo, atha ca pana nevattānaṃ pātukarissati’! Ekamekañcepi ito, udāyi, dhammaṃ aññatitthiyā paribbājakā attani samanupasseyyuṃ, te tāvatakeneva paṭākaṃ parihareyyuṃ. Passa kho tvaṃ, udāyi, ‘tathāgatassa appicchatā santuṭṭhitā sallekhatā. Yatra hi nāma tathāgato evaṃmahiddhiko evaṃmahānubhāvo, atha ca pana nevattānaṃ pātukarissatī’’’ti!

\paragraph{163.} Atha kho bhagavā āyasmantaṃ sāriputtaṃ āmantesi – ‘‘tasmā tiha tvaṃ, sāriputta, imaṃ dhammapariyāyaṃ abhikkhaṇaṃ bhāseyyāsi bhikkhūnaṃ bhikkhunīnaṃ upāsakānaṃ upāsikānaṃ. Yesampi hi, sāriputta, moghapurisānaṃ bhavissati tathāgate kaṅkhā vā vimati vā, tesamimaṃ dhammapariyāyaṃ sutvā tathāgate kaṅkhā vā vimati vā, sā pahīyissatī’’ti. Iti hidaṃ āyasmā sāriputto bhagavato sammukhā sampasādaṃ pavedesi. Tasmā imassa veyyākaraṇassa sampasādanīyaṃ tveva adhivacananti.

\xsectionEnd{Sampasādanīyasuttaṃ niṭṭhitaṃ pañcamaṃ.}


\section{Pāsādikasuttaṃ}

\paragraph{164.} Evaṃ me sutaṃ – ekaṃ samayaṃ bhagavā sakkesu viharati vedhaññā nāma sakyā, tesaṃ ambavane pāsāde.

\subsubsection{Nigaṇṭhanāṭaputtakālaṅkiriyā}

Tena kho pana samayena nigaṇṭho nāṭaputto\footnote{nāthaputto (sī. pī.)} pāvāyaṃ adhunākālaṅkato hoti. Tassa kālaṅkiriyāya bhinnā nigaṇṭhā dvedhikajātā bhaṇḍanajātā kalahajātā vivādāpannā aññamaññaṃ mukhasattīhi vitudantā viharanti – ‘‘na tvaṃ imaṃ dhammavinayaṃ ājānāsi, ahaṃ imaṃ dhammavinayaṃ ājānāmi, kiṃ tvaṃ imaṃ dhammavinayaṃ ājānissasi? Micchāpaṭipanno tvamasi, ahamasmi sammāpaṭipanno. Sahitaṃ me, asahitaṃ te. Purevacanīyaṃ pacchā avaca, pacchāvacanīyaṃ pure avaca. Adhiciṇṇaṃ te viparāvattaṃ, āropito te vādo, niggahito tvamasi, cara vādappamokkhāya, nibbeṭhehi vā sace pahosī’’ti. Vadhoyeva kho\footnote{vadhoyeveko (ka.)} maññe nigaṇṭhesu nāṭaputtiyesu vattati\footnote{anuvattati (syā. ka.)}. Yepi nigaṇṭhassa nāṭaputtassa sāvakā gihī odātavasanā , tepi\footnote{te tesu (ka.)} nigaṇṭhesu nāṭaputtiyesu nibbinnarūpā\footnote{nibbindarūpā (ka.)} virattarūpā paṭivānarūpā, yathā taṃ durakkhāte dhammavinaye duppavedite aniyyānike anupasamasaṃvattanike asammāsambuddhappavedite bhinnathūpe appaṭisaraṇe.

\paragraph{165.} Atha kho cundo samaṇuddeso pāvāyaṃ vassaṃvuṭṭho\footnote{vassaṃvuttho (sī. syā. pī.)} yena sāmagāmo, yenāyasmā ānando tenupasaṅkami; upasaṅkamitvā āyasmantaṃ ānandaṃ abhivādetvā ekamantaṃ nisīdi. Ekamantaṃ nisinno kho cundo samaṇuddeso āyasmantaṃ ānandaṃ etadavoca – ‘‘nigaṇṭho, bhante, nāṭaputto pāvāyaṃ adhunākālaṅkato. Tassa kālaṅkiriyāya bhinnā nigaṇṭhā dvedhikajātā…pe… bhinnathūpe appaṭisaraṇe’’ti.

Evaṃ vutte, āyasmā ānando cundaṃ samaṇuddesaṃ etadavoca – ‘‘atthi kho idaṃ, āvuso cunda, kathāpābhataṃ bhagavantaṃ dassanāya. Āyāmāvuso cunda, yena bhagavā tenupasaṅkamissāma; upasaṅkamitvā etamatthaṃ bhagavato ārocessāmā’’ti\footnote{āroceyyāmāti (syā.)}. ‘‘Evaṃ, bhante’’ti kho cundo samaṇuddeso āyasmato ānandassa paccassosi.

Atha kho āyasmā ca ānando cundo ca samaṇuddeso yena bhagavā tenupasaṅkamiṃsu; upasaṅkamitvā bhagavantaṃ abhivādetvā ekamantaṃ nisīdiṃsu. Ekamantaṃ nisinno kho āyasmā ānando bhagavantaṃ etadavoca – ‘‘ayaṃ, bhante, cundo samaṇuddeso evamāha, ‘nigaṇṭho, bhante, nāṭaputto pāvāyaṃ adhunākālaṅkato, tassa kālaṅkiriyāya bhinnā nigaṇṭhā…pe… bhinnathūpe appaṭisaraṇe’’’ti.

\subsubsection{Asammāsambuddhappaveditadhammavinayo}

\paragraph{166.} ‘‘Evaṃ hetaṃ, cunda, hoti durakkhāte dhammavinaye duppavedite aniyyānike anupasamasaṃvattanike asammāsambuddhappavedite. Idha, cunda, satthā ca hoti asammāsambuddho, dhammo ca durakkhāto duppavedito aniyyāniko anupasamasaṃvattaniko asammāsambuddhappavedito, sāvako ca tasmiṃ dhamme na dhammānudhammappaṭipanno viharati na sāmīcippaṭipanno na anudhammacārī, vokkamma ca tamhā dhammā vattati. So evamassa vacanīyo – ‘tassa te, āvuso, lābhā, tassa te suladdhaṃ, satthā ca te asammāsambuddho, dhammo ca durakkhāto duppavedito aniyyāniko anupasamasaṃvattaniko asammāsambuddhappavedito. Tvañca tasmiṃ dhamme na dhammānudhammappaṭipanno viharasi, na sāmīcippaṭipanno, na anudhammacārī, vokkamma ca tamhā dhammā vattasī’ti. Iti kho, cunda, satthāpi tattha gārayho, dhammopi tattha gārayho, sāvako ca tattha evaṃ pāsaṃso. Yo kho, cunda, evarūpaṃ sāvakaṃ evaṃ vadeyya – ‘etāyasmā tathā paṭipajjatu, yathā te satthārā dhammo desito paññatto’ti. Yo ca samādapeti\footnote{samādāpeti (sī. ṭṭha.)}, yañca samādapeti, yo ca samādapito\footnote{samādāpito (sī. ṭṭha.)} tathattāya paṭipajjati. Sabbe te bahuṃ apuññaṃ pasavanti. Taṃ kissa hetu? Evaṃ hetaṃ, cunda, hoti durakkhāte dhammavinaye duppavedite aniyyānike anupasamasaṃvattanike asammāsambuddhappavedite.

\paragraph{167.} ‘‘Idha pana, cunda, satthā ca hoti asammāsambuddho, dhammo ca durakkhāto duppavedito aniyyāniko anupasamasaṃvattaniko asammāsambuddhappavedito, sāvako ca tasmiṃ dhamme dhammānudhammappaṭipanno viharati sāmīcippaṭipanno anudhammacārī, samādāya taṃ dhammaṃ vattati. So evamassa vacanīyo – ‘tassa te, āvuso, alābhā, tassa te dulladdhaṃ, satthā ca te asammāsambuddho , dhammo ca durakkhāto duppavedito aniyyāniko anupasamasaṃvattaniko asammāsambuddhappavedito. Tvañca tasmiṃ dhamme dhammānudhammappaṭipanno viharasi sāmīcippaṭipanno anudhammacārī, samādāya taṃ dhammaṃ vattasī’ti. Iti kho, cunda, satthāpi tattha gārayho, dhammopi tattha gārayho, sāvakopi tattha evaṃ gārayho. Yo kho, cunda, evarūpaṃ sāvakaṃ evaṃ vadeyya – ‘addhāyasmā ñāyappaṭipanno ñāyamārādhessatī’ti. Yo ca pasaṃsati, yañca pasaṃsati, yo ca pasaṃsito bhiyyoso mattāya vīriyaṃ ārabhati. Sabbe te bahuṃ apuññaṃ pasavanti. Taṃ kissa hetu? Evañhetaṃ, cunda, hoti durakkhāte dhammavinaye duppavedite aniyyānike anupasamasaṃvattanike asammāsambuddhappavedite.

\subsubsection{Sammāsambuddhappaveditadhammavinayo}

\paragraph{168.} ‘‘Idha pana, cunda, satthā ca hoti sammāsambuddho, dhammo ca svākkhāto suppavedito niyyāniko upasamasaṃvattaniko sammāsambuddhappavedito, sāvako ca tasmiṃ dhamme na dhammānudhammappaṭipanno viharati, na sāmīcippaṭipanno, na anudhammacārī, vokkamma ca tamhā dhammā vattati. So evamassa vacanīyo – ‘tassa te, āvuso, alābhā, tassa te dulladdhaṃ, satthā ca te sammāsambuddho, dhammo ca svākkhāto suppavedito niyyāniko upasamasaṃvattaniko sammāsambuddhappavedito. Tvañca tasmiṃ dhamme na dhammānudhammappaṭipanno viharasi, na sāmīcippaṭipanno, na anudhammacārī, vokkamma ca tamhā dhammā vattasī’ti. Iti kho, cunda, satthāpi tattha pāsaṃso, dhammopi tattha pāsaṃso, sāvako ca tattha evaṃ gārayho. Yo kho, cunda, evarūpaṃ sāvakaṃ evaṃ vadeyya – ‘etāyasmā tathā paṭipajjatu yathā te satthārā dhammo desito paññatto’ti. Yo ca samādapeti, yañca samādapeti, yo ca samādapito tathattāya paṭipajjati. Sabbe te bahuṃ puññaṃ pasavanti. Taṃ kissa hetu? Evañhetaṃ , cunda, hoti svākkhāte dhammavinaye suppavedite niyyānike upasamasaṃvattanike sammāsambuddhappavedite.

\paragraph{169.} ‘‘Idha pana, cunda, satthā ca hoti sammāsambuddho, dhammo ca svākkhāto suppavedito niyyāniko upasamasaṃvattaniko sammāsambuddhappavedito, sāvako ca tasmiṃ dhamme dhammānudhammappaṭipanno viharati sāmīcippaṭipanno anudhammacārī, samādāya taṃ dhammaṃ vattati. So evamassa vacanīyo – ‘tassa te, āvuso, lābhā, tassa te suladdhaṃ, satthā ca te\footnote{satthā ca te arahaṃ (syā.)} sammāsambuddho , dhammo ca svākkhāto suppavedito niyyāniko upasamasaṃvattaniko sammāsambuddhappavedito. Tvañca tasmiṃ dhamme dhammānudhammappaṭipanno viharasi sāmīcippaṭipanno anudhammacārī, samādāya taṃ dhammaṃ vattasī’ti. Iti kho, cunda, satthāpi tattha pāsaṃso, dhammopi tattha pāsaṃso, sāvakopi tattha evaṃ pāsaṃso. Yo kho, cunda, evarūpaṃ sāvakaṃ evaṃ vadeyya – ‘addhāyasmā ñāyappaṭipanno ñāyamārādhessatī’ti. Yo ca pasaṃsati, yañca pasaṃsati, yo ca pasaṃsito\footnote{pasattho (syā.)} bhiyyoso mattāya vīriyaṃ ārabhati. Sabbe te bahuṃ puññaṃ pasavanti. Taṃ kissa hetu? Evañhetaṃ, cunda, hoti svākkhāte dhammavinaye suppavedite niyyānike upasamasaṃvattanike sammāsambuddhappavedite.

\subsubsection{Sāvakānutappasatthu}

\paragraph{170.} ‘‘Idha pana, cunda, satthā ca loke udapādi arahaṃ sammāsambuddho, dhammo ca svākkhāto suppavedito niyyāniko upasamasaṃvattaniko sammāsambuddhappavedito, aviññāpitatthā cassa honti sāvakā saddhamme, na ca tesaṃ kevalaṃ paripūraṃ brahmacariyaṃ āvikataṃ hoti uttānīkataṃ sabbasaṅgāhapadakataṃ sappāṭihīrakataṃ yāva devamanussehi suppakāsitaṃ. Atha nesaṃ satthuno antaradhānaṃ hoti. Evarūpo kho, cunda, satthā sāvakānaṃ kālaṅkato anutappo hoti. Taṃ kissa hetu? Satthā ca no loke udapādi arahaṃ sammāsambuddho, dhammo ca svākkhāto suppavedito niyyāniko upasamasaṃvattaniko sammāsambuddhappavedito, aviññāpitatthā camha saddhamme, na ca no kevalaṃ paripūraṃ brahmacariyaṃ āvikataṃ hoti uttānīkataṃ sabbasaṅgāhapadakataṃ sappāṭihīrakataṃ yāva devamanussehi suppakāsitaṃ. Atha no satthuno antaradhānaṃ hotīti. Evarūpo kho, cunda, satthā sāvakānaṃ kālaṅkato anutappo hoti.

\subsubsection{Sāvakānanutappasatthu}

\paragraph{171.} ‘‘Idha pana, cunda, satthā ca loke udapādi arahaṃ sammāsambuddho. Dhammo ca svākkhāto suppavedito niyyāniko upasamasaṃvattaniko sammāsambuddhappavedito. Viññāpitatthā cassa honti sāvakā saddhamme, kevalañca tesaṃ paripūraṃ brahmacariyaṃ āvikataṃ hoti uttānīkataṃ sabbasaṅgāhapadakataṃ sappāṭihīrakataṃ yāva devamanussehi suppakāsitaṃ. Atha nesaṃ satthuno antaradhānaṃ hoti. Evarūpo kho, cunda, satthā sāvakānaṃ kālaṅkato ananutappo hoti . Taṃ kissa hetu? Satthā ca no loke udapādi arahaṃ sammāsambuddho. Dhammo ca svākkhāto suppavedito niyyāniko upasamasaṃvattaniko sammāsambuddhappavedito. Viññāpitatthā camha saddhamme, kevalañca no paripūraṃ brahmacariyaṃ āvikataṃ hoti uttānīkataṃ sabbasaṅgāhapadakataṃ sappāṭihīrakataṃ yāva devamanussehi suppakāsitaṃ . Atha no satthuno antaradhānaṃ hotīti. Evarūpo kho, cunda, satthā sāvakānaṃ kālaṅkato ananutappo hoti.

\subsubsection{Brahmacariyaaparipūrādikathā}

\paragraph{172.} ‘‘Etehi cepi, cunda, aṅgehi samannāgataṃ brahmacariyaṃ hoti, no ca kho satthā hoti thero rattaññū cirapabbajito addhagato vayoanuppatto. Evaṃ taṃ brahmacariyaṃ aparipūraṃ hoti tenaṅgena.

‘‘Yato ca kho, cunda, etehi ceva aṅgehi samannāgataṃ brahmacariyaṃ hoti, satthā ca hoti thero rattaññū cirapabbajito addhagato vayoanuppatto. Evaṃ taṃ brahmacariyaṃ paripūraṃ hoti tenaṅgena.

\paragraph{173.} ‘‘Etehi cepi, cunda, aṅgehi samannāgataṃ brahmacariyaṃ hoti, satthā ca hoti thero rattaññū cirapabbajito addhagato vayoanuppatto , no ca khvassa therā bhikkhū sāvakā honti viyattā vinītā visāradā pattayogakkhemā. Alaṃ samakkhātuṃ saddhammassa, alaṃ uppannaṃ parappavādaṃ sahadhammehi suniggahitaṃ niggahetvā sappāṭihāriyaṃ dhammaṃ desetuṃ. Evaṃ taṃ brahmacariyaṃ aparipūraṃ hoti tenaṅgena.

‘‘Yato ca kho, cunda, etehi ceva aṅgehi samannāgataṃ brahmacariyaṃ hoti, satthā ca hoti thero rattaññū cirapabbajito addhagato vayoanuppatto, therā cassa bhikkhū sāvakā honti viyattā vinītā visāradā pattayogakkhemā. Alaṃ samakkhātuṃ saddhammassa, alaṃ uppannaṃ parappavādaṃ sahadhammehi suniggahitaṃ niggahetvā sappāṭihāriyaṃ dhammaṃ desetuṃ. Evaṃ taṃ brahmacariyaṃ paripūraṃ hoti tenaṅgena.

\paragraph{174.} ‘‘Etehi cepi, cunda, aṅgehi samannāgataṃ brahmacariyaṃ hoti, satthā ca hoti thero rattaññū cirapabbajito addhagato vayoanuppatto, therā cassa bhikkhū sāvakā honti viyattā vinītā visāradā pattayogakkhemā. Alaṃ samakkhātuṃ saddhammassa, alaṃ uppannaṃ parappavādaṃ sahadhammehi suniggahitaṃ niggahetvā sappāṭihāriyaṃ dhammaṃ desetuṃ. No ca khvassa majjhimā bhikkhū sāvakā honti…pe… majjhimā cassa bhikkhū sāvakā honti, no ca khvassa navā bhikkhū sāvakā honti…pe… navā cassa bhikkhū sāvakā honti, no ca khvassa therā bhikkhuniyo sāvikā honti…pe… therā cassa bhikkhuniyo sāvikā honti, no ca khvassa majjhimā bhikkhuniyo sāvikā honti…pe… majjhimā cassa bhikkhuniyo sāvikā honti , no ca khvassa navā bhikkhuniyo sāvikā honti…pe… navā cassa bhikkhuniyo sāvikā honti, no ca khvassa upāsakā sāvakā honti gihī odātavasanā brahmacārino…pe… upāsakā cassa sāvakā honti gihī odātavasanā brahmacārino, no ca khvassa upāsakā sāvakā honti gihī odātavasanā kāmabhogino…pe… upāsakā cassa sāvakā honti gihī odātavasanā kāmabhogino, no ca khvassa upāsikā sāvikā honti gihiniyo odātavasanā brahmacāriniyo…pe… upāsikā cassa sāvikā honti gihiniyo odātavasanā brahmacāriniyo, no ca khvassa upāsikā sāvikā honti gihiniyo odātavasanā kāmabhoginiyo…pe… upāsikā cassa sāvikā honti gihiniyo odātavasanā kāmabhoginiyo, no ca khvassa brahmacariyaṃ hoti iddhañceva phītañca vitthārikaṃ bāhujaññaṃ puthubhūtaṃ yāva devamanussehi suppakāsitaṃ…pe… brahmacariyañcassa hoti iddhañceva phītañca vitthārikaṃ bāhujaññaṃ puthubhūtaṃ yāva devamanussehi suppakāsitaṃ, no ca kho lābhaggayasaggappattaṃ. Evaṃ taṃ brahmacariyaṃ aparipūraṃ hoti tenaṅgena.

‘‘Yato ca kho, cunda, etehi ceva aṅgehi samannāgataṃ brahmacariyaṃ hoti, satthā ca hoti thero rattaññū cirapabbajito addhagato vayoanuppatto, therā cassa bhikkhū sāvakā honti viyattā vinītā visāradā pattayogakkhemā. Alaṃ samakkhātuṃ saddhammassa, alaṃ uppannaṃ parappavādaṃ sahadhammehi suniggahitaṃ niggahetvā sappāṭihāriyaṃ dhammaṃ desetuṃ. Majjhimā cassa bhikkhū sāvakā honti…pe… navā cassa bhikkhū sāvakā honti…pe… therā cassa bhikkhuniyo sāvikā honti…pe… majjhimā cassa bhikkhuniyo sāvikā honti…pe… navā cassa bhikkhuniyo sāvikā honti…pe… upāsakā cassa sāvakā honti…pe… gihī odātavasanā brahmacārino . Upāsakā cassa sāvakā honti gihī odātavasanā kāmabhogino…pe… upāsikā cassa sāvikā honti gihiniyo odātavasanā brahmacāriniyo…pe… upāsikā cassa sāvikā honti gihiniyo odātavasanā kāmabhoginiyo…pe… brahmacariyañcassa hoti iddhañceva phītañca vitthārikaṃ bāhujaññaṃ puthubhūtaṃ yāva devamanussehi suppakāsitaṃ, lābhaggappattañca yasaggappattañca. Evaṃ taṃ brahmacariyaṃ paripūraṃ hoti tenaṅgena.

\paragraph{175.} ‘‘Ahaṃ kho pana, cunda, etarahi satthā loke uppanno arahaṃ sammāsambuddho. Dhammo ca svākkhāto suppavedito niyyāniko upasamasaṃvattaniko sammāsambuddhappavedito. Viññāpitatthā ca me sāvakā saddhamme, kevalañca tesaṃ paripūraṃ brahmacariyaṃ āvikataṃ uttānīkataṃ sabbasaṅgāhapadakataṃ sappāṭihīrakataṃ yāva devamanussehi suppakāsitaṃ. Ahaṃ kho pana, cunda, etarahi satthā thero rattaññū cirapabbajito addhagato vayoanuppatto.

‘‘Santi kho pana me, cunda, etarahi therā bhikkhū sāvakā honti viyattā vinītā visāradā pattayogakkhemā. Alaṃ samakkhātuṃ saddhammassa, alaṃ uppannaṃ parappavādaṃ sahadhammehi suniggahitaṃ niggahetvā sappāṭihāriyaṃ dhammaṃ desetuṃ. Santi kho pana me, cunda , etarahi majjhimā bhikkhū sāvakā…pe… santi kho pana me, cunda, etarahi navā bhikkhū sāvakā…pe… santi kho pana me, cunda, etarahi therā bhikkhuniyo sāvikā…pe… santi kho pana me, cunda, etarahi majjhimā bhikkhuniyo sāvikā…pe… santi kho pana me, cunda, etarahi navā bhikkhuniyo sāvikā…pe… santi kho pana me, cunda, etarahi upāsakā sāvakā gihī odātavasanā brahmacārino…pe… santi kho pana me, cunda, etarahi upāsakā sāvakā gihī odātavasanā kāmabhogino…pe… santi kho pana me, cunda, etarahi upāsikā sāvikā gihiniyo odātavasanā brahmacāriniyo…pe… santi kho pana me, cunda, etarahi upāsikā sāvikā gihiniyo odātavasanā kāmabhoginiyo…pe… etarahi kho pana me, cunda, brahmacariyaṃ iddhañceva phītañca vitthārikaṃ bāhujaññaṃ puthubhūtaṃ yāva devamanussehi suppakāsitaṃ.

\paragraph{176.} ‘‘Yāvatā kho, cunda, etarahi satthāro loke uppannā, nāhaṃ, cunda, aññaṃ ekasatthārampi samanupassāmi evaṃlābhaggayasaggappattaṃ yatharivāhaṃ. Yāvatā kho pana, cunda, etarahi saṅgho vā gaṇo vā loke uppanno; nāhaṃ, cunda, aññaṃ ekaṃ saṃghampi samanupassāmi evaṃlābhaggayasaggappattaṃ yatharivāyaṃ, cunda, bhikkhusaṅgho. Yaṃ kho taṃ, cunda, sammā vadamāno vadeyya – ‘sabbākārasampannaṃ sabbākāraparipūraṃ anūnamanadhikaṃ svākkhātaṃ kevalaṃ paripūraṃ brahmacariyaṃ suppakāsita’nti. Idameva taṃ sammā vadamāno vadeyya – ‘sabbākārasampannaṃ…pe… suppakāsita’nti.

‘‘Udako\footnote{uddako (sī. syā. pī.)} sudaṃ, cunda, rāmaputto evaṃ vācaṃ bhāsati – ‘passaṃ na passatī’ti. Kiñca passaṃ na passatīti? Khurassa sādhunisitassa talamassa passati, dhārañca khvassa na passati. Idaṃ vuccati – ‘passaṃ na passatī’ti. Yaṃ kho panetaṃ, cunda, udakena rāmaputtena bhāsitaṃ hīnaṃ gammaṃ pothujjanikaṃ anariyaṃ anatthasaṃhitaṃ khurameva sandhāya. Yañca taṃ\footnote{yañcetaṃ (syā. ka.)}, cunda, sammā vadamāno vadeyya – ‘passaṃ na passatī’ti, idameva taṃ\footnote{idamevetaṃ (ka.)} sammā vadamāno vadeyya – ‘passaṃ na passatī’ti. Kiñca passaṃ na passatīti? Evaṃ sabbākārasampannaṃ sabbākāraparipūraṃ anūnamanadhikaṃ svākkhātaṃ kevalaṃ paripūraṃ brahmacariyaṃ suppakāsitanti, iti hetaṃ passati\footnote{suppakāsitaṃ, iti hetaṃ na passatīti (syā. ka.)}. Idamettha apakaḍḍheyya, evaṃ taṃ parisuddhataraṃ assāti, iti hetaṃ na passati\footnote{na passatīti (syā. ka.)}. Idamettha upakaḍḍheyya, evaṃ taṃ paripūraṃ\footnote{parisuddhataraṃ (syā. ka.), paripūrataraṃ (?)} assāti, iti hetaṃ na passati. Idaṃ vuccati cunda – ‘passaṃ na passatī’ti. Yaṃ kho taṃ, cunda, sammā vadamāno vadeyya – ‘sabbākārasampannaṃ…pe… brahmacariyaṃ suppakāsita’nti. Idameva taṃ sammā vadamāno vadeyya – ‘sabbākārasampannaṃ sabbākāraparipūraṃ anūnamanadhikaṃ svākkhātaṃ kevalaṃ paripūraṃ brahmacariyaṃ suppakāsita’nti.

\subsubsection{Saṅgāyitabbadhammo}

\paragraph{177.} Tasmātiha, cunda, ye vo mayā dhammā abhiññā desitā, tattha sabbeheva saṅgamma samāgamma atthena atthaṃ byañjanena byañjanaṃ saṅgāyitabbaṃ na vivaditabbaṃ, yathayidaṃ brahmacariyaṃ addhaniyaṃ assa ciraṭṭhitikaṃ, tadassa bahujanahitāya bahujanasukhāya lokānukampāya atthāya hitāya sukhāya devamanussānaṃ. Katame ca te, cunda , dhammā mayā abhiññā desitā, yattha sabbeheva saṅgamma samāgamma atthena atthaṃ byañjanena byañjanaṃ saṅgāyitabbaṃ na vivaditabbaṃ, yathayidaṃ brahmacariyaṃ addhaniyaṃ assa ciraṭṭhitikaṃ, tadassa bahujanahitāya bahujanasukhāya lokānukampāya atthāya hitāya sukhāya devamanussānaṃ? Seyyathidaṃ – cattāro satipaṭṭhānā, cattāro sammappadhānā, cattāro iddhipādā, pañcindriyāni, pañca balāni, satta bojjhaṅgā , ariyo aṭṭhaṅgiko maggo. Ime kho te, cunda, dhammā mayā abhiññā desitā. Yattha sabbeheva saṅgamma samāgamma atthena atthaṃ byañjanena byañjanaṃ saṅgāyitabbaṃ na vivaditabbaṃ, yathayidaṃ brahmacariyaṃ addhaniyaṃ assa ciraṭṭhitikaṃ, tadassa bahujanahitāya bahujanasukhāya lokānukampāya atthāya hitāya sukhāya devamanussānaṃ.

\subsubsection{Saññāpetabbavidhi}

\paragraph{178.} ‘‘Tesañca vo, cunda, samaggānaṃ sammodamānānaṃ avivadamānānaṃ sikkhataṃ\footnote{sikkhitabbaṃ (bahūsu)} aññataro sabrahmacārī saṅghe dhammaṃ bhāseyya. Tatra ce tumhākaṃ evamassa – ‘ayaṃ kho āyasmā atthañceva micchā gaṇhāti, byañjanāni ca micchā ropetī’ti. Tassa neva abhinanditabbaṃ na paṭikkositabbaṃ, anabhinanditvā appaṭikkositvā so evamassa vacanīyo – ‘imassa nu kho, āvuso, atthassa imāni vā byañjanāni etāni vā byañjanāni katamāni opāyikatarāni, imesañca\footnote{imesaṃ vā (syā. pī. ka.), imesaṃ (sī.)} byañjanānaṃ ayaṃ vā attho eso vā attho katamo opāyikataro’ti ? So ce evaṃ vadeyya – ‘imassa kho, āvuso, atthassa imāneva byañjanāni opāyikatarāni, yā ceva\footnote{yañceva (sī. ka.), ṭīkā oloketabbā} etāni; imesañca\footnote{imedaṃ (sabbattha)} byañjanānaṃ ayameva attho opāyikataro, yā ceva\footnote{yañceva (sī. ka.), ṭīkā oloketabbā} eso’ti. So neva ussādetabbo na apasādetabbo, anussādetvā anapasādetvā sveva sādhukaṃ saññāpetabbo tassa ca atthassa tesañca byañjanānaṃ nisantiyā.

\paragraph{179.} ‘‘Aparopi ce, cunda, sabrahmacārī saṅghe dhammaṃ bhāseyya. Tatra ce tumhākaṃ evamassa – ‘ayaṃ kho āyasmā atthañhi kho micchā gaṇhāti byañjanāni sammā ropetī’ti. Tassa neva abhinanditabbaṃ na paṭikkositabbaṃ, anabhinanditvā appaṭikkositvā so evamassa vacanīyo – ‘imesaṃ nu kho, āvuso, byañjanānaṃ ayaṃ vā attho eso vā attho katamo opāyikataro’ti? So ce evaṃ vadeyya – ‘imesaṃ kho, āvuso, byañjanānaṃ ayameva attho opāyikataro, yā ceva eso’ti. So neva ussādetabbo na apasādetabbo, anussādetvā anapasādetvā sveva sādhukaṃ saññāpetabbo tasseva atthassa nisantiyā.

\paragraph{180.} ‘‘Aparopi ce, cunda, sabrahmacārī saṅghe dhammaṃ bhāseyya. Tatra ce tumhākaṃ evamassa – ‘ayaṃ kho āyasmā atthañhi kho sammā gaṇhāti byañjanāni micchā ropetī’ti. Tassa neva abhinanditabbaṃ na paṭikkositabbaṃ; anabhinanditvā appaṭikkositvā so evamassa vacanīyo – ‘imassa nu kho, āvuso, atthassa imāni vā byañjanāni etāni vā byañjanāni katamāni opāyikatarānī’ti? So ce evaṃ vadeyya – ‘imassa kho, āvuso, atthassa imāneva byañjanāni opayikatarāni, yāni ceva etānī’ti . So neva ussādetabbo na apasādetabbo; anussādetvā anapasādetvā sveva sādhukaṃ saññāpetabbo tesaññeva byañjanānaṃ nisantiyā.

\paragraph{181.} ‘‘Aparopi ce, cunda, sabrahmacārī saṅghe dhammaṃ bhāseyya. Tatra ce tumhākaṃ evamassa – ‘ayaṃ kho āyasmā atthañceva sammā gaṇhāti byañjanāni ca sammā ropetī’ti. Tassa ‘sādhū’ti bhāsitaṃ abhinanditabbaṃ anumoditabbaṃ; tassa ‘sādhū’ti bhāsitaṃ abhinanditvā anumoditvā so evamassa vacanīyo – ‘lābhā no āvuso, suladdhaṃ no āvuso, ye mayaṃ āyasmantaṃ tādisaṃ sabrahmacāriṃ passāma evaṃ atthupetaṃ byañjanupeta’nti.

Paccayānuññātakāraṇaṃ

\paragraph{182.} ‘‘Na vo ahaṃ, cunda, diṭṭhadhammikānaṃyeva āsavānaṃ saṃvarāya dhammaṃ desemi. Na panāhaṃ, cunda, samparāyikānaṃyeva āsavānaṃ paṭighātāya dhammaṃ desemi. Diṭṭhadhammikānaṃ cevāhaṃ, cunda, āsavānaṃ saṃvarāya dhammaṃ desemi; samparāyikānañca āsavānaṃ paṭighātāya. Tasmātiha, cunda, yaṃ vo mayā cīvaraṃ anuññātaṃ, alaṃ vo taṃ – yāvadeva sītassa paṭighātāya, uṇhassa paṭighātāya, ḍaṃsamakasavātātapasarīsapa\footnote{siriṃsapa (syā.)} samphassānaṃ paṭighātāya, yāvadeva hirikopīnapaṭicchādanatthaṃ. Yo vo mayā piṇḍapāto anuññāto, alaṃ vo so yāvadeva imassa kāyassa ṭhitiyā yāpanāya vihiṃsūparatiyā brahmacariyānuggahāya, iti purāṇañca vedanaṃ paṭihaṅkhāmi, navañca vedanaṃ na uppādessāmi, yātrā ca me bhavissati anavajjatā ca phāsuvihāro ca\footnote{cāti (bahūsu)}. Yaṃ vo mayā senāsanaṃ anuññātaṃ, alaṃ vo taṃ yāvadeva sītassa paṭighātāya, uṇhassa paṭighātāya, ḍaṃsamakasavātātapasarīsapasamphassānaṃ paṭighātāya, yāvadeva utuparissayavinodana paṭisallānārāmatthaṃ. Yo vo mayā gilānapaccayabhesajja parikkhāro anuññāto, alaṃ vo so yāvadeva uppannānaṃ veyyābādhikānaṃ vedanānaṃ paṭighātāya abyāpajjaparamatāya\footnote{abyāpajjhaparamatāyāti (sī. syā. pī.), abyābajjhaparamatāya (?)}.

\subsubsection{Sukhallikānuyogo}

\paragraph{183.} ‘‘Ṭhānaṃ kho panetaṃ, cunda, vijjati yaṃ aññatitthiyā paribbājakā evaṃ vadeyyuṃ – ‘sukhallikānuyogamanuyuttā samaṇā sakyaputtiyā viharantī’ti. Evaṃvādino\footnote{vadamānā (syā.)}, cunda, aññatitthiyā paribbājakā evamassu vacanīyā – ‘katamo so , āvuso, sukhallikānuyogo? Sukhallikānuyogā hi bahū anekavihitā nānappakārakā’ti.

‘‘Cattārome, cunda, sukhallikānuyogā hīnā gammā pothujjanikā anariyā anatthasaṃhitā na nibbidāya na virāgāya na nirodhāya na upasamāya na abhiññāya na sambodhāya na nibbānāya saṃvattanti. Katame cattāro?

‘‘Idha, cunda, ekacco bālo pāṇe vadhitvā vadhitvā attānaṃ sukheti pīṇeti. Ayaṃ paṭhamo sukhallikānuyogo.

‘‘Puna caparaṃ, cunda, idhekacco adinnaṃ ādiyitvā ādiyitvā attānaṃ sukheti pīṇeti. Ayaṃ dutiyo sukhallikānuyogo.

‘‘Puna caparaṃ, cunda, idhekacco musā bhaṇitvā bhaṇitvā attānaṃ sukheti pīṇeti. Ayaṃ tatiyo sukhallikānuyogo.

‘‘Puna caparaṃ, cunda, idhekacco pañcahi kāmaguṇehi samappito samaṅgībhūto paricāreti. Ayaṃ catuttho sukhallikānuyogo.

‘‘Ime kho, cunda, cattāro sukhallikānuyogā hīnā gammā pothujjanikā anariyā anatthasaṃhitā na nibbidāya na virāgāya na nirodhāya na upasamāya na abhiññāya na sambodhāya na nibbānāya saṃvattanti.

‘‘Ṭhānaṃ kho panetaṃ, cunda, vijjati yaṃ aññatitthiyā paribbājakā evaṃ vadeyyuṃ – ‘‘ime cattāro sukhallikānuyoge anuyuttā samaṇā sakyaputtiyā viharantī’ti. Te vo\footnote{te (sī. pī.)} ‘māhevaṃ’ tissu vacanīyā. Na te vo sammā vadamānā vadeyyuṃ, abbhācikkheyyuṃ asatā abhūtena.

\paragraph{184.} ‘‘Cattārome, cunda, sukhallikānuyogā ekantanibbidāya virāgāya nirodhāya upasamāya abhiññāya sambodhāya nibbānāya saṃvattanti. Katame cattāro?

‘‘Idha , cunda, bhikkhu vivicceva kāmehi vivicca akusalehi dhammehi savitakkaṃ savicāraṃ vivekajaṃ pītisukhaṃ paṭhamaṃ jhānaṃ upasampajja viharati. Ayaṃ paṭhamo sukhallikānuyogo.

‘‘Puna caparaṃ, cunda, bhikkhu vitakkavicārānaṃ vūpasamā…pe… dutiyaṃ jhānaṃ upasampajja viharati. Ayaṃ dutiyo sukhallikānuyogo.

‘‘Puna caparaṃ, cunda, bhikkhu pītiyā ca virāgā…pe… tatiyaṃ jhānaṃ upasampajja viharati. Ayaṃ tatiyo sukhallikānuyogo.

‘‘Puna caparaṃ, cunda, bhikkhu sukhassa ca pahānā dukkhassa ca pahānā…pe… catutthaṃ jhānaṃ upasampajja viharati. Ayaṃ catuttho sukhallikānuyogo.

‘‘Ime kho, cunda, cattāro sukhallikānuyogā ekantanibbidāya virāgāya nirodhāya upasamāya abhiññāya sambodhāya nibbānāya saṃvattanti.

‘‘Ṭhānaṃ kho panetaṃ, cunda, vijjati yaṃ aññatitthiyā paribbājakā evaṃ vadeyyuṃ – ‘‘ime cattāro sukhallikānuyoge anuyuttā samaṇā sakyaputtiyā viharantī’ti. Te vo ‘evaṃ’ tissu vacanīyā. Sammā te vo vadamānā vadeyyuṃ, na te vo abbhācikkheyyuṃ asatā abhūtena.

\subsubsection{Sukhallikānuyogānisaṃso}

\paragraph{185.} ‘‘Ṭhānaṃ kho panetaṃ, cunda, vijjati, yaṃ aññatitthiyā paribbājakā evaṃ vadeyyuṃ – ‘ime panāvuso, cattāro sukhallikānuyoge anuyuttānaṃ viharataṃ kati phalāni katānisaṃsā pāṭikaṅkhā’ti? Evaṃvādino, cunda, aññatitthiyā paribbājakā evamassu vacanīyā – ‘ime kho, āvuso, cattāro sukhallikānuyoge anuyuttānaṃ viharataṃ cattāri phalāni cattāro ānisaṃsā pāṭikaṅkhā. Katame cattāro? Idhāvuso, bhikkhu tiṇṇaṃ saṃyojanānaṃ parikkhayā sotāpanno hoti avinipātadhammo niyato sambodhiparāyaṇo. Idaṃ paṭhamaṃ phalaṃ, paṭhamo ānisaṃso. Puna caparaṃ, āvuso, bhikkhu tiṇṇaṃ saṃyojanānaṃ parikkhayā rāgadosamohānaṃ tanuttā sakadāgāmī hoti, sakideva imaṃ lokaṃ āgantvā dukkhassantaṃ karoti. Idaṃ dutiyaṃ phalaṃ, dutiyo ānisaṃso. Puna caparaṃ, āvuso, bhikkhu pañcannaṃ orambhāgiyānaṃ saṃyojanānaṃ parikkhayā opapātiko hoti, tattha parinibbāyī anāvattidhammo tasmā lokā. Idaṃ tatiyaṃ phalaṃ, tatiyo ānisaṃso. Puna caparaṃ, āvuso, bhikkhu āsavānaṃ khayā anāsavaṃ cetovimuttiṃ paññāvimuttiṃ diṭṭheva dhamme sayaṃ abhiññā sacchikatvā upasampajja viharati. Idaṃ catutthaṃ phalaṃ catuttho ānisaṃso. Ime kho, āvuso, cattāro sukhallikānuyoge anuyuttānaṃ viharataṃ imāni cattāri phalāni, cattāro ānisaṃsā pāṭikaṅkhā’’ti.

\subsubsection{Khīṇāsavaabhabbaṭhānaṃ}

\paragraph{186.} ‘‘Ṭhānaṃ kho panetaṃ, cunda, vijjati yaṃ aññatitthiyā paribbājakā evaṃ vadeyyuṃ – ‘aṭṭhitadhammā samaṇā sakyaputtiyā viharantī’ti. Evaṃvādino, cunda, aññatitthiyā paribbājakā evamassu vacanīyā – ‘atthi kho, āvuso, tena bhagavatā jānatā passatā arahatā sammāsambuddhena sāvakānaṃ dhammā desitā paññattā yāvajīvaṃ anatikkamanīyā. Seyyathāpi, āvuso, indakhīlo vā ayokhīlo vā gambhīranemo sunikhāto acalo asampavedhī. Evameva kho, āvuso, tena bhagavatā jānatā passatā arahatā sammāsambuddhena sāvakānaṃ dhammā desitā paññattā yāvajīvaṃ anatikkamanīyā. Yo so, āvuso, bhikkhu arahaṃ khīṇāsavo vusitavā katakaraṇīyo ohitabhāro anuppattasadattho parikkhīṇabhavasaṃyojano sammadaññā vimutto, abhabbo so nava ṭhānāni ajjhācarituṃ. Abhabbo, āvuso, khīṇāsavo bhikkhu sañcicca pāṇaṃ jīvitā voropetuṃ; abhabbo khīṇāsavo bhikkhu adinnaṃ theyyasaṅkhātaṃ ādiyituṃ; abhabbo khīṇāsavo bhikkhu methunaṃ dhammaṃ paṭisevituṃ; abhabbo khīṇāsavo bhikkhu sampajānamusā bhāsituṃ; abhabbo khīṇāsavo bhikkhu sannidhikārakaṃ kāme paribhuñjituṃ seyyathāpi pubbe āgārikabhūto; abhabbo khīṇāsavo bhikkhu chandāgatiṃ gantuṃ; abhabbo khīṇāsavo bhikkhu dosāgatiṃ gantuṃ; abhabbo khīṇāsavo bhikkhu mohāgatiṃ gantuṃ; abhabbo khīṇāsavo bhikkhu bhayāgatiṃ gantuṃ. Yo so, āvuso, bhikkhu arahaṃ khīṇāsavo vusitavā katakaraṇīyo ohitabhāro anuppattasadattho parikkhīṇabhavasaṃyojano sammadaññā vimutto, abhabbo so imāni nava ṭhānāni ajjhācaritu’’nti.

\subsubsection{Pañhābyākaraṇaṃ}

\paragraph{187.} ‘‘Ṭhānaṃ kho panetaṃ, cunda, vijjati, yaṃ aññatitthiyā paribbājakā evaṃ vadeyyuṃ – ‘atītaṃ kho addhānaṃ ārabbha samaṇo gotamo atīrakaṃ ñāṇadassanaṃ paññapeti, no ca kho anāgataṃ addhānaṃ ārabbha atīrakaṃ ñāṇadassanaṃ paññapeti, tayidaṃ kiṃsu tayidaṃ kathaṃsū’ti? Te ca aññatitthiyā paribbājakā aññavihitakena ñāṇadassanena aññavihitakaṃ ñāṇadassanaṃ paññapetabbaṃ maññanti yathariva bālā abyattā. Atītaṃ kho, cunda, addhānaṃ ārabbha tathāgatassa satānusāri ñāṇaṃ hoti; so yāvatakaṃ ākaṅkhati tāvatakaṃ anussarati. Anāgatañca kho addhānaṃ ārabbha tathāgatassa bodhijaṃ ñāṇaṃ uppajjati – ‘ayamantimā jāti, natthidāni punabbhavo’ti. ‘Atītaṃ cepi, cunda, hoti abhūtaṃ atacchaṃ anatthasaṃhitaṃ, na taṃ tathāgato byākaroti. Atītaṃ cepi, cunda, hoti bhūtaṃ tacchaṃ anatthasaṃhitaṃ, tampi tathāgato na byākaroti. Atītaṃ cepi cunda, hoti bhūtaṃ tacchaṃ atthasaṃhitaṃ, tatra kālaññū tathāgato hoti tassa pañhassa veyyākaraṇāya. Anāgataṃ cepi, cunda, hoti abhūtaṃ atacchaṃ anatthasaṃhitaṃ, na taṃ tathāgato byākaroti…pe… tassa pañhassa veyyākaraṇāya. Paccuppannaṃ cepi, cunda, hoti abhūtaṃ atacchaṃ anatthasaṃhitaṃ, na taṃ tathāgato byākaroti. Paccuppannaṃ cepi, cunda, hoti bhūtaṃ tacchaṃ anatthasaṃhitaṃ, tampi tathāgato na byākaroti. Paccuppannaṃ cepi, cunda, hoti bhūtaṃ tacchaṃ atthasaṃhitaṃ, tatra kālaññū tathāgato hoti tassa pañhassa veyyākaraṇāya.

\paragraph{188.} ‘‘Iti kho, cunda, atītānāgatapaccuppannesu dhammesu tathāgato kālavādī\footnote{kālavādī saccavādī (syā.)} bhūtavādī atthavādī dhammavādī vinayavādī, tasmā ‘tathāgato’ti vuccati. Yañca kho, cunda, sadevakassa lokassa samārakassa sabrahmakassa sassamaṇabrāhmaṇiyā pajāya sadevamanussāya diṭṭhaṃ sutaṃ mutaṃ viññātaṃ pattaṃ pariyesitaṃ anuvicaritaṃ manasā, sabbaṃ tathāgatena abhisambuddhaṃ, tasmā ‘tathāgato’ti vuccati. Yañca, cunda, rattiṃ tathāgato anuttaraṃ sammāsambodhiṃ abhisambujjhati, yañca rattiṃ anupādisesāya nibbānadhātuyā parinibbāyati, yaṃ etasmiṃ antare bhāsati lapati niddisati. Sabbaṃ taṃ tatheva hoti no aññathā, tasmā ‘tathāgato’ti vuccati. Yathāvādī, cunda, tathāgato tathākārī, yathākārī tathāvādī. Iti yathāvādī tathākārī, yathākārī tathāvādī, tasmā ‘tathāgato’ti vuccati. Sadevake loke, cunda, samārake sabrahmake sassamaṇabrāhmaṇiyā pajāya sadevamanussāya tathāgato abhibhū anabhibhūto aññadatthudaso vasavattī, tasmā ‘tathāgato’ti vuccati.

\subsubsection{Abyākataṭṭhānaṃ}

\paragraph{189.} ‘‘Ṭhānaṃ kho panetaṃ, cunda, vijjati yaṃ aññatitthiyā paribbājakā evaṃ vadeyyuṃ – ‘kiṃ nu kho, āvuso, hoti tathāgato paraṃ maraṇā, idameva saccaṃ moghamañña’nti? Evaṃvādino, cunda, aññatitthiyā paribbājakā evamassu vacanīyā – ‘abyākataṃ kho, āvuso, bhagavatā – ‘‘hoti tathāgato paraṃ maraṇā, idameva saccaṃ moghamañña’’’nti.

‘‘Ṭhānaṃ kho panetaṃ, cunda, vijjati, yaṃ aññatitthiyā paribbājakā evaṃ vadeyyuṃ – ‘kiṃ panāvuso, na hoti tathāgato paraṃ maraṇā, idameva saccaṃ moghamañña’nti? Evaṃvādino, cunda, aññatitthiyā paribbājakā evamassu vacanīyā – ‘etampi kho, āvuso, bhagavatā abyākataṃ – ‘‘na hoti tathāgato paraṃ maraṇā, idameva saccaṃ moghamañña’’’nti.

‘‘Ṭhānaṃ kho panetaṃ, cunda, vijjati, yaṃ aññatitthiyā paribbājakā evaṃ vadeyyuṃ – ‘kiṃ panāvuso, hoti ca na ca hoti tathāgato paraṃ maraṇā, idameva saccaṃ moghamañña’nti? Evaṃvādino, cunda, aññatitthiyā paribbājakā evamassu vacanīyā – ‘abyākataṃ kho etaṃ, āvuso, bhagavatā – ‘‘hoti ca na ca hoti tathāgato paraṃ maraṇā, idameva saccaṃ moghamañña’’’nti.

‘‘Ṭhānaṃ kho panetaṃ, cunda, vijjati, yaṃ aññatitthiyā paribbājakā evaṃ vadeyyuṃ – ‘kiṃ panāvuso, neva hoti na na hoti tathāgato paraṃ maraṇā, idameva saccaṃ moghamañña’nti? Evaṃvādino, cunda, aññatitthiyā paribbājakā evamassu vacanīyā – ‘etampi kho, āvuso, bhagavatā abyākataṃ – ‘‘neva hoti na na hoti tathāgato paraṃ maraṇā, idameva saccaṃ moghamañña’’’nti.

‘‘Ṭhānaṃ kho panetaṃ, cunda, vijjati, yaṃ aññatitthiyā paribbājakā evaṃ vadeyyuṃ – ‘kasmā panetaṃ, āvuso, samaṇena gotamena abyākata’nti? Evaṃvādino, cunda, aññatitthiyā paribbājakā evamassu vacanīyā – ‘na hetaṃ, āvuso, atthasaṃhitaṃ na dhammasaṃhitaṃ na ādibrahmacariyakaṃ na nibbidāya na virāgāya na nirodhāya na upasamāya na abhiññāya na sambodhāya na nibbānāya saṃvattati, tasmā taṃ bhagavatā abyākata’nti.

\subsubsection{Byākataṭṭhānaṃ}

\paragraph{190.} ‘‘Ṭhānaṃ kho panetaṃ, cunda, vijjati, yaṃ aññatitthiyā paribbājakā evaṃ vadeyyuṃ – ‘kiṃ panāvuso, samaṇena gotamena byākata’nti? Evaṃvādino, cunda, aññatitthiyā paribbājakā evamassu vacanīyā – ‘idaṃ dukkhanti kho, āvuso, bhagavatā byākataṃ, ayaṃ dukkhasamudayoti kho, āvuso, bhagavatā byākataṃ, ayaṃ dukkhanirodhoti kho, āvuso, bhagavatā byākataṃ, ayaṃ dukkhanirodhagāminī paṭipadāti kho, āvuso, bhagavatā byākata’nti.

‘‘Ṭhānaṃ kho panetaṃ, cunda, vijjati, yaṃ aññatitthiyā paribbājakā evaṃ vadeyyuṃ – ‘kasmā panetaṃ, āvuso, samaṇena gotamena byākata’nti? Evaṃvādino, cunda, aññatitthiyā paribbājakā evamassu vacanīyā – ‘etañhi, āvuso, atthasaṃhitaṃ, etaṃ dhammasaṃhitaṃ, etaṃ ādibrahmacariyakaṃ ekantanibbidāya virāgāya nirodhāya upasamāya abhiññāya sambodhāya nibbānāya saṃvattati. Tasmā taṃ bhagavatā byākata’nti.

\subsubsection{Pubbantasahagatadiṭṭhinissayā}

\paragraph{191.} ‘‘Yepi te, cunda, pubbantasahagatā diṭṭhinissayā, tepi vo mayā byākatā, yathā te byākātabbā. Yathā ca te na byākātabbā, kiṃ vo ahaṃ te tathā\footnote{tattha (syā. ka.)} byākarissāmi? Yepi te, cunda, aparantasahagatā diṭṭhinissayā, tepi vo mayā byākatā, yathā te byākātabbā. Yathā ca te na byākātabbā, kiṃ vo ahaṃ te tathā byākarissāmi? Katame ca te, cunda, pubbantasahagatā diṭṭhinissayā, ye vo mayā byākatā, yathā te byākātabbā. (Yathā ca te na byākātabbā, kiṃ vo ahaṃ te tathā byākarissāmi)\footnote{(yathā ca te na byākātabbā) sabbattha}? Santi kho, cunda, eke samaṇabrāhmaṇā evaṃvādino evaṃdiṭṭhino – ‘sassato attā ca loko ca, idameva saccaṃ moghamañña’nti. Santi pana, cunda, eke samaṇabrāhmaṇā evaṃvādino evaṃdiṭṭhino – ‘asassato attā ca loko ca…pe… sassato ca asassato ca attā ca loko ca… neva sassato nāsassato attā ca loko ca… sayaṃkato attā ca loko ca… paraṃkato attā ca loko ca… sayaṃkato ca paraṃkato ca attā ca loko ca… asayaṃkāro aparaṃkāro adhiccasamuppanno attā ca loko ca, idameva saccaṃ moghamañña’nti. Sassataṃ sukhadukkhaṃ… asassataṃ sukhadukkhaṃ… sassatañca asassatañca sukhadukkhaṃ… nevasassataṃ nāsassataṃ sukhadukkhaṃ… sayaṃkataṃ sukhadukkhaṃ… paraṃkataṃ sukhadukkhaṃ… sayaṃkatañca paraṃkatañca sukhadukkhaṃ… asayaṃkāraṃ aparaṃkāraṃ adhiccasamuppannaṃ sukhadukkhaṃ, idameva saccaṃ moghamañña’nti.

\paragraph{192.} ‘‘Tatra, cunda, ye te samaṇabrāhmaṇā evaṃvādino evaṃdiṭṭhino – ‘sassato attā ca loko ca, idameva saccaṃ moghamañña’nti. Tyāhaṃ upasaṅkamitvā evaṃ vadāmi – ‘atthi nu kho idaṃ, āvuso, vuccati – ‘‘sassato attā ca loko cā’’ti? Yañca kho te evamāhaṃsu – ‘idameva saccaṃ moghamañña’nti. Taṃ tesaṃ nānujānāmi. Taṃ kissa hetu? Aññathāsaññinopi hettha, cunda, santeke sattā. Imāyapi kho ahaṃ, cunda, paññattiyā neva attanā samasamaṃ samanupassāmi kuto bhiyyo. Atha kho ahameva tattha bhiyyo yadidaṃ adhipaññatti.

\paragraph{193.} ‘‘Tatra, cunda, ye te samaṇabrāhmaṇā evaṃvādino evaṃdiṭṭhino – ‘asassato attā ca loko ca…pe… sassato ca asassato ca attā ca loko ca… nevasassato nāsassato attā ca loko ca… sayaṃkato attā ca loko ca… paraṃkato attā ca loko ca… sayaṃkato ca paraṃkato ca attā ca loko ca… asayaṃkāro aparaṃkāro adhiccasamuppanno attā ca loko ca… sassataṃ sukhadukkhaṃ… asassataṃ sukhadukkhaṃ… sassatañca asassatañca sukhadukkhaṃ… nevasassataṃ nāsassataṃ sukhadukkhaṃ… sayaṃkataṃ sukhadukkhaṃ… paraṃkataṃ sukhadukkhaṃ… sayaṃkatañca paraṃkatañca sukhadukkhaṃ… asayaṃkāraṃ aparaṃkāraṃ adhiccasamuppannaṃ sukhadukkhaṃ, idameva saccaṃ moghamañña’nti. Tyāhaṃ upasaṅkamitvā evaṃ vadāmi – ‘atthi nu kho idaṃ, āvuso, vuccati – ‘‘asayaṃkāraṃ aparaṃkāraṃ adhiccasamuppannaṃ sukhadukkha’’’nti? Yañca kho te evamāhaṃsu – ‘idameva saccaṃ moghamañña’nti. Taṃ tesaṃ nānujānāmi. Taṃ kissa hetu? Aññathāsaññinopi hettha, cunda, santeke sattā. Imāyapi kho ahaṃ, cunda, paññattiyā neva attanā samasamaṃ samanupassāmi kuto bhiyyo. Atha kho ahameva tattha bhiyyo yadidaṃ adhipaññatti. Ime kho te, cunda, pubbantasahagatā diṭṭhinissayā, ye vo mayā byākatā, yathā te byākātabbā . Yathā ca te na byākātabbā, kiṃ vo ahaṃ te tathā byākarissāmīti\footnote{byākarissāmīti (sī. ka.)}?

\subsubsection{Aparantasahagatadiṭṭhinissayā}

\paragraph{194.} ‘‘Katame ca te, cunda, aparantasahagatā diṭṭhinissayā, ye vo mayā byākatā, yathā te byākātabbā. (Yathā ca te na byākātabbā, kiṃ vo ahaṃ te tathā byākarissāmī)\footnote{( ) etthantare pāṭho sabbatthapi paripuṇṇo dissati}? Santi, cunda, eke samaṇabrāhmaṇā evaṃvādino evaṃdiṭṭhino – ‘rūpī attā hoti arogo paraṃ maraṇā, idameva saccaṃ moghamañña’nti. Santi pana, cunda, eke samaṇabrāhmaṇā evaṃvādino evaṃdiṭṭhino – ‘arūpī attā hoti…pe… rūpī ca arūpī ca attā hoti… nevarūpī nārūpī attā hoti… saññī attā hoti… asaññī attā hoti… nevasaññīnāsaññī attā hoti… attā ucchijjati vinassati na hoti paraṃ maraṇā, idameva saccaṃ moghamañña’nti. Tatra, cunda, ye te samaṇabrāhmaṇā evaṃvādino evaṃdiṭṭhino – ‘rūpī attā hoti arogo paraṃ maraṇā, idameva saccaṃ moghamañña’nti. Tyāhaṃ upasaṅkamitvā evaṃ vadāmi – ‘atthi nu kho idaṃ, āvuso, vuccati – ‘‘rūpī attā hoti arogo paraṃ maraṇā’’’ti? Yañca kho te evamāhaṃsu – ‘idameva saccaṃ moghamañña’nti. Taṃ tesaṃ nānujānāmi. Taṃ kissa hetu? Aññathāsaññinopi hettha, cunda, santeke sattā. Imāyapi kho ahaṃ, cunda, paññattiyā neva attanā samasamaṃ samanupassāmi kuto bhiyyo. Atha kho ahameva tattha bhiyyo yadidaṃ adhipaññatti.

\paragraph{195.} ‘‘Tatra, cunda, ye te samaṇabrāhmaṇā evaṃvādino evaṃdiṭṭhino – ‘arūpī attā hoti…pe… rūpī ca arūpī ca attā hoti… nevarūpīnārūpī attā hoti… saññī attā hoti… asaññī attā hoti… nevasaññīnāsaññī attā hoti… attā ucchijjati vinassati na hoti paraṃ maraṇā, idameva saccaṃ moghamañña’nti. Tyāhaṃ upasaṅkamitvā evaṃ vadāmi – ‘atthi nu kho idaṃ, āvuso, vuccati – ‘‘attā ucchijjati vinassati na hoti paraṃ maraṇā’’’ti? Yañca kho te, cunda, evamāhaṃsu – ‘idameva saccaṃ moghamañña’nti. Taṃ tesaṃ nānujānāmi. Taṃ kissa hetu? Aññathāsaññinopi hettha, cunda, santeke sattā. Imāyapi kho ahaṃ, cunda, paññattiyā neva attanā samasamaṃ samanupassāmi, kuto bhiyyo. Atha kho ahameva tattha bhiyyo yadidaṃ adhipaññatti. Ime kho te, cunda, aparantasahagatā diṭṭhinissayā, ye vo mayā byākatā , yathā te byākātabbā. Yathā ca te na byākātabbā, kiṃ vo ahaṃ te tathā byākarissāmīti\footnote{byākarissāmīti (sī. ka.)}?

\paragraph{196.} ‘‘Imesañca, cunda, pubbantasahagatānaṃ diṭṭhinissayānaṃ imesañca aparantasahagatānaṃ diṭṭhinissayānaṃ pahānāya samatikkamāya evaṃ mayā cattāro satipaṭṭhānā desitā paññattā. Katame cattāro? Idha, cunda, bhikkhu kāye kāyānupassī viharati ātāpī sampajāno satimā vineyya loke abhijjhādomanassaṃ. Vedanāsu vedanānupassī…pe… citte cittānupassī… dhammesu dhammānupassī viharati ātāpī sampajāno satimā, vineyya loke abhijjhādomanassaṃ. Imesañca cunda, pubbantasahagatānaṃ diṭṭhinissayānaṃ imesañca aparantasahagatānaṃ diṭṭhinissayānaṃ pahānāya samatikkamāya. Evaṃ mayā ime cattāro satipaṭṭhānā desitā paññattā’’ti.

\paragraph{197.} Tena kho pana samayena āyasmā upavāṇo bhagavato piṭṭhito ṭhito hoti bhagavantaṃ bījayamāno. Atha kho āyasmā upavāṇo bhagavantaṃ etadavoca – ‘‘acchariyaṃ, bhante, abbhutaṃ, bhante! Pāsādiko vatāyaṃ, bhante, dhammapariyāyo; supāsādiko vatāyaṃ bhante, dhammapariyāyo, ko nāmāyaṃ bhante dhammapariyāyo’’ti? ‘‘Tasmātiha tvaṃ, upavāṇa, imaṃ dhammapariyāyaṃ ‘pāsādiko’ tveva naṃ dhārehī’’ti. Idamavoca bhagavā. Attamano āyasmā upavāṇo bhagavato bhāsitaṃ abhinandīti.

\xsectionEnd{Pāsādikasuttaṃ niṭṭhitaṃ chaṭṭhaṃ.}


\section{Lakkhaṇasuttaṃ}

\subsubsection{Dvattiṃsamahāpurisalakkhaṇāni}

\paragraph{198.} Evaṃ me sutaṃ – ekaṃ samayaṃ bhagavā sāvatthiyaṃ viharati jetavane anāthapiṇḍikassa ārāme. Tatra kho bhagavā bhikkhū āmantesi – ‘‘bhikkhavo’’ti. ‘‘Bhaddante’’ti\footnote{bhadanteti (sī. syā. pī.)} te bhikkhū bhagavato paccassosuṃ. Bhagavā etadavoca –

\paragraph{199.} ‘‘Dvattiṃsimāni, bhikkhave, mahāpurisassa mahāpurisalakkhaṇāni, yehi samannāgatassa mahāpurisassa dveva gatiyo bhavanti anaññā. Sace agāraṃ ajjhāvasati, rājā hoti cakkavattī dhammiko dhammarājā cāturanto vijitāvī janapadatthāvariyappatto sattaratanasamannāgato. Tassimāni satta ratanāni bhavanti; seyyathidaṃ, cakkaratanaṃ hatthiratanaṃ assaratanaṃ maṇiratanaṃ itthiratanaṃ gahapatiratanaṃ pariṇāyakaratanameva sattamaṃ. Parosahassaṃ kho panassa puttā bhavanti sūrā vīraṅgarūpā parasenappamaddanā. So imaṃ pathaviṃ sāgarapariyantaṃ adaṇḍena asatthena dhammena abhivijiya ajjhāvasati. Sace kho pana agārasmā anagāriyaṃ pabbajati, arahaṃ hoti sammāsambuddho loke vivaṭṭacchado\footnote{vivaṭacchado (syā. ka.), vivattacchado (sī. pī.)}.

\paragraph{200.} ‘‘Katamāni ca tāni, bhikkhave, dvattiṃsa mahāpurisassa mahāpurisalakkhaṇāni, yehi samannāgatassa mahāpurisassa dveva gatiyo bhavanti anaññā? Sace agāraṃ ajjhāvasati, rājā hoti cakkavattī…pe… sace kho pana agārasmā anagāriyaṃ pabbajati, arahaṃ hoti sammāsambuddho loke vivaṭṭacchado.

‘‘Idha, bhikkhave, mahāpuriso suppatiṭṭhitapādo hoti. Yampi, bhikkhave, mahāpuriso suppatiṭṭhitapādo hoti, idampi, bhikkhave, mahāpurisassa mahāpurisalakkhaṇaṃ bhavati.

‘‘Puna caparaṃ, bhikkhave, mahāpurisassa heṭṭhāpādatalesu cakkāni jātāni honti sahassārāni sanemikāni sanābhikāni sabbākāraparipūrāni\footnote{sabbākāraparipūrāni suvibhattantarāni (sī. pī.)}. Yampi , bhikkhave, mahāpurisassa heṭṭhāpādatalesu cakkāni jātāni honti sahassārāni sanemikāni sanābhikāni sabbākāraparipūrāni, idampi, bhikkhave, mahāpurisassa mahāpurisalakkhaṇaṃ bhavati.

‘‘Puna caparaṃ, bhikkhave, mahāpuriso āyatapaṇhi hoti…pe… dīghaṅguli hoti… mudutalunahatthapādo hoti… jālahatthapādo hoti… ussaṅkhapādo hoti… eṇijaṅgho hoti… ṭhitakova anonamanto ubhohi pāṇitalehi jaṇṇukāni parimasati parimajjati… kosohitavatthaguyho hoti… suvaṇṇavaṇṇo hoti kañcanasannibhattaco… sukhumacchavi hoti, sukhumattā chaviyā rajojallaṃ kāye na upalimpati… ekekalomo hoti, ekekāni lomāni lomakūpesu jātāni… uddhaggalomo hoti, uddhaggāni lomāni jātāni nīlāni añjanavaṇṇāni kuṇḍalāvaṭṭāni\footnote{kuṇḍalāvattāni (bahūsu)} dakkhiṇāvaṭṭakajātāni\footnote{dakkhiṇāvattakajātāni (sī. syā. pī.)} … brahmujugatto hoti… sattussado hoti… sīhapubbaddhakāyo hoti… citantaraṃso\footnote{pitantaraṃso (syā.)} hoti… nigrodhaparimaṇḍalo hoti, yāvatakvassa kāyo tāvatakvassa byāmo yāvatakvassa byāmo tāvatakvassa kāyo… samavaṭṭakkhandho hoti… rasaggasaggī hoti… sīhahanu hoti… cattālīsadanto hoti … samadanto hoti… aviraḷadanto hoti… susukkadāṭho hoti… pahūtajivho hoti… brahmassaro hoti karavīkabhāṇī… abhinīlanetto hoti… gopakhumo hoti… uṇṇā bhamukantare jātā hoti, odātā mudutūlasannibhā. Yampi, bhikkhave, mahāpurisassa uṇṇā bhamukantare jātā hoti, odātā mudutūlasannibhā, idampi, bhikkhave, mahāpurisassa mahāpurisalakkhaṇaṃ bhavati.

‘‘Puna caparaṃ, bhikkhave, mahāpuriso uṇhīsasīso hoti. Yampi, bhikkhave, mahāpuriso uṇhīsasīso hoti, idampi, bhikkhave, mahāpurisassa mahāpurisalakkhaṇaṃ bhavati.

‘‘Imāni kho tāni, bhikkhave, dvattiṃsa mahāpurisassa mahāpurisalakkhaṇāni, yehi samannāgatassa mahāpurisassa dveva gatiyo bhavanti anaññā. Sace agāraṃ ajjhāvasati, rājā hoti cakkavattī…pe… sace kho pana agārasmā anagāriyaṃ pabbajati, arahaṃ hoti sammāsambuddho loke vivaṭṭacchado.

‘‘Imāni kho, bhikkhave, dvattiṃsa mahāpurisassa mahāpurisalakkhaṇāni bāhirakāpi isayo dhārenti, no ca kho te jānanti – ‘imassa kammassa kaṭattā idaṃ lakkhaṇaṃ paṭilabhatī’ti.

\subsubsection{(1) Suppatiṭṭhitapādatālakkhaṇaṃ}

\paragraph{201.} ‘‘Yampi, bhikkhave, tathāgato purimaṃ jātiṃ purimaṃ bhavaṃ purimaṃ niketaṃ pubbe manussabhūto samāno daḷhasamādāno ahosi kusalesu dhammesu, avatthitasamādāno kāyasucarite vacīsucarite manosucarite dānasaṃvibhāge sīlasamādāne uposathupavāse matteyyatāya petteyyatāya sāmaññatāya brahmaññatāya kule jeṭṭhāpacāyitāya aññataraññataresu ca adhikusalesu dhammesu . So tassa kammassa kaṭattā upacitattā ussannattā vipulattā kāyassa bhedā paraṃ maraṇā sugatiṃ saggaṃ lokaṃ upapajjati. So tattha aññe deve dasahi ṭhānehi adhiggaṇhāti dibbena āyunā dibbena vaṇṇena dibbena sukhena dibbena yasena dibbena ādhipateyyena dibbehi rūpehi dibbehi saddehi dibbehi gandhehi dibbehi rasehi dibbehi phoṭṭhabbehi. So tato cuto itthattaṃ āgato samāno imaṃ mahāpurisalakkhaṇaṃ paṭilabhati. Suppatiṭṭhitapādo hoti. Samaṃ pādaṃ bhūmiyaṃ nikkhipati, samaṃ uddharati, samaṃ sabbāvantehi pādatalehi bhūmiṃ phusati.

\paragraph{202.} ‘‘So tena lakkhaṇena samannāgato sace agāraṃ ajjhāvasati, rājā hoti cakkavattī dhammiko dhammarājā cāturanto vijitāvī janapadatthāvariyappatto sattaratanasamannāgato. Tassimāni satta ratanāni bhavanti; seyyathidaṃ, cakkaratanaṃ hatthiratanaṃ assaratanaṃ maṇiratanaṃ itthiratanaṃ gahapatiratanaṃ pariṇāyakaratanameva sattamaṃ. Parosahassaṃ kho panassa puttā bhavanti sūrā vīraṅgarūpā parasenappamaddanā. So imaṃ pathaviṃ sāgarapariyantaṃ akhilamanimittamakaṇṭakaṃ iddhaṃ phītaṃ khemaṃ sivaṃ nirabbudaṃ adaṇḍena asatthena dhammena abhivijiya ajjhāvasati . Rājā samāno kiṃ labhati? Akkhambhiyo\footnote{avikkhambhiyo (sī. pī.)} hoti kenaci manussabhūtena paccatthikena paccāmittena. Rājā samāno idaṃ labhati. ‘‘Sace kho pana agārasmā anagāriyaṃ pabbajati, arahaṃ hoti sammāsambuddho loke vivaṭṭacchado. Buddho samāno kiṃ labhati? Akkhambhiyo hoti abbhantarehi vā bāhirehi vā paccatthikehi paccāmittehi rāgena vā dosena vā mohena vā samaṇena vā brāhmaṇena vā devena vā mārena vā brahmunā vā kenaci vā lokasmiṃ. Buddho samāno idaṃ labhati’’. Etamatthaṃ bhagavā avoca.

\paragraph{203.} Tatthetaṃ vuccati –

‘‘Sacce ca dhamme ca dame ca saṃyame,

Soceyyasīlālayuposathesu ca;

Dāne ahiṃsāya asāhase rato,

Daḷhaṃ samādāya samattamācari\footnote{samantamācari (syā. ka.)}.

‘‘So tena kammena divaṃ samakkami\footnote{apakkami (syā. ka.)},

Sukhañca khiḍḍāratiyo ca anvabhi\footnote{aṃnvabhi (ṭīkā)};

Tato cavitvā punarāgato idha,

Samehi pādehi phusī vasundharaṃ.

‘‘Byākaṃsu veyyañjanikā samāgatā,

Samappatiṭṭhassa na hoti khambhanā;

Gihissa vā pabbajitassa vā puna\footnote{pana (syā.)},

Taṃ lakkhaṇaṃ bhavati tadatthajotakaṃ.

‘‘Akkhambhiyo hoti agāramāvasaṃ,

Parābhibhū sattubhi nappamaddano;

Manussabhūtenidha hoti kenaci,

Akkhambhiyo tassa phalena kammuno.

‘‘Sace ca pabbajjamupeti tādiso,

Nekkhammachandābhirato vicakkhaṇo;

Aggo na so gacchati jātu khambhanaṃ,

Naruttamo esa hi tassa dhammatā’’ti.

\subsubsection{(2) Pādatalacakkalakkhaṇaṃ}

\paragraph{204.} ‘‘Yampi, bhikkhave, tathāgato purimaṃ jātiṃ purimaṃ bhavaṃ purimaṃ niketaṃ pubbe manussabhūto samāno bahujanassa sukhāvaho ahosi, ubbegauttāsabhayaṃ apanuditā, dhammikañca rakkhāvaraṇaguttiṃ saṃvidhātā, saparivārañca dānaṃ adāsi. So tassa kammassa kaṭattā upacitattā ussannattā vipulattā kāyassa bhedā paraṃ maraṇā sugatiṃ saggaṃ lokaṃ upapajjati…pe… so tato cuto itthattaṃ āgato samāno imaṃ mahāpurisalakkhaṇaṃ paṭilabhati. Heṭṭhāpādatalesu cakkāni jātāni honti sahassārāni sanemikāni sanābhikāni sabbākāraparipūrāni suvibhattantarāni.

‘‘So tena lakkhaṇena samannāgato sace agāraṃ ajjhāvasati, rājā hoti cakkavattī…pe… rājā samāno kiṃ labhati? Mahāparivāro hoti; mahāssa honti parivārā brāhmaṇagahapatikā negamajānapadā gaṇakamahāmattā anīkaṭṭhā dovārikā amaccā pārisajjā rājāno bhogiyā kumārā. Rājā samāno idaṃ labhati. Sace kho pana agārasmā anagāriyaṃ pabbajati, arahaṃ hoti sammāsambuddho loke vivaṭṭacchado. Buddho samāno kiṃ labhati? Mahāparivāro hoti; mahāssa honti parivārā bhikkhū bhikkhuniyo upāsakā upāsikāyo devā manussā asurā nāgā gandhabbā. Buddho samāno idaṃ labhati’’. Etamatthaṃ bhagavā avoca.

\paragraph{205.} Tatthetaṃ vuccati –

‘‘Pure puratthā purimāsu jātisu,

Manussabhūto bahunaṃ sukhāvaho;

Ubbhegauttāsabhayāpanūdano,

Guttīsu rakkhāvaraṇesu ussuko.

‘‘So tena kammena divaṃ samakkami,

Sukhañca khiḍḍāratiyo ca anvabhi;

Tato cavitvā punarāgato idha,

Cakkāni pādesu duvesu vindati.

‘‘Samantanemīni sahassarāni ca,

Byākaṃsu veyyañjanikā samāgatā;

Disvā kumāraṃ satapuññalakkhaṇaṃ,

Parivāravā hessati sattumaddano.

Tathā hī cakkāni samantanemini,

Sace na pabbajjamupeti tādiso;

Vatteti cakkaṃ pathaviṃ pasāsati,

Tassānuyantādha\footnote{tassānuyuttā idha (sī. pī.), tassānuyantā idha (syā. ka.)} bhavanti khattiyā.

‘‘Mahāyasaṃ saṃparivārayanti naṃ,

Sace ca pabbajjamupeti tādiso;

Nekkhammachandābhirato vicakkhaṇo,

Devāmanussāsurasakkarakkhasā\footnote{sattarakkhasā (ka.) sī. syāaṭṭhakathā oloketabbā}.

‘‘Gandhabbanāgā vihagā catuppadā,

Anuttaraṃ devamanussapūjitaṃ;

Mahāyasaṃ saṃparivārayanti na’’nti.

\subsubsection{(3-5) Āyatapaṇhitāditilakkhaṇaṃ}

\paragraph{206.} ‘‘Yampi, bhikkhave, tathāgato purimaṃ jātiṃ purimaṃ bhavaṃ purimaṃ niketaṃ pubbe manussabhūto samāno pāṇātipātaṃ pahāya pāṇātipātā paṭivirato ahosi nihitadaṇḍo nihitasattho lajjī dayāpanno, sabbapāṇabhūtahitānukampī vihāsi. So tassa kammassa kaṭattā upacitattā ussannattā vipulattā…pe… so tato cuto itthattaṃ āgato samāno imāni tīṇi mahāpurisalakkhaṇāni paṭilabhati. Āyatapaṇhi ca hoti, dīghaṅguli ca brahmujugatto ca.

‘‘So tehi lakkhaṇehi samannāgato sace agāraṃ ajjhāvasati, rājā hoti cakkavattī…pe… rājā samāno kiṃ labhati? Dīghāyuko hoti ciraṭṭhitiko, dīghamāyuṃ pāleti, na sakkā hoti antarā jīvitā voropetuṃ kenaci manussabhūtena paccatthikena paccāmittena . Rājā samāno idaṃ labhati… buddho samāno kiṃ labhati? Dīghāyuko hoti ciraṭṭhitiko, dīghamāyuṃ pāleti, na sakkā hoti antarā jīvitā voropetuṃ paccatthikehi paccāmittehi samaṇena vā brāhmaṇena vā devena vā mārena vā brahmunā vā kenaci vā lokasmiṃ. Buddho samāno idaṃ labhati’’. Etamatthaṃ bhagavā avoca.

\paragraph{207.} Tatthetaṃ vuccati –

‘‘Māraṇavadhabhayattano\footnote{maraṇavadhabhayattano (sī. pī. ka.), maraṇavadhabhayamattano (syā.)} viditvā,

Paṭivirato paraṃ māraṇāyahosi;

Tena sucaritena saggamagamā\footnote{tena so sucaritena saggamagamāsi (syā.)},

Sukataphalavipākamanubhosi.

‘‘Caviya punaridhāgato samāno,

Paṭilabhati idha tīṇi lakkhaṇāni;

Bhavati vipuladīghapāsaṇhiko,

Brahmāva suju subho sujātagatto.

‘‘Subhujo susu susaṇṭhito sujāto,

Mudutalunaṅguliyassa honti;

Dīghā tībhi purisavaraggalakkhaṇehi,

Cirayapanāya\footnote{cirayāpanāya (syā.)} kumāramādisanti.

‘‘Bhavati yadi gihī ciraṃ yapeti,

Cirataraṃ pabbajati yadi tato hi;

Yāpayati ca vasiddhibhāvanāya,

Iti dīghāyukatāya taṃ nimitta’’nti.

\subsubsection{(6) Sattussadatālakkhaṇaṃ}

\paragraph{208.} ‘‘Yampi , bhikkhave, tathāgato purimaṃ jātiṃ purimaṃ bhavaṃ purimaṃ niketaṃ pubbe manussabhūto samāno dātā ahosi paṇītānaṃ rasitānaṃ khādanīyānaṃ bhojanīyānaṃ sāyanīyānaṃ lehanīyānaṃ pānānaṃ. So tassa kammassa kaṭattā…pe… so tato cuto itthattaṃ āgato samāno imaṃ mahāpurisalakkhaṇaṃ paṭilabhati, sattussado hoti, sattassa ussadā honti; ubhosu hatthesu ussadā honti, ubhosu pādesu ussadā honti, ubhosu aṃsakūṭesu ussadā honti, khandhe ussado hoti.

‘‘So tena lakkhaṇena samannāgato sace agāraṃ ajjhāvasati, rājā hoti cakkavattī…pe… rājā samāno kiṃ labhati? Lābhī hoti paṇītānaṃ rasitānaṃ khādanīyānaṃ bhojanīyānaṃ sāyanīyānaṃ lehanīyānaṃ pānānaṃ. Rājā samāno idaṃ labhati… buddho samāno kiṃ labhati? Lābhī hoti paṇītānaṃ rasitānaṃ khādanīyānaṃ bhojanīyānaṃ sāyanīyānaṃ lehanīyānaṃ pānānaṃ. Buddho samāno idaṃ labhati’’. Etamatthaṃ bhagavā avoca.

\paragraph{209.} Tatthetaṃ vuccati –

‘‘Khajjabhojjamatha leyya sāyiyaṃ,

Uttamaggarasadāyako ahu;

Tena so sucaritena kammunā,

Nandane ciramabhippamodati.

‘‘Satta cussade idhādhigacchati,

Hatthapādamudutañca vindati;

Āhu byañjananimittakovidā,

Khajjabhojjarasalābhitāya naṃ.

‘‘Yaṃ gihissapi\footnote{na taṃ gihissāpi (syā.)} tadatthajotakaṃ,

Pabbajjampi ca tadādhigacchati;

Khajjabhojjarasalābhiruttamaṃ,

Āhu sabbagihibandhanacchida’’nti.

\subsubsection{(7-8) Karacaraṇamudujālatālakkhaṇāni}

\paragraph{210.} ‘‘Yampi , bhikkhave, tathāgato purimaṃ jātiṃ purimaṃ bhavaṃ purimaṃ niketaṃ pubbe manussabhūto samāno catūhi saṅgahavatthūhi janaṃ saṅgāhako ahosi – dānena peyyavajjena\footnote{piyavācena (syā. ka.)} atthacariyāya samānattatāya. So tassa kammassa kaṭattā…pe… so tato cuto itthattaṃ āgato samāno imāni dve mahāpurisalakkhaṇāni paṭilabhati. Mudutalunahatthapādo ca hoti jālahatthapādo ca.

‘‘So tehi lakkhaṇehi samannāgato sace agāraṃ ajjhāvasati, rājā hoti cakkavattī…pe… rājā samāno kiṃ labhati? Susaṅgahitaparijano hoti, susaṅgahitāssa honti brāhmaṇagahapatikā negamajānapadā gaṇakamahāmattā anīkaṭṭhā dovārikā amaccā pārisajjā rājāno bhogiyā kumārā. Rājā samāno idaṃ labhati… buddho samāno kiṃ labhati? Susaṅgahitaparijano hoti, susaṅgahitāssa honti bhikkhū bhikkhuniyo upāsakā upāsikāyo devā manussā asurā nāgā gandhabbā. Buddho samāno idaṃ labhati’’. Etamatthaṃ bhagavā avoca.

\paragraph{211.} Tatthetaṃ vuccati –

‘‘Dānampi catthacariyatañca\footnote{dānampi ca atthacariyatampi ca (sī. pī.)},

Piyavāditañca samānattatañca\footnote{piyavadatañca samānachandatañca (sī. pī.)};

Kariyacariyasusaṅgahaṃ bahūnaṃ,

Anavamatena guṇena yāti saggaṃ.

‘‘Caviya punaridhāgato samāno,

Karacaraṇamudutañca jālino ca;

Atirucirasuvaggudassaneyyaṃ,

Paṭilabhati daharo susu kumāro.

‘‘Bhavati parijanassavo vidheyyo,

Mahimaṃ āvasito susaṅgahito;

Piyavadū hitasukhataṃ jigīsamāno\footnote{jigiṃ samāno (sī. syā. pī.)},

Abhirucitāni guṇāni ācarati.

‘‘Yadi ca jahati sabbakāmabhogaṃ,

Kathayati dhammakathaṃ jino janassa;

Vacanapaṭikarassābhippasannā ,

Sutvāna dhammānudhammamācarantī’’ti.

\subsubsection{(9-10) Ussaṅkhapādauddhaggalomatālakkhaṇāni}

\paragraph{212.} ‘‘Yampi, bhikkhave, tathāgato purimaṃ jātiṃ purimaṃ bhavaṃ purimaṃ niketaṃ pubbe manussabhūto samāno\footnote{samāno bahuno janassa (sī. pī.)} atthūpasaṃhitaṃ dhammūpasaṃhitaṃ vācaṃ bhāsitā ahosi, bahujanaṃ nidaṃsesi, pāṇīnaṃ hitasukhāvaho dhammayāgī. So tassa kammassa kaṭattā…pe… so tato cuto itthattaṃ āgato samāno imāni dve mahāpurisalakkhaṇāni paṭilabhati. Ussaṅkhapādo ca hoti, uddhaggalomo ca.

‘‘So tehi lakkhaṇehi samannāgato, sace agāraṃ ajjhāvasati, rājā hoti cakkavattī…pe… rājā samāno kiṃ labhati? Aggo ca hoti seṭṭho ca pāmokkho ca uttamo ca pavaro ca kāmabhogīnaṃ. Rājā samāno idaṃ labhati… buddho samāno kiṃ labhati? Aggo ca hoti seṭṭho ca pāmokkho ca uttamo ca pavaro ca sabbasattānaṃ. Buddho samāno idaṃ labhati’’. Etamatthaṃ bhagavā avoca.

\paragraph{213.} Tatthetaṃ vuccati –

‘‘Atthadhammasahitaṃ\footnote{atthadhammasaṃhitaṃ (ka. sī. pī.), atthadhammupasaṃhitaṃ (ka.)} pure giraṃ,

Erayaṃ bahujanaṃ nidaṃsayi;

Pāṇinaṃ hitasukhāvaho ahu,

Dhammayāgamayajī\footnote{dhammayāgaṃ assaji (ka.)} amaccharī.

‘‘Tena so sucaritena kammunā,

Suggatiṃ vajati tattha modati;

Lakkhaṇāni ca duve idhāgato,

Uttamappamukhatāya\footnote{uttamasukhatāya (syā.), uttamapamukkhatāya (ka.)} vindati.

‘‘Ubbhamuppatitalomavā saso,

Pādagaṇṭhirahu sādhusaṇṭhitā;

Maṃsalohitācitā tacotthatā,

Uparicaraṇasobhanā\footnote{uparijānusobhanā (syā.), upari ca pana sobhanā (sī. pī.)} ahu.

‘‘Gehamāvasati ce tathāvidho,

Aggataṃ vajati kāmabhoginaṃ;

Tena uttaritaro na vijjati,

Jambudīpamabhibhuyya iriyati.

‘‘Pabbajampi ca anomanikkamo,

Aggataṃ vajati sabbapāṇinaṃ;

Tena uttaritaro na vijjati,

Sabbalokamabhibhuyya viharatī’’ti.

\subsubsection{(11) Eṇijaṅghalakkhaṇaṃ}

\paragraph{214.} ‘‘Yampi, bhikkhave, tathāgato purimaṃ jātiṃ purimaṃ bhavaṃ purimaṃ niketaṃ pubbe manussabhūto samāno sakkaccaṃ vācetā ahosi sippaṃ vā vijjaṃ vā caraṇaṃ vā kammaṃ vā – ‘kiṃ time khippaṃ vijāneyyuṃ, khippaṃ paṭipajjeyyuṃ, na ciraṃ kilisseyyu’’nti. So tassa kammassa kaṭattā…pe… so tato cuto itthattaṃ āgato samāno imaṃ mahāpurisalakkhaṇaṃ paṭilabhati. Eṇijaṅgho hoti.

‘‘So tena lakkhaṇena samannāgato sace agāraṃ ajjhāvasati, rājā hoti cakkavattī…pe… rājā samāno kiṃ labhati? Yāni tāni rājārahāni rājaṅgāni rājūpabhogāni rājānucchavikāni tāni khippaṃ paṭilabhati. Rājā samāno idaṃ labhati… buddho samāno kiṃ labhati? Yāni tāni samaṇārahāni samaṇaṅgāni samaṇūpabhogāni samaṇānucchavikāni, tāni khippaṃ paṭilabhati. Buddho samāno idaṃ labhati’’. Etamatthaṃ bhagavā avoca.

\paragraph{215.} Tatthetaṃ vuccati –

‘‘Sippesu vijjācaraṇesu kammesu\footnote{kammasu (sī. pī.)},

Kathaṃ vijāneyyuṃ\footnote{vijāneyya (sī. pī.), vijāneyyu (syā.)} lahunti icchati;

Yadūpaghātāya na hoti kassaci,

Vāceti khippaṃ na ciraṃ kilissati.

‘‘Taṃ kammaṃ katvā kusalaṃ sukhudrayaṃ\footnote{sukhindriyaṃ (ka.)},

Jaṅghā manuññā labhate susaṇṭhitā;

Vaṭṭā sujātā anupubbamuggatā,

Uddhaggalomā sukhumattacotthatā.

‘‘Eṇeyyajaṅghoti tamāhu puggalaṃ,

Sampattiyā khippamidhāhu\footnote{khippamidāhu (?)} lakkhaṇaṃ;

Gehānulomāni yadābhikaṅkhati,

Apabbajaṃ khippamidhādhigacchati\footnote{khippamidādhigacchati (?)}.

‘‘Sace ca pabbajjamupeti tādiso,

Nekkhammachandābhirato vicakkhaṇo;

Anucchavikassa yadānulomikaṃ,

Taṃ vindati khippamanomavikkamo\footnote{nikkamo (sī. syā. pī.)}’’ti.

\subsubsection{(12) Sukhumacchavilakkhaṇaṃ}

\paragraph{216.} ‘‘Yampi, bhikkhave, tathāgato purimaṃ jātiṃ purimaṃ bhavaṃ purimaṃ niketaṃ pubbe manussabhūto samāno samaṇaṃ vā brāhmaṇaṃ vā upasaṅkamitvā paripucchitā ahosi – ‘‘kiṃ, bhante, kusalaṃ, kiṃ akusalaṃ, kiṃ sāvajjaṃ, kiṃ anavajjaṃ, kiṃ sevitabbaṃ, kiṃ na sevitabbaṃ, kiṃ me karīyamānaṃ dīgharattaṃ ahitāya dukkhāya assa, kiṃ vā pana me karīyamānaṃ dīgharattaṃ hitāya sukhāya assā’’ti. So tassa kammassa kaṭattā…pe… so tato cuto itthattaṃ āgato samāno imaṃ mahāpurisalakkhaṇaṃ paṭilabhati. Sukhumacchavi hoti, sukhumattā chaviyā rajojallaṃ kāye na upalimpati.

‘‘So tena lakkhaṇena samannāgato sace agāraṃ ajjhāvasati, rājā hoti cakkavattī…pe… rājā samāno kiṃ labhati? Mahāpañño hoti, nāssa hoti koci paññāya sadiso vā seṭṭho vā kāmabhogīnaṃ. Rājā samāno idaṃ labhati… buddho samāno kiṃ labhati? Mahāpañño hoti puthupañño hāsapañño\footnote{hāsupañño (sī. pī.)} javanapañño tikkhapañño nibbedhikapañño, nāssa hoti koci paññāya sadiso vā seṭṭho vā sabbasattānaṃ. Buddho samāno idaṃ labhati’’. Etamatthaṃ bhagavā avoca.

\paragraph{217.} Tatthetaṃ vuccati –

‘‘Pure puratthā purimāsu jātisu,

Aññātukāmo paripucchitā ahu;

Sussūsitā pabbajitaṃ upāsitā,

Atthantaro atthakathaṃ nisāmayi.

‘‘Paññāpaṭilābhagatena\footnote{paññāpaṭilābhakatena (sī. pī.) ṭīkā oloketabbā} kammunā,

Manussabhūto sukhumacchavī ahu;

Byākaṃsu uppādanimittakovidā,

Sukhumāni atthāni avecca dakkhiti.

‘‘Sace na pabbajjamupeti tādiso,

Vatteti cakkaṃ pathaviṃ pasāsati;

Atthānusiṭṭhīsu pariggahesu ca,

Na tena seyyo sadiso ca vijjati.

‘‘Sace ca pabbajjamupeti tādiso,

Nekkhammachandābhirato vicakkhaṇo;

Paññāvisiṭṭhaṃ labhate anuttaraṃ,

Pappoti bodhiṃ varabhūrimedhaso’’ti.

\subsubsection{(13) Suvaṇṇavaṇṇalakkhaṇaṃ}

\paragraph{218.} ‘‘Yampi , bhikkhave, tathāgato purimaṃ jātiṃ purimaṃ bhavaṃ purimaṃ niketaṃ pubbe manussabhūto samāno akkodhano ahosi anupāyāsabahulo, bahumpi vutto samāno nābhisajji na kuppi na byāpajji na patitthīyi, na kopañca dosañca appaccayañca pātvākāsi. Dātā ca ahosi sukhumānaṃ mudukānaṃ attharaṇānaṃ pāvuraṇānaṃ\footnote{pāpuraṇānaṃ (sī. syā. pī.)} khomasukhumānaṃ kappāsikasukhumānaṃ koseyyasukhumānaṃ kambalasukhumānaṃ. So tassa kammassa kaṭattā upacitattā…pe… so tato cuto itthattaṃ āgato samāno imaṃ mahāpurisalakkhaṇaṃ paṭilabhati. Suvaṇṇavaṇṇo hoti kañcanasannibhattaco.

‘‘So tena lakkhaṇena samannāgato sace agāraṃ ajjhāvasati, rājā hoti cakkavattī…pe… rājā samāno kiṃ labhati? Lābhī hoti sukhumānaṃ mudukānaṃ attharaṇānaṃ pāvuraṇānaṃ khomasukhumānaṃ kappāsikasukhumānaṃ koseyyasukhumānaṃ kambalasukhumānaṃ. Rājā samāno idaṃ labhati… buddho samāno kiṃ labhati? Lābhī hoti sukhumānaṃ mudukānaṃ attharaṇānaṃ pāvuraṇānaṃ khomasukhumānaṃ kappāsikasukhumānaṃ koseyyasukhumānaṃ kambalasukhumānaṃ. Buddho samāno idaṃ labhati’’. Etamatthaṃ bhagavā avoca.

\paragraph{219.} Tatthetaṃ vuccati –

‘‘Akkodhañca adhiṭṭhahi adāsi\footnote{adāsi ca (sī. pī.)},

Dānañca vatthāni sukhumāni succhavīni;

Purimatarabhave ṭhito abhivissaji,

Mahimiva suro abhivassaṃ.

‘‘Taṃ katvāna ito cuto dibbaṃ,

Upapajji\footnote{upapajja (sī. pī.)} sukataphalavipākamanubhutvā;

Kanakatanusannibho idhābhibhavati,

Suravarataroriva indo.

‘‘Gehañcāvasati naro apabbajja,

Micchaṃ mahatimahiṃ anusāsati\footnote{pasāsati (syā.)};

Pasayha sahidha sattaratanaṃ\footnote{pasayha abhivasana-varataraṃ (sī. pī.)},

Paṭilabhati vimala\footnote{vipula (syā.), vipulaṃ (sī. pī.)} sukhumacchaviṃ suciñca.

‘‘Lābhī acchādanavatthamokkhapāvuraṇānaṃ,

Bhavati yadi anāgāriyataṃ upeti;

Sahito\footnote{suhita (syā.), sa hi (sī. pī.)} purimakataphalaṃ anubhavati,

Na bhavati katassa panāso’’ti.

\subsubsection{(14) Kosohitavatthaguyhalakkhaṇaṃ}

\paragraph{220.} Yampi, bhikkhave, tathāgato purimaṃ jātiṃ purimaṃ bhavaṃ purimaṃ niketaṃ pubbe manussabhūto samāno cirappanaṭṭhe sucirappavāsino ñātimitte suhajje sakhino samānetā ahosi. Mātarampi puttena samānetā ahosi, puttampi mātarā samānetā ahosi, pitarampi puttena samānetā ahosi, puttampi pitarā samānetā ahosi, bhātarampi bhātarā samānetā ahosi, bhātarampi bhaginiyā samānetā ahosi, bhaginimpi bhātarā samānetā ahosi, bhaginimpi bhaginiyā samānetā ahosi, samaṅgīkatvā\footnote{samaggiṃ katvā (sī. syā. pī.)} ca abbhanumoditā ahosi. So tassa kammassa kaṭattā…pe… so tato cuto itthattaṃ āgato samāno imaṃ mahāpurisalakkhaṇaṃ paṭilabhati – kosohitavatthaguyho hoti.

‘‘So tena lakkhaṇena samannāgato sace agāraṃ ajjhāvasati, rājā hoti cakkavattī…pe… rājā samāno kiṃ labhati? Pahūtaputto hoti, parosahassaṃ kho panassa puttā bhavanti sūrā vīraṅgarūpā parasenappamaddanā. Rājā samāno idaṃ labhati… buddho samāno kiṃ labhati? Pahūtaputto hoti, anekasahassaṃ kho panassa puttā bhavanti sūrā vīraṅgarūpā parasenappamaddanā. Buddho samāno idaṃ labhati’’. Etamatthaṃ bhagavā avoca.

\paragraph{221.} Tatthetaṃ vuccati –

‘‘Pure puratthā purimāsu jātisu,

Cirappanaṭṭhe sucirappavāsino;

Ñātī suhajje sakhino samānayi,

Samaṅgikatvā anumoditā ahu.

‘‘So tena\footnote{sa tena (ka.)} kammena divaṃ samakkami,

Sukhañca khiḍḍāratiyo ca anvabhi;

Tato cavitvā punarāgato idha,

Kosohitaṃ vindati vatthachādiyaṃ.

‘‘Pahūtaputto bhavatī tathāvidho,

Parosahassañca\footnote{parosahassassa (sī. pī.)} bhavanti atrajā;

Sūrā ca vīrā ca\footnote{sūrā ca vīraṅgarūpā (ka.)} amittatāpanā,

Gihissa pītiṃjananā piyaṃvadā.

‘‘Bahūtarā pabbajitassa iriyato,

Bhavanti puttā vacanānusārino;

Gihissa vā pabbajitassa vā puna,

Taṃ lakkhaṇaṃ jāyati tadatthajotaka’’nti.

\xsubsubsectionEnd{Paṭhamabhāṇavāro niṭṭhito.}

\subsubsection{(15-16) Parimaṇḍalaanonamajaṇṇuparimasanalakkhaṇāni}

\paragraph{222.} ‘‘Yampi , bhikkhave, tathāgato purimaṃ jātiṃ purimaṃ bhavaṃ purimaṃ niketaṃ pubbe manussabhūto samāno mahājanasaṅgahaṃ\footnote{mahājanasaṅgāhakaṃ (ka.)} samekkhamāno\footnote{samapekkhamāno (ka.)} samaṃ jānāti sāmaṃ jānāti, purisaṃ jānāti purisavisesaṃ jānāti – ‘ayamidamarahati ayamidamarahatī’ti tattha tattha purisavisesakaro ahosi. So tassa kammassa kaṭattā…pe… so tato cuto itthattaṃ āgato samāno imāni dve mahāpurisalakkhaṇāni paṭilabhati. Nigrodha parimaṇḍalo ca hoti, ṭhitakoyeva ca anonamanto ubhohi pāṇitalehi jaṇṇukāni parimasati parimajjati.

‘‘So tehi lakkhaṇehi samannāgato sace agāraṃ ajjhāvasati, rājā hoti cakkavattī…pe… rājā samāno kiṃ labhati ? Aḍḍho hoti mahaddhano mahābhogo pahūtajātarūparajato pahūtavittūpakaraṇo pahūtadhanadhañño paripuṇṇakosakoṭṭhāgāro. Rājā samāno idaṃ labhati…pe… buddho samāno kiṃ labhati? Aḍḍho hoti mahaddhano mahābhogo. Tassimāni dhanāni honti, seyyathidaṃ, saddhādhanaṃ sīladhanaṃ hiridhanaṃ ottappadhanaṃ sutadhanaṃ cāgadhanaṃ paññādhanaṃ. Buddho samāno idaṃ labhati’’. Etamatthaṃ bhagavā avoca.

\paragraph{223.} Tatthetaṃ vuccati –

‘‘Tuliya paṭivicaya cintayitvā,

Mahājanasaṅgahanaṃ\footnote{mahājanaṃ saṅgāhakaṃ (ka.)} samekkhamāno;

Ayamidamarahati tattha tattha,

Purisavisesakaro pure ahosi.

‘‘Mahiñca pana\footnote{samā ca pana (syā.), sa hi ca pana (sī. pī.)} ṭhito anonamanto,

Phusati karehi ubhohi jaṇṇukāni;

Mahiruhaparimaṇḍalo ahosi,

Sucaritakammavipākasesakena.

‘‘Bahuvividhanimittalakkhaṇaññū,

Atinipuṇā manujā byākariṃsu;

Bahuvividhā gihīnaṃ arahāni,

Paṭilabhati daharo susu kumāro.

‘‘Idha ca mahīpatissa kāmabhogī,

Gihipatirūpakā bahū bhavanti;

Yadi ca jahati sabbakāmabhogaṃ,

Labhati anuttaraṃ uttamadhanagga’’nti.

\subsubsection{(17-19) Sīhapubbaddhakāyāditilakkhaṇaṃ}

\paragraph{224.} ‘‘Yampi , bhikkhave, tathāgato purimaṃ jātiṃ purimaṃ bhavaṃ purimaṃ niketaṃ pubbe manussabhūto samāno bahujanassa atthakāmo ahosi hitakāmo phāsukāmo yogakkhemakāmo – ‘kintime saddhāya vaḍḍheyyuṃ, sīlena vaḍḍheyyuṃ, sutena vaḍḍheyyuṃ\footnote{sutena vaḍḍheyyuṃ, buddhiyā vaḍḍheyyuṃ (syā.)}, cāgena vaḍḍheyyuṃ, dhammena vaḍḍheyyuṃ, paññāya vaḍḍheyyuṃ, dhanadhaññena vaḍḍheyyuṃ, khettavatthunā vaḍḍheyyuṃ, dvipadacatuppadehi vaḍḍheyyuṃ, puttadārehi vaḍḍheyyuṃ, dāsakammakaraporisehi vaḍḍheyyuṃ, ñātīhi vaḍḍheyyuṃ, mittehi vaḍḍheyyuṃ, bandhavehi vaḍḍheyyu’nti. So tassa kammassa kaṭattā…pe… so tato cuto itthattaṃ āgato samāno imāni tīṇi mahāpurisalakkhaṇāni paṭilabhati. Sīhapubbaddhakāyo ca hoti citantaraṃso ca samavaṭṭakkhandho ca.

‘‘So tehi lakkhaṇehi samannāgato sace agāraṃ ajjhāvasati, rājā hoti cakkavattī…pe… rājā samāno kiṃ labhati? Aparihānadhammo hoti, na parihāyati dhanadhaññena khettavatthunā dvipadacatuppadehi puttadārehi dāsakammakaraporisehi ñātīhi mittehi bandhavehi, na parihāyati sabbasampattiyā. Rājā samāno idaṃ labhati… buddho samāno kiṃ labhati? Aparihānadhammo hoti, na parihāyati saddhāya sīlena sutena cāgena paññāya, na parihāyati sabbasampattiyā. Buddho samāno idaṃ labhati’’. Etamatthaṃ bhagavā avoca.

\paragraph{225.} Tatthetaṃ vuccati –

‘‘Saddhāya sīlena sutena buddhiyā,

Cāgena dhammena bahūhi sādhuhi;

Dhanena dhaññena ca khettavatthunā,

Puttehi dārehi catuppadehi ca.

‘‘Ñātīhi mittehi ca bandhavehi ca,

Balena vaṇṇena sukhena cūbhayaṃ;

Kathaṃ na hāyeyyuṃ pareti icchati,

Atthassa middhī ca\footnote{idaṃ samiddhañca (ka.), addhaṃ samiddhañca (syā.)} panābhikaṅkhati.

‘‘Sa sīhapubbaddhasusaṇṭhito ahu,

Samavaṭṭakkhandho ca citantaraṃso;

Pubbe suciṇṇena katena kammunā,

Ahāniyaṃ pubbanimittamassa taṃ.

‘‘Gihīpi dhaññena dhanena vaḍḍhati,

Puttehi dārehi catuppadehi ca;

Akiñcano pabbajito anuttaraṃ,

Pappoti bodhiṃ asahānadhammata’’nti\footnote{sambodhimahānadhammatanti (syā. ka.) ṭīkā oloketabbā}.

\subsubsection{(20) Rasaggasaggitālakkhaṇaṃ}

\paragraph{226.} ‘‘Yampi , bhikkhave, tathāgato purimaṃ jātiṃ purimaṃ bhavaṃ purimaṃ niketaṃ pubbe manussabhūto samāno sattānaṃ aviheṭhakajātiko ahosi pāṇinā vā leḍḍunā vā daṇḍena vā satthena vā. So tassa kammassa kaṭattā upacitattā…pe… so tato cuto itthattaṃ āgato samāno imaṃ mahāpurisalakkhaṇaṃ paṭilabhati, rasaggasaggī hoti, uddhaggāssa rasaharaṇīyo gīvāya jātā honti samābhivāhiniyo\footnote{samavāharasaharaṇiyo (syā.)}.

‘‘So tena lakkhaṇena samannāgato sace agāraṃ ajjhāvasati, rājā hoti cakkavattī…pe… rājā samāno kiṃ labhati? Appābādho hoti appātaṅko, samavepākiniyā gahaṇiyā samannāgato nātisītāya nāccuṇhāya. Rājā samāno idaṃ labhati… buddho samāno kiṃ labhati? Appābādho hoti appātaṅko samavepākiniyā gahaṇiyā samannāgato nātisītāya nāccuṇhāya majjhimāya padhānakkhamāya. Buddho samāno idaṃ labhati’’. Etamatthaṃ bhagavā avoca.

\paragraph{227.} Tatthetaṃ vuccati –

‘‘Na pāṇidaṇḍehi panātha leḍḍunā,

Satthena vā maraṇavadhena\footnote{māraṇavadhena (ka.)} vā pana;

Ubbādhanāya paritajjanāya vā,

Na heṭhayī janatamaheṭhako ahu.

‘‘Teneva so sugatimupecca modati,

Sukhapphalaṃ kariya sukhāni vindati;

Samojasā\footnote{sampajjasā (sī. pī.), pāmuñjasā (syā.), sāmañca sā (ka.)} rasaharaṇī susaṇṭhitā,

Idhāgato labhati rasaggasaggitaṃ.

‘‘Tenāhu naṃ atinipuṇā vicakkhaṇā,

Ayaṃ naro sukhabahulo bhavissati;

Gihissa vā pabbajitassa vā puna\footnote{pana (syā.)},

Taṃ lakkhaṇaṃ bhavati tadatthajotaka’’nti.

\subsubsection{(21-22) Abhinīlanettagopakhumalakkhaṇāni}

\paragraph{228.} ‘‘Yampi, bhikkhave, tathāgato purimaṃ jātiṃ purimaṃ bhavaṃ purimaṃ niketaṃ pubbe manussabhūto samāno na ca visaṭaṃ, na ca visāci\footnote{na ca visācitaṃ (sī. pī.), na ca visāvi (syā.)}, na ca pana viceyya pekkhitā, ujuṃ tathā pasaṭamujumano, piyacakkhunā bahujanaṃ udikkhitā ahosi. So tassa kammassa kaṭattā…pe… so tato cuto itthattaṃ āgato samāno imāni dve mahāpurisalakkhaṇāni paṭilabhati. Abhinīlanetto ca hoti gopakhumo ca.

‘‘So tehi lakkhaṇehi samannāgato, sace agāraṃ ajjhāvasati, rājā hoti cakkavattī…pe… rājā samāno kiṃ labhati? Piyadassano hoti bahuno janassa, piyo hoti manāpo brāhmaṇagahapatikānaṃ negamajānapadānaṃ gaṇakamahāmattānaṃ anīkaṭṭhānaṃ dovārikānaṃ amaccānaṃ pārisajjānaṃ rājūnaṃ bhogiyānaṃ kumārānaṃ. Rājā samāno idaṃ labhati…pe… buddho samāno kiṃ labhati? Piyadassano hoti bahuno janassa, piyo hoti manāpo bhikkhūnaṃ bhikkhunīnaṃ upāsakānaṃ upāsikānaṃ devānaṃ manussānaṃ asurānaṃ nāgānaṃ gandhabbānaṃ. Buddho samāno idaṃ labhati’’. Etamatthaṃ bhagavā avoca.

\paragraph{229.} Tatthetaṃ vuccati –

‘‘Na ca visaṭaṃ na ca visāci\footnote{visācitaṃ (sī. pī.), visāvi (syā.)}, na ca pana viceyyapekkhitā;

Ujuṃ tathā pasaṭamujumano, piyacakkhunā bahujanaṃ udikkhitā.

‘‘Sugatīsu so phalavipākaṃ,

Anubhavati tattha modati;

Idha ca pana bhavati gopakhumo,

Abhinīlanettanayano sudassano.

‘‘Abhiyogino ca nipuṇā,

Bahū pana nimittakovidā;

Sukhumanayanakusalā manujā,

Piyadassanoti abhiniddisanti naṃ.

‘‘Piyadassano gihīpi santo ca,

Bhavati bahujanapiyāyito;

Yadi ca na bhavati gihī samaṇo hoti,

Piyo bahūnaṃ sokanāsano’’ti.

\subsubsection{(23) Uṇhīsasīsalakkhaṇaṃ}

\paragraph{230.} ‘‘Yampi, bhikkhave, tathāgato purimaṃ jātiṃ purimaṃ bhavaṃ purimaṃ niketaṃ pubbe manussabhūto samāno bahujanapubbaṅgamo ahosi kusalesu dhammesu bahujanapāmokkho kāyasucarite vacīsucarite manosucarite dānasaṃvibhāge sīlasamādāne uposathupavāse matteyyatāya petteyyatāya sāmaññatāya brahmaññatāya kule jeṭṭhāpacāyitāya aññataraññataresu ca adhikusalesu dhammesu. So tassa kammassa kaṭattā…pe… so tato cuto itthattaṃ āgato samāno imaṃ mahāpurisalakkhaṇaṃ paṭilabhati – uṇhīsasīso hoti.

‘‘So tena lakkhaṇena samannāgato sace agāraṃ ajjhāvasati, rājā hoti cakkavattī…pe… rājā samāno kiṃ labhati? Mahāssa jano anvāyiko hoti, brāhmaṇagahapatikā negamajānapadā gaṇakamahāmattā anīkaṭṭhā dovārikā amaccā pārisajjā rājāno bhogiyā kumārā. Rājā samāno idaṃ labhati… buddho samāno kiṃ labhati? Mahāssa jano anvāyiko hoti, bhikkhū bhikkhuniyo upāsakā upāsikāyo devā manussā asurā nāgā gandhabbā. Buddho samāno idaṃ labhati’’. Etamatthaṃ bhagavā avoca.

\paragraph{231.} Tatthetaṃ vuccati –

‘‘Pubbaṅgamo sucaritesu ahu,

Dhammesu dhammacariyābhirato;

Anvāyiko bahujanassa ahu,

Saggesu vedayittha puññaphalaṃ.

‘‘Veditvā so sucaritassa phalaṃ,

Uṇhīsasīsattamidhajjhagamā;

Byākaṃsu byañjananimittadharā,

Pubbaṅgamo bahujanaṃ hessati.

‘‘Paṭibhogiyā manujesu idha,

Pubbeva tassa abhiharanti tadā;

Yadi khattiyo bhavati bhūmipati,

Paṭihārakaṃ bahujane labhati.

‘‘Atha cepi pabbajati so manujo,

Dhammesu hoti paguṇo visavī;

Tassānusāsaniguṇābhirato,

Anvāyiko bahujano bhavatī’’ti.

\subsubsection{(24-25) Ekekalomatāuṇṇālakkhaṇāni}

\paragraph{232.} ‘‘Yampi, bhikkhave, tathāgato purimaṃ jātiṃ purimaṃ bhavaṃ purimaṃ niketaṃ pubbe manussabhūto samāno musāvādaṃ pahāya musāvādā paṭivirato ahosi, saccavādī saccasandho theto paccayiko avisaṃvādako lokassa . So tassa kammassa kaṭattā upacitattā…pe… so tato cuto itthattaṃ āgato samāno imāni dve mahāpurisalakkhaṇāni paṭilabhati. Ekekalomo ca hoti, uṇṇā ca bhamukantare jātā hoti odātā mudutūlasannibhā.

‘‘So tehi lakkhaṇehi samannāgato, sace agāraṃ ajjhāvasati, rājā hoti cakkavattī…pe… rājā samāno kiṃ labhati? Mahāssa jano upavattati, brāhmaṇagahapatikā negamajānapadā gaṇakamahāmattā anīkaṭṭhā dovārikā amaccā pārisajjā rājāno bhogiyā kumārā. Rājā samāno idaṃ labhati… buddho samāno kiṃ labhati? Mahāssa jano upavattati, bhikkhū bhikkhuniyo upāsakā upāsikāyo devā manussā asurā nāgā gandhabbā. Buddho samāno idaṃ labhati’’. Etamatthaṃ bhagavā avoca.

\paragraph{233.} Tatthetaṃ vuccati –

‘‘Saccappaṭiñño purimāsu jātisu,

Advejjhavāco alikaṃ vivajjayi;

Na so visaṃvādayitāpi kassaci,

Bhūtena tacchena tathena bhāsayi\footnote{tosayi (sī. pī.)}.

‘‘Setā susukkā mudutūlasannibhā,

Uṇṇā sujātā\footnote{uṇṇāssa jātā (ka. sī.)} bhamukantare ahu;

Na lomakūpesu duve ajāyisuṃ,

Ekekalomūpacitaṅgavā ahu.

‘‘Taṃ lakkhaṇaññū bahavo samāgatā,

Byākaṃsu uppādanimittakovidā;

Uṇṇā ca lomā ca yathā susaṇṭhitā,

Upavattatī īdisakaṃ bahujjano.

‘‘Gihimpi santaṃ upavattatī jano,

Bahu puratthāpakatena kammunā;

Akiñcanaṃ pabbajitaṃ anuttaraṃ,

Buddhampi santaṃ upavattati jano’’ti.

\subsubsection{(26-27) Cattālīsaaviraḷadantalakkhaṇāni}

\paragraph{234.} ‘‘Yampi, bhikkhave tathāgato purimaṃ jātiṃ purimaṃ bhavaṃ purimaṃ niketaṃ pubbe manussabhūto samāno pisuṇaṃ vācaṃ pahāya pisuṇāya vācāya paṭivirato ahosi. Ito sutvā na amutra akkhātā imesaṃ bhedāya, amutra vā sutvā na imesaṃ akkhātā amūsaṃ bhedāya , iti bhinnānaṃ vā sandhātā, sahitānaṃ vā anuppadātā, samaggārāmo samaggarato samagganandī samaggakaraṇiṃ vācaṃ bhāsitā ahosi. So tassa kammassa kaṭattā…pe… so tato cuto itthattaṃ āgato samāno imāni dve mahāpurisalakkhaṇāni paṭilabhati. Cattālīsadanto ca hoti aviraḷadanto ca.

‘‘So tehi lakkhaṇehi samannāgato sace agāraṃ ajjhāvasati, rājā hoti cakkavattī…pe… rājā samāno kiṃ labhati? Abhejjapariso hoti, abhejjāssa honti parisā, brāhmaṇagahapatikā negamajānapadā gaṇakamahāmattā anīkaṭṭhā dovārikā amaccā pārisajjā rājāno bhogiyā kumārā. Rājā samāno idaṃ labhati … buddho samāno kiṃ labhati? Abhejjapariso hoti, abhejjāssa honti parisā, bhikkhū bhikkhuniyo upāsakā upāsikāyo devā manussā asurā nāgā gandhabbā. Buddho samāno idaṃ labhati’’. Etamatthaṃ bhagavā avoca.

\paragraph{235.} Tatthetaṃ vuccati –

‘‘Vebhūtiyaṃ sahitabhedakāriṃ,

Bhedappavaḍḍhanavivādakāriṃ;

Kalahappavaḍḍhanaākiccakāriṃ,

Sahitānaṃ bhedajananiṃ na bhaṇi.

‘‘Avivādavaḍḍhanakariṃ sugiraṃ,

Bhinnānusandhijananiṃ abhaṇi;

Kalahaṃ janassa panudī samaṅgī,

Sahitehi nandati pamodati ca.

‘‘Sugatīsu so phalavipākaṃ,

Anubhavati tattha modati;

Dantā idha honti aviraḷā sahitā,

Caturo dasassa mukhajā susaṇṭhitā.

‘‘Yadi khattiyo bhavati bhūmipati,

Avibhediyāssa parisā bhavati;

Samaṇo ca hoti virajo vimalo,

Parisāssa hoti anugatā acalā’’ti.

\subsubsection{(28-29) Pahūtajivhābrahmassaralakkhaṇāni}

\paragraph{236.} ‘‘Yampi , bhikkhave, tathāgato purimaṃ jātiṃ purimaṃ bhavaṃ purimaṃ niketaṃ pubbe manussabhūto samāno pharusaṃ vācaṃ pahāya pharusāya vācāya paṭivirato ahosi. Yā sā vācā nelā kaṇṇasukhā pemanīyā hadayaṅgamā porī bahujanakantā bahujanamanāpā, tathārūpiṃ vācaṃ bhāsitā ahosi. So tassa kammassa kaṭattā upacitattā…pe… so tato cuto itthattaṃ āgato samāno imāni dve mahāpurisalakkhaṇāni paṭilabhati. Pahūtajivho ca hoti brahmassaro ca karavīkabhāṇī.

‘‘So tehi lakkhaṇehi samannāgato sace agāraṃ ajjhāvasati, rājā hoti cakkavattī…pe… rājā samāno kiṃ labhati? Ādeyyavāco hoti, ādiyantissa vacanaṃ brāhmaṇagahapatikā negamajānapadā gaṇakamahāmattā anīkaṭṭhā dovārikā amaccā pārisajjā rājāno bhogiyā kumārā. Rājā samāno idaṃ labhati… buddho samāno kiṃ labhati? Ādeyyavāco hoti, ādiyantissa vacanaṃ bhikkhū bhikkhuniyo upāsakā upāsikāyo devā manussā asurā nāgā gandhabbā. Buddho samāno idaṃ labhati’’. Etamatthaṃ bhagavā avoca.

\paragraph{237.} Tatthetaṃ vuccati –

‘‘Akkosabhaṇḍanavihesakāriṃ,

Ubbādhikaṃ\footnote{ubbādhakaraṃ (syā.)} bahujanappamaddanaṃ;

Abāḷhaṃ giraṃ so na bhaṇi pharusaṃ,

Madhuraṃ bhaṇi susaṃhitaṃ\footnote{susahitaṃ (syā.)} sakhilaṃ.

‘‘Manaso piyā hadayagāminiyo,

Vācā so erayati kaṇṇasukhā;

Vācāsuciṇṇaphalamanubhavi,

Saggesu vedayatha\footnote{vedayati (?) ṭīkā oloketabbā} puññaphalaṃ.

‘‘Veditvā so sucaritassa phalaṃ,

Brahmassarattamidhamajjhagamā;

Jivhāssa hoti vipulā puthulā,

Ādeyyavākyavacano bhavati.

‘‘Gihinopi ijjhati yathā bhaṇato,

Atha ce pabbajati so manujo;

Ādiyantissa vacanaṃ janatā,

Bahuno bahuṃ subhaṇitaṃ bhaṇato’’ti.

\subsubsection{(30) Sīhahanulakkhaṇaṃ}

\paragraph{238.} ‘‘Yampi , bhikkhave, tathāgato purimaṃ jātiṃ purimaṃ bhavaṃ purimaṃ niketaṃ pubbe manussabhūto samāno samphappalāpaṃ pahāya samphappalāpā paṭivirato ahosi kālavādī bhūtavādī atthavādī dhammavādī vinayavādī, nidhānavatiṃ vācaṃ bhāsitā ahosi kālena sāpadesaṃ pariyantavatiṃ atthasaṃhitaṃ. So tassa kammassa kaṭattā…pe… so tato cuto itthattaṃ āgato samāno imaṃ mahāpurisalakkhaṇaṃ paṭilabhati, sīhahanu hoti.

‘‘So tena lakkhaṇena samannāgato sace agāraṃ ajjhāvasati, rājā hoti cakkavattī…pe… rājā samāno kiṃ labhati? Appadhaṃsiyo hoti kenaci manussabhūtena paccatthikena paccāmittena. Rājā samāno idaṃ labhati… buddho samāno kiṃ labhati? Appadhaṃsiyo hoti abbhantarehi vā bāhirehi vā paccatthikehi paccāmittehi, rāgena vā dosena vā mohena vā samaṇena vā brāhmaṇena vā devena vā mārena vā brahmunā vā kenaci vā lokasmiṃ. Buddho samāno idaṃ labhati’’. Etamatthaṃ bhagavā avoca.

\paragraph{239.} Tatthetaṃ vuccati –

‘‘Na samphappalāpaṃ na muddhataṃ\footnote{buddhatanti (ka.)},

Avikiṇṇavacanabyappatho ahosi;

Ahitamapi ca apanudi,

Hitamapi ca bahujanasukhañca abhaṇi.

‘‘Taṃ katvā ito cuto divamupapajji,

Sukataphalavipākamanubhosi;

Caviya punaridhāgato samāno,

Dvidugamavaratarahanuttamalattha.

‘‘Rājā hoti suduppadhaṃsiyo,

Manujindo manujādhipati mahānubhāvo;

Tidivapuravarasamo bhavati,

Suravarataroriva indo.

‘‘Gandhabbāsurayakkharakkhasebhi\footnote{surasakkarakkhasebhi (syā.)},

Surehi na hi bhavati suppadhaṃsiyo;

Tathatto yadi bhavati tathāvidho,

Idha disā ca paṭidisā ca vidisā cā’’ti.

\subsubsection{(31-32) Samadantasusukkadāṭhālakkhaṇāni}

\paragraph{240.} ‘‘Yampi, bhikkhave, tathāgato purimaṃ jātiṃ purimaṃ bhavaṃ purimaṃ niketaṃ pubbe manussabhūto samāno micchājīvaṃ pahāya sammāājīvena jīvikaṃ kappesi, tulākūṭa kaṃsakūṭa mānakūṭa ukkoṭana vañcana nikati sāciyoga chedana vadha bandhana viparāmosa ālopa sahasākārā\footnote{sāhasākārā (sī. syā. pī.)} paṭivirato ahosi. So tassa kammassa kaṭattā upacitattā ussannattā vipulattā kāyassa bhedā paraṃ maraṇā sugatiṃ saggaṃ lokaṃ upapajjati. So tattha aññe deve dasahi ṭhānehi adhigaṇhāti dibbena āyunā dibbena vaṇṇena dibbena sukhena dibbena yasena dibbena ādhipateyyena dibbehi rūpehi dibbehi saddehi dibbehi gandhehi dibbehi rasehi dibbehi phoṭṭhabbehi. So tato cuto itthattaṃ āgato samāno imāni dve mahāpurisalakkhaṇāni paṭilabhati, samadanto ca hoti susukkadāṭho ca.

‘‘So tehi lakkhaṇehi samannāgato sace agāraṃ ajjhāvasati, rājā hoti cakkavattī dhammiko dhammarājā cāturanto vijitāvī janapadatthāvariyappatto sattaratanasamannāgato. Tassimāni satta ratanāni bhavanti, seyyathidaṃ – cakkaratanaṃ hatthiratanaṃ assaratanaṃ maṇiratanaṃ itthiratanaṃ gahapatiratanaṃ pariṇāyakaratanameva sattamaṃ. Parosahassaṃ kho panassa puttā bhavanti sūrā vīraṅgarūpā parasenappamaddanā. So imaṃ pathaviṃ sāgarapariyantaṃ akhilamanimittamakaṇṭakaṃ iddhaṃ phītaṃ khemaṃ sivaṃ nirabbudaṃ adaṇḍena asatthena dhammena abhivijiya ajjhāvasati. Rājā samāno kiṃ labhati? Suciparivāro hoti sucissa honti parivārā brāhmaṇagahapatikā negamajānapadā gaṇakamahāmattā anīkaṭṭhā dovārikā amaccā pārisajjā rājāno bhogiyā kumārā. Rājā samāno idaṃ labhati.

‘‘Sace kho pana agārasmā anagāriyaṃ pabbajati, arahaṃ hoti sammāsambuddho loke vivaṭṭacchado. Buddho samāno kiṃ labhati? Suciparivāro hoti, sucissa honti parivārā, bhikkhū bhikkhuniyo upāsakā upāsikāyo devā manussā asurā nāgā gandhabbā. Buddho samāno idaṃ labhati’’. Etamatthaṃ bhagavā avoca.

\paragraph{241.} Tatthetaṃ vuccati –

‘‘Micchājīvañca avassaji samena vuttiṃ,

Sucinā so janayittha dhammikena;

Ahitamapi ca apanudi,

Hitamapi ca bahujanasukhañca acari.

‘‘Sagge vedayati naro sukhapphalāni,

Karitvā nipuṇebhi vidūhi sabbhi;

Vaṇṇitāni tidivapuravarasamo,

Abhiramati ratikhiḍḍāsamaṅgī.

‘‘Laddhānaṃ mānusakaṃ bhavaṃ tato,

Cavitvāna sukataphalavipākaṃ;

Sesakena paṭilabhati lapanajaṃ,

Samamapi sucisusukkaṃ\footnote{laddhāna manussakaṃ bhavaṃ tato caviya, puna sukataphalavipākasesakena; paṭilabhati lapanajaṃ samamapi, suci ca suvisuddhasusukkaṃ (syā.)}.

‘‘Taṃ veyyañjanikā samāgatā bahavo,

Byākaṃsu nipuṇasammatā manujā;

Sucijanaparivāragaṇo bhavati,

Dijasamasukkasucisobhanadanto.

‘‘Rañño hoti bahujano,

Suciparivāro mahatiṃ mahiṃ anusāsato;

Pasayha na ca janapadatudanaṃ,

Hitamapi ca bahujanasukhañca caranti.

‘‘Atha ce pabbajati bhavati vipāpo,

Samaṇo samitarajo vivaṭṭacchado;

Vigatadarathakilamatho,

Imamapi ca paramapi ca\footnote{imampi ca parampi ca (pī.), paraṃpi paramaṃpi ca (syā.)} passati lokaṃ.

‘‘Tassovādakarā bahugihī ca pabbajitā ca,

Asuciṃ garahitaṃ dhunanti pāpaṃ;

Sa hi sucibhi parivuto bhavati,

Malakhilakalikilese panudehī’’ti\footnote{tassovādakarā bahugihī ca, pabbajitā ca asucivigarahita; panudipāpassa hi sucibhiparivuto, bhavati malakhilakakilese panudeti (syā.)}.

Idamavoca bhagavā. Attamanā te bhikkhū bhagavato bhāsitaṃ abhinandunti.

\xsectionEnd{Lakkhaṇasuttaṃ niṭṭhitaṃ sattamaṃ.}


\section{Siṅgālasuttaṃ}

\paragraph{242.} Evaṃ me sutaṃ – ekaṃ samayaṃ bhagavā rājagahe viharati veḷuvane kalandakanivāpe. Tena kho pana samayena siṅgālako\footnote{sigālako (sī.)} gahapatiputto kālasseva uṭṭhāya rājagahā nikkhamitvā allavattho allakeso pañjaliko puthudisā\footnote{puthuddisā (sī. syā. pī.)} namassati – puratthimaṃ disaṃ dakkhiṇaṃ disaṃ pacchimaṃ disaṃ uttaraṃ disaṃ heṭṭhimaṃ disaṃ uparimaṃ disaṃ.

\paragraph{243.} Atha kho bhagavā pubbaṇhasamayaṃ nivāsetvā pattacīvaramādāya rājagahaṃ piṇḍāya pāvisi. Addasā kho bhagavā siṅgālakaṃ gahapatiputtaṃ kālasseva vuṭṭhāya rājagahā nikkhamitvā allavatthaṃ allakesaṃ pañjalikaṃ puthudisā namassantaṃ – puratthimaṃ disaṃ dakkhiṇaṃ disaṃ pacchimaṃ disaṃ uttaraṃ disaṃ heṭṭhimaṃ disaṃ uparimaṃ disaṃ. Disvā siṅgālakaṃ gahapatiputtaṃ etadavoca – ‘‘kiṃ nu kho tvaṃ, gahapatiputta, kālasseva uṭṭhāya rājagahā nikkhamitvā allavattho allakeso pañjaliko puthudisā namassasi – puratthimaṃ disaṃ dakkhiṇaṃ disaṃ pacchimaṃ disaṃ uttaraṃ disaṃ heṭṭhimaṃ disaṃ uparimaṃ disa’’nti? ‘‘Pitā maṃ, bhante, kālaṃ karonto evaṃ avaca – ‘disā, tāta, namasseyyāsī’ti. So kho ahaṃ, bhante, pituvacanaṃ sakkaronto garuṃ karonto mānento pūjento kālasseva uṭṭhāya rājagahā nikkhamitvā allavattho allakeso pañjaliko puthudisā namassāmi – puratthimaṃ disaṃ dakkhiṇaṃ disaṃ pacchimaṃ disaṃ uttaraṃ disaṃ heṭṭhimaṃ disaṃ uparimaṃ disa’’nti.

\subsubsection{Cha disā}

\paragraph{244.} ‘‘Na kho, gahapatiputta, ariyassa vinaye evaṃ cha disā\footnote{chaddisā (sī. pī.)} namassitabbā’’ti. ‘‘Yathā kathaṃ pana, bhante, ariyassa vinaye cha disā\footnote{chaddisā (sī. pī.)} namassitabbā? Sādhu me, bhante, bhagavā tathā dhammaṃ desetu, yathā ariyassa vinaye cha disā\footnote{chaddisā (sī. pī.)} namassitabbā’’ti.

‘‘Tena hi, gahapatiputta suṇohi sādhukaṃ manasikarohi bhāsissāmī’’ti. ‘‘Evaṃ, bhante’’ti kho siṅgālako gahapatiputto bhagavato paccassosi. Bhagavā etadavoca –

‘‘Yato kho, gahapatiputta, ariyasāvakassa cattāro kammakilesā pahīnā honti, catūhi ca ṭhānehi pāpakammaṃ na karoti, cha ca bhogānaṃ apāyamukhāni na sevati, so evaṃ cuddasa pāpakāpagato chaddisāpaṭicchādī\footnote{paṭicchādī hoti (syā.)} ubholokavijayāya paṭipanno hoti. Tassa ayañceva loko āraddho hoti paro ca loko. So kāyassa bhedā paraṃ maraṇā sugatiṃ saggaṃ lokaṃ upapajjati.

\subsubsection{Cattārokammakilesā}

\paragraph{245.} ‘‘Katamassa cattāro kammakilesā pahīnā honti? Pāṇātipāto kho, gahapatiputta, kammakileso, adinnādānaṃ kammakileso, kāmesumicchācāro kammakileso, musāvādo kammakileso. Imassa cattāro kammakilesā pahīnā hontī’’ti. Idamavoca bhagavā, idaṃ vatvāna\footnote{idaṃ vatvā (sī. pī.) evamīdisesu ṭhānesu} sugato athāparaṃ etadavoca satthā –

‘‘Pāṇātipāto adinnādānaṃ, musāvādo ca vuccati;

Paradāragamanañceva, nappasaṃsanti paṇḍitā’’ti.

\subsubsection{Catuṭṭhānaṃ}

\paragraph{246.} ‘‘Katamehi catūhi ṭhānehi pāpakammaṃ na karoti? Chandāgatiṃ gacchanto pāpakammaṃ karoti, dosāgatiṃ gacchanto pāpakammaṃ karoti, mohāgatiṃ gacchanto pāpakammaṃ karoti, bhayāgatiṃ gacchanto pāpakammaṃ karoti. Yato kho, gahapatiputta, ariyasāvako neva chandāgatiṃ gacchati, na dosāgatiṃ gacchati, na mohāgatiṃ gacchati, na bhayāgatiṃ gacchati; imehi catūhi ṭhānehi pāpakammaṃ na karotī’’ti. Idamavoca bhagavā, idaṃ vatvāna sugato athāparaṃ etadavoca satthā –

‘‘Chandā dosā bhayā mohā, yo dhammaṃ ativattati;

Nihīyati yaso tassa\footnote{tassa yeso (bahūsu, vinayepi)}, kāḷapakkheva candimā.

‘‘Chandā dosā bhayā mohā, yo dhammaṃ nātivattati;

Āpūrati yaso tassa\footnote{tassa yeso (bahūsu, vinayepi)}, sukkapakkheva\footnote{juṇhapakkheva (ka.)} candimā’’ti.

\subsubsection{Cha apāyamukhāni}

\paragraph{247.} ‘‘Katamāni cha bhogānaṃ apāyamukhāni na sevati? Surāmerayamajjappamādaṭṭhānānuyogo kho, gahapatiputta, bhogānaṃ apāyamukhaṃ, vikālavisikhācariyānuyogo bhogānaṃ apāyamukhaṃ, samajjābhicaraṇaṃ bhogānaṃ apāyamukhaṃ, jūtappamādaṭṭhānānuyogo bhogānaṃ apāyamukhaṃ, pāpamittānuyogo bhogānaṃ apāyamukhaṃ, ālasyānuyogo\footnote{ālasānuyogo (sī. syā. pī.)} bhogānaṃ apāyamukhaṃ.

\subsubsection{Surāmerayassa cha ādīnavā}

\paragraph{248.} ‘‘Cha khome, gahapatiputta, ādīnavā surāmerayamajjappamādaṭṭhānānuyoge. Sandiṭṭhikā dhanajāni\footnote{dhanañjāni (sī. pī.)}, kalahappavaḍḍhanī, rogānaṃ āyatanaṃ, akittisañjananī, kopīnanidaṃsanī , paññāya dubbalikaraṇītveva chaṭṭhaṃ padaṃ bhavati. Ime kho, gahapatiputta, cha ādīnavā surāmerayamajjappamādaṭṭhānānuyoge.

\subsubsection{Vikālacariyāya cha ādīnavā}

\paragraph{249.} ‘‘Cha khome, gahapatiputta, ādīnavā vikālavisikhācariyānuyoge. Attāpissa agutto arakkhito hoti, puttadāropissa agutto arakkhito hoti, sāpateyyaṃpissa aguttaṃ arakkhitaṃ hoti, saṅkiyo ca hoti pāpakesu ṭhānesu\footnote{tesu tesu ṭhānesu (syā.)}, abhūtavacanañca tasmiṃ rūhati, bahūnañca dukkhadhammānaṃ purakkhato hoti. Ime kho, gahapatiputta, cha ādīnavā vikālavisikhācariyānuyoge.

\subsubsection{Samajjābhicaraṇassa cha ādīnavā}

\paragraph{250.} ‘‘Cha khome, gahapatiputta, ādīnavā samajjābhicaraṇe. Kva\footnote{kuvaṃ (ka. sī. pī.)} naccaṃ, kva gītaṃ, kva vāditaṃ, kva akkhānaṃ, kva pāṇissaraṃ, kva kumbhathunanti. Ime kho, gahapatiputta, cha ādīnavā samajjābhicaraṇe.

\subsubsection{Jūtappamādassa cha ādīnavā}

\paragraph{251.} ‘‘Cha khome, gahapatiputta, ādīnavā jūtappamādaṭṭhānānuyoge. Jayaṃ veraṃ pasavati, jino vittamanusocati, sandiṭṭhikā dhanajāni, sabhāgatassa\footnote{sabhāye tassa (ka.)} vacanaṃ na rūhati, mittāmaccānaṃ paribhūto hoti, āvāhavivāhakānaṃ apatthito hoti – ‘akkhadhutto ayaṃ purisapuggalo nālaṃ dārabharaṇāyā’ti. Ime kho, gahapatiputta, cha ādīnavā jūtappamādaṭṭhānānuyoge.

\subsubsection{Pāpamittatāya cha ādīnavā}

\paragraph{252.} ‘‘Cha khome, gahapatiputta, ādīnavā pāpamittānuyoge. Ye dhuttā, ye soṇḍā, ye pipāsā, ye nekatikā, ye vañcanikā, ye sāhasikā. Tyāssa mittā honti te sahāyā. Ime kho, gahapatiputta, cha ādīnavā pāpamittānuyoge.

\subsubsection{Ālasyassa cha ādīnavā}

\paragraph{253.} ‘‘Cha khome, gahapatiputta, ādīnavā ālasyānuyoge. Atisītanti kammaṃ na karoti, atiuṇhanti kammaṃ na karoti, atisāyanti kammaṃ na karoti, atipātoti kammaṃ na karoti, atichātosmīti kammaṃ na karoti, atidhātosmīti kammaṃ na karoti. Tassa evaṃ kiccāpadesabahulassa viharato anuppannā ceva bhogā nuppajjanti, uppannā ca bhogā parikkhayaṃ gacchanti. Ime kho, gahapatiputta, cha ādīnavā ālasyānuyoge’’ti. Idamavoca bhagavā, idaṃ vatvāna sugato athāparaṃ etadavoca satthā –

‘‘Hoti pānasakhā nāma,

Hoti sammiyasammiyo;

Yo ca atthesu jātesu,

Sahāyo hoti so sakhā.

‘‘Ussūraseyyā paradārasevanā,

Verappasavo\footnote{verappasaṅgo (sī. syā. pī.)} ca anatthatā ca;

Pāpā ca mittā sukadariyatā ca,

Ete cha ṭhānā purisaṃ dhaṃsayanti.

‘‘Pāpamitto pāpasakho,

Pāpaācāragocaro;

Asmā lokā paramhā ca,

Ubhayā dhaṃsate naro.

‘‘Akkhitthiyo vāruṇī naccagītaṃ,

Divā soppaṃ pāricariyā akāle;

Pāpā ca mittā sukadariyatā ca,

Ete cha ṭhānā purisaṃ dhaṃsayanti.

‘‘Akkhehi dibbanti suraṃ pivanti,

Yantitthiyo pāṇasamā paresaṃ;

Nihīnasevī na ca vuddhasevī\footnote{vuddhisevī (syā.), buddhisevī (ka.)},

Nihīyate kāḷapakkheva cando.

‘‘Yo vāruṇī addhano akiñcano,

Pipāso pivaṃ papāgato\footnote{pipāsosi atthapāgato (syā.), pipāsopi samappapāgato (ka.)};

Udakamiva iṇaṃ vigāhati,

Akulaṃ\footnote{ākulaṃ (syā. ka.)} kāhiti khippamattano.

‘‘Na divā soppasīlena, rattimuṭṭhānadessinā\footnote{rattinuṭṭhānadassinā (sī. pī.), rattinuṭṭhānasīlinā (?)};

Niccaṃ mattena soṇḍena, sakkā āvasituṃ gharaṃ.

‘‘Atisītaṃ atiuṇhaṃ, atisāyamidaṃ ahu;

Iti vissaṭṭhakammante, atthā accenti māṇave.

‘‘Yodha sītañca uṇhañca, tiṇā bhiyyo na maññati;

Karaṃ purisakiccāni, so sukhaṃ\footnote{sukhā (sabbattha) aṭṭhakathā oloketabbā} na vihāyatī’’ti.

\subsubsection{Mittapatirūpakā}

\paragraph{254.} ‘‘Cattārome, gahapatiputta, amittā mittapatirūpakā veditabbā. Aññadatthuharo amitto mittapatirūpako veditabbo, vacīparamo amitto mittapatirūpako veditabbo, anuppiyabhāṇī amitto mittapatirūpako veditabbo, apāyasahāyo amitto mittapatirūpako veditabbo.

\paragraph{255.} ‘‘Catūhi kho, gahapatiputta, ṭhānehi aññadatthuharo amitto mittapatirūpako veditabbo.

‘‘Aññadatthuharo hoti, appena bahumicchati ;

Bhayassa kiccaṃ karoti, sevati atthakāraṇā.

‘‘Imehi kho, gahapatiputta, catūhi ṭhānehi aññadatthuharo amitto mittapatirūpako veditabbo.

\paragraph{256.} ‘‘Catūhi kho, gahapatiputta, ṭhānehi vacīparamo amitto mittapatirūpako veditabbo. Atītena paṭisantharati\footnote{paṭisandharati (ka.)}, anāgatena paṭisantharati, niratthakena saṅgaṇhāti, paccuppannesu kiccesu byasanaṃ dasseti. Imehi kho, gahapatiputta, catūhi ṭhānehi vacīparamo amitto mittapatirūpako veditabbo.

\paragraph{257.} ‘‘Catūhi kho, gahapatiputta, ṭhānehi anuppiyabhāṇī amitto mittapatirūpako veditabbo. Pāpakaṃpissa\footnote{pāpakammaṃpissa (syā.)} anujānāti, kalyāṇaṃpissa anujānāti, sammukhāssa vaṇṇaṃ bhāsati, parammukhāssa avaṇṇaṃ bhāsati. Imehi kho, gahapatiputta, catūhi ṭhānehi anuppiyabhāṇī amitto mittapatirūpako veditabbo.

\paragraph{258.} ‘‘Catūhi kho, gahapatiputta, ṭhānehi apāyasahāyo amitto mittapatirūpako veditabbo . Surāmeraya majjappamādaṭṭhānānuyoge sahāyo hoti, vikāla visikhā cariyānuyoge sahāyo hoti, samajjābhicaraṇe sahāyo hoti, jūtappamādaṭṭhānānuyoge sahāyo hoti. Imehi kho, gahapatiputta, catūhi ṭhānehi apāyasahāyo amitto mittapatirūpako veditabbo’’ti.

\paragraph{259.} Idamavoca bhagavā, idaṃ vatvāna sugato athāparaṃ etadavoca satthā –

‘‘Aññadatthuharo mitto, yo ca mitto vacīparo\footnote{vacīparamo (syā.)};

Anuppiyañca yo āha, apāyesu ca yo sakhā.

Ete amitte cattāro, iti viññāya paṇḍito;

Ārakā parivajjeyya, maggaṃ paṭibhayaṃ yathā’’ti.

\subsubsection{Suhadamitto}

\paragraph{260.} ‘‘Cattārome , gahapatiputta, mittā suhadā veditabbā. Upakāro\footnote{upakārako (syā.)} mitto suhado veditabbo, samānasukhadukkho mitto suhado veditabbo, atthakkhāyī mitto suhado veditabbo, anukampako mitto suhado veditabbo.

\paragraph{261.} ‘‘Catūhi kho, gahapatiputta, ṭhānehi upakāro mitto suhado veditabbo. Pamattaṃ rakkhati, pamattassa sāpateyyaṃ rakkhati, bhītassa saraṇaṃ hoti, uppannesu kiccakaraṇīyesu taddiguṇaṃ bhogaṃ anuppadeti. Imehi kho, gahapatiputta, catūhi ṭhānehi upakāro mitto suhado veditabbo.

\paragraph{262.} ‘‘Catūhi kho, gahapatiputta, ṭhānehi samānasukhadukkho mitto suhado veditabbo. Guyhamassa ācikkhati, guyhamassa parigūhati, āpadāsu na vijahati, jīvitaṃpissa atthāya pariccattaṃ hoti. Imehi kho, gahapatiputta, catūhi ṭhānehi samānasukhadukkho mitto suhado veditabbo.

\paragraph{263.} ‘‘Catūhi kho, gahapatiputta, ṭhānehi atthakkhāyī mitto suhado veditabbo. Pāpā nivāreti, kalyāṇe niveseti, assutaṃ sāveti, saggassa maggaṃ ācikkhati. Imehi kho, gahapatiputta, catūhi ṭhānehi atthakkhāyī mitto suhado veditabbo.

\paragraph{264.} ‘‘Catūhi kho, gahapatiputta, ṭhānehi anukampako mitto suhado veditabbo. Abhavenassa na nandati, bhavenassa nandati, avaṇṇaṃ bhaṇamānaṃ nivāreti, vaṇṇaṃ bhaṇamānaṃ pasaṃsati. Imehi kho, gahapatiputta, catūhi ṭhānehi anukampako mitto suhado veditabbo’’ti.

\paragraph{265.} Idamavoca bhagavā, idaṃ vatvāna sugato athāparaṃ etadavoca satthā –

‘‘Upakāro ca yo mitto, sukhe dukkhe\footnote{sukhadukkho (syā. ka.)} ca yo sakhā\footnote{yo ca mitto sukhe dukkhe (sī. pī.)};

Atthakkhāyī ca yo mitto, yo ca mittānukampako.

‘‘Etepi mitte cattāro, iti viññāya paṇḍito;

Sakkaccaṃ payirupāseyya, mātā puttaṃ va orasaṃ;

Paṇḍito sīlasampanno, jalaṃ aggīva bhāsati.

‘‘Bhoge saṃharamānassa, bhamarasseva irīyato;

Bhogā sannicayaṃ yanti, vammikovupacīyati.

‘‘Evaṃ bhoge samāhatvā\footnote{samāharitvā (syā.)}, alamatto kule gihī;

Catudhā vibhaje bhoge, sa ve mittāni ganthati.

‘‘Ekena bhoge bhuñjeyya, dvīhi kammaṃ payojaye;

Catutthañca nidhāpeyya, āpadāsu bhavissatī’’ti.

\subsubsection{Chaddisāpaṭicchādanakaṇḍaṃ}

\paragraph{266.} ‘‘Kathañca, gahapatiputta, ariyasāvako chaddisāpaṭicchādī hoti? Cha imā, gahapatiputta, disā veditabbā. Puratthimā disā mātāpitaro veditabbā, dakkhiṇā disā ācariyā veditabbā, pacchimā disā puttadārā veditabbā, uttarā disā mittāmaccā veditabbā, heṭṭhimā disā dāsakammakarā veditabbā, uparimā disā samaṇabrāhmaṇā veditabbā.

\paragraph{267.} ‘‘Pañcahi kho, gahapatiputta, ṭhānehi puttena puratthimā disā mātāpitaro paccupaṭṭhātabbā – bhato ne\footnote{nesaṃ (bahūsu)} bharissāmi, kiccaṃ nesaṃ karissāmi, kulavaṃsaṃ ṭhapessāmi, dāyajjaṃ paṭipajjāmi, atha vā pana petānaṃ kālaṅkatānaṃ dakkhiṇaṃ anuppadassāmīti. Imehi kho, gahapatiputta, pañcahi ṭhānehi puttena puratthimā disā mātāpitaro paccupaṭṭhitā pañcahi ṭhānehi puttaṃ anukampanti. Pāpā nivārenti, kalyāṇe nivesenti, sippaṃ sikkhāpenti, patirūpena dārena saṃyojenti, samaye dāyajjaṃ niyyādenti\footnote{niyyātenti (ka. sī.)}. Imehi kho, gahapatiputta, pañcahi ṭhānehi puttena puratthimā disā mātāpitaro paccupaṭṭhitā imehi pañcahi ṭhānehi puttaṃ anukampanti. Evamassa esā puratthimā disā paṭicchannā hoti khemā appaṭibhayā.

\paragraph{268.} ‘‘Pañcahi kho, gahapatiputta, ṭhānehi antevāsinā dakkhiṇā disā ācariyā paccupaṭṭhātabbā – uṭṭhānena upaṭṭhānena sussusāya pāricariyāya sakkaccaṃ sippapaṭiggahaṇena\footnote{sippaṃ paṭiggahaṇena (syā.), sippauggahaṇena (ka.)}. Imehi kho, gahapatiputta, pañcahi ṭhānehi antevāsinā dakkhiṇā disā ācariyā paccupaṭṭhitā pañcahi ṭhānehi antevāsiṃ anukampanti – suvinītaṃ vinenti, suggahitaṃ gāhāpenti, sabbasippassutaṃ samakkhāyino bhavanti, mittāmaccesu paṭiyādenti\footnote{paṭivedenti (syā.)}, disāsu parittāṇaṃ karonti. Imehi kho, gahapatiputta, pañcahi ṭhānehi antevāsinā dakkhiṇā disā ācariyā paccupaṭṭhitā imehi pañcahi ṭhānehi antevāsiṃ anukampanti. Evamassa esā dakkhiṇā disā paṭicchannā hoti khemā appaṭibhayā.

\paragraph{269.} ‘‘Pañcahi kho, gahapatiputta, ṭhānehi sāmikena pacchimā disā bhariyā paccupaṭṭhātabbā – sammānanāya anavamānanāya\footnote{avimānanāya (syā. pī.)} anaticariyāya issariyavossaggena alaṅkārānuppadānena. Imehi kho, gahapatiputta, pañcahi ṭhānehi sāmikena pacchimā disā bhariyā paccupaṭṭhitā pañcahi ṭhānehi sāmikaṃ anukampati – susaṃvihitakammantā ca hoti, saṅgahitaparijanā\footnote{susaṅgahitaparijanā (sī. syā. pī.)} ca, anaticārinī ca, sambhatañca anurakkhati, dakkhā ca hoti analasā sabbakiccesu. Imehi kho, gahapatiputta, pañcahi ṭhānehi sāmikena pacchimā disā bhariyā paccupaṭṭhitā imehi pañcahi ṭhānehi sāmikaṃ anukampati. Evamassa esā pacchimā disā paṭicchannā hoti khemā appaṭibhayā.

\paragraph{270.} ‘‘Pañcahi kho, gahapatiputta, ṭhānehi kulaputtena uttarā disā mittāmaccā paccupaṭṭhātabbā – dānena peyyavajjena\footnote{viyavajjena (syā. ka.)} atthacariyāya samānattatāya avisaṃvādanatāya. Imehi kho, gahapatiputta, pañcahi ṭhānehi kulaputtena uttarā disā mittāmaccā paccupaṭṭhitā pañcahi ṭhānehi kulaputtaṃ anukampanti – pamattaṃ rakkhanti, pamattassa sāpateyyaṃ rakkhanti, bhītassa saraṇaṃ honti, āpadāsu na vijahanti, aparapajā cassa paṭipūjenti. Imehi kho, gahapatiputta, pañcahi ṭhānehi kulaputtena uttarā disā mittāmaccā paccupaṭṭhitā imehi pañcahi ṭhānehi kulaputtaṃ anukampanti. Evamassa esā uttarā disā paṭicchannā hoti khemā appaṭibhayā.

\paragraph{271.} ‘‘Pañcahi kho, gahapatiputta, ṭhānehi ayyirakena\footnote{ayirakena (sī. syā. pī.)} heṭṭhimā disā dāsakammakarā paccupaṭṭhātabbā – yathābalaṃ kammantasaṃvidhānena bhattavetanānuppadānena gilānupaṭṭhānena acchariyānaṃ rasānaṃ saṃvibhāgena samaye vossaggena. Imehi kho, gahapatiputta, pañcahi ṭhānehi ayyirakena heṭṭhimā disā dāsakammakarā paccupaṭṭhitā pañcahi ṭhānehi ayyirakaṃ anukampanti – pubbuṭṭhāyino ca honti, pacchā nipātino ca, dinnādāyino ca, sukatakammakarā ca, kittivaṇṇaharā ca. Imehi kho, gahapatiputta, pañcahi ṭhānehi ayyirakena heṭṭhimā disā dāsakammakarā paccupaṭṭhitā imehi pañcahi ṭhānehi ayyirakaṃ anukampanti. Evamassa esā heṭṭhimā disā paṭicchannā hoti khemā appaṭibhayā.

\paragraph{272.} ‘‘Pañcahi kho, gahapatiputta, ṭhānehi kulaputtena uparimā disā samaṇabrāhmaṇā paccupaṭṭhātabbā – mettena kāyakammena mettena vacīkammena mettena manokammena anāvaṭadvāratāya āmisānuppadānena. Imehi kho, gahapatiputta, pañcahi ṭhānehi kulaputtena uparimā disā samaṇabrāhmaṇā paccupaṭṭhitā chahi ṭhānehi kulaputtaṃ anukampanti – pāpā nivārenti, kalyāṇe nivesenti, kalyāṇena manasā anukampanti, assutaṃ sāventi, sutaṃ pariyodāpenti, saggassa maggaṃ ācikkhanti. Imehi kho, gahapatiputta, pañcahi ṭhānehi kulaputtena uparimā disā samaṇabrāhmaṇā paccupaṭṭhitā imehi chahi ṭhānehi kulaputtaṃ anukampanti. Evamassa esā uparimā disā paṭicchannā hoti khemā appaṭibhayā’’ti.

\paragraph{273.} Idamavoca bhagavā. Idaṃ vatvāna sugato athāparaṃ etadavoca satthā –

‘‘Mātāpitā disā pubbā, ācariyā dakkhiṇā disā;

Puttadārā disā pacchā, mittāmaccā ca uttarā.

‘‘Dāsakammakarā heṭṭhā, uddhaṃ samaṇabrāhmaṇā;

Etā disā namasseyya, alamatto kule gihī.

‘‘Paṇḍito sīlasampanno, saṇho ca paṭibhānavā;

Nivātavutti atthaddho, tādiso labhate yasaṃ.

‘‘Uṭṭhānako analaso, āpadāsu na vedhati;

Acchinnavutti medhāvī, tādiso labhate yasaṃ.

‘‘Saṅgāhako mittakaro, vadaññū vītamaccharo;

Netā vinetā anunetā, tādiso labhate yasaṃ.

‘‘Dānañca peyyavajjañca, atthacariyā ca yā idha;

Samānattatā ca dhammesu, tattha tattha yathārahaṃ;

Ete kho saṅgahā loke, rathassāṇīva yāyato.

‘‘Ete ca saṅgahā nāssu, na mātā puttakāraṇā;

Labhetha mānaṃ pūjaṃ vā, pitā vā puttakāraṇā.

‘‘Yasmā ca saṅgahā\footnote{saṅgahe (ka.) aṭṭhakathāyaṃ icchitapāṭho} ete, sammapekkhanti\footnote{samavekkhanti (sī. pī. ka.)} paṇḍitā;

Tasmā mahattaṃ papponti, pāsaṃsā ca bhavanti te’’ti.

\paragraph{274.} Evaṃ vutte, siṅgālako gahapatiputto bhagavantaṃ etadavoca – ‘‘abhikkantaṃ, bhante! Abhikkantaṃ, bhante! Seyyathāpi, bhante, nikkujjitaṃ vā ukkujjeyya, paṭicchannaṃ vā vivareyya, mūḷhassa vā maggaṃ ācikkheyya, andhakāre vā telapajjotaṃ dhāreyya ‘cakkhumanto rūpāni dakkhantī’ti. Evamevaṃ bhagavatā anekapariyāyena dhammo pakāsito. Esāhaṃ, bhante, bhagavantaṃ saraṇaṃ gacchāmi dhammañca bhikkhusaṃghañca. Upāsakaṃ maṃ bhagavā dhāretu, ajjatagge pāṇupetaṃ saraṇaṃ gata’’nti.

\xsectionEnd{Siṅgālasuttaṃ\footnote{siṅgālovādasuttantaṃ (pī.)} niṭṭhitaṃ aṭṭhamaṃ.}


\section{Āṭānāṭiyasuttaṃ}

\subsubsection{Paṭhamabhāṇavāro}

\paragraph{275.} Evaṃ me sutaṃ – ekaṃ samayaṃ bhagavā rājagahe viharati gijjhakūṭe pabbate. Atha kho cattāro mahārājā\footnote{mahārājāno (ka.)} mahatiyā ca yakkhasenāya mahatiyā ca gandhabbasenāya mahatiyā ca kumbhaṇḍasenāya mahatiyā ca nāgasenāya catuddisaṃ rakkhaṃ ṭhapetvā catuddisaṃ gumbaṃ ṭhapetvā catuddisaṃ ovaraṇaṃ ṭhapetvā abhikkantāya rattiyā abhikkantavaṇṇā kevalakappaṃ gijjhakūṭaṃ pabbataṃ obhāsetvā\footnote{gijjhakūṭaṃ obhāsetvā (sī. syā. pī.)} yena bhagavā tenupasaṅkamiṃsu; upasaṅkamitvā bhagavantaṃ abhivādetvā ekamantaṃ nisīdiṃsu. Tepi kho yakkhā appekacce bhagavantaṃ abhivādetvā ekamantaṃ nisīdiṃsu, appekacce bhagavatā saddhiṃ sammodiṃsu, sammodanīyaṃ kathaṃ sāraṇīyaṃ vītisāretvā ekamantaṃ nisīdiṃsu, appekacce yena bhagavā tenañjaliṃ paṇāmetvā ekamantaṃ nisīdiṃsu, appekacce nāmagottaṃ sāvetvā ekamantaṃ nisīdiṃsu, appekacce tuṇhībhūtā ekamantaṃ nisīdiṃsu.

\paragraph{276.} Ekamantaṃ nisinno kho vessavaṇo mahārājā bhagavantaṃ etadavoca – ‘‘santi hi, bhante, uḷārā yakkhā bhagavato appasannā. Santi hi, bhante, uḷārā yakkhā bhagavato pasannā. Santi hi , bhante, majjhimā yakkhā bhagavato appasannā. Santi hi, bhante, majjhimā yakkhā bhagavato pasannā. Santi hi, bhante, nīcā yakkhā bhagavato appasannā. Santi hi, bhante, nīcā yakkhā bhagavato pasannā. Yebhuyyena kho pana, bhante, yakkhā appasannāyeva bhagavato. Taṃ kissa hetu? Bhagavā hi, bhante, pāṇātipātā veramaṇiyā dhammaṃ deseti, adinnādānā veramaṇiyā dhammaṃ deseti, kāmesumicchācārā veramaṇiyā dhammaṃ deseti, musāvādā veramaṇiyā dhammaṃ deseti, surāmerayamajjappamādaṭṭhānā veramaṇiyā dhammaṃ deseti. Yebhuyyena kho pana, bhante, yakkhā appaṭiviratāyeva pāṇātipātā, appaṭiviratā adinnādānā, appaṭiviratā kāmesumicchācārā, appaṭiviratā musāvādā, appaṭiviratā surāmerayamajjappamādaṭṭhānā . Tesaṃ taṃ hoti appiyaṃ amanāpaṃ. Santi hi, bhante, bhagavato sāvakā araññavanapatthāni pantāni senāsanāni paṭisevanti appasaddāni appanigghosāni vijanavātāni manussarāhasseyyakāni\footnote{manussarāhaseyyakāni (sī. syā. pī.)} paṭisallānasāruppāni. Tattha santi uḷārā yakkhā nivāsino, ye imasmiṃ bhagavato pāvacane appasannā. Tesaṃ pasādāya uggaṇhātu, bhante, bhagavā āṭānāṭiyaṃ rakkhaṃ bhikkhūnaṃ bhikkhunīnaṃ upāsakānaṃ upāsikānaṃ guttiyā rakkhāya avihiṃsāya phāsuvihārāyā’’ti. Adhivāsesi bhagavā tuṇhībhāvena.

Atha kho vessavaṇo mahārājā bhagavato adhivāsanaṃ viditvā tāyaṃ velāyaṃ imaṃ āṭānāṭiyaṃ rakkhaṃ abhāsi –

\paragraph{277.} ‘‘Vipassissa ca\footnote{ime cakārā porāṇapotthakesu natthi} namatthu, cakkhumantassa sirīmato.

Sikhissapi ca\footnote{ime cakārā porāṇapotthakesu natthi} namatthu, sabbabhūtānukampino.

‘‘Vessabhussa ca\footnote{ime cakārā porāṇapotthakesu natthi} namatthu, nhātakassa tapassino;

Namatthu kakusandhassa, mārasenāpamaddino.

‘‘Koṇāgamanassa namatthu, brāhmaṇassa vusīmato;

Kassapassa ca\footnote{ime cakārā porāṇapotthakesu natthi} namatthu, vippamuttassa sabbadhi.

‘‘Aṅgīrasassa namatthu, sakyaputtassa sirīmato;

Yo imaṃ dhammaṃ desesi\footnote{dhammamadesesi (sī. syā. pī.), dhammaṃ deseti (?)}, sabbadukkhāpanūdanaṃ.

‘‘Ye cāpi nibbutā loke, yathābhūtaṃ vipassisuṃ;

Te janā apisuṇātha\footnote{apisuṇā (sī. syā. pī.)}, mahantā vītasāradā.

‘‘Hitaṃ devamanussānaṃ, yaṃ namassanti gotamaṃ;

Vijjācaraṇasampannaṃ, mahantaṃ vītasāradaṃ.

\paragraph{278.} ‘‘Yato uggacchati sūriyo\footnote{suriyo (sī. syā. pī.)}, ādicco maṇḍalī mahā.

Yassa cuggacchamānassa, saṃvarīpi nirujjhati;

Yassa cuggate sūriye, ‘divaso’ti pavuccati.

‘‘Rahadopi tattha gambhīro, samuddo saritodako;

Evaṃ taṃ tattha jānanti, ‘samuddo saritodako’.

‘‘Ito ‘sā purimā disā’, iti naṃ ācikkhatī jano;

Yaṃ disaṃ abhipāleti, mahārājā yasassi so.

‘‘Gandhabbānaṃ adhipati\footnote{ādhipati (sī. syā. pī.) evamuparipi}, ‘dhataraṭṭho’ti nāmaso;

Ramatī naccagītehi, gandhabbehi purakkhato.

‘‘Puttāpi tassa bahavo, ekanāmāti me sutaṃ;

Asīti dasa eko ca, indanāmā mahabbalā.

Te cāpi buddhaṃ disvāna, buddhaṃ ādiccabandhunaṃ;

Dūratova namassanti, mahantaṃ vītasāradaṃ.

‘‘Namo te purisājañña, namo te purisuttama;

Kusalena samekkhasi, amanussāpi taṃ vandanti;

Sutaṃ netaṃ abhiṇhaso, tasmā evaṃ vademase.

‘‘‘Jinaṃ vandatha gotamaṃ, jinaṃ vandāma gotamaṃ;

Vijjācaraṇasampannaṃ, buddhaṃ vandāma gotamaṃ’.

\paragraph{279.} ‘‘Yena petā pavuccanti, pisuṇā piṭṭhimaṃsikā.

Pāṇātipātino luddā\footnote{luddhā (pī. ka.)}, corā nekatikā janā.

‘‘Ito ‘sā dakkhiṇā disā’, iti naṃ ācikkhatī jano;

Yaṃ disaṃ abhipāleti, mahārājā yasassi so.

‘‘Kumbhaṇḍānaṃ adhipati, ‘virūḷho’ iti nāmaso;

Ramatī naccagītehi, kumbhaṇḍehi purakkhato.

‘‘Puttāpi tassa bahavo, ekanāmāti me sutaṃ;

Asīti dasa eko ca, indanāmā mahabbalā.

Te cāpi buddhaṃ disvāna, buddhaṃ ādiccabandhunaṃ;

Dūratova namassanti, mahantaṃ vītasāradaṃ.

‘‘Namo te purisājañña, namo te purisuttama;

Kusalena samekkhasi, amanussāpi taṃ vandanti;

Sutaṃ netaṃ abhiṇhaso, tasmā evaṃ vademase.

‘‘‘Jinaṃ vandatha gotamaṃ, jinaṃ vandāma gotamaṃ;

Vijjācaraṇasampannaṃ, buddhaṃ vandāma gotamaṃ’.

\paragraph{280.} ‘‘Yattha coggacchati sūriyo, ādicco maṇḍalī mahā.

Yassa coggacchamānassa, divasopi nirujjhati;

Yassa coggate sūriye, ‘saṃvarī’ti pavuccati.

‘‘Rahadopi tattha gambhīro, samuddo saritodako;

Evaṃ taṃ tattha jānanti, ‘samuddo saritodako’.

‘‘Ito ‘sā pacchimā disā’, iti naṃ ācikkhatī jano;

Yaṃ disaṃ abhipāleti, mahārājā yasassi so.

‘‘Nāgānañca adhipati, ‘virūpakkho’ti nāmaso;

Ramatī naccagītehi, nāgeheva purakkhato.

‘‘Puttāpi tassa bahavo, ekanāmāti me sutaṃ;

Asīti dasa eko ca, indanāmā mahabbalā.

Te cāpi buddhaṃ disvāna, buddhaṃ ādiccabandhunaṃ;

Dūratova namassanti, mahantaṃ vītasāradaṃ.

‘‘Namo te purisājañña, namo te purisuttama;

Kusalena samekkhasi, amanussāpi taṃ vandanti;

Sutaṃ netaṃ abhiṇhaso, tasmā evaṃ vademase.

‘‘‘Jinaṃ vandatha gotamaṃ, jinaṃ vandāma gotamaṃ;

Vijjācaraṇasampannaṃ, buddhaṃ vandāma gotamaṃ’.

\paragraph{281.} ‘‘Yena uttarakuruvho\footnote{uttarakurū rammā (sī. syā. pī.)}, mahāneru sudassano.

Manussā tattha jāyanti, amamā apariggahā.

‘‘Na te bījaṃ pavapanti, napi nīyanti naṅgalā;

Akaṭṭhapākimaṃ sāliṃ, paribhuñjanti mānusā.

‘‘Akaṇaṃ athusaṃ suddhaṃ, sugandhaṃ taṇḍulapphalaṃ;

Tuṇḍikīre pacitvāna, tato bhuñjanti bhojanaṃ.

‘‘Gāviṃ ekakhuraṃ katvā, anuyanti disodisaṃ;

Pasuṃ ekakhuraṃ katvā, anuyanti disodisaṃ.

‘‘Itthiṃ vā vāhanaṃ\footnote{itthī-vāhanaṃ (sī. pī.), itthīṃ vāhanaṃ (syā.)} katvā, anuyanti disodisaṃ;

Purisaṃ vāhanaṃ katvā, anuyanti disodisaṃ.

‘‘Kumāriṃ vāhanaṃ katvā, anuyanti disodisaṃ;

Kumāraṃ vāhanaṃ katvā, anuyanti disodisaṃ.

‘‘Te yāne abhiruhitvā,

Sabbā disā anupariyāyanti\footnote{anupariyanti (syā.)};

Pacārā tassa rājino.

‘‘Hatthiyānaṃ assayānaṃ, dibbaṃ yānaṃ upaṭṭhitaṃ;

Pāsādā sivikā ceva, mahārājassa yasassino.

‘‘Tassa ca nagarā ahu,

Antalikkhe sumāpitā;

Āṭānāṭā kusināṭā parakusināṭā,

Nāṭasuriyā\footnote{nāṭapuriyā (sī. pī.), nāṭapariyā (syā.)} parakusiṭanāṭā.

‘‘Uttarena kasivanto\footnote{kapivanto (sī. syā. pī)},

Janoghamaparena ca;

Navanavutiyo ambaraambaravatiyo,

Āḷakamandā nāma rājadhānī.

‘‘Kuverassa kho pana, mārisa, mahārājassa visāṇā nāma rājadhānī;

Tasmā kuvero mahārājā, ‘vessavaṇo’ti pavuccati.

‘‘Paccesanto pakāsenti, tatolā tattalā tatotalā;

Ojasi tejasi tatojasī, sūro rājā ariṭṭho nemi.

‘‘Rahadopi tattha dharaṇī nāma, yato meghā pavassanti;

Vassā yato patāyanti, sabhāpi tattha sālavatī\footnote{bhagalavatī (sī. syā. pī.)} nāma.

‘‘Yattha yakkhā payirupāsanti, tattha niccaphalā rukkhā;

Nānā dijagaṇā yutā, mayūrakoñcābhirudā;

Kokilādīhi vagguhi.

‘‘Jīvañjīvakasaddettha, atho oṭṭhavacittakā;

Kukkuṭakā\footnote{kukutthakā (sī. pī.)} kuḷīrakā, vane pokkharasātakā.

‘‘Sukasāḷikasaddettha, daṇḍamāṇavakāni ca;

Sobhati sabbakālaṃ sā, kuveranaḷinī sadā.

‘‘Ito ‘sā uttarā disā’, iti naṃ ācikkhatī jano;

Yaṃ disaṃ abhipāleti, mahārājā yasassi so.

‘‘Yakkhānañca adhipati, ‘kuvero’ iti nāmaso;

Ramatī naccagītehi, yakkheheva purakkhato.

‘‘Puttāpi tassa bahavo, ekanāmāti me sutaṃ;

Asīti dasa eko ca, indanāmā mahabbalā.

‘‘Te cāpi buddhaṃ disvāna, buddhaṃ ādiccabandhunaṃ;

Dūratova namassanti, mahantaṃ vītasāradaṃ.

‘‘Namo te purisājañña, namo te purisuttama;

Kusalena samekkhasi, amanussāpi taṃ vandanti;

Sutaṃ netaṃ abhiṇhaso, tasmā evaṃ vademase.

‘‘‘Jinaṃ vandatha gotamaṃ, jinaṃ vandāma gotamaṃ;

Vijjācaraṇasampannaṃ, buddhaṃ vandāma gotama’’’nti.

‘‘Ayaṃ kho sā, mārisa, āṭānāṭiyā rakkhā bhikkhūnaṃ bhikkhunīnaṃ upāsakānaṃ upāsikānaṃ guttiyā rakkhāya avihiṃsāya phāsuvihārāya.

\paragraph{282.} ‘‘Yassa kassaci, mārisa, bhikkhussa vā bhikkhuniyā vā upāsakassa vā upāsikāya vā ayaṃ āṭānāṭiyā rakkhā suggahitā bhavissati samattā pariyāputā\footnote{pariyāpuṭā (ka.)}. Taṃ ce amanusso yakkho vā yakkhinī vā yakkhapotako vā yakkhapotikā vā yakkhamahāmatto vā yakkhapārisajjo vā yakkhapacāro vā, gandhabbo vā gandhabbī vā gandhabbapotako vā gandhabbapotikā vā gandhabbamahāmatto vā gandhabbapārisajjo vā gandhabbapacāro vā, kumbhaṇḍo vā kumbhaṇḍī vā kumbhaṇḍapotako vā kumbhaṇḍapotikā vā kumbhaṇḍamahāmatto vā kumbhaṇḍapārisajjo vā kumbhaṇḍapacāro vā, nāgo vā nāgī vā nāgapotako vā nāgapotikā vā nāgamahāmatto vā nāgapārisajjo vā nāgapacāro vā, paduṭṭhacitto bhikkhuṃ vā bhikkhuniṃ vā upāsakaṃ vā upāsikaṃ vā gacchantaṃ vā anugaccheyya, ṭhitaṃ vā upatiṭṭheyya, nisinnaṃ vā upanisīdeyya, nipannaṃ vā upanipajjeyya. Na me so, mārisa, amanusso labheyya gāmesu vā nigamesu vā sakkāraṃ vā garukāraṃ vā. Na me so, mārisa, amanusso labheyya āḷakamandāya nāma rājadhāniyā vatthuṃ vā vāsaṃ vā. Na me so, mārisa, amanusso labheyya yakkhānaṃ samitiṃ gantuṃ. Apissu naṃ, mārisa, amanussā anāvayhampi naṃ kareyyuṃ avivayhaṃ. Apissu naṃ, mārisa, amanussā attāhipi paripuṇṇāhi paribhāsāhi paribhāseyyuṃ. Apissu naṃ, mārisa, amanussā rittaṃpissa pattaṃ sīse nikkujjeyyuṃ. Apissu naṃ, mārisa, amanussā sattadhāpissa muddhaṃ phāleyyuṃ.

‘‘Santi hi, mārisa, amanussā caṇḍā ruddhā\footnote{ruddā (sī. pī.)} rabhasā, te neva mahārājānaṃ ādiyanti, na mahārājānaṃ purisakānaṃ ādiyanti, na mahārājānaṃ purisakānaṃ purisakānaṃ ādiyanti. Te kho te, mārisa, amanussā mahārājānaṃ avaruddhā nāma vuccanti. Seyyathāpi, mārisa, rañño māgadhassa vijite mahācorā. Te neva rañño māgadhassa ādiyanti, na rañño māgadhassa purisakānaṃ ādiyanti, na rañño māgadhassa purisakānaṃ purisakānaṃ ādiyanti. Te kho te, mārisa, mahācorā rañño māgadhassa avaruddhā nāma vuccanti. Evameva kho, mārisa, santi amanussā caṇḍā ruddhā rabhasā, te neva mahārājānaṃ ādiyanti, na mahārājānaṃ purisakānaṃ ādiyanti, na mahārājānaṃ purisakānaṃ purisakānaṃ ādiyanti. Te kho te, mārisa, amanussā mahārājānaṃ avaruddhā nāma vuccanti. Yo hi koci, mārisa, amanusso yakkho vā yakkhinī vā…pe… gandhabbo vā gandhabbī vā … kumbhaṇḍo vā kumbhaṇḍī vā… nāgo vā nāgī vā nāgapotako vā nāgapotikā vā nāgamahāmatto vā nāgapārisajjo vā nāgapacāro vā paduṭṭhacitto bhikkhuṃ vā bhikkhuniṃ vā upāsakaṃ vā upāsikaṃ vā gacchantaṃ vā anugaccheyya, ṭhitaṃ vā upatiṭṭheyya, nisinnaṃ vā upanisīdeyya, nipannaṃ vā upanipajjeyya. Imesaṃ yakkhānaṃ mahāyakkhānaṃ senāpatīnaṃ mahāsenāpatīnaṃ ujjhāpetabbaṃ vikkanditabbaṃ viravitabbaṃ – ‘ayaṃ yakkho gaṇhāti, ayaṃ yakkho āvisati, ayaṃ yakkho heṭheti, ayaṃ yakkho viheṭheti, ayaṃ yakkho hiṃsati, ayaṃ yakkho vihiṃsati, ayaṃ yakkho na muñcatī’ti.

\paragraph{283.} ‘‘Katamesaṃ yakkhānaṃ mahāyakkhānaṃ senāpatīnaṃ mahāsenāpatīnaṃ?

‘‘Indo somo varuṇo ca, bhāradvājo pajāpati;

Candano kāmaseṭṭho ca, kinnughaṇḍu nighaṇḍu ca.

‘‘Panādo opamañño ca, devasūto ca mātali;

Cittaseno ca gandhabbo, naḷo rājā janesabho\footnote{janosabho (syā.)}.

‘‘Sātāgiro hemavato, puṇṇako karatiyo guḷo;

Sivako mucalindo ca, vessāmitto yugandharo.

‘‘Gopālo supparodho ca\footnote{suppagedho ca (sī. syā. pī.)}, hiri netti\footnote{hirī nettī (sī. pī.)} ca mandiyo;

Pañcālacaṇḍo āḷavako, pajjunno sumano sumukho;

Dadhimukho maṇi māṇivaro\footnote{maṇi mānicaro (syā. pī.)} dīgho, atho serīsako saha.

‘‘Imesaṃ yakkhānaṃ mahāyakkhānaṃ senāpatīnaṃ mahāsenāpatīnaṃ ujjhāpetabbaṃ vikkanditabbaṃ viravitabbaṃ – ‘ayaṃ yakkho gaṇhāti, ayaṃ yakkho āvisati, ayaṃ yakkho heṭheti, ayaṃ yakkho viheṭheti, ayaṃ yakkho hiṃsati, ayaṃ yakkho vihiṃsati, ayaṃ yakkho na muñcatī’ti.

‘‘Ayaṃ kho sā, mārisa, āṭānāṭiyā rakkhā bhikkhūnaṃ bhikkhunīnaṃ upāsakānaṃ upāsikānaṃ guttiyā rakkhāya avihiṃsāya phāsuvihārāya. Handa ca dāni mayaṃ, mārisa, gacchāma bahukiccā mayaṃ bahukaraṇīyā’’ti. ‘‘Yassadāni tumhe mahārājāno kālaṃ maññathā’’ti.

\paragraph{284.} Atha kho cattāro mahārājā uṭṭhāyāsanā bhagavantaṃ abhivādetvā padakkhiṇaṃ katvā tatthevantaradhāyiṃsu. Tepi kho yakkhā uṭṭhāyāsanā appekacce bhagavantaṃ abhivādetvā padakkhiṇaṃ katvā tatthevantaradhāyiṃsu. Appekacce bhagavatā saddhiṃ sammodiṃsu, sammodanīyaṃ kathaṃ sāraṇīyaṃ vītisāretvā tatthevantaradhāyiṃsu . Appekacce yena bhagavā tenañjaliṃ paṇāmetvā tatthevantaradhāyiṃsu. Appekacce nāmagottaṃ sāvetvā tatthevantaradhāyiṃsu. Appekacce tuṇhībhūtā tatthevantaradhāyiṃsūti.

\xsubsubsectionEnd{Paṭhamabhāṇavāro niṭṭhito.}

\subsubsection{Dutiyabhāṇavāro}

\paragraph{285.} Atha kho bhagavā tassā rattiyā accayena bhikkhū āmantesi – ‘‘imaṃ, bhikkhave, rattiṃ cattāro mahārājā mahatiyā ca yakkhasenāya mahatiyā ca gandhabbasenāya mahatiyā ca kumbhaṇḍasenāya mahatiyā ca nāgasenāya catuddisaṃ rakkhaṃ ṭhapetvā catuddisaṃ gumbaṃ ṭhapetvā catuddisaṃ ovaraṇaṃ ṭhapetvā abhikkantāya rattiyā abhikkantavaṇṇā kevalakappaṃ gijjhakūṭaṃ pabbataṃ obhāsetvā yenāhaṃ tenupasaṅkamiṃsu; upasaṅkamitvā maṃ abhivādetvā ekamantaṃ nisīdiṃsu. Tepi kho, bhikkhave, yakkhā appekacce maṃ abhivādetvā ekamantaṃ nisīdiṃsu. Appekacce mayā saddhiṃ sammodiṃsu, sammodanīyaṃ kathaṃ sāraṇīyaṃ vītisāretvā ekamantaṃ nisīdiṃsu. Appekacce yenāhaṃ tenañjaliṃ paṇāmetvā ekamantaṃ nisīdiṃsu. Appekacce nāmagottaṃ sāvetvā ekamantaṃ nisīdiṃsu. Appekacce tuṇhībhūtā ekamantaṃ nisīdiṃsu.

\paragraph{286.} ‘‘Ekamantaṃ nisinno kho, bhikkhave, vessavaṇo mahārājā maṃ etadavoca – ‘santi hi, bhante, uḷārā yakkhā bhagavato appasannā…pe… santi hi , bhante nīcā yakkhā bhagavato pasannā. Yebhuyyena kho pana, bhante, yakkhā appasannāyeva bhagavato. Taṃ kissa hetu? Bhagavā hi, bhante, pāṇātipātā veramaṇiyā dhammaṃ deseti… surāmerayamajjappamādaṭṭhānā veramaṇiyā dhammaṃ deseti. Yebhuyyena kho pana, bhante, yakkhā appaṭiviratāyeva pāṇātipātā… appaṭiviratā surāmerayamajjappamādaṭṭhānā. Tesaṃ taṃ hoti appiyaṃ amanāpaṃ. Santi hi, bhante, bhagavato sāvakā araññavanapatthāni pantāni senāsanāni paṭisevanti appasaddāni appanigghosāni vijanavātāni manussarāhasseyyakāni paṭisallānasāruppāni. Tattha santi uḷārā yakkhā nivāsino, ye imasmiṃ bhagavato pāvacane appasannā, tesaṃ pasādāya uggaṇhātu, bhante, bhagavā āṭānāṭiyaṃ rakkhaṃ bhikkhūnaṃ bhikkhunīnaṃ upāsakānaṃ upāsikānaṃ guttiyā rakkhāya avihiṃsāya phāsuvihārāyā’ti. Adhivāsesiṃ kho ahaṃ, bhikkhave, tuṇhībhāvena. Atha kho, bhikkhave, vessavaṇo mahārājā me adhivāsanaṃ viditvā tāyaṃ velāyaṃ imaṃ āṭānāṭiyaṃ rakkhaṃ abhāsi –

\paragraph{287.} ‘Vipassissa ca namatthu, cakkhumantassa sirīmato.

Sikhissapi ca namatthu, sabbabhūtānukampino.

‘Vessabhussa ca namatthu, nhātakassa tapassino;

Namatthu kakusandhassa, mārasenāpamaddino.

‘Koṇāgamanassa namatthu, brāhmaṇassa vusīmato;

Kassapassa ca namatthu, vippamuttassa sabbadhi.

‘Aṅgīrasassa namatthu, sakyaputtassa sirīmato;

Yo imaṃ dhammaṃ desesi, sabbadukkhāpanūdanaṃ.

‘Ye cāpi nibbutā loke, yathābhūtaṃ vipassisuṃ;

Te janā apisuṇātha, mahantā vītasāradā.

‘Hitaṃ devamanussānaṃ, yaṃ namassanti gotamaṃ;

Vijjācaraṇasampannaṃ, mahantaṃ vītasāradaṃ.

\paragraph{288.} ‘Yato uggacchati sūriyo, ādicco maṇḍalī mahā.

Yassa cuggacchamānassa, saṃvarīpi nirujjhati;

Yassa cuggate sūriye, ‘‘divaso’’ti pavuccati.

‘Rahadopi tattha gambhīro, samuddo saritodako;

Evaṃ taṃ tattha jānanti, ‘‘samuddo saritodako’’.

‘Ito ‘‘sā purimā disā’’, iti naṃ ācikkhatī jano;

Yaṃ disaṃ abhipāleti, mahārājā yasassi so.

‘Gandhabbānaṃ adhipati, ‘‘dhataraṭṭho’’ti nāmaso;

Ramatī naccagītehi, gandhabbehi purakkhato.

‘Puttāpi tassa bahavo, ekanāmāti me sutaṃ;

Asīti dasa eko ca, indanāmā mahabbalā.

‘Te cāpi buddhaṃ disvāna, buddhaṃ ādiccabandhunaṃ;

Dūratova namassanti, mahantaṃ vītasāradaṃ.

‘Namo te purisājañña, namo te purisuttama;

Kusalena samekkhasi, amanussāpi taṃ vandanti;

Sutaṃ netaṃ abhiṇhaso, tassā evaṃ vademase.

‘‘Jinaṃ vandatha gotamaṃ, jinaṃ vandāma gotamaṃ;

Vijjācaraṇasampannaṃ, buddhaṃ vandāma gotamaṃ’’.

\paragraph{289.} ‘Yena petā pavuccanti, pisuṇā piṭṭhimaṃsikā.

Pāṇātipātino luddā, corā nekatikā janā.

‘Ito ‘‘sā dakkhiṇā disā’’, iti naṃ ācikkhatī jano;

Yaṃ disaṃ abhipāleti, mahārājā yasassi so.

‘Kumbhaṇḍānaṃ adhipati, ‘‘virūḷho’’ iti nāmaso;

Ramatī naccagītehi, kumbhaṇḍehi purakkhato.

‘Puttāpi tassa bahavo, ekanāmāti me sutaṃ;

Asīti dasa eko ca, indanāmā mahabbalā.

‘Te cāpi buddhaṃ disvāna, buddhaṃ ādiccabandhunaṃ;

Dūratova namassanti, mahantaṃ vītasāradaṃ.

‘Namo te purisājañña, namo te purisuttama;

Kusalena samekkhasi, amanussāpi taṃ vandanti;

Sutaṃ netaṃ abhiṇhaso, tasmā evaṃ vademase.

‘‘Jinaṃ vandatha gotamaṃ, jinaṃ vandāma gotamaṃ;

Vijjācaraṇasampannaṃ, buddhaṃ vandāma gotamaṃ’’.

\paragraph{290.} ‘Yattha coggacchati sūriyo, ādicco maṇḍalī mahā.

Yassa coggacchamānassa, divasopi nirujjhati;

Yassa coggate sūriye, ‘‘saṃvarī’’ti pavuccati.

‘Rahadopi tattha gambhīro, samuddo saritodako;

Evaṃ taṃ tattha jānanti, samuddo saritodako.

‘Ito ‘‘sā pacchimā disā’’, iti naṃ ācikkhatī jano;

Yaṃ disaṃ abhipāleti, mahārājā yasassi so.

‘Nāgānañca adhipati, ‘‘virūpakkho’’ti nāmaso;

Ramatī naccagītehi, nāgeheva purakkhato.

‘Puttāpi tassa bahavo, ekanāmāti me sutaṃ;

Asīti dasa eko ca, indanāmā mahabbalā.

‘Te cāpi buddhaṃ disvāna, buddhaṃ ādiccabandhunaṃ;

Dūratova namassanti, mahantaṃ vītasāradaṃ.

‘Namo te purisājañña, namo te purisuttama;

Kusalena samekkhasi, amanussāpi taṃ vandanti;

Sutaṃ netaṃ abhiṇhaso, tasmā evaṃ vademase.

‘‘Jinaṃ vandatha gotamaṃ, jinaṃ vandāma gotamaṃ;

Vijjācaraṇasampannaṃ, buddhaṃ vandāma gotamaṃ’’.

\paragraph{291.} ‘Yena uttarakuruvho, mahāneru sudassano.

Manussā tattha jāyanti, amamā apariggahā.

‘Na te bījaṃ pavapanti, nāpi nīyanti naṅgalā;

Akaṭṭhapākimaṃ sāliṃ, paribhuñjanti mānusā.

‘Akaṇaṃ athusaṃ suddhaṃ, sugandhaṃ taṇḍulapphalaṃ;

Tuṇḍikīre pacitvāna, tato bhuñjanti bhojanaṃ.

‘Gāviṃ ekakhuraṃ katvā, anuyanti disodisaṃ;

Pasuṃ ekakhuraṃ katvā, anuyanti disodisaṃ.

‘Itthiṃ vā vāhanaṃ katvā, anuyanti disodisaṃ;

Purisaṃ vāhanaṃ katvā, anuyanti disodisaṃ.

‘Kumāriṃ vāhanaṃ katvā, anuyanti disodisaṃ;

Kumāraṃ vāhanaṃ katvā, anuyanti disodisaṃ.

‘Te yāne abhiruhitvā,

Sabbā disā anupariyāyanti;

Pacārā tassa rājino.

‘Hatthiyānaṃ assayānaṃ,

Dibbaṃ yānaṃ upaṭṭhitaṃ;

Pāsādā sivikā ceva,

Mahārājassa yasassino.

‘Tassa ca nagarā ahu,

Antalikkhe sumāpitā;

Āṭānāṭā kusināṭā parakusināṭā,

Nāṭasuriyā parakusiṭanāṭā.

‘Uttarena kasivanto,

Janoghamaparena ca;

Navanavutiyo ambaraambaravatiyo,

Āḷakamandā nāma rājadhānī.

‘Kuverassa kho pana, mārisa, mahārājassa visāṇā nāma rājadhānī;

Tasmā kuvero mahārājā, ‘‘vessavaṇo’’ti pavuccati.

‘Paccesanto pakāsenti, tatolā tattalā tatotalā;

Ojasi tejasi tatojasī, sūro rājā ariṭṭho nemi.

‘Rahadopi tattha dharaṇī nāma, yato meghā pavassanti;

Vassā yato patāyanti, sabhāpi tattha sālavatī nāma.

‘Yattha yakkhā payirupāsanti, tattha niccaphalā rukkhā;

Nānā dijagaṇā yutā, mayūrakoñcābhirudā;

Kokilādīhi vagguhi.

‘Jīvañjīvakasaddettha, atho oṭṭhavacittakā;

Kukkuṭakā kuḷīrakā, vane pokkharasātakā.

‘Sukasāḷika saddettha, daṇḍamāṇavakāni ca;

Sobhati sabbakālaṃ sā, kuveranaḷinī sadā.

‘Ito ‘‘sā uttarā disā’’, iti naṃ ācikkhatī jano;

Yaṃ disaṃ abhipāleti, mahārājā yasassi so.

‘Yakkhānañca adhipati, ‘‘kuvero’’ iti nāmaso;

Ramatī naccagītehi, yakkheheva purakkhato.

‘Puttāpi tassa bahavo, ekanāmāti me sutaṃ;

Asīti dasa eko ca, indanāmā mahabbalā.

‘Te cāpi buddhaṃ disvāna, buddhaṃ ādiccabandhunaṃ;

Dūratova namassanti, mahantaṃ vītasāradaṃ.

‘Namo te purisājañña, namo te purisuttama;

Kusalena samekkhasi, amanussāpi taṃ vandanti;

Sutaṃ netaṃ abhiṇhaso, tasmā evaṃ vademase.

‘‘Jinaṃ vandatha gotamaṃ, jinaṃ vandāma gotamaṃ;

Vijjācaraṇasampannaṃ, buddhaṃ vandāma gotama’’nti.

\paragraph{292.} ‘Ayaṃ kho sā, mārisa , āṭānāṭiyā rakkhā bhikkhūnaṃ bhikkhunīnaṃ upāsakānaṃ upāsikānaṃ guttiyā rakkhāya avihiṃsāya phāsuvihārāya. Yassa kassaci, mārisa, bhikkhussa vā bhikkhuniyā vā upāsakassa vā upāsikāya vā ayaṃ āṭānāṭiyā rakkhā suggahitā bhavissati samattā pariyāputā taṃ ce amanusso yakkho vā yakkhinī vā…pe… gandhabbo vā gandhabbī vā…pe… kumbhaṇḍo vā kumbhaṇḍī vā…pe… nāgo vā nāgī vā nāgapotako vā nāgapotikā vā nāgamahāmatto vā nāgapārisajjo vā nāgapacāro vā, paduṭṭhacitto bhikkhuṃ vā bhikkhuniṃ vā upāsakaṃ vā upāsikaṃ vā gacchantaṃ vā anugaccheyya, ṭhitaṃ vā upatiṭṭheyya, nisinnaṃ vā upanisīdeyya, nipannaṃ vā upanipajjeyya. Na me so, mārisa, amanusso labheyya gāmesu vā nigamesu vā sakkāraṃ vā garukāraṃ vā. Na me so, mārisa, amanusso labheyya āḷakamandāya nāma rājadhāniyā vatthuṃ vā vāsaṃ vā. Na me so, mārisa, amanusso labheyya yakkhānaṃ samitiṃ gantuṃ. Apissu naṃ, mārisa, amanussā anāvayhampi naṃ kareyyuṃ avivayhaṃ. Apissu naṃ, mārisa, amanussā attāhi paripuṇṇāhi paribhāsāhi paribhāseyyuṃ. Apissu naṃ, mārisa, amanussā rittaṃpissa pattaṃ sīse nikkujjeyyuṃ. Apissu naṃ, mārisa, amanussā sattadhāpissa muddhaṃ phāleyyuṃ. Santi hi, mārisa, amanussā caṇḍā ruddhā rabhasā, te neva mahārājānaṃ ādiyanti, na mahārājānaṃ purisakānaṃ ādiyanti, na mahārājānaṃ purisakānaṃ purisakānaṃ ādiyanti. Te kho te, mārisa, amanussā mahārājānaṃ avaruddhā nāma vuccanti. Seyyathāpi, mārisa, rañño māgadhassa vijite mahācorā. Te neva rañño māgadhassa ādiyanti, na rañño māgadhassa purisakānaṃ ādiyanti, na rañño māgadhassa purisakānaṃ purisakānaṃ ādiyanti. Te kho te, mārisa, mahācorā rañño māgadhassa avaruddhā nāma vuccanti. Evameva kho, mārisa, santi amanussā caṇḍā ruddhā rabhasā, te neva mahārājānaṃ ādiyanti, na mahārājānaṃ purisakānaṃ ādiyanti, na mahārājānaṃ purisakānaṃ purisakānaṃ ādiyanti. Te kho te, mārisa, amanussā mahārājānaṃ avaruddhā nāma vuccanti. Yo hi koci, mārisa, amanusso yakkho vā yakkhinī vā…pe… gandhabbo vā gandhabbī vā…pe… kumbhaṇḍo vā kumbhaṇḍī vā…pe… nāgo vā nāgī vā…pe… paduṭṭhacitto bhikkhuṃ vā bhikkhuniṃ vā upāsakaṃ vā upāsikaṃ vā gacchantaṃ vā upagaccheyya, ṭhitaṃ vā upatiṭṭheyya, nisinnaṃ vā upanisīdeyya, nipannaṃ vā upanipajjeyya. Imesaṃ yakkhānaṃ mahāyakkhānaṃ senāpatīnaṃ mahāsenāpatīnaṃ ujjhāpetabbaṃ vikkanditabbaṃ viravitabbaṃ – ‘ayaṃ yakkho gaṇhāti, ayaṃ yakkho āvisati, ayaṃ yakkho heṭheti, ayaṃ yakkho viheṭheti, ayaṃ yakkho hiṃsati, ayaṃ yakkho vihiṃsati, ayaṃ yakkho na muñcatī’ti.

\paragraph{293.} ‘Katamesaṃ yakkhānaṃ mahāyakkhānaṃ senāpatīnaṃ mahāsenāpatīnaṃ?

‘Indo somo varuṇo ca, bhāradvājo pajāpati;

Candano kāmaseṭṭho ca, kinnughaṇḍu nighaṇḍu ca.

‘Panādo opamañño ca, devasūto ca mātali;

Cittaseno ca gandhabbo, naḷo rājā janesabho.

‘Sātāgiro hevamato, puṇṇako karatiyo guḷo;

Sivako mucalindo ca, vessāmitto yugandharo.

‘Gopālo supparodho ca, hiri netti ca mandiyo;

Pañcālacaṇḍo āḷavako, pajjunno sumano sumukho;

Dadhimukho maṇi māṇivaro dīgho, atho serīsako saha.

‘Imesaṃ yakkhānaṃ mahāyakkhānaṃ senāpatīnaṃ mahāsenāpatīnaṃ ujjhāpetabbaṃ vikkanditabbaṃ viravitabbaṃ – ‘‘ayaṃ yakkho gaṇhāti, ayaṃ yakkho āvisati, ayaṃ yakkho heṭheti, ayaṃ yakkho viheṭheti, ayaṃ yakkho hiṃsati, ayaṃ yakkho vihiṃsati, ayaṃ yakkho na muñcatī’’ti. Ayaṃ kho, mārisa, āṭānāṭiyā rakkhā bhikkhūnaṃ bhikkhunīnaṃ upāsakānaṃ upāsikānaṃ guttiyā rakkhāya avihiṃsāya phāsuvihārāya. Handa ca dāni mayaṃ, mārisa, gacchāma, bahukiccā mayaṃ bahukaraṇīyā’’’ti. ‘‘‘Yassa dāni tumhe mahārājāno kālaṃ maññathā’’’ti.

\paragraph{294.} ‘‘Atha kho, bhikkhave, cattāro mahārājā uṭṭhāyāsanā maṃ abhivādetvā padakkhiṇaṃ katvā tatthevantaradhāyiṃsu. Tepi kho, bhikkhave, yakkhā uṭṭhāyāsanā appekacce maṃ abhivādetvā padakkhiṇaṃ katvā tatthevantaradhāyiṃsu. Appekacce mayā saddhiṃ sammodiṃsu, sammodanīyaṃ kathaṃ sāraṇīyaṃ vītisāretvā tatthevantaradhāyiṃsu. Appekacce yenāhaṃ tenañjaliṃ paṇāmetvā tatthevantaradhāyiṃsu. Appekacce nāmagottaṃ sāvetvā tatthevantaradhāyiṃsu. Appekacce tuṇhībhūtā tatthevantaradhāyiṃsu.

\paragraph{295.} ‘‘Uggaṇhātha , bhikkhave, āṭānāṭiyaṃ rakkhaṃ. Pariyāpuṇātha, bhikkhave, āṭānāṭiyaṃ rakkhaṃ. Dhāretha, bhikkhave, āṭānāṭiyaṃ rakkhaṃ. Atthasaṃhitā\footnote{atthasaṃhitāyaṃ (syā.)}, bhikkhave, āṭānāṭiyā rakkhā bhikkhūnaṃ bhikkhunīnaṃ upāsakānaṃ upāsikānaṃ guttiyā rakkhāya avihiṃsāya phāsuvihārāyā’’ti. Idamavoca bhagavā. Attamanā te bhikkhū bhagavato bhāsitaṃ abhinandunti.

\xsectionEnd{Āṭānāṭiyasuttaṃ niṭṭhitaṃ navamaṃ.}


\section{Saṅgītisuttaṃ}

\paragraph{296.} Evaṃ me sutaṃ – ekaṃ samayaṃ bhagavā mallesu cārikaṃ caramāno mahatā bhikkhusaṅghena saddhiṃ pañcamattehi bhikkhusatehi yena pāvā nāma mallānaṃ nagaraṃ tadavasari. Tatra sudaṃ bhagavā pāvāyaṃ viharati cundassa kammāraputtassa ambavane.

\subsubsection{Ubbhatakanavasandhāgāraṃ}

\paragraph{297.} Tena kho pana samayena pāveyyakānaṃ mallānaṃ ubbhatakaṃ nāma navaṃ sandhāgāraṃ\footnote{santhāgāraṃ (sī. pī.), saṇṭhāgāraṃ (syā. kaṃ.)} acirakāritaṃ hoti anajjhāvuṭṭhaṃ\footnote{anajjhāvutthaṃ (sī. syā. pī. ka.)} samaṇena vā brāhmaṇena vā kenaci vā manussabhūtena. Assosuṃ kho pāveyyakā mallā – ‘‘bhagavā kira mallesu cārikaṃ caramāno mahatā bhikkhusaṅghena saddhiṃ pañcamattehi bhikkhusatehi pāvaṃ anuppatto pāvāyaṃ viharati cundassa kammāraputtassa ambavane’’ti. Atha kho pāveyyakā mallā yena bhagavā tenupasaṅkamiṃsu; upasaṅkamitvā bhagavantaṃ abhivādetvā ekamantaṃ nisīdiṃsu. Ekamantaṃ nisinnā kho pāveyyakā mallā bhagavantaṃ etadavocuṃ – ‘‘idha, bhante, pāveyyakānaṃ mallānaṃ ubbhatakaṃ nāma navaṃ sandhāgāraṃ acirakāritaṃ hoti anajjhāvuṭṭhaṃ samaṇena vā brāhmaṇena vā kenaci vā manussabhūtena. Tañca kho, bhante, bhagavā paṭhamaṃ paribhuñjatu, bhagavatā paṭhamaṃ paribhuttaṃ pacchā pāveyyakā mallā paribhuñjissanti. Tadassa pāveyyakānaṃ mallānaṃ dīgharattaṃ hitāya sukhāyā’’ti. Adhivāsesi kho bhagavā tuṇhībhāvena.

\paragraph{298.} Atha kho pāveyyakā mallā bhagavato adhivāsanaṃ viditvā uṭṭhāyāsanā bhagavantaṃ abhivādetvā padakkhiṇaṃ katvā yena sandhāgāraṃ tenupasaṅkamiṃsu; upasaṅkamitvā sabbasanthariṃ\footnote{sabbasanthariṃ santhataṃ (ka.)} sandhāgāraṃ santharitvā bhagavato āsanāni paññāpetvā udakamaṇikaṃ patiṭṭhapetvā telapadīpaṃ āropetvā yena bhagavā tenupasaṅkamiṃsu; upasaṅkamitvā bhagavantaṃ abhivādetvā ekamantaṃ aṭṭhaṃsu. Ekamantaṃ ṭhitā kho te pāveyyakā mallā bhagavantaṃ etadavocuṃ – ‘‘sabbasantharisanthataṃ\footnote{sabbasanthariṃ santhataṃ (sī. pī. ka.)}, bhante, sandhāgāraṃ, bhagavato āsanāni paññattāni, udakamaṇiko patiṭṭhāpito, telapadīpo āropito. Yassadāni, bhante, bhagavā kālaṃ maññatī’’ti.

\paragraph{299.} Atha kho bhagavā nivāsetvā pattacīvaramādāya saddhiṃ bhikkhusaṅghena yena sandhāgāraṃ tenupasaṅkami; upasaṅkamitvā pāde pakkhāletvā sandhāgāraṃ pavisitvā majjhimaṃ thambhaṃ nissāya puratthābhimukho nisīdi. Bhikkhusaṅghopi kho pāde pakkhāletvā sandhāgāraṃ pavisitvā pacchimaṃ bhittiṃ nissāya puratthābhimukho nisīdi bhagavantaṃyeva purakkhatvā. Pāveyyakāpi kho mallā pāde pakkhāletvā sandhāgāraṃ pavisitvā puratthimaṃ bhittiṃ nissāya pacchimābhimukhā nisīdiṃsu bhagavantaṃyeva purakkhatvā. Atha kho bhagavā pāveyyake malle bahudeva rattiṃ dhammiyā kathāya sandassetvā samādapetvā samuttejetvā sampahaṃsetvā uyyojesi – ‘‘abhikkantā kho, vāseṭṭhā, ratti. Yassadāni tumhe kālaṃ maññathā’’ti. ‘‘Evaṃ, bhante’’ti kho pāveyyakā mallā bhagavato paṭissutvā uṭṭhāyāsanā bhagavantaṃ abhivādetvā padakkhiṇaṃ katvā pakkamiṃsu.

\paragraph{300.} Atha kho bhagavā acirapakkantesu pāveyyakesu mallesu tuṇhībhūtaṃ tuṇhībhūtaṃ bhikkhusaṃghaṃ anuviloketvā āyasmantaṃ sāriputtaṃ āmantesi – ‘‘vigatathinamiddho\footnote{vigatathīnamiddho (sī. syā. kaṃ. pī.)} kho, sāriputta, bhikkhusaṅgho. Paṭibhātu taṃ, sāriputta, bhikkhūnaṃ dhammīkathā. Piṭṭhi me āgilāyati. Tamahaṃ āyamissāmī’’ti\footnote{āyameyyāmīti (syā. kaṃ.)}. ‘‘Evaṃ, bhante’’ti kho āyasmā sāriputto bhagavato paccassosi. Atha kho bhagavā catugguṇaṃ saṅghāṭiṃ paññapetvā dakkhiṇena passena sīhaseyyaṃ kappesi pāde pādaṃ accādhāya, sato sampajāno uṭṭhānasaññaṃ manasi karitvā.

\subsubsection{Bhinnanigaṇṭhavatthu}

\paragraph{301.} Tena kho pana samayena nigaṇṭho nāṭaputto pāvāyaṃ adhunākālaṅkato hoti. Tassa kālaṅkiriyāya bhinnā nigaṇṭhā dvedhikajātā\footnote{ddheḷhakajātā (syā. kaṃ.)} bhaṇḍanajātā kalahajātā vivādāpannā aññamaññaṃ mukhasattīhi vitudantā viharanti\footnote{vicaranti (syā. kaṃ.)} – ‘‘na tvaṃ imaṃ dhammavinayaṃ ājānāsi, ahaṃ imaṃ dhammavinayaṃ ājānāmi, kiṃ tvaṃ imaṃ dhammavinayaṃ ājānissasi! Micchāpaṭipanno tvamasi, ahamasmi sammāpaṭipanno. Sahitaṃ me, asahitaṃ te. Purevacanīyaṃ pacchā avaca, pacchāvacanīyaṃ pure avaca. Adhiciṇṇaṃ te viparāvattaṃ, āropito te vādo, niggahito tvamasi, cara vādappamokkhāya, nibbeṭhehi vā sace pahosī’’ti. Vadhoyeva kho maññe nigaṇṭhesu nāṭaputtiyesu vattati. Yepi\footnote{yepi te (sī. pī.)} nigaṇṭhassa nāṭaputtassa sāvakā gihī odātavasanā , tepi nigaṇṭhesu nāṭaputtiyesu nibbinnarūpā virattarūpā paṭivānarūpā, yathā taṃ durakkhāte dhammavinaye duppavedite aniyyānike anupasamasaṃvattanike asammāsambuddhappavedite bhinnathūpe appaṭisaraṇe.

\paragraph{302.} Atha kho āyasmā sāriputto bhikkhū āmantesi – ‘‘nigaṇṭho, āvuso, nāṭaputto pāvāyaṃ adhunākālaṅkato, tassa kālaṅkiriyāya bhinnā nigaṇṭhā dvedhikajātā…pe… bhinnathūpe appaṭisaraṇe’’. ‘‘Evañhetaṃ, āvuso, hoti durakkhāte dhammavinaye duppavedite aniyyānike anupasamasaṃvattanike asammāsambuddhappavedite. Ayaṃ kho panāvuso amhākaṃ\footnote{asmākaṃ (pī.)} bhagavatā\footnote{bhagavato (ka. sī.)} dhammo svākkhāto suppavedito niyyāniko upasamasaṃvattaniko sammāsambuddhappavedito. Tattha sabbeheva saṅgāyitabbaṃ, na vivaditabbaṃ, yathayidaṃ brahmacariyaṃ addhaniyaṃ assa ciraṭṭhitikaṃ, tadassa bahujanahitāya bahujanasukhāya lokānukampāya atthāya hitāya sukhāya devamanussānaṃ.

‘‘Katamo cāvuso, amhākaṃ bhagavatā\footnote{bhagavato (ka. sī.)} dhammo svākkhāto suppavedito niyyāniko upasamasaṃvattaniko sammāsambuddhappavedito; yattha sabbeheva saṅgāyitabbaṃ, na vivaditabbaṃ, yathayidaṃ brahmacariyaṃ addhaniyaṃ assa ciraṭṭhitikaṃ, tadassa bahujanahitāya bahujanasukhāya lokānukampāya atthāya hitāya sukhāya devamanussānaṃ?

\subsubsection{Ekakaṃ}

\paragraph{303.} ‘‘Atthi kho, āvuso, tena bhagavatā jānatā passatā arahatā sammāsambuddhena eko dhammo sammadakkhāto. Tattha sabbeheva saṅgāyitabbaṃ, na vivaditabbaṃ, yathayidaṃ brahmacariyaṃ addhaniyaṃ assa ciraṭṭhitikaṃ , tadassa bahujanahitāya bahujanasukhāya lokānukampāya atthāya hitāya sukhāya devamanussānaṃ. Katamo eko dhammo? Sabbe sattā āhāraṭṭhitikā. Sabbe sattā saṅkhāraṭṭhitikā. Ayaṃ kho, āvuso, tena bhagavatā jānatā passatā arahatā sammāsambuddhena eko dhammo sammadakkhāto. Tattha sabbeheva saṅgāyitabbaṃ, na vivaditabbaṃ , yathayidaṃ brahmacariyaṃ addhaniyaṃ assa ciraṭṭhitikaṃ, tadassa bahujanahitāya bahujanasukhāya lokānukampāya atthāya hitāya sukhāya devamanussānaṃ.

\subsubsection{Dukaṃ}

\paragraph{304.} ‘‘Atthi kho, āvuso, tena bhagavatā jānatā passatā arahatā sammāsambuddhena dve dhammā sammadakkhātā. Tattha sabbeheva saṅgāyitabbaṃ, na vivaditabbaṃ, yathayidaṃ brahmacariyaṃ addhaniyaṃ assa ciraṭṭhitikaṃ, tadassa bahujanahitāya bahujanasukhāya lokānukampāya atthāya hitāya sukhāya devamanussānaṃ. Katame dve\footnote{dve dhammo (syā. kaṃ.) evamuparipi}?

‘‘Nāmañca rūpañca.

‘‘Avijjā ca bhavataṇhā ca.

‘‘Bhavadiṭṭhi ca vibhavadiṭṭhi ca.

‘‘Ahirikañca\footnote{ahirīkañca (katthaci)} anottappañca.

‘‘Hirī ca ottappañca.

‘‘Dovacassatā ca pāpamittatā ca.

‘‘Sovacassatā ca kalyāṇamittatā ca.

‘‘Āpattikusalatā ca āpattivuṭṭhānakusalatā ca.

‘‘Samāpattikusalatā ca samāpattivuṭṭhānakusalatā ca.

‘‘Dhātukusalatā ca manasikārakusalatā ca.

‘‘Āyatanakusalatā ca paṭiccasamuppādakusalatā ca.

‘‘Ṭhānakusalatā ca aṭṭhānakusalatā ca.

‘‘Ajjavañca lajjavañca.

‘‘Khanti ca soraccañca.

‘‘Sākhalyañca paṭisanthāro ca.

‘‘Avihiṃsā ca soceyyañca.

‘‘Muṭṭhassaccañca asampajaññañca.

‘‘Sati ca sampajaññañca .

‘‘Indriyesu aguttadvāratā ca bhojane amattaññutā ca.

‘‘Indriyesu guttadvāratā ca bhojane mattaññutā ca.

‘‘Paṭisaṅkhānabalañca\footnote{paṭisandhānabalañca (syā.)} bhāvanābalañca.

‘‘Satibalañca samādhibalañca.

‘‘Samatho ca vipassanā ca.

‘‘Samathanimittañca paggahanimittañca.

‘‘Paggaho ca avikkhepo ca.

‘‘Sīlavipatti ca diṭṭhivipatti ca.

‘‘Sīlasampadā ca diṭṭhisampadā ca.

‘‘Sīlavisuddhi ca diṭṭhivisuddhi ca.

‘‘Diṭṭhivisuddhi kho pana yathā diṭṭhissa ca padhānaṃ.

‘‘Saṃvego ca saṃvejanīyesu ṭhānesu saṃviggassa ca yoniso padhānaṃ.

‘‘Asantuṭṭhitā ca kusalesu dhammesu appaṭivānitā ca padhānasmiṃ.

‘‘Vijjā ca vimutti ca.

‘‘Khayeñāṇaṃ anuppādeñāṇaṃ.

‘‘Ime kho, āvuso, tena bhagavatā jānatā passatā arahatā sammāsambuddhena dve dhammā sammadakkhātā. Tattha sabbeheva saṅgāyitabbaṃ, na vivaditabbaṃ, yathayidaṃ brahmacariyaṃ addhaniyaṃ assa ciraṭṭhitikaṃ, tadassa bahujanahitāya bahujanasukhāya lokānukampāya atthāya hitāya sukhāya devamanussānaṃ.

\subsubsection{Tikaṃ}

\paragraph{305.} ‘‘Atthi kho, āvuso, tena bhagavatā jānatā passatā arahatā sammāsambuddhena tayo dhammā sammadakkhātā. Tattha sabbeheva saṅgāyitabbaṃ…pe… atthāya hitāya sukhāya devamanussānaṃ. Katame tayo?

‘‘Tīṇi akusalamūlāni – lobho akusalamūlaṃ, doso akusalamūlaṃ, moho akusalamūlaṃ.

‘‘Tīṇi kusalamūlāni – alobho kusalamūlaṃ, adoso kusalamūlaṃ, amoho kusalamūlaṃ.

‘‘Tīṇi duccaritāni – kāyaduccaritaṃ, vacīduccaritaṃ, manoduccaritaṃ.

‘‘Tīṇi sucaritāni – kāyasucaritaṃ, vacīsucaritaṃ , manosucaritaṃ.

‘‘Tayo akusalavitakkā – kāmavitakko, byāpādavitakko, vihiṃsāvitakko.

‘‘Tayo kusalavitakkā – nekkhammavitakko, abyāpādavitakko, avihiṃsāvitakko.

‘‘Tayo akusalasaṅkappā – kāmasaṅkappo, byāpādasaṅkappo, vihiṃsāsaṅkappo.

‘‘Tayo kusalasaṅkappā – nekkhammasaṅkappo, abyāpādasaṅkappo, avihiṃsāsaṅkappo.

‘‘Tisso akusalasaññā – kāmasaññā, byāpādasaññā, vihiṃsāsaññā.

‘‘Tisso kusalasaññā – nekkhammasaññā, abyāpādasaññā, avihiṃsāsaññā.

‘‘Tisso akusaladhātuyo – kāmadhātu, byāpādadhātu, vihiṃsādhātu.

‘‘Tisso kusaladhātuyo – nekkhammadhātu, abyāpādadhātu, avihiṃsādhātu.

‘‘Aparāpi tisso dhātuyo – kāmadhātu, rūpadhātu, arūpadhātu.

‘‘Aparāpi tisso dhātuyo – rūpadhātu, arūpadhātu, nirodhadhātu.

‘‘Aparāpi tisso dhātuyo – hīnadhātu, majjhimadhātu, paṇītadhātu.

‘‘Tisso taṇhā – kāmataṇhā, bhavataṇhā, vibhavataṇhā.

‘‘Aparāpi tisso taṇhā – kāmataṇhā, rūpataṇhā, arūpataṇhā.

‘‘Aparāpi tisso taṇhā – rūpataṇhā, arūpataṇhā, nirodhataṇhā.

‘‘Tīṇi saṃyojanāni – sakkāyadiṭṭhi, vicikicchā, sīlabbataparāmāso.

‘‘Tayo āsavā – kāmāsavo, bhavāsavo, avijjāsavo.

‘‘Tayo bhavā – kāmabhavo, rūpabhavo, arūpabhavo.

‘‘Tisso esanā – kāmesanā, bhavesanā, brahmacariyesanā.

‘‘Tisso vidhā – seyyohamasmīti vidhā, sadisohamasmīti vidhā, hīnohamasmīti vidhā.

‘‘Tayo addhā – atīto addhā, anāgato addhā, paccuppanno addhā.

‘‘Tayo antā – sakkāyo anto, sakkāyasamudayo anto, sakkāyanirodho anto.

‘‘Tisso vedanā – sukhā vedanā, dukkhā vedanā, adukkhamasukhā vedanā.

‘‘Tisso dukkhatā – dukkhadukkhatā, saṅkhāradukkhatā, vipariṇāmadukkhatā.

‘‘Tayo rāsī – micchattaniyato rāsi, sammattaniyato rāsi, aniyato rāsi.

‘‘Tayo tamā\footnote{tisso kaṅkhā (bahūsu) aṭṭhakathā oloketabbā} – atītaṃ vā addhānaṃ ārabbha kaṅkhati vicikicchati nādhimuccati na sampasīdati, anāgataṃ vā addhānaṃ ārabbha kaṅkhati vicikicchati nādhimuccati na sampasīdati, etarahi vā paccuppannaṃ addhānaṃ ārabbha kaṅkhati vicikicchati nādhimuccati na sampasīdati.

‘‘Tīṇi tathāgatassa arakkheyyāni – parisuddhakāyasamācāro āvuso tathāgato, natthi tathāgatassa kāyaduccaritaṃ, yaṃ tathāgato rakkheyya – ‘mā me idaṃ paro aññāsī’ti. Parisuddhavacīsamācāro āvuso, tathāgato, natthi tathāgatassa vacīduccaritaṃ, yaṃ tathāgato rakkheyya – ‘mā me idaṃ paro aññāsī’ti. Parisuddhamanosamācāro, āvuso, tathāgato, natthi tathāgatassa manoduccaritaṃ yaṃ tathāgato rakkheyya – ‘mā me idaṃ paro aññāsī’ti.

‘‘Tayo kiñcanā – rāgo kiñcanaṃ, doso kiñcanaṃ, moho kiñcanaṃ.

‘‘Tayo aggī – rāgaggi, dosaggi, mohaggi.

‘‘Aparepi tayo aggī – āhuneyyaggi, gahapataggi, dakkhiṇeyyaggi.

‘‘Tividhena rūpasaṅgaho – sanidassanasappaṭighaṃ rūpaṃ\footnote{sanidassanasappaṭigharūpaṃ (syā. kaṃ.) evamitaradvayepi}, anidassanasappaṭighaṃ rūpaṃ, anidassanaappaṭighaṃ rūpaṃ.

‘‘Tayo saṅkhārā – puññābhisaṅkhāro, apuññābhisaṅkhāro , āneñjābhisaṅkhāro.

‘‘Tayo puggalā – sekkho puggalo, asekkho puggalo, nevasekkhonāsekkho puggalo.

‘‘Tayo therā – jātithero, dhammathero, sammutithero\footnote{sammatithero (syā. kaṃ.)}.

‘‘Tīṇi puññakiriyavatthūni – dānamayaṃ puññakiriyavatthu, sīlamayaṃ puññakiriyavatthu, bhāvanāmayaṃ puññakiriyavatthu.

‘‘Tīṇi codanāvatthūni – diṭṭhena, sutena, parisaṅkāya.

‘‘Tisso kāmūpapattiyo\footnote{kāmuppattiyo (sī.), kāmupapattiyo (syā. pī. ka.)} – santāvuso sattā paccupaṭṭhitakāmā, te paccupaṭṭhitesu kāmesu vasaṃ vattenti, seyyathāpi manussā ekacce ca devā ekacce ca vinipātikā. Ayaṃ paṭhamā kāmūpapatti. Santāvuso, sattā nimmitakāmā, te nimminitvā nimminitvā kāmesu vasaṃ vattenti, seyyathāpi devā nimmānaratī. Ayaṃ dutiyā kāmūpapatti. Santāvuso sattā paranimmitakāmā, te paranimmitesu kāmesu vasaṃ vattenti, seyyathāpi devā paranimmitavasavattī. Ayaṃ tatiyā kāmūpapatti.

‘‘Tisso sukhūpapattiyo\footnote{sukhupapattiyo (syā. pī. ka.)} – santāvuso sattā\footnote{sattā sukhaṃ (syā. kaṃ.)} uppādetvā uppādetvā sukhaṃ viharanti, seyyathāpi devā brahmakāyikā. Ayaṃ paṭhamā sukhūpapatti. Santāvuso, sattā sukhena abhisannā parisannā paripūrā paripphuṭā. Te kadāci karahaci udānaṃ udānenti – ‘aho sukhaṃ, aho sukha’nti , seyyathāpi devā ābhassarā. Ayaṃ dutiyā sukhūpapatti. Santāvuso, sattā sukhena abhisannā parisannā paripūrā paripphuṭā. Te santaṃyeva tusitā\footnote{santusitā (syā. kaṃ.)} sukhaṃ\footnote{cittasukhaṃ (syā. ka.)} paṭisaṃvedenti, seyyathāpi devā subhakiṇhā. Ayaṃ tatiyā sukhūpapatti .

‘‘Tisso paññā – sekkhā paññā, asekkhā paññā, nevasekkhānāsekkhā paññā.

‘‘Aparāpi tisso paññā – cintāmayā paññā, sutamayā paññā, bhāvanāmayā paññā.

‘‘Tīṇāvudhāni – sutāvudhaṃ, pavivekāvudhaṃ, paññāvudhaṃ.

‘‘Tīṇindriyāni – anaññātaññassāmītindriyaṃ, aññindriyaṃ, aññātāvindriyaṃ.

‘‘Tīṇi cakkhūni – maṃsacakkhu, dibbacakkhu, paññācakkhu.

‘‘Tisso sikkhā – adhisīlasikkhā, adhicittasikkhā, adhipaññāsikkhā.

‘‘Tisso bhāvanā – kāyabhāvanā, cittabhāvanā, paññābhāvanā.

‘‘Tīṇi anuttariyāni – dassanānuttariyaṃ, paṭipadānuttariyaṃ, vimuttānuttariyaṃ.

‘‘Tayo samādhī – savitakkasavicāro samādhi, avitakkavicāramatto samādhi, avitakkaavicāro samādhi.

‘‘Aparepi tayo samādhī – suññato samādhi, animitto samādhi, appaṇihito samādhi.

‘‘Tīṇi soceyyāni – kāyasoceyyaṃ, vacīsoceyyaṃ, manosoceyyaṃ.

‘‘Tīṇi moneyyāni – kāyamoneyyaṃ, vacīmoneyyaṃ, manomoneyyaṃ.

‘‘Tīṇi kosallāni – āyakosallaṃ, apāyakosallaṃ, upāyakosallaṃ.

‘‘Tayo madā – ārogyamado, yobbanamado, jīvitamado.

‘‘Tīṇi ādhipateyyāni – attādhipateyyaṃ, lokādhipateyyaṃ, dhammādhipateyyaṃ.

‘‘Tīṇi kathāvatthūni – atītaṃ vā addhānaṃ ārabbha kathaṃ katheyya – ‘evaṃ ahosi atītamaddhāna’nti; anāgataṃ vā addhānaṃ ārabbha kathaṃ katheyya – ‘evaṃ bhavissati anāgatamaddhāna’nti; etarahi vā paccuppannaṃ addhānaṃ ārabbha kathaṃ katheyya – ‘evaṃ hoti etarahi paccuppannaṃ addhāna’nti.

‘‘Tisso vijjā – pubbenivāsānussatiñāṇaṃ vijjā, sattānaṃ cutūpapāteñāṇaṃ vijjā, āsavānaṃ khayeñāṇaṃ vijjā.

‘‘Tayo vihārā – dibbo vihāro, brahmā vihāro, ariyo vihāro.

‘‘Tīṇi pāṭihāriyāni – iddhipāṭihāriyaṃ, ādesanāpāṭihāriyaṃ, anusāsanīpāṭihāriyaṃ.

‘‘Ime kho, āvuso, tena bhagavatā jānatā passatā arahatā sammāsambuddhena tayo dhammā sammadakkhātā. Tattha sabbeheva saṅgāyitabbaṃ…pe… atthāya hitāya sukhāya devamanussānaṃ.

\subsubsection{Catukkaṃ}

\paragraph{306.} ‘‘Atthi kho, āvuso, tena bhagavatā jānatā passatā arahatā sammāsambuddhena cattāro dhammā sammadakkhātā. Tattha sabbeheva saṅgāyitabbaṃ, na vivaditabbaṃ…pe… atthāya hitāya sukhāya devamanussānaṃ. Katame cattāro?

‘‘Cattāro satipaṭṭhānā. Idhāvuso, bhikkhu kāye kāyānupassī viharati ātāpī sampajāno satimā, vineyya loke abhijjhādomanassaṃ. Vedanāsu vedanānupassī…pe… citte cittānupassī…pe… dhammesu dhammānupassī viharati ātāpī sampajāno satimā vineyya loke abhijjhādomanassaṃ.

‘‘Cattāro sammappadhānā. Idhāvuso, bhikkhu anuppannānaṃ pāpakānaṃ akusalānaṃ dhammānaṃ anuppādāya chandaṃ janeti vāyamati vīriyaṃ ārabhati cittaṃ paggaṇhāti padahati. Uppannānaṃ pāpakānaṃ akusalānaṃ dhammānaṃ pahānāya chandaṃ janeti vāyamati vīriyaṃ ārabhati cittaṃ paggaṇhāti padahati. Anuppannānaṃ kusalānaṃ dhammānaṃ uppādāya chandaṃ janeti vāyamati vīriyaṃ ārabhati cittaṃ paggaṇhāti padahati. Uppannānaṃ kusalānaṃ dhammānaṃ ṭhitiyā asammosāya bhiyyobhāvāya vepullāya bhāvanāya pāripūriyā chandaṃ janeti vāyamati vīriyaṃ ārabhati cittaṃ paggaṇhāti padahati.

‘‘Cattāro iddhipādā. Idhāvuso, bhikkhu chandasamādhipadhānasaṅkhārasamannāgataṃ iddhipādaṃ bhāveti. Cittasamādhipadhānasaṅkhārasamannāgataṃ iddhipādaṃ bhāveti. Vīriyasamādhipadhānasaṅkhārasamannāgataṃ iddhipādaṃ bhāveti. Vīmaṃsāsamādhipadhānasaṅkhārasamannāgataṃ iddhipādaṃ bhāveti.

‘‘Cattāri jhānāni. Idhāvuso, bhikkhu vivicceva kāmehi vivicca akusalehi dhammehi savitakkaṃ savicāraṃ vivekajaṃ pītisukhaṃ paṭhamaṃ jhānaṃ\footnote{paṭhamajjhānaṃ (syā. kaṃ.)} upasampajja viharati. Vitakkavicārānaṃ vūpasamā ajjhattaṃ sampasādanaṃ cetaso ekodibhāvaṃ avitakkaṃ avicāraṃ samādhijaṃ pītisukhaṃ dutiyaṃ jhānaṃ\footnote{dutiyajjhānaṃ (syā. kaṃ.)} upasampajja viharati. Pītiyā ca virāgā upekkhako ca viharati sato ca sampajāno, sukhañca kāyena paṭisaṃvedeti, yaṃ taṃ ariyā ācikkhanti – ‘upekkhako satimā sukhavihārī’ti tatiyaṃ jhānaṃ\footnote{tatiyajjhānaṃ (syā. kaṃ.)} upasampajja viharati. Sukhassa ca pahānā dukkhassa ca pahānā, pubbeva somanassadomanassānaṃ atthaṅgamā, adukkhamasukhaṃ upekkhāsatipārisuddhiṃ catutthaṃ jhānaṃ\footnote{catutthajjhānaṃ (syā. kaṃ.)} upasampajja viharati.

\paragraph{307.} ‘‘Catasso samādhibhāvanā. Atthāvuso, samādhibhāvanā bhāvitā bahulīkatā diṭṭhadhammasukhavihārāya saṃvattati. Atthāvuso, samādhibhāvanā bhāvitā bahulīkatā ñāṇadassanapaṭilābhāya saṃvattati. Atthāvuso samādhibhāvanā bhāvitā bahulīkatā satisampajaññāya saṃvattati. Atthāvuso samādhibhāvanā bhāvitā bahulīkatā āsavānaṃ khayāya saṃvattati.

‘‘Katamā cāvuso, samādhibhāvanā bhāvitā bahulīkatā diṭṭhadhammasukhavihārāya saṃvattati? Idhāvuso, bhikkhu vivicceva kāmehi vivicca akusalehi dhammehi savitakkaṃ…pe… catutthaṃ jhānaṃ upasampajja viharati. Ayaṃ, āvuso , samādhibhāvanā bhāvitā bahulīkatā diṭṭhadhammasukhavihārāya saṃvattati.

‘‘Katamā cāvuso, samādhibhāvanā bhāvitā bahulīkatā ñāṇadassanapaṭilābhāya saṃvattati? Idhāvuso, bhikkhu ālokasaññaṃ manasi karoti, divāsaññaṃ adhiṭṭhāti yathā divā tathā rattiṃ, yathā rattiṃ tathā divā. Iti vivaṭena cetasā apariyonaddhena sappabhāsaṃ cittaṃ bhāveti. Ayaṃ, āvuso samādhibhāvanā bhāvitā bahulīkatā ñāṇadassanapaṭilābhāya saṃvattati.

‘‘Katamā cāvuso, samādhibhāvanā bhāvitā bahulīkatā satisampajaññāya saṃvattati? Idhāvuso, bhikkhuno viditā vedanā uppajjanti, viditā upaṭṭhahanti, viditā abbhatthaṃ gacchanti. Viditā saññā uppajjanti, viditā upaṭṭhahanti, viditā abbhatthaṃ gacchanti. Viditā vitakkā uppajjanti, viditā upaṭṭhahanti, viditā abbhatthaṃ gacchanti. Ayaṃ, āvuso, samādhibhāvanā bhāvitā bahulīkatā satisampajaññāya saṃvattati.

‘‘Katamā cāvuso, samādhibhāvanā bhāvitā bahulīkatā āsavānaṃ khayāya saṃvattati? Idhāvuso, bhikkhu pañcasu upādānakkhandhesu udayabbayānupassī viharati. Iti rūpaṃ, iti rūpassa samudayo, iti rūpassa atthaṅgamo. Iti vedanā…pe… iti saññā… iti saṅkhārā… iti viññāṇaṃ, iti viññāṇassa samudayo, iti viññāṇassa atthaṅgamo. Ayaṃ, āvuso, samādhibhāvanā bhāvitā bahulīkatā āsavānaṃ khayāya saṃvattati.

\paragraph{308.} ‘‘Catasso appamaññā. Idhāvuso, bhikkhu mettāsahagatena cetasā ekaṃ disaṃ pharitvā viharati. Tathā dutiyaṃ. Tathā tatiyaṃ. Tathā catutthaṃ. Iti uddhamadho tiriyaṃ sabbadhi sabbattatāya sabbāvantaṃ lokaṃ mettāsahagatena cetasā vipulena mahaggatena appamāṇena averena abyāpajjena\footnote{abyāpajjhena (sī. syā. kaṃ. pī.)} pharitvā viharati. Karuṇāsahagatena cetasā…pe… muditāsahagatena cetasā…pe… upekkhāsahagatena cetasā ekaṃ disaṃ pharitvā viharati. Tathā dutiyaṃ. Tathā tatiyaṃ. Tathā catutthaṃ. Iti uddhamadho tiriyaṃ sabbadhi sabbattatāya sabbāvantaṃ lokaṃ upekkhāsahagatena cetasā vipulena mahaggatena appamāṇena averena abyāpajjena pharitvā viharati.

‘‘Cattāro āruppā.\footnote{arūpā (syā. kaṃ. pī.)} Idhāvuso, bhikkhu sabbaso rūpasaññānaṃ samatikkamā paṭighasaññānaṃ atthaṅgamā nānattasaññānaṃ amanasikārā ‘ananto ākāso’ti ākāsānañcāyatanaṃ upasampajja viharati. Sabbaso ākāsānañcāyatanaṃ samatikkamma ‘anantaṃ viññāṇa’nti viññāṇañcāyatanaṃ upasampajja viharati. Sabbaso viññāṇañcāyatanaṃ samatikkamma ‘natthi kiñcī’ti ākiñcaññāyatanaṃ upasampajja viharati. Sabbaso ākiñcaññāyatanaṃ samatikkamma nevasaññānāsaññāyatanaṃ upasampajja viharati.

‘‘Cattāri apassenāni. Idhāvuso, bhikkhu saṅkhāyekaṃ paṭisevati, saṅkhāyekaṃ adhivāseti, saṅkhāyekaṃ parivajjeti, saṅkhāyekaṃ vinodeti.

\paragraph{309.} ‘‘Cattāro ariyavaṃsā. Idhāvuso, bhikkhu santuṭṭho hoti itarītarena cīvarena, itarītaracīvarasantuṭṭhiyā ca vaṇṇavādī, na ca cīvarahetu anesanaṃ appatirūpaṃ āpajjati; aladdhā ca cīvaraṃ na paritassati, laddhā ca cīvaraṃ agadhito\footnote{agathito (sī. pī.)} amucchito anajjhāpanno ādīnavadassāvī nissaraṇapañño paribhuñjati; tāya ca pana itarītaracīvarasantuṭṭhiyā nevattānukkaṃseti na paraṃ vambheti. Yo hi tattha dakkho analaso sampajāno paṭissato, ayaṃ vuccatāvuso – ‘bhikkhu porāṇe aggaññe ariyavaṃse ṭhito’.

‘‘Puna caparaṃ, āvuso, bhikkhu santuṭṭho hoti itarītarena piṇḍapātena, itarītarapiṇḍapātasantuṭṭhiyā ca vaṇṇavādī, na ca piṇḍapātahetu anesanaṃ appatirūpaṃ āpajjati; aladdhā ca piṇḍapātaṃ na paritassati, laddhā ca piṇḍapātaṃ agadhito amucchito anajjhāpanno ādīnavadassāvī nissaraṇapañño paribhuñjati; tāya ca pana itarītarapiṇḍapātasantuṭṭhiyā nevattānukkaṃseti na paraṃ vambheti. Yo hi tattha dakkho analaso sampajāno paṭissato , ayaṃ vuccatāvuso – ‘bhikkhu porāṇe aggaññe ariyavaṃse ṭhito’.

‘‘Puna caparaṃ, āvuso, bhikkhu santuṭṭho hoti itarītarena senāsanena, itarītarasenāsanasantuṭṭhiyā ca vaṇṇavādī, na ca senāsanahetu anesanaṃ appatirūpaṃ āpajjati; aladdhā ca senāsanaṃ na paritassati, laddhā ca senāsanaṃ agadhito amucchito anajjhāpanno ādīnavadassāvī nissaraṇapañño paribhuñjati; tāya ca pana itarītarasenāsanasantuṭṭhiyā nevattānukkaṃseti na paraṃ vambheti. Yo hi tattha dakkho analaso sampajāno paṭissato, ayaṃ vuccatāvuso – ‘bhikkhu porāṇe aggaññe ariyavaṃse ṭhito’.

‘‘Puna caparaṃ, āvuso, bhikkhu pahānārāmo hoti pahānarato, bhāvanārāmo hoti bhāvanārato; tāya ca pana pahānārāmatāya pahānaratiyā bhāvanārāmatāya bhāvanāratiyā nevattānukkaṃseti na paraṃ vambheti. Yo hi tattha dakkho analaso sampajāno paṭissato ayaṃ vuccatāvuso – ‘bhikkhu porāṇe aggaññe ariyavaṃse ṭhito’.

\paragraph{310.} ‘‘Cattāri padhānāni. Saṃvarapadhānaṃ pahānapadhānaṃ bhāvanāpadhānaṃ\footnote{bhāvanāppadhānaṃ (syā.)} anurakkhaṇāpadhānaṃ\footnote{anurakkhanāppadhānaṃ (syā.)}. Katamañcāvuso, saṃvarapadhānaṃ? Idhāvuso, bhikkhu cakkhunā rūpaṃ disvā na nimittaggāhī hoti nānubyañjanaggāhī. Yatvādhikaraṇamenaṃ cakkhundriyaṃ asaṃvutaṃ viharantaṃ abhijjhādomanassā pāpakā akusalā dhammā anvāssaveyyuṃ, tassa saṃvarāya paṭipajjati, rakkhati cakkhundriyaṃ, cakkhundriye saṃvaraṃ āpajjati. Sotena saddaṃ sutvā… ghānena gandhaṃ ghāyitvā… jivhāya rasaṃ sāyitvā… kāyena phoṭṭhabbaṃ phusitvā… manasā dhammaṃ viññāya na nimittaggāhī hoti nānubyañjanaggāhī. Yatvādhikaraṇamenaṃ manindriyaṃ asaṃvutaṃ viharantaṃ abhijjhādomanassā pāpakā akusalā dhammā anvāssaveyyuṃ, tassa saṃvarāya paṭipajjati, rakkhati manindriyaṃ, manindriye saṃvaraṃ āpajjati. Idaṃ vuccatāvuso, saṃvarapadhānaṃ.

‘‘Katamañcāvuso, pahānapadhānaṃ? Idhāvuso, bhikkhu uppannaṃ kāmavitakkaṃ nādhivāseti pajahati vinodeti byantiṃ karoti\footnote{byantī karoti (syā. kaṃ.)} anabhāvaṃ gameti. Uppannaṃ byāpādavitakkaṃ…pe… uppannaṃ vihiṃsāvitakkaṃ… uppannuppanne pāpake akusale dhamme nādhivāseti pajahati vinodeti byantiṃ karoti anabhāvaṃ gameti. Idaṃ vuccatāvuso, pahānapadhānaṃ.

‘‘Katamañcāvuso , bhāvanāpadhānaṃ? Idhāvuso, bhikkhu satisambojjhaṅgaṃ bhāveti vivekanissitaṃ virāganissitaṃ nirodhanissitaṃ vossaggapariṇāmiṃ. Dhammavicayasambojjhaṅgaṃ bhāveti… vīriyasambojjhaṅgaṃ bhāveti… pītisambojjhaṅgaṃ bhāveti… passaddhisambojjhaṅgaṃ bhāveti… samādhisambojjhaṅgaṃ bhāveti… upekkhāsambojjhaṅgaṃ bhāveti vivekanissitaṃ virāganissitaṃ nirodhanissitaṃ vossaggapariṇāmiṃ. Idaṃ vuccatāvuso, bhāvanāpadhānaṃ.

‘‘Katamañcāvuso, anurakkhaṇāpadhānaṃ? Idhāvuso, bhikkhu uppannaṃ bhadrakaṃ\footnote{bhaddakaṃ (syā. kaṃ. pī.)} samādhinimittaṃ anurakkhati – aṭṭhikasaññaṃ, puḷuvakasaññaṃ\footnote{puḷavakasaññaṃ (sī. pī.)}, vinīlakasaññaṃ, vicchiddakasaññaṃ, uddhumātakasaññaṃ. Idaṃ vuccatāvuso, anurakkhaṇāpadhānaṃ.

‘‘Cattāri ñāṇāni – dhamme ñāṇaṃ, anvaye ñāṇaṃ, pariye\footnote{paricce (sī. ka.), paricchede (syā. pī. ka.) ṭīkā oloketabbā} ñāṇaṃ, sammutiyā ñāṇaṃ\footnote{sammatiñāṇaṃ (syā. kaṃ.)}.

‘‘Aparānipi cattāri ñāṇāni – dukkhe ñāṇaṃ, dukkhasamudaye ñāṇaṃ, dukkhanirodhe ñāṇaṃ, dukkhanirodhagāminiyā paṭipadāya ñāṇaṃ.

\paragraph{311.} ‘‘Cattāri sotāpattiyaṅgāni – sappurisasaṃsevo, saddhammassavanaṃ, yonisomanasikāro, dhammānudhammappaṭipatti.

‘‘Cattāri sotāpannassa aṅgāni. Idhāvuso, ariyasāvako buddhe aveccappasādena samannāgato hoti – ‘itipi so bhagavā arahaṃ sammāsambuddho vijjācaraṇasampanno sugato lokavidū anuttaro purisadammasārathi satthā devamanussānaṃ buddho, bhagavā’ti. Dhamme aveccappasādena samannāgato hoti – ‘svākkhāto bhagavatā dhammo sandiṭṭhiko akāliko ehipassiko opaneyyiko\footnote{opanayiko (syā. kaṃ.)} paccattaṃ veditabbo viññūhī’ti. Saṅghe aveccappasādena samannāgato hoti – ‘suppaṭipanno bhagavato sāvakasaṅgho ujuppaṭipanno bhagavato sāvakasaṅgho ñāyappaṭipanno bhagavato sāvakasaṅgho sāmīcippaṭipanno bhagavato sāvakasaṅgho yadidaṃ cattāri purisayugāni aṭṭha purisapuggalā, esa bhagavato sāvakasaṅgho āhuneyyo pāhuneyyo dakkhiṇeyyo añjalikaraṇīyo anuttaraṃ puññakkhettaṃ lokassā’ti. Ariyakantehi sīlehi samannāgato hoti akhaṇḍehi acchiddehi asabalehi akammāsehi bhujissehi viññuppasatthehi aparāmaṭṭhehi samādhisaṃvattanikehi.

‘‘Cattāri sāmaññaphalāni – sotāpattiphalaṃ, sakadāgāmiphalaṃ, anāgāmiphalaṃ, arahattaphalaṃ.

‘‘Catasso dhātuyo – pathavīdhātu, āpodhātu, tejodhātu, vāyodhātu.

‘‘Cattāro āhārā – kabaḷīkāro āhāro oḷāriko vā sukhumo vā, phasso dutiyo, manosañcetanā tatiyā, viññāṇaṃ catutthaṃ.

‘‘Catasso viññāṇaṭṭhitiyo. Rūpūpāyaṃ vā, āvuso, viññāṇaṃ tiṭṭhamānaṃ tiṭṭhati rūpārammaṇaṃ\footnote{rūpāramaṇaṃ (?)} rūpappatiṭṭhaṃ nandūpasecanaṃ vuddhiṃ virūḷhiṃ vepullaṃ āpajjati; vedanūpāyaṃ vā āvuso…pe… saññūpāyaṃ vā, āvuso…pe… saṅkhārūpāyaṃ vā, āvuso, viññāṇaṃ tiṭṭhamānaṃ tiṭṭhati saṅkhārārammaṇaṃ saṅkhārappatiṭṭhaṃ nandūpasecanaṃ vuddhiṃ virūḷhiṃ vepullaṃ āpajjati.

‘‘Cattāri agatigamanāni – chandāgatiṃ gacchati, dosāgati gacchati, mohāgatiṃ gacchati, bhayāgatiṃ gacchati.

‘‘Cattāro taṇhuppādā – cīvarahetu vā, āvuso, bhikkhuno taṇhā uppajjamānā uppajjati; piṇḍapātahetu vā, āvuso, bhikkhuno taṇhā uppajjamānā uppajjati; senāsanahetu vā, āvuso, bhikkhuno taṇhā uppajjamānā uppajjati; itibhavābhavahetu vā, āvuso, bhikkhuno taṇhā uppajjamānā uppajjati.

‘‘Catasso paṭipadā – dukkhā paṭipadā dandhābhiññā, dukkhā paṭipadā khippābhiññā, sukhā paṭipadā dandhābhiññā, sukhā paṭipadā khippābhiññā.

‘‘Aparāpi catasso paṭipadā – akkhamā paṭipadā, khamā paṭipadā, damā paṭipadā, samā paṭipadā.

‘‘Cattāri dhammapadāni – anabhijjhā dhammapadaṃ, abyāpādo dhammapadaṃ, sammāsati dhammapadaṃ, sammāsamādhi dhammapadaṃ.

‘‘Cattāri dhammasamādānāni – atthāvuso, dhammasamādānaṃ paccuppannadukkhañceva āyatiñca dukkhavipākaṃ. Atthāvuso, dhammasamādānaṃ paccuppannadukkhaṃ āyatiṃ sukhavipākaṃ. Atthāvuso, dhammasamādānaṃ paccuppannasukhaṃ āyatiṃ dukkhavipākaṃ. Atthāvuso, dhammasamādānaṃ paccuppannasukhañceva āyatiñca sukhavipākaṃ.

‘‘Cattāro dhammakkhandhā – sīlakkhandho, samādhikkhandho, paññākkhandho, vimuttikkhandho.

‘‘Cattāri balāni – vīriyabalaṃ, satibalaṃ, samādhibalaṃ, paññābalaṃ.

‘‘Cattāri adhiṭṭhānāni – paññādhiṭṭhānaṃ, saccādhiṭṭhānaṃ, cāgādhiṭṭhānaṃ, upasamādhiṭṭhānaṃ.

\paragraph{312.} ‘‘Cattāri pañhabyākaraṇāni\footnote{cattāro pañhābyākaraṇā (sī. syā. kaṃ. pī.)} - ekaṃsabyākaraṇīyo pañho, paṭipucchābyākaraṇīyo pañho, vibhajjabyākaraṇīyo pañho, ṭhapanīyo pañho.

‘‘Cattāri kammāni – atthāvuso, kammaṃ kaṇhaṃ kaṇhavipākaṃ; atthāvuso, kammaṃ sukkaṃ sukkavipākaṃ; atthāvuso, kammaṃ kaṇhasukkaṃ kaṇhasukkavipākaṃ; atthāvuso, kammaṃ akaṇhaasukkaṃ akaṇhaasukkavipākaṃ kammakkhayāya saṃvattati.

‘‘Cattāro sacchikaraṇīyā dhammā – pubbenivāso satiyā sacchikaraṇīyo; sattānaṃ cutūpapāto cakkhunā sacchikaraṇīyo; aṭṭha vimokkhā kāyena sacchikaraṇīyā; āsavānaṃ khayo paññāya sacchikaraṇīyo.

‘‘Cattāro oghā – kāmogho, bhavogho, diṭṭhogho, avijjogho.

‘‘Cattāro yogā – kāmayogo, bhavayogo, diṭṭhiyogo, avijjāyogo.

‘‘Cattāro visaññogā – kāmayogavisaññogo, bhavayogavisaññogo, diṭṭhiyogavisaññogo, avijjāyogavisaññogo.

‘‘Cattāro ganthā – abhijjhā kāyagantho, byāpādo kāyagantho, sīlabbataparāmāso kāyagantho, idaṃsaccābhiniveso kāyagantho.

‘‘Cattāri upādānāni – kāmupādānaṃ\footnote{kāmūpādānaṃ (sī. pī.) evamitaresupi}, diṭṭhupādānaṃ, sīlabbatupādānaṃ, attavādupādānaṃ.

‘‘Catasso yoniyo – aṇḍajayoni, jalābujayoni, saṃsedajayoni, opapātikayoni.

‘‘Catasso gabbhāvakkantiyo. Idhāvuso, ekacco asampajāno mātukucchiṃ okkamati, asampajāno mātukucchismiṃ ṭhāti, asampajāno mātukucchimhā nikkhamati, ayaṃ paṭhamā gabbhāvakkanti. Puna caparaṃ, āvuso, idhekacco sampajāno mātukucchiṃ okkamati, asampajāno mātukucchismiṃ ṭhāti, asampajāno mātukucchimhā nikkhamati, ayaṃ dutiyā gabbhāvakkanti. Puna caparaṃ, āvuso, idhekacco sampajāno mātukucchiṃ okkamati, sampajāno mātukucchismiṃ ṭhāti, asampajāno mātukucchimhā nikkhamati, ayaṃ tatiyā gabbhāvakkanti. Puna caparaṃ, āvuso, idhekacco sampajāno mātukucchiṃ okkamati, sampajāno mātukucchismiṃ ṭhāti, sampajāno mātukucchimhā nikkhamati, ayaṃ catutthā gabbhāvakkanti.

‘‘Cattāro attabhāvapaṭilābhā. Atthāvuso, attabhāvapaṭilābho, yasmiṃ attabhāvapaṭilābhe attasañcetanāyeva kamati, no parasañcetanā. Atthāvuso, attabhāvapaṭilābho, yasmiṃ attabhāvapaṭilābhe parasañcetanāyeva kamati, no attasañcetanā. Atthāvuso, attabhāvapaṭilābho, yasmiṃ attabhāvapaṭilābhe attasañcetanā ceva kamati parasañcetanā ca. Atthāvuso, attabhāvapaṭilābho, yasmiṃ attabhāvapaṭilābhe neva attasañcetanā kamati, no parasañcetanā.

\paragraph{313.} ‘‘Catasso dakkhiṇāvisuddhiyo. Atthāvuso, dakkhiṇā dāyakato visujjhati no paṭiggāhakato. Atthāvuso, dakkhiṇā paṭiggāhakato visujjhati no dāyakato. Atthāvuso, dakkhiṇā neva dāyakato visujjhati no paṭiggāhakato. Atthāvuso, dakkhiṇā dāyakato ceva visujjhati paṭiggāhakato ca.

‘‘Cattāri saṅgahavatthūni – dānaṃ, peyyavajjaṃ\footnote{piyavajjaṃ (syā. kaṃ. ka.)}, atthacariyā, samānattatā.

‘‘Cattāro anariyavohārā – musāvādo, pisuṇāvācā, pharusāvācā, samphappalāpo.

‘‘Cattāro ariyavohārā – musāvādā veramaṇī\footnote{veramaṇi (ka.)}, pisuṇāya vācāya veramaṇī, pharusāya vācāya veramaṇī, samphappalāpā veramaṇī.

‘‘Aparepi cattāro anariyavohārā – adiṭṭhe diṭṭhavāditā, assute sutavāditā, amute mutavāditā, aviññāte viññātavāditā.

‘‘Aparepi cattāro ariyavohārā – adiṭṭhe adiṭṭhavāditā, assute assutavāditā, amute amutavāditā, aviññāte aviññātavāditā.

‘‘Aparepi cattāro anariyavohārā – diṭṭhe adiṭṭhavāditā, sute assutavāditā, mute amutavāditā, viññāte aviññātavāditā.

‘‘Aparepi cattāro ariyavohārā – diṭṭhe diṭṭhavāditā, sute sutavāditā, mute mutavāditā, viññāte viññātavāditā.

\paragraph{314.} ‘‘Cattāro puggalā. Idhāvuso, ekacco puggalo attantapo hoti attaparitāpanānuyogamanuyutto. Idhāvuso, ekacco puggalo parantapo hoti paraparitāpanānuyogamanuyutto. Idhāvuso, ekacco puggalo attantapo ca hoti attaparitāpanānuyogamanuyutto, parantapo ca paraparitāpanānuyogamanuyutto. Idhāvuso, ekacco puggalo neva attantapo hoti na attaparitāpanānuyogamanuyutto na parantapo na paraparitāpanānuyogamanuyutto. So anattantapo aparantapo diṭṭheva dhamme nicchāto nibbuto sītībhūto\footnote{sītibhūto (ka.)} sukhappaṭisaṃvedī brahmabhūtena attanā viharati.

‘‘Aparepi cattāro puggalā. Idhāvuso, ekacco puggalo attahitāya paṭipanno hoti no parahitāya. Idhāvuso, ekacco puggalo parahitāya paṭipanno hoti no attahitāya. Idhāvuso , ekacco puggalo neva attahitāya paṭipanno hoti no parahitāya. Idhāvuso, ekacco puggalo attahitāya ceva paṭipanno hoti parahitāya ca.

‘‘Aparepi cattāro puggalā – tamo tamaparāyano, tamo jotiparāyano, joti tamaparāyano, joti jotiparāyano.

‘‘Aparepi cattāro puggalā – samaṇamacalo, samaṇapadumo, samaṇapuṇḍarīko, samaṇesu samaṇasukhumālo.

‘‘Ime kho, āvuso, tena bhagavatā jānatā passatā arahatā sammāsambuddhena cattāro dhammā sammadakkhātā; tattha sabbeheva saṅgāyitabbaṃ…pe… atthāya hitāya sukhāya devamanussānaṃ.

\xsubsubsectionEnd{Paṭhamabhāṇavāro niṭṭhito.}

\subsubsection{Pañcakaṃ}

\paragraph{315.} ‘‘Atthi kho, āvuso, tena bhagavatā jānatā passatā arahatā sammāsambuddhena pañca dhammā sammadakkhātā. Tattha sabbeheva saṅgāyitabbaṃ…pe… atthāya hitāya sukhāya devamanussānaṃ. Katame pañca?

‘‘Pañcakkhandhā. Rūpakkhandho vedanākkhandho saññākkhandho saṅkhārakkhandho viññāṇakkhandho.

‘‘Pañcupādānakkhandhā. Rūpupādānakkhandho\footnote{rūpūpādānakkhandho (sī. syā. kaṃ. pī.) evamitaresupi} vedanupādānakkhandho saññupādānakkhandho saṅkhārupādānakkhandho viññāṇupādānakkhandho.

‘‘Pañca kāmaguṇā. Cakkhuviññeyyā rūpā iṭṭhā kantā manāpā piyarūpā kāmūpasañhitā rajanīyā , sotaviññeyyā saddā… ghānaviññeyyā gandhā… jivhāviññeyyā rasā… kāyaviññeyyā phoṭṭhabbā iṭṭhā kantā manāpā piyarūpā kāmūpasañhitā rajanīyā.

‘‘Pañca gatiyo – nirayo, tiracchānayoni, pettivisayo, manussā, devā.

‘‘Pañca macchariyāni – āvāsamacchariyaṃ, kulamacchariyaṃ, lābhamacchariyaṃ, vaṇṇamacchariyaṃ, dhammamacchariyaṃ.

‘‘Pañca nīvaraṇāni – kāmacchandanīvaraṇaṃ, byāpādanīvaraṇaṃ, thinamiddhanīvaraṇaṃ, uddhaccakukkuccanīvaraṇaṃ, vicikicchānīvaraṇaṃ.

‘‘Pañca orambhāgiyāni saññojanāni – sakkāyadiṭṭhi, vicikicchā, sīlabbataparāmāso, kāmacchando, byāpādo.

‘‘Pañca uddhambhāgiyāni saññojanāni – rūparāgo, arūparāgo, māno, uddhaccaṃ, avijjā.

‘‘Pañca sikkhāpadāni – pāṇātipātā veramaṇī, adinnādānā veramaṇī, kāmesumicchācārā veramaṇī, musāvādā veramaṇī, surāmerayamajjappamādaṭṭhānā veramaṇī.

\paragraph{316.} ‘‘Pañca abhabbaṭṭhānāni. Abhabbo, āvuso, khīṇāsavo bhikkhu sañcicca pāṇaṃ jīvitā voropetuṃ. Abhabbo khīṇāsavo bhikkhu adinnaṃ theyyasaṅkhātaṃ ādiyituṃ\footnote{ādātuṃ (syā. kaṃ. pī.)}. Abhabbo khīṇāsavo bhikkhu methunaṃ dhammaṃ paṭisevituṃ. Abhabbo khīṇāsavo bhikkhu sampajānamusā bhāsituṃ. Abhabbo khīṇāsavo bhikkhu sannidhikārakaṃ kāme paribhuñjituṃ, seyyathāpi pubbe āgārikabhūto.

‘‘Pañca byasanāni – ñātibyasanaṃ, bhogabyasanaṃ, rogabyasanaṃ, sīlabyasanaṃ, diṭṭhibyasanaṃ. Nāvuso, sattā ñātibyasanahetu vā bhogabyasanahetu vā rogabyasanahetu vā kāyassa bhedā paraṃ maraṇā apāyaṃ duggatiṃ vinipātaṃ nirayaṃ upapajjanti. Sīlabyasanahetu vā, āvuso, sattā diṭṭhibyasanahetu vā kāyassa bhedā paraṃ maraṇā apāyaṃ duggatiṃ vinipātaṃ nirayaṃ upapajjanti.

‘‘Pañca sampadā – ñātisampadā, bhogasampadā, ārogyasampadā, sīlasampadā, diṭṭhisampadā. Nāvuso, sattā ñātisampadāhetu vā bhogasampadāhetu vā ārogyasampadāhetu vā kāyassa bhedā paraṃ maraṇā sugatiṃ saggaṃ lokaṃ upapajjanti. Sīlasampadāhetu vā, āvuso, sattā diṭṭhisampadāhetu vā kāyassa bhedā paraṃ maraṇā sugatiṃ saggaṃ lokaṃ upapajjanti.

‘‘Pañca ādīnavā dussīlassa sīlavipattiyā. Idhāvuso , dussīlo sīlavipanno pamādādhikaraṇaṃ mahatiṃ bhogajāniṃ nigacchati, ayaṃ paṭhamo ādīnavo dussīlassa sīlavipattiyā. Puna caparaṃ, āvuso, dussīlassa sīlavipannassa pāpako kittisaddo abbhuggacchati, ayaṃ dutiyo ādīnavo dussīlassa sīlavipattiyā. Puna caparaṃ, āvuso, dussīlo sīlavipanno yaññadeva parisaṃ upasaṅkamati yadi khattiyaparisaṃ yadi brāhmaṇaparisaṃ yadi gahapatiparisaṃ yadi samaṇaparisaṃ, avisārado upasaṅkamati maṅkubhūto, ayaṃ tatiyo ādīnavo dussīlassa sīlavipattiyā. Puna caparaṃ, āvuso, dussīlo sīlavipanno sammūḷho kālaṃ karoti, ayaṃ catuttho ādīnavo dussīlassa sīlavipattiyā. Puna caparaṃ, āvuso, dussīlo sīlavipanno kāyassa bhedā paraṃ maraṇā apāyaṃ duggatiṃ vinipātaṃ nirayaṃ upapajjati, ayaṃ pañcamo ādīnavo dussīlassa sīlavipattiyā.

‘‘Pañca ānisaṃsā sīlavato sīlasampadāya. Idhāvuso, sīlavā sīlasampanno appamādādhikaraṇaṃ mahantaṃ bhogakkhandhaṃ adhigacchati, ayaṃ paṭhamo ānisaṃso sīlavato sīlasampadāya. Puna caparaṃ, āvuso, sīlavato sīlasampannassa kalyāṇo kittisaddo abbhuggacchati, ayaṃ dutiyo ānisaṃso sīlavato sīlasampadāya. Puna caparaṃ, āvuso, sīlavā sīlasampanno yaññadeva parisaṃ upasaṅkamati yadi khattiyaparisaṃ yadi brāhmaṇaparisaṃ yadi gahapatiparisaṃ yadi samaṇaparisaṃ, visārado upasaṅkamati amaṅkubhūto, ayaṃ tatiyo ānisaṃso sīlavato sīlasampadāya. Puna caparaṃ, āvuso, sīlavā sīlasampanno asammūḷho kālaṃ karoti, ayaṃ catuttho ānisaṃso sīlavato sīlasampadāya. Puna caparaṃ, āvuso, sīlavā sīlasampanno kāyassa bhedā paraṃ maraṇā sugatiṃ saggaṃ lokaṃ upapajjati, ayaṃ pañcamo ānisaṃso sīlavato sīlasampadāya.

‘‘Codakena , āvuso, bhikkhunā paraṃ codetukāmena pañca dhamme ajjhattaṃ upaṭṭhapetvā paro codetabbo. Kālena vakkhāmi no akālena, bhūtena vakkhāmi no abhūtena, saṇhena vakkhāmi no pharusena, atthasaṃhitena vakkhāmi no anatthasaṃhitena, mettacittena\footnote{mettācittena (katthaci)} vakkhāmi no dosantarenāti. Codakena, āvuso, bhikkhunā paraṃ codetukāmena ime pañca dhamme ajjhattaṃ upaṭṭhapetvā paro codetabbo.

\paragraph{317.} ‘‘Pañca padhāniyaṅgāni. Idhāvuso, bhikkhu saddho hoti, saddahati tathāgatassa bodhiṃ – ‘itipi so bhagavā arahaṃ sammāsambuddho vijjācaraṇasampanno sugato, lokavidū anuttaro purisadammasārathi satthā devamanussānaṃ buddho bhagavā’ti. Appābādho hoti appātaṅko, samavepākiniyā gahaṇiyā samannāgato nātisītāya nāccuṇhāya majjhimāya padhānakkhamāya. Asaṭho hoti amāyāvī, yathābhūtaṃ attānaṃ āvikattā satthari vā viññūsu vā sabrahmacārīsu. Āraddhavīriyo viharati akusalānaṃ dhammānaṃ pahānāya kusalānaṃ dhammānaṃ upasampadāya thāmavā daḷhaparakkamo anikkhittadhuro kusalesu dhammesu. Paññavā hoti udayatthagāminiyā paññāya samannāgato ariyāya nibbedhikāya sammādukkhakkhayagāminiyā.

\paragraph{318.} ‘‘Pañca suddhāvāsā – avihā, atappā, sudassā, sudassī, akaniṭṭhā.

‘‘Pañca anāgāmino – antarāparinibbāyī, upahaccaparinibbāyī, asaṅkhāraparinibbāyī, sasaṅkhāraparinibbāyī, uddhaṃsotoakaniṭṭhagāmī.

\paragraph{319.} ‘‘Pañca cetokhilā. Idhāvuso, bhikkhu satthari kaṅkhati vicikicchati nādhimuccati na sampasīdati. Yo so, āvuso, bhikkhu satthari kaṅkhati vicikicchati nādhimuccati na sampasīdati, tassa cittaṃ na namati ātappāya anuyogāya sātaccāya padhānāya, yassa cittaṃ na namati ātappāya anuyogāya sātaccāya padhānāya, ayaṃ paṭhamo cetokhilo. Puna caparaṃ, āvuso, bhikkhu dhamme kaṅkhati vicikicchati…pe… saṅghe kaṅkhati vicikicchati… sikkhāya kaṅkhati vicikicchati… sabrahmacārīsu kupito hoti anattamano āhatacitto khilajāto. Yo so, āvuso, bhikkhu sabrahmacārīsu kupito hoti anattamano āhatacitto khilajāto, tassa cittaṃ na namati ātappāya anuyogāya sātaccāya padhānāya, yassa cittaṃ na namati ātappāya anuyogāya sātaccāya padhānāya, ayaṃ pañcamo cetokhilo.

\paragraph{320.} ‘‘Pañca cetasovinibandhā. Idhāvuso, bhikkhu kāmesu avītarāgo hoti avigatacchando avigatapemo avigatapipāso avigatapariḷāho avigatataṇho. Yo so, āvuso, bhikkhu kāmesu avītarāgo hoti avigatacchando avigatapemo avigatapipāso avigatapariḷāho avigatataṇho, tassa cittaṃ na namati ātappāya anuyogāya sātaccāya padhānāya. Yassa cittaṃ na namati ātappāya anuyogāya sātaccāya padhānāya. Ayaṃ paṭhamo cetaso vinibandho. Puna caparaṃ, āvuso, bhikkhu kāye avītarāgo hoti…pe… rūpe avītarāgo hoti…pe… puna caparaṃ, āvuso, bhikkhu yāvadatthaṃ udarāvadehakaṃ bhuñjitvā seyyasukhaṃ passasukhaṃ middhasukhaṃ anuyutto viharati…pe… puna caparaṃ, āvuso, bhikkhu aññataraṃ devanikāyaṃ paṇidhāya brahmacariyaṃ carati – ‘imināhaṃ sīlena vā vatena vā tapena vā brahmacariyena vā devo vā bhavissāmi devaññataro vā’ti. Yo so, āvuso, bhikkhu aññataraṃ devanikāyaṃ paṇidhāya brahmacariyaṃ carati – ‘imināhaṃ sīlena vā vatena vā tapena vā brahmacariyena vā devo vā bhavissāmi devaññataro vā’ti, tassa cittaṃ na namati ātappāya anuyogāya sātaccāya padhānāya. Yassa cittaṃ na namati ātappāya anuyogāya sātaccāya padhānāya. Ayaṃ pañcamo cetaso vinibandho.

‘‘Pañcindriyāni – cakkhundriyaṃ, sotindriyaṃ, ghānindriyaṃ, jivhindriyaṃ, kāyindriyaṃ.

‘‘Aparānipi pañcindriyāni – sukhindriyaṃ, dukkhindriyaṃ, somanassindriyaṃ, domanassindriyaṃ, upekkhindriyaṃ.

‘‘Aparānipi pañcindriyāni – saddhindriyaṃ, vīriyindriyaṃ, satindriyaṃ, samādhindriyaṃ, paññindriyaṃ.

\paragraph{321.} ‘‘Pañca nissaraṇiyā\footnote{nissāraṇīyā (sī. syā. kaṃ. pī.) ṭīkā oloketabbā}dhātuyo. Idhāvuso, bhikkhuno kāme manasikaroto kāmesu cittaṃ na pakkhandati na pasīdati na santiṭṭhati na vimuccati. Nekkhammaṃ kho panassa manasikaroto nekkhamme cittaṃ pakkhandati pasīdati santiṭṭhati vimuccati. Tassa taṃ cittaṃ sugataṃ subhāvitaṃ suvuṭṭhitaṃ suvimuttaṃ visaṃyuttaṃ kāmehi. Ye ca kāmapaccayā uppajjanti āsavā vighātā pariḷāhā\footnote{vighātapariḷāhā (syā. kaṃ.)}, mutto so tehi, na so taṃ vedanaṃ vedeti. Idamakkhātaṃ kāmānaṃ nissaraṇaṃ.

‘‘Puna caparaṃ, āvuso, bhikkhuno byāpādaṃ manasikaroto byāpāde cittaṃ na pakkhandati na pasīdati na santiṭṭhati na vimuccati. Abyāpādaṃ kho panassa manasikaroto abyāpāde cittaṃ pakkhandati pasīdati santiṭṭhati vimuccati. Tassa taṃ cittaṃ sugataṃ subhāvitaṃ suvuṭṭhitaṃ suvimuttaṃ visaṃyuttaṃ byāpādena. Ye ca byāpādapaccayā uppajjanti āsavā vighātā pariḷāhā, mutto so tehi, na so taṃ vedanaṃ vedeti. Idamakkhātaṃ byāpādassa nissaraṇaṃ.

‘‘Puna caparaṃ, āvuso, bhikkhuno vihesaṃ manasikaroto vihesāya cittaṃ na pakkhandati na pasīdati na santiṭṭhati na vimuccati. Avihesaṃ kho panassa manasikaroto avihesāya cittaṃ pakkhandati pasīdati santiṭṭhati vimuccati. Tassa taṃ cittaṃ sugataṃ subhāvitaṃ suvuṭṭhitaṃ suvimuttaṃ visaṃyuttaṃ vihesāya. Ye ca vihesāpaccayā uppajjanti āsavā vighātā pariḷāhā, mutto so tehi, na so taṃ vedanaṃ vedeti. Idamakkhātaṃ vihesāya nissaraṇaṃ.

‘‘Puna caparaṃ, āvuso, bhikkhuno rūpe manasikaroto rūpesu cittaṃ na pakkhandati na pasīdati na santiṭṭhati na vimuccati. Arūpaṃ kho panassa manasikaroto arūpe cittaṃ pakkhandati pasīdati santiṭṭhati vimuccati. Tassa taṃ cittaṃ sugataṃ subhāvitaṃ suvuṭṭhitaṃ suvimuttaṃ visaṃyuttaṃ rūpehi. Ye ca rūpapaccayā uppajjanti āsavā vighātā pariḷāhā, mutto so tehi, na so taṃ vedanaṃ vedeti. Idamakkhātaṃ rūpānaṃ nissaraṇaṃ.

‘‘Puna caparaṃ, āvuso, bhikkhuno sakkāyaṃ manasikaroto sakkāye cittaṃ na pakkhandati na pasīdati na santiṭṭhati na vimuccati. Sakkāyanirodhaṃ kho panassa manasikaroto sakkāyanirodhe cittaṃ pakkhandati pasīdati santiṭṭhati vimuccati. Tassa taṃ cittaṃ sugataṃ subhāvitaṃ suvuṭṭhitaṃ suvimuttaṃ visaṃyuttaṃ sakkāyena. Ye ca sakkāyapaccayā uppajjanti āsavā vighātā pariḷāhā, mutto so tehi, na so taṃ vedanaṃ vedeti. Idamakkhātaṃ sakkāyassa nissaraṇaṃ.

\paragraph{322.} ‘‘Pañca vimuttāyatanāni. Idhāvuso, bhikkhuno satthā dhammaṃ deseti aññataro vā garuṭṭhāniyo sabrahmacārī. Yathā yathā, āvuso, bhikkhuno satthā dhammaṃ deseti aññataro vā garuṭṭhāniyo sabrahmacārī . Tathā tathā so tasmiṃ dhamme atthapaṭisaṃvedī ca hoti dhammapaṭisaṃvedī ca. Tassa atthapaṭisaṃvedino dhammapaṭisaṃvedino pāmojjaṃ jāyati, pamuditassa pīti jāyati, pītimanassa kāyo passambhati, passaddhakāyo sukhaṃ vedeti, sukhino cittaṃ samādhiyati. Idaṃ paṭhamaṃ vimuttāyatanaṃ.

‘‘Puna caparaṃ, āvuso, bhikkhuno na heva kho satthā dhammaṃ deseti aññataro vā garuṭṭhāniyo sabrahmacārī, api ca kho yathāsutaṃ yathāpariyattaṃ dhammaṃ vitthārena paresaṃ deseti…pe… api ca kho yathāsutaṃ yathāpariyattaṃ dhammaṃ vitthārena sajjhāyaṃ karoti…pe… api ca kho yathāsutaṃ yathāpariyattaṃ dhammaṃ cetasā anuvitakketi anuvicāreti manasānupekkhati…pe… api ca khvassa aññataraṃ samādhinimittaṃ suggahitaṃ hoti sumanasikataṃ sūpadhāritaṃ suppaṭividdhaṃ paññāya. Yathā yathā, āvuso, bhikkhuno aññataraṃ samādhinimittaṃ suggahitaṃ hoti sumanasikataṃ sūpadhāritaṃ suppaṭividdhaṃ paññāya tathā tathā so tasmiṃ dhamme atthapaṭisaṃvedī ca hoti dhammapaṭisaṃvedī ca. Tassa atthapaṭisaṃvedino dhammapaṭisaṃvedino pāmojjaṃ jāyati, pamuditassa pīti jāyati, pītimanassa kāyo passambhati, passaddhakāyo sukhaṃ vedeti , sukhino cittaṃ samādhiyati. Idaṃ pañcamaṃ vimuttāyatanaṃ.

‘‘Pañca vimuttiparipācanīyā saññā – aniccasaññā, anicce dukkhasaññā, dukkhe anattasaññā, pahānasaññā, virāgasaññā.

‘‘Ime kho, āvuso, tena bhagavatā jānatā passatā arahatā sammāsambuddhena pañca dhammā sammadakkhātā; tattha sabbeheva saṅgāyitabbaṃ…pe… atthāya hitāya sukhāya devamanussānaṃ\footnote{saṅgitiyapañcakaṃ niṭṭhitaṃ (syā. kaṃ.)}.

\subsubsection{Chakkaṃ}

\paragraph{323.} ‘‘Atthi kho, āvuso, tena bhagavatā jānatā passatā arahatā sammāsambuddhena cha dhammā sammadakkhātā; tattha sabbeheva saṅgāyitabbaṃ…pe… atthāya hitāya sukhāya devamanussānaṃ. Katame cha?

‘‘Cha ajjhattikāniāyatanāni – cakkhāyatanaṃ, sotāyatanaṃ, ghānāyatanaṃ, jivhāyatanaṃ, kāyāyatanaṃ, manāyatanaṃ.

‘‘Cha bāhirāni āyatanāni – rūpāyatanaṃ, saddāyatanaṃ, gandhāyatanaṃ, rasāyatanaṃ, phoṭṭhabbāyatanaṃ, dhammāyatanaṃ.

‘‘Cha viññāṇakāyā – cakkhuviññāṇaṃ, sotaviññāṇaṃ, ghānaviññāṇaṃ, jivhāviññāṇaṃ, kāyaviññāṇaṃ, manoviññāṇaṃ.

‘‘Cha phassakāyā – cakkhusamphasso, sotasamphasso, ghānasamphasso, jivhāsamphasso, kāyasamphasso, manosamphasso.

‘‘Cha vedanākāyā – cakkhusamphassajā vedanā, sotasamphassajā vedanā, ghānasamphassajā vedanā, jivhāsamphassajā vedanā, kāyasamphassajā vedanā, manosamphassajā vedanā.

‘‘Cha saññākāyā – rūpasaññā, saddasaññā, gandhasaññā, rasasaññā, phoṭṭhabbasaññā, dhammasaññā.

‘‘Cha sañcetanākāyā – rūpasañcetanā, saddasañcetanā, gandhasañcetanā, rasasañcetanā, phoṭṭhabbasañcetanā, dhammasañcetanā.

‘‘Cha taṇhākāyā – rūpataṇhā, saddataṇhā, gandhataṇhā, rasataṇhā, phoṭṭhabbataṇhā, dhammataṇhā.

\paragraph{324.} ‘‘Cha agāravā. Idhāvuso, bhikkhu satthari agāravo viharati appatisso; dhamme agāravo viharati appatisso; saṅghe agāravo viharati appatisso; sikkhāya agāravo viharati appatisso; appamāde agāravo viharati appatisso; paṭisanthāre\footnote{paṭisandhāre (ka.)} agāravo viharati appatisso.

‘‘Cha gāravā. Idhāvuso, bhikkhu satthari sagāravo viharati sappatisso; dhamme sagāravo viharati sappatisso; saṅghe sagāravo viharati sappatisso; sikkhāya sagāravo viharati sappatisso; appamāde sagāravo viharati sappatisso; paṭisanthāre sagāravo viharati sappatisso.

‘‘Cha somanassūpavicārā. Cakkhunā rūpaṃ disvā somanassaṭṭhāniyaṃ rūpaṃ upavicarati; sotena saddaṃ sutvā… ghānena gandhaṃ ghāyitvā… jivhāya rasaṃ sāyitvā… kāyena phoṭṭhabbaṃ phusitvā. Manasā dhammaṃ viññāya somanassaṭṭhāniyaṃ dhammaṃ upavicarati.

‘‘Cha domanassūpavicārā. Cakkhunā rūpaṃ disvā domanassaṭṭhāniyaṃ rūpaṃ upavicarati…pe… manasā dhammaṃ viññāya domanassaṭṭhāniyaṃ dhammaṃ upavicarati.

‘‘Cha upekkhūpavicārā. Cakkhunā rūpaṃ disvā upekkhāṭṭhāniyaṃ\footnote{upekkhāṭhāniyaṃ (ka.)} rūpaṃ upavicarati…pe… manasā dhammaṃ viññāya upekkhāṭṭhāniyaṃ dhammaṃ upavicarati.

‘‘Cha sāraṇīyā dhammā. Idhāvuso, bhikkhuno mettaṃ kāyakammaṃ paccupaṭṭhitaṃ hoti sabrahmacārīsu āvi\footnote{āvī (ka. sī. pī. ka.)} ceva raho ca. Ayampi dhammo sāraṇīyo piyakaraṇo garukaraṇo saṅgahāya avivādāya sāmaggiyā ekībhāvāya saṃvattati.

‘‘Puna caparaṃ, āvuso, bhikkhuno mettaṃ vacīkammaṃ paccupaṭṭhitaṃ hoti sabrahmacārīsu āvi ceva raho ca. Ayampi dhammo sāraṇīyo…pe… ekībhāvāya saṃvattati.

‘‘Puna caparaṃ, āvuso, bhikkhuno mettaṃ manokammaṃ paccupaṭṭhitaṃ hoti sabrahmacārīsu āvi ceva raho ca. Ayampi dhammo sāraṇīyo…pe… ekībhāvāya saṃvattati.

‘‘Puna caparaṃ, āvuso, bhikkhu ye te lābhā dhammikā dhammaladdhā antamaso pattapariyāpannamattampi, tathārūpehi lābhehi appaṭivibhattabhogī hoti sīlavantehi sabrahmacārīhi sādhāraṇabhogī. Ayampi dhammo sāraṇīyo…pe… ekībhāvāya saṃvattati.

‘‘Puna caparaṃ, āvuso, bhikkhu yāni tāni sīlāni akhaṇḍāni acchiddāni asabalāni akammāsāni bhujissāni viññuppasatthāni aparāmaṭṭhāni samādhisaṃvattanikāni, tathārūpesu sīlesu sīlasāmaññagato viharati sabrahmacārīhi āvi ceva raho ca. Ayampi dhammo sāraṇīyo…pe… ekībhāvāya saṃvattati.

‘‘Puna caparaṃ, āvuso, bhikkhu yāyaṃ diṭṭhi ariyā niyyānikā niyyāti takkarassa sammā dukkhakkhayāya, tathārūpāya diṭṭhiyā diṭṭhisāmaññagato viharati sabrahmacārīhi āvi ceva raho ca. Ayampi dhammo sāraṇīyo piyakaraṇo garukaraṇo saṅgahāya avivādāya sāmaggiyā ekībhāvāya saṃvattati.

\paragraph{325.} Cha vivādamūlāni. Idhāvuso, bhikkhu kodhano hoti upanāhī. Yo so, āvuso, bhikkhu kodhano hoti upanāhī, so sattharipi agāravo viharati appatisso, dhammepi agāravo viharati appatisso, saṅghepi agāravo viharati appatisso, sikkhāyapi na paripūrakārī\footnote{paripūrīkārī (syā. kaṃ.)} hoti. Yo so, āvuso, bhikkhu satthari agāravo viharati appatisso, dhamme agāravo viharati appatisso, saṅghe agāravo viharati appatisso, sikkhāya na paripūrakārī, so saṅghe vivādaṃ janeti. Yo hoti vivādo bahujanaahitāya bahujanaasukhāya anatthāya ahitāya dukkhāya devamanussānaṃ. Evarūpaṃ ce tumhe, āvuso, vivādamūlaṃ ajjhattaṃ vā bahiddhā vā samanupasseyyātha. Tatra tumhe, āvuso, tasseva pāpakassa vivādamūlassa pahānāya vāyameyyātha. Evarūpaṃ ce tumhe, āvuso, vivādamūlaṃ ajjhattaṃ vā bahiddhā vā na samanupasseyyātha. Tatra tumhe, āvuso, tasseva pāpakassa vivādamūlassa āyatiṃ anavassavāya paṭipajjeyyātha. Evametassa pāpakassa vivādamūlassa pahānaṃ hoti. Evametassa pāpakassa vivādamūlassa āyatiṃ anavassavo hoti.

‘‘Puna caparaṃ, āvuso, bhikkhu makkhī hoti paḷāsī…pe… issukī hoti maccharī…pe… saṭho hoti māyāvī… pāpiccho hoti micchādiṭṭhī… sandiṭṭhiparāmāsī hoti ādhānaggāhī duppaṭinissaggī…pe… yo so, āvuso, bhikkhu sandiṭṭhiparāmāsī hoti ādhānaggāhī duppaṭinissaggī, so sattharipi agāravo viharati appatisso, dhammepi agāravo viharati appatisso, saṅghepi agāravo viharati appatisso, sikkhāyapi na paripūrakārī hoti. Yo so, āvuso, bhikkhu satthari agāravo viharati appatisso, dhamme agāravo viharati appatisso, saṅghe agāravo viharati appatisso , sikkhāya na paripūrakārī, so saṅghe vivādaṃ janeti. Yo hoti vivādo bahujanaahitāya bahujanaasukhāya anatthāya ahitāya dukkhāya devamanussānaṃ. Evarūpaṃ ce tumhe, āvuso, vivādamūlaṃ ajjhattaṃ vā bahiddhā vā samanupasseyyātha. Tatra tumhe, āvuso, tasseva pāpakassa vivādamūlassa pahānāya vāyameyyātha. Evarūpaṃ ce tumhe, āvuso, vivādamūlaṃ ajjhattaṃ vā bahiddhā vā na samanupasseyyātha. Tatra tumhe, āvuso, tasseva pāpakassa vivādamūlassa āyatiṃ anavassavāya paṭipajjeyyātha. Evametassa pāpakassa vivādamūlassa pahānaṃ hoti. Evametassa pāpakassa vivādamūlassa āyatiṃ anavassavo hoti.

‘‘Cha dhātuyo – pathavīdhātu, āpodhātu, tejodhātu, vāyodhātu, ākāsadhātu, viññāṇadhātu.

\paragraph{326.} ‘‘Cha nissaraṇiyā dhātuyo. Idhāvuso, bhikkhu evaṃ vadeyya – ‘mettā hi kho me cetovimutti bhāvitā bahulīkatā yānīkatā vatthukatā anuṭṭhitā paricitā susamāraddhā, atha ca pana me byāpādo cittaṃ pariyādāya tiṭṭhatī’ti. So ‘mā hevaṃ’, tissa vacanīyo, ‘māyasmā evaṃ avaca, mā bhagavantaṃ abbhācikkhi, na hi sādhu bhagavato abbhakkhānaṃ, na hi bhagavā evaṃ vadeyya. Aṭṭhānametaṃ, āvuso, anavakāso, yaṃ mettāya cetovimuttiyā bhāvitāya bahulīkatāya yānīkatāya vatthukatāya anuṭṭhitāya paricitāya susamāraddhāya. Atha ca panassa byāpādo cittaṃ pariyādāya ṭhassati, netaṃ ṭhānaṃ vijjati. Nissaraṇaṃ hetaṃ, āvuso, byāpādassa, yadidaṃ mettā cetovimuttī’ti.

‘‘Idha panāvuso, bhikkhu evaṃ vadeyya – ‘karuṇā hi kho me cetovimutti bhāvitā bahulīkatā yānīkatā vatthukatā anuṭṭhitā paricitā susamāraddhā. Atha ca pana me vihesā cittaṃ pariyādāya tiṭṭhatī’ti, so ‘mā hevaṃ’ tissa vacanīyo ‘māyasmā evaṃ avaca, mā bhagavantaṃ abbhācikkhi, na hi sādhu bhagavato abbhakkhānaṃ, na hi bhagavā evaṃ vadeyya. Aṭṭhānametaṃ āvuso, anavakāso, yaṃ karuṇāya cetovimuttiyā bhāvitāya bahulīkatāya yānīkatāya vatthukatāya anuṭṭhitāya paricitāya susamāraddhāya, atha ca panassa vihesā cittaṃ pariyādāya ṭhassati, netaṃ ṭhānaṃ vijjati. Nissaraṇaṃ hetaṃ, āvuso, vihesāya, yadidaṃ karuṇā cetovimuttī’ti.

‘‘Idha panāvuso, bhikkhu evaṃ vadeyya – ‘muditā hi kho me cetovimutti bhāvitā bahulīkatā yānīkatā vatthukatā anuṭṭhitā paricitā susamāraddhā. Atha ca pana me arati cittaṃ pariyādāya tiṭṭhatī’ti, so ‘mā hevaṃ’ tissa vacanīyo ‘‘māyasmā evaṃ avaca, mā bhagavantaṃ abbhācikkhi, na hi sādhu bhagavato abbhakkhānaṃ, na hi bhagavā evaṃ vadeyya. Aṭṭhānametaṃ, āvuso, anavakāso, yaṃ muditāya cetovimuttiyā bhāvitāya bahulīkatāya yānīkatāya vatthukatāya anuṭṭhitāya paricitāya susamāraddhāya, atha ca panassa arati cittaṃ pariyādāya ṭhassati, netaṃ ṭhānaṃ vijjati. Nissaraṇaṃ hetaṃ, āvuso, aratiyā, yadidaṃ muditā cetovimuttī’ti.

‘‘Idha panāvuso, bhikkhu evaṃ vadeyya – ‘upekkhā hi kho me cetovimutti bhāvitā bahulīkatā yānīkatā vatthukatā anuṭṭhitā paricitā susamāraddhā. Atha ca pana me rāgo cittaṃ pariyādāya tiṭṭhatī’ti. So ‘mā hevaṃ’ tissa vacanīyo ‘māyasmā evaṃ avaca, mā bhagavantaṃ abbhācikkhi, na hi sādhu bhagavato abbhakkhānaṃ, na hi bhagavā evaṃ vadeyya. Aṭṭhānametaṃ, āvuso, anavakāso, yaṃ upekkhāya cetovimuttiyā bhāvitāya bahulīkatāya yānīkatāya vatthukatāya anuṭṭhitāya paricitāya susamāraddhāya, atha ca panassa rāgo cittaṃ pariyādāya ṭhassati netaṃ ṭhānaṃ vijjati. Nissaraṇaṃ hetaṃ, āvuso, rāgassa, yadidaṃ upekkhā cetovimuttī’ti.

‘‘Idha panāvuso, bhikkhu evaṃ vadeyya – ‘animittā hi kho me cetovimutti bhāvitā bahulīkatā yānīkatā vatthukatā anuṭṭhitā paricitā susamāraddhā. Atha ca pana me nimittānusāri viññāṇaṃ hotī’ti. So ‘mā hevaṃ’ tissa vacanīyo ‘māyasmā evaṃ avaca, mā bhagavantaṃ abbhācikkhi, na hi sādhu bhagavato abbhakkhānaṃ, na hi bhagavā evaṃ vadeyya. Aṭṭhānametaṃ, āvuso, anavakāso, yaṃ animittāya cetovimuttiyā bhāvitāya bahulīkatāya yānīkatāya vatthukatāya anuṭṭhitāya paricitāya susamāraddhāya, atha ca panassa nimittānusāri viññāṇaṃ bhavissati, netaṃ ṭhānaṃ vijjati. Nissaraṇaṃ hetaṃ, āvuso, sabbanimittānaṃ, yadidaṃ animittā cetovimuttī’ti.

‘‘Idha panāvuso, bhikkhu evaṃ vadeyya – ‘asmīti kho me vigataṃ\footnote{vighātaṃ (sī. pī.), vigate (syā. ka.)}, ayamahamasmīti na samanupassāmi, atha ca pana me vicikicchākathaṅkathāsallaṃ cittaṃ pariyādāya tiṭṭhatī’ti. So ‘mā hevaṃ’ tissa vacanīyo ‘māyasmā evaṃ avaca, mā bhagavantaṃ abbhācikkhi, na hi sādhu bhagavato abbhakkhānaṃ, na hi bhagavā evaṃ vadeyya. Aṭṭhānametaṃ, āvuso, anavakāso, yaṃ asmīti vigate\footnote{vighāte (sī. pī.)} ayamahamasmīti asamanupassato, atha ca panassa vicikicchākathaṅkathāsallaṃ cittaṃ pariyādāya ṭhassati, netaṃ ṭhānaṃ vijjati. Nissaraṇaṃ hetaṃ, āvuso, vicikicchākathaṅkathāsallassa, yadidaṃ asmimānasamugghāto’ti.

\paragraph{327.} ‘‘Cha anuttariyāni – dassanānuttariyaṃ, savanānuttariyaṃ, lābhānuttariyaṃ, sikkhānuttariyaṃ, pāricariyānuttariyaṃ, anussatānuttariyaṃ.

‘‘Cha anussatiṭṭhānāni – buddhānussati, dhammānussati, saṅghānussati, sīlānussati, cāgānussati, devatānussati.

\paragraph{328.} ‘‘Cha satatavihārā. Idhāvuso, bhikkhu cakkhunā rūpaṃ disvā neva sumano hoti na dummano, upekkhako\footnote{upekkhako ca (syā. ka.)} viharati sato sampajāno. Sotena saddaṃ sutvā…pe… manasā dhammaṃ viññāya neva sumano hoti na dummano, upekkhako viharati sato sampajāno.

\paragraph{329.}‘‘Chaḷābhijātiyo. Idhāvuso, ekacco kaṇhābhijātiko samāno kaṇhaṃ dhammaṃ abhijāyati. Idha panāvuso, ekacco kaṇhābhijātiko samāno sukkaṃ dhammaṃ abhijāyati. Idha panāvuso, ekacco kaṇhābhijātiko samāno akaṇhaṃ asukkaṃ nibbānaṃ abhijāyati. Idha panāvuso, ekacco sukkābhijātiko samāno sukkaṃ dhammaṃ abhijāyati. Idha panāvuso, ekacco sukkābhijātiko samāno kaṇhaṃ dhammaṃ abhijāyati. Idha panāvuso, ekacco sukkābhijātiko samāno akaṇhaṃ asukkaṃ nibbānaṃ abhijāyati.

‘‘Cha nibbedhabhāgiyā saññā\footnote{nibbedhabhāgiyasaññā (syā. kaṃ.)} – aniccasaññā anicce, dukkhasaññā dukkhe, anattasaññā, pahānasaññā, virāgasaññā, nirodhasaññā.

‘‘Ime kho, āvuso, tena bhagavatā jānatā passatā arahatā sammāsambuddhena cha dhammā sammadakkhātā; tattha sabbeheva saṅgāyitabbaṃ…pe… atthāya hitāya sukhāya devamanussānaṃ.

\subsubsection{Sattakaṃ}

\paragraph{330.} ‘‘Atthi kho, āvuso, tena bhagavatā jānatā passatā arahatā sammāsambuddhena satta dhammā sammadakkhātā; tattha sabbeheva saṅgāyitabbaṃ…pe… atthāya hitāya sukhāya devamanussānaṃ. Katame satta?

‘‘Satta ariyadhanāni – saddhādhanaṃ, sīladhanaṃ, hiridhanaṃ, ottappadhanaṃ, sutadhanaṃ, cāgadhanaṃ, paññādhanaṃ.

‘‘Satta bojjhaṅgā – satisambojjhaṅgo, dhammavicayasambojjhaṅgo , vīriyasambojjhaṅgo, pītisambojjhaṅgo, passaddhisambojjhaṅgo, samādhisambojjhaṅgo, upekkhāsambojjhaṅgo.

‘‘Satta samādhiparikkhārā – sammādiṭṭhi, sammāsaṅkappo, sammāvācā, sammākammanto, sammāājīvo, sammāvāyāmo, sammāsati.

‘‘Satta asaddhammā – idhāvuso, bhikkhu assaddho hoti, ahiriko hoti, anottappī hoti, appassuto hoti, kusīto hoti, muṭṭhassati hoti, duppañño hoti.

‘‘Satta saddhammā – idhāvuso, bhikkhu saddho hoti, hirimā hoti, ottappī hoti, bahussuto hoti, āraddhavīriyo hoti, upaṭṭhitassati hoti, paññavā hoti.

‘‘Satta sappurisadhammā – idhāvuso, bhikkhu dhammaññū ca hoti atthaññū ca attaññū ca mattaññū ca kālaññū ca parisaññū ca puggalaññū ca.

\paragraph{331.} ‘‘Satta niddasavatthūni. Idhāvuso, bhikkhu sikkhāsamādāne tibbacchando hoti, āyatiñca sikkhāsamādāne avigatapemo. Dhammanisantiyā tibbacchando hoti, āyatiñca dhammanisantiyā avigatapemo. Icchāvinaye tibbacchando hoti, āyatiñca icchāvinaye avigatapemo. Paṭisallāne tibbacchando hoti, āyatiñca paṭisallāne avigatapemo. Vīriyārambhe tibbacchando hoti, āyatiñca vīriyārambhe avigatapemo. Satinepakke tibbacchando hoti, āyatiñca satinepakke avigatapemo . Diṭṭhipaṭivedhe tibbacchando hoti, āyatiñca diṭṭhipaṭivedhe avigatapemo.

‘‘Satta saññā – aniccasaññā, anattasaññā, asubhasaññā, ādīnavasaññā, pahānasaññā, virāgasaññā, nirodhasaññā.

‘‘Sattabalāni – saddhābalaṃ, vīriyabalaṃ, hiribalaṃ, ottappabalaṃ, satibalaṃ, samādhibalaṃ, paññābalaṃ.

\paragraph{332.} ‘‘Satta viññāṇaṭṭhitiyo. Santāvuso, sattā nānattakāyā nānattasaññino, seyyathāpi manussā ekacce ca devā ekacce ca vinipātikā. Ayaṃ paṭhamā viññāṇaṭṭhiti.

‘‘Santāvuso, sattā nānattakāyā ekattasaññino seyyathāpi devā brahmakāyikā paṭhamābhinibbattā. Ayaṃ dutiyā viññāṇaṭṭhiti.

‘‘Santāvuso, sattā ekattakāyā nānattasaññino seyyathāpi devā ābhassarā. Ayaṃ tatiyā viññāṇaṭṭhiti.

‘‘Santāvuso, sattā ekattakāyā ekattasaññino seyyathāpi devā subhakiṇhā. Ayaṃ catutthī viññāṇaṭṭhiti.

‘‘Santāvuso, sattā sabbaso rūpasaññānaṃ samatikkamā paṭighasaññānaṃ atthaṅgamā nānattasaññānaṃ amanasikārā ‘ananto ākāso’ti ākāsānañcāyatanūpagā. Ayaṃ pañcamī viññāṇaṭṭhiti.

‘‘Santāvuso, sattā sabbaso ākāsānañcāyatanaṃ samatikkamma ‘anantaṃ viññāṇa’nti viññāṇañcāyatanūpagā. Ayaṃ chaṭṭhī viññāṇaṭṭhiti.

‘‘Santāvuso , sattā sabbaso viññāṇañcāyatanaṃ samatikkamma ‘natthi kiñcī’ti ākiñcaññāyatanūpagā. Ayaṃ sattamī viññāṇaṭṭhiti.

‘‘Satta puggalā dakkhiṇeyyā – ubhatobhāgavimutto , paññāvimutto, kāyasakkhi, diṭṭhippatto, saddhāvimutto, dhammānusārī, saddhānusārī.

‘‘Satta anusayā – kāmarāgānusayo, paṭighānusayo, diṭṭhānusayo, vicikicchānusayo, mānānusayo, bhavarāgānusayo, avijjānusayo.

‘‘Satta saññojanāni – anunayasaññojanaṃ\footnote{kāmasaññojanaṃ (syā. kaṃ.)}, paṭighasaññojanaṃ, diṭṭhisaññojanaṃ, vicikicchāsaññojanaṃ, mānasaññojanaṃ, bhavarāgasaññojanaṃ, avijjāsaññojanaṃ.

‘‘Satta adhikaraṇasamathā – uppannuppannānaṃ adhikaraṇānaṃ samathāya vūpasamāya sammukhāvinayo dātabbo, sativinayo dātabbo, amūḷhavinayo dātabbo, paṭiññāya kāretabbaṃ, yebhuyyasikā, tassapāpiyasikā, tiṇavatthārako.

‘‘Ime kho, āvuso, tena bhagavatā jānatā passatā arahatā sammāsambuddhena satta dhammā sammadakkhātā; tattha sabbeheva saṅgāyitabbaṃ…pe… atthāya hitāya sukhāya devamanussānaṃ.

\xsubsubsectionEnd{Dutiyabhāṇavāro niṭṭhito.}

\subsubsection{Aṭṭhakaṃ}

\paragraph{333.} ‘‘Atthi kho, āvuso, tena bhagavatā jānatā passatā arahatā sammāsambuddhena aṭṭha dhammā sammadakkhātā; tattha sabbeheva saṅgāyitabbaṃ…pe… atthāya hitāya sukhāya devamanussānaṃ. Katame aṭṭha?

‘‘Aṭṭha micchattā – micchādiṭṭhi, micchāsaṅkappo, micchāvācā, micchākammanto, micchāājīvo, micchāvāyāmo micchāsati, micchāsamādhi.

‘‘Aṭṭha sammattā – sammādiṭṭhi, sammāsaṅkappo, sammāvācā, sammākammanto, sammāājīvo, sammāvāyāmo, sammāsati, sammāsamādhi.

‘‘Aṭṭha puggalā dakkhiṇeyyā – sotāpanno, sotāpattiphalasacchikiriyāya paṭipanno; sakadāgāmī, sakadāgāmiphalasacchikiriyāya paṭipanno; anāgāmī, anāgāmiphalasacchikiriyāya paṭipanno; arahā, arahattaphalasacchikiriyāya paṭipanno.

\paragraph{334.} ‘‘Aṭṭha kusītavatthūni. Idhāvuso, bhikkhunā kammaṃ kātabbaṃ hoti. Tassa evaṃ hoti – ‘kammaṃ kho me kātabbaṃ bhavissati, kammaṃ kho pana me karontassa kāyo kilamissati, handāhaṃ nipajjāmī’ti! So nipajjati na vīriyaṃ ārabhati appattassa pattiyā anadhigatassa adhigamāya asacchikatassa sacchikiriyāya. Idaṃ paṭhamaṃ kusītavatthu.

‘‘Puna caparaṃ, āvuso, bhikkhunā kammaṃ kataṃ hoti. Tassa evaṃ hoti – ‘ahaṃ kho kammaṃ akāsiṃ, kammaṃ kho pana me karontassa kāyo kilanto, handāhaṃ nipajjāmī’ti! So nipajjati na vīriyaṃ ārabhati…pe… idaṃ dutiyaṃ kusītavatthu.

‘‘Puna caparaṃ, āvuso, bhikkhunā maggo gantabbo hoti. Tassa evaṃ hoti – ‘maggo kho me gantabbo bhavissati, maggaṃ kho pana me gacchantassa kāyo kilamissati, handāhaṃ nipajjāmī’ti! So nipajjati na vīriyaṃ ārabhati… idaṃ tatiyaṃ kusītavatthu.

‘‘Puna caparaṃ, āvuso, bhikkhunā maggo gato hoti. Tassa evaṃ hoti – ‘ahaṃ kho maggaṃ agamāsiṃ, maggaṃ kho pana me gacchantassa kāyo kilanto, handāhaṃ nipajjāmī’ti! So nipajjati na vīriyaṃ ārabhati… idaṃ catutthaṃ kusītavatthu.

‘‘Puna caparaṃ, āvuso, bhikkhu gāmaṃ vā nigamaṃ vā piṇḍāya caranto na labhati lūkhassa vā paṇītassa vā bhojanassa yāvadatthaṃ pāripūriṃ. Tassa evaṃ hoti – ‘ahaṃ kho gāmaṃ vā nigamaṃ vā piṇḍāya caranto nālatthaṃ lūkhassa vā paṇītassa vā bhojanassa yāvadatthaṃ pāripūriṃ, tassa me kāyo kilanto akammañño, handāhaṃ nipajjāmī’ti! So nipajjati na vīriyaṃ ārabhati… idaṃ pañcamaṃ kusītavatthu.

‘‘Puna caparaṃ, āvuso, bhikkhu gāmaṃ vā nigamaṃ vā piṇḍāya caranto labhati lūkhassa vā paṇītassa vā bhojanassa yāvadatthaṃ pāripūriṃ. Tassa evaṃ hoti – ‘ahaṃ kho gāmaṃ vā nigamaṃ vā piṇḍāya caranto alatthaṃ lūkhassa vā paṇītassa vā bhojanassa yāvadatthaṃ pāripūriṃ, tassa me kāyo garuko akammañño, māsācitaṃ maññe , handāhaṃ nipajjāmī’ti! So nipajjati na vīriyaṃ ārabhati… idaṃ chaṭṭhaṃ kusītavatthu.

‘‘Puna caparaṃ, āvuso, bhikkhuno uppanno hoti appamattako ābādho. Tassa evaṃ hoti – ‘uppanno kho me ayaṃ appamattako ābādho; atthi kappo nipajjituṃ, handāhaṃ nipajjāmī’ti! So nipajjati na vīriyaṃ ārabhati… idaṃ sattamaṃ kusītavatthu.

‘‘Puna caparaṃ, āvuso, bhikkhu gilānā vuṭṭhito\footnote{gilānavuṭṭhito (saddanīti) a. ni. 6.16 nakulapitusuttaṭīkā passitabbā} hoti aciravuṭṭhito gelaññā. Tassa evaṃ hoti – ‘ahaṃ kho gilānā vuṭṭhito aciravuṭṭhito gelaññā, tassa me kāyo dubbalo akammañño, handāhaṃ nipajjāmī’ti! So nipajjati na vīriyaṃ ārabhati appattassa pattiyā anadhigatassa adhigamāya asacchikatassa sacchikiriyāya. Idaṃ aṭṭhamaṃ kusītavatthu.

\paragraph{335.} ‘‘Aṭṭha ārambhavatthūni. Idhāvuso, bhikkhunā kammaṃ kātabbaṃ hoti. Tassa evaṃ hoti – ‘kammaṃ kho me kātabbaṃ bhavissati, kammaṃ kho pana me karontena na sukaraṃ buddhānaṃ sāsanaṃ manasi kātuṃ, handāhaṃ vīriyaṃ ārabhāmi appattassa pattiyā anadhigatassa adhigamāya, asacchikatassa sacchikiriyāyā’ti! So vīriyaṃ ārabhati appattassa pattiyā, anadhigatassa adhigamāya asacchikatassa sacchikiriyāya. Idaṃ paṭhamaṃ ārambhavatthu.

‘‘Puna caparaṃ, āvuso, bhikkhunā kammaṃ kataṃ hoti. Tassa evaṃ hoti – ‘ahaṃ kho kammaṃ akāsiṃ, kammaṃ kho panāhaṃ karonto nāsakkhiṃ buddhānaṃ sāsanaṃ manasi kātuṃ, handāhaṃ vīriyaṃ ārabhāmi…pe… so vīriyaṃ ārabhati… idaṃ dutiyaṃ ārambhavatthu.

‘‘Puna caparaṃ, āvuso, bhikkhunā maggo gantabbo hoti. Tassa evaṃ hoti – ‘maggo kho me gantabbo bhavissati, maggaṃ kho pana me gacchantena na sukaraṃ buddhānaṃ sāsanaṃ manasi kātuṃ. Handāhaṃ vīriyaṃ ārabhāmi…pe… so vīriyaṃ ārabhati… idaṃ tatiyaṃ ārambhavatthu.

‘‘Puna caparaṃ, āvuso, bhikkhunā maggo gato hoti. Tassa evaṃ hoti – ‘ahaṃ kho maggaṃ agamāsiṃ, maggaṃ kho panāhaṃ gacchanto nāsakkhiṃ buddhānaṃ sāsanaṃ manasi kātuṃ, handāhaṃ vīriyaṃ ārabhāmi…pe… so vīriyaṃ ārabhati… idaṃ catutthaṃ ārambhavatthu.

‘‘Puna caparaṃ, āvuso, bhikkhu gāmaṃ vā nigamaṃ vā piṇḍāya caranto na labhati lūkhassa vā paṇītassa vā bhojanassa yāvadatthaṃ pāripūriṃ . Tassa evaṃ hoti – ‘ahaṃ kho gāmaṃ vā nigamaṃ vā piṇḍāya caranto nālatthaṃ lūkhassa vā paṇītassa vā bhojanassa yāvadatthaṃ pāripūriṃ, tassa me kāyo lahuko kammañño, handāhaṃ vīriyaṃ ārabhāmi…pe… so vīriyaṃ ārabhati… idaṃ pañcamaṃ ārambhavatthu.

‘‘Puna caparaṃ, āvuso, bhikkhu gāmaṃ vā nigamaṃ vā piṇḍāya caranto labhati lūkhassa vā paṇītassa vā bhojanassa yāvadatthaṃ pāripūriṃ. Tassa evaṃ hoti – ‘ahaṃ kho gāmaṃ vā nigamaṃ vā piṇḍāya caranto alatthaṃ lūkhassa vā paṇītassa vā bhojanassa yāvadatthaṃ pāripūriṃ, tassa me kāyo balavā kammañño, handāhaṃ vīriyaṃ ārabhāmi…pe… so vīriyaṃ ārabhati… idaṃ chaṭṭhaṃ ārambhavatthu .

‘‘Puna caparaṃ, āvuso, bhikkhuno uppanno hoti appamattako ābādho. Tassa evaṃ hoti – ‘uppanno kho me ayaṃ appamattako ābādho, ṭhānaṃ kho panetaṃ vijjati yaṃ me ābādho pavaḍḍheyya, handāhaṃ vīriyaṃ ārabhāmi…pe… so vīriyaṃ ārabhati… idaṃ sattamaṃ ārambhavatthu.

‘‘Puna caparaṃ, āvuso, bhikkhu gilānā vuṭṭhito hoti aciravuṭṭhito gelaññā. Tassa evaṃ hoti – ‘ahaṃ kho gilānā vuṭṭhito aciravuṭṭhito gelaññā, ṭhānaṃ kho panetaṃ vijjati yaṃ me ābādho paccudāvatteyya, handāhaṃ vīriyaṃ ārabhāmi appattassa pattiyā anadhigatassa adhigamāya asacchikatassa sacchikiriyāyā’’ti! So vīriyaṃ ārabhati appattassa pattiyā anadhigatassa adhigamāya asacchikatassa sacchikiriyāya. Idaṃ aṭṭhamaṃ ārambhavatthu.

\paragraph{336.} ‘‘Aṭṭha dānavatthūni. Āsajja dānaṃ deti, bhayā dānaṃ deti, ‘adāsi me’ti dānaṃ deti, ‘dassati me’ti dānaṃ deti, ‘sāhu dāna’nti dānaṃ deti, ‘ahaṃ pacāmi, ime na pacanti, nārahāmi pacanto apacantānaṃ dānaṃ na dātu’nti dānaṃ deti, ‘idaṃ me dānaṃ dadato kalyāṇo kittisaddo abbhuggacchatī’ti dānaṃ deti. Cittālaṅkāra-cittaparikkhāratthaṃ dānaṃ deti.

\paragraph{337.} ‘‘Aṭṭha dānūpapattiyo. Idhāvuso, ekacco dānaṃ deti samaṇassa vā brāhmaṇassa vā annaṃ pānaṃ vatthaṃ yānaṃ mālāgandhavilepanaṃ seyyāvasathapadīpeyyaṃ. So yaṃ deti taṃ paccāsīsati\footnote{paccāsiṃsati (sī. syā. kaṃ. pī.)}. So passati khattiyamahāsālaṃ vā brāhmaṇamahāsālaṃ vā gahapatimahāsālaṃ vā pañcahi kāmaguṇehi samappitaṃ samaṅgībhūtaṃ paricārayamānaṃ. Tassa evaṃ hoti – ‘aho vatāhaṃ kāyassa bhedā paraṃ maraṇā khattiyamahāsālānaṃ vā brāhmaṇamahāsālānaṃ vā gahapatimahāsālānaṃ vā sahabyataṃ upapajjeyya’nti! So taṃ cittaṃ dahati, taṃ cittaṃ adhiṭṭhāti, taṃ cittaṃ bhāveti, tassa taṃ cittaṃ hīne vimuttaṃ uttari abhāvitaṃ tatrūpapattiyā saṃvattati . Tañca kho sīlavato vadāmi no dussīlassa. Ijjhatāvuso, sīlavato cetopaṇidhi visuddhattā.

‘‘Puna caparaṃ, āvuso, idhekacco dānaṃ deti samaṇassa vā brāhmaṇassa vā annaṃ pānaṃ…pe… seyyāvasathapadīpeyyaṃ. So yaṃ deti taṃ paccāsīsati. Tassa sutaṃ hoti – ‘cātumahārājikā\footnote{cātummahārājikā (sī. syā. pī.)} devā dīghāyukā vaṇṇavanto sukhabahulā’’ti. Tassa evaṃ hoti – ‘aho vatāhaṃ kāyassa bhedā paraṃ maraṇā cātumahārājikānaṃ devānaṃ sahabyataṃ upapajjeyya’’nti! So taṃ cittaṃ dahati, taṃ cittaṃ adhiṭṭhāti, taṃ cittaṃ bhāveti, tassa taṃ cittaṃ hīne vimuttaṃ uttari abhāvitaṃ tatrūpapattiyā saṃvattati. Tañca kho sīlavato vadāmi no dussīlassa. Ijjhatāvuso, sīlavato cetopaṇidhi visuddhattā.

‘‘Puna caparaṃ, āvuso, idhekacco dānaṃ deti samaṇassa vā brāhmaṇassa vā annaṃ pānaṃ…pe… seyyāvasathapadīpeyyaṃ. So yaṃ deti taṃ paccāsīsati. Tassa sutaṃ hoti – ‘tāvatiṃsā devā…pe… yāmā devā…pe… tusitā devā…pe… nimmānaratī devā…pe… paranimmitavasavattī devā dīghāyukā vaṇṇavanto sukhabahulā’ti. Tassa evaṃ hoti – ‘aho vatāhaṃ kāyassa bhedā paraṃ maraṇā paranimmitavasavattīnaṃ devānaṃ sahabyataṃ upapajjeyya’’nti! So taṃ cittaṃ dahati, taṃ cittaṃ adhiṭṭhāti, taṃ cittaṃ bhāveti, tassa taṃ cittaṃ hīne vimuttaṃ uttari abhāvitaṃ tatrūpapattiyā saṃvattati. Tañca kho sīlavato vadāmi no dussīlassa. Ijjhatāvuso, sīlavato cetopaṇidhi visuddhattā.

‘‘Puna caparaṃ, āvuso, idhekacco dānaṃ deti samaṇassa vā brāhmaṇassa vā annaṃ pānaṃ vatthaṃ yānaṃ mālāgandhavilepanaṃ seyyāvasathapadīpeyyaṃ. So yaṃ deti taṃ paccāsīsati. Tassa sutaṃ hoti – ‘brahmakāyikā devā dīghāyukā vaṇṇavanto sukhabahulā’ti. Tassa evaṃ hoti – ‘aho vatāhaṃ kāyassa bhedā paraṃ maraṇā brahmakāyikānaṃ devānaṃ sahabyataṃ upapajjeyya’nti! So taṃ cittaṃ dahati, taṃ cittaṃ adhiṭṭhāti, taṃ cittaṃ bhāveti, tassa taṃ cittaṃ hīne vimuttaṃ uttari abhāvitaṃ tatrūpapattiyā saṃvattati. Tañca kho sīlavato vadāmi no dussīlassa; vītarāgassa no sarāgassa. Ijjhatāvuso, sīlavato cetopaṇidhi vītarāgattā.

‘‘Aṭṭha parisā – khattiyaparisā, brāhmaṇaparisā, gahapatiparisā, samaṇaparisā, cātumahārājikaparisā, tāvatiṃsaparisā, māraparisā, brahmaparisā .

‘‘Aṭṭha lokadhammā – lābho ca, alābho ca, yaso ca, ayaso ca, nindā ca, pasaṃsā ca, sukhañca, dukkhañca.

\paragraph{338.} ‘‘Aṭṭha abhibhāyatanāni. Ajjhattaṃ rūpasaññī eko bahiddhā rūpāni passati parittāni suvaṇṇadubbaṇṇāni, ‘tāni abhibhuyya jānāmi passāmī’ti evaṃsaññī hoti. Idaṃ paṭhamaṃ abhibhāyatanaṃ.

‘‘Ajjhattaṃ rūpasaññī eko bahiddhā rūpāni passati appamāṇāni suvaṇṇadubbaṇṇāni, ‘tāni abhibhuyya jānāmi passāmī’ti – evaṃsaññī hoti. Idaṃ dutiyaṃ abhibhāyatanaṃ.

‘‘Ajjhattaṃ arūpasaññī eko bahiddhā rūpāni passati parittāni suvaṇṇadubbaṇṇāni, ‘tāni abhibhuyya jānāmi passāmī’ti evaṃsaññī hoti. Idaṃ tatiyaṃ abhibhāyatanaṃ.

‘‘Ajjhattaṃ arūpasaññī eko bahiddhā rūpāni passati appamāṇāni suvaṇṇadubbaṇṇāni, ‘tāni abhibhuyya jānāmi passāmī’ti evaṃsaññī hoti. Idaṃ catutthaṃ abhibhāyatanaṃ.

‘‘Ajjhattaṃ arūpasaññī eko bahiddhā rūpāni passati nīlāni nīlavaṇṇāni nīlanidassanāni nīlanibhāsāni. Seyyathāpi nāma umāpupphaṃ nīlaṃ nīlavaṇṇaṃ nīlanidassanaṃ nīlanibhāsaṃ, seyyathā vā pana taṃ vatthaṃ bārāṇaseyyakaṃ ubhatobhāgavimaṭṭhaṃ nīlaṃ nīlavaṇṇaṃ nīlanidassanaṃ nīlanibhāsaṃ. Evameva\footnote{evamevaṃ (ka.)} ajjhattaṃ arūpasaññī eko bahiddhā rūpāni passati nīlāni nīlavaṇṇāni nīlanidassanāni nīlanibhāsāni, ‘tāni abhibhuyya jānāmi passāmī’ti evaṃsaññī hoti. Idaṃ pañcamaṃ abhibhāyatanaṃ.

‘‘Ajjhattaṃ arūpasaññī eko bahiddhā rūpāni passati pītāni pītavaṇṇāni pītanidassanāni pītanibhāsāni. Seyyathāpi nāma kaṇikārapupphaṃ\footnote{kaṇṇikārapupphaṃ (syā. kaṃ.)} pītaṃ pītavaṇṇaṃ pītanidassanaṃ pītanibhāsaṃ, seyyathā vā pana taṃ vatthaṃ bārāṇaseyyakaṃ ubhatobhāgavimaṭṭhaṃ pītaṃ pītavaṇṇaṃ pītanidassanaṃ pītanibhāsaṃ. Evameva ajjhattaṃ arūpasaññī eko bahiddhā rūpāni passati pītāni pītavaṇṇāni pītanidassanāni pītanibhāsāni, ‘tāni abhibhuyya jānāmi passāmī’ti evaṃsaññī hoti. Idaṃ chaṭṭhaṃ abhibhāyatanaṃ.

‘‘Ajjhattaṃ arūpasaññī eko bahiddhā rūpāni passati lohitakāni lohitakavaṇṇāni lohitakanidassanāni lohitakanibhāsāni. Seyyathāpi nāma bandhujīvakapupphaṃ lohitakaṃ lohitakavaṇṇaṃ lohitakanidassanaṃ lohitakanibhāsaṃ, seyyathā vā pana taṃ vatthaṃ bārāṇaseyyakaṃ ubhatobhāgavimaṭṭhaṃ lohitakaṃ lohitakavaṇṇaṃ lohitakanidassanaṃ lohitakanibhāsaṃ. Evameva ajjhattaṃ arūpasaññī eko bahiddhā rūpāni passati lohitakāni lohitakavaṇṇāni lohitakanidassanāni lohitakanibhāsāni, ‘tāni abhibhuyya jānāmi passāmī’ti evaṃsaññī hoti. Idaṃ sattamaṃ abhibhāyatanaṃ.

‘‘Ajjhattaṃ arūpasaññī eko bahiddhā rūpāni passati odātāni odātavaṇṇāni odātanidassanāni odātanibhāsāni. Seyyathāpi nāma osadhitārakā odātā odātavaṇṇā odātanidassanā odātanibhāsā, seyyathā vā pana taṃ vatthaṃ bārāṇaseyyakaṃ ubhatobhāgavimaṭṭhaṃ odātaṃ odātavaṇṇaṃ odātanidassanaṃ odātanibhāsaṃ. Evameva ajjhattaṃ arūpasaññī eko bahiddhā rūpāni passati odātāni odātavaṇṇāni odātanidassanāni odātanibhāsāni , ‘tāni abhibhuyya jānāmi passāmī’ti evaṃsaññī hoti. Idaṃ aṭṭhamaṃ abhibhāyatanaṃ.

\paragraph{339.} ‘‘Aṭṭha vimokkhā. Rūpī rūpāni passati. Ayaṃ paṭhamo vimokkho.

‘‘Ajjhattaṃ arūpasaññī bahiddhā rūpāni passati. Ayaṃ dutiyo vimokkho.

‘‘Subhanteva adhimutto hoti. Ayaṃ tatiyo vimokkho.

‘‘Sabbaso rūpasaññānaṃ samatikkamā paṭighasaññānaṃ atthaṅgamā nānattasaññānaṃ amanasikārā ‘ananto ākāso’ti ākāsānañcāyatanaṃ upasampajja viharati. Ayaṃ catuttho vimokkho.

‘‘Sabbaso ākāsānañcāyatanaṃ samatikkamma ‘anantaṃ viññāṇa’nti viññāṇañcāyatanaṃ upasampajja viharati. Ayaṃ pañcamo vimokkho.

‘‘Sabbaso viññāṇañcāyatanaṃ samatikkamma ‘natthi kiñcī’ti ākiñcaññāyatanaṃ upasampajja viharati. Ayaṃ chaṭṭho vimokkho.

‘‘Sabbaso ākiñcaññāyatanaṃ samatikkamma nevasaññānāsaññāyatanaṃ upasampajja viharati. Ayaṃ sattamo vimokkho.

‘‘Sabbaso nevasaññānāsaññāyatanaṃ samatikkamma saññāvedayita nirodhaṃ upasampajja viharati. Ayaṃ aṭṭhamo vimokkho.

‘‘Ime kho, āvuso, tena bhagavatā jānatā passatā arahatā sammāsambuddhena aṭṭha dhammā sammadakkhātā; tattha sabbeheva saṅgāyitabbaṃ…pe… atthāya hitāya sukhāya devamanussānaṃ.

\subsubsection{Navakaṃ}

\paragraph{340.} ‘‘Atthi kho, āvuso, tena bhagavatā jānatā passatā arahatā sammāsambuddhena nava dhammā sammadakkhātā; tattha sabbeheva saṅgāyitabbaṃ…pe… atthāya hitāya sukhāya devamanussānaṃ. Katame nava?

‘‘Nava āghātavatthūni. ‘Anatthaṃ me acarī’ti āghātaṃ bandhati; ‘anatthaṃ me caratī’ti āghātaṃ bandhati; ‘anatthaṃ me carissatī’ti āghātaṃ bandhati; ‘piyassa me manāpassa anatthaṃ acarī’ti āghātaṃ bandhati…pe… anatthaṃ caratīti āghātaṃ bandhati…pe… anatthaṃ carissatīti āghātaṃ bandhati; ‘appiyassa me amanāpassa atthaṃ acarī’ti āghātaṃ bandhati…pe… atthaṃ caratīti āghātaṃ bandhati…pe… atthaṃ carissatīti āghātaṃ bandhati.

‘‘Nava āghātapaṭivinayā. ‘Anatthaṃ me acari\footnote{acarīti (syā. ka.) evaṃ ‘‘carati carissati’’ padesupi}, taṃ kutettha labbhā’ti āghātaṃ paṭivineti ; ‘anatthaṃ me carati, taṃ kutettha labbhā’ti āghātaṃ paṭivineti; ‘anatthaṃ me carissati, taṃ kutettha labbhā’ti āghātaṃ paṭivineti; ‘piyassa me manāpassa anatthaṃ acari…pe… anatthaṃ carati…pe… anatthaṃ carissati, taṃ kutettha labbhā’ti āghātaṃ paṭivineti; ‘appiyassa me amanāpassa atthaṃ acari…pe… atthaṃ carati…pe… atthaṃ carissati, taṃ kutettha labbhā’ti āghātaṃ paṭivineti.

\paragraph{341.} ‘‘Nava sattāvāsā. Santāvuso, sattā nānattakāyā nānattasaññino, seyyathāpi manussā ekacce ca devā ekacce ca vinipātikā. Ayaṃ paṭhamo sattāvāso.

‘‘Santāvuso, sattā nānattakāyā ekattasaññino, seyyathāpi devā brahmakāyikā paṭhamābhinibbattā. Ayaṃ dutiyo sattāvāso.

‘‘Santāvuso, sattā ekattakāyā nānattasaññino, seyyathāpi devā ābhassarā. Ayaṃ tatiyo sattāvāso.

‘‘Santāvuso , sattā ekattakāyā ekattasaññino, seyyathāpi devā subhakiṇhā. Ayaṃ catuttho sattāvāso.

‘‘Santāvuso, sattā asaññino appaṭisaṃvedino, seyyathāpi devā asaññasattā\footnote{asaññisattā (syā. kaṃ.)}. Ayaṃ pañcamo sattāvāso.

‘‘Santāvuso, sattā sabbaso rūpasaññānaṃ samatikkamā paṭighasaññānaṃ atthaṅgamā nānattasaññānaṃ amanasikārā ‘ananto ākāso’ti ākāsānañcāyatanūpagā. Ayaṃ chaṭṭho sattāvāso.

‘‘Santāvuso, sattā sabbaso ākāsānañcāyatanaṃ samatikkamma ‘anantaṃ viññāṇa’nti viññāṇañcāyatanūpagā. Ayaṃ sattamo sattāvāso.

‘‘Santāvuso, sattā sabbaso viññāṇañcāyatanaṃ samatikkamma ‘natthi kiñcī’ti ākiñcāññāyatanūpagā. Ayaṃ aṭṭhamo sattāvāso.

‘‘Santāvuso , sattā sabbaso ākiñcaññāyatanaṃ samatikkamma\footnote{samatikkamma santametaṃ paṇītametanti (syā. kaṃ.)} nevasaññānāsaññāyatanūpagā. Ayaṃ navamo sattāvāso.

\paragraph{342.} ‘‘Nava akkhaṇā asamayā brahmacariyavāsāya. Idhāvuso , tathāgato ca loke uppanno hoti arahaṃ sammāsambuddho, dhammo ca desiyati opasamiko parinibbāniko sambodhagāmī sugatappavedito. Ayañca puggalo nirayaṃ upapanno hoti. Ayaṃ paṭhamo akkhaṇo asamayo brahmacariyavāsāya.

‘‘Puna caparaṃ, āvuso, tathāgato ca loke uppanno hoti arahaṃ sammāsambuddho, dhammo ca desiyati opasamiko parinibbāniko sambodhagāmī sugatappavedito. Ayañca puggalo tiracchānayoniṃ upapanno hoti. Ayaṃ dutiyo akkhaṇo asamayo brahmacariyavāsāya.

‘‘Puna caparaṃ…pe… pettivisayaṃ upapanno hoti. Ayaṃ tatiyo akkhaṇo asamayo brahmacariyavāsāya.

‘‘Puna caparaṃ…pe… asurakāyaṃ upapanno hoti. Ayaṃ catuttho akkhaṇo asamayo brahmacariyavāsāya.

‘‘Puna caparaṃ…pe… aññataraṃ dīghāyukaṃ devanikāyaṃ upapanno hoti. Ayaṃ pañcamo akkhaṇo asamayo brahmacariyavāsāya.

‘‘Puna caparaṃ…pe… paccantimesu janapadesu paccājāto hoti milakkhesu\footnote{milakkhakesu (syā. kaṃ.) milakkhūsu (ka.)} aviññātāresu, yattha natthi gati bhikkhūnaṃ bhikkhunīnaṃ upāsakānaṃ upāsikānaṃ. Ayaṃ chaṭṭho akkhaṇo asamayo brahmacariyavāsāya.

‘‘Puna caparaṃ…pe… majjhimesu janapadesu paccājāto hoti. So ca hoti micchādiṭṭhiko viparītadassano – ‘natthi dinnaṃ, natthi yiṭṭhaṃ, natthi hutaṃ, natthi sukatadukkaṭānaṃ\footnote{sukaṭa dukkaṭānaṃ (sī. pī.)} kammānaṃ phalaṃ vipāko, natthi ayaṃ loko, natthi paro loko, natthi mātā, natthi pitā, natthi sattā opapātikā, natthi loke samaṇabrāhmaṇā sammaggatā sammāpaṭipannā ye imañca lokaṃ parañca lokaṃ sayaṃ abhiññā sacchikatvā pavedentī’ti. Ayaṃ sattamo akkhaṇo asamayo brahmacariyavāsāya.

‘‘Puna caparaṃ…pe… majjhimesu janapadesu paccājāto hoti. So ca hoti duppañño jaḷo eḷamūgo, nappaṭibalo subhāsitadubbhāsitānamatthamaññātuṃ. Ayaṃ aṭṭhamo akkhaṇo asamayo brahmacariyavāsāya.

‘‘Puna caparaṃ, āvuso, tathāgato ca loke na\footnote{katthaci nakāro na dissati} uppanno hoti arahaṃ sammāsambuddho, dhammo ca na desiyati opasamiko parinibbāniko sambodhagāmī sugatappavedito. Ayañca puggalo majjhimesu janapadesu paccājāto hoti, so ca hoti paññavā ajaḷo aneḷamūgo, paṭibalo subhāsita-dubbhāsitānamatthamaññātuṃ. Ayaṃ navamo akkhaṇo asamayo brahmacariyavāsāya.

\paragraph{343.} ‘‘Nava anupubbavihārā. Idhāvuso, bhikkhu vivicceva kāmehi vivicca akusalehi dhammehi savitakkaṃ savicāraṃ vivekajaṃ pītisukhaṃ paṭhamaṃ jhānaṃ upasampajja viharati. Vitakkavicārānaṃ vūpasamā…pe… dutiyaṃ jhānaṃ upasampajja viharati. Pītiyā ca virāgā…pe… tatiyaṃ jhānaṃ upasampajja viharati. Sukhassa ca pahānā…pe… catutthaṃ jhānaṃ upasampajja viharati. Sabbaso rūpasaññānaṃ samatikkamā…pe… ākāsānañcāyatanaṃ upasampajja viharati. Sabbaso ākāsānañcāyatanaṃ samatikkamma ‘anantaṃ viññāṇa’nti viññāṇañcāyatanaṃ upasampajja viharati. Sabbaso viññāṇañcāyatanaṃ samatikkamma ‘natthi kiñcī’ti ākiñcaññāyatanaṃ upasampajja viharati. Sabbaso ākiñcaññāyatanaṃ samatikkamma nevasaññānāsaññāyatanaṃ upasampajja viharati. Sabbaso nevasaññānāsaññāyatanaṃ samatikkamma saññāvedayitanirodhaṃ upasampajja viharati.

\paragraph{344.} ‘‘Nava anupubbanirodhā. Paṭhamaṃ jhānaṃ samāpannassa kāmasaññā niruddhā hoti. Dutiyaṃ jhānaṃ samāpannassa vitakkavicārā niruddhā honti. Tatiyaṃ jhānaṃ samāpannassa pīti niruddhā hoti. Catutthaṃ jhānaṃ samāpannassa assāsapassāssā niruddhā honti. Ākāsānañcāyatanaṃ samāpannassa rūpasaññā niruddhā hoti. Viññāṇañcāyatanaṃ samāpannassa ākāsānañcāyatanasaññā niruddhā hoti. Ākiñcaññāyatanaṃ samāpannassa viññāṇañcāyatanasaññā niruddhā hoti. Nevasaññānāsaññāyatanaṃ samāpannassa ākiñcaññāyatanasaññā niruddhā hoti. Saññāvedayitanirodhaṃ samāpannassa saññā ca vedanā ca niruddhā honti.

‘‘Ime kho, āvuso, tena bhagavatā jānatā passatā arahatā sammāsambuddhena nava dhammā sammadakkhātā. Tattha sabbeheva saṅgāyitabbaṃ…pe… atthāya hitāya sukhāya devamanussānaṃ.

\subsubsection{Dasakaṃ}

\paragraph{345.} ‘‘Atthi kho, āvuso, tena bhagavatā jānatā passatā arahatā sammāsambuddhena dasa dhammā sammadakkhātā. Tattha sabbeheva saṅgāyitabbaṃ…pe… atthāya hitāya sukhāya devamanussānaṃ. Katame dasa?

‘‘Dasa nāthakaraṇā dhammā. Idhāvuso, bhikkhu sīlavā hoti. Pātimokkhasaṃvarasaṃvuto viharati ācāragocarasampanno, aṇumattesu vajjesu bhayadassāvī samādāya sikkhati sikkhāpadesu. Yaṃpāvuso, bhikkhu sīlavā hoti, pātimokkhasaṃvarasaṃvuto viharati, ācāragocarasampanno, aṇumattesu vajjesu bhayadassāvī samādāya sikkhati sikkhāpadesu. Ayampi dhammo nāthakaraṇo.

‘‘Puna caparaṃ, āvuso, bhikkhu bahussuto hoti sutadharo sutasannicayo. Ye te dhammā ādikalyāṇā majjhekalyāṇā pariyosānakalyāṇā sātthā sabyañjanā\footnote{sātthaṃ sabyañjanaṃ (sī. syā. pī.)} kevalaparipuṇṇaṃ parisuddhaṃ brahmacariyaṃ abhivadanti, tathārūpāssa dhammā bahussutā honti\footnote{dhatā (ka. sī. syā. kaṃ.)} dhātā vacasā paricitā manasānupekkhitā diṭṭhiyā suppaṭividdhā, yaṃpāvuso, bhikkhu bahussuto hoti…pe… diṭṭhiyā suppaṭividdhā. Ayampi dhammo nāthakaraṇo.

‘‘Puna caparaṃ, āvuso, bhikkhu kalyāṇamitto hoti kalyāṇasahāyo kalyāṇasampavaṅko. Yaṃpāvuso, bhikkhu kalyāṇamitto hoti kalyāṇasahāyo kalyāṇasampavaṅko. Ayampi dhammo nāthakaraṇo.

‘‘Puna caparaṃ, āvuso, bhikkhu suvaco hoti sovacassakaraṇehi dhammehi samannāgato khamo padakkhiṇaggāhī anusāsaniṃ. Yaṃpāvuso, bhikkhu suvaco hoti…pe… padakkhiṇaggāhī anusāsaniṃ. Ayampi dhammo nāthakaraṇo.

‘‘Puna caparaṃ, āvuso, bhikkhu yāni tāni sabrahmacārīnaṃ uccāvacāni kiṃkaraṇīyāni, tattha dakkho hoti analaso tatrupāyāya vīmaṃsāya samannāgato, alaṃ kātuṃ alaṃ saṃvidhātuṃ. Yaṃpāvuso, bhikkhu yāni tāni sabrahmacārīnaṃ…pe… alaṃ saṃvidhātuṃ. Ayampi dhammo nāthakaraṇo.

‘‘Puna caparaṃ, āvuso, bhikkhu dhammakāmo hoti piyasamudāhāro, abhidhamme abhivinaye uḷārapāmojjo\footnote{uḷārapāmujjo (sī. pī.), oḷārapāmojjo (syā. kaṃ.)}. Yaṃpāvuso, bhikkhu dhammakāmo hoti…pe… uḷārapāmojjo\footnote{uḷārapāmujjo (sī. pī.), oḷārapāmojjo (syā. kaṃ.)}. Ayampi dhammo nāthakaraṇo.

‘‘Puna caparaṃ, āvuso , bhikkhu santuṭṭho hoti itarītarehi cīvarapiṇḍapātasenāsanagilānappaccayabhesajjaparikkhārehi . Yaṃpāvuso, bhikkhu santuṭṭho hoti…pe… parikkhārehi. Ayampi dhammo nāthakaraṇo.

‘‘Puna caparaṃ, āvuso, bhikkhu āraddhavīriyo viharati akusalānaṃ dhammānaṃ pahānāya kusalānaṃ dhammānaṃ upasampadāya, thāmavā daḷhaparakkamo anikkhittadhuro kusalesu dhammesu. Yaṃpāvuso, bhikkhu āraddhavīriyo viharati…pe… anikkhittadhuro kusalesu dhammesu. Ayampi dhammo nāthakaraṇo.

‘‘Puna caparaṃ, āvuso, bhikkhu satimā hoti paramena satinepakkena samannāgato cirakatampi cirabhāsitampi saritā anussaritā. Yaṃpāvuso, bhikkhu satimā hoti…pe… saritā anussaritā. Ayampi dhammo nāthakaraṇo.

‘‘Puna caparaṃ, āvuso, bhikkhu paññavā hoti, udayatthagāminiyā paññāya samannāgato ariyāya nibbedhikāya sammādukkhakkhayagāminiyā. Yaṃpāvuso, bhikkhu paññavā hoti…pe… sammādukkhakkhayagāminiyā. Ayampi dhammo nāthakaraṇo.

\paragraph{346.} Dasa kasiṇāyatanāni. Pathavīkasiṇameko sañjānāti, uddhaṃ adho tiriyaṃ advayaṃ appamāṇaṃ. Āpokasiṇameko sañjānāti…pe… tejokasiṇameko sañjānāti… vāyokasiṇameko sañjānāti… nīlakasiṇameko sañjānāti… pītakasiṇameko sañjānāti… lohitakasiṇameko sañjānāti… odātakasiṇameko sañjānāti… ākāsakasiṇameko sañjānāti… viññāṇakasiṇameko sañjānāti, uddhaṃ adho tiriyaṃ advayaṃ appamāṇaṃ.

\paragraph{347.} ‘‘Dasa akusalakammapathā – pāṇātipāto, adinnādānaṃ, kāmesumicchācāro, musāvādo, pisuṇā vācā, pharusā vācā, samphappalāpo, abhijjhā, byāpādo, micchādiṭṭhi.

‘‘Dasa kusalakammapathā – pāṇātipātā veramaṇī, adinnādānā veramaṇī, kāmesumicchācārā veramaṇī, musāvādā veramaṇī, pisuṇāya vācāya veramaṇī, pharusāya vācāya veramaṇī, samphappalāpā veramaṇī, anabhijjhā, abyāpādo, sammādiṭṭhi.

\paragraph{348.} ‘‘Dasa ariyavāsā. Idhāvuso, bhikkhu pañcaṅgavippahīno hoti, chaḷaṅgasamannāgato, ekārakkho, caturāpasseno, paṇunnapaccekasacco, samavayasaṭṭhesano, anāvilasaṅkappo, passaddhakāyasaṅkhāro, suvimuttacitto, suvimuttapañño.

‘‘Kathañcāvuso, bhikkhu pañcaṅgavippahīno hoti? Idhāvuso, bhikkhuno kāmacchando pahīno hoti, byāpādo pahīno hoti, thinamiddhaṃ pahīnaṃ hoti, uddhaccakukuccaṃ pahīnaṃ hoti, vicikicchā pahīnā hoti. Evaṃ kho, āvuso, bhikkhu pañcaṅgavippahīno hoti.

‘‘Kathañcāvuso, bhikkhu chaḷaṅgasamannāgato hoti? Idhāvuso, bhikkhu cakkhunā rūpaṃ disvā neva sumano hoti na dummano, upekkhako viharati sato sampajāno. Sotena saddaṃ sutvā…pe… manasā dhammaṃ viññāya neva sumano hoti na dummano, upekkhako viharati sato sampajāno. Evaṃ kho, āvuso, bhikkhu chaḷaṅgasamannāgato hoti.

‘‘Kathañcāvuso , bhikkhu ekārakkho hoti? Idhāvuso, bhikkhu satārakkhena cetasā samannāgato hoti. Evaṃ kho, āvuso, bhikkhu ekārakkho hoti .

‘‘Kathañcāvuso, bhikkhu caturāpasseno hoti? Idhāvuso, bhikkhu saṅkhāyekaṃ paṭisevati, saṅkhāyekaṃ adhivāseti, saṅkhāyekaṃ parivajjeti, saṅkhāyekaṃ vinodeti. Evaṃ kho, āvuso, bhikkhu caturāpasseno hoti.

‘‘Kathañcāvuso, bhikkhu paṇunnapaccekasacco hoti? Idhāvuso, bhikkhuno yāni tāni puthusamaṇabrāhmaṇānaṃ puthupaccekasaccāni, sabbāni tāni nunnāni honti paṇunnāni cattāni vantāni muttāni pahīnāni paṭinissaṭṭhāni. Evaṃ kho, āvuso, bhikkhu paṇunnapaccekasacco hoti.

‘‘Kathañcāvuso , bhikkhu samavayasaṭṭhesano hoti? Idhāvuso, bhikkhuno kāmesanā pahīnā hoti, bhavesanā pahīnā hoti, brahmacariyesanā paṭippassaddhā. Evaṃ kho, āvuso, bhikkhu samavayasaṭṭhesano hoti.

‘‘Kathañcāvuso, bhikkhu anāvilasaṅkappo hoti? Idhāvuso, bhikkhuno kāmasaṅkappo pahīno hoti, byāpādasaṅkappo pahīno hoti, vihiṃsāsaṅkappo pahīno hoti. Evaṃ kho, āvuso, bhikkhu anāvilasaṅkappo hoti.

‘‘Kathañcāvuso, bhikkhu passaddhakāyasaṅkhāro hoti ? Idhāvuso, bhikkhu sukhassa ca pahānā dukkhassa ca pahānā pubbeva somanassadomanassānaṃ atthaṅgamā adukkhamasukhaṃ upekkhāsatipārisuddhiṃ catutthaṃ jhānaṃ upasampajja viharati. Evaṃ kho, āvuso, bhikkhu passaddhakāyasaṅkhāro hoti.

‘‘Kathañcāvuso, bhikkhu suvimuttacitto hoti? Idhāvuso, bhikkhuno rāgā cittaṃ vimuttaṃ hoti, dosā cittaṃ vimuttaṃ hoti, mohā cittaṃ vimuttaṃ hoti. Evaṃ kho, āvuso, bhikkhu suvimuttacitto hoti.

‘‘Kathañcāvuso, bhikkhu suvimuttapañño hoti? Idhāvuso, bhikkhu ‘rāgo me pahīno ucchinnamūlo tālāvatthukato anabhāvaṃkato āyatiṃ anuppādadhammo’ti pajānāti. ‘Doso me pahīno ucchinnamūlo tālāvatthukato anabhāvaṃkato āyatiṃ anuppādadhammo’ti pajānāti. ‘Moho me pahīno ucchinnamūlo tālāvatthukato anabhāvaṃkato āyatiṃ anuppādadhammo’ti pajānāti. Evaṃ kho, āvuso, bhikkhu suvimuttapañño hoti.

‘‘Dasa asekkhā dhammā – asekkhā sammādiṭṭhi, asekkho sammāsaṅkappo, asekkhā sammāvācā, asekkho sammākammanto, asekkho sammāājīvo, asekkho sammāvāyāmo, asekkhā sammāsati, asekkho sammāsamādhi, asekkhaṃ sammāñāṇaṃ, asekkhā sammāvimutti.

‘‘Ime kho, āvuso, tena bhagavatā jānatā passatā arahatā sammāsambuddhena dasa dhammā sammadakkhātā. Tattha sabbeheva saṅgāyitabbaṃ na vivaditabbaṃ, yathayidaṃ brahmacariyaṃ addhaniyaṃ assa ciraṭṭhitikaṃ, tadassa bahujanahitāya bahujanasukhāya lokānukampāya atthāya hitāya sukhāya devamanussāna’’nti.

\paragraph{349.} Atha kho bhagavā uṭṭhahitvā āyasmantaṃ sāriputtaṃ āmantesi – ‘sādhu sādhu, sāriputta, sādhu kho tvaṃ, sāriputta, bhikkhūnaṃ saṅgītipariyāyaṃ abhāsī’ti. Idamavocāyasmā sāriputto, samanuñño satthā ahosi. Attamanā te bhikkhū āyasmato sāriputtassa bhāsitaṃ abhinandunti.

\xsectionEnd{Saṅgītisuttaṃ niṭṭhitaṃ dasamaṃ.}


\section{Dasuttarasuttaṃ}

\paragraph{350.} Evaṃ me sutaṃ – ekaṃ samayaṃ bhagavā campāyaṃ viharati gaggarāya pokkharaṇiyā tīre mahatā bhikkhusaṅghena saddhiṃ pañcamattehi bhikkhusatehi. Tatra kho āyasmā sāriputto bhikkhū āmantesi – ‘‘āvuso bhikkhave’’ti! ‘‘Āvuso’’ti kho te bhikkhū āyasmato sāriputtassa paccassosuṃ. Āyasmā sāriputto etadavoca –

‘‘Dasuttaraṃ pavakkhāmi, dhammaṃ nibbānapattiyā;

Dukkhassantakiriyāya, sabbaganthappamocanaṃ’’.

\subsubsection{Eko dhammo}

\paragraph{351.} ‘‘Eko, āvuso, dhammo bahukāro, eko dhammo bhāvetabbo, eko dhammo pariññeyyo, eko dhammo pahātabbo, eko dhammo hānabhāgiyo, eko dhammo visesabhāgiyo, eko dhammo duppaṭivijjho, eko dhammo uppādetabbo, eko dhammo abhiññeyyo, eko dhammo sacchikātabbo.

(Ka) ‘‘katamo eko dhammo bahukāro? Appamādo kusalesu dhammesu. Ayaṃ eko dhammo bahukāro.

(Kha) ‘‘katamo eko dhammo bhāvetabbo? Kāyagatāsati sātasahagatā. Ayaṃ eko dhammo bhāvetabbo.

(Ga) ‘‘katamo eko dhammo pariññeyyo? Phasso sāsavo upādāniyo. Ayaṃ eko dhammo pariññeyyo.

(Gha) ‘‘katamo eko dhammo pahātabbo? Asmimāno. Ayaṃ eko dhammo pahātabbo.

(Ṅa) ‘‘katamo eko dhammo hānabhāgiyo? Ayoniso manasikāro. Ayaṃ eko dhammo hānabhāgiyo.

(Ca) ‘‘katamo eko dhammo visesabhāgiyo? Yoniso manasikāro. Ayaṃ eko dhammo visesabhāgiyo.

(Cha) ‘‘katamo eko dhammo duppaṭivijjho? Ānantariko cetosamādhi. Ayaṃ eko dhammo duppaṭivijjho.

(Ja) ‘‘katamo eko dhammo uppādetabbo? Akuppaṃ ñāṇaṃ. Ayaṃ eko dhammo uppādetabbo.

(Jha) ‘‘katamo eko dhammo abhiññeyyo? Sabbe sattā āhāraṭṭhitikā. Ayaṃ eko dhammo abhiññeyyo.

(Ña) ‘‘katamo eko dhammo sacchikātabbo? Akuppā cetovimutti. Ayaṃ eko dhammo sacchikātabbo.

‘‘Iti ime dasa dhammā bhūtā tacchā tathā avitathā anaññathā sammā tathāgatena abhisambuddhā.

\subsubsection{Dve dhammā}

\paragraph{352.} ‘‘Dve dhammā bahukārā, dve dhammā bhāvetabbā, dve dhammā pariññeyyā, dve dhammā pahātabbā , dve dhammā hānabhāgiyā, dve dhammā visesabhāgiyā, dve dhammā duppaṭivijjhā, dve dhammā uppādetabbā, dve dhammā abhiññeyyā, dve dhammā sacchikātabbā.

(Ka) ‘‘katame dve dhammā bahukārā? Sati ca sampajaññañca. Ime dve dhammā bahukārā.

(Kha) ‘‘katame dve dhammā bhāvetabbā? Samatho ca vipassanā ca. Ime dve dhammā bhāvetabbā.

(Ga) ‘‘katame dve dhammā pariññeyyā? Nāmañca rūpañca. Ime dve dhammā pariññeyyā.

(Gha) ‘‘katame dve dhammā pahātabbā? Avijjā ca bhavataṇhā ca. Ime dve dhammā pahātabbā.

(Ṅa) ‘‘katame dve dhammā hānabhāgiyā? Dovacassatā ca pāpamittatā ca. Ime dve dhammā hānabhāgiyā.

(Ca) ‘‘katame dve dhammā visesabhāgiyā? Sovacassatā ca kalyāṇamittatā ca. Ime dve dhammā visesabhāgiyā.

(Cha) ‘‘katame dve dhammā duppaṭivijjhā? Yo ca hetu yo ca paccayo sattānaṃ saṃkilesāya, yo ca hetu yo ca paccayo sattānaṃ visuddhiyā. Ime dve dhammā duppaṭivijjhā.

(Ja) ‘‘katame dve dhammā uppādetabbā? Dve ñāṇāni – khaye ñāṇaṃ, anuppāde ñāṇaṃ. Ime dve dhammā uppādetabbā.

(Jha) ‘‘katame dve dhammā abhiññeyyā? Dve dhātuyo – saṅkhatā ca dhātu asaṅkhatā ca dhātu. Ime dve dhammā abhiññeyyā.

(Ña) ‘‘katame dve dhammā sacchikātabbā? Vijjā ca vimutti ca. Ime dve dhammā sacchikātabbā.

‘‘Iti ime vīsati dhammā bhūtā tacchā tathā avitathā anaññathā sammā tathāgatena abhisambuddhā.

\subsubsection{Tayo dhammā}

\paragraph{353.} ‘‘Tayo dhammā bahukārā, tayo dhammā bhāvetabbā…pe… tayo dhammā sacchikātabbā.

(Ka) ‘‘katame tayo dhammā bahukārā? Sappurisasaṃsevo, saddhammassavanaṃ, dhammānudhammappaṭipatti. Ime tayo dhammā bahukārā.

(Kha) ‘‘katame tayo dhammā bhāvetabbā? Tayo samādhī – savitakko savicāro samādhi, avitakko vicāramatto samādhi, avitakko avicāro samādhi. Ime tayo dhammā bhāvetabbā.

(Ga) ‘‘katame tayo dhammā pariññeyyā? Tisso vedanā – sukhā vedanā, dukkhā vedanā, adukkhamasukhā vedanā. Ime tayo dhammā pariññeyyā.

(Gha) ‘‘katame tayo dhammā pahātabbā? Tisso taṇhā – kāmataṇhā, bhavataṇhā, vibhavataṇhā. Ime tayo dhammā pahātabbā.

(Ṅa) ‘‘katame tayo dhammā hānabhāgiyā? Tīṇi akusalamūlāni – lobho akusalamūlaṃ, doso akusalamūlaṃ, moho akusalamūlaṃ. Ime tayo dhammā hānabhāgiyā.

(Ca) ‘‘katame tayo dhammā visesabhāgiyā? Tīṇi kusalamūlāni – alobho kusalamūlaṃ, adoso kusalamūlaṃ, amoho kusalamūlaṃ. Ime tayo dhammā visesabhāgiyā.

(Cha) ‘‘katame tayo dhammā duppaṭivijjhā? Tisso nissaraṇiyā dhātuyo – kāmānametaṃ nissaraṇaṃ yadidaṃ nekkhammaṃ, rūpānametaṃ nissaraṇaṃ yadidaṃ arūpaṃ, yaṃ kho pana kiñci bhūtaṃ saṅkhataṃ paṭiccasamuppannaṃ, nirodho tassa nissaraṇaṃ. Ime tayo dhammā duppaṭivijjhā.

(Ja) ‘‘katame tayo dhammā uppādetabbā? Tīṇi ñāṇāni – atītaṃse ñāṇaṃ, anāgataṃse ñāṇaṃ, paccuppannaṃse ñāṇaṃ. Ime tayo dhammā uppādetabbā.

(Jha) ‘‘katame tayo dhammā abhiññeyyā? Tisso dhātuyo – kāmadhātu, rūpadhātu, arūpadhātu. Ime tayo dhammā abhiññeyyā.

(Ña) ‘‘katame tayo dhammā sacchikātabbā? Tisso vijjā – pubbenivāsānussatiñāṇaṃ vijjā, sattānaṃ cutūpapāte ñāṇaṃ vijjā, āsavānaṃ khaye ñāṇaṃ vijjā. Ime tayo dhammā sacchikātabbā.

‘‘Iti ime tiṃsa dhammā bhūtā tacchā tathā avitathā anaññathā sammā tathāgatena abhisambuddhā.

\subsubsection{Cattāro dhammā}

\paragraph{354.} ‘‘Cattāro dhammā bahukārā, cattāro dhammā bhāvetabbā…pe… cattāro dhammā sacchikātabbā.

(Ka) ‘‘katame cattāro dhammā bahukārā? Cattāri cakkāni – patirūpadesavāso, sappurisūpanissayo\footnote{sappurisupassayo (syā. kaṃ.)}, attasammāpaṇidhi, pubbe ca katapuññatā. Ime cattāro dhammā bahukārā.

(Kha) ‘‘katame cattāro dhammā bhāvetabbā? Cattāro satipaṭṭhānā – idhāvuso, bhikkhu kāye kāyānupassī viharati ātāpī sampajāno satimā vineyya loke abhijjhādomanassaṃ. Vedanāsu…pe… citte… dhammesu dhammānupassī viharati ātāpī sampajāno satimā vineyya loke abhijjhādomanassaṃ. Ime cattāro dhammā bhāvetabbā.

(Ga) ‘‘katame cattāro dhammā pariññeyyā? Cattāro āhārā – kabaḷīkāro\footnote{kavaḷīkāro (syā. kaṃ.)} āhāro oḷāriko vā sukhumo vā, phasso dutiyo, manosañcetanā tatiyā, viññāṇaṃ catutthaṃ. Ime cattāro dhammā pariññeyyā.

(Gha) ‘‘katame cattāro dhammā pahātabbā? Cattāro oghā – kāmogho, bhavogho, diṭṭhogho, avijjogho. Ime cattāro dhammā pahātabbā.

(Ṅa) ‘‘katame cattāro dhammā hānabhāgiyā? Cattāro yogā – kāmayogo, bhavayogo, diṭṭhiyogo, avijjāyogo. Ime cattāro dhammā hānabhāgiyā.

(Ca) ‘‘katame cattāro dhammā visesabhāgiyā? Cattāro visaññogā – kāmayogavisaṃyogo, bhavayogavisaṃyogo, diṭṭhiyogavisaṃyogo, avijjāyogavisaṃyogo. Ime cattāro dhammā visesabhāgiyā.

(Cha) ‘‘katame cattāro dhammā duppaṭivijjhā? Cattāro samādhī – hānabhāgiyo samādhi, ṭhitibhāgiyo samādhi, visesabhāgiyo samādhi, nibbedhabhāgiyo samādhi. Ime cattāro dhammā duppaṭivijjhā.

(Ja) ‘‘katame cattāro dhammā uppādetabbā? Cattāri ñāṇāni – dhamme ñāṇaṃ, anvaye ñāṇaṃ, pariye ñāṇaṃ, sammutiyā ñāṇaṃ. Ime cattāro dhammā uppādetabbā.

(Jha) ‘‘katame cattāro dhammā abhiññeyyā? Cattāri ariyasaccāni – dukkhaṃ ariyasaccaṃ, dukkhasamudayaṃ\footnote{dukkhasamudayo (syā. kaṃ.)} ariyasaccaṃ, dukkhanirodhaṃ\footnote{dukkhanirodho (syā. kaṃ.)} ariyasaccaṃ, dukkhanirodhagāminī paṭipadā ariyasaccaṃ. Ime cattāro dhammā abhiññeyyā.

(Ña) ‘‘katame cattāro dhammā sacchikātabbā? Cattāri sāmaññaphalāni – sotāpattiphalaṃ, sakadāgāmiphalaṃ, anāgāmiphalaṃ, arahattaphalaṃ . Ime cattāro dhammā sacchikātabbā.

‘‘Iti ime cattārīsadhammā bhūtā tacchā tathā avitathā anaññathā sammā tathāgatena abhisambuddhā.

\subsubsection{Pañca dhammā}

\paragraph{355.} ‘‘Pañca dhammā bahukārā…pe… pañca dhammā sacchikātabbā.

(Ka) ‘‘katame pañca dhammā bahukārā? Pañca padhāniyaṅgāni – idhāvuso, bhikkhu saddho hoti, saddahati tathāgatassa bodhiṃ – ‘itipi so bhagavā arahaṃ sammāsambuddho vijjācaraṇasampanno sugato lokavidū anuttaro purisadammasārathi satthā devamanussānaṃ buddho bhagavā’ti. Appābādho hoti appātaṅko samavepākiniyā gahaṇiyā samannāgato nātisītāya nāccuṇhāya majjhimāya padhānakkhamāya. Asaṭho hoti amāyāvī yathābhūtamattānaṃ āvīkattā satthari vā viññūsu vā sabrahmacārīsu. Āraddhavīriyo viharati akusalānaṃ dhammānaṃ pahānāya, kusalānaṃ dhammānaṃ upasampadāya, thāmavā daḷhaparakkamo anikkhittadhuro kusalesu dhammesu. Paññavā hoti udayatthagāminiyā paññāya samannāgato ariyāya nibbedhikāya sammā dukkhakkhayagāminiyā. Ime pañca dhammā bahukārā.

(Kha) ‘‘katame pañca dhammā bhāvetabbā? Pañcaṅgiko sammāsamādhi – pītipharaṇatā, sukhapharaṇatā, cetopharaṇatā , ālokapharaṇatā, paccavekkhaṇanimittaṃ\footnote{paccavekkhaṇānimittaṃ (syā. kaṃ.)}. Ime pañca dhammā bhāvetabbā.

(Ga) ‘‘katame pañca dhammā pariññeyyā? Pañcupādānakkhandhā\footnote{seyyathīdaṃ (sī. syā. kaṃ. pī.)} – rūpupādānakkhandho, vedanupādānakkhandho, saññupādānakkhandho, saṅkhārupādānakkhandho viññāṇupādānakkhandho. Ime pañca dhammā pariññeyyā.

(Gha) ‘‘katame pañca dhammā pahātabbā? Pañca nīvaraṇāni – kāmacchandanīvaraṇaṃ, byāpādanīvaraṇaṃ, thinamiddhanīvaraṇaṃ, uddhaccakukuccanīvaraṇaṃ, vicikicchānīvaraṇaṃ. Ime pañca dhammā pahātabbā.

(Ṅa) ‘‘katame pañca dhammā hānabhāgiyā? Pañca cetokhilā – idhāvuso, bhikkhu satthari kaṅkhati vicikicchati nādhimuccati na sampasīdati. Yo so, āvuso, bhikkhu satthari kaṅkhati vicikicchati nādhimuccati na sampasīdati, tassa cittaṃ na namati ātappāya anuyogāya sātaccāya padhānāya. Yassa cittaṃ na namati ātappāya anuyogāya sātaccāya padhānāya . Ayaṃ paṭhamo cetokhilo. Puna caparaṃ, āvuso, bhikkhu dhamme kaṅkhati vicikicchati…pe… saṅghe kaṅkhati vicikicchati…pe… sikkhāya kaṅkhati vicikicchati…pe… sabrahmacārīsu kupito hoti anattamano āhatacitto khilajāto, yo so, āvuso, bhikkhu sabrahmacārīsu kupito hoti anattamano āhatacitto khilajāto, tassa cittaṃ na namati ātappāya anuyogāya sātaccāya padhānāya. Yassa cittaṃ na namati ātappāya anuyogāya sātaccāya padhānāya. Ayaṃ pañcamo cetokhilo. Ime pañca dhammā hānabhāgiyā.

(Ca) ‘‘katame pañca dhammā visesabhāgiyā? Pañcindriyāni – saddhindriyaṃ, vīriyindriyaṃ, satindriyaṃ, samādhindriyaṃ, paññindriyaṃ. Ime pañca dhammā visesabhāgiyā.

(Cha) ‘‘katame pañca dhammā duppaṭivijjhā? Pañca nissaraṇiyā dhātuyo – idhāvuso, bhikkhuno kāme manasikaroto kāmesu cittaṃ na pakkhandati na pasīdati na santiṭṭhati na vimuccati. Nekkhammaṃ kho panassa manasikaroto nekkhamme cittaṃ pakkhandati pasīdati santiṭṭhati vimuccati. Tassa taṃ cittaṃ sugataṃ subhāvitaṃ suvuṭṭhitaṃ suvimuttaṃ visaṃyuttaṃ kāmehi. Ye ca kāmapaccayā uppajjanti āsavā vighātā pariḷāhā, mutto so tehi. Na so taṃ vedanaṃ vedeti. Idamakkhātaṃ kāmānaṃ nissaraṇaṃ.

‘‘Puna caparaṃ, āvuso, bhikkhuno byāpādaṃ manasikaroto byāpāde cittaṃ na pakkhandati na pasīdati na santiṭṭhati na vimuccati. Abyāpādaṃ kho panassa manasikaroto abyāpāde cittaṃ pakkhandati pasīdati santiṭṭhati vimuccati. Tassa taṃ cittaṃ sugataṃ subhāvitaṃ suvuṭṭhitaṃ suvimuttaṃ visaṃyuttaṃ byāpādena. Ye ca byāpādapaccayā uppajjanti āsavā vighātā pariḷāhā, mutto so tehi. Na so taṃ vedanaṃ vedeti. Idamakkhātaṃ byāpādassa nissaraṇaṃ.

‘‘Puna caparaṃ, āvuso, bhikkhuno vihesaṃ manasikaroto vihesāya cittaṃ na pakkhandati na pasīdati na santiṭṭhati na vimuccati. Avihesaṃ kho panassa manasikaroto avihesāya cittaṃ pakkhandati pasīdati santiṭṭhati vimuccati . Tassa taṃ cittaṃ sugataṃ subhāvitaṃ suvuṭṭhitaṃ suvimuttaṃ visaṃyuttaṃ vihesāya. Ye ca vihesāpaccayā uppajjanti āsavā vighātā pariḷāhā, mutto so tehi. Na so taṃ vedanaṃ vedeti. Idamakkhātaṃ vihesāya nissaraṇaṃ.

‘‘Puna caparaṃ, āvuso, bhikkhuno rūpe manasikaroto rūpesu cittaṃ na pakkhandati na pasīdati na santiṭṭhati na vimuccati. Arūpaṃ kho panassa manasikaroto arūpe cittaṃ pakkhandati pasīdati santiṭṭhati vimuccati. Tassa taṃ cittaṃ sugataṃ subhāvitaṃ suvuṭṭhitaṃ suvimuttaṃ visaṃyuttaṃ rūpehi. Ye ca rūpapaccayā uppajjanti āsavā vighātā pariḷāhā, mutto so tehi. Na so taṃ vedanaṃ vedeti. Idamakkhātaṃ rūpānaṃ nissaraṇaṃ.

‘‘Puna caparaṃ, āvuso, bhikkhuno sakkāyaṃ manasikaroto sakkāye cittaṃ na pakkhandati na pasīdati na santiṭṭhati na vimuccati. Sakkāyanirodhaṃ kho panassa manasikaroto sakkāyanirodhe cittaṃ pakkhandati pasīdati santiṭṭhati vimuccati. Tassa taṃ cittaṃ sugataṃ subhāvitaṃ suvuṭṭhitaṃ suvimuttaṃ visaṃyuttaṃ sakkāyena. Ye ca sakkāyapaccayā uppajjanti āsavā vighātā pariḷāhā, mutto so tehi. Na so taṃ vedanaṃ vedeti. Idamakkhātaṃ sakkāyassa nissaraṇaṃ. Ime pañca dhammā duppaṭivijjhā.

(Ja) ‘‘katame pañca dhammā uppādetabbā? Pañca ñāṇiko sammāsamādhi – ‘ayaṃ samādhi paccuppannasukho ceva āyatiñca sukhavipāko’ti paccattaṃyeva ñāṇaṃ uppajjati. ‘Ayaṃ samādhi ariyo nirāmiso’ti paccattaññeva ñāṇaṃ uppajjati. ‘Ayaṃ samādhi akāpurisasevito’ti paccattaṃyeva ñāṇaṃ uppajjati. ‘Ayaṃ samādhi santo paṇīto paṭippassaddhaladdho ekodibhāvādhigato, na sasaṅkhāraniggayhavāritagato’ti\footnote{na ca sasaṅkhāraniggayha vāritavatoti (sī. syā. kaṃ. pī.), na sasaṅkhāraniggayhavārivāvato (ka.), na sasaṅkhāraniggayhavāriyādhigato (?)} paccattaṃyeva ñāṇaṃ uppajjati. ‘So kho panāhaṃ imaṃ samādhiṃ satova samāpajjāmi sato vuṭṭhahāmī’ti paccattaṃyeva ñāṇaṃ uppajjati. Ime pañca dhammā uppādetabbā.

(Jha) ‘‘katame pañca dhammā abhiññeyyā? Pañca vimuttāyatanāni – idhāvuso, bhikkhuno satthā dhammaṃ deseti aññataro vā garuṭṭhāniyo sabrahmacārī. Yathā yathā, āvuso, bhikkhuno satthā dhammaṃ deseti, aññataro vā garuṭṭhāniyo sabrahmacārī, tathā tathā so\footnote{bhikkhu (syā. kaṃ.)} tasmiṃ dhamme atthappaṭisaṃvedī ca hoti dhammapaṭisaṃvedī ca. Tassa atthappaṭisaṃvedino dhammapaṭisaṃvedino pāmojjaṃ jāyati, pamuditassa pīti jāyati, pītimanassa kāyo passambhati, passaddhakāyo sukhaṃ vedeti, sukhino cittaṃ samādhiyati. Idaṃ paṭhamaṃ vimuttāyatanaṃ.

‘‘Puna caparaṃ, āvuso, bhikkhuno na heva kho satthā dhammaṃ deseti, aññataro vā garuṭṭhāniyo sabrahmacārī, api ca kho yathāsutaṃ yathāpariyattaṃ dhammaṃ vitthārena paresaṃ deseti yathā yathā, āvuso, bhikkhu yathāsutaṃ yathāpariyattaṃ dhammaṃ vitthārena paresaṃ deseti. Tathā tathā so tasmiṃ dhamme atthappaṭisaṃvedī ca hoti dhammapaṭisaṃvedī ca. Tassa atthappaṭisaṃvedino dhammapaṭisaṃvedino pāmojjaṃ jāyati, pamuditassa pīti jāyati, pītimanassa kāyo passambhati, passaddhakāyo sukhaṃ vedeti, sukhino cittaṃ samādhiyati. Idaṃ dutiyaṃ vimuttāyatanaṃ.

‘‘Puna caparaṃ, āvuso, bhikkhuno na heva kho satthā dhammaṃ deseti, aññataro vā garuṭṭhāniyo sabrahmacārī, nāpi yathāsutaṃ yathāpariyattaṃ dhammaṃ vitthārena paresaṃ deseti. Api ca kho, yathāsutaṃ yathāpariyattaṃ dhammaṃ vitthārena sajjhāyaṃ karoti. Yathā yathā, āvuso, bhikkhu yathāsutaṃ yathāpariyattaṃ dhammaṃ vitthārena sajjhāyaṃ karoti tathā tathā so tasmiṃ dhamme atthappaṭisaṃvedī ca hoti dhammapaṭisaṃvedī ca. Tassa atthappaṭisaṃvedino dhammapaṭisaṃvedino pāmojjaṃ jāyati, pamuditassa pīti jāyati, pītimanassa kāyo passambhati, passaddhakāyo sukhaṃ vedeti, sukhino cittaṃ samādhiyati. Idaṃ tatiyaṃ vimuttāyatanaṃ.

‘‘Puna caparaṃ, āvuso, bhikkhuno na heva kho satthā dhammaṃ deseti, aññataro vā garuṭṭhāniyo sabrahmacārī, nāpi yathāsutaṃ yathāpariyattaṃ dhammaṃ vitthārena paresaṃ deseti, nāpi yathāsutaṃ yathāpariyattaṃ dhammaṃ vitthārena sajjhāyaṃ karoti. Api ca kho, yathāsutaṃ yathāpariyattaṃ dhammaṃ cetasā anuvitakketi anuvicāreti manasānupekkhati. Yathā yathā , āvuso , bhikkhu yathāsutaṃ yathāpariyattaṃ dhammaṃ cetasā anuvitakketi anuvicāreti manasānupekkhati tathā tathā so tasmiṃ dhamme atthappaṭisaṃvedī ca hoti dhammapaṭisaṃvedī ca. Tassa atthappaṭisaṃvedino dhammapaṭisaṃvedino pāmojjaṃ jāyati, pamuditassa pīti jāyati, pītimanassa kāyo passambhati, passaddhakāyo sukhaṃ vedeti, sukhino cittaṃ samādhiyati. Idaṃ catutthaṃ vimuttāyatanaṃ.

‘‘Puna caparaṃ, āvuso, bhikkhuno na heva kho satthā dhammaṃ deseti, aññataro vā garuṭṭhāniyo sabrahmacārī, nāpi yathāsutaṃ yathāpariyattaṃ dhammaṃ vitthārena paresaṃ deseti, nāpi yathāsutaṃ yathāpariyattaṃ dhammaṃ vitthārena sajjhāyaṃ karoti, nāpi yathāsutaṃ yathāpariyattaṃ dhammaṃ cetasā anuvitakketi anuvicāreti manasānupekkhati; api ca khvassa aññataraṃ samādhinimittaṃ suggahitaṃ hoti sumanasikataṃ sūpadhāritaṃ suppaṭividdhaṃ paññāya. Yathā yathā, āvuso, bhikkhuno aññataraṃ samādhinimittaṃ suggahitaṃ hoti sumanasikataṃ sūpadhāritaṃ suppaṭividdhaṃ paññāya tathā tathā so tasmiṃ dhamme atthappaṭisaṃvedī ca hoti dhammappaṭisaṃvedī ca. Tassa atthappaṭisaṃvedino dhammappaṭisaṃvedino pāmojjaṃ jāyati, pamuditassa pīti jāyati, pītimanassa kāyo passambhati, passaddhakāyo sukhaṃ vedeti, sukhino cittaṃ samādhiyati. Idaṃ pañcamaṃ vimuttāyatanaṃ. Ime pañca dhammā abhiññeyyā.

(Ña) ‘‘katame pañca dhammā sacchikātabbā? Pañca dhammakkhandhā – sīlakkhandho , samādhikkhandho, paññākkhandho, vimuttikkhandho, vimuttiñāṇadassanakkhandho. Ime pañca dhammā sacchikātabbā.

‘‘Iti ime paññāsa dhammā bhūtā tacchā tathā avitathā anaññathā sammā tathāgatena abhisambuddhā.

\subsubsection{Cha dhammā}

\paragraph{356.} ‘‘Cha dhammā bahukārā…pe… cha dhammā sacchikātabbā.

(Ka) ‘‘katame cha dhammā bahukārā? Cha sāraṇīyā dhammā. Idhāvuso, bhikkhuno mettaṃ kāyakammaṃ paccupaṭṭhitaṃ hoti sabrahmacārīsu āvi ceva raho ca, ayampi dhammo sāraṇīyo piyakaraṇo garukaraṇo saṅgahāya avivādāya sāmaggiyā ekībhāvāya saṃvattati.

‘‘Puna caparaṃ, āvuso, bhikkhuno mettaṃ vacīkammaṃ…pe… ekībhāvāya saṃvattati.

‘‘Puna caparaṃ, āvuso, bhikkhuno mettaṃ manokammaṃ…pe… ekībhāvāya saṃvattati.

‘‘Puna caparaṃ, āvuso, bhikkhu ye te lābhā dhammikā dhammaladdhā antamaso pattapariyāpannamattampi, tathārūpehi lābhehi appaṭivibhattabhogī hoti sīlavantehi sabrahmacārīhi sādhāraṇabhogī, ayampi dhammo sāraṇīyo…pe… ekībhāvāya saṃvattati.

‘‘Puna caparaṃ, āvuso, bhikkhu, yāni tāni sīlāni akhaṇḍāni acchiddāni asabalāni akammāsāni bhujissāni viññuppasatthāni aparāmaṭṭhāni samādhisaṃvattanikāni, tathārūpesu sīlesu sīlasāmaññagato viharati sabrahmacārīhi āvi ceva raho ca, ayampi dhammo sāraṇīyo…pe… ekībhāvāya saṃvattati.

‘‘Puna caparaṃ, āvuso, bhikkhu yāyaṃ diṭṭhi ariyā niyyānikā niyyāti takkarassa sammā dukkhakkhayāya, tathārūpāya diṭṭhiyā diṭṭhi sāmaññagato viharati sabrahmacārīhi āvi ceva raho ca, ayampi dhammo sāraṇīyo piyakaraṇo garukaraṇo, saṅgahāya avivādāya sāmaggiyā ekībhāvāya saṃvattati. Ime cha dhammā bahukārā.

(Kha) ‘‘katame cha dhammā bhāvetabbā? Cha anussatiṭṭhānāni – buddhānussati, dhammānussati, saṅghānussati, sīlānussati, cāgānussati, devatānussati. Ime cha dhammā bhāvetabbā.

(Ga) ‘‘katame cha dhammā pariññeyyā? Cha ajjhattikāni āyatanāni – cakkhāyatanaṃ, sotāyatanaṃ, ghānāyatanaṃ, jivhāyatanaṃ, kāyāyatanaṃ, manāyatanaṃ. Ime cha dhammā pariññeyyā.

(Gha) ‘‘katame cha dhammā pahātabbā? Cha taṇhākāyā – rūpataṇhā, saddataṇhā, gandhataṇhā, rasataṇhā, phoṭṭhabbataṇhā, dhammataṇhā. Ime cha dhammā pahātabbā.

(Ṅa) ‘‘katame cha dhammā hānabhāgiyā? Cha agāravā – idhāvuso, bhikkhu satthari agāravo viharati appatisso. Dhamme…pe… saṅghe… sikkhāya… appamāde… paṭisanthāre agāravo viharati appatisso. Ime cha dhammā hānabhāgiyā.

(Ca) ‘‘katame cha dhammā visesabhāgiyā? Cha gāravā – idhāvuso, bhikkhu satthari sagāravo viharati sappatisso dhamme…pe… saṅghe… sikkhāya… appamāde… paṭisanthāre sagāravo viharati sappatisso. Ime cha dhammā visesabhāgiyā.

(Cha) ‘‘katame cha dhammā duppaṭivijjhā? Cha nissaraṇiyā dhātuyo – idhāvuso, bhikkhu evaṃ vadeyya – ‘mettā hi kho me, cetovimutti bhāvitā bahulīkatā yānīkatā vatthukatā anuṭṭhitā paricitā susamāraddhā, atha ca pana me byāpādo cittaṃ pariyādāya tiṭṭhatī’ti. So ‘mā hevaṃ’ tissa vacanīyo ‘māyasmā evaṃ avaca, mā bhagavantaṃ abbhācikkhi. Na hi sādhu bhagavato abbhakkhānaṃ, na hi bhagavā evaṃ vadeyya. Aṭṭhānametaṃ āvuso anavakāso yaṃ mettāya cetovimuttiyā bhāvitāya bahulīkatāya yānīkatāya vatthukatāya anuṭṭhitāya paricitāya susamāraddhāya. Atha ca panassa byāpādo cittaṃ pariyādāya ṭhassatīti, netaṃ ṭhānaṃ vijjati. Nissaraṇaṃ hetaṃ, āvuso, byāpādassa, yadidaṃ mettācetovimuttī’ti.

‘‘Idha panāvuso, bhikkhu evaṃ vadeyya – ‘karuṇā hi kho me cetovimutti bhāvitā bahulīkatā yānīkatā vatthukatā anuṭṭhitā paricitā susamāraddhā. Atha ca pana me vihesā cittaṃ pariyādāya tiṭṭhatī’ti. So – ‘mā hevaṃ’ tissa vacanīyo, ‘māyasmā evaṃ avaca, mā bhagavantaṃ abbhācikkhi…pe… nissaraṇaṃ hetaṃ, āvuso, vihesāya, yadidaṃ karuṇācetovimuttī’ti.

‘‘Idha panāvuso, bhikkhu evaṃ vadeyya – ‘muditā hi kho me cetovimutti bhāvitā…pe… atha ca pana me arati cittaṃ pariyādāya tiṭṭhatī’ti. So – ‘mā hevaṃ’ tissa vacanīyo ‘māyasmā evaṃ avaca…pe… nissaraṇaṃ hetaṃ, āvuso aratiyā, yadidaṃ muditācetovimuttī’ti.

‘‘Idha panāvuso, bhikkhu evaṃ vadeyya – ‘upekkhā hi kho me cetovimutti bhāvitā…pe… atha ca pana me rāgo cittaṃ pariyādāya tiṭṭhatī’ti. So – ‘mā hevaṃ’ tissa vacanīyo ‘māyasmā evaṃ avaca…pe… nissaraṇaṃ hetaṃ, āvuso, rāgassa yadidaṃ upekkhācetovimuttī’ti.

‘‘Idha panāvuso, bhikkhu evaṃ vadeyya – ‘animittā hi kho me cetovimutti bhāvitā…pe… atha ca pana me nimittānusāri viññāṇaṃ hotī’ti. So – ‘mā hevaṃ’ tissa vacanīyo ‘māyasmā evaṃ avaca…pe… nissaraṇaṃ hetaṃ, āvuso, sabbanimittānaṃ yadidaṃ animittā cetovimuttī’ti.

‘‘Idha panāvuso, bhikkhu evaṃ vadeyya – ‘asmīti kho me vigataṃ, ayamahamasmīti na samanupassāmi, atha ca pana me vicikicchākathaṃkathāsallaṃ cittaṃ pariyādāya tiṭṭhatī’ti. So – ‘mā hevaṃ’ tissa vacanīyo ‘māyasmā evaṃ avaca, mā bhagavantaṃ abbhācikkhi, na hi sādhu bhagavato abbhakkhānaṃ, na hi bhagavā evaṃ vadeyya. Aṭṭhānametaṃ, āvuso, anavakāso yaṃ asmīti vigate ayamahamasmīti asamanupassato. Atha ca panassa vicikicchākathaṃkathāsallaṃ cittaṃ pariyādāya ṭhassati, netaṃ ṭhānaṃ vijjati. Nissaraṇaṃ hetaṃ, āvuso, vicikicchākathaṃkathāsallassa, yadidaṃ asmimānasamugghāṭo’ti. Ime cha dhammā duppaṭivijjhā.

(Ja) ‘‘katame cha dhammā uppādetabbā? Cha satatavihārā. Idhāvuso, bhikkhu cakkhunā rūpaṃ disvā neva sumano hoti na dummano, upekkhako viharati sato sampajāno. Sotena saddaṃ sutvā…pe… ghānena gandhaṃ ghāyitvā… jivhāya rasaṃ sāyitvā… kāyena phoṭṭhabbaṃ phusitvā… manasā dhammaṃ viññāya neva sumano hoti na dummano, upekkhako viharati sato sampajāno. Ime cha dhammā uppādetabbā.

(Jha) ‘‘katame cha dhammā abhiññeyyā? Cha anuttariyāni – dassanānuttariyaṃ, savanānuttariyaṃ, lābhānuttariyaṃ, sikkhānuttariyaṃ, pāricariyānuttariyaṃ, anussatānuttariyaṃ. Ime cha dhammā abhiññeyyā.

(Ña) ‘‘katame cha dhammā sacchikātabbā? Cha abhiññā – idhāvuso, bhikkhu anekavihitaṃ iddhividhaṃ paccanubhoti – ekopi hutvā bahudhā hoti , bahudhāpi hutvā eko hoti. Āvibhāvaṃ tirobhāvaṃ. Tirokuṭṭaṃ tiropākāraṃ tiropabbataṃ asajjamāno gacchati seyyathāpi ākāse . Pathaviyāpi ummujjanimujjaṃ karoti seyyathāpi udake. Udakepi abhijjamāne gacchati seyyathāpi pathaviyaṃ. Ākāsepi pallaṅkena kamati seyyathāpi pakkhī sakuṇo. Imepi candimasūriye evaṃmahiddhike evaṃmahānubhāve pāṇinā parāmasati parimajjati. Yāva brahmalokāpi kāyena vasaṃ vatteti.

‘‘Dibbāya sotadhātuyā visuddhāya atikkantamānusikāya ubho sadde suṇāti dibbe ca mānuse ca, ye dūre santike ca.

‘‘Parasattānaṃ parapuggalānaṃ cetasā ceto paricca pajānāti\footnote{jānāti (syā. kaṃ.)}, sarāgaṃ vā cittaṃ sarāgaṃ cittanti pajānāti …pe… avimuttaṃ vā cittaṃ avimuttaṃ cittanti pajānāti.

‘‘So anekavihitaṃ pubbenivāsaṃ anussarati, seyyathidaṃ ekampi jātiṃ…pe… iti sākāraṃ sauddesaṃ anekavihitaṃ pubbenivāsaṃ anussarati.

‘‘Dibbena cakkhunā visuddhena atikkantamānusakena satte passati cavamāne upapajjamāne hīne paṇīte suvaṇṇe dubbaṇṇe sugate duggate yathākammūpage satte pajānāti …pe…

‘‘Āsavānaṃ khayā anāsavaṃ cetovimuttiṃ paññāvimuttiṃ diṭṭheva dhamme sayaṃ abhiññā sacchikatvā upasampajja viharati. Ime cha dhammā sacchikātabbā.

‘‘Iti ime saṭṭhi dhammā bhūtā tacchā tathā avitathā anaññathā sammā tathāgatena abhisambuddhā.

\subsubsection{Satta dhammā}

\paragraph{357.} ‘‘Satta dhammā bahukārā…pe… satta dhammā sacchikātabbā.

(Ka) ‘‘katame satta dhammā bahukārā? Satta ariyadhanāni – saddhādhanaṃ, sīladhanaṃ, hiridhanaṃ, ottappadhanaṃ, sutadhanaṃ, cāgadhanaṃ, paññādhanaṃ. Ime satta dhammā bahukārā.

(Kha) ‘‘katame satta dhammā bhāvetabbā? Satta sambojjhaṅgā – satisambojjhaṅgo, dhammavicayasambojjhaṅgo, vīriyasambojjhaṅgo, pītisambojjhaṅgo, passaddhisambojjhaṅgo, samādhisambojjhaṅgo, upekkhāsambojjhaṅgo . Ime satta dhammā bhāvetabbā.

(Ga) ‘‘katame satta dhammā pariññeyyā? Satta viññāṇaṭṭhitiyo – santāvuso, sattā nānattakāyā nānattasaññino, seyyathāpi manussā ekacce ca devā ekacce ca vinipātikā. Ayaṃ paṭhamā viññāṇaṭṭhiti.

‘‘Santāvuso , sattā nānattakāyā ekattasaññino, seyyathāpi devā brahmakāyikā paṭhamābhinibbattā. Ayaṃ dutiyā viññāṇaṭṭhiti.

‘‘Santāvuso, sattā ekattakāyā nānattasaññino, seyyathāpi devā ābhassarā. Ayaṃ tatiyā viññāṇaṭṭhiti.

‘‘Santāvuso, sattā ekattakāyā ekattasaññino, seyyathāpi devā subhakiṇhā. Ayaṃ catutthī viññāṇaṭṭhiti.

‘‘Santāvuso, sattā sabbaso rūpasaññānaṃ samatikkamā…pe… ‘ananto ākāso’ti ākāsānañcāyatanūpagā. Ayaṃ pañcamī viññāṇaṭṭhiti.

‘‘Santāvuso, sattā sabbaso ākāsānañcāyatanaṃ samatikkamma ‘anantaṃ viññāṇa’nti viññāṇañcāyatanūpagā. Ayaṃ chaṭṭhī viññāṇaṭṭhiti.

‘‘Santāvuso, sattā sabbaso viññāṇañcāyatanaṃ samatikkamma ‘natthi kiñcī’ti ākiñcaññāyatanūpagā. Ayaṃ sattamī viññāṇaṭṭhiti. Ime satta dhammā pariññeyyā.

(Gha) ‘‘katame satta dhammā pahātabbā? Sattānusayā – kāmarāgānusayo, paṭighānusayo, diṭṭhānusayo, vicikicchānusayo, mānānusayo, bhavarāgānusayo , avijjānusayo. Ime satta dhammā pahātabbā.

(Ṅa) ‘‘katame satta dhammā hānabhāgiyā? Satta asaddhammā – idhāvuso, bhikkhu assaddho hoti, ahiriko hoti, anottappī hoti, appassuto hoti, kusīto hoti, muṭṭhassati hoti, duppañño hoti. Ime satta dhammā hānabhāgiyā.

(Ca) ‘‘katame satta dhammā visesabhāgiyā? Satta saddhammā – idhāvuso, bhikkhu saddho hoti, hirimā\footnote{hiriko (syā. kaṃ.)} hoti, ottappī hoti, bahussuto hoti, āraddhavīriyo hoti, upaṭṭhitassati hoti, paññavā hoti. Ime satta dhammā visesabhāgiyā.

(Cha) ‘‘katame satta dhammā duppaṭivijjhā? Satta sappurisadhammā – idhāvuso, bhikkhu dhammaññū ca hoti atthaññū ca attaññū ca mattaññū ca kālaññū ca parisaññū ca puggalaññū ca. Ime satta dhammā duppaṭivijjhā.

(Ja) ‘‘katame satta dhammā uppādetabbā? Satta saññā – aniccasaññā, anattasaññā, asubhasaññā, ādīnavasaññā, pahānasaññā, virāgasaññā, nirodhasaññā. Ime satta dhammā uppādetabbā.

(Jha) ‘‘katame satta dhammā abhiññeyyā? Satta niddasavatthūni – idhāvuso, bhikkhu sikkhāsamādāne tibbacchando hoti, āyatiñca sikkhāsamādāne avigatapemo. Dhammanisantiyā tibbacchando hoti, āyatiñca dhammanisantiyā avigatapemo. Icchāvinaye tibbacchando hoti, āyatiñca icchāvinaye avigatapemo. Paṭisallāne tibbacchando hoti, āyatiñca paṭisallāne avigatapemo. Vīriyāramme tibbacchando hoti, āyatiñca vīriyāramme avigatapemo. Satinepakke tibbacchando hoti, āyatiñca satinepakke avigatapemo. Diṭṭhipaṭivedhe tibbacchando hoti, āyatiñca diṭṭhipaṭivedhe avigatapemo. Ime satta dhammā abhiññeyyā.

(Ña) ‘‘katame satta dhammā sacchikātabbā? Satta khīṇāsavabalāni – idhāvuso, khīṇāsavassa bhikkhuno aniccato sabbe saṅkhārā yathābhūtaṃ sammappaññāya sudiṭṭhā honti. Yaṃpāvuso, khīṇāsavassa bhikkhuno aniccato sabbe saṅkhārā yathābhūtaṃ sammappaññāya sudiṭṭhā honti, idampi khīṇāsavassa bhikkhuno balaṃ hoti, yaṃ balaṃ āgamma khīṇāsavo bhikkhu āsavānaṃ khayaṃ paṭijānāti – ‘khīṇā me āsavā’ti.

‘‘Puna caparaṃ, āvuso, khīṇāsavassa bhikkhuno aṅgārakāsūpamā kāmā yathābhūtaṃ sammappaññāya sudiṭṭhā honti. Yaṃpāvuso…pe… ‘khīṇā me āsavā’ti.

‘‘Puna caparaṃ, āvuso, khīṇāsavassa bhikkhuno vivekaninnaṃ cittaṃ hoti vivekapoṇaṃ vivekapabbhāraṃ vivekaṭṭhaṃ nekkhammābhirataṃ byantībhūtaṃ sabbaso āsavaṭṭhāniyehi dhammehi. Yaṃpāvuso…pe… ‘khīṇā me āsavā’ti.

‘‘Puna caparaṃ, āvuso, khīṇāsavassa bhikkhuno cattāro satipaṭṭhānā bhāvitā honti subhāvitā . Yaṃpāvuso…pe… ‘khīṇā me āsavā’ti.

‘‘Puna caparaṃ, āvuso, khīṇāsavassa bhikkhuno pañcindriyāni bhāvitāni honti subhāvitāni. Yaṃpāvuso…pe… ‘khīṇā me āsavā’ti.

‘‘Puna caparaṃ, āvuso, khīṇāsavassa bhikkhuno satta bojjhaṅgā bhāvitā honti subhāvitā. Yaṃpāvuso…pe… ‘khīṇā me āsavā’ti.

‘‘Puna caparaṃ, āvuso, khīṇāsavassa bhikkhuno ariyo aṭṭhaṅgiko maggo bhāvito hoti subhāvito. Yaṃpāvuso, khīṇāsavassa bhikkhuno ariyo aṭṭhaṅgiko maggo bhāvito hoti subhāvito, idampi khīṇāsavassa bhikkhuno balaṃ hoti, yaṃ balaṃ āgamma khīṇāsavo bhikkhu āsavānaṃ khayaṃ paṭijānāti – ‘khīṇā me āsavā’ti. Ime satta dhammā sacchikātabbā.

‘‘Itime sattati dhammā bhūtā tacchā tathā avitathā anaññathā sammā tathāgatena abhisambuddhā.

Paṭhamabhāṇavāro niṭṭhito.

\subsubsection{Aṭṭha dhammā}

\paragraph{358.} ‘‘Aṭṭha dhammā bahukārā…pe… aṭṭha dhammā sacchikātabbā.

(Ka) ‘‘katame aṭṭha dhammā bahukārā? Aṭṭha hetū aṭṭha paccayā ādibrahmacariyikāya paññāya appaṭiladdhāya paṭilābhāya paṭiladdhāya bhiyyobhāvāya vepullāya bhāvanāya pāripūriyā saṃvattanti. Katame aṭṭha? Idhāvuso, bhikkhu satthāraṃ\footnote{satthāraṃ vā (syā. ka.)} upanissāya viharati aññataraṃ vā garuṭṭhāniyaṃ sabrahmacāriṃ, yatthassa tibbaṃ hirottappaṃ paccupaṭṭhitaṃ hoti pemañca gāravo ca. Ayaṃ paṭhamo hetu paṭhamo paccayo ādibrahmacariyikāya paññāya appaṭiladdhāya paṭilābhāya . Paṭiladdhāya bhiyyobhāvāya vepullāya bhāvanāya pāripūriyā saṃvattati.

‘‘Taṃ kho pana satthāraṃ upanissāya viharati aññataraṃ vā garuṭṭhāniyaṃ sabrahmacāriṃ , yatthassa tibbaṃ hirottappaṃ paccupaṭṭhitaṃ hoti pemañca gāravo ca. Te kālena kālaṃ upasaṅkamitvā paripucchati paripañhati – ‘idaṃ, bhante, kathaṃ? Imassa ko attho’ti? Tassa te āyasmanto avivaṭañceva vivaranti, anuttānīkatañca uttānī\footnote{anuttānikatañca uttāniṃ (ka.)} karonti, anekavihitesu ca kaṅkhāṭṭhāniyesu dhammesu kaṅkhaṃ paṭivinodenti. Ayaṃ dutiyo hetu dutiyo paccayo ādibrahmacariyikāya paññāya appaṭiladdhāya paṭilābhāya, paṭiladdhāya bhiyyobhāvāya, vepullāya bhāvanāya pāripūriyā saṃvattati.

‘‘Taṃ kho pana dhammaṃ sutvā dvayena vūpakāsena sampādeti – kāyavūpakāsena ca cittavūpakāsena ca. Ayaṃ tatiyo hetu tatiyo paccayo ādibrahmacariyikāya paññāya appaṭiladdhāya paṭilābhāya, paṭiladdhāya bhiyyobhāvāya vepullāya bhāvanāya pāripūriyā saṃvattati.

‘‘Puna caparaṃ, āvuso, bhikkhu sīlavā hoti, pātimokkhasaṃvarasaṃvuto viharati ācāragocarasampanno, aṇumattesu vajjesu bhayadassāvī samādāya sikkhati sikkhāpadesu. Ayaṃ catuttho hetu catuttho paccayo ādibrahmacariyikāya paññāya appaṭiladdhāya paṭilābhāya, paṭiladdhāya bhiyyobhāvāya vepullāya bhāvanāya pāripūriyā saṃvattati.

‘‘Puna caparaṃ, āvuso, bhikkhu bahussuto hoti sutadharo sutasannicayo. Ye te dhammā ādikalyāṇā majjhekalyāṇā pariyosānakalyāṇā sātthā sabyañjanā kevalaparipuṇṇaṃ parisuddhaṃ brahmacariyaṃ abhivadanti, tathārūpāssa dhammā bahussutā honti dhātā vacasā paricitā manasānupekkhitā diṭṭhiyā suppaṭividdhā. Ayaṃ pañcamo hetu pañcamo paccayo ādibrahmacariyikāya paññāya appaṭiladdhāya paṭilābhāya, paṭiladdhāya bhiyyobhāvāya vepullāya bhāvanāya pāripūriyā saṃvattati.

‘‘Puna caparaṃ, āvuso, bhikkhu āraddhavīriyo viharati akusalānaṃ dhammānaṃ pahānāya, kusalānaṃ dhammānaṃ upasampadāya, thāmavā daḷhaparakkamo anikkhittadhuro kusalesu dhammesu. Ayaṃ chaṭṭho hetu chaṭṭho paccayo ādibrahmacariyikāya paññāya appaṭiladdhāya paṭilābhāya, paṭiladdhāya bhiyyobhāvāya vepullāya bhāvanāya pāripūriyā saṃvattati.

‘‘Puna caparaṃ, āvuso, bhikkhu satimā hoti paramena satinepakkena samannāgato. Cirakatampi cirabhāsitampi saritā anussaritā. Ayaṃ sattamo hetu sattamo paccayo ādibrahmacariyikāya paññāya appaṭiladdhāya paṭilābhāya, paṭiladdhāya bhiyyobhāvāya vepullāya bhāvanāya pāripūriyā saṃvattati.

‘‘Puna caparaṃ, āvuso, bhikkhu pañcasu upādānakkhandhesu, udayabbayānupassī viharati – ‘iti rūpaṃ iti rūpassa samudayo iti rūpassa atthaṅgamo; iti vedanā iti vedanāya samudayo iti vedanāya atthaṅgamo; iti saññā iti saññāya samudayo iti saññāya atthaṅgamo; iti saṅkhārā iti saṅkhārānaṃ samudayo iti saṅkhārānaṃ atthaṅgamo; iti viññāṇaṃ iti viññāṇassa samudayo iti viññāṇassa atthaṅgamo’ti. Ayaṃ aṭṭhamo hetu aṭṭhamo paccayo ādibrahmacariyikāya paññāya appaṭiladdhāya paṭilābhāya, paṭiladdhāya bhiyyobhāvāya vepullāya bhāvanāya pāripūriyā saṃvattati. Ime aṭṭha dhammā bahukārā.

(Kha) ‘‘katame aṭṭha dhammā bhāvetabbā? Ariyo aṭṭhaṅgiko maggo seyyathidaṃ – sammādiṭṭhi, sammāsaṅkappo, sammāvācā, sammākammanto, sammāājīvo, sammāvāyāmo, sammāsati, sammāsamādhi. Ime aṭṭha dhammā bhāvetabbā.

(Ga) ‘‘katame aṭṭha dhammā pariññeyyā? Aṭṭha lokadhammā – lābho ca, alābho ca, yaso ca, ayaso ca, nindā ca, pasaṃsā ca, sukhañca, dukkhañca. Ime aṭṭha dhammā pariññeyyā.

(Gha) ‘‘katame aṭṭha dhammā pahātabbā? Aṭṭha micchattā – micchādiṭṭhi, micchāsaṅkappo, micchāvācā, micchākammanto, micchāājīvo, micchāvāyāmo, micchāsati, micchāsamādhi. Ime aṭṭha dhammā pahātabbā.

(Ṅa) ‘‘katame aṭṭha dhammā hānabhāgiyā? Aṭṭha kusītavatthūni. Idhāvuso, bhikkhunā kammaṃ kātabbaṃ hoti, tassa evaṃ hoti – ‘kammaṃ kho me kātabbaṃ bhavissati, kammaṃ kho pana me karontassa kāyo kilamissati, handāhaṃ nipajjāmī’ti. So nipajjati, na vīriyaṃ ārabhati appattassa pattiyā anadhigatassa adhigamāya asacchikatassa sacchikiriyāya. Idaṃ paṭhamaṃ kusītavatthu.

‘‘Puna caparaṃ, āvuso, bhikkhunā kammaṃ kataṃ hoti . Tassa evaṃ hoti – ‘ahaṃ kho kammaṃ akāsiṃ, kammaṃ kho pana me karontassa kāyo kilanto, handāhaṃ nipajjāmī’ti. So nipajjati, na vīriyaṃ ārabhati…pe… idaṃ dutiyaṃ kusītavatthu.

‘‘Puna caparaṃ, āvuso, bhikkhunā maggo gantabbo hoti. Tassa evaṃ hoti – ‘maggo kho me gantabbo bhavissati, maggaṃ kho pana me gacchantassa kāyo kilamissati, handāhaṃ nipajjāmī’ti. So nipajjati, na vīriyaṃ ārabhati…pe… idaṃ tatiyaṃ kusītavatthu.

‘‘Puna caparaṃ, āvuso, bhikkhunā maggo gato hoti. Tassa evaṃ hoti – ‘ahaṃ kho maggaṃ agamāsiṃ, maggaṃ kho pana me gacchantassa kāyo kilanto, handāhaṃ nipajjāmī’ti. So nipajjati, na vīriyaṃ ārabhati…pe… idaṃ catutthaṃ kusītavatthu.

‘‘Puna caparaṃ, āvuso, bhikkhu gāmaṃ vā nigamaṃ vā piṇḍāya caranto na labhati lūkhassa vā paṇītassa vā bhojanassa yāvadatthaṃ pāripūriṃ. Tassa evaṃ hoti – ‘ahaṃ kho gāmaṃ vā nigamaṃ vā piṇḍāya caranto nālatthaṃ lūkhassa vā paṇītassa vā bhojanassa yāvadatthaṃ pāripūriṃ, tassa me kāyo kilanto akammañño, handāhaṃ nipajjāmī’ti…pe… idaṃ pañcamaṃ kusītavatthu.

‘‘Puna caparaṃ, āvuso, bhikkhu gāmaṃ vā nigamaṃ vā piṇḍāya caranto labhati lūkhassa vā paṇītassa vā bhojanassa yāvadatthaṃ pāripūriṃ. Tassa evaṃ hoti – ‘ahaṃ kho gāmaṃ vā nigamaṃ vā piṇḍāya caranto alatthaṃ lūkhassa vā paṇītassa vā bhojanassa yāvadatthaṃ pāripūriṃ , tassa me kāyo garuko akammañño, māsācitaṃ maññe, handāhaṃ nipajjāmī’ti. So nipajjati…pe… idaṃ chaṭṭhaṃ kusītavatthu.

‘‘Puna caparaṃ, āvuso, bhikkhuno uppanno hoti appamattako ābādho, tassa evaṃ hoti – ‘uppanno kho me ayaṃ appamattako ābādho atthi kappo nipajjituṃ, handāhaṃ nipajjāmī’ti. So nipajjati…pe… idaṃ sattamaṃ kusītavatthu.

‘‘Puna caparaṃ, āvuso, bhikkhu gilānāvuṭṭhito hoti aciravuṭṭhito gelaññā. Tassa evaṃ hoti – ‘ahaṃ kho gilānāvuṭṭhito aciravuṭṭhito gelaññā. Tassa me kāyo dubbalo akammañño, handāhaṃ nipajjāmī’ti. So nipajjati…pe… idaṃ aṭṭhamaṃ kusītavatthu. Ime aṭṭha dhammā hānabhāgiyā.

(Ca) ‘‘katame aṭṭha dhammā visesabhāgiyā? Aṭṭha ārambhavatthūni. Idhāvuso, bhikkhunā kammaṃ kātabbaṃ hoti, tassa evaṃ hoti – ‘kammaṃ kho me kātabbaṃ bhavissati, kammaṃ kho pana me karontena na sukaraṃ buddhānaṃ sāsanaṃ manasikātuṃ, handāhaṃ vīriyaṃ ārabhāmi appattassa pattiyā anadhigatassa adhigamāya asacchikatassa sacchikiriyāyā’ti. So vīriyaṃ ārabhati appattassa pattiyā anadhigatassa adhigamāya asacchikatassa sacchikiriyāya. Idaṃ paṭhamaṃ ārambhavatthu.

‘‘Puna caparaṃ, āvuso, bhikkhunā kammaṃ kataṃ hoti. Tassa evaṃ hoti – ‘ahaṃ kho kammaṃ akāsiṃ, kammaṃ kho panāhaṃ karonto nāsakkhiṃ buddhānaṃ sāsanaṃ manasikātuṃ, handāhaṃ vīriyaṃ ārabhāmi…pe… idaṃ dutiyaṃ ārambhavatthu.

‘‘Puna caparaṃ, āvuso, bhikkhunā maggo gantabbo hoti. Tassa evaṃ hoti – ‘maggo kho me gantabbo bhavissati, maggaṃ kho pana me gacchantena na sukaraṃ buddhānaṃ sāsanaṃ manasikātuṃ, handāhaṃ vīriyaṃ ārabhāmi…pe… idaṃ tatiyaṃ ārambhavatthu.

‘‘Puna caparaṃ, āvuso, bhikkhunā maggo gato hoti. Tassa evaṃ hoti – ‘ahaṃ kho maggaṃ agamāsiṃ, maggaṃ kho panāhaṃ gacchanto nāsakkhiṃ buddhānaṃ sāsanaṃ manasikātuṃ, handāhaṃ vīriyaṃ ārabhāmi…pe… idaṃ catutthaṃ ārambhavatthu.

‘‘Puna caparaṃ, āvuso, bhikkhu gāmaṃ vā nigamaṃ vā piṇḍāya caranto na labhati lūkhassa vā paṇītassa vā bhojanassa yāvadatthaṃ pāripūriṃ. Tassa evaṃ hoti – ‘ahaṃ kho gāmaṃ vā nigamaṃ vā piṇḍāya caranto nālatthaṃ lūkhassa vā paṇītassa vā bhojanassa yāvadatthaṃ pāripūriṃ , tassa me kāyo lahuko kammañño, handāhaṃ vīriyaṃ ārabhāmi…pe… idaṃ pañcamaṃ ārambhavatthu.

‘‘Puna caparaṃ, āvuso, bhikkhu gāmaṃ vā nigamaṃ vā piṇḍāya caranto labhati lūkhassa vā paṇītassa vā bhojanassa yāvadatthaṃ pāripūriṃ. Tassa evaṃ hoti – ‘ahaṃ kho gāmaṃ vā nigamaṃ vā piṇḍāya caranto alatthaṃ lūkhassa vā paṇītassa vā bhojanassa yāvadatthaṃ pāripūriṃ. Tassa me kāyo balavā kammañño, handāhaṃ vīriyaṃ ārabhāmi…pe… idaṃ chaṭṭhaṃ ārambhavatthu.

‘‘Puna caparaṃ, āvuso, bhikkhuno uppanno hoti appamattako ābādho. Tassa evaṃ hoti – ‘uppanno kho me ayaṃ appamattako ābādho ṭhānaṃ kho panetaṃ vijjati, yaṃ me ābādho pavaḍḍheyya, handāhaṃ vīriyaṃ ārabhāmi…pe… idaṃ sattamaṃ ārambhavatthu.

‘‘Puna caparaṃ, āvuso, bhikkhu gilānā vuṭṭhito hoti aciravuṭṭhito gelaññā. Tassa evaṃ hoti – ‘ahaṃ kho gilānā vuṭṭhito aciravuṭṭhito gelaññā, ṭhānaṃ kho panetaṃ vijjati, yaṃ me ābādho paccudāvatteyya, handāhaṃ vīriyaṃ ārabhāmi appattassa pattiyā anadhigatassa adhigamāya asacchikatassa sacchikiriyāyā’ti. So vīriyaṃ ārabhati appattassa pattiyā anadhigatassa adhigamāya asacchikatassa sacchikiriyāya. Idaṃ aṭṭhamaṃ ārambhavatthu. Ime aṭṭha dhammā visesabhāgiyā.

(Cha) ‘‘katame aṭṭha dhammā duppaṭivijjhā? Aṭṭha akkhaṇā asamayā brahmacariyavāsāya. Idhāvuso, tathāgato ca loke uppanno hoti arahaṃ sammāsambuddho, dhammo ca desiyati opasamiko parinibbāniko sambodhagāmī sugatappavedito. Ayañca puggalo nirayaṃ upapanno hoti. Ayaṃ paṭhamo akkhaṇo asamayo brahmacariyavāsāya.

‘‘Puna caparaṃ, āvuso, tathāgato ca loke uppanno hoti arahaṃ sammāsambuddho, dhammo ca desiyati opasamiko parinibbāniko sambodhagāmī sugatappavedito, ayañca puggalo tiracchānayoniṃ upapanno hoti. Ayaṃ dutiyo akkhaṇo asamayo brahmacariyavāsāya.

‘‘Puna caparaṃ…pe… pettivisayaṃ upapanno hoti. Ayaṃ tatiyo akkhaṇo asamayo brahmacariyavāsāya.

‘‘Puna caparaṃ…pe… aññataraṃ dīghāyukaṃ devanikāyaṃ upapanno hoti. Ayaṃ catuttho akkhaṇo asamayo brahmacariyavāsāya.

‘‘Puna caparaṃ…pe… paccantimesu janapadesu paccājāto hoti milakkhesu aviññātāresu, yattha natthi gati bhikkhūnaṃ bhikkhunīnaṃ upāsakānaṃ upāsikānaṃ. Ayaṃ pañcamo akkhaṇo asamayo brahmacariyavāsāya.

‘‘Puna caparaṃ…pe… ayañca puggalo majjhimesu janapadesu paccājāto hoti, so ca hoti micchādiṭṭhiko viparītadassano – ‘natthi dinnaṃ, natthi yiṭṭhaṃ, natthi hutaṃ, natthi sukatadukkaṭānaṃ kammānaṃ phalaṃ vipāko, natthi ayaṃ loko, natthi paro loko, natthi mātā, natthi pitā, natthi sattā opapātikā, natthi loke samaṇabrāhmaṇā sammaggatā sammāpaṭipannā ye imañca lokaṃ parañca lokaṃ sayaṃ abhiññā sacchikatvā pavedentī’ti. Ayaṃ chaṭṭho akkhaṇo asamayo brahmacariyavāsāya.

‘‘Puna caparaṃ…pe… ayañca puggalo majjhimesu janapadesu paccājāto hoti , so ca hoti duppañño jaḷo eḷamūgo, nappaṭibalo subhāsitadubbhāsitānamatthamaññātuṃ. Ayaṃ sattamo akkhaṇo asamayo brahmacariyavāsāya.

‘‘Puna caparaṃ…pe… ayañca puggalo majjhimesu janapadesu paccājāto hoti, so ca hoti paññavā ajaḷo aneḷamūgo, paṭibalo subhāsitadubbhāsitānamatthamaññātuṃ. Ayaṃ aṭṭhamo akkhaṇo asamayo brahmacariyavāsāya. Ime aṭṭha dhammā duppaṭivijjhā.

(Ja) ‘‘katame aṭṭha dhammā uppādetabbā? Aṭṭha mahāpurisavitakkā – appicchassāyaṃ dhammo, nāyaṃ dhammo mahicchassa. Santuṭṭhassāyaṃ dhammo, nāyaṃ dhammo asantuṭṭhassa. Pavivittassāyaṃ dhammo, nāyaṃ dhammo saṅgaṇikārāmassa. Āraddhavīriyassāyaṃ dhammo, nāyaṃ dhammo kusītassa. Upaṭṭhitasatissāyaṃ dhammo, nāyaṃ dhammo muṭṭhassatissa. Samāhitassāyaṃ dhammo, nāyaṃ dhammo asamāhitassa . Paññavato\footnote{paññāvato (sī. pī.)} ayaṃ dhammo, nāyaṃ dhammo duppaññassa. Nippapañcassāyaṃ dhammo, nāyaṃ dhammo papañcārāmassāti\footnote{nippapañcārāmassa ayaṃ dhammo nippapañcaratino, nāyaṃ dhammo papañcārāmassa papañcaratinoti (sī. syā. pī.) aṅguttarepi tatheva dissati. aṭṭhakathāṭīkā pana oloketabbā} ime aṭṭha dhammā uppādetabbā.

(Jha) ‘‘katame aṭṭha dhammā abhiññeyyā? Aṭṭha abhibhāyatanāni – ajjhattaṃ rūpasaññī eko bahiddhā rūpāni passati parittāni suvaṇṇadubbaṇṇāni , ‘tāni abhibhuyya jānāmi passāmī’ti – evaṃsaññī hoti. Idaṃ paṭhamaṃ abhibhāyatanaṃ.

‘‘Ajjhattaṃ rūpasaññī eko bahiddhā rūpāni passati appamāṇāni suvaṇṇadubbaṇṇāni, ‘tāni abhibhuyya jānāmi passāmī’ti – evaṃsaññī hoti. Idaṃ dutiyaṃ abhibhāyatanaṃ.

‘‘Ajjhattaṃ arūpasaññī eko bahiddhā rūpāni passati parittāni suvaṇṇadubbaṇṇāni, ‘tāni abhibhuyya jānāmi passāmī’ti evaṃsaññī hoti. Idaṃ tatiyaṃ abhibhāyatanaṃ.

‘‘Ajjhattaṃ arūpasaññī eko bahiddhā rūpāni passati appamāṇāni suvaṇṇadubbaṇṇāni, ‘tāni abhibhuyya jānāmi passāmī’ti evaṃsaññī hoti. Idaṃ catutthaṃ abhibhāyatanaṃ.

‘‘Ajjhattaṃ arūpasaññī eko bahiddhā rūpāni passati nīlāni nīlavaṇṇāni nīlanidassanāni nīlanibhāsāni. Seyyathāpi nāma umāpupphaṃ nīlaṃ nīlavaṇṇaṃ nīlanidassanaṃ nīlanibhāsaṃ. Seyyathā vā pana taṃ vatthaṃ bārāṇaseyyakaṃ ubhatobhāgavimaṭṭhaṃ nīlaṃ nīlavaṇṇaṃ nīlanidassanaṃ nīlanibhāsaṃ, evameva ajjhattaṃ arūpasaññī eko bahiddhā rūpāni passati nīlāni nīlavaṇṇāni nīlanidassanāni nīlanibhāsāni, ‘tāni abhibhuyya jānāmi passāmī’ti evaṃsaññī hoti. Idaṃ pañcamaṃ abhibhāyatanaṃ.

‘‘Ajjhattaṃ arūpasaññī eko bahiddhā rūpāni passati pītāni pītavaṇṇāni pītanidassanāni pītanibhāsāni. Seyyathāpi nāma kaṇikārapupphaṃ pītaṃ pītavaṇṇaṃ pītanidassanaṃ pītanibhāsaṃ. Seyyathā vā pana taṃ vatthaṃ bārāṇaseyyakaṃ ubhatobhāgavimaṭṭhaṃ pītaṃ pītavaṇṇaṃ pītanidassanaṃ pītanibhāsaṃ , evameva ajjhattaṃ arūpasaññī eko bahiddhā rūpāni passati pītāni pītavaṇṇāni pītanidassanāni pītanibhāsāni, ‘tāni abhibhuyya jānāmi passāmī’ti evaṃsaññī hoti. Idaṃ chaṭṭhaṃ abhibhāyatanaṃ.

‘‘Ajjhattaṃ arūpasaññī eko bahiddhā rūpāni passati lohitakāni lohitakavaṇṇāni lohitakanidassanāni lohitakanibhāsāni. Seyyathāpi nāma bandhujīvakapupphaṃ lohitakaṃ lohitakavaṇṇaṃ lohitakanidassanaṃ lohitakanibhāsaṃ, seyyathā vā pana taṃ vatthaṃ bārāṇaseyyakaṃ ubhatobhāgavimaṭṭhaṃ lohitakaṃ lohitakavaṇṇaṃ lohitakanidassanaṃ lohitakanibhāsaṃ, evameva ajjhattaṃ arūpasaññī eko bahiddhā rūpāni passati lohitakāni lohitakavaṇṇāni lohitakanidassanāni lohitakanibhāsāni, ‘tāni abhibhuyya jānāmi passāmī’ti evaṃsaññī hoti. Idaṃ sattamaṃ abhibhāyatanaṃ.

‘‘Ajjhattaṃ arūpasaññī eko bahiddhā rūpāni passati odātāni odātavaṇṇāni odātanidassanāni odātanibhāsāni. Seyyathāpi nāma osadhitārakā odātā odātavaṇṇā odātanidassanā odātanibhāsā, seyyathā vā pana taṃ vatthaṃ bārāṇaseyyakaṃ ubhatobhāgavimaṭṭhaṃ odātaṃ odātavaṇṇaṃ odātanidassanaṃ odātanibhāsaṃ, evameva ajjhattaṃ arūpasaññī eko bahiddhā rūpāni passati odātāni odātavaṇṇāni odātanidassanāni odātanibhāsāni, ‘tāni abhibhuyya jānāmi passāmī’ti evaṃsaññī hoti. Idaṃ aṭṭhamaṃ abhibhāyatanaṃ. Ime aṭṭha dhammā abhiññeyyā.

(Ña) ‘‘katame aṭṭha dhammā sacchikātabbā? Aṭṭha vimokkhā – rūpī rūpāni passati. Ayaṃ paṭhamo vimokkho.

‘‘Ajjhattaṃ arūpasaññī eko bahiddhā rūpāni passati. Ayaṃ dutiyo vimokkho.

‘‘Subhanteva adhimutto hoti. Ayaṃ tatiyo vimokkho.

‘‘Sabbaso rūpasaññānaṃ samatikkamā paṭighasaññānaṃ atthaṅgamā nānattasaññānaṃ amanasikārā ‘ananto ākāso’ti ākāsānañcāyatanaṃ upasampajja viharati. Ayaṃ catuttho vimokkho.

‘‘Sabbaso ākāsānañcāyatanaṃ samatikkamma ‘anantaṃ viññāṇa’nti viññāṇañcāyatanaṃ upasampajja viharati. Ayaṃ pañcamo vimokkho.

‘‘Sabbaso viññāṇañcāyatanaṃ samatikkamma ‘natthi kiñcī’ti ākiñcaññāyatanaṃ upasampajja viharati. Ayaṃ chaṭṭho vimokkho.

‘‘Sabbaso ākiñcaññāyatanaṃ samatikkamma nevasaññānāsaññāyatanaṃ upasampajja viharati. Ayaṃ sattamo vimokkho.

‘‘Sabbaso nevasaññānāsaññāyatanaṃ samatikkamma saññāvedayitanirodhaṃ upasampajja viharati. Ayaṃ aṭṭhamo vimokkho. Ime aṭṭha dhammā sacchikātabbā.

‘‘Iti ime asīti dhammā bhūtā tacchā tathā avitathā anaññathā sammā tathāgatena abhisambuddhā.

\subsubsection{Nava dhammā}

\paragraph{359.} ‘‘Nava dhammā bahukārā…pe… nava dhammā sacchikātabbā.

(Ka) ‘‘katame nava dhammā bahukārā? Nava yonisomanasikāramūlakā dhammā, yonisomanasikaroto pāmojjaṃ jāyati, pamuditassa pīti jāyati, pītimanassa kāyo passambhati, passaddhakāyo sukhaṃ vedeti, sukhino cittaṃ samādhiyati, samāhite citte yathābhūtaṃ jānāti passati, yathābhūtaṃ jānaṃ passaṃ nibbindati, nibbindaṃ virajjati, virāgā vimuccati. Ime nava dhammā bahukārā.

(Kha) ‘‘katame nava dhammā bhāvetabbā? Nava pārisuddhipadhāniyaṅgāni – sīlavisuddhi pārisuddhipadhāniyaṅgaṃ, cittavisuddhi pārisuddhipadhāniyaṅgaṃ, diṭṭhivisuddhi pārisuddhipadhāniyaṅgaṃ, kaṅkhāvitaraṇavisuddhi pārisuddhipadhāniyaṅgaṃ, maggāmaggañāṇadassana – visuddhi pārisuddhipadhāniyaṅgaṃ, paṭipadāñāṇadassanavisuddhi pārisuddhipadhāniyaṅgaṃ, ñāṇadassanavisuddhi pārisuddhipadhāniyaṅgaṃ, paññāvisuddhi pārisuddhipadhāniyaṅgaṃ, vimuttivisuddhi pārisuddhipadhāniyaṅgaṃ. Ime nava dhammā bhāvetabbā.

(Ga) ‘‘katame nava dhammā pariññeyyā? Nava sattāvāsā – santāvuso, sattā nānattakāyā nānattasaññino, seyyathāpi manussā ekacce ca devā ekacce ca vinipātikā. Ayaṃ paṭhamo sattāvāso.

‘‘Santāvuso , sattā nānattakāyā ekattasaññino, seyyathāpi devā brahmakāyikā paṭhamābhinibbattā. Ayaṃ dutiyo sattāvāso.

‘‘Santāvuso, sattā ekattakāyā nānattasaññino, seyyathāpi devā ābhassarā. Ayaṃ tatiyo sattāvāso.

‘‘Santāvuso, sattā ekattakāyā ekattasaññino, seyyathāpi devā subhakiṇhā. Ayaṃ catuttho sattāvāso.

‘‘Santāvuso, sattā asaññino appaṭisaṃvedino, seyyathāpi devā asaññasattā. Ayaṃ pañcamo sattāvāso.

‘‘Santāvuso, sattā sabbaso rūpasaññānaṃ samatikkamā paṭighasaññānaṃ atthaṅgamā nānattasaññānaṃ amanasikārā ‘ananto ākāso’ti ākāsānañcāyatanūpagā. Ayaṃ chaṭṭho sattāvāso.

‘‘Santāvuso, sattā sabbaso ākāsānañcāyatanaṃ samatikkamma ‘anantaṃ viññāṇa’nti viññāṇañcāyatanūpagā. Ayaṃ sattamo sattāvāso.

‘‘Santāvuso, sattā sabbaso viññāṇañcāyatanaṃ samatikkamma ‘natthi kiñcī’ti ākiñcaññāyatanūpagā. Ayaṃ aṭṭhamo sattāvāso.

‘‘Santāvuso, sattā sabbaso ākiñcaññāyatanaṃ samatikkamma nevasaññānāsaññāyatanūpagā. Ayaṃ navamo sattāvāso. Ime nava dhammā pariññeyyā.

(Gha) ‘‘katame nava dhammā pahātabbā? Nava taṇhāmūlakā dhammā – taṇhaṃ paṭicca pariyesanā, pariyesanaṃ paṭicca lābho, lābhaṃ paṭicca vinicchayo, vinicchayaṃ paṭicca chandarāgo , chandarāgaṃ paṭicca ajjhosānaṃ, ajjhosānaṃ paṭicca pariggaho, pariggahaṃ paṭicca macchariyaṃ, macchariyaṃ paṭicca ārakkho, ārakkhādhikaraṇaṃ\footnote{ārakkhādhikaraṇaṃ paṭicca (syā. pī. ka.)} daṇḍādānasatthādānakalahaviggahavivādatuvaṃtuvaṃpesuññamusāvādā aneke pāpakā akusalā dhammā sambhavanti. Ime nava dhammā pahātabbā.

(Ṅa) ‘‘katame nava dhammā hānabhāgiyā? Nava āghātavatthūni – ‘anatthaṃ me acarī’ti āghātaṃ bandhati, ‘anatthaṃ me caratī’ti āghātaṃ bandhati, ‘anatthaṃ me carissatī’ti āghātaṃ bandhati; ‘piyassa me manāpassa anatthaṃ acarī’ti āghātaṃ bandhati…pe… ‘anatthaṃ caratī’ti āghātaṃ bandhati…pe… ‘anatthaṃ carissatī’ti āghātaṃ bandhati; ‘appiyassa me amanāpassa atthaṃ acarī’ti āghātaṃ bandhati…pe… ‘atthaṃ caratī’ti āghātaṃ bandhati…pe… ‘atthaṃ carissatī’ti āghātaṃ bandhati. Ime nava dhammā hānabhāgiyā.

(Ca) ‘‘katame nava dhammā visesabhāgiyā? Nava āghātapaṭivinayā – ‘anatthaṃ me acari, taṃ kutettha labbhā’ti āghātaṃ paṭivineti; ‘anatthaṃ me carati, taṃ kutettha labbhā’ti āghātaṃ paṭivineti; ‘anatthaṃ me carissati, taṃ kutettha labbhā’ti āghātaṃ paṭivineti; ‘piyassa me manāpassa anatthaṃ acari…pe… anatthaṃ carati…pe… anatthaṃ carissati, taṃ kutettha labbhā’ti āghātaṃ paṭivineti; ‘appiyassa me amanāpassa atthaṃ acari…pe… atthaṃ carati…pe… atthaṃ carissati, taṃ kutettha labbhā’ti āghātaṃ paṭivineti. Ime nava dhammā visesabhāgiyā.

(Cha) ‘‘katame nava dhammā duppaṭivijjhā? Nava nānattā – dhātunānattaṃ paṭicca uppajjati phassanānattaṃ, phassanānattaṃ paṭicca uppajjati vedanānānattaṃ, vedanānānattaṃ paṭicca uppajjati saññānānattaṃ, saññānānattaṃ paṭicca uppajjati saṅkappanānattaṃ, saṅkappanānattaṃ paṭicca uppajjati chandanānattaṃ, chandanānattaṃ paṭicca uppajjati pariḷāhanānattaṃ, pariḷāhanānattaṃ paṭicca uppajjati pariyesanānānattaṃ, pariyesanānānattaṃ paṭicca uppajjati lābhanānattaṃ. Ime nava dhammā duppaṭivijjhā.

(Ja) ‘‘katame nava dhammā uppādetabbā? Nava saññā – asubhasaññā, maraṇasaññā, āhārepaṭikūlasaññā , sabbalokeanabhiratisaññā\footnote{anabhiratasaññā (syā. ka.)}, aniccasaññā, anicce dukkhasaññā, dukkhe anattasaññā, pahānasaññā, virāgasaññā. Ime nava dhammā uppādetabbā.

(Jha) ‘‘katame nava dhammā abhiññeyyā? Nava anupubbavihārā – idhāvuso, bhikkhu vivicceva kāmehi vivicca akusalehi dhammehi savitakkaṃ savicāraṃ vivekajaṃ pītisukhaṃ paṭhamaṃ jhānaṃ upasampajja viharati. Vitakkavicārānaṃ vūpasamā…pe… dutiyaṃ jhānaṃ upasampajja viharati. Pītiyā ca virāgā …pe… tatiyaṃ jhānaṃ upasampajja viharati. Sukhassa ca pahānā…pe… catutthaṃ jhānaṃ upasampajja viharati. Sabbaso rūpasaññānaṃ samatikkamā…pe… ākāsānañcāyatanaṃ upasampajja viharati. Sabbaso ākāsānañcāyatanaṃ samatikkamma ‘anantaṃ viññāṇa’nti viññāṇañcāyatanaṃ upasampajja viharati. Sabbaso viññāṇañcāyatanaṃ samatikkamma ‘natthi kiñcī’ti ākiñcaññāyatanaṃ upasampajja viharati. Sabbaso ākiñcaññāyatanaṃ samatikkamma nevasaññānāsaññāyatanaṃ upasampajja viharati. Sabbaso nevasaññānāsaññāyatanaṃ samatikkamma saññāvedayitanirodhaṃ upasampajja viharati. Ime nava dhammā abhiññeyyā.

(Ña) ‘‘katame nava dhammā sacchikātabbā? Nava anupubbanirodhā – paṭhamaṃ jhānaṃ samāpannassa kāmasaññā niruddhā hoti, dutiyaṃ jhānaṃ samāpannassa vitakkavicārā niruddhā honti, tatiyaṃ jhānaṃ samāpannassa pīti niruddhā hoti, catutthaṃ jhānaṃ samāpannassa assāsapassāssā niruddhā honti, ākāsānañcāyatanaṃ samāpannassa rūpasaññā niruddhā hoti, viññāṇañcāyatanaṃ samāpannassa ākāsānañcāyatanasaññā niruddhā hoti, ākiñcaññāyatanaṃ samāpannassa viññāṇañcāyatanasaññā niruddhā hoti, nevasaññānāsaññāyatanaṃ samāpannassa ākiñcaññāyatanasaññā niruddhā hoti, saññāvedayitanirodhaṃ samāpannassa saññā ca vedanā ca niruddhā honti. Ime nava dhammā sacchikātabbā.

‘‘Iti ime navuti dhammā bhūtā tacchā tathā avitathā anaññathā sammā tathāgatena abhisambuddhā.

\subsubsection{Dasa dhammā}

\paragraph{360.} ‘‘Dasa dhammā bahukārā…pe… dasa dhammā sacchikātabbā.

(Ka) ‘‘katame dasa dhammā bahukārā? Dasa nāthakaraṇādhammā – idhāvuso, bhikkhu sīlavā hoti, pātimokkhasaṃvarasaṃvuto viharati ācāragocarasampanno, aṇumattesu vajjesu bhayadassāvī samādāya sikkhati sikkhāpadesu, yaṃpāvuso, bhikkhu sīlavā hoti…pe… sikkhati sikkhāpadesu. Ayampi dhammo nāthakaraṇo.

‘‘Puna caparaṃ, āvuso, bhikkhu bahussuto …pe… diṭṭhiyā suppaṭividdhā, yaṃpāvuso, bhikkhu bahussuto…pe… ayampi dhammo nāthakaraṇo.

‘‘Puna caparaṃ, āvuso, bhikkhu kalyāṇamitto hoti kalyāṇasahāyo kalyāṇasampavaṅko. Yaṃpāvuso, bhikkhu…pe… kalyāṇasampavaṅko. Ayampi dhammo nāthakaraṇo.

‘‘Puna caparaṃ, āvuso, bhikkhu suvaco hoti sovacassakaraṇehi dhammehi samannāgato, khamo padakkhiṇaggāhī anusāsaniṃ. Yaṃpāvuso, bhikkhu…pe… anusāsaniṃ. Ayampi dhammo nāthakaraṇo.

‘‘Puna caparaṃ, āvuso, bhikkhu yāni tāni sabrahmacārīnaṃ uccāvacāni kiṃkaraṇīyāni tattha dakkho hoti analaso tatrupāyāya vīmaṃsāya samannāgato, alaṃ kātuṃ, alaṃ saṃvidhātuṃ. Yaṃpāvuso, bhikkhu…pe… alaṃ saṃvidhātuṃ. Ayampi dhammo nāthakaraṇo.

‘‘Puna caparaṃ, āvuso, bhikkhu dhammakāmo hoti piyasamudāhāro abhidhamme abhivinaye uḷārapāmojjo. Yaṃpāvuso, bhikkhu…pe… uḷārapāmojjo. Ayampi dhammo nāthakaraṇo.

‘‘Puna caparaṃ, āvuso, bhikkhu santuṭṭho hoti itarītarehi cīvarapiṇḍapātasenāsanagilānappaccayabhesajjaparikkhārehi. Yaṃpāvuso, bhikkhu …pe… ayampi dhammo nāthakaraṇo.

‘‘Puna caparaṃ, āvuso, bhikkhu āraddhavīriyo viharati…pe… kusalesu dhammesu. Yaṃpāvuso, bhikkhu…pe… ayampi dhammo nāthakaraṇo.

‘‘Puna caparaṃ, āvuso, bhikkhu satimā hoti, paramena satinepakkena samannāgato, cirakatampi cirabhāsitampi saritā anussaritā. Yaṃpāvuso, bhikkhu…pe… ayampi dhammo nāthakaraṇo.

‘‘Puna caparaṃ, āvuso, bhikkhu paññavā hoti udayatthagāminiyā paññāya samannāgato, ariyāya nibbedhikāya sammā dukkhakkhayagāminiyā. Yaṃpāvuso, bhikkhu…pe… ayampi dhammo nāthakaraṇo. Ime dasa dhammā bahukārā.

(Kha) ‘‘katame dasa dhammā bhāvetabbā? Dasa kasiṇāyatanāni – pathavīkasiṇameko sañjānāti uddhaṃ adho tiriyaṃ advayaṃ appamāṇaṃ. Āpokasiṇameko sañjānāti…pe… tejokasiṇameko sañjānāti… vāyokasiṇameko sañjānāti… nīlakasiṇameko sañjānāti… pītakasiṇameko sañjānāti… lohitakasiṇameko sañjānāti… odātakasiṇameko sañjānāti… ākāsakasiṇameko sañjānāti… viññāṇakasiṇameko sañjānāti uddhaṃ adho tiriyaṃ advayaṃ appamāṇaṃ . Ime dasa dhammā bhāvetabbā.

(Ga) ‘‘katame dasa dhammā pariññeyyā? Dasāyatanāni – cakkhāyatanaṃ, rūpāyatanaṃ, sotāyatanaṃ, saddāyatanaṃ, ghānāyatanaṃ, gandhāyatanaṃ, jivhāyatanaṃ, rasāyatanaṃ, kāyāyatanaṃ, phoṭṭhabbāyatanaṃ. Ime dasa dhammā pariññeyyā.

(Gha) ‘‘katame dasa dhammā pahātabbā? Dasa micchattā – micchādiṭṭhi, micchāsaṅkappo, micchāvācā, micchākammanto, micchāājīvo, micchāvāyāmo, micchāsati, micchāsamādhi, micchāñāṇaṃ, micchāvimutti. Ime dasa dhammā pahātabbā.

(Ṅa) ‘‘katame dasa dhammā hānabhāgiyā? Dasa akusalakammapathā – pāṇātipāto, adinnādānaṃ, kāmesumicchācāro, musāvādo, pisuṇā vācā, pharusā vācā, samphappalāpo, abhijjhā, byāpādo, micchādiṭṭhi. Ime dasa dhammā hānabhāgiyā.

(Ca) ‘‘katame dasa dhammā visesabhāgiyā? Dasa kusalakammapathā – pāṇātipātā veramaṇī, adinnādānā veramaṇī, kāmesumicchācārā veramaṇī, musāvādā veramaṇī, pisuṇāya vācāya veramaṇī, pharusāya vācāya veramaṇī, samphappalāpā veramaṇī, anabhijjhā, abyāpādo, sammādiṭṭhi. Ime dasa dhammā visesabhāgiyā.

(Cha) ‘‘katame dasa dhammā duppaṭivijjhā? Dasa ariyavāsā – idhāvuso , bhikkhu pañcaṅgavippahīno hoti, chaḷaṅgasamannāgato, ekārakkho, caturāpasseno, paṇunnapaccekasacco, samavayasaṭṭhesano, anāvilasaṅkappo, passaddhakāyasaṅkhāro, suvimuttacitto, suvimuttapañño.

‘‘Kathañcāvuso , bhikkhu pañcaṅgavippahīno hoti? Idhāvuso, bhikkhuno kāmacchando pahīno hoti, byāpādo pahīno hoti, thinamiddhaṃ pahīnaṃ hoti, uddhaccakukkuccaṃ pahīnaṃ hoti, vicikicchā pahīnā hoti. Evaṃ kho, āvuso, bhikkhu pañcaṅgavippahīno hoti.

‘‘Kathañcāvuso, bhikkhu chaḷaṅgasamannāgato hoti? Idhāvuso, bhikkhu cakkhunā rūpaṃ disvā neva sumano hoti na dummano, upekkhako viharati sato sampajāno. Sotena saddaṃ sutvā…pe… ghānena gandhaṃ ghāyitvā… jivhāya rasaṃ sāyitvā… kāyena phoṭṭhabbaṃ phusitvā… manasā dhammaṃ viññāya neva sumano hoti na dummano, upekkhako viharati sato sampajāno. Evaṃ kho, āvuso, bhikkhu chaḷaṅgasamannāgato hoti.

‘‘Kathañcāvuso, bhikkhu ekārakkho hoti? Idhāvuso, bhikkhu satārakkhena cetasā samannāgato hoti. Evaṃ kho, āvuso, bhikkhu ekārakkho hoti.

‘‘Kathañcāvuso, bhikkhu caturāpasseno hoti? Idhāvuso, bhikkhu saṅkhāyekaṃ paṭisevati, saṅkhāyekaṃ adhivāseti, saṅkhāyekaṃ parivajjeti, saṅkhāyekaṃ vinodeti. Evaṃ kho, āvuso, bhikkhu caturāpasseno hoti.

‘‘Kathañcāvuso, bhikkhu paṇunnapaccekasacco hoti? Idhāvuso, bhikkhuno yāni tāni puthusamaṇabrāhmaṇānaṃ puthupaccekasaccāni, sabbāni tāni nunnāni honti paṇunnāni cattāni vantāni muttāni pahīnāni paṭinissaṭṭhāni. Evaṃ kho, āvuso, bhikkhu paṇunnapaccekasacco hoti.

‘‘Kathañcāvuso, bhikkhu samavayasaṭṭhesano hoti? Idhāvuso, bhikkhuno kāmesanā pahīnā hoti, bhavesanā pahīnā hoti, brahmacariyesanā paṭippassaddhā. Evaṃ kho, āvuso, bhikkhu samavayasaṭṭhesano hoti.

‘‘Kathañcāvuso , bhikkhu anāvilasaṅkappā hoti? Idhāvuso, bhikkhuno kāmasaṅkappo pahīno hoti, byāpādasaṅkappo pahīno hoti, vihiṃsāsaṅkappo pahīno hoti. Evaṃ kho, āvuso, bhikkhu anāvilasaṅkappo hoti.

‘‘Kathañcāvuso, bhikkhu passaddhakāyasaṅkhāro hoti? Idhāvuso, bhikkhu sukhassa ca pahānā dukkhassa ca pahānā pubbeva somanassadomanassānaṃ atthaṅgamā adukkhamasukhaṃ upekkhāsatipārisuddhiṃ catutthaṃ jhānaṃ upasampajja viharati. Evaṃ kho, āvuso, bhikkhu passaddhakāyasaṅkhāro hoti.

‘‘Kathañcāvuso, bhikkhu suvimuttacitto hoti? Idhāvuso, bhikkhuno rāgā cittaṃ vimuttaṃ hoti, dosā cittaṃ vimuttaṃ hoti, mohā cittaṃ vimuttaṃ hoti. Evaṃ kho, āvuso, bhikkhu suvimuttacitto hoti.

‘‘Kathañcāvuso, bhikkhu suvimuttapañño hoti? Idhāvuso, bhikkhu ‘rāgo me pahīno ucchinnamūlo tālāvatthukato anabhāvaṃkato āyatiṃ anuppādadhammo’ti pajānāti. ‘Doso me pahīno…pe… āyatiṃ anuppādadhammo’ti pajānāti. ‘Moho me pahīno …pe… āyatiṃ anuppādadhammo’ti pajānāti. Evaṃ kho, āvuso, bhikkhu suvimuttapañño hoti. Ime dasa dhammā duppaṭivijjhā.

(Ja) ‘‘katame dasa dhammā uppādetabbā? Dasa saññā – asubhasaññā, maraṇasaññā, āhārepaṭikūlasaññā, sabbalokeanabhiratisaññā, aniccasaññā, anicce dukkhasaññā, dukkhe anattasaññā, pahānasaññā, virāgasaññā, nirodhasaññā. Ime dasa dhammā uppādetabbā.

(Jha) ‘‘katame dasa dhammā abhiññeyyā? Dasa nijjaravatthūni – sammādiṭṭhissa micchādiṭṭhi nijjiṇṇā hoti. Ye ca micchādiṭṭhipaccayā aneke pāpakā akusalā dhammā sambhavanti, te cassa nijjiṇṇā honti. Sammāsaṅkappassa micchāsaṅkappo…pe… sammāvācassa micchāvācā… sammākammantassa micchākammanto… sammāājīvassa micchāājīvo… sammāvāyāmassa micchāvāyāmo… sammāsatissa micchāsati… sammāsamādhissa micchāsamādhi… sammāñāṇassa micchāñāṇaṃ nijjiṇṇaṃ hoti. Sammāvimuttissa micchāvimutti nijjiṇṇā hoti. Ye ca micchāvimuttipaccayā aneke pāpakā akusalā dhammā sambhavanti, te cassa nijjiṇṇā honti. Ime dasa dhammā abhiññeyyā.

(Ña) ‘‘katame dasa dhammā sacchikātabbā? Dasa asekkhā dhammā – asekkhā sammādiṭṭhi, asekkho sammāsaṅkappo, asekkhā sammāvācā, asekkho sammākammanto, asekkho sammāājīvo, asekkho sammāvāyāmo, asekkhā sammāsati, asekkho sammāsamādhi, asekkhaṃ sammāñāṇaṃ, asekkhā sammāvimutti. Ime dasa dhammā sacchikātabbā.

‘‘Iti ime satadhammā bhūtā tacchā tathā avitathā anaññathā sammā tathāgatena abhisambuddhā’’ti. Idamavocāyasmā sāriputto. Attamanā te bhikkhū āyasmato sāriputtassa bhāsitaṃ abhinandunti.

\xsectionEnd{Dasuttarasuttaṃ niṭṭhitaṃ ekādasamaṃ. \\ Pāthikavaggo\footnote{pāṭikavaggo (sī. syā. pī.)} niṭṭhito.}

\paragraph{}
Tassuddānaṃ –

Pāthiko ca\footnote{pāṭikañca (syā. kaṃ.)} udumbaraṃ\footnote{pāṭikodumbarīceva (sī. pī.)}, cakkavatti aggaññakaṃ;

Sampasādanapāsādaṃ\footnote{sampasādañca pāsādaṃ (sī. syā. kaṃ. pī.)}, mahāpurisalakkhaṇaṃ.

Siṅgālāṭānāṭiyakaṃ , saṅgīti ca dasuttaraṃ;

Ekādasahi suttehi, pāthikavaggoti vuccati.

\xsectionEnd{Pāthikavaggapāḷi niṭṭhitā. \\ Tīhi vaggehi paṭimaṇḍito sakalo \\ Dīghanikāyo samatto.}





% \newpage
%\section{1. Brahmajālasuttaṃ}

%\subsection{Paribbājakakathā}

%\paragraph{1.}





\end{document}
