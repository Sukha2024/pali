\section{Mahāgovindasuttaṃ}

\paragraph{293.} Evaṃ me sutaṃ – ekaṃ samayaṃ bhagavā rājagahe viharati gijjhakūṭe pabbate. Atha kho pañcasikho gandhabbaputto abhikkantāya rattiyā abhikkantavaṇṇo kevalakappaṃ gijjhakūṭaṃ pabbataṃ obhāsetvā yena bhagavā tenupasaṅkami; upasaṅkamitvā bhagavantaṃ abhivādetvā ekamantaṃ aṭṭhāsi. Ekamantaṃ ṭhito kho pañcasikho gandhabbaputto bhagavantaṃ etadavoca – ‘‘yaṃ kho me, bhante, devānaṃ tāvatiṃsānaṃ sammukhā sutaṃ sammukhā paṭiggahitaṃ, ārocemi taṃ bhagavato’’ti. ‘‘Ārocehi me tvaṃ, pañcasikhā’’ti bhagavā avoca.

\subsubsection{Devasabhā}

\paragraph{294.} ‘‘Purimāni, bhante, divasāni purimatarāni tadahuposathe pannarase pavāraṇāya puṇṇāya puṇṇamāya rattiyā kevalakappā ca devā tāvatiṃsā sudhammāyaṃ sabhāyaṃ sannisinnā honti sannipatitā; mahatī ca dibbaparisā samantato nisinnā honti, cattāro ca mahārājāno catuddisā nisinnā honti; puratthimāya disāya dhataraṭṭho mahārājā pacchimābhimukho nisinno hoti deve purakkhatvā; dakkhiṇāya disāya virūḷhako mahārājā uttarābhimukho nisinno hoti deve purakkhatvā; pacchimāya disāya virūpakkho mahārājā puratthābhimukho nisinno hoti deve purakkhatvā; uttarāya disāya vessavaṇo mahārājā dakkhiṇābhimukho nisinno hoti deve purakkhatvā. Yadā bhante, kevalakappā ca devā tāvatiṃsā sudhammāyaṃ sabhāyaṃ sannisinnā honti sannipatitā, mahatī ca dibbaparisā samantato nisinnā honti, cattāro ca mahārājāno catuddisā nisinnā honti, idaṃ nesaṃ hoti āsanasmiṃ; atha pacchā amhākaṃ āsanaṃ hoti.

‘‘Ye te, bhante, devā bhagavati brahmacariyaṃ caritvā adhunūpapannā tāvatiṃsakāyaṃ, te aññe deve atirocanti vaṇṇena ceva yasasā ca. Tena sudaṃ, bhante, devā tāvatiṃsā attamanā honti pamuditā pītisomanassajātā; ‘dibbā vata, bho, kāyā paripūrenti, hāyanti asurakāyā’ti.

\paragraph{295.} ‘‘Atha kho, bhante, sakko devānamindo devānaṃ tāvatiṃsānaṃ sampasādaṃ viditvā imāhi gāthāhi anumodi –

‘Modanti vata bho devā, tāvatiṃsā sahindakā;

Tathāgataṃ namassantā, dhammassa ca sudhammataṃ.

Nave deve ca passantā, vaṇṇavante yasassine;

Sugatasmiṃ brahmacariyaṃ, caritvāna idhāgate.

Te aññe atirocanti, vaṇṇena yasasāyunā;

Sāvakā bhūripaññassa, visesūpagatā idha.

Idaṃ disvāna nandanti, tāvatiṃsā sahindakā;

Tathāgataṃ namassantā, dhammassa ca sudhammata’nti.

‘‘Tena sudaṃ , bhante, devā tāvatiṃsā bhiyyoso mattāya attamanā honti pamuditā pītisomanassajātā; ‘dibbā vata, bho, kāyā paripūrenti, hāyanti asurakāyā’’’ti.

\subsubsection{Aṭṭha yathābhuccavaṇṇā}

\paragraph{296.} ‘‘Atha kho, bhante, sakko devānamindo devānaṃ tāvatiṃsānaṃ sampasādaṃ viditvā deve tāvatiṃse āmantesi – ‘iccheyyātha no tumhe, mārisā, tassa bhagavato aṭṭha yathābhucce vaṇṇe sotu’nti? ‘Icchāma mayaṃ, mārisa, tassa bhagavato aṭṭha yathābhucce vaṇṇe sotu’nti. Atha kho, bhante, sakko devānamindo devānaṃ tāvatiṃsānaṃ bhagavato aṭṭha yathābhucce vaṇṇe payirudāhāsi – ‘taṃ kiṃ maññanti, bhonto devā tāvatiṃsā? Yāvañca so bhagavā bahujanahitāya paṭipanno bahujanasukhāya lokānukampāya atthāya hitāya sukhāya devamanussānaṃ. Evaṃ bahujanahitāya paṭipannaṃ bahujanasukhāya lokānukampāya atthāya hitāya sukhāya devamanussānaṃ imināpaṅgena samannāgataṃ satthāraṃ neva atītaṃse samanupassāma, na panetarahi, aññatra tena bhagavatā.

‘‘Svākkhāto kho pana tena bhagavatā dhammo sandiṭṭhiko akāliko ehipassiko opaneyyiko paccattaṃ veditabbo viññūhi. Evaṃ opaneyyikassa dhammassa desetāraṃ imināpaṅgena samannāgataṃ satthāraṃ neva atītaṃse samanupassāma, na panetarahi, aññatra tena bhagavatā.

‘‘Idaṃ kusalanti kho pana tena bhagavatā supaññattaṃ, idaṃ akusalanti supaññattaṃ. Idaṃ sāvajjaṃ idaṃ anavajjaṃ, idaṃ sevitabbaṃ idaṃ na sevitabbaṃ, idaṃ hīnaṃ idaṃ paṇītaṃ, idaṃ kaṇhasukkasappaṭibhāganti supaññattaṃ. Evaṃ kusalākusalasāvajjānavajjasevitabbāsevitabbahīna-paṇītakaṇhasukkasappaṭibhāgānaṃ dhammānaṃ paññapetāraṃ imināpaṅgena samannāgataṃ satthāraṃ neva atītaṃse samanupassāma, na panetarahi, aññatra tena bhagavatā.

‘‘Supaññattā kho pana tena bhagavatā sāvakānaṃ nibbānagāminī paṭipadā, saṃsandati nibbānañca paṭipadā ca. Seyyathāpi nāma gaṅgodakaṃ yamunodakena saṃsandati sameti, evameva supaññattā tena bhagavatā sāvakānaṃ nibbānagāminī paṭipadā, saṃsandati nibbānañca paṭipadā ca. Evaṃ nibbānagāminiyā paṭipadāya paññapetāraṃ imināpaṅgena samannāgataṃ satthāraṃ neva atītaṃse samanupassāma, na panetarahi, aññatra tena bhagavatā.

‘‘Abhinipphanno\footnote{abhinippanno (pī. ka.)} kho pana tassa bhagavato lābho abhinipphanno siloko, yāva maññe khattiyā sampiyāyamānarūpā viharanti, vigatamado kho pana so bhagavā āhāraṃ āhāreti. Evaṃ vigatamadaṃ āhāraṃ āharayamānaṃ imināpaṅgena samannāgataṃ satthāraṃ neva atītaṃse samanupassāma, na panetarahi, aññatra tena bhagavatā.

‘‘Laddhasahāyo kho pana so bhagavā sekhānañceva paṭipannānaṃ khīṇāsavānañca vusitavataṃ. Te bhagavā apanujja ekārāmataṃ anuyutto viharati. Evaṃ ekārāmataṃ anuyuttaṃ imināpaṅgena samannāgataṃ satthāraṃ neva atītaṃse samanupassāma, na panetarahi, aññatra tena bhagavatā.

‘‘Yathāvādī kho pana so bhagavā tathākārī, yathākārī tathāvādī, iti yathāvādī tathākārī, yathākārī tathāvādī. Evaṃ dhammānudhammappaṭipannaṃ imināpaṅgena samannāgataṃ satthāraṃ neva atītaṃse samanupassāma, na panetarahi, aññatra tena bhagavatā.

‘‘Tiṇṇavicikiccho kho pana so bhagavā vigatakathaṃkatho pariyositasaṅkappo ajjhāsayaṃ ādibrahmacariyaṃ. Evaṃ tiṇṇavicikicchaṃ vigatakathaṃkathaṃ pariyositasaṅkappaṃ ajjhāsayaṃ ādibrahmacariyaṃ imināpaṅgena samannāgataṃ satthāraṃ neva atītaṃse samanupassāma, na panetarahi, aññatra tena bhagavatā’ti.

\paragraph{297.} ‘‘Ime kho, bhante, sakko devānamindo devānaṃ tāvatiṃsānaṃ bhagavato aṭṭha yathābhucce vaṇṇe payirudāhāsi. Tena sudaṃ, bhante, devā tāvatiṃsā bhiyyoso mattāya attamanā honti pamuditā pītisomanassajātā bhagavato aṭṭha yathābhucce vaṇṇe sutvā. Tatra, bhante, ekacce devā evamāhaṃsu – ‘aho vata, mārisā, cattāro sammāsambuddhā loke uppajjeyyuṃ dhammañca deseyyuṃ yathariva bhagavā. Tadassa bahujanahitāya bahujanasukhāya lokānukampāya atthāya hitāya sukhāya devamanussāna’nti. Ekacce devā evamāhaṃsu – ‘tiṭṭhantu, mārisā, cattāro sammāsambuddhā, aho vata, mārisā, tayo sammāsambuddhā loke uppajjeyyuṃ dhammañca deseyyuṃ yathariva bhagavā. Tadassa bahujanahitāya bahujanasukhāya lokānukampāya atthāya hitāya sukhāya devamanussāna’nti. Ekacce devā evamāhaṃsu – ‘tiṭṭhantu, mārisā, tayo sammāsambuddhā, aho vata, mārisā, dve sammāsambuddhā loke uppajjeyyuṃ dhammañca deseyyuṃ yathariva bhagavā. Tadassa bahujanahitāya bahujanasukhāya lokānukampāya atthāya hitāya sukhāya devamanussāna’nti.

\paragraph{298.} ‘‘Evaṃ vutte , bhante, sakko devānamindo deve tāvatiṃse etadavoca – ‘aṭṭhānaṃ kho etaṃ, mārisā, anavakāso, yaṃ ekissā lokadhātuyā dve arahanto sammāsambuddhā apubbaṃ acarimaṃ uppajjeyyuṃ, netaṃ ṭhānaṃ vijjati. Aho vata, mārisā, so bhagavā appābādho appātaṅko ciraṃ dīghamaddhānaṃ tiṭṭheyya. Tadassa bahujanahitāya bahujanasukhāya lokānukampāya atthāya hitāya sukhāya devamanussāna’nti. Atha kho, bhante, yenatthena devā tāvatiṃsā sudhammāyaṃ sabhāyaṃ sannisinnā honti sannipatitā, taṃ atthaṃ cintayitvā taṃ atthaṃ mantayitvā vuttavacanāpi taṃ cattāro mahārājāno tasmiṃ atthe honti. Paccānusiṭṭhavacanāpi taṃ cattāro mahārājāno tasmiṃ atthe honti, sakesu sakesu āsanesu ṭhitā avipakkantā.

Te vuttavākyā rājāno, paṭiggayhānusāsaniṃ;

Vippasannamanā santā, aṭṭhaṃsu samhi āsaneti.

\paragraph{299.} ‘‘Atha kho, bhante, uttarāya disāya uḷāro āloko sañjāyi, obhāso pāturahosi atikkammeva devānaṃ devānubhāvaṃ. Atha kho, bhante, sakko devānamindo deve tāvatiṃse āmantesi – ‘yathā kho, mārisā, nimittāni dissanti, uḷāro āloko sañjāyati, obhāso pātubhavati , brahmā pātubhavissati; brahmuno hetaṃ pubbanimittaṃ pātubhāvāya, yadidaṃ āloko sañjāyati obhāso pātubhavatīti.

‘Yathā nimittā dissanti, brahmā pātubhavissati;

Brahmuno hetaṃ nimittaṃ, obhāso vipulo mahā’ti.

\subsubsection{Sanaṅkumārakathā}

\paragraph{300.} ‘‘Atha kho, bhante, devā tāvatiṃsā yathāsakesu āsanesu nisīdiṃsu – ‘obhāsametaṃ ñassāma, yaṃvipāko bhavissati, sacchikatvāva naṃ gamissāmā’ti. Cattāropi mahārājāno yathāsakesu āsanesu nisīdiṃsu – ‘obhāsametaṃ ñassāma, yaṃvipāko bhavissati, sacchikatvāva naṃ gamissāmā’ti. Idaṃ sutvā devā tāvatiṃsā ekaggā samāpajjiṃsu – ‘obhāsametaṃ ñassāma, yaṃvipāko bhavissati, sacchikatvāva naṃ gamissāmā’ti.

‘‘Yadā, bhante, brahmā sanaṅkumāro devānaṃ tāvatiṃsānaṃ pātubhavati, oḷārikaṃ attabhāvaṃ abhinimminitvā pātubhavati. Yo kho pana, bhante, brahmuno pakativaṇṇo, anabhisambhavanīyo so devānaṃ tāvatiṃsānaṃ cakkhupathasmiṃ. Yadā, bhante, brahmā sanaṅkumāro devānaṃ tāvatiṃsānaṃ pātubhavati, so aññe deve atirocati vaṇṇena ceva yasasā ca. Seyyathāpi, bhante, sovaṇṇo viggaho mānusaṃ viggahaṃ atirocati, evameva kho, bhante, yadā brahmā sanaṅkumāro devānaṃ tāvatiṃsānaṃ pātubhavati, so aññe deve atirocati vaṇṇena ceva yasasā ca. Yadā, bhante, brahmā sanaṅkumāro devānaṃ tāvatiṃsānaṃ pātubhavati, na tassaṃ parisāyaṃ koci devo abhivādeti vā paccuṭṭheti vā āsanena vā nimanteti. Sabbeva tuṇhībhūtā pañjalikā pallaṅkena nisīdanti – ‘yassadāni devassa pallaṅkaṃ icchissati brahmā sanaṅkumāro, tassa devassa pallaṅke nisīdissatī’ti. Yassa kho pana, bhante, devassa brahmā sanaṅkumāro pallaṅke nisīdati, uḷāraṃ so labhati devo vedapaṭilābhaṃ, uḷāraṃ so labhati devo somanassapaṭilābhaṃ . Seyyathāpi, bhante, rājā khattiyo muddhāvasitto adhunābhisitto rajjena, uḷāraṃ so labhati vedapaṭilābhaṃ, uḷāraṃ so labhati somanassapaṭilābhaṃ, evameva kho, bhante, yassa devassa brahmā sanaṅkumāro pallaṅke nisīdati, uḷāraṃ so labhati devo vedapaṭilābhaṃ, uḷāraṃ so labhati devo somanassapaṭilābhaṃ. Atha, bhante, brahmā sanaṅkumāro devānaṃ tāvatiṃsānaṃ sampasādaṃ viditvā antarahito imāhi gāthāhi anumodi –

‘Modanti vata bho devā, tāvatiṃsā sahindakā;

Tathāgataṃ namassantā, dhammassa ca sudhammataṃ.

‘Nave deve ca passantā, vaṇṇavante yasassine;

Sugatasmiṃ brahmacariyaṃ, caritvāna idhāgate.

‘Te aññe atirocanti, vaṇṇena yasasāyunā;

Sāvakā bhūripaññassa, visesūpagatā idha.

‘Idaṃ disvāna nandanti, tāvatiṃsā sahindakā;

Tathāgataṃ namassantā, dhammassa ca sudhammata’nti.

\paragraph{301.} ‘‘Imamatthaṃ, bhante, brahmā sanaṅkumāro abhāsittha. Imamatthaṃ, bhante , brahmuno sanaṅkumārassa bhāsato aṭṭhaṅgasamannāgato saro hoti vissaṭṭho ca viññeyyo ca mañju ca savanīyo ca bindu ca avisārī ca gambhīro ca ninnādī ca. Yathāparisaṃ kho pana, bhante, brahmā sanaṅkumāro sarena viññāpeti, na cassa bahiddhā parisāya ghoso niccharati. Yassa kho pana, bhante, evaṃ aṭṭhaṅgasamannāgato saro hoti, so vuccati ‘brahmassaro’ti. Atha kho, bhante, devā tāvatiṃsā brahmānaṃ sanaṅkumāraṃ etadavocuṃ – ‘sādhu, mahābrahme, etadeva mayaṃ saṅkhāya modāma; atthi ca sakkena devānamindena tassa bhagavato aṭṭha yathābhuccā vaṇṇā bhāsitā; te ca mayaṃ saṅkhāya modāmā’ti.

\subsubsection{Aṭṭha yathābhuccavaṇṇā}

\paragraph{302.} ‘‘Atha , bhante, brahmā sanaṅkumāro sakkaṃ devānamindaṃ etadavoca – ‘sādhu, devānaminda, mayampi tassa bhagavato aṭṭha yathābhucce vaṇṇe suṇeyyāmā’ti. ‘Evaṃ mahābrahme’ti kho, bhante, sakko devānamindo brahmuno sanaṅkumārassa bhagavato aṭṭha yathābhucce vaṇṇe payirudāhāsi.

‘‘Taṃ kiṃ maññati, bhavaṃ mahābrahmā? Yāvañca so bhagavā bahujanahitāya paṭipanno bahujanasukhāya lokānukampāya atthāya hitāya sukhāya devamanussānaṃ. Evaṃ bahujanahitāya paṭipannaṃ bahujanasukhāya lokānukampāya atthāya hitāya sukhāya devamanussānaṃ imināpaṅgena samannāgataṃ satthāraṃ neva atītaṃse samanupassāma, na panetarahi, aññatra tena bhagavatā.

‘‘Svākkhāto kho pana tena bhagavatā dhammo sandiṭṭhiko akāliko ehipassiko opaneyyiko paccattaṃ veditabbo viññūhi. Evaṃ opaneyyikassa dhammassa desetāraṃ imināpaṅgena samannāgataṃ satthāraṃ neva atītaṃse samanupassāma, na panetarahi, aññatra tena bhagavatā.

‘‘Idaṃ kusala’nti kho pana tena bhagavatā supaññattaṃ, ‘idaṃ akusala’nti supaññattaṃ, ‘idaṃ sāvajjaṃ idaṃ anavajjaṃ, idaṃ sevitabbaṃ idaṃ na sevitabbaṃ, idaṃ hīnaṃ idaṃ paṇītaṃ, idaṃ kaṇhasukkasappaṭibhāga’nti supaññattaṃ. Evaṃ kusalākusalasāvajjānavajjasevitabbāsevitabbahīnapaṇītakaṇhasukkasappaṭibhāgānaṃ dhammānaṃ paññāpetāraṃ. Imināpaṅgena samannāgataṃ satthāraṃ neva atītaṃse samanupassāma, na panetarahi, aññatra tena bhagavatā.

‘‘Supaññattā kho pana tena bhagavatā sāvakānaṃ nibbānagāminī paṭipadā saṃsandati nibbānañca paṭipadā ca. Seyyathāpi nāma gaṅgodakaṃ yamunodakena saṃsandati sameti, evameva supaññattā tena bhagavatā sāvakānaṃ nibbānagāminī paṭipadā saṃsandati nibbānañca paṭipadā ca. Evaṃ nibbānagāminiyā paṭipadāya paññāpetāraṃ imināpaṅgena samannāgataṃ satthāraṃ neva atītaṃse samanupassāma, na panetarahi, aññatra tena bhagavatā.

‘‘Abhinipphanno kho pana tassa bhagavato lābho abhinipphanno siloko, yāva maññe khattiyā sampiyāyamānarūpā viharanti. Vigatamado kho pana so bhagavā āhāraṃ āhāreti. Evaṃ vigatamadaṃ āhāraṃ āharayamānaṃ imināpaṅgena samannāgataṃ satthāraṃ neva atītaṃse samanupassāma, na panetarahi, aññatra tena bhagavatā.

‘‘Laddhasahāyo kho pana so bhagavā sekhānañceva paṭipannānaṃ khīṇāsavānañca vusitavataṃ, te bhagavā apanujja ekārāmataṃ anuyutto viharati. Evaṃ ekārāmataṃ anuyuttaṃ imināpaṅgena samannāgataṃ satthāraṃ neva atītaṃse samanupassāma, na panetarahi, aññatra tena bhagavatā.

‘‘Yathāvādī kho pana so bhagavā tathākārī, yathākārī tathāvādī; iti yathāvādī tathākārī, yathākārī tathāvādī. Evaṃ dhammānudhammappaṭippannaṃ imināpaṅgena samannāgataṃ satthāraṃ neva atītaṃse samanupassāma, na panetarahi, aññatra tena bhagavatā.

‘‘Tiṇṇavicikiccho kho pana so bhagavā vigatakathaṃkatho pariyositasaṅkappo ajjhāsayaṃ ādibrahmacariyaṃ . Evaṃ tiṇṇavicikicchaṃ vigatakathaṃkathaṃ pariyositasaṅkappaṃ ajjhāsayaṃ ādibrahmacariyaṃ. Imināpaṅgena samannāgataṃ satthāraṃ neva atītaṃse samanupassāma, na panetarahi, aññatra tena bhagavatā’ti.

\paragraph{303.} ‘‘Ime kho, bhante, sakko devānamindo brahmuno sanaṅkumārassa bhagavato aṭṭha yathābhucce vaṇṇe payirudāhāsi. Tena sudaṃ, bhante, brahmā sanaṅkumāro attamano hoti pamudito pītisomanassajāto bhagavato aṭṭha yathābhucce vaṇṇe sutvā. Atha, bhante, brahmā sanaṅkumāro oḷārikaṃ attabhāvaṃ abhinimminitvā kumāravaṇṇī hutvā pañcasikho devānaṃ tāvatiṃsānaṃ pāturahosi . So vehāsaṃ abbhuggantvā ākāse antalikkhe pallaṅkena nisīdi. Seyyathāpi, bhante, balavā puriso supaccatthate vā pallaṅke same vā bhūmibhāge pallaṅkena nisīdeyya, evameva kho, bhante, brahmā sanaṅkumāro vehāsaṃ abbhuggantvā ākāse antalikkhe pallaṅkena nisīditvā deve tāvatiṃse āmantesi –

\subsubsection{Govindabrāhmaṇavatthu}

\paragraph{304.} ‘‘Taṃ kiṃ maññanti, bhonto devā tāvatiṃsā, yāva dīgharattaṃ mahāpaññova so bhagavā ahosi. Bhūtapubbaṃ, bho, rājā disampati nāma ahosi. Disampatissa rañño govindo nāma brāhmaṇo purohito ahosi. Disampatissa rañño reṇu nāma kumāro putto ahosi. Govindassa brāhmaṇassa jotipālo nāma māṇavo putto ahosi. Iti reṇu ca rājaputto jotipālo ca māṇavo aññe ca cha khattiyā iccete aṭṭha sahāyā ahesuṃ. Atha kho, bho, ahorattānaṃ accayena govindo brāhmaṇo kālamakāsi. Govinde brāhmaṇe kālaṅkate rājā disampati paridevesi – ‘‘yasmiṃ vata, bho, mayaṃ samaye govinde brāhmaṇe sabbakiccāni sammā vossajjitvā pañcahi kāmaguṇehi samappitā samaṅgībhūtā paricārema, tasmiṃ no samaye govindo brāhmaṇo kālaṅkato’’ti. Evaṃ vutte bho reṇu rājaputto rājānaṃ disampatiṃ etadavoca – ‘‘mā kho tvaṃ, deva, govinde brāhmaṇe kālaṅkate atibāḷhaṃ paridevesi. Atthi, deva, govindassa brāhmaṇassa jotipālo nāma māṇavo putto paṇḍitataro ceva pitarā, alamatthadasataro ceva pitarā; yepissa pitā atthe anusāsi, tepi jotipālasseva māṇavassa anusāsaniyā’’ti. ‘‘Evaṃ kumārā’’ti? ‘‘Evaṃ devā’’ti.

\subsubsection{Mahāgovindavatthu}

\paragraph{305.} ‘‘Atha kho, bho, rājā disampati aññataraṃ purisaṃ āmantesi – ‘‘ehi tvaṃ, ambho purisa, yena jotipālo nāma māṇavo tenupasaṅkama; upasaṅkamitvā jotipālaṃ māṇavaṃ evaṃ vadehi – ‘bhavamatthu bhavantaṃ jotipālaṃ, rājā disampati bhavantaṃ jotipālaṃ māṇavaṃ āmantayati, rājā disampati bhoto jotipālassa māṇavassa dassanakāmo’’’ti. ‘‘Evaṃ, devā’’ti kho, bho, so puriso disampatissa rañño paṭissutvā yena jotipālo māṇavo tenupasaṅkami; upasaṅkamitvā jotipālaṃ māṇavaṃ etadavoca – ‘‘bhavamatthu bhavantaṃ jotipālaṃ, rājā disampati bhavantaṃ jotipālaṃ māṇavaṃ āmantayati , rājā disampati bhoto jotipālassa māṇavassa dassanakāmo’’ti. ‘‘Evaṃ, bho’’ti kho bho jotipālo māṇavo tassa purisassa paṭissutvā yena rājā disampati tenupasaṅkami; upasaṅkamitvā disampatinā raññā saddhiṃ sammodi; sammodanīyaṃ kathaṃ sāraṇīyaṃ vītisāretvā ekamantaṃ nisīdi. Ekamantaṃ nisinnaṃ kho, bho, jotipālaṃ māṇavaṃ rājā disampati etadavoca – ‘‘anusāsatu no bhavaṃ jotipālo, mā no bhavaṃ jotipālo anusāsaniyā paccabyāhāsi. Pettike taṃ ṭhāne ṭhapessāmi, govindiye abhisiñcissāmī’’ti. ‘‘Evaṃ, bho’’ti kho, bho, so jotipālo māṇavo disampatissa rañño paccassosi. Atha kho, bho, rājā disampati jotipālaṃ māṇavaṃ govindiye abhisiñci, taṃ pettike ṭhāne ṭhapesi. Abhisitto jotipālo māṇavo govindiye pettike ṭhāne ṭhapito yepissa pitā atthe anusāsi tepi atthe anusāsati, yepissa pitā atthe nānusāsi, tepi atthe anusāsati; yepissa pitā kammante abhisambhosi, tepi kammante abhisambhoti, yepissa pitā kammante nābhisambhosi, tepi kammante abhisambhoti. Tamenaṃ manussā evamāhaṃsu – ‘‘govindo vata, bho, brāhmaṇo, mahāgovindo vata, bho, brāhmaṇo’’ti. Iminā kho evaṃ, bho, pariyāyena jotipālassa māṇavassa govindo mahāgovindotveva samaññā udapādi.

\subsubsection{Rajjasaṃvibhajanaṃ}

\paragraph{306.} ‘‘Atha kho, bho, mahāgovindo brāhmaṇo yena te cha khattiyā tenupasaṅkami; upasaṅkamitvā te cha khattiye etadavoca – ‘‘disampati kho, bho, rājā jiṇṇo vuddho mahallako addhagato vayoanuppatto, ko nu kho pana, bho, jānāti jīvitaṃ? Ṭhānaṃ kho panetaṃ vijjati, yaṃ disampatimhi raññe kālaṅkate rājakattāro reṇuṃ rājaputtaṃ rajje abhisiñceyyuṃ. Āyantu, bhonto, yena reṇu rājaputto tenupasaṅkamatha; upasaṅkamitvā reṇuṃ rājaputtaṃ evaṃ vadetha – ‘‘mayaṃ kho bhoto reṇussa sahāyā piyā manāpā appaṭikūlā, yaṃsukho bhavaṃ taṃsukhā mayaṃ, yaṃdukkho bhavaṃ taṃdukkhā mayaṃ. Disampati kho, bho, rājā jiṇṇo vuddho mahallako addhagato vayoanuppatto, ko nu kho pana, bho, jānāti jīvitaṃ? Ṭhānaṃ kho panetaṃ vijjati, yaṃ disampatimhi raññe kālaṅkate rājakattāro bhavantaṃ reṇuṃ rajje abhisiñceyyuṃ. Sace bhavaṃ reṇu rajjaṃ labhetha, saṃvibhajetha no rajjenā’’ti. ‘‘Evaṃ bho’’ti kho, bho, te cha khattiyā mahāgovindassa brāhmaṇassa paṭissutvā yena reṇu rājaputto tenupasaṅkamiṃsu; upasaṅkamitvā reṇuṃ rājaputtaṃ etadavocuṃ – ‘‘mayaṃ kho bhoto reṇussa sahāyā piyā manāpā appaṭikūlā ; yaṃsukho bhavaṃ taṃsukhā mayaṃ, yaṃdukkho bhavaṃ taṃdukkhā mayaṃ. Disampati kho, bho, rājā jiṇṇo vuddho mahallako addhagato vayoanuppatto, ko nu kho pana bho jānāti jīvitaṃ? Ṭhānaṃ kho panetaṃ vijjati, yaṃ disampatimhi raññe kālaṅkate rājakattāro bhavantaṃ reṇuṃ rajje abhisiñceyyuṃ. Sace bhavaṃ reṇu rajjaṃ labhetha, saṃvibhajetha no rajjenā’’ti. ‘‘Ko nu kho, bho, añño mama vijite sukho bhavetha\footnote{sukhā bhaveyyātha (ka.), sukhaṃ bhaveyyātha, sukhamedheyyātha (sī. pī.),sukha medhetha (?)}, aññatra bhavantebhi? Sacāhaṃ, bho, rajjaṃ labhissāmi, saṃvibhajissāmi vo rajjenā’’’ti.

\paragraph{307.} ‘‘Atha kho, bho, ahorattānaṃ accayena rājā disampati kālamakāsi. Disampatimhi raññe kālaṅkate rājakattāro reṇuṃ rājaputtaṃ rajje abhisiñciṃsu. Abhisitto reṇu rajjena pañcahi kāmaguṇehi samappito samaṅgībhūto paricāreti. Atha kho, bho, mahāgovindo brāhmaṇo yena te cha khattiyā tenupasaṅkami; upasaṅkamitvā te cha khattiye etadavoca – ‘‘disampati kho, bho, rājā kālaṅkato. Abhisitto reṇu rajjena pañcahi kāmaguṇehi samappito samaṅgībhūto paricāreti. Ko nu kho pana, bho, jānāti, madanīyā kāmā? Āyantu, bhonto, yena reṇu rājā tenupasaṅkamatha; upasaṅkamitvā reṇuṃ rājānaṃ evaṃ vadetha – disampati kho, bho, rājā kālaṅkato, abhisitto bhavaṃ reṇu rajjena, sarati bhavaṃ taṃ vacana’’’nti?

\paragraph{308.} ‘‘‘Evaṃ , bho’’ti kho, bho, te cha khattiyā mahāgovindassa brāhmaṇassa paṭissutvā yena reṇu rājā tenupasaṅkamiṃsu; upasaṅkamitvā reṇuṃ rājānaṃ etadavocuṃ – ‘‘disampati kho, bho, rājā kālaṅkato, abhisitto bhavaṃ reṇu rajjena, sarati bhavaṃ taṃ vacana’’nti? ‘‘Sarāmahaṃ, bho, taṃ vacanaṃ\footnote{vacananti (syā. ka.)}. Ko nu kho, bho, pahoti imaṃ mahāpathaviṃ uttarena āyataṃ dakkhiṇena sakaṭamukhaṃ sattadhā samaṃ suvibhattaṃ vibhajitu’’nti? ‘‘Ko nu kho, bho, añño pahoti, aññatra mahāgovindena brāhmaṇenā’’ti? Atha kho, bho, reṇu rājā aññataraṃ purisaṃ āmantesi – ‘‘ehi tvaṃ, ambho purisa, yena mahāgovindo brāhmaṇo tenupasaṅkama; upasaṅkamitvā mahāgovindaṃ brāhmaṇaṃ evaṃ vadehi – ‘rājā taṃ, bhante, reṇu āmantetī’’’ti. ‘‘Evaṃ devā’’ti kho, bho, so puriso reṇussa rañño paṭissutvā yena mahāgovindo brāhmaṇo tenupasaṅkami; upasaṅkamitvā mahāgovindaṃ brāhmaṇaṃ etadavoca – ‘‘rājā taṃ, bhante, reṇu āmantetī’’ti. ‘‘Evaṃ, bho’’ti kho, bho, mahāgovindo brāhmaṇo tassa purisassa paṭissutvā yena reṇu rājā tenupasaṅkami; upasaṅkamitvā reṇunā raññā saddhiṃ sammodi. Sammodanīyaṃ kathaṃ sāraṇīyaṃ vītisāretvā ekamantaṃ nisīdi. Ekamantaṃ nisinnaṃ kho, bho, mahāgovindaṃ brāhmaṇaṃ reṇu rājā etadavoca – ‘‘etu, bhavaṃ govindo, imaṃ mahāpathaviṃ uttarena āyataṃ dakkhiṇena sakaṭamukhaṃ sattadhā samaṃ suvibhattaṃ vibhajatū’’ti. ‘‘Evaṃ, bho’’ti kho mahāgovindo brāhmaṇo reṇussa rañño paṭissutvā imaṃ mahāpathaviṃ uttarena āyataṃ dakkhiṇena sakaṭamukhaṃ sattadhā samaṃ suvibhattaṃ vibhaji. Sabbāni sakaṭamukhāni paṭṭhapesi\footnote{aṭṭhapesi (sī. pī.)}. Tatra sudaṃ majjhe reṇussa rañño janapado hoti.

\paragraph{309.} Dantapuraṃ kaliṅgānaṃ\footnote{kāliṅgānaṃ (syā. pī. ka.)}, assakānañca potanaṃ.

Mahesayaṃ\footnote{māhissati (sī. syā. pī.)} avantīnaṃ, sovīrānañca rorukaṃ.

Mithilā ca videhānaṃ, campā aṅgesu māpitā;

Bārāṇasī ca kāsīnaṃ, ete govindamāpitāti.

\paragraph{310.} ‘‘Atha kho, bho, te cha khattiyā yathāsakena lābhena attamanā ahesuṃ paripuṇṇasaṅkappā – ‘‘yaṃ vata no ahosi icchitaṃ, yaṃ ākaṅkhitaṃ, yaṃ adhippetaṃ, yaṃ abhipatthitaṃ, taṃ no laddha’’nti.

‘‘Sattabhū brahmadatto ca, vessabhū bharato saha;

Reṇu dve dhataraṭṭhā ca, tadāsuṃ satta bhāradhā’ti.

\xsubsubsectionEnd{Paṭhamabhāṇavāro niṭṭhito.}

\subsubsection{Kittisaddaabbhuggamanaṃ}

\paragraph{311.} ‘‘Atha kho, bho, te cha khattiyā yena mahāgovindo brāhmaṇo tenupasaṅkamiṃsu; upasaṅkamitvā mahāgovindaṃ brāhmaṇaṃ etadavocuṃ – ‘‘yathā kho bhavaṃ govindo reṇussa rañño sahāyo piyo manāpo appaṭikūlo. Evameva kho bhavaṃ govindo amhākampi sahāyo piyo manāpo appaṭikūlo, anusāsatu no bhavaṃ govindo; mā no bhavaṃ govindo anusāsaniyā paccabyāhāsī’’ti. ‘‘Evaṃ, bho’’ti kho mahāgovindo brāhmaṇo tesaṃ channaṃ khattiyānaṃ paccassosi. Atha kho, bho, mahāgovindo brāhmaṇo satta ca rājāno khattiye muddhāvasitte rajje\footnote{muddhābhisitte rajjena (syā.)} anusāsi, satta ca brāhmaṇamahāsāle satta ca nhātakasatāni mante vācesi.

\paragraph{312.} ‘‘Atha kho, bho, mahāgovindassa brāhmaṇassa aparena samayena evaṃ kalyāṇo kittisaddo abbhuggacchi\footnote{abbhuggañchi (sī. pī.)} – ‘‘sakkhi mahāgovindo brāhmaṇo brahmānaṃ passati, sakkhi mahāgovindo brāhmaṇo brahmunā sākaccheti sallapati mantetī’’ti. Atha kho, bho, mahāgovindassa brāhmaṇassa etadahosi – ‘‘mayhaṃ kho evaṃ kalyāṇo kittisaddo abbhuggato – ‘sakkhi mahāgovindo brāhmaṇo brahmānaṃ passati, sakkhi mahāgovindo brāhmaṇo brahmunā sākaccheti sallapati mantetī’ti. Na kho panāhaṃ brahmānaṃ passāmi, na brahmunā sākacchemi, na brahmunā sallapāmi , na brahmunā mantemi. Sutaṃ kho pana metaṃ brāhmaṇānaṃ vuddhānaṃ mahallakānaṃ ācariyapācariyānaṃ bhāsamānānaṃ – ‘yo vassike cattāro māse paṭisallīyati, karuṇaṃ jhānaṃ jhāyati, so brahmānaṃ passati brahmunā sākaccheti brahmunā sallapati brahmunā mantetī’ti. Yaṃnūnāhaṃ vassike cattāro māse paṭisallīyeyyaṃ, karuṇaṃ jhānaṃ jhāyeyya’’nti.

\paragraph{313.} ‘‘Atha kho, bho, mahāgovindo brāhmaṇo yena reṇu rājā tenupasaṅkami; upasaṅkamitvā reṇuṃ rājānaṃ etadavoca – ‘‘mayhaṃ kho, bho, evaṃ kalyāṇo kittisaddo abbhuggato – ‘sakkhi mahāgovindo brāhmaṇo brahmānaṃ passati, sakkhi mahāgovindo brāhmaṇo brahmunā sākaccheti sallapati mantetī’ti. Na kho panāhaṃ, bho, brahmānaṃ passāmi, na brahmunā sākacchemi, na brahmunā sallapāmi, na brahmunā mantemi. Sutaṃ kho pana metaṃ brāhmaṇānaṃ vuddhānaṃ mahallakānaṃ ācariyapācariyānaṃ bhāsamānānaṃ – ‘yo vassike cattāro māse paṭisallīyati, karuṇaṃ jhānaṃ jhāyati, so brahmānaṃ passati, brahmunā sākaccheti brahmunā sallapati brahmunā mantetī’ti. Icchāmahaṃ, bho, vassike cattāro māse paṭisallīyituṃ, karuṇaṃ jhānaṃ jhāyituṃ; namhi kenaci upasaṅkamitabbo aññatra ekena bhattābhihārenā’’ti. ‘‘Yassadāni bhavaṃ govindo kālaṃ maññatī’’ti.

\paragraph{314.} ‘‘Atha kho, bho, mahāgovindo brāhmaṇo yena te cha khattiyā tenupasaṅkami; upasaṅkamitvā te cha khattiye etadavoca – ‘‘mayhaṃ kho, bho, evaṃ kalyāṇo kittisaddo abbhuggato – ‘sakkhi mahāgovindo brāhmaṇo brahmānaṃ passati, sakkhi mahāgovindo brāhmaṇo brahmunā sākaccheti sallapati mantetī’ti. Na kho panāhaṃ, bho, brahmānaṃ passāmi, na brahmunā sākacchemi, na brahmunā sallapāmi, na brahmunā mantemi. Sutaṃ kho pana metaṃ brāhmaṇānaṃ vuddhānaṃ mahallakānaṃ ācariyapācariyānaṃ bhāsamānānaṃ, ‘yo vassike cattāro māse paṭisallīyati, karuṇaṃ jhānaṃ jhāyati, so brahmānaṃ passati brahmunā sākaccheti brahmunā sallapati brahmunā mantetī’ti. Icchāmahaṃ, bho, vassike cattāro māse paṭisallīyituṃ, karuṇaṃ jhānaṃ jhāyituṃ; namhi kenaci upasaṅkamitabbo aññatra ekena bhattābhihārenā’’ti. ‘‘Yassadāni bhavaṃ govindo kālaṃ maññatī’’’ti.

\paragraph{315.} ‘‘Atha kho, bho, mahāgovindo brāhmaṇo yena te satta ca brāhmaṇamahāsālā satta ca nhātakasatāni tenupasaṅkami; upasaṅkamitvā te satta ca brāhmaṇamahāsāle satta ca nhātakasatāni etadavoca – ‘‘mayhaṃ kho, bho, evaṃ kalyāṇo kittisaddo abbhuggato – ‘sakkhi mahāgovindo brāhmaṇo brahmānaṃ passati, sakkhi mahāgovindo brāhmaṇo brahmunā sākaccheti sallapati mantetī’ti. Na kho panāhaṃ, bho, brahmānaṃ passāmi, na brahmunā sākacchemi, na brahmunā sallapāmi, na brahmunā mantemi. Sutaṃ kho pana metaṃ brāhmaṇānaṃ vuddhānaṃ mahallakānaṃ ācariyapācariyānaṃ bhāsamānānaṃ – ‘yo vassike cattāro māse paṭisallīyati, karuṇaṃ jhānaṃ jhāyati, so brahmānaṃ passati, brahmunā sākaccheti, brahmunā sallapati, brahmunā mantetī’ti. Tena hi, bho, yathāsute yathāpariyatte mante vitthārena sajjhāyaṃ karotha, aññamaññañca mante vācetha; icchāmahaṃ, bho, vassike cattāro māse paṭisallīyituṃ, karuṇaṃ jhānaṃ jhāyituṃ; namhi kenaci upasaṅkamitabbo aññatra ekena bhattābhihārenā’’ti. ‘‘Yassa dāni bhavaṃ govindo kālaṃ maññatī’’ti.

\paragraph{316.} ‘‘Atha kho, bho, mahāgovindo brāhmaṇo yena cattārīsā bhariyā sādisiyo tenupasaṅkami; upasaṅkamitvā cattārīsā bhariyā sādisiyo etadavoca – ‘‘mayhaṃ kho, bhotī, evaṃ kalyāṇo kittisaddo abbhuggato – ‘sakkhi mahāgovindo brāhmaṇo brahmānaṃ passati, sakkhi mahāgovindo brāhmaṇo brahmunā sākaccheti sallapati mantetī’ti. Na kho panāhaṃ, bhotī, brahmānaṃ passāmi, na brahmunā sākacchemi, na brahmunā sallapāmi, na brahmunā mantemi. Sutaṃ kho pana metaṃ brāhmaṇānaṃ vuddhānaṃ mahallakānaṃ ācariyapācariyānaṃ bhāsamānānaṃ ‘yo vassike cattāro māse paṭisallīyati, karuṇaṃ jhānaṃ jhāyati, so brahmānaṃ passati, brahmunā sākaccheti, brahmunā sallapati, brahmunā mantetīti, icchāmahaṃ, bhotī, vassike cattāro māse paṭisallīyituṃ, karuṇaṃ jhānaṃ jhāyituṃ; namhi kenaci upasaṅkamitabbo aññatra ekena bhattābhihārenā’’ti. ‘‘Yassa dāni bhavaṃ govindo kālaṃ maññatī’’’ti.

\paragraph{317.} ‘‘Atha kho, bho, mahāgovindo brāhmaṇo puratthimena nagarassa navaṃ sandhāgāraṃ kārāpetvā vassike cattāro māse paṭisallīyi, karuṇaṃ jhānaṃ jhāyi; nāssudha koci upasaṅkamati\footnote{upasaṅkami (pī.)} aññatra ekena bhattābhihārena. Atha kho, bho, mahāgovindassa brāhmaṇassa catunnaṃ māsānaṃ accayena ahudeva ukkaṇṭhanā ahu paritassanā – ‘‘sutaṃ kho pana metaṃ brāhmaṇānaṃ vuddhānaṃ mahallakānaṃ ācariyapācariyānaṃ bhāsamānānaṃ – ‘yo vassike cattāro māse paṭisallīyati, karuṇaṃ jhānaṃ jhāyati, so brahmānaṃ passati, brahmunā sākaccheti brahmunā sallapati brahmunā mantetī’ti. Na kho panāhaṃ brahmānaṃ passāmi, na brahmunā sākacchemi na brahmunā sallapāmi na brahmunā mantemī’’’ti.

\subsubsection{Brahmunā sākacchā}

\paragraph{318.} ‘‘Atha kho, bho, brahmā sanaṅkumāro mahāgovindassa brāhmaṇassa cetasā cetoparivitakkamaññāya seyyathāpi nāma balavā puriso samiñjitaṃ vā bāhaṃ pasāreyya, pasāritaṃ vā bāhaṃ samiñjeyya, evameva, brahmaloke antarahito mahāgovindassa brāhmaṇassa sammukhe pāturahosi. Atha kho, bho, mahāgovindassa brāhmaṇassa ahudeva bhayaṃ ahu chambhitattaṃ ahu lomahaṃso yathā taṃ adiṭṭhapubbaṃ rūpaṃ disvā. Atha kho, bho, mahāgovindo brāhmaṇo bhīto saṃviggo lomahaṭṭhajāto brahmānaṃ sanaṅkumāraṃ gāthāya ajjhabhāsi –

‘‘‘Vaṇṇavā yasavā sirimā, ko nu tvamasi mārisa;

Ajānantā taṃ pucchāma, kathaṃ jānemu taṃ maya’’nti.

‘‘Maṃ ve kumāraṃ jānanti, brahmaloke sanantanaṃ\footnote{sanantica (ka.)};

Sabbe jānanti maṃ devā, evaṃ govinda jānahi’’.

‘‘‘Āsanaṃ udakaṃ pajjaṃ, madhusākañca\footnote{madhupākañca (sī. syā. pī.)} brahmuno;

Agghe bhavantaṃ pucchāma, agghaṃ kurutu no bhavaṃ’’.

‘‘Paṭiggaṇhāma te agghaṃ, yaṃ tvaṃ govinda bhāsasi;

Diṭṭhadhammahitatthāya, samparāya sukhāya ca;

Katāvakāso pucchassu, yaṃ kiñci abhipatthita’’nti.

\paragraph{319.} ‘‘Atha kho, bho, mahāgovindassa brāhmaṇassa etadahosi – ‘‘katāvakāso khomhi brahmunā sanaṅkumārena. Kiṃ nu kho ahaṃ brahmānaṃ sanaṅkumāraṃ puccheyyaṃ diṭṭhadhammikaṃ vā atthaṃ samparāyikaṃ vā’ti? Atha kho, bho, mahāgovindassa brāhmaṇassa etadahosi – ‘kusalo kho ahaṃ diṭṭhadhammikānaṃ atthānaṃ, aññepi maṃ diṭṭhadhammikaṃ atthaṃ pucchanti. Yaṃnūnāhaṃ brahmānaṃ sanaṅkumāraṃ samparāyikaññeva atthaṃ puccheyya’nti. Atha kho, bho, mahāgovindo brāhmaṇo brahmānaṃ sanaṅkumāraṃ gāthāya ajjhabhāsi –

‘‘Pucchāmi brahmānaṃ sanaṅkumāraṃ,

Kaṅkhī akaṅkhiṃ paravediyesu;

Katthaṭṭhito kimhi ca sikkhamāno,

Pappoti macco amataṃ brahmaloka’’nti.

‘‘Hitvā mamattaṃ manujesu brahme,

Ekodibhūto karuṇedhimutto\footnote{karuṇādhimutto (sī. syā. pī.)};

Nirāmagandho virato methunasmā,

Etthaṭṭhito ettha ca sikkhamāno;

Pappoti macco amataṃ brahmaloka’’nti.

\paragraph{320.} ‘‘Hitvā mamatta’nti ahaṃ bhoto ājānāmi. Idhekacco appaṃ vā bhogakkhandhaṃ pahāya mahantaṃ vā bhogakkhandhaṃ pahāya appaṃ vā ñātiparivaṭṭaṃ pahāya mahantaṃ vā ñātiparivaṭṭaṃ pahāya kesamassuṃ ohāretvā kāsāyāni vatthāni acchādetvā agārasmā anagāriyaṃ pabbajati, ‘iti hitvā mamatta’nti ahaṃ bhoto ājānāmi. ‘Ekodibhūto’ti ahaṃ bhoto ājānāmi. Idhekacco vivittaṃ senāsanaṃ bhajati araññaṃ rukkhamūlaṃ pabbataṃ kandaraṃ giriguhaṃ susānaṃ vanapatthaṃ abbhokāsaṃ palālapuñjaṃ, iti ekodibhūto’ti ahaṃ bhoto ājānāmi. ‘Karuṇedhimutto’ti ahaṃ bhoto ājānāmi. Idhekacco karuṇāsahagatena cetasā ekaṃ disaṃ pharitvā viharati, tathā dutiyaṃ, tathā tatiyaṃ, tathā catutthaṃ. Iti uddhamadhotiriyaṃ sabbadhi sabbattatāya sabbāvantaṃ lokaṃ karuṇāsahagatena cetasā vipulena mahaggatena appamāṇena averena abyāpajjena pharitvā viharati. Iti ‘karuṇedhimutto’ti ahaṃ bhoto ājānāmi. Āmagandhe ca kho ahaṃ bhoto bhāsamānassa na ājānāmi.

‘‘Ke āmagandhā manujesu brahme,

Ete avidvā idha brūhi dhīra;

Kenāvaṭā\footnote{kenāvuṭā (syā.)} vāti pajā kurutu\footnote{kururū (syā.), kuruṭṭharū (pī.), kurūru (?)},

Āpāyikā nivutabrahmalokā’’ti.

‘‘Kodho mosavajjaṃ nikati ca dubbho,

Kadariyatā atimāno usūyā;

Icchā vivicchā paraheṭhanā ca,

Lobho ca doso ca mado ca moho;

Etesu yuttā anirāmagandhā,

Āpāyikā nivutabrahmalokā’’ti.

‘‘Yathā kho ahaṃ bhoto āmagandhe bhāsamānassa ājānāmi. Te na sunimmadayā agāraṃ ajjhāvasatā. Pabbajissāmahaṃ, bho, agārasmā anagāriya’’nti. ‘‘Yassadāni bhavaṃ govindo kālaṃ maññatī’’ti.

\subsubsection{Reṇurājaāmantanā}

\paragraph{321.} ‘‘Atha kho, bho, mahāgovindo brāhmaṇo yena reṇu rājā tenupasaṅkami; upasaṅkamitvā reṇuṃ rājānaṃ etadavoca – ‘‘aññaṃ dāni bhavaṃ purohitaṃ pariyesatu, yo bhoto rajjaṃ anusāsissati. Icchāmahaṃ, bho , agārasmā anagāriyaṃ pabbajituṃ. Yathā kho pana me sutaṃ brahmuno āmagandhe bhāsamānassa, te na sunimmadayā agāraṃ ajjhāvasatā. Pabbajissāmahaṃ, bho, agārasmā anagāriya’’nti.

‘‘Āmantayāmi rājānaṃ, reṇuṃ bhūmipatiṃ ahaṃ;

Tvaṃ pajānassu rajjena, nāhaṃ porohicce rame’’.

‘‘Sace te ūnaṃ kāmehi, ahaṃ paripūrayāmi te;

Yo taṃ hiṃsati vāremi, bhūmisenāpati ahaṃ;

Tuvaṃ pitā ahaṃ putto, mā no govinda pājahi’’\footnote{pājehi (aṭṭhakathāyaṃ saṃvaṇṇitapāṭhantaraṃ)}.

‘‘Namatthi ūnaṃ kāmehi, hiṃsitā me na vijjati;

Amanussavaco sutvā, tasmāhaṃ na gahe rame’’.

‘‘Amanusso kathaṃvaṇṇo, kiṃ te atthaṃ abhāsatha;

Yañca sutvā jahāsi no, gehe amhe ca kevalī’’.

‘‘Upavutthassa me pubbe, yiṭṭhukāmassa me sato;

Aggi pajjalito āsi, kusapattaparitthato’’.

‘‘Tato me brahmā pāturahu, brahmalokā sanantano;

So me pañhaṃ viyākāsi, taṃ sutvā na gahe rame’’.

‘‘Saddahāmi ahaṃ bhoto, yaṃ tvaṃ govinda bhāsasi;

Amanussavaco sutvā, kathaṃ vattetha aññathā.

‘‘Te taṃ anuvattissāma, satthā govinda no bhavaṃ;

Maṇi yathā veḷuriyo, akāco vimalo subho;

Evaṃ suddhā carissāma, govindassānusāsane’’ti.

‘‘‘Sace bhavaṃ govindo agārasmā anagāriyaṃ pabbajissati, mayampi agārasmā anagāriyaṃ pabbajissāma. Atha yā te gati, sā no gati bhavissatī’’ti.

\subsubsection{Cha khattiyaāmantanā}

\paragraph{322.} ‘‘Atha kho, bho, mahāgovindo brāhmaṇo yena te cha khattiyā tenupasaṅkami; upasaṅkamitvā te cha khattiye etadavoca – ‘‘aññaṃ dāni bhavanto purohitaṃ pariyesantu, yo bhavantānaṃ rajje anusāsissati. Icchāmahaṃ, bho, agārasmā anagāriyaṃ pabbajituṃ. Yathā kho pana me sutaṃ brahmuno āmagandhe bhāsamānassa, te na sunimmadayā agāraṃ ajjhāvasatā. Pabbajissāmahaṃ, bho, agārasmā anagāriya’’nti. Atha kho, bho, te cha khattiyā ekamantaṃ apakkamma evaṃ samacintesuṃ – ‘‘ime kho brāhmaṇā nāma dhanaluddhā; yaṃnūna mayaṃ mahāgovindaṃ brāhmaṇaṃ dhanena sikkheyyāmā’’ti. Te mahāgovindaṃ brāhmaṇaṃ upasaṅkamitvā evamāhaṃsu – ‘‘saṃvijjati kho, bho, imesu sattasu rajjesu pahūtaṃ sāpateyyaṃ, tato bhoto yāvatakena attho, tāvatakaṃ āharīyata’’nti. ‘‘Alaṃ, bho, mamapidaṃ pahūtaṃ sāpateyyaṃ bhavantānaṃyeva vāhasā. Tamahaṃ sabbaṃ pahāya agārasmā anagāriyaṃ pabbajissāmi. Yathā kho pana me sutaṃ brahmuno āmagandhe bhāsamānassa, te na sunimmadayā agāraṃ ajjhāvasatā, pabbajissāmahaṃ, bho, agārasmā anagāriya’’nti. Atha kho, bho, te cha khattiyā ekamantaṃ apakkamma evaṃ samacintesuṃ – ‘‘ime kho brāhmaṇā nāma itthiluddhā; yaṃnūna mayaṃ mahāgovindaṃ brāhmaṇaṃ itthīhi sikkheyyāmā’’ti. Te mahāgovindaṃ brāhmaṇaṃ upasaṅkamitvā evamāhaṃsu – ‘‘saṃvijjanti kho, bho, imesu sattasu rajjesu pahūtā itthiyo, tato bhoto yāvatikāhi attho, tāvatikā ānīyata’’nti. ‘‘Alaṃ, bho, mamapimā\footnote{mamapitā (ka.), mamapi (sī.)} cattārīsā bhariyā sādisiyo. Tāpāhaṃ sabbā pahāya agārasmā anagāriyaṃ pabbajissāmi. Yathā kho pana me sutaṃ brahmuno āmagandhe bhāsamānassa, te na sunimmadayā agāraṃ ajjhāvasatā, pabbajissāmahaṃ, bho, agārasmā anagāriyanti’’.

\paragraph{323.} ‘‘Sace bhavaṃ govindo agārasmā anagāriyaṃ pabbajissati, mayampi agārasmā anagāriyaṃ pabbajissāma, atha yā te gati, sā no gati bhavissatīti.

‘‘Sace jahatha kāmāni, yattha satto puthujjano;

Ārambhavho daḷhā hotha, khantibalasamāhitā.

‘‘Esa maggo ujumaggo, esa maggo anuttaro;

Saddhammo sabbhi rakkhito, brahmalokūpapattiyāti.

‘‘Tena hi bhavaṃ govindo satta vassāni āgametu. Sattannaṃ vassānaṃ accayena mayampi agārasmā anagāriyaṃ pabbajissāma, atha yā te gati, sā no gati bhavissatī’’ti.

‘‘‘Aticiraṃ kho, bho, satta vassāni, nāhaṃ sakkomi, bhavante, satta vassāni āgametuṃ. Ko nu kho pana, bho, jānāti jīvitānaṃ! Gamanīyo samparāyo, mantāyaṃ\footnote{mantāya (bahūsu)} boddhabbaṃ, kattabbaṃ kusalaṃ, caritabbaṃ brahmacariyaṃ, natthi jātassa amaraṇaṃ. Yathā kho pana me sutaṃ brahmuno āmagandhe bhāsamānassa, te na sunimmadayā agāraṃ ajjhāvasatā, pabbajissāmahaṃ, bho, agārasmā anagāriya’’’nti. ‘‘Tena hi bhavaṃ govindo chabbassāni āgametu…pe… pañca vassāni āgametu… cattāri vassāni āgametu… tīṇi vassāni āgametu… dve vassāni āgametu… ekaṃ vassaṃ āgametu, ekassa vassassa accayena mayampi agārasmā anagāriyaṃ pabbajissāma, atha yā te gati, sā no gati bhavissatī’’ti.

‘‘‘Aticiraṃ kho, bho, ekaṃ vassaṃ, nāhaṃ sakkomi bhavante ekaṃ vassaṃ āgametuṃ. Ko nu kho pana, bho, jānāti jīvitānaṃ! Gamanīyo samparāyo, mantāyaṃ boddhabbaṃ, kattabbaṃ kusalaṃ , caritabbaṃ brahmacariyaṃ, natthi jātassa amaraṇaṃ. Yathā kho pana me sutaṃ brahmuno āmagandhe bhāsamānassa, te na sunimmadayā agāraṃ ajjhāvasatā, pabbajissāmahaṃ, bho, agārasmā anagāriya’’nti. ‘‘Tena hi bhavaṃ govindo satta māsāni āgametu, sattannaṃ māsānaṃ accayena mayampi agārasmā anagāriyaṃ pabbajissāma, atha yā te gati, sā no gati bhavissatī’’ti.

‘‘‘Aticiraṃ kho, bho, satta māsāni, nāhaṃ sakkomi bhavante satta māsāni āgametuṃ. Ko nu kho pana, bho, jānāti jīvitānaṃ. Gamanīyo samparāyo, mantāyaṃ boddhabbaṃ , kattabbaṃ kusalaṃ, caritabbaṃ brahmacariyaṃ, natthi jātassa amaraṇaṃ. Yathā kho pana me sutaṃ brahmuno āmagandhe bhāsamānassa, te na sunimmadayā agāraṃ ajjhāvasatā, pabbajissāmahaṃ, bho, agārasmā anagāriya’’nti.

‘‘‘Tena hi bhavaṃ govindo cha māsāni āgametu…pe… pañca māsāni āgametu… cattāri māsāni āgametu… tīṇi māsāni āgametu… dve māsāni āgametu… ekaṃ māsaṃ āgametu… addhamāsaṃ āgametu, addhamāsassa accayena mayampi agārasmā anagāriyaṃ pabbajissāma, atha yā te gati, sā no gati bhavissatī’’ti.

‘‘‘Aticiraṃ kho, bho, addhamāso, nāhaṃ sakkomi bhavante addhamāsaṃ āgametuṃ. Ko nu kho pana, bho, jānāti jīvitānaṃ! Gamanīyo samparāyo, mantāyaṃ boddhabbaṃ, kattabbaṃ kusalaṃ, caritabbaṃ brahmacariyaṃ, natthi jātassa amaraṇaṃ. Yathā kho pana me sutaṃ brahmuno āmagandhe bhāsamānassa, te na sunimmadayā agāraṃ ajjhāvasatā, pabbajissāmahaṃ, bho, agārasmā anagāriya’’nti. ‘‘Tena hi bhavaṃ govindo sattāhaṃ āgametu, yāva mayaṃ sake puttabhātaro rajjena\footnote{rajje (syā.)} anusāsissāma, sattāhassa accayena mayampi agārasmā anagāriyaṃ pabbajissāma, atha yā te gati, sā no gati bhavissatī’’ti. ‘‘Na ciraṃ kho, bho, sattāhaṃ, āgamessāmahaṃ bhavante sattāha’’nti.

\subsubsection{Brāhmaṇamahāsālādīnaṃ āmantanā}

\paragraph{324.} ‘‘Atha kho, bho, mahāgovindo brāhmaṇo yena te satta ca brāhmaṇamahāsālā satta ca nhātakasatāni tenupasaṅkami; upasaṅkamitvā te satta ca brāhmaṇamahāsāle satta ca nhātakasatāni etadavoca – ‘‘aññaṃ dāni bhavanto ācariyaṃ pariyesantu, yo bhavantānaṃ mante vācessati. Icchāmahaṃ, bho, agārasmā anagāriyaṃ pabbajituṃ. Yathā kho pana me sutaṃ brahmuno āmagandhe bhāsamānassa. Te na sunimmadayā agāraṃ ajjhāvasatā, pabbajissāmahaṃ, bho, agārasmā anagāriya’’nti. ‘‘Mā bhavaṃ govindo agārasmā anagāriyaṃ pabbaji. Pabbajjā, bho, appesakkhā ca appalābhā ca; brahmaññaṃ mahesakkhañca mahālābhañcā’’ti. ‘‘Mā bhavanto evaṃ avacuttha – ‘‘pabbajjā appesakkhā ca appalābhā ca, brahmaññaṃ mahesakkhañca mahālābhañcā’’ti. Ko nu kho, bho, aññatra mayā mahesakkhataro vā mahālābhataro vā! Ahañhi, bho, etarahi rājāva raññaṃ brahmāva brāhmaṇānaṃ\footnote{brahmānaṃ (sī. pī. ka.)} devatāva gahapatikānaṃ. Tamahaṃ sabbaṃ pahāya agārasmā anagāriyaṃ pabbajissāmi. Yathā kho pana me sutaṃ brahmuno āmagandhe bhāsamānassa, te na sunimmadayā agāraṃ ajjhāvasatā. Pabbajissāmahaṃ, bho, agārasmā anagāriya’’nti. ‘‘Sace bhavaṃ govindo agārasmā anagāriyaṃ pabbajissati, mayampi agārasmā anagāriyaṃ pabbajissāma, atha yā te gati, sā no gati bhavissatī’’ti.

\subsubsection{Bhariyānaṃ āmantanā}

\paragraph{325.} ‘‘Atha kho, bho, mahāgovindo brāhmaṇo yena cattārīsā bhariyā sādisiyo tenupasaṅkami; upasaṅkamitvā cattārīsā bhariyā sādisiyo etadavoca – ‘‘yā bhotīnaṃ icchati, sakāni vā ñātikulāni gacchatu aññaṃ vā bhattāraṃ pariyesatu. Icchāmahaṃ, bhotī, agārasmā anagāriyaṃ pabbajituṃ. Yathā kho pana me sutaṃ brahmuno āmagandhe bhāsamānassa, te na sunimmadayā agāraṃ ajjhāvasatā. Pabbajissāmahaṃ, bhotī, agārasmā anagāriya’’nti. ‘‘Tvaññeva no ñāti ñātikāmānaṃ, tvaṃ pana bhattā bhattukāmānaṃ. Sace bhavaṃ govindo agārasmā anagāriyaṃ pabbajissati, mayampi agārasmā anagāriyaṃ pabbajissāma, atha yā te gati, sā no gati bhavissatī’’ti.

\subsubsection{Mahāgovindapabbajjā}

\paragraph{326.} ‘‘Atha kho, bho, mahāgovindo brāhmaṇo tassa sattāhassa accayena kesamassuṃ ohāretvā kāsāyāni vatthāni acchādetvā agārasmā anagāriyaṃ pabbaji. Pabbajitaṃ pana mahāgovindaṃ brāhmaṇaṃ satta ca rājāno khattiyā muddhāvasittā satta ca brāhmaṇamahāsālā satta ca nhātakasatāni cattārīsā ca bhariyā sādisiyo anekāni ca khattiyasahassāni anekāni ca brāhmaṇasahassāni anekāni ca gahapatisahassāni anekehi ca itthāgārehi itthiyo kesamassuṃ ohāretvā kāsāyāni vatthāni acchādetvā mahāgovindaṃ brāhmaṇaṃ agārasmā anagāriyaṃ pabbajitaṃ anupabbajiṃsu. Tāya sudaṃ, bho, parisāya parivuto mahāgovindo brāhmaṇo gāmanigamarājadhānīsu cārikaṃ carati. Yaṃ kho pana, bho, tena samayena mahāgovindo brāhmaṇo gāmaṃ vā nigamaṃ vā upasaṅkamati, tattha rājāva hoti raññaṃ, brahmāva brāhmaṇānaṃ, devatāva gahapatikānaṃ. Tena kho pana samayena manussā khipanti vā upakkhalanti vā te evamāhaṃsu – ‘‘namatthu mahāgovindassa brāhmaṇassa, namatthu satta purohitassā’’’ti.

\paragraph{327.} ‘‘Mahāgovindo, bho, brāhmaṇo mettāsahagatena cetasā ekaṃ disaṃ pharitvā vihāsi, tathā dutiyaṃ, tathā tatiyaṃ, tathā catutthaṃ. Iti uddhamadho tiriyaṃ sabbadhi sabbattatāya sabbāvantaṃ lokaṃ mettāsahagatena cetasā vipulena mahaggatena appamāṇena averena abyāpajjena pharitvā vihāsi. Karuṇāsahagatena cetasā…pe… muditāsahagatena cetasā…pe… upekkhāsahagatena cetasā…pe… abyāpajjena pharitvā vihāsi sāvakānañca brahmalokasahabyatāya maggaṃ desesi.

\paragraph{328.} ‘‘Ye kho pana, bho, tena samayena mahāgovindassa brāhmaṇassa sāvakā sabbena sabbaṃ sāsanaṃ ājāniṃsu. Te kāyassa bhedā paraṃ maraṇā sugatiṃ brahmalokaṃ upapajjiṃsu. Ye na sabbena sabbaṃ sāsanaṃ ājāniṃsu, te kāyassa bhedā paraṃ maraṇā appekacce paranimmitavasavattīnaṃ devānaṃ sahabyataṃ upapajjiṃsu; appekacce nimmānaratīnaṃ devānaṃ sahabyataṃ upapajjiṃsu; appekacce tusitānaṃ devānaṃ sahabyataṃ upapajjiṃsu; appekacce yāmānaṃ devānaṃ sahabyataṃ upapajjiṃsu; appekacce tāvatiṃsānaṃ devānaṃ sahabyataṃ upapajjiṃsu; appekacce cātumahārājikānaṃ devānaṃ sahabyataṃ upapajjiṃsu; ye sabbanihīnaṃ kāyaṃ paripūresuṃ te gandhabbakāyaṃ paripūresuṃ. Iti kho, bho\footnote{pana (syā. ka.)}, sabbesaṃyeva tesaṃ kulaputtānaṃ amoghā pabbajjā ahosi avañjhā saphalā saudrayā’’’ti.

\paragraph{329.} ‘‘Sarati taṃ bhagavā’’ti? ‘‘Sarāmahaṃ, pañcasikha. Ahaṃ tena samayena mahāgovindo brāhmaṇo ahosiṃ. Ahaṃ tesaṃ sāvakānaṃ brahmalokasahabyatāya maggaṃ desesiṃ. Taṃ kho pana me, pañcasikha, brahmacariyaṃ na nibbidāya na virāgāya na nirodhāya na upasamāya na abhiññāya na sambodhāya na nibbānāya saṃvattati, yāvadeva brahmalokūpapattiyā.

Idaṃ kho pana me, pañcasikha, brahmacariyaṃ ekantanibbidāya virāgāya nirodhāya upasamāya abhiññāya sambodhāya nibbānāya saṃvattati. Katamañca taṃ, pañcasikha, brahmacariyaṃ ekantanibbidāya virāgāya nirodhāya upasamāya abhiññāya sambodhāya nibbānāya saṃvattati? Ayameva ariyo aṭṭhaṅgiko maggo. Seyyathidaṃ – sammādiṭṭhi sammāsaṅkappo sammāvācā sammākammanto sammāājīvo sammāvāyāmo sammāsati sammāsamādhi. Idaṃ kho taṃ, pañcasikha, brahmacariyaṃ ekantanibbidāya virāgāya nirodhāya upasamāya abhiññāya sambodhāya nibbānāya saṃvattati.

\paragraph{330.} ‘‘Ye kho pana me, pañcasikha, sāvakā sabbena sabbaṃ sāsanaṃ ājānanti, te āsavānaṃ khayā anāsavaṃ cetovimuttiṃ paññāvimuttiṃ diṭṭheva dhamme sayaṃ abhiññā sacchikatvā upasampajja viharanti; ye na sabbena sabbaṃ sāsanaṃ ājānanti, te pañcannaṃ orambhāgiyānaṃ saṃyojanānaṃ parikkhayā opapātikā honti tattha parinibbāyino anāvattidhammā tasmā lokā. Ye na sabbena sabbaṃ sāsanaṃ ājānanti, appekacce tiṇṇaṃ saṃyojanānaṃ parikkhayā rāgadosamohānaṃ tanuttā sakadāgāmino honti sakideva imaṃ lokaṃ āgantvā dukkhassantaṃ karissanti\footnote{karonti (sī. pī.)}. Ye na sabbena sabbaṃ sāsanaṃ ājānanti, appekacce tiṇṇaṃ saṃyojanānaṃ parikkhayā sotāpannā honti avinipātadhammā niyatā sambodhiparāyaṇā. Iti kho, pañcasikha, sabbesaṃyeva imesaṃ kulaputtānaṃ amoghā pabbajjā\footnote{pabbajā ahosi (ka.)} avañjhā saphalā saudrayā’’ti.

Idamavoca bhagavā. Attamano pañcasikho gandhabbaputto bhagavato bhāsitaṃ abhinanditvā anumoditvā bhagavantaṃ abhivādetvā padakkhiṇaṃ katvā tatthevantaradhāyīti.

\xsectionEnd{Mahāgovindasuttaṃ niṭṭhitaṃ chaṭṭhaṃ.}
