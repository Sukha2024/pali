\section{Mahānidānasuttaṃ}

\subsubsection{Paṭiccasamuppādo}

\paragraph{95.} Evaṃ me sutaṃ – ekaṃ samayaṃ bhagavā kurūsu viharati kammāsadhammaṃ nāma\footnote{kammāsadammaṃ nāma (syā.)} kurūnaṃ nigamo. Atha kho āyasmā ānando yena bhagavā tenupasaṅkami, upasaṅkamitvā bhagavantaṃ abhivādetvā ekamantaṃ nisīdi. Ekamantaṃ nisinno kho āyasmā ānando bhagavantaṃ etadavoca – ‘‘acchariyaṃ, bhante, abbhutaṃ, bhante! Yāva gambhīro cāyaṃ, bhante, paṭiccasamuppādo gambhīrāvabhāso ca, atha ca pana me uttānakuttānako viya khāyatī’’ti. ‘‘Mā hevaṃ, ānanda, avaca, mā hevaṃ, ānanda, avaca. Gambhīro cāyaṃ, ānanda, paṭiccasamuppādo gambhīrāvabhāso ca. Etassa, ānanda, dhammassa ananubodhā appaṭivedhā evamayaṃ pajā tantākulakajātā kulagaṇṭhikajātā\footnote{gulāguṇṭhikajātā (sī. pī.), guṇagaṇṭhikajātā (syā.)} muñjapabbajabhūtā apāyaṃ duggatiṃ vinipātaṃ saṃsāraṃ nātivattati.

\paragraph{96.} ‘‘‘Atthi idappaccayā jarāmaraṇa’nti iti puṭṭhena satā, ānanda, atthītissa vacanīyaṃ. ‘Kiṃpaccayā jarāmaraṇa’nti iti ce vadeyya, ‘jātipaccayā jarāmaraṇa’nti iccassa vacanīyaṃ.

‘‘‘Atthi idappaccayā jātī’ti iti puṭṭhena satā, ānanda, atthītissa vacanīyaṃ. ‘Kiṃpaccayā jātī’ti iti ce vadeyya, ‘bhavapaccayā jātī’ti iccassa vacanīyaṃ.

‘‘‘Atthi idappaccayā bhavo’ti iti puṭṭhena satā, ānanda, atthītissa vacanīyaṃ . ‘Kiṃpaccayā bhavo’ti iti ce vadeyya, ‘upādānapaccayā bhavo’ti iccassa vacanīyaṃ.

‘‘‘Atthi idappaccayā upādāna’nti iti puṭṭhena satā, ānanda, atthītissa vacanīyaṃ. ‘Kiṃpaccayā upādāna’nti iti ce vadeyya, ‘taṇhāpaccayā upādāna’nti iccassa vacanīyaṃ.

‘‘‘Atthi idappaccayā taṇhā’ti iti puṭṭhena satā, ānanda, atthītissa vacanīyaṃ. ‘Kiṃpaccayā taṇhā’ti iti ce vadeyya, ‘vedanāpaccayā taṇhā’ti iccassa vacanīyaṃ.

‘‘‘Atthi idappaccayā vedanā’ti iti puṭṭhena satā, ānanda, atthītissa vacanīyaṃ. ‘Kiṃpaccayā vedanā’ti iti ce vadeyya, ‘phassapaccayā vedanā’ti iccassa vacanīyaṃ.

‘‘‘Atthi idappaccayā phasso’ti iti puṭṭhena satā, ānanda, atthītissa vacanīyaṃ. ‘Kiṃpaccayā phasso’ti iti ce vadeyya, ‘nāmarūpapaccayā phasso’ti iccassa vacanīyaṃ.

‘‘‘Atthi idappaccayā nāmarūpa’nti iti puṭṭhena satā, ānanda, atthītissa vacanīyaṃ. ‘Kiṃpaccayā nāmarūpa’nti iti ce vadeyya, ‘viññāṇapaccayā nāmarūpa’nti iccassa vacanīyaṃ.

‘‘‘Atthi idappaccayā viññāṇa’nti iti puṭṭhena satā, ānanda, atthītissa vacanīyaṃ. ‘Kiṃpaccayā viññāṇa’nti iti ce vadeyya, ‘nāmarūpapaccayā viññāṇa’nti iccassa vacanīyaṃ.

\paragraph{97.} ‘‘Iti kho, ānanda, nāmarūpapaccayā viññāṇaṃ, viññāṇapaccayā nāmarūpaṃ, nāmarūpapaccayā phasso, phassapaccayā vedanā, vedanāpaccayā taṇhā, taṇhāpaccayā upādānaṃ, upādānapaccayā bhavo, bhavapaccayā jāti , jātipaccayā jarāmaraṇaṃ sokaparidevadukkhadomanassupāyāsā sambhavanti. Evametassa kevalassa dukkhakkhandhassa samudayo hoti.

\paragraph{98.} ‘‘‘Jātipaccayā jarāmaraṇa’nti iti kho panetaṃ vuttaṃ, tadānanda, imināpetaṃ pariyāyena veditabbaṃ, yathā jātipaccayā jarāmaraṇaṃ. Jāti ca hi, ānanda, nābhavissa sabbena sabbaṃ sabbathā sabbaṃ kassaci kimhici, seyyathidaṃ – devānaṃ vā devattāya, gandhabbānaṃ vā gandhabbattāya, yakkhānaṃ vā yakkhattāya, bhūtānaṃ vā bhūtattāya, manussānaṃ vā manussattāya, catuppadānaṃ vā catuppadattāya, pakkhīnaṃ vā pakkhittāya, sarīsapānaṃ vā sarīsapattāya\footnote{siriṃsapānaṃ siriṃsapattāya (sī. syā.)}, tesaṃ tesañca hi, ānanda, sattānaṃ tadattāya jāti nābhavissa. Sabbaso jātiyā asati jātinirodhā api nu kho jarāmaraṇaṃ paññāyethā’’ti? ‘‘No hetaṃ, bhante’’. ‘‘Tasmātihānanda, eseva hetu etaṃ nidānaṃ esa samudayo esa paccayo jarāmaraṇassa, yadidaṃ jāti’’.

\paragraph{99.} ‘‘‘Bhavapaccayā jātī’ti iti kho panetaṃ vuttaṃ, tadānanda, imināpetaṃ pariyāyena veditabbaṃ, yathā bhavapaccayā jāti. Bhavo ca hi, ānanda, nābhavissa sabbena sabbaṃ sabbathā sabbaṃ kassaci kimhici, seyyathidaṃ – kāmabhavo vā rūpabhavo vā arūpabhavo vā, sabbaso bhave asati bhavanirodhā api nu kho jāti paññāyethā’’ti? ‘‘No hetaṃ, bhante’’. ‘‘Tasmātihānanda, eseva hetu etaṃ nidānaṃ esa samudayo esa paccayo jātiyā, yadidaṃ bhavo’’.

\paragraph{100.} ‘‘‘Upādānapaccayā bhavo’ti iti kho panetaṃ vuttaṃ, tadānanda, imināpetaṃ pariyāyena veditabbaṃ, yathā upādānapaccayā bhavo. Upādānañca hi, ānanda, nābhavissa sabbena sabbaṃ sabbathā sabbaṃ kassaci kimhici , seyyathidaṃ – kāmupādānaṃ vā diṭṭhupādānaṃ vā sīlabbatupādānaṃ vā attavādupādānaṃ vā, sabbaso upādāne asati upādānanirodhā api nu kho bhavo paññāyethā’’ti? ‘‘No hetaṃ, bhante’’. ‘‘Tasmātihānanda, eseva hetu etaṃ nidānaṃ esa samudayo esa paccayo bhavassa, yadidaṃ upādānaṃ’’.

\paragraph{101.} ‘‘‘Taṇhāpaccayā upādāna’nti iti kho panetaṃ vuttaṃ tadānanda, imināpetaṃ pariyāyena veditabbaṃ, yathā taṇhāpaccayā upādānaṃ. Taṇhā ca hi, ānanda, nābhavissa sabbena sabbaṃ sabbathā sabbaṃ kassaci kimhici, seyyathidaṃ – rūpataṇhā saddataṇhā gandhataṇhā rasataṇhā phoṭṭhabbataṇhā dhammataṇhā, sabbaso taṇhāya asati taṇhānirodhā api nu kho upādānaṃ paññāyethā’’ti? ‘‘No hetaṃ, bhante’’. ‘‘Tasmātihānanda, eseva hetu etaṃ nidānaṃ esa samudayo esa paccayo upādānassa, yadidaṃ taṇhā’’.

\paragraph{102.} ‘‘‘Vedanāpaccayā taṇhā’ti iti kho panetaṃ vuttaṃ, tadānanda, imināpetaṃ pariyāyena veditabbaṃ, yathā vedanāpaccayā taṇhā. Vedanā ca hi, ānanda, nābhavissa sabbena sabbaṃ sabbathā sabbaṃ kassaci kimhici, seyyathidaṃ – cakkhusamphassajā vedanā sotasamphassajā vedanā ghānasamphassajā vedanā jivhāsamphassajā vedanā kāyasamphassajā vedanā manosamphassajā vedanā, sabbaso vedanāya asati vedanānirodhā api nu kho taṇhā paññāyethā’’ti ? ‘‘No hetaṃ, bhante’’. ‘‘Tasmātihānanda, eseva hetu etaṃ nidānaṃ esa samudayo esa paccayo taṇhāya, yadidaṃ vedanā’’.

\paragraph{103.} ‘‘Iti kho panetaṃ, ānanda, vedanaṃ paṭicca taṇhā, taṇhaṃ paṭicca pariyesanā, pariyesanaṃ paṭicca lābho, lābhaṃ paṭicca vinicchayo, vinicchayaṃ paṭicca chandarāgo, chandarāgaṃ paṭicca ajjhosānaṃ, ajjhosānaṃ paṭicca pariggaho, pariggahaṃ paṭicca macchariyaṃ, macchariyaṃ paṭicca ārakkho. Ārakkhādhikaraṇaṃ daṇḍādānasatthādānakalahaviggahavivādatuvaṃtuvaṃpesuññamusāvādā aneke pāpakā akusalā dhammā sambhavanti.

\paragraph{104.} ‘‘‘Ārakkhādhikaraṇaṃ\footnote{ārakkhaṃ paṭicca ārakkhādhikaraṇaṃ (syā.)} daṇḍādānasatthādānakalahaviggahavivādatuvaṃtuvaṃpesuññamusāvādā aneke pāpakā akusalā dhammā sambhavantī’ti iti kho panetaṃ vuttaṃ, tadānanda, imināpetaṃ pariyāyena veditabbaṃ, yathā ārakkhādhikaraṇaṃ daṇḍādānasatthādānakalahaviggahavivādatuvaṃtuvaṃpesuññamusāvādā aneke pāpakā akusalā dhammā sambhavanti. Ārakkho ca hi, ānanda, nābhavissa sabbena sabbaṃ sabbathā sabbaṃ kassaci kimhici, sabbaso ārakkhe asati ārakkhanirodhā api nu kho daṇḍādānasatthādānakalahaviggahavivādatuvaṃtuvaṃpesuññamusāvādā aneke pāpakā akusalā dhammā sambhaveyyu’’nti? ‘‘No hetaṃ, bhante’’. ‘‘Tasmātihānanda, eseva hetu etaṃ nidānaṃ esa samudayo esa paccayo daṇḍādānasatthādānakalahaviggahavivādatuvaṃtuvaṃpesuññamusāvādānaṃ anekesaṃ pāpakānaṃ akusalānaṃ dhammānaṃ sambhavāya yadidaṃ ārakkho.

\paragraph{105.} ‘‘‘Macchariyaṃ paṭicca ārakkho’ti iti kho panetaṃ vuttaṃ, tadānanda, imināpetaṃ pariyāyena veditabbaṃ, yathā macchariyaṃ paṭicca ārakkho. Macchariyañca hi, ānanda, nābhavissa sabbena sabbaṃ sabbathā sabbaṃ kassaci kimhici , sabbaso macchariye asati macchariyanirodhā api nu kho ārakkho paññāyethā’’ti? ‘‘No hetaṃ, bhante’’. ‘‘Tasmātihānanda, eseva hetu etaṃ nidānaṃ esa samudayo esa paccayo ārakkhassa, yadidaṃ macchariyaṃ’’.

\paragraph{106.} ‘‘‘Pariggahaṃ paṭicca macchariya’nti iti kho panetaṃ vuttaṃ, tadānanda, imināpetaṃ pariyāyena veditabbaṃ, yathā pariggahaṃ paṭicca macchariyaṃ. Pariggaho ca hi, ānanda, nābhavissa sabbena sabbaṃ sabbathā sabbaṃ kassaci kimhici, sabbaso pariggahe asati pariggahanirodhā api nu kho macchariyaṃ paññāyethā’’ti? ‘‘No hetaṃ, bhante’’. ‘‘Tasmātihānanda, eseva hetu etaṃ nidānaṃ esa samudayo esa paccayo macchariyassa, yadidaṃ pariggaho’’.

\paragraph{107.} ‘‘‘Ajjhosānaṃ paṭicca pariggaho’ti iti kho panetaṃ vuttaṃ, tadānanda, imināpetaṃ pariyāyena veditabbaṃ, yathā ajjhosānaṃ paṭicca pariggaho. Ajjhosānañca hi, ānanda, nābhavissa sabbena sabbaṃ sabbathā sabbaṃ kassaci kimhici, sabbaso ajjhosāne asati ajjhosānanirodhā api nu kho pariggaho paññāyethā’’ti ? ‘‘No hetaṃ, bhante’’. ‘‘Tasmātihānanda, eseva hetu etaṃ nidānaṃ esa samudayo esa paccayo pariggahassa – yadidaṃ ajjhosānaṃ’’.

\paragraph{108.} ‘‘‘Chandarāgaṃ paṭicca ajjhosāna’nti iti kho panetaṃ vuttaṃ, tadānanda, imināpetaṃ pariyāyena veditabbaṃ, yathā chandarāgaṃ paṭicca ajjhosānaṃ. Chandarāgo ca hi, ānanda, nābhavissa sabbena sabbaṃ sabbathā sabbaṃ kassaci kimhici, sabbaso chandarāge asati chandarāganirodhā api nu kho ajjhosānaṃ paññāyethā’’ti? ‘‘No hetaṃ, bhante’’. ‘‘Tasmātihānanda, eseva hetu etaṃ nidānaṃ esa samudayo esa paccayo ajjhosānassa, yadidaṃ chandarāgo’’.

\paragraph{109.} ‘‘‘Vinicchayaṃ paṭicca chandarāgo’ti iti kho panetaṃ vuttaṃ, tadānanda, imināpetaṃ pariyāyena veditabbaṃ, yathā vinicchayaṃ paṭicca chandarāgo. Vinicchayo ca hi, ānanda, nābhavissa sabbena sabbaṃ sabbathā sabbaṃ kassaci kimhici, sabbaso vinicchaye asati vinicchayanirodhā api nu kho chandarāgo paññāyethā’’ti? ‘‘No hetaṃ , bhante’’. ‘‘Tasmātihānanda, eseva hetu etaṃ nidānaṃ esa samudayo esa paccayo chandarāgassa, yadidaṃ vinicchayo’’.

\paragraph{110.} ‘‘‘Lābhaṃ paṭicca vinicchayo’ti iti kho panetaṃ vuttaṃ, tadānanda, imināpetaṃ pariyāyena veditabbaṃ, yathā lābhaṃ paṭicca vinicchayo. Lābho ca hi, ānanda, nābhavissa sabbena sabbaṃ sabbathā sabbaṃ kassaci kimhici, sabbaso lābhe asati lābhanirodhā api nu kho vinicchayo paññāyethā’’ti? ‘‘No hetaṃ, bhante’’. ‘‘Tasmātihānanda eseva hetu etaṃ nidānaṃ esa samudayo esa paccayo vinicchayassa, yadidaṃ lābho’’.

\paragraph{111.} ‘‘‘Pariyesanaṃ paṭicca lābho’ti iti kho panetaṃ vuttaṃ, tadānanda, imināpetaṃ pariyāyena veditabbaṃ, yathā pariyesanaṃ paṭicca lābho. Pariyesanā ca hi, ānanda, nābhavissa sabbena sabbaṃ sabbathā sabbaṃ kassaci kimhici, sabbaso pariyesanāya asati pariyesanānirodhā api nu kho lābho paññāyethā’’ti? ‘‘No hetaṃ, bhante’’. ‘‘Tasmātihānanda, eseva hetu etaṃ nidānaṃ esa samudayo esa paccayo lābhassa, yadidaṃ pariyesanā’’.

\paragraph{112.} ‘‘‘Taṇhaṃ paṭicca pariyesanā’ti iti kho panetaṃ vuttaṃ, tadānanda, imināpetaṃ pariyāyena veditabbaṃ, yathā taṇhaṃ paṭicca pariyesanā. Taṇhā ca hi, ānanda, nābhavissa sabbena sabbaṃ sabbathā sabbaṃ kassaci kimhici, seyyathidaṃ – kāmataṇhā bhavataṇhā vibhavataṇhā, sabbaso taṇhāya asati taṇhānirodhā api nu kho pariyesanā paññāyethā’’ti? ‘‘No hetaṃ, bhante’’. ‘‘Tasmātihānanda, eseva hetu etaṃ nidānaṃ esa samudayo esa paccayo pariyesanāya, yadidaṃ taṇhā. Iti kho, ānanda, ime dve dhammā\footnote{ime dhammā (ka.)} dvayena vedanāya ekasamosaraṇā bhavanti’’.

\paragraph{113.} ‘‘‘Phassapaccayā vedanā’ti iti kho panetaṃ vuttaṃ, tadānanda, imināpetaṃ pariyāyena veditabbaṃ, yathā ‘phassapaccayā vedanā. Phasso ca hi, ānanda, nābhavissa sabbena sabbaṃ sabbathā sabbaṃ kassaci kimhici, seyyathidaṃ – cakkhusamphasso sotasamphasso ghānasamphasso jivhāsamphasso kāyasamphasso manosamphasso, sabbaso phasse asati phassanirodhā api nu kho vedanā paññāyethā’’ti? ‘‘No hetaṃ, bhante’’. ‘‘Tasmātihānanda , eseva hetu etaṃ nidānaṃ esa samudayo esa paccayo vedanāya, yadidaṃ phasso’’.

\paragraph{114.} ‘‘‘Nāmarūpapaccayā phasso’ti iti kho panetaṃ vuttaṃ, tadānanda, imināpetaṃ pariyāyena veditabbaṃ, yathā nāmarūpapaccayā phasso. Yehi, ānanda, ākārehi yehi liṅgehi yehi nimittehi yehi uddesehi nāmakāyassa paññatti hoti, tesu ākāresu tesu liṅgesu tesu nimittesu tesu uddesesu asati api nu kho rūpakāye adhivacanasamphasso paññāyethā’’ti? ‘‘No hetaṃ, bhante’’. ‘‘Yehi, ānanda, ākārehi yehi liṅgehi yehi nimittehi yehi uddesehi rūpakāyassa paññatti hoti, tesu ākāresu…pe… tesu uddesesu asati api nu kho nāmakāye paṭighasamphasso paññāyethā’’ti? ‘‘No hetaṃ, bhante’’. ‘‘Yehi, ānanda, ākārehi…pe… yehi uddesehi nāmakāyassa ca rūpakāyassa ca paññatti hoti , tesu ākāresu…pe… tesu uddesesu asati api nu kho adhivacanasamphasso vā paṭighasamphasso vā paññāyethā’’ti? ‘‘No hetaṃ, bhante’’. ‘‘Yehi, ānanda, ākārehi…pe… yehi uddesehi nāmarūpassa paññatti hoti, tesu ākāresu …pe… tesu uddesesu asati api nu kho phasso paññāyethā’’ti? ‘‘No hetaṃ, bhante’’. ‘‘Tasmātihānanda, eseva hetu etaṃ nidānaṃ esa samudayo esa paccayo phassassa, yadidaṃ nāmarūpaṃ’’.

\paragraph{115.} ‘‘‘Viññāṇapaccayā nāmarūpa’nti iti kho panetaṃ vuttaṃ, tadānanda, imināpetaṃ pariyāyena veditabbaṃ, yathā viññāṇapaccayā nāmarūpaṃ. Viññāṇañca hi, ānanda, mātukucchismiṃ na okkamissatha, api nu kho nāmarūpaṃ mātukucchismiṃ samuccissathā’’ti? ‘‘No hetaṃ, bhante’’. ‘‘Viññāṇañca hi, ānanda, mātukucchismiṃ okkamitvā vokkamissatha, api nu kho nāmarūpaṃ itthattāya abhinibbattissathā’’ti? ‘‘No hetaṃ, bhante’’. ‘‘Viññāṇañca hi, ānanda, daharasseva sato vocchijjissatha kumārakassa vā kumārikāya vā, api nu kho nāmarūpaṃ vuddhiṃ virūḷhiṃ vepullaṃ āpajjissathā’’ti? ‘‘No hetaṃ, bhante’’. ‘‘Tasmātihānanda, eseva hetu etaṃ nidānaṃ esa samudayo esa paccayo nāmarūpassa – yadidaṃ viññāṇaṃ’’.

\paragraph{116.} ‘‘‘Nāmarūpapaccayā viññāṇa’nti iti kho panetaṃ vuttaṃ, tadānanda, imināpetaṃ pariyāyena veditabbaṃ, yathā nāmarūpapaccayā viññāṇaṃ. Viññāṇañca hi, ānanda, nāmarūpe patiṭṭhaṃ na labhissatha, api nu kho āyatiṃ jātijarāmaraṇaṃ dukkhasamudayasambhavo\footnote{jātijarāmaraṇadukkhasamudayasambhavo (sī. syā. pī.)} paññāyethā’’ti? ‘‘No hetaṃ, bhante’’. ‘‘Tasmātihānanda, eseva hetu etaṃ nidānaṃ esa samudayo esa paccayo viññāṇassa yadidaṃ nāmarūpaṃ. Ettāvatā kho, ānanda, jāyetha vā jīyetha\footnote{jiyyetha (ka.)} vā mīyetha\footnote{miyyetha (ka.)} vā cavetha vā upapajjetha vā. Ettāvatā adhivacanapatho, ettāvatā niruttipatho, ettāvatā paññattipatho, ettāvatā paññāvacaraṃ, ettāvatā vaṭṭaṃ vattati itthattaṃ paññāpanāya yadidaṃ nāmarūpaṃ saha viññāṇena aññamaññapaccayatā pavattati.

\subsubsection{Attapaññatti}

\paragraph{117.} ‘‘Kittāvatā ca, ānanda, attānaṃ paññapento paññapeti? Rūpiṃ vā hi, ānanda, parittaṃ attānaṃ paññapento paññapeti – ‘‘rūpī me paritto attā’’ti. Rūpiṃ vā hi , ānanda, anantaṃ attānaṃ paññapento paññapeti – ‘rūpī me ananto attā’ti. Arūpiṃ vā hi, ānanda, parittaṃ attānaṃ paññapento paññapeti – ‘arūpī me paritto attā’ti. Arūpiṃ vā hi, ānanda, anantaṃ attānaṃ paññapento paññapeti – ‘arūpī me ananto attā’ti.

\paragraph{118.} ‘‘Tatrānanda, yo so rūpiṃ parittaṃ attānaṃ paññapento paññapeti. Etarahi vā so rūpiṃ parittaṃ attānaṃ paññapento paññapeti, tattha bhāviṃ vā so rūpiṃ parittaṃ attānaṃ paññapento paññapeti, ‘atathaṃ vā pana santaṃ tathattāya upakappessāmī’ti iti vā panassa hoti. Evaṃ santaṃ kho, ānanda, rūpiṃ\footnote{rūpī (ka.)} parittattānudiṭṭhi anusetīti iccālaṃ vacanāya.

‘‘Tatrānanda, yo so rūpiṃ anantaṃ attānaṃ paññapento paññapeti. Etarahi vā so rūpiṃ anantaṃ attānaṃ paññapento paññapeti, tattha bhāviṃ vā so rūpiṃ anantaṃ attānaṃ paññapento paññapeti, ‘atathaṃ vā pana santaṃ tathattāya upakappessāmī’ti iti vā panassa hoti. Evaṃ santaṃ kho, ānanda, rūpiṃ\footnote{rūpī (ka.)} anantattānudiṭṭhi anusetīti iccālaṃ vacanāya.

‘‘Tatrānanda, yo so arūpiṃ parittaṃ attānaṃ paññapento paññapeti. Etarahi vā so arūpiṃ parittaṃ attānaṃ paññapento paññapeti, tattha bhāviṃ vā so arūpiṃ parittaṃ attānaṃ paññapento paññapeti, ‘atathaṃ vā pana santaṃ tathattāya upakappessāmī’ti iti vā panassa hoti. Evaṃ santaṃ kho, ānanda, arūpiṃ\footnote{arūpī (ka.)} parittattānudiṭṭhi anusetīti iccālaṃ vacanāya.

‘‘Tatrānanda, yo so arūpiṃ anantaṃ attānaṃ paññapento paññapeti. Etarahi vā so arūpiṃ anantaṃ attānaṃ paññapento paññapeti, tattha bhāviṃ vā so arūpiṃ anantaṃ attānaṃ paññapento paññapeti, ‘atathaṃ vā pana santaṃ tathattāya upakappessāmī’ti iti vā panassa hoti. Evaṃ santaṃ kho, ānanda, arūpiṃ\footnote{arūpī (ka.)} anantattānudiṭṭhi anusetīti iccālaṃ vacanāya. Ettāvatā kho, ānanda, attānaṃ paññapento paññapeti.

\subsubsection{Naattapaññatti}

\paragraph{119.} ‘‘Kittāvatā ca, ānanda, attānaṃ na paññapento na paññapeti? Rūpiṃ vā hi, ānanda, parittaṃ attānaṃ na paññapento na paññapeti – ‘rūpī me paritto attā’ti. Rūpiṃ vā hi, ānanda, anantaṃ attānaṃ na paññapento na paññapeti – ‘rūpī me ananto attā’ti. Arūpiṃ vā hi, ānanda, parittaṃ attānaṃ na paññapento na paññapeti – ‘arūpī me paritto attā’ti. Arūpiṃ vā hi, ānanda, anantaṃ attānaṃ na paññapento na paññapeti – ‘arūpī me ananto attā’ti.

\paragraph{120.} ‘‘Tatrānanda, yo so rūpiṃ parittaṃ attānaṃ na paññapento na paññapeti. Etarahi vā so rūpiṃ parittaṃ attānaṃ na paññapento na paññapeti, tattha bhāviṃ vā so rūpiṃ parittaṃ attānaṃ na paññapento na paññapeti, ‘atathaṃ vā pana santaṃ tathattāya upakappessāmī’ti iti vā panassa na hoti. Evaṃ santaṃ kho, ānanda, rūpiṃ parittattānudiṭṭhi nānusetīti iccālaṃ vacanāya.

‘‘Tatrānanda , yo so rūpiṃ anantaṃ attānaṃ na paññapento na paññapeti. Etarahi vā so rūpiṃ anantaṃ attānaṃ na paññapento na paññapeti, tattha bhāviṃ vā so rūpiṃ anantaṃ attānaṃ na paññapento na paññapeti, ‘atathaṃ vā pana santaṃ tathattāya upakappessāmī’ti iti vā panassa na hoti. Evaṃ santaṃ kho, ānanda, rūpiṃ anantattānudiṭṭhi nānusetīti iccālaṃ vacanāya.

‘‘Tatrānanda, yo so arūpiṃ parittaṃ attānaṃ na paññapento na paññapeti. Etarahi vā so arūpiṃ parittaṃ attānaṃ na paññapento na paññapeti, tattha bhāviṃ vā so arūpiṃ parittaṃ attānaṃ na paññapento na paññapeti, ‘atathaṃ vā pana santaṃ tathattāya upakappessāmī’ti iti vā panassa na hoti. Evaṃ santaṃ kho, ānanda, arūpiṃ parittattānudiṭṭhi nānusetīti iccālaṃ vacanāya.

‘‘Tatrānanda, yo so arūpiṃ anantaṃ attānaṃ na paññapento na paññapeti. Etarahi vā so arūpiṃ anantaṃ attānaṃ na paññapento na paññapeti, tattha bhāviṃ vā so arūpiṃ anantaṃ attānaṃ na paññapento na paññapeti, ‘atathaṃ vā pana santaṃ tathattāya upakappessāmī’ti iti vā panassa na hoti. Evaṃ santaṃ kho, ānanda, arūpiṃ anantattānudiṭṭhi nānusetīti iccālaṃ vacanāya. Ettāvatā kho, ānanda, attānaṃ na paññapento na paññapeti.

\subsubsection{Attasamanupassanā}

\paragraph{121.} ‘‘Kittāvatā ca, ānanda, attānaṃ samanupassamāno samanupassati? Vedanaṃ vā hi, ānanda, attānaṃ samanupassamāno samanupassati – ‘vedanā me attā’ti. ‘Na heva kho me vedanā attā, appaṭisaṃvedano me attā’ti iti vā hi, ānanda, attānaṃ samanupassamāno samanupassati. ‘Na heva kho me vedanā attā, nopi appaṭisaṃvedano me attā, attā me vediyati, vedanādhammo hi me attā’ti iti vā hi, ānanda, attānaṃ samanupassamāno samanupassati.

\paragraph{122.} ‘‘Tatrānanda, yo so evamāha – ‘vedanā me attā’ti, so evamassa vacanīyo – ‘tisso kho imā, āvuso, vedanā – sukhā vedanā dukkhā vedanā adukkhamasukhā vedanā. Imāsaṃ kho tvaṃ tissannaṃ vedanānaṃ katamaṃ attato samanupassasī’ti? Yasmiṃ, ānanda, samaye sukhaṃ vedanaṃ vedeti, neva tasmiṃ samaye dukkhaṃ vedanaṃ vedeti, na adukkhamasukhaṃ vedanaṃ vedeti; sukhaṃyeva tasmiṃ samaye vedanaṃ vedeti. Yasmiṃ, ānanda, samaye dukkhaṃ vedanaṃ vedeti, neva tasmiṃ samaye sukhaṃ vedanaṃ vedeti, na adukkhamasukhaṃ vedanaṃ vedeti; dukkhaṃyeva tasmiṃ samaye vedanaṃ vedeti. Yasmiṃ, ānanda, samaye adukkhamasukhaṃ vedanaṃ vedeti, neva tasmiṃ samaye sukhaṃ vedanaṃ vedeti, na dukkhaṃ vedanaṃ vedeti; adukkhamasukhaṃyeva tasmiṃ samaye vedanaṃ vedeti.

\paragraph{123.} ‘‘Sukhāpi kho, ānanda, vedanā aniccā saṅkhatā paṭiccasamuppannā khayadhammā vayadhammā virāgadhammā nirodhadhammā. Dukkhāpi kho, ānanda, vedanā aniccā saṅkhatā paṭiccasamuppannā khayadhammā vayadhammā virāgadhammā nirodhadhammā. Adukkhamasukhāpi kho, ānanda, vedanā aniccā saṅkhatā paṭiccasamuppannā khayadhammā vayadhammā virāgadhammā nirodhadhammā. Tassa sukhaṃ vedanaṃ vediyamānassa ‘eso me attā’ti hoti. Tassāyeva sukhāya vedanāya nirodhā ‘byagā\footnote{byaggā (sī. ka.)} me attā’ti hoti. Dukkhaṃ vedanaṃ vediyamānassa ‘eso me attā’ti hoti. Tassāyeva dukkhāya vedanāya nirodhā ‘byagā me attā’ti hoti. Adukkhamasukhaṃ vedanaṃ vediyamānassa ‘eso me attā’ti hoti. Tassāyeva adukkhamasukhāya vedanāya nirodhā ‘byagā me attā’ti hoti. Iti so diṭṭheva dhamme aniccasukhadukkhavokiṇṇaṃ uppādavayadhammaṃ attānaṃ samanupassamāno samanupassati, yo so evamāha – ‘vedanā me attā’ti. Tasmātihānanda, etena petaṃ nakkhamati – ‘vedanā me attā’ti samanupassituṃ.

\paragraph{124.} ‘‘Tatrānanda , yo so evamāha – ‘na heva kho me vedanā attā, appaṭisaṃvedano me attā’ti, so evamassa vacanīyo – ‘yattha panāvuso, sabbaso vedayitaṃ natthi api nu kho, tattha ‘‘ayamahamasmī’’ti siyā’’’ti ? ‘‘No hetaṃ, bhante’’. ‘‘Tasmātihānanda, etena petaṃ nakkhamati – ‘na heva kho me vedanā attā, appaṭisaṃvedano me attā’ti samanupassituṃ.

\paragraph{125.} ‘‘Tatrānanda , yo so evamāha – ‘na heva kho me vedanā attā, nopi appaṭisaṃvedano me attā, attā me vediyati, vedanādhammo hi me attā’ti. So evamassa vacanīyo – vedanā ca hi, āvuso, sabbena sabbaṃ sabbathā sabbaṃ aparisesā nirujjheyyuṃ. Sabbaso vedanāya asati vedanānirodhā api nu kho tattha ‘ayamahamasmī’ti siyā’’ti? ‘No hetaṃ, bhante’’. ‘‘Tasmātihānanda, etena petaṃ nakkhamati – ‘‘na heva kho me vedanā attā, nopi appaṭisaṃvedano me attā, attā me vediyati, vedanādhammo hi me attā’ti samanupassituṃ.

\paragraph{126.} ‘‘Yato kho, ānanda, bhikkhu neva vedanaṃ attānaṃ samanupassati, nopi appaṭisaṃvedanaṃ attānaṃ samanupassati, nopi ‘attā me vediyati, vedanādhammo hi me attā’ti samanupassati. So evaṃ na samanupassanto na ca kiñci loke upādiyati, anupādiyaṃ na paritassati, aparitassaṃ\footnote{aparitassanaṃ (ka.)} paccattaññeva parinibbāyati, ‘khīṇā jāti, vusitaṃ brahmacariyaṃ, kataṃ karaṇīyaṃ, nāparaṃ itthattāyā’ti pajānāti. Evaṃ vimuttacittaṃ kho, ānanda, bhikkhuṃ yo evaṃ vadeyya – ‘hoti tathāgato paraṃ maraṇā itissa\footnote{iti sā (aṭṭhakathāyaṃ pāṭhantaraṃ)} diṭṭhī’ti, tadakallaṃ. ‘Na hoti tathāgato paraṃ maraṇā itissa diṭṭhī’ti, tadakallaṃ. ‘Hoti ca na ca hoti tathāgato paraṃ maraṇā itissa diṭṭhī’ti, tadakallaṃ. ‘Neva hoti na na hoti tathāgato paraṃ maraṇā itissa diṭṭhī’ti, tadakallaṃ. Taṃ kissa hetu? Yāvatā, ānanda, adhivacanaṃ yāvatā adhivacanapatho, yāvatā nirutti yāvatā niruttipatho, yāvatā paññatti yāvatā paññattipatho, yāvatā paññā yāvatā paññāvacaraṃ, yāvatā vaṭṭaṃ\footnote{yāvatā vaṭṭaṃ vaṭṭati (ka. sī.)}, yāvatā vaṭṭati\footnote{yāvatā vaṭṭaṃ vaṭṭati (ka. sī.)}, tadabhiññāvimutto bhikkhu, tadabhiññāvimuttaṃ bhikkhuṃ ‘na jānāti na passati itissa diṭṭhī’ti, tadakallaṃ.

\subsubsection{Satta viññāṇaṭṭhiti}

\paragraph{127.} ‘‘Satta kho, ānanda\footnote{satta kho imā ānanda (ka. sī. syā.)}, viññāṇaṭṭhitiyo, dve āyatanāni. Katamā satta? Santānanda, sattā nānattakāyā nānattasaññino, seyyathāpi manussā , ekacce ca devā, ekacce ca vinipātikā. Ayaṃ paṭhamā viññāṇaṭṭhiti. Santānanda, sattā nānattakāyā ekattasaññino, seyyathāpi devā brahmakāyikā paṭhamābhinibbattā. Ayaṃ dutiyā viññāṇaṭṭhiti. Santānanda, sattā ekattakāyā nānattasaññino, seyyathāpi devā ābhassarā. Ayaṃ tatiyā viññāṇaṭṭhiti. Santānanda, sattā ekattakāyā ekattasaññino, seyyathāpi devā subhakiṇhā. Ayaṃ catutthī viññāṇaṭṭhiti. Santānanda, sattā sabbaso rūpasaññānaṃ samatikkamā paṭighasaññānaṃ atthaṅgamā nānattasaññānaṃ amanasikārā ‘ananto ākāso’ti ākāsānañcāyatanūpagā. Ayaṃ pañcamī viññāṇaṭṭhiti . Santānanda, sattā sabbaso ākāsānañcāyatanaṃ samatikkamma ‘anantaṃ viññāṇa’nti viññāṇañcāyatanūpagā. Ayaṃ chaṭṭhī viññāṇaṭṭhiti. Santānanda, sattā sabbaso viññāṇañcāyatanaṃ samatikkamma ‘natthi kiñcī’ti ākiñcaññāyatanūpagā. Ayaṃ sattamī viññāṇaṭṭhiti. Asaññasattāyatanaṃ nevasaññānāsaññāyatanameva dutiyaṃ.

\paragraph{128.} ‘‘Tatrānanda, yāyaṃ paṭhamā viññāṇaṭṭhiti nānattakāyā nānattasaññino, seyyathāpi manussā, ekacce ca devā, ekacce ca vinipātikā. Yo nu kho, ānanda, tañca pajānāti, tassā ca samudayaṃ pajānāti, tassā ca atthaṅgamaṃ pajānāti, tassā ca assādaṃ pajānāti, tassā ca ādīnavaṃ pajānāti, tassā ca nissaraṇaṃ pajānāti, kallaṃ nu tena tadabhinanditu’’nti? ‘‘No hetaṃ, bhante’’…pe… ‘‘tatrānanda, yamidaṃ asaññasattāyatanaṃ. Yo nu kho, ānanda, tañca pajānāti, tassa ca samudayaṃ pajānāti, tassa ca atthaṅgamaṃ pajānāti, tassa ca assādaṃ pajānāti, tassa ca ādīnavaṃ pajānāti, tassa ca nissaraṇaṃ pajānāti, kallaṃ nu tena tadabhinanditu’’nti? ‘‘No hetaṃ, bhante’’. ‘‘Tatrānanda, yamidaṃ nevasaññānāsaññāyatanaṃ. Yo nu kho, ānanda, tañca pajānāti, tassa ca samudayaṃ pajānāti, tassa ca atthaṅgamaṃ pajānāti, tassa ca assādaṃ pajānāti, tassa ca ādīnavaṃ pajānāti, tassa ca nissaraṇaṃ pajānāti, kallaṃ nu tena tadabhinanditu’’nti? ‘‘No hetaṃ, bhante’’. Yato kho, ānanda, bhikkhu imāsañca sattannaṃ viññāṇaṭṭhitīnaṃ imesañca dvinnaṃ āyatanānaṃ samudayañca atthaṅgamañca assādañca ādīnavañca nissaraṇañca yathābhūtaṃ viditvā anupādā vimutto hoti, ayaṃ vuccatānanda, bhikkhu paññāvimutto.

\subsubsection{Aṭṭha vimokkhā}

\paragraph{129.} ‘‘Aṭṭha kho ime, ānanda, vimokkhā. Katame aṭṭha? Rūpī rūpāni passati ayaṃ paṭhamo vimokkho. Ajjhattaṃ arūpasaññī bahiddhā rūpāni passati, ayaṃ dutiyo vimokkho. Subhanteva adhimutto hoti, ayaṃ tatiyo vimokkho. Sabbaso rūpasaññānaṃ samatikkamā paṭighasaññānaṃ atthaṅgamā nānattasaññānaṃ amanasikārā ‘ananto ākāso’ti ākāsānañcāyatanaṃ upasampajja viharati, ayaṃ catuttho vimokkho. Sabbaso ākāsānañcāyatanaṃ samatikkamma ‘anantaṃ viññāṇa’nti viññāṇañcāyatanaṃ upasampajja viharati, ayaṃ pañcamo vimokkho. Sabbaso viññāṇañcāyatanaṃ samatikkamma ‘natthi kiñcī’ti ākiñcaññāyatanaṃ upasampajja viharati, ayaṃ chaṭṭho vimokkho. Sabbaso ākiñcaññāyatanaṃ samatikkamma ‘nevasaññānāsaññā’yatanaṃ upasampajja viharati, ayaṃ sattamo vimokkho. Sabbaso nevasaññānāsaññāyatanaṃ samatikkamma saññāvedayitanirodhaṃ upasampajja viharati, ayaṃ aṭṭhamo vimokkho. Ime kho, ānanda, aṭṭha vimokkhā.

\paragraph{130.} ‘‘Yato kho, ānanda, bhikkhu ime aṭṭha vimokkhe anulomampi samāpajjati, paṭilomampi samāpajjati, anulomapaṭilomampi samāpajjati, yatthicchakaṃ yadicchakaṃ yāvaticchakaṃ samāpajjatipi vuṭṭhātipi. Āsavānañca khayā anāsavaṃ cetovimuttiṃ paññāvimuttiṃ diṭṭheva dhamme sayaṃ abhiññā sacchikatvā upasampajja viharati, ayaṃ vuccatānanda, bhikkhu ubhatobhāgavimutto. Imāya ca ānanda ubhatobhāgavimuttiyā aññā ubhatobhāgavimutti uttaritarā vā paṇītatarā vā natthī’’ti. Idamavoca bhagavā. Attamano āyasmā ānando bhagavato bhāsitaṃ abhinandīti.

\xsectionEnd{Mahānidānasuttaṃ niṭṭhitaṃ dutiyaṃ.}
