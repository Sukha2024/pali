\section{Sakkapañhasuttaṃ}

\paragraph{344.} Evaṃ me sutaṃ – ekaṃ samayaṃ bhagavā magadhesu viharati, pācīnato rājagahassa ambasaṇḍā nāma brāhmaṇagāmo, tassuttarato vediyake pabbate indasālaguhāyaṃ. Tena kho pana samayena sakkassa devānamindassa ussukkaṃ udapādi bhagavantaṃ dassanāya. Atha kho sakkassa devānamindassa etadahosi – ‘‘kahaṃ nu kho bhagavā etarahi viharati arahaṃ sammāsambuddho’’ti? Addasā kho sakko devānamindo bhagavantaṃ magadhesu viharantaṃ pācīnato rājagahassa ambasaṇḍā nāma brāhmaṇagāmo, tassuttarato vediyake pabbate indasālaguhāyaṃ. Disvāna deve tāvatiṃse āmantesi – ‘‘ayaṃ, mārisā, bhagavā magadhesu viharati, pācīnato rājagahassa ambasaṇḍā nāma brāhmaṇagāmo, tassuttarato vediyake pabbate indasālaguhāyaṃ. Yadi pana, mārisā, mayaṃ taṃ bhagavantaṃ dassanāya upasaṅkameyyāma arahantaṃ sammāsambuddha’’nti? ‘‘Evaṃ bhaddantavā’’ti kho devā tāvatiṃsā sakkassa devānamindassa paccassosuṃ.

\paragraph{345.} Atha kho sakko devānamindo pañcasikhaṃ gandhabbadevaputtaṃ\footnote{gandhabbaputtaṃ (syā.)} āmantesi – ‘‘ayaṃ, tāta pañcasikha, bhagavā magadhesu viharati pācīnato rājagahassa ambasaṇḍā nāma brāhmaṇagāmo, tassuttarato vediyake pabbate indasālaguhāyaṃ. Yadi pana , tāta pañcasikha, mayaṃ taṃ bhagavantaṃ dassanāya upasaṅkameyyāma arahantaṃ sammāsambuddha’’nti? ‘‘Evaṃ bhaddantavā’’ti kho pañcasikho gandhabbadevaputto sakkassa devānamindassa paṭissutvā beluvapaṇḍuvīṇaṃ ādāya sakkassa devānamindassa anucariyaṃ upāgami.

\paragraph{346.} Atha kho sakko devānamindo devehi tāvatiṃsehi parivuto pañcasikhena gandhabbadevaputtena purakkhato seyyathāpi nāma balavā puriso samiñjitaṃ vā bāhaṃ pasāreyya pasāritaṃ vā bāhaṃ samiñjeyya; evameva devesu tāvatiṃsesu antarahito magadhesu pācīnato rājagahassa ambasaṇḍā nāma brāhmaṇagāmo, tassuttarato vediyake pabbate paccuṭṭhāsi. Tena kho pana samayena vediyako pabbato atiriva obhāsajāto hoti ambasaṇḍā ca brāhmaṇagāmo yathā taṃ devānaṃ devānubhāvena. Apissudaṃ parito gāmesu manussā evamāhaṃsu – ‘‘ādittassu nāmajja vediyako pabbato jhāyatisu\footnote{jhāyatassu (syā.), pajjhāyitassu (sī. pī.)} nāmajja vediyako pabbato jalatisu\footnote{jalatassu (syā.), jalitassu (sī. pī.)} nāmajja vediyako pabbato kiṃsu nāmajja vediyako pabbato atiriva obhāsajāto ambasaṇḍā ca brāhmaṇagāmo’’ti saṃviggā lomahaṭṭhajātā ahesuṃ.

\paragraph{347.} Atha kho sakko devānamindo pañcasikhaṃ gandhabbadevaputtaṃ āmantesi – ‘‘durupasaṅkamā kho, tāta pañcasikha, tathāgatā mādisena, jhāyī jhānaratā, tadantaraṃ\footnote{tadanantaraṃ (sī. syā. pī. ka.)} paṭisallīnā. Yadi pana tvaṃ, tāta pañcasikha, bhagavantaṃ paṭhamaṃ pasādeyyāsi, tayā, tāta, paṭhamaṃ pasāditaṃ pacchā mayaṃ taṃ bhagavantaṃ dassanāya upasaṅkameyyāma arahantaṃ sammāsambuddha’’nti. ‘‘Evaṃ bhaddantavā’’ti kho pañcasikho gandhabbadevaputto sakkassa devānamindassa paṭissutvā beluvapaṇḍuvīṇaṃ ādāya yena indasālaguhā tenupasaṅkami; upasaṅkamitvā ‘‘ettāvatā me bhagavā neva atidūre bhavissati nāccāsanne, saddañca me sossatī’’ti ekamantaṃ aṭṭhāsi.

\subsubsection{Pañcasikhagītagāthā}

\paragraph{348.} Ekamantaṃ ṭhito kho pañcasikho gandhabbadevaputto beluvapaṇḍuvīṇaṃ\footnote{veḷuvapaṇḍuvīṇaṃ ādāya (syā.)} assāvesi, imā ca gāthā abhāsi buddhūpasañhitā dhammūpasañhitā saṅghūpasañhitā arahantūpasañhitā kāmūpasañhitā –

‘‘Vande te pitaraṃ bhadde, timbaruṃ sūriyavacchase;

Yena jātāsi kalyāṇī, ānandajananī mama.

‘‘Vātova sedataṃ kanto, pānīyaṃva pipāsato;

Aṅgīrasi piyāmesi, dhammo arahatāmiva.

‘‘Āturasseva bhesajjaṃ, bhojanaṃva jighacchato;

Parinibbāpaya maṃ bhadde, jalantamiva vārinā.

‘‘Sītodakaṃ pokkharaṇiṃ, yuttaṃ kiñjakkhareṇunā;

Nāgo ghammābhitattova, ogāhe te thanūdaraṃ.

‘‘Accaṅkusova nāgova, jitaṃ me tuttatomaraṃ;

Kāraṇaṃ nappajānāmi, sammatto lakkhaṇūruyā.

‘‘Tayi gedhitacittosmi, cittaṃ vipariṇāmitaṃ;

Paṭigantuṃ na sakkomi, vaṅkaghastova ambujo.

‘‘Vāmūru saja maṃ bhadde, saja maṃ mandalocane;

Palissaja maṃ kalyāṇi, etaṃ me abhipatthitaṃ.

‘‘Appako vata me santo, kāmo vellitakesiyā;

Anekabhāvo samuppādi, arahanteva dakkhiṇā.

‘‘Yaṃ me atthi kataṃ puññaṃ, arahantesu tādisu;

Taṃ me sabbaṅgakalyāṇi, tayā saddhiṃ vipaccataṃ.

‘‘Yaṃ me atthi kataṃ puññaṃ, asmiṃ pathavimaṇḍale;

Taṃ me sabbaṅgakalyāṇi, tayā saddhiṃ vipaccataṃ.

‘‘Sakyaputtova jhānena, ekodi nipako sato;

Amataṃ muni jigīsāno\footnote{jigiṃsāno (sī. syā. pī.)}, tamahaṃ sūriyavacchase.

‘‘Yathāpi muni nandeyya, patvā sambodhimuttamaṃ;

Evaṃ nandeyyaṃ kalyāṇi, missībhāvaṃ gato tayā.

‘‘Sakko ce me varaṃ dajjā, tāvatiṃsānamissaro;

Tāhaṃ bhadde vareyyāhe, evaṃ kāmo daḷho mama.

‘‘Sālaṃva na ciraṃ phullaṃ, pitaraṃ te sumedhase;

Vandamāno namassāmi, yassā setādisī pajā’’ti.

\paragraph{349.} Evaṃ vutte bhagavā pañcasikhaṃ gandhabbadevaputtaṃ etadavoca – ‘‘saṃsandati kho te, pañcasikha, tantissaro gītassarena, gītassaro ca tantissarena; na ca pana\footnote{neva pana (syā.)} te pañcasikha, tantissaro gītassaraṃ ativattati, gītassaro ca tantissaraṃ. Kadā saṃyūḷhā pana te, pañcasikha, imā gāthā buddhūpasañhitā dhammūpasañhitā saṅghūpasañhitā arahantūpasañhitā kāmūpasañhitā’’ti? ‘‘Ekamidaṃ, bhante, samayaṃ bhagavā uruvelāyaṃ viharati najjā nerañjarāya tīre ajapālanigrodhe paṭhamābhisambuddho . Tena kho panāhaṃ, bhante, samayena bhaddā nāma sūriyavacchasā timbaruno gandhabbarañño dhītā, tamabhikaṅkhāmi. Sā kho pana, bhante, bhaginī parakāminī hoti; sikhaṇḍī nāma mātalissa saṅgāhakassa putto, tamabhikaṅkhati. Yato kho ahaṃ, bhante, taṃ bhaginiṃ nālatthaṃ kenaci pariyāyena. Athāhaṃ beluvapaṇḍuvīṇaṃ ādāya yena timbaruno gandhabbarañño nivesanaṃ tenupasaṅkamiṃ; upasaṅkamitvā beluvapaṇḍuvīṇaṃ assāvesiṃ, imā ca gāthā abhāsiṃ buddhūpasañhitā dhammūpasañhitā saṅghūpasañhitā arahantūpasañhitā kāmūpasañhitā –

‘‘Vande te pitaraṃ bhadde, timbaruṃ sūriyavacchase;

Yena jātāsi kalyāṇī, ānandajananī mama. …pe…

Sālaṃva na ciraṃ phullaṃ, pitaraṃ te sumedhase;

Vandamāno namassāmi, yassā setādisī pajā’’ti.

‘‘Evaṃ vutte, bhante, bhaddā sūriyavacchasā maṃ etadavoca – ‘na kho me, mārisa, so bhagavā sammukhā diṭṭho api ca sutoyeva me so bhagavā devānaṃ tāvatiṃsānaṃ sudhammāyaṃ sabhāyaṃ upanaccantiyā. Yato kho tvaṃ, mārisa, taṃ bhagavantaṃ kittesi, hotu no ajja samāgamo’ti. Soyeva no, bhante, tassā bhaginiyā saddhiṃ samāgamo ahosi. Na ca dāni tato pacchā’’ti.

\subsubsection{Sakkūpasaṅkama}

\paragraph{350.} Atha kho sakkassa devānamindassa etadahosi – ‘‘paṭisammodati pañcasikho gandhabbadevaputto bhagavatā, bhagavā ca pañcasikhenā’’ti. Atha kho sakko devānamindo pañcasikhaṃ gandhabbadevaputtaṃ āmantesi – ‘‘abhivādehi me tvaṃ, tāta pañcasikha, bhagavantaṃ – ‘sakko, bhante, devānamindo sāmacco saparijano bhagavato pāde sirasā vandatī’ti’’. ‘‘Evaṃ bhaddantavā’’ti kho pañcasikho gandhabbadevaputto sakkassa devānamindassa paṭissutvā bhagavantaṃ abhivādeti – ‘‘sakko, bhante, devānamindo sāmacco saparijano bhagavato pāde sirasā vandatī’’ti. ‘‘Evaṃ sukhī hotu, pañcasikha, sakko devānamindo sāmacco saparijano; sukhakāmā hi devā manussā asurā nāgā gandhabbā ye caññe santi puthukāyā’’ti.

\paragraph{351.} Evañca pana tathāgatā evarūpe mahesakkhe yakkhe abhivadanti. Abhivadito sakko devānamindo bhagavato indasālaguhaṃ pavisitvā bhagavantaṃ abhivādetvā ekamantaṃ aṭṭhāsi. Devāpi tāvatiṃsā indasālaguhaṃ pavisitvā bhagavantaṃ abhivādetvā ekamantaṃ aṭṭhaṃsu. Pañcasikhopi gandhabbadevaputto indasālaguhaṃ pavisitvā bhagavantaṃ abhivādetvā ekamantaṃ aṭṭhāsi.

Tena kho pana samayena indasālaguhā visamā santī samā samapādi, sambādhā santī urundā\footnote{uruddā (ka.)} samapādi, andhakāro guhāyaṃ antaradhāyi, āloko udapādi yathā taṃ devānaṃ devānubhāvena.

\paragraph{352.} Atha kho bhagavā sakkaṃ devānamindaṃ etadavoca – ‘‘acchariyamidaṃ āyasmato kosiyassa, abbhutamidaṃ āyasmato kosiyassa tāva bahukiccassa bahukaraṇīyassa yadidaṃ idhāgamana’’nti. ‘‘Cirapaṭikāhaṃ, bhante, bhagavantaṃ dassanāya upasaṅkamitukāmo; api ca devānaṃ tāvatiṃsānaṃ kehici kehici\footnote{kehici (syā.)} kiccakaraṇīyehi byāvaṭo; evāhaṃ nāsakkhiṃ bhagavantaṃ dassanāya upasaṅkamituṃ. Ekamidaṃ, bhante, samayaṃ bhagavā sāvatthiyaṃ viharati salaḷāgārake. Atha khvāhaṃ, bhante, sāvatthiṃ agamāsiṃ bhagavantaṃ dassanāya. Tena kho pana, bhante, samayena bhagavā aññatarena samādhinā nisinno hoti, bhūjati\footnote{bhuñjatī ca (sī. pī.), bhujagī (syā.)} ca nāma vessavaṇassa mahārājassa paricārikā bhagavantaṃ paccupaṭṭhitā hoti, pañjalikā namassamānā tiṭṭhati. Atha khvāhaṃ, bhante, bhūjatiṃ etadavocaṃ – ‘abhivādehi me tvaṃ, bhagini, bhagavantaṃ – ‘‘sakko, bhante, devānamindo sāmacco saparijano bhagavato pāde sirasā vandatī’’ti. Evaṃ vutte, bhante, sā bhūjati maṃ etadavoca – ‘akālo kho, mārisa, bhagavantaṃ dassanāya; paṭisallīno bhagavā’ti. ‘Tena hī, bhagini, yadā bhagavā tamhā samādhimhā vuṭṭhito hoti, atha mama vacanena bhagavantaṃ abhivādehi – ‘‘sakko, bhante, devānamindo sāmacco saparijano bhagavato pāde sirasā vandatī’’ti. Kacci me sā, bhante, bhaginī bhagavantaṃ abhivādesi? Sarati bhagavā tassā bhaginiyā vacana’’nti? ‘‘Abhivādesi maṃ sā, devānaminda, bhaginī, sarāmahaṃ tassā bhaginiyā vacanaṃ. Api cāhaṃ āyasmato nemisaddena\footnote{cakkanemisaddena (syā.)} tamhā samādhimhā vuṭṭhito’’ti. ‘‘Ye te, bhante, devā amhehi paṭhamataraṃ tāvatiṃsakāyaṃ upapannā, tesaṃ me sammukhā sutaṃ sammukhā paṭiggahitaṃ – ‘yadā tathāgatā loke uppajjanti arahanto sammāsambuddhā, dibbā kāyā paripūrenti, hāyanti asurakāyā’ti. Taṃ me idaṃ, bhante, sakkhidiṭṭhaṃ yato tathāgato loke uppanno arahaṃ sammāsambuddho, dibbā kāyā paripūrenti, hāyanti asurakāyāti.

\subsubsection{Gopakavatthu}

\paragraph{353.} ‘‘Idheva, bhante, kapilavatthusmiṃ gopikā nāma sakyadhītā ahosi buddhe pasannā dhamme pasannā saṅghe pasannā sīlesu paripūrakārinī. Sā itthittaṃ\footnote{itthicittaṃ (syā.)} virājetvā purisattaṃ\footnote{purisacittaṃ (syā.)} bhāvetvā kāyassa bhedā paraṃ maraṇā sugatiṃ saggaṃ lokaṃ upapannā. Devānaṃ tāvatiṃsānaṃ sahabyataṃ amhākaṃ puttattaṃ ajjhupagatā. Tatrapi naṃ evaṃ jānanti – ‘gopako devaputto, gopako devaputto’ti. Aññepi, bhante, tayo bhikkhū bhagavati brahmacariyaṃ caritvā hīnaṃ gandhabbakāyaṃ upapannā. Te pañcahi kāmaguṇehi samappitā samaṅgībhūtā paricārayamānā amhākaṃ upaṭṭhānaṃ āgacchanti amhākaṃ pāricariyaṃ. Te amhākaṃ upaṭṭhānaṃ āgate amhākaṃ pāricariyaṃ gopako devaputto paṭicodesi – ‘kutomukhā nāma tumhe , mārisā, tassa bhagavato dhammaṃ assuttha\footnote{āyuhittha (syā.)} – ahañhi nāma itthikā samānā buddhe pasannā dhamme pasannā saṅghe pasannā sīlesu paripūrakārinī itthittaṃ virājetvā purisattaṃ bhāvetvā kāyassa bhedā paraṃ maraṇā sugatiṃ saggaṃ lokaṃ upapannā, devānaṃ tāvatiṃsānaṃ sahabyataṃ sakkassa devānamindassa puttattaṃ ajjhupagatā. Idhāpi maṃ evaṃ jānanti ‘‘gopako devaputto gopako devaputto’ti. Tumhe pana, mārisā, bhagavati brahmacariyaṃ caritvā hīnaṃ gandhabbakāyaṃ upapannā. Duddiṭṭharūpaṃ vata, bho, addasāma, ye mayaṃ addasāma sahadhammike hīnaṃ gandhabbakāyaṃ upapanne’ti. Tesaṃ, bhante, gopakena devaputtena paṭicoditānaṃ dve devā diṭṭheva dhamme satiṃ paṭilabhiṃsu kāyaṃ brahmapurohitaṃ, eko pana devo kāme ajjhāvasi.

\paragraph{354.}‘‘‘Upāsikā cakkhumato ahosiṃ,

Nāmampi mayhaṃ ahu ‘gopikā’ti;

Buddhe ca dhamme ca abhippasannā,

Saṅghañcupaṭṭhāsiṃ pasannacittā.

‘‘‘Tasseva buddhassa sudhammatāya,

Sakkassa puttomhi mahānubhāvo;

Mahājutīko tidivūpapanno,

Jānanti maṃ idhāpi ‘gopako’ti.

‘‘‘Athaddasaṃ bhikkhavo diṭṭhapubbe,

Gandhabbakāyūpagate vasīne;

Imehi te gotamasāvakāse,

Ye ca mayaṃ pubbe manussabhūtā.

‘‘‘Annena pānena upaṭṭhahimhā,

Pādūpasaṅgayha sake nivesane;

Kutomukhā nāma ime bhavanto,

Buddhassa dhammāni paṭiggahesuṃ\footnote{buddhassa dhammaṃ na paṭiggahesuṃ (syā.)}.

‘‘‘Paccattaṃ veditabbo hi dhammo,

Sudesito cakkhumatānubuddho;

Ahañhi tumheva upāsamāno,

Sutvāna ariyāna subhāsitāni.

‘‘‘Sakkassa puttomhi mahānubhāvo,

Mahājutīko tidivūpapanno;

Tumhe pana seṭṭhamupāsamānā,

Anuttaraṃ brahmacariyaṃ caritvā.

‘‘‘Hīnaṃ kāyaṃ upapannā bhavanto,

Anānulomā bhavatūpapatti;

Duddiṭṭharūpaṃ vata addasāma,

Sahadhammike hīnakāyūpapanne.

‘‘‘Gandhabbakāyūpagatā bhavanto,

Devānamāgacchatha pāricariyaṃ;

Agāre vasato mayhaṃ,

Imaṃ passa visesataṃ.

‘‘‘Itthī hutvā svajja pumomhi devo,

Dibbehi kāmehi samaṅgibhūto’;

Te coditā gotamasāvakena,

Saṃvegamāpādu samecca gopakaṃ.

‘‘‘Handa viyāyāma\footnote{vigāyāma (syā.), vitāyāma (pī.)} byāyāma\footnote{viyāyamāma (sī. pī.)},

Mā no mayaṃ parapessā ahumhā’;

Tesaṃ duve vīriyamārabhiṃsu,

Anussaraṃ gotamasāsanāni.

‘‘Idheva cittāni virājayitvā,

Kāmesu ādīnavamaddasaṃsu;

Te kāmasaṃyojanabandhanāni,

Pāpimayogāni duraccayāni.

‘‘Nāgova sannāni guṇāni\footnote{sandānaguṇāni (sī. pī.), santāni guṇāni (syā.)} chetvā,

Deve tāvatiṃse atikkamiṃsu;

Saindā devā sapajāpatikā,

Sabbe sudhammāya sabhāyupaviṭṭhā.

‘‘Tesaṃ nisinnānaṃ abhikkamiṃsu,

Vīrā virāgā virajaṃ karontā;

Te disvā saṃvegamakāsi vāsavo,

Devābhibhū devagaṇassa majjhe.

‘‘‘Imehi te hīnakāyūpapannā,

Deve tāvatiṃse abhikkamanti’;

Saṃvegajātassa vaco nisamma,

So gopako vāsavamajjhabhāsi.

‘‘‘Buddho janindatthi manussaloke,

Kāmābhibhū sakyamunīti ñāyati;

Tasseva te puttā satiyā vihīnā,

Coditā mayā te satimajjhalatthuṃ.

‘‘‘Tiṇṇaṃ tesaṃ āvasinettha\footnote{avasīnettha (pī.)} eko,

Gandhabbakāyūpagato vasīno;

Dve ca sambodhipathānusārino,

Devepi hīḷenti samāhitattā.

‘‘‘Etādisī dhammappakāsanettha,

Na tattha kiṃkaṅkhati koci sāvako;

Nitiṇṇaoghaṃ vicikicchachinnaṃ,

Buddhaṃ namassāma jinaṃ janindaṃ’.

‘‘Yaṃ te dhammaṃ idhaññāya,

Visesaṃ ajjhagaṃsu\footnote{ajjhagamaṃsu (syā.)} te;

Kāyaṃ brahmapurohitaṃ,

Duve tesaṃ visesagū.

‘‘Tassa dhammassa pattiyā,

Āgatamhāsi mārisa;

Katāvakāsā bhagavatā,

Pañhaṃ pucchemu mārisā’’ti.

\paragraph{355.} Atha kho bhagavato etadahosi – ‘‘dīgharattaṃ visuddho kho ayaṃ yakkho\footnote{sakko (sī. syā. pī.)}, yaṃ kiñci maṃ pañhaṃ pucchissati, sabbaṃ taṃ atthasañhitaṃyeva pucchissati, no anatthasañhitaṃ. Yañcassāhaṃ puṭṭho byākarissāmi, taṃ khippameva ājānissatī’’ti.

\paragraph{356.} Atha kho bhagavā sakkaṃ devānamindaṃ gāthāya ajjhabhāsi –

‘‘Puccha vāsava maṃ pañhaṃ, yaṃ kiñci manasicchasi;

Tassa tasseva pañhassa, ahaṃ antaṃ karomi te’’ti.

\subsubsection{Paṭhamabhāṇavāro niṭṭhito.}

\paragraph{357.} Katāvakāso sakko devānamindo bhagavatā imaṃ bhagavantaṃ\footnote{devānamindo bhagavantaṃ imaṃ (sī. pī.)} paṭhamaṃ pañhaṃ apucchi –

‘‘Kiṃ saṃyojanā nu kho, mārisa, devā manussā asurā nāgā gandhabbā ye caññe santi puthukāyā, te – ‘averā adaṇḍā asapattā abyāpajjā viharemu averino’ti iti ca nesaṃ hoti, atha ca pana saverā sadaṇḍā sasapattā sabyāpajjā viharanti saverino’’ti? Itthaṃ sakko devānamindo bhagavantaṃ pañhaṃ\footnote{imaṃ paṭhamaṃ pañhaṃ (sī. pī.)} apucchi. Tassa bhagavā pañhaṃ puṭṭho byākāsi –

‘‘Issāmacchariyasaṃyojanā kho, devānaminda, devā manussā asurā nāgā gandhabbā ye caññe santi puthukāyā, te – ‘averā adaṇḍā asapattā abyāpajjā viharemu averino’ti iti ca nesaṃ hoti, atha ca pana saverā sadaṇḍā sasapattā sabyāpajjā viharanti saverino’’ti. Itthaṃ bhagavā sakkassa devānamindassa pañhaṃ puṭṭho byākāsi. Attamano sakko devānamindo bhagavato bhāsitaṃ abhinandi anumodi – ‘‘evametaṃ, bhagavā, evametaṃ, sugata. Tiṇṇā mettha kaṅkhā vigatā kathaṃkathā bhagavato pañhaveyyākaraṇaṃ sutvā’’ti.

\paragraph{358.} Itiha sakko devānamindo bhagavato bhāsitaṃ abhinanditvā anumoditvā bhagavantaṃ uttariṃ\footnote{uttariṃ (sī. syā. pī.)} pañhaṃ apucchi –

‘‘Issāmacchariyaṃ pana, mārisa, kiṃnidānaṃ kiṃsamudayaṃ kiṃjātikaṃ kiṃpabhavaṃ; kismiṃ sati issāmacchariyaṃ hoti; kismiṃ asati issāmacchariyaṃ na hotī’’ti? ‘‘Issāmacchariyaṃ kho, devānaminda, piyāppiyanidānaṃ piyāppiyasamudayaṃ piyāppiyajātikaṃ piyāppiyapabhavaṃ; piyāppiye sati issāmacchariyaṃ hoti, piyāppiye asati issāmacchariyaṃ na hotī’’ti.

‘‘Piyāppiyaṃ kho pana, mārisa, kiṃnidānaṃ kiṃsamudayaṃ kiṃjātikaṃ kiṃpabhavaṃ; kismiṃ sati piyāppiyaṃ hoti; kismiṃ asati piyāppiyaṃ na hotī’’ti? ‘‘Piyāppiyaṃ kho, devānaminda, chandanidānaṃ chandasamudayaṃ chandajātikaṃ chandapabhavaṃ; chande sati piyāppiyaṃ hoti; chande asati piyāppiyaṃ na hotī’’ti.

‘‘Chando kho pana, mārisa, kiṃnidāno kiṃsamudayo kiṃjātiko kiṃpabhavo; kismiṃ sati chando hoti; kismiṃ asati chando na hotī’’ti? ‘‘Chando kho, devānaminda, vitakkanidāno vitakkasamudayo vitakkajātiko vitakkapabhavo; vitakke sati chando hoti; vitakke asati chando na hotī’’ti.

‘‘Vitakko kho pana, mārisa, kiṃnidāno kiṃsamudayo kiṃjātiko kiṃpabhavo; kismiṃ sati vitakko hoti; kismiṃ asati vitakko na hotī’’ti? ‘‘Vitakko kho, devānaminda, papañcasaññāsaṅkhānidāno papañcasaññāsaṅkhāsamudayo papañcasaññāsaṅkhājātiko papañcasaññāsaṅkhāpabhavo; papañcasaññāsaṅkhāya sati vitakko hoti; papañcasaññāsaṅkhāya asati vitakko na hotī’’ti.

‘‘Kathaṃ paṭipanno pana, mārisa, bhikkhu papañcasaññāsaṅkhānirodhasāruppagāminiṃ paṭipadaṃ paṭipanno hotī’’ti?

\subsubsection{Vedanākammaṭṭhānaṃ}

\paragraph{359.} ‘‘Somanassaṃpāhaṃ\footnote{pahaṃ (sī. pī.), cāhaṃ (syā. kaṃ.)}, devānaminda, duvidhena vadāmi – sevitabbampi, asevitabbampi. Domanassaṃpāhaṃ, devānaminda, duvidhena vadāmi – sevitabbampi, asevitabbampi. Upekkhaṃpāhaṃ, devānaminda, duvidhena vadāmi – sevitabbampi, asevitabbampi.

\paragraph{360.} ‘‘Somanassaṃpāhaṃ, devānaminda, duvidhena vadāmi sevitabbampi, asevitabbampīti iti kho panetaṃ vuttaṃ, kiñcetaṃ paṭicca vuttaṃ? Tattha yaṃ jaññā somanassaṃ ‘imaṃ kho me somanassaṃ sevato akusalā dhammā abhivaḍḍhanti, kusalā dhammā parihāyantī’ti, evarūpaṃ somanassaṃ na sevitabbaṃ. Tattha yaṃ jaññā somanassaṃ ‘imaṃ kho me somanassaṃ sevato akusalā dhammā parihāyanti, kusalā dhammā abhivaḍḍhantī’ti, evarūpaṃ somanassaṃ sevitabbaṃ. Tattha yaṃ ce savitakkaṃ savicāraṃ, yaṃ ce avitakkaṃ avicāraṃ, ye avitakke avicāre, te\footnote{se (sī. pī.)} paṇītatare. Somanassaṃpāhaṃ, devānaminda, duvidhena vadāmi sevitabbampi, asevitabbampīti. Iti yaṃ taṃ vuttaṃ, idametaṃ paṭicca vuttaṃ.

\paragraph{361.} ‘‘Domanassaṃpāhaṃ, devānaminda, duvidhena vadāmi sevitabbampi , asevitabbampīti. Iti kho panetaṃ vuttaṃ, kiñcetaṃ paṭicca vuttaṃ? Tattha yaṃ jaññā domanassaṃ ‘imaṃ kho me domanassaṃ sevato akusalā dhammā abhivaḍḍhanti, kusalā dhammā parihāyantī’ti, evarūpaṃ domanassaṃ na sevitabbaṃ. Tattha yaṃ jaññā domanassaṃ ‘imaṃ kho me domanassaṃ sevato akusalā dhammā parihāyanti, kusalā dhammā abhivaḍḍhantī’ti, evarūpaṃ domanassaṃ sevitabbaṃ. Tattha yaṃ ce savitakkaṃ savicāraṃ, yaṃ ce avitakkaṃ avicāraṃ, ye avitakke avicāre, te paṇītatare. Domanassaṃpāhaṃ, devānaminda, duvidhena vadāmi sevitabbampi, asevitabbampī’ti iti yaṃ taṃ vuttaṃ, idametaṃ paṭicca vuttaṃ.

\paragraph{362.} ‘‘Upekkhaṃpāhaṃ, devānaminda, duvidhena vadāmi sevitabbampi, asevitabbampīti iti kho panetaṃ vuttaṃ, kiñcetaṃ paṭicca vuttaṃ? Tattha yaṃ jaññā upekkhaṃ ‘imaṃ kho me upekkhaṃ sevato akusalā dhammā abhivaḍḍhanti, kusalā dhammā parihāyantī’ti, evarūpā upekkhā na sevitabbā. Tattha yaṃ jaññā upekkhaṃ ‘imaṃ kho me upekkhaṃ sevato akusalā dhammā parihāyanti, kusalā dhammā abhivaḍḍhantī’ti, evarūpā upekkhā sevitabbā. Tattha yaṃ ce savitakkaṃ savicāraṃ, yaṃ ce avitakkaṃ avicāraṃ, ye avitakke avicāre, te paṇītatare. Upekkhaṃpāhaṃ, devānaminda, duvidhena vadāmi sevitabbampi, asevitabbampīti iti yaṃ taṃ vuttaṃ, idametaṃ paṭicca vuttaṃ.

\paragraph{363.} ‘‘Evaṃ paṭipanno kho, devānaminda, bhikkhu papañcasaññāsaṅkhānirodhasāruppagāminiṃ paṭipadaṃ paṭipanno hotī’’ti. Itthaṃ bhagavā sakkassa devānamindassa pañhaṃ puṭṭho byākāsi. Attamano sakko devānamindo bhagavato bhāsitaṃ abhinandi anumodi – ‘‘evametaṃ, bhagavā, evametaṃ, sugata, tiṇṇā mettha kaṅkhā vigatā kathaṃkathā bhagavato pañhaveyyākaraṇaṃ sutvā’’ti.

\subsubsection{Pātimokkhasaṃvaro}

\paragraph{364.} Itiha sakko devānamindo bhagavato bhāsitaṃ abhinanditvā anumoditvā bhagavantaṃ uttariṃ pañhaṃ apucchi –

‘‘Kathaṃ paṭipanno pana, mārisa, bhikkhu pātimokkhasaṃvarāya paṭipanno hotī’’ti? ‘‘Kāyasamācāraṃpāhaṃ, devānaminda, duvidhena vadāmi – sevitabbampi, asevitabbampi. Vacīsamācāraṃpāhaṃ, devānaminda, duvidhena vadāmi – sevitabbampi, asevitabbampi. Pariyesanaṃpāhaṃ, devānaminda, duvidhena vadāmi – sevitabbampi, asevitabba’’mpi.

‘‘Kāyasamācāraṃpāhaṃ , devānaminda, duvidhena vadāmi sevitabbampi asevitabbampīti iti kho panetaṃ vuttaṃ, kiñcetaṃ paṭicca vuttaṃ? Tattha yaṃ jaññā kāyasamācāraṃ ‘imaṃ kho me kāyasamācāraṃ sevato akusalā dhammā abhivaḍḍhanti, kusalā dhammā parihāyantī’ti, evarūpo kāyasamācāro na sevitabbo. Tattha yaṃ jaññā kāyasamācāraṃ ‘imaṃ kho me kāyasamācāraṃ sevato akusalā dhammā parihāyanti, kusalā dhammā abhivaḍḍhantī’ti, evarūpo kāyasamācāro sevitabbo. Kāyasamācāraṃpāhaṃ, devānaminda, duvidhena vadāmi – sevitabbampi, asevitabbampīti iti yaṃ taṃ vuttaṃ, idametaṃ paṭicca vuttaṃ.

‘‘Vacīsamācāraṃpāhaṃ , devānaminda, duvidhena vadāmi – sevitabbampi, asevitabbampī’ti. Iti kho panetaṃ vuttaṃ, kiñcetaṃ paṭicca vuttaṃ? Tattha yaṃ jaññā vacīsamācāraṃ ‘imaṃ kho me vacīsamācāraṃ sevato akusalā dhammā abhivaḍḍhanti, kusalā dhammā parihāyantī’ti, evarūpo vacīsamācāro na sevitabbo. Tattha yaṃ jaññā vacīsamācāraṃ ‘imaṃ kho me vacīsamācāraṃ sevato akusalā dhammā parihāyanti, kusalā dhammā abhivaḍḍhantī’ti, evarūpo vacīsamācāro sevitabbo. Vacīsamācāraṃpāhaṃ, devānaminda, duvidhena vadāmi – sevitabbampi, asevitabbampīti iti yaṃ taṃ vuttaṃ, idametaṃ paṭicca vuttaṃ.

‘‘Pariyesanaṃpāhaṃ , devānaminda, duvidhena vadāmi – sevitabbampi, asevitabbampīti iti kho panetaṃ vuttaṃ, kiñcetaṃ paṭicca vuttaṃ? Tattha yaṃ jaññā pariyesanaṃ ‘imaṃ kho me pariyesanaṃ sevato akusalā dhammā abhivaḍḍhanti, kusalā dhammā parihāyantī’ti, evarūpā pariyesanā na sevitabbā. Tattha yaṃ jaññā pariyesanaṃ ‘imaṃ kho me pariyesanaṃ sevato akusalā dhammā parihāyanti, kusalā dhammā abhivaḍḍhantī’ti, evarūpā pariyesanā sevitabbā. Pariyesanaṃpāhaṃ, devānaminda, duvidhena vadāmi – sevitabbampi, asevitabbampīti iti yaṃ taṃ vuttaṃ, idametaṃ paṭicca vuttaṃ.

‘‘Evaṃ paṭipanno kho, devānaminda, bhikkhu pātimokkhasaṃvarāya paṭipanno hotī’’ti. Itthaṃ bhagavā sakkassa devānamindassa pañhaṃ puṭṭho byākāsi. Attamano sakko devānamindo bhagavato bhāsitaṃ abhinandi anumodi – ‘‘evametaṃ, bhagavā, evametaṃ, sugata. Tiṇṇā mettha kaṅkhā vigatā kathaṃkathā bhagavato pañhaveyyākaraṇaṃ sutvā’’ti.

\subsubsection{Indriyasaṃvaro}

\paragraph{365.} Itiha sakko devānamindo bhagavato bhāsitaṃ abhinanditvā anumoditvā bhagavantaṃ uttariṃ pañhaṃ apucchi –

‘‘Kathaṃ paṭipanno pana, mārisa, bhikkhu indriyasaṃvarāya paṭipanno hotī’’ti? ‘‘Cakkhuviññeyyaṃ rūpaṃpāhaṃ, devānaminda, duvidhena vadāmi – sevitabbampi, asevitabbampi. Sotaviññeyyaṃ saddaṃpāhaṃ, devānaminda, duvidhena vadāmi – sevitabbampi, asevitabbampi. Ghānaviññeyyaṃ gandhaṃpāhaṃ, devānaminda, duvidhena vadāmi – sevitabbampi, asevitabbampi. Jivhāviññeyyaṃ rasaṃpāhaṃ, devānaminda, duvidhena vadāmi – sevitabbampi, asevitabbampi. Kāyaviññeyyaṃ phoṭṭhabbaṃpāhaṃ, devānaminda, duvidhena vadāmi – sevitabbampi, asevitabbampi. Manoviññeyyaṃ dhammaṃpāhaṃ, devānaminda, duvidhena vadāmi – sevitabbampi, asevitabbampī’’ti.

Evaṃ vutte, sakko devānamindo bhagavantaṃ etadavoca –

‘‘Imassa kho ahaṃ, bhante, bhagavatā saṅkhittena bhāsitassa evaṃ vitthārena atthaṃ ājānāmi. Yathārūpaṃ, bhante, cakkhuviññeyyaṃ rūpaṃ sevato akusalā dhammā abhivaḍḍhanti, kusalā dhammā parihāyanti, evarūpaṃ cakkhuviññeyyaṃ rūpaṃ na sevitabbaṃ . Yathārūpañca kho, bhante, cakkhuviññeyyaṃ rūpaṃ sevato akusalā dhammā parihāyanti, kusalā dhammā abhivaḍḍhanti, evarūpaṃ cakkhuviññeyyaṃ rūpaṃ sevitabbaṃ. Yathārūpañca kho, bhante, sotaviññeyyaṃ saddaṃ sevato…pe… ghānaviññeyyaṃ gandhaṃ sevato… jivhāviññeyyaṃ rasaṃ sevato… kāyaviññeyyaṃ phoṭṭhabbaṃ sevato… manoviññeyyaṃ dhammaṃ sevato akusalā dhammā abhivaḍḍhanti, kusalā dhammā parihāyanti, evarūpo manoviññeyyo dhammo na sevitabbo. Yathārūpañca kho, bhante, manoviññeyyaṃ dhammaṃ sevato akusalā dhammā parihāyanti, kusalā dhammā abhivaḍḍhanti, evarūpo manoviññeyyo dhammo sevitabbo.

‘‘Imassa kho me, bhante, bhagavatā saṅkhittena bhāsitassa evaṃ vitthārena atthaṃ ājānato tiṇṇā mettha kaṅkhā vigatā kathaṃkathā bhagavato pañhaveyyākaraṇaṃ sutvā’’ti.

\paragraph{366.} Itiha sakko devānamindo bhagavato bhāsitaṃ abhinanditvā anumoditvā bhagavantaṃ uttariṃ pañhaṃ apucchi –

‘‘Sabbeva nu kho, mārisa, samaṇabrāhmaṇā ekantavādā ekantasīlā ekantachandā ekantaajjhosānā’’ti? ‘‘Na kho, devānaminda, sabbe samaṇabrāhmaṇā ekantavādā ekantasīlā ekantachandā ekantaajjhosānā’’ti.

‘‘Kasmā pana, mārisa, na sabbe samaṇabrāhmaṇā ekantavādā ekantasīlā ekantachandā ekantaajjhosānā’’ti? ‘‘Anekadhātu nānādhātu kho, devānaminda, loko. Tasmiṃ anekadhātunānādhātusmiṃ loke yaṃ yadeva sattā dhātuṃ abhinivisanti, taṃ tadeva thāmasā parāmāsā abhinivissa voharanti – ‘idameva saccaṃ moghamañña’nti. Tasmā na sabbe samaṇabrāhmaṇā ekantavādā ekantasīlā ekantachandā ekantaajjhosānā’’ti.

‘‘Sabbeva nu kho, mārisa, samaṇabrāhmaṇā accantaniṭṭhā accantayogakkhemī accantabrahmacārī accantapariyosānā’’ti? ‘‘Na kho, devānaminda, sabbe samaṇabrāhmaṇā accantaniṭṭhā accantayogakkhemī accantabrahmacārī accantapariyosānā’’ti.

‘‘Kasmā pana, mārisa, na sabbe samaṇabrāhmaṇā accantaniṭṭhā accantayogakkhemī accantabrahmacārī accantapariyosānā’’ti? ‘‘Ye kho, devānaminda, bhikkhū taṇhāsaṅkhayavimuttā te accantaniṭṭhā accantayogakkhemī accantabrahmacārī accantapariyosānā. Tasmā na sabbe samaṇabrāhmaṇā accantaniṭṭhā accantayogakkhemī accantabrahmacārī accantapariyosānā’’ti.

Itthaṃ bhagavā sakkassa devānamindassa pañhaṃ puṭṭho byākāsi. Attamano sakko devānamindo bhagavato bhāsitaṃ abhinandi anumodi – ‘‘evametaṃ, bhagavā, evametaṃ, sugata. Tiṇṇā mettha kaṅkhā vigatā kathaṃkathā bhagavato pañhaveyyākaraṇaṃ sutvā’’ti.

\paragraph{367.} Itiha sakko devānamindo bhagavato bhāsitaṃ abhinanditvā anumoditvā bhagavantaṃ etadavoca –

‘‘Ejā, bhante, rogo, ejā gaṇḍo, ejā sallaṃ, ejā imaṃ purisaṃ parikaḍḍhati tassa tasseva bhavassa abhinibbattiyā. Tasmā ayaṃ puriso uccāvacamāpajjati . Yesāhaṃ, bhante, pañhānaṃ ito bahiddhā aññesu samaṇabrāhmaṇesu okāsakammampi nālatthaṃ, te me bhagavatā byākatā. Dīgharattānusayitañca pana\footnote{dīgharattānupassatā, yañca pana (syā.), dīgharattānusayino, yañca pana (sī. pī.)} me vicikicchākathaṃkathāsallaṃ, tañca bhagavatā abbuḷha’’nti.

‘‘Abhijānāsi no tvaṃ, devānaminda, ime pañhe aññe samaṇabrāhmaṇe pucchitā’’ti? ‘‘Abhijānāmahaṃ, bhante, ime pañhe aññe samaṇabrāhmaṇe pucchitā’’ti. ‘‘Yathā kathaṃ pana te, devānaminda, byākaṃsu? Sace te agaru bhāsassū’’ti. ‘‘Na kho me, bhante, garu yatthassa bhagavā nisinno bhagavantarūpo vā’’ti. ‘‘Tena hi, devānaminda, bhāsassū’’ti. ‘‘Yesvāhaṃ\footnote{yesāhaṃ (sī. syā. pī.)}, bhante , maññāmi samaṇabrāhmaṇā āraññikā pantasenāsanāti, tyāhaṃ upasaṅkamitvā ime pañhe pucchāmi, te mayā puṭṭhā na sampāyanti, asampāyantā mamaṃyeva paṭipucchanti – ‘ko nāmo āyasmā’ti? Tesāhaṃ puṭṭho byākaromi – ‘ahaṃ kho, mārisa, sakko devānamindo’ti. Te mamaṃyeva uttari paṭipucchanti – ‘kiṃ panāyasmā, devānaminda\footnote{devānamindo (sī. pī.)}, kammaṃ katvā imaṃ ṭhānaṃ patto’ti? Tesāhaṃ yathāsutaṃ yathāpariyattaṃ dhammaṃ desemi. Te tāvatakeneva attamanā honti – ‘sakko ca no devānamindo diṭṭho, yañca no apucchimhā, tañca no byākāsī’ti. Te aññadatthu mamaṃyeva sāvakā sampajjanti, na cāhaṃ tesaṃ. Ahaṃ kho pana, bhante, bhagavato sāvako sotāpanno avinipātadhammo niyato sambodhiparāyaṇo’’ti .

\subsubsection{Somanassapaṭilābhakathā}

\paragraph{368.} ‘‘Abhijānāsi no tvaṃ, devānaminda, ito pubbe evarūpaṃ vedapaṭilābhaṃ somanassapaṭilābha’’nti? ‘‘Abhijānāmahaṃ , bhante, ito pubbe evarūpaṃ vedapaṭilābhaṃ somanassapaṭilābha’’nti. ‘‘Yathā kathaṃ pana tvaṃ, devānaminda, abhijānāsi ito pubbe evarūpaṃ vedapaṭilābhaṃ somanassapaṭilābha’’nti?

‘‘Bhūtapubbaṃ, bhante, devāsurasaṅgāmo samupabyūḷho\footnote{samūpabbuḷho (sī. pī.)} ahosi. Tasmiṃ kho pana, bhante, saṅgāme devā jiniṃsu, asurā parājayiṃsu\footnote{parājiṃsu (sī. pī.)}. Tassa mayhaṃ, bhante, taṃ saṅgāmaṃ abhivijinitvā vijitasaṅgāmassa etadahosi – ‘yā ceva dāni dibbā ojā yā ca asurā ojā, ubhayametaṃ\footnote{ubhayamettha (syā.)} devā paribhuñjissantī’ti. So kho pana me, bhante, vedapaṭilābho somanassapaṭilābho sadaṇḍāvacaro sasatthāvacaro na nibbidāya na virāgāya na nirodhāya na upasamāya na abhiññāya na sambodhāya na nibbānāya saṃvattati. Yo kho pana me ayaṃ, bhante, bhagavato dhammaṃ sutvā vedapaṭilābho somanassapaṭilābho, so adaṇḍāvacaro asatthāvacaro ekantanibbidāya virāgāya nirodhāya upasamāya abhiññāya sambodhāya nibbānāya saṃvattatī’’ti.

\paragraph{369.} ‘‘Kiṃ pana tvaṃ, devānaminda, atthavasaṃ sampassamāno evarūpaṃ vedapaṭilābhaṃ somanassapaṭilābhaṃ pavedesī’’ti? ‘‘Cha kho ahaṃ, bhante, atthavase sampassamāno evarūpaṃ vedapaṭilābhaṃ somanassapaṭilābhaṃ pavedemi.

‘‘Idheva tiṭṭhamānassa, devabhūtassa me sato;

Punarāyu ca me laddho, evaṃ jānāhi mārisa.

‘‘Imaṃ kho ahaṃ, bhante, paṭhamaṃ atthavasaṃ sampassamāno evarūpaṃ vedapaṭilābhaṃ somanassapaṭilābhaṃ pavedemi.

‘‘Cutāhaṃ diviyā kāyā, āyuṃ hitvā amānusaṃ;

Amūḷho gabbhamessāmi, yattha me ramatī mano.

‘‘Imaṃ kho ahaṃ, bhante, dutiyaṃ atthavasaṃ sampassamāno evarūpaṃ vedapaṭilābhaṃ somanassapaṭilābhaṃ pavedemi.

‘‘Svāhaṃ amūḷhapaññassa\footnote{amūḷhapañhassa (?)}, viharaṃ sāsane rato;

Ñāyena viharissāmi, sampajāno paṭissato.

‘‘Imaṃ kho ahaṃ, bhante, tatiyaṃ atthavasaṃ sampassamāno evarūpaṃ vedapaṭilābhaṃ somanassapaṭilābhaṃ pavedemi.

‘‘Ñāyena me carato ca, sambodhi ce bhavissati;

Aññātā viharissāmi, sveva anto bhavissati.

‘‘Imaṃ kho ahaṃ, bhante, catutthaṃ atthavasaṃ sampassamāno evarūpaṃ vedapaṭilābhaṃ somanassapaṭilābhaṃ pavedemi.

‘‘Cutāhaṃ mānusā kāyā, āyuṃ hitvāna mānusaṃ;

Puna devo bhavissāmi, devalokamhi uttamo.

‘‘Imaṃ kho ahaṃ, bhante, pañcamaṃ atthavasaṃ sampassamāno evarūpaṃ vedapaṭilābhaṃ somanassapaṭilābhaṃ pavedemi.

‘‘Te\footnote{ye (?)} paṇītatarā devā, akaniṭṭhā yasassino;

Antime vattamānamhi, so nivāso bhavissati.

‘‘Imaṃ kho ahaṃ, bhante, chaṭṭhaṃ atthavasaṃ sampassamāno evarūpaṃ vedapaṭilābhaṃ somanassapaṭilābhaṃ pavedemi.

‘‘Ime kho ahaṃ, bhante, cha atthavase sampassamāno evarūpaṃ vedapaṭilābhaṃ somanassapaṭilābhaṃ pavedemi.

\paragraph{370.}‘‘Apariyositasaṅkappo , vicikiccho kathaṃkathī.

Vicariṃ dīghamaddhānaṃ, anvesanto tathāgataṃ.

‘‘Yassu maññāmi samaṇe, pavivittavihārino;

Sambuddhā iti maññāno, gacchāmi te upāsituṃ.

‘‘‘Kathaṃ ārādhanā hoti, kathaṃ hoti virādhanā’;

Iti puṭṭhā na sampāyanti\footnote{sambhonti (syā.)}, magge paṭipadāsu ca.

‘‘Tyassu yadā maṃ jānanti, sakko devānamāgato;

Tyassu mameva pucchanti, ‘kiṃ katvā pāpuṇī idaṃ’.

‘‘Tesaṃ yathāsutaṃ dhammaṃ, desayāmi jane sutaṃ\footnote{janesuta (ka. sī.)};

Tena attamanā honti, ‘diṭṭho no vāsavoti ca’.

‘‘Yadā ca buddhamaddakkhiṃ, vicikicchāvitāraṇaṃ;

Somhi vītabhayo ajja, sambuddhaṃ payirupāsiya\footnote{payirupāsayiṃ (syā. ka.)}.

‘‘Taṇhāsallassa hantāraṃ, buddhaṃ appaṭipuggalaṃ;

Ahaṃ vande mahāvīraṃ, buddhamādiccabandhunaṃ.

‘‘Yaṃ karomasi brahmuno, samaṃ devehi mārisa;

Tadajja tuyhaṃ kassāma\footnote{dassāma (syā. ka.)}, handa sāmaṃ karoma te.

‘‘Tvameva asi\footnote{tuvamevasi (pī.)} sambuddho, tuvaṃ satthā anuttaro;

Sadevakasmiṃ lokasmiṃ, natthi te paṭipuggalo’’ti.

\paragraph{371.} Atha kho sakko devānamindo pañcasikhaṃ gandhabbaputtaṃ āmantesi – ‘‘bahūpakāro kho mesi tvaṃ, tāta pañcasikha, yaṃ tvaṃ bhagavantaṃ paṭhamaṃ pasādesi. Tayā, tāta, paṭhamaṃ pasāditaṃ pacchā mayaṃ taṃ bhagavantaṃ dassanāya upasaṅkamimhā arahantaṃ sammāsambuddhaṃ. Pettike vā ṭhāne ṭhapayissāmi , gandhabbarājā bhavissasi, bhaddañca te sūriyavacchasaṃ dammi, sā hi te abhipatthitā’’ti.

Atha kho sakko devānamindo pāṇinā pathaviṃ parāmasitvā tikkhattuṃ udānaṃ udānesi – ‘‘namo tassa bhagavato arahato sammāsambuddhassā’’ti.

Imasmiñca pana veyyākaraṇasmiṃ bhaññamāne sakkassa devānamindassa virajaṃ vītamalaṃ dhammacakkhuṃ udapādi – ‘‘yaṃ kiñci samudayadhammaṃ, sabbaṃ taṃ nirodhadhamma’’nti. Aññesañca asītiyā devatāsahassānaṃ , iti ye sakkena devānamindena ajjhiṭṭhapañhā puṭṭhā , te bhagavatā byākatā. Tasmā imassa veyyākaraṇassa sakkapañhātveva adhivacananti.

\xsectionEnd{Sakkapañhasuttaṃ niṭṭhitaṃ aṭṭhamaṃ.}
