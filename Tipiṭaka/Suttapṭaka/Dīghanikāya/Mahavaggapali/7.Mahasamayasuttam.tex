\section{Mahāsamayasuttaṃ}

\paragraph{331.} Evaṃ me sutaṃ – ekaṃ samayaṃ bhagavā sakkesu viharati kapilavatthusmiṃ mahāvane mahatā bhikkhusaṅghena saddhiṃ pañcamattehi bhikkhusatehi sabbeheva arahantehi; dasahi ca lokadhātūhi devatā yebhuyyena sannipatitā honti bhagavantaṃ dassanāya bhikkhusaṅghañca. Atha kho catunnaṃ suddhāvāsakāyikānaṃ devatānaṃ\footnote{devānaṃ (sī. syā. pī.)} etadahosi – ‘‘ayaṃ kho bhagavā sakkesu viharati kapilavatthusmiṃ mahāvane mahatā bhikkhusaṅghena saddhiṃ pañcamattehi bhikkhusatehi sabbeheva arahantehi; dasahi ca lokadhātūhi devatā yebhuyyena sannipatitā honti bhagavantaṃ dassanāya bhikkhusaṅghañca. Yaṃnūna mayampi yena bhagavā tenupasaṅkameyyāma; upasaṅkamitvā bhagavato santike paccekaṃ gāthaṃ\footnote{paccekagāthaṃ (sī. syā. pī.), paccekagāthā (ka. sī.)} bhāseyyāmā’’ti.

\paragraph{332.} Atha kho tā devatā seyyathāpi nāma balavā puriso samiñjitaṃ vā bāhaṃ pasāreyya pasāritaṃ vā bāhaṃ samiñjeyya , evameva suddhāvāsesu devesu antarahitā bhagavato purato pāturahesuṃ. Atha kho tā devatā bhagavantaṃ abhivādetvā ekamantaṃ aṭṭhaṃsu. Ekamantaṃ ṭhitā kho ekā devatā bhagavato santike imaṃ gāthaṃ abhāsi –

‘‘Mahāsamayo pavanasmiṃ, devakāyā samāgatā;

Āgatamha imaṃ dhammasamayaṃ, dakkhitāye aparājitasaṅgha’’nti.

Atha kho aparā devatā bhagavato santike imaṃ gāthaṃ abhāsi –

‘‘Tatra bhikkhavo samādahaṃsu, cittamattano ujukaṃ akaṃsu\footnote{ujukamakaṃsu (sī. syā. pī.)};

Sārathīva nettāni gahetvā, indriyāni rakkhanti paṇḍitā’’ti.

Atha kho aparā devatā bhagavato santike imaṃ gāthaṃ abhāsi –

‘‘Chetvā khīlaṃ chetvā palighaṃ, indakhīlaṃ ūhacca\footnote{uhacca (ka.)} manejā;

Te caranti suddhā vimalā, cakkhumatā sudantā susunāgā’’ti.

Atha kho aparā devatā bhagavato santike imaṃ gāthaṃ abhāsi –

‘‘Yekeci buddhaṃ saraṇaṃ gatāse, na te gamissanti apāyabhūmiṃ;

Pahāya mānusaṃ dehaṃ, devakāyaṃ paripūressantī’’ti.

\subsubsection{Devatāsannipātā}

\paragraph{333.} Atha kho bhagavā bhikkhū āmantesi – ‘‘yebhuyyena, bhikkhave, dasasu lokadhātūsu devatā sannipatitā honti\footnote{( ) sī. ipotthakesu natthi}, tathāgataṃ dassanāya bhikkhusaṅghañca . Yepi te, bhikkhave, ahesuṃ atītamaddhānaṃ arahanto sammāsambuddhā, tesampi bhagavantānaṃ etaṃparamāyeva\footnote{etaparamāyeva (sī. syā. pī.)} devatā sannipatitā ahesuṃ seyyathāpi mayhaṃ etarahi. Yepi te, bhikkhave, bhavissanti anāgatamaddhānaṃ arahanto sammāsambuddhā, tesampi bhagavantānaṃ etaṃparamāyeva devatā sannipatitā bhavissanti seyyathāpi mayhaṃ etarahi. Ācikkhissāmi, bhikkhave, devakāyānaṃ nāmāni; kittayissāmi, bhikkhave, devakāyānaṃ nāmāni; desessāmi, bhikkhave, devakāyānaṃ nāmāni. Taṃ suṇātha, sādhukaṃ manasikarotha, bhāsissāmī’’ti. ‘‘Evaṃ, bhante’’ti kho te bhikkhū bhagavato paccassosuṃ.

\paragraph{334.} Bhagavā etadavoca –

‘‘Silokamanukassāmi, yattha bhummā tadassitā;

Ye sitā girigabbharaṃ, pahitattā samāhitā.

‘‘Puthūsīhāva sallīnā, lomahaṃsābhisambhuno;

Odātamanasā suddhā, vippasannamanāvilā’’\footnote{vippasannāmanāvilā (pī. ka.)}.

Bhiyyo pañcasate ñatvā, vane kāpilavatthave;

Tato āmantayī satthā, sāvake sāsane rate.

‘‘Devakāyā abhikkantā, te vijānātha bhikkhavo’’;

Te ca ātappamakaruṃ, sutvā buddhassa sāsanaṃ.

Tesaṃ pāturahu ñāṇaṃ, amanussānadassanaṃ;

Appeke satamaddakkhuṃ, sahassaṃ atha sattariṃ.

Sataṃ eke sahassānaṃ, amanussānamaddasuṃ;

Appekenantamaddakkhuṃ , disā sabbā phuṭā ahuṃ.

Tañca sabbaṃ abhiññāya, vavatthitvāna\footnote{vavakkhitvāna (sī. syā. pī.), avekkhitvāna (ṭīkā)} cakkhumā;

Tato āmantayī satthā, sāvake sāsane rate.

‘‘Devakāyā abhikkantā, te vijānātha bhikkhavo;

Ye vohaṃ kittayissāmi, girāhi anupubbaso.

\paragraph{335.}‘‘Sattasahassā te yakkhā, bhummā kāpilavatthavā.

Iddhimanto jutimanto, vaṇṇavanto yasassino;

Modamānā abhikkāmuṃ, bhikkhūnaṃ samitiṃ vanaṃ.

‘‘Chasahassā hemavatā, yakkhā nānattavaṇṇino;

Iddhimanto jutīmanto\footnote{jutīmanto (sī. pī.)}, vaṇṇavanto yasassino;

Modamānā abhikkāmuṃ, bhikkhūnaṃ samitiṃ vanaṃ.

‘‘Sātāgirā tisahassā, yakkhā nānattavaṇṇino;

Iddhimanto jutimanto, vaṇṇavanto yasassino;

Modamānā abhikkāmuṃ, bhikkhūnaṃ samitiṃ vanaṃ.

‘‘Iccete soḷasasahassā, yakkhā nānattavaṇṇino;

Iddhimanto jutimanto, vaṇṇavanto yasassino;

Modamānā abhikkāmuṃ, bhikkhūnaṃ samitiṃ vanaṃ.

‘‘Vessāmittā pañcasatā, yakkhā nānattavaṇṇino;

Iddhimanto jutimanto, vaṇṇavanto yasassino;

Modamānā abhikkāmuṃ, bhikkhūnaṃ samitiṃ vanaṃ.

‘‘Kumbhīro rājagahiko, vepullassa nivesanaṃ;

Bhiyyo naṃ satasahassaṃ, yakkhānaṃ payirupāsati;

Kumbhīro rājagahiko, sopāgā samitiṃ vanaṃ.

\paragraph{336.}‘‘Purimañca disaṃ rājā, dhataraṭṭho pasāsati.

Gandhabbānaṃ adhipati, mahārājā yasassiso.

‘‘Puttāpi tassa bahavo, indanāmā mahabbalā;

Iddhimanto jutimanto, vaṇṇavanto yasassino;

Modamānā abhikkāmuṃ, bhikkhūnaṃ samitiṃ vanaṃ.

‘‘Dakkhiṇañca disaṃ rājā, virūḷho taṃ pasāsati\footnote{tappasāsati (syā.)};

Kumbhaṇḍānaṃ adhipati, mahārājā yasassiso.

‘‘Puttāpi tassa bahavo, indanāmā mahabbalā;

Iddhimanto jutimanto, vaṇṇavanto yasassino;

Modamānā abhikkāmuṃ, bhikkhūnaṃ samitiṃ vanaṃ.

‘‘Pacchimañca disaṃ rājā, virūpakkho pasāsati;

Nāgānañca adhipati, mahārājā yasassiso.

‘‘Puttāpi tassa bahavo, indanāmā mahabbalā;

Iddhimanto jutimanto, vaṇṇavanto yasassino;

Modamānā abhikkāmuṃ, bhikkhūnaṃ samitiṃ vanaṃ.

‘‘Uttarañca disaṃ rājā, kuvero taṃ pasāsati;

Yakkhānañca adhipati, mahārājā yasassiso.

‘‘Puttāpi tassa bahavo, indanāmā mahabbalā;

Iddhimanto jutimanto, vaṇṇavanto yasassino;

Modamānā abhikkāmuṃ, bhikkhūnaṃ samitiṃ vanaṃ.

‘‘Purimaṃ disaṃ dhataraṭṭho, dakkhiṇena virūḷhako;

Pacchimena virūpakkho, kuvero uttaraṃ disaṃ.

‘‘Cattāro te mahārājā, samantā caturo disā;

Daddallamānā\footnote{daddaḷhamānā (ka.)} aṭṭhaṃsu, vane kāpilavatthave.

\paragraph{337.}‘‘Tesaṃ māyāvino dāsā, āguṃ\footnote{āgū (syā.), āgu (sī. pī.) evamuparipi} vañcanikā saṭhā.

Māyā kuṭeṇḍu viṭeṇḍu\footnote{veṭeṇḍu (sī. syā. pī.)}, viṭucca\footnote{viṭū ca (syā.)} viṭuṭo saha.

‘‘Candano kāmaseṭṭho ca, kinnighaṇḍu\footnote{kinnughaṇḍu (sī. syā. pī.)} nighaṇḍu ca;

Panādo opamañño ca, devasūto ca mātali.

‘‘Cittaseno ca gandhabbo, naḷorājā janesabho\footnote{janosabho (syā.)};

Āgā pañcasikho ceva, timbarū sūriyavaccasā\footnote{suriyavaccasā (sī. pī.)}.

‘‘Ete caññe ca rājāno, gandhabbā saha rājubhi;

Modamānā abhikkāmuṃ, bhikkhūnaṃ samitiṃ vanaṃ.

\paragraph{338.}‘‘Athāguṃ nāgasā nāgā, vesālā sahatacchakā.

Kambalassatarā āguṃ, pāyāgā saha ñātibhi.

‘‘Yāmunā dhataraṭṭhā ca, āgū nāgā yasassino;

Erāvaṇo mahānāgo, sopāgā samitiṃ vanaṃ.

‘‘Ye nāgarāje sahasā haranti, dibbā dijā pakkhi visuddhacakkhū;

Vehāyasā\footnote{vehāsayā (sī. pī.)} te vanamajjhapattā, citrā supaṇṇā iti tesa nāmaṃ.

‘‘Abhayaṃ tadā nāgarājānamāsi, supaṇṇato khemamakāsi buddho;

Saṇhāhi vācāhi upavhayantā, nāgā supaṇṇā saraṇamakaṃsu buddhaṃ.

\paragraph{339.}‘‘Jitā vajirahatthena, samuddaṃ asurāsitā.

Bhātaro vāsavassete, iddhimanto yasassino.

‘‘Kālakañcā mahābhismā\footnote{kālakañjā mahābhiṃsā (sī. pī.)}, asurā dānaveghasā;

Vepacitti sucitti ca, pahārādo namucī saha.

‘‘Satañca baliputtānaṃ, sabbe verocanāmakā;

Sannayhitvā balisenaṃ\footnote{balīsenaṃ (syā.)}, rāhubhaddamupāgamuṃ;

Samayodāni bhaddante, bhikkhūnaṃ samitiṃ vanaṃ.

\paragraph{340.}‘‘Āpo ca devā pathavī, tejo vāyo tadāgamuṃ.

Varuṇā vāraṇā\footnote{vāruṇā (syā.)} devā, somo ca yasasā saha.

‘‘Mettā karuṇā kāyikā, āguṃ devā yasassino;

Dasete dasadhā kāyā, sabbe nānattavaṇṇino.

‘‘Iddhimanto jutimanto, vaṇṇavanto yasassino;

Modamānā abhikkāmuṃ, bhikkhūnaṃ samitiṃ vanaṃ.

‘‘Veṇḍudevā sahali ca\footnote{veṇhūca devā sahalīca (sī. pī.)}, asamā ca duve yamā;

Candassūpanisā devā, candamāguṃ purakkhatvā.

‘‘Sūriyassūpanisā\footnote{suriyassūpanisā (sī. syā. pī.)} devā, sūriyamāguṃ purakkhatvā;

Nakkhattāni purakkhatvā, āguṃ mandavalāhakā.

‘‘Vasūnaṃ vāsavo seṭṭho, sakkopāgā purindado;

Dasete dasadhā kāyā, sabbe nānattavaṇṇino.

‘‘Iddhimanto jutimanto, vaṇṇavanto yasassino;

Modamānā abhikkāmuṃ, bhikkhūnaṃ samitiṃ vanaṃ.

‘‘Athāguṃ sahabhū devā, jalamaggisikhāriva;

Ariṭṭhakā ca rojā ca, umāpupphanibhāsino.

‘‘Varuṇā sahadhammā ca, accutā ca anejakā;

Sūleyyarucirā āguṃ, āguṃ vāsavanesino;

Dasete dasadhā kāyā, sabbe nānattavaṇṇino.

‘‘Iddhimanto jutimanto, vaṇṇavanto yasassino;

Modamānā abhikkāmuṃ, bhikkhūnaṃ samitiṃ vanaṃ.

‘‘Samānā mahāsamanā, mānusā mānusuttamā;

Khiḍḍāpadosikā āguṃ, āguṃ manopadosikā.

‘‘Athāguṃ harayo devā, ye ca lohitavāsino;

Pāragā mahāpāragā, āguṃ devā yasassino;

Dasete dasadhā kāyā, sabbe nānattavaṇṇino.

‘‘Iddhimanto jutimanto, vaṇṇavanto yasassino;

Modamānā abhikkāmuṃ, bhikkhūnaṃ samitiṃ vanaṃ.

‘‘Sukkā karambhā\footnote{karumhā (sī. syā. pī.)} aruṇā, āguṃ veghanasā saha;

Odātagayhā pāmokkhā, āguṃ devā vicakkhaṇā.

‘‘Sadāmattā hāragajā, missakā ca yasassino;

Thanayaṃ āga pajjunno, yo disā abhivassati.

‘‘Dasete dasadhā kāyā, sabbe nānattavaṇṇino;

Iddhimanto jutimanto, vaṇṇavanto yasassino;

Modamānā abhikkāmuṃ, bhikkhūnaṃ samitiṃ vanaṃ.

‘‘Khemiyā tusitā yāmā, kaṭṭhakā ca yasassino;

Lambītakā lāmaseṭṭhā, jotināmā ca āsavā;

Nimmānaratino āguṃ, athāguṃ paranimmitā.

‘‘Dasete dasadhā kāyā, sabbe nānattavaṇṇino;

Iddhimanto jutimanto, vaṇṇavanto yasassino;

Modamānā abhikkāmuṃ, bhikkhūnaṃ samitiṃ vanaṃ.

‘‘Saṭṭhete devanikāyā, sabbe nānattavaṇṇino;

Nāmanvayena āgacchuṃ\footnote{āgañchuṃ (sī. syā. pī.)}, ye caññe sadisā saha.

‘‘‘Pavuṭṭhajātimakhilaṃ\footnote{pavutthajātiṃ akhilaṃ (sī. pī.)}, oghatiṇṇamanāsavaṃ;

Dakkhemoghataraṃ nāgaṃ, candaṃva asitātigaṃ’.

\paragraph{341.}‘‘Subrahmā paramatto ca\footnote{paramattho ca (ka.)}, puttā iddhimato saha.

Sanaṅkumāro tisso ca, sopāga samitiṃ vanaṃ.

‘‘Sahassaṃ brahmalokānaṃ, mahābrahmābhitiṭṭhati;

Upapanno jutimanto, bhismākāyo yasassiso.

‘‘Dasettha issarā āguṃ, paccekavasavattino;

Tesañca majjhato āga, hārito parivārito.

\paragraph{342.}‘‘Te ca sabbe abhikkante, sainde\footnote{sinde (syā.)} deve sabrahmake.

Mārasenā abhikkāmi, passa kaṇhassa mandiyaṃ.

‘‘‘Etha gaṇhatha bandhatha, rāgena baddhamatthu vo;

Samantā parivāretha, mā vo muñcittha koci naṃ’.

‘‘Iti tattha mahāseno, kaṇho senaṃ apesayi;

Pāṇinā talamāhacca, saraṃ katvāna bheravaṃ.

‘‘Yathā pāvussako megho, thanayanto savijjuko; +

Tadā so paccudāvatti, saṅkuddho asayaṃvase\footnote{asayaṃvasī (sī. pī.)}.

\paragraph{343.} Tañca sabbaṃ abhiññāya, vavatthitvāna cakkhumā.

Tato āmantayī satthā, sāvake sāsane rate.

‘‘Mārasenā abhikkantā, te vijānātha bhikkhavo;

Te ca ātappamakaruṃ, sutvā buddhassa sāsanaṃ;

Vītarāgehi pakkāmuṃ, nesaṃ lomāpi iñjayuṃ.

‘‘‘Sabbe vijitasaṅgāmā, bhayātītā yasassino;

Modanti saha bhūtehi, sāvakā te janesutā’’ti.

\xsectionEnd{Mahāsamayasuttaṃ niṭṭhitaṃ sattamaṃ.}
