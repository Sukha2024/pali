\section{Saṅgītisuttaṃ}

\paragraph{296.} Evaṃ me sutaṃ – ekaṃ samayaṃ bhagavā mallesu cārikaṃ caramāno mahatā bhikkhusaṅghena saddhiṃ pañcamattehi bhikkhusatehi yena pāvā nāma mallānaṃ nagaraṃ tadavasari. Tatra sudaṃ bhagavā pāvāyaṃ viharati cundassa kammāraputtassa ambavane.

\subsubsection{Ubbhatakanavasandhāgāraṃ}

\paragraph{297.} Tena kho pana samayena pāveyyakānaṃ mallānaṃ ubbhatakaṃ nāma navaṃ sandhāgāraṃ\footnote{santhāgāraṃ (sī. pī.), saṇṭhāgāraṃ (syā. kaṃ.)} acirakāritaṃ hoti anajjhāvuṭṭhaṃ\footnote{anajjhāvutthaṃ (sī. syā. pī. ka.)} samaṇena vā brāhmaṇena vā kenaci vā manussabhūtena. Assosuṃ kho pāveyyakā mallā – ‘‘bhagavā kira mallesu cārikaṃ caramāno mahatā bhikkhusaṅghena saddhiṃ pañcamattehi bhikkhusatehi pāvaṃ anuppatto pāvāyaṃ viharati cundassa kammāraputtassa ambavane’’ti. Atha kho pāveyyakā mallā yena bhagavā tenupasaṅkamiṃsu; upasaṅkamitvā bhagavantaṃ abhivādetvā ekamantaṃ nisīdiṃsu. Ekamantaṃ nisinnā kho pāveyyakā mallā bhagavantaṃ etadavocuṃ – ‘‘idha, bhante, pāveyyakānaṃ mallānaṃ ubbhatakaṃ nāma navaṃ sandhāgāraṃ acirakāritaṃ hoti anajjhāvuṭṭhaṃ samaṇena vā brāhmaṇena vā kenaci vā manussabhūtena. Tañca kho, bhante, bhagavā paṭhamaṃ paribhuñjatu, bhagavatā paṭhamaṃ paribhuttaṃ pacchā pāveyyakā mallā paribhuñjissanti. Tadassa pāveyyakānaṃ mallānaṃ dīgharattaṃ hitāya sukhāyā’’ti. Adhivāsesi kho bhagavā tuṇhībhāvena.

\paragraph{298.} Atha kho pāveyyakā mallā bhagavato adhivāsanaṃ viditvā uṭṭhāyāsanā bhagavantaṃ abhivādetvā padakkhiṇaṃ katvā yena sandhāgāraṃ tenupasaṅkamiṃsu; upasaṅkamitvā sabbasanthariṃ\footnote{sabbasanthariṃ santhataṃ (ka.)} sandhāgāraṃ santharitvā bhagavato āsanāni paññāpetvā udakamaṇikaṃ patiṭṭhapetvā telapadīpaṃ āropetvā yena bhagavā tenupasaṅkamiṃsu; upasaṅkamitvā bhagavantaṃ abhivādetvā ekamantaṃ aṭṭhaṃsu. Ekamantaṃ ṭhitā kho te pāveyyakā mallā bhagavantaṃ etadavocuṃ – ‘‘sabbasantharisanthataṃ\footnote{sabbasanthariṃ santhataṃ (sī. pī. ka.)}, bhante, sandhāgāraṃ, bhagavato āsanāni paññattāni, udakamaṇiko patiṭṭhāpito, telapadīpo āropito. Yassadāni, bhante, bhagavā kālaṃ maññatī’’ti.

\paragraph{299.} Atha kho bhagavā nivāsetvā pattacīvaramādāya saddhiṃ bhikkhusaṅghena yena sandhāgāraṃ tenupasaṅkami; upasaṅkamitvā pāde pakkhāletvā sandhāgāraṃ pavisitvā majjhimaṃ thambhaṃ nissāya puratthābhimukho nisīdi. Bhikkhusaṅghopi kho pāde pakkhāletvā sandhāgāraṃ pavisitvā pacchimaṃ bhittiṃ nissāya puratthābhimukho nisīdi bhagavantaṃyeva purakkhatvā. Pāveyyakāpi kho mallā pāde pakkhāletvā sandhāgāraṃ pavisitvā puratthimaṃ bhittiṃ nissāya pacchimābhimukhā nisīdiṃsu bhagavantaṃyeva purakkhatvā. Atha kho bhagavā pāveyyake malle bahudeva rattiṃ dhammiyā kathāya sandassetvā samādapetvā samuttejetvā sampahaṃsetvā uyyojesi – ‘‘abhikkantā kho, vāseṭṭhā, ratti. Yassadāni tumhe kālaṃ maññathā’’ti. ‘‘Evaṃ, bhante’’ti kho pāveyyakā mallā bhagavato paṭissutvā uṭṭhāyāsanā bhagavantaṃ abhivādetvā padakkhiṇaṃ katvā pakkamiṃsu.

\paragraph{300.} Atha kho bhagavā acirapakkantesu pāveyyakesu mallesu tuṇhībhūtaṃ tuṇhībhūtaṃ bhikkhusaṃghaṃ anuviloketvā āyasmantaṃ sāriputtaṃ āmantesi – ‘‘vigatathinamiddho\footnote{vigatathīnamiddho (sī. syā. kaṃ. pī.)} kho, sāriputta, bhikkhusaṅgho. Paṭibhātu taṃ, sāriputta, bhikkhūnaṃ dhammīkathā. Piṭṭhi me āgilāyati. Tamahaṃ āyamissāmī’’ti\footnote{āyameyyāmīti (syā. kaṃ.)}. ‘‘Evaṃ, bhante’’ti kho āyasmā sāriputto bhagavato paccassosi. Atha kho bhagavā catugguṇaṃ saṅghāṭiṃ paññapetvā dakkhiṇena passena sīhaseyyaṃ kappesi pāde pādaṃ accādhāya, sato sampajāno uṭṭhānasaññaṃ manasi karitvā.

\subsubsection{Bhinnanigaṇṭhavatthu}

\paragraph{301.} Tena kho pana samayena nigaṇṭho nāṭaputto pāvāyaṃ adhunākālaṅkato hoti. Tassa kālaṅkiriyāya bhinnā nigaṇṭhā dvedhikajātā\footnote{ddheḷhakajātā (syā. kaṃ.)} bhaṇḍanajātā kalahajātā vivādāpannā aññamaññaṃ mukhasattīhi vitudantā viharanti\footnote{vicaranti (syā. kaṃ.)} – ‘‘na tvaṃ imaṃ dhammavinayaṃ ājānāsi, ahaṃ imaṃ dhammavinayaṃ ājānāmi, kiṃ tvaṃ imaṃ dhammavinayaṃ ājānissasi! Micchāpaṭipanno tvamasi, ahamasmi sammāpaṭipanno. Sahitaṃ me, asahitaṃ te. Purevacanīyaṃ pacchā avaca, pacchāvacanīyaṃ pure avaca. Adhiciṇṇaṃ te viparāvattaṃ, āropito te vādo, niggahito tvamasi, cara vādappamokkhāya, nibbeṭhehi vā sace pahosī’’ti. Vadhoyeva kho maññe nigaṇṭhesu nāṭaputtiyesu vattati. Yepi\footnote{yepi te (sī. pī.)} nigaṇṭhassa nāṭaputtassa sāvakā gihī odātavasanā , tepi nigaṇṭhesu nāṭaputtiyesu nibbinnarūpā virattarūpā paṭivānarūpā, yathā taṃ durakkhāte dhammavinaye duppavedite aniyyānike anupasamasaṃvattanike asammāsambuddhappavedite bhinnathūpe appaṭisaraṇe.

\paragraph{302.} Atha kho āyasmā sāriputto bhikkhū āmantesi – ‘‘nigaṇṭho, āvuso, nāṭaputto pāvāyaṃ adhunākālaṅkato, tassa kālaṅkiriyāya bhinnā nigaṇṭhā dvedhikajātā…pe… bhinnathūpe appaṭisaraṇe’’. ‘‘Evañhetaṃ, āvuso, hoti durakkhāte dhammavinaye duppavedite aniyyānike anupasamasaṃvattanike asammāsambuddhappavedite. Ayaṃ kho panāvuso amhākaṃ\footnote{asmākaṃ (pī.)} bhagavatā\footnote{bhagavato (ka. sī.)} dhammo svākkhāto suppavedito niyyāniko upasamasaṃvattaniko sammāsambuddhappavedito. Tattha sabbeheva saṅgāyitabbaṃ, na vivaditabbaṃ, yathayidaṃ brahmacariyaṃ addhaniyaṃ assa ciraṭṭhitikaṃ, tadassa bahujanahitāya bahujanasukhāya lokānukampāya atthāya hitāya sukhāya devamanussānaṃ.

‘‘Katamo cāvuso, amhākaṃ bhagavatā\footnote{bhagavato (ka. sī.)} dhammo svākkhāto suppavedito niyyāniko upasamasaṃvattaniko sammāsambuddhappavedito; yattha sabbeheva saṅgāyitabbaṃ, na vivaditabbaṃ, yathayidaṃ brahmacariyaṃ addhaniyaṃ assa ciraṭṭhitikaṃ, tadassa bahujanahitāya bahujanasukhāya lokānukampāya atthāya hitāya sukhāya devamanussānaṃ?

\subsubsection{Ekakaṃ}

\paragraph{303.} ‘‘Atthi kho, āvuso, tena bhagavatā jānatā passatā arahatā sammāsambuddhena eko dhammo sammadakkhāto. Tattha sabbeheva saṅgāyitabbaṃ, na vivaditabbaṃ, yathayidaṃ brahmacariyaṃ addhaniyaṃ assa ciraṭṭhitikaṃ , tadassa bahujanahitāya bahujanasukhāya lokānukampāya atthāya hitāya sukhāya devamanussānaṃ. Katamo eko dhammo? Sabbe sattā āhāraṭṭhitikā. Sabbe sattā saṅkhāraṭṭhitikā. Ayaṃ kho, āvuso, tena bhagavatā jānatā passatā arahatā sammāsambuddhena eko dhammo sammadakkhāto. Tattha sabbeheva saṅgāyitabbaṃ, na vivaditabbaṃ , yathayidaṃ brahmacariyaṃ addhaniyaṃ assa ciraṭṭhitikaṃ, tadassa bahujanahitāya bahujanasukhāya lokānukampāya atthāya hitāya sukhāya devamanussānaṃ.

\subsubsection{Dukaṃ}

\paragraph{304.} ‘‘Atthi kho, āvuso, tena bhagavatā jānatā passatā arahatā sammāsambuddhena dve dhammā sammadakkhātā. Tattha sabbeheva saṅgāyitabbaṃ, na vivaditabbaṃ, yathayidaṃ brahmacariyaṃ addhaniyaṃ assa ciraṭṭhitikaṃ, tadassa bahujanahitāya bahujanasukhāya lokānukampāya atthāya hitāya sukhāya devamanussānaṃ. Katame dve\footnote{dve dhammo (syā. kaṃ.) evamuparipi}?

‘‘Nāmañca rūpañca.

‘‘Avijjā ca bhavataṇhā ca.

‘‘Bhavadiṭṭhi ca vibhavadiṭṭhi ca.

‘‘Ahirikañca\footnote{ahirīkañca (katthaci)} anottappañca.

‘‘Hirī ca ottappañca.

‘‘Dovacassatā ca pāpamittatā ca.

‘‘Sovacassatā ca kalyāṇamittatā ca.

‘‘Āpattikusalatā ca āpattivuṭṭhānakusalatā ca.

‘‘Samāpattikusalatā ca samāpattivuṭṭhānakusalatā ca.

‘‘Dhātukusalatā ca manasikārakusalatā ca.

‘‘Āyatanakusalatā ca paṭiccasamuppādakusalatā ca.

‘‘Ṭhānakusalatā ca aṭṭhānakusalatā ca.

‘‘Ajjavañca lajjavañca.

‘‘Khanti ca soraccañca.

‘‘Sākhalyañca paṭisanthāro ca.

‘‘Avihiṃsā ca soceyyañca.

‘‘Muṭṭhassaccañca asampajaññañca.

‘‘Sati ca sampajaññañca .

‘‘Indriyesu aguttadvāratā ca bhojane amattaññutā ca.

‘‘Indriyesu guttadvāratā ca bhojane mattaññutā ca.

‘‘Paṭisaṅkhānabalañca\footnote{paṭisandhānabalañca (syā.)} bhāvanābalañca.

‘‘Satibalañca samādhibalañca.

‘‘Samatho ca vipassanā ca.

‘‘Samathanimittañca paggahanimittañca.

‘‘Paggaho ca avikkhepo ca.

‘‘Sīlavipatti ca diṭṭhivipatti ca.

‘‘Sīlasampadā ca diṭṭhisampadā ca.

‘‘Sīlavisuddhi ca diṭṭhivisuddhi ca.

‘‘Diṭṭhivisuddhi kho pana yathā diṭṭhissa ca padhānaṃ.

‘‘Saṃvego ca saṃvejanīyesu ṭhānesu saṃviggassa ca yoniso padhānaṃ.

‘‘Asantuṭṭhitā ca kusalesu dhammesu appaṭivānitā ca padhānasmiṃ.

‘‘Vijjā ca vimutti ca.

‘‘Khayeñāṇaṃ anuppādeñāṇaṃ.

‘‘Ime kho, āvuso, tena bhagavatā jānatā passatā arahatā sammāsambuddhena dve dhammā sammadakkhātā. Tattha sabbeheva saṅgāyitabbaṃ, na vivaditabbaṃ, yathayidaṃ brahmacariyaṃ addhaniyaṃ assa ciraṭṭhitikaṃ, tadassa bahujanahitāya bahujanasukhāya lokānukampāya atthāya hitāya sukhāya devamanussānaṃ.

\subsubsection{Tikaṃ}

\paragraph{305.} ‘‘Atthi kho, āvuso, tena bhagavatā jānatā passatā arahatā sammāsambuddhena tayo dhammā sammadakkhātā. Tattha sabbeheva saṅgāyitabbaṃ…pe… atthāya hitāya sukhāya devamanussānaṃ. Katame tayo?

‘‘Tīṇi akusalamūlāni – lobho akusalamūlaṃ, doso akusalamūlaṃ, moho akusalamūlaṃ.

‘‘Tīṇi kusalamūlāni – alobho kusalamūlaṃ, adoso kusalamūlaṃ, amoho kusalamūlaṃ.

‘‘Tīṇi duccaritāni – kāyaduccaritaṃ, vacīduccaritaṃ, manoduccaritaṃ.

‘‘Tīṇi sucaritāni – kāyasucaritaṃ, vacīsucaritaṃ , manosucaritaṃ.

‘‘Tayo akusalavitakkā – kāmavitakko, byāpādavitakko, vihiṃsāvitakko.

‘‘Tayo kusalavitakkā – nekkhammavitakko, abyāpādavitakko, avihiṃsāvitakko.

‘‘Tayo akusalasaṅkappā – kāmasaṅkappo, byāpādasaṅkappo, vihiṃsāsaṅkappo.

‘‘Tayo kusalasaṅkappā – nekkhammasaṅkappo, abyāpādasaṅkappo, avihiṃsāsaṅkappo.

‘‘Tisso akusalasaññā – kāmasaññā, byāpādasaññā, vihiṃsāsaññā.

‘‘Tisso kusalasaññā – nekkhammasaññā, abyāpādasaññā, avihiṃsāsaññā.

‘‘Tisso akusaladhātuyo – kāmadhātu, byāpādadhātu, vihiṃsādhātu.

‘‘Tisso kusaladhātuyo – nekkhammadhātu, abyāpādadhātu, avihiṃsādhātu.

‘‘Aparāpi tisso dhātuyo – kāmadhātu, rūpadhātu, arūpadhātu.

‘‘Aparāpi tisso dhātuyo – rūpadhātu, arūpadhātu, nirodhadhātu.

‘‘Aparāpi tisso dhātuyo – hīnadhātu, majjhimadhātu, paṇītadhātu.

‘‘Tisso taṇhā – kāmataṇhā, bhavataṇhā, vibhavataṇhā.

‘‘Aparāpi tisso taṇhā – kāmataṇhā, rūpataṇhā, arūpataṇhā.

‘‘Aparāpi tisso taṇhā – rūpataṇhā, arūpataṇhā, nirodhataṇhā.

‘‘Tīṇi saṃyojanāni – sakkāyadiṭṭhi, vicikicchā, sīlabbataparāmāso.

‘‘Tayo āsavā – kāmāsavo, bhavāsavo, avijjāsavo.

‘‘Tayo bhavā – kāmabhavo, rūpabhavo, arūpabhavo.

‘‘Tisso esanā – kāmesanā, bhavesanā, brahmacariyesanā.

‘‘Tisso vidhā – seyyohamasmīti vidhā, sadisohamasmīti vidhā, hīnohamasmīti vidhā.

‘‘Tayo addhā – atīto addhā, anāgato addhā, paccuppanno addhā.

‘‘Tayo antā – sakkāyo anto, sakkāyasamudayo anto, sakkāyanirodho anto.

‘‘Tisso vedanā – sukhā vedanā, dukkhā vedanā, adukkhamasukhā vedanā.

‘‘Tisso dukkhatā – dukkhadukkhatā, saṅkhāradukkhatā, vipariṇāmadukkhatā.

‘‘Tayo rāsī – micchattaniyato rāsi, sammattaniyato rāsi, aniyato rāsi.

‘‘Tayo tamā\footnote{tisso kaṅkhā (bahūsu) aṭṭhakathā oloketabbā} – atītaṃ vā addhānaṃ ārabbha kaṅkhati vicikicchati nādhimuccati na sampasīdati, anāgataṃ vā addhānaṃ ārabbha kaṅkhati vicikicchati nādhimuccati na sampasīdati, etarahi vā paccuppannaṃ addhānaṃ ārabbha kaṅkhati vicikicchati nādhimuccati na sampasīdati.

‘‘Tīṇi tathāgatassa arakkheyyāni – parisuddhakāyasamācāro āvuso tathāgato, natthi tathāgatassa kāyaduccaritaṃ, yaṃ tathāgato rakkheyya – ‘mā me idaṃ paro aññāsī’ti. Parisuddhavacīsamācāro āvuso, tathāgato, natthi tathāgatassa vacīduccaritaṃ, yaṃ tathāgato rakkheyya – ‘mā me idaṃ paro aññāsī’ti. Parisuddhamanosamācāro, āvuso, tathāgato, natthi tathāgatassa manoduccaritaṃ yaṃ tathāgato rakkheyya – ‘mā me idaṃ paro aññāsī’ti.

‘‘Tayo kiñcanā – rāgo kiñcanaṃ, doso kiñcanaṃ, moho kiñcanaṃ.

‘‘Tayo aggī – rāgaggi, dosaggi, mohaggi.

‘‘Aparepi tayo aggī – āhuneyyaggi, gahapataggi, dakkhiṇeyyaggi.

‘‘Tividhena rūpasaṅgaho – sanidassanasappaṭighaṃ rūpaṃ\footnote{sanidassanasappaṭigharūpaṃ (syā. kaṃ.) evamitaradvayepi}, anidassanasappaṭighaṃ rūpaṃ, anidassanaappaṭighaṃ rūpaṃ.

‘‘Tayo saṅkhārā – puññābhisaṅkhāro, apuññābhisaṅkhāro , āneñjābhisaṅkhāro.

‘‘Tayo puggalā – sekkho puggalo, asekkho puggalo, nevasekkhonāsekkho puggalo.

‘‘Tayo therā – jātithero, dhammathero, sammutithero\footnote{sammatithero (syā. kaṃ.)}.

‘‘Tīṇi puññakiriyavatthūni – dānamayaṃ puññakiriyavatthu, sīlamayaṃ puññakiriyavatthu, bhāvanāmayaṃ puññakiriyavatthu.

‘‘Tīṇi codanāvatthūni – diṭṭhena, sutena, parisaṅkāya.

‘‘Tisso kāmūpapattiyo\footnote{kāmuppattiyo (sī.), kāmupapattiyo (syā. pī. ka.)} – santāvuso sattā paccupaṭṭhitakāmā, te paccupaṭṭhitesu kāmesu vasaṃ vattenti, seyyathāpi manussā ekacce ca devā ekacce ca vinipātikā. Ayaṃ paṭhamā kāmūpapatti. Santāvuso, sattā nimmitakāmā, te nimminitvā nimminitvā kāmesu vasaṃ vattenti, seyyathāpi devā nimmānaratī. Ayaṃ dutiyā kāmūpapatti. Santāvuso sattā paranimmitakāmā, te paranimmitesu kāmesu vasaṃ vattenti, seyyathāpi devā paranimmitavasavattī. Ayaṃ tatiyā kāmūpapatti.

‘‘Tisso sukhūpapattiyo\footnote{sukhupapattiyo (syā. pī. ka.)} – santāvuso sattā\footnote{sattā sukhaṃ (syā. kaṃ.)} uppādetvā uppādetvā sukhaṃ viharanti, seyyathāpi devā brahmakāyikā. Ayaṃ paṭhamā sukhūpapatti. Santāvuso, sattā sukhena abhisannā parisannā paripūrā paripphuṭā. Te kadāci karahaci udānaṃ udānenti – ‘aho sukhaṃ, aho sukha’nti , seyyathāpi devā ābhassarā. Ayaṃ dutiyā sukhūpapatti. Santāvuso, sattā sukhena abhisannā parisannā paripūrā paripphuṭā. Te santaṃyeva tusitā\footnote{santusitā (syā. kaṃ.)} sukhaṃ\footnote{cittasukhaṃ (syā. ka.)} paṭisaṃvedenti, seyyathāpi devā subhakiṇhā. Ayaṃ tatiyā sukhūpapatti .

‘‘Tisso paññā – sekkhā paññā, asekkhā paññā, nevasekkhānāsekkhā paññā.

‘‘Aparāpi tisso paññā – cintāmayā paññā, sutamayā paññā, bhāvanāmayā paññā.

‘‘Tīṇāvudhāni – sutāvudhaṃ, pavivekāvudhaṃ, paññāvudhaṃ.

‘‘Tīṇindriyāni – anaññātaññassāmītindriyaṃ, aññindriyaṃ, aññātāvindriyaṃ.

‘‘Tīṇi cakkhūni – maṃsacakkhu, dibbacakkhu, paññācakkhu.

‘‘Tisso sikkhā – adhisīlasikkhā, adhicittasikkhā, adhipaññāsikkhā.

‘‘Tisso bhāvanā – kāyabhāvanā, cittabhāvanā, paññābhāvanā.

‘‘Tīṇi anuttariyāni – dassanānuttariyaṃ, paṭipadānuttariyaṃ, vimuttānuttariyaṃ.

‘‘Tayo samādhī – savitakkasavicāro samādhi, avitakkavicāramatto samādhi, avitakkaavicāro samādhi.

‘‘Aparepi tayo samādhī – suññato samādhi, animitto samādhi, appaṇihito samādhi.

‘‘Tīṇi soceyyāni – kāyasoceyyaṃ, vacīsoceyyaṃ, manosoceyyaṃ.

‘‘Tīṇi moneyyāni – kāyamoneyyaṃ, vacīmoneyyaṃ, manomoneyyaṃ.

‘‘Tīṇi kosallāni – āyakosallaṃ, apāyakosallaṃ, upāyakosallaṃ.

‘‘Tayo madā – ārogyamado, yobbanamado, jīvitamado.

‘‘Tīṇi ādhipateyyāni – attādhipateyyaṃ, lokādhipateyyaṃ, dhammādhipateyyaṃ.

‘‘Tīṇi kathāvatthūni – atītaṃ vā addhānaṃ ārabbha kathaṃ katheyya – ‘evaṃ ahosi atītamaddhāna’nti; anāgataṃ vā addhānaṃ ārabbha kathaṃ katheyya – ‘evaṃ bhavissati anāgatamaddhāna’nti; etarahi vā paccuppannaṃ addhānaṃ ārabbha kathaṃ katheyya – ‘evaṃ hoti etarahi paccuppannaṃ addhāna’nti.

‘‘Tisso vijjā – pubbenivāsānussatiñāṇaṃ vijjā, sattānaṃ cutūpapāteñāṇaṃ vijjā, āsavānaṃ khayeñāṇaṃ vijjā.

‘‘Tayo vihārā – dibbo vihāro, brahmā vihāro, ariyo vihāro.

‘‘Tīṇi pāṭihāriyāni – iddhipāṭihāriyaṃ, ādesanāpāṭihāriyaṃ, anusāsanīpāṭihāriyaṃ.

‘‘Ime kho, āvuso, tena bhagavatā jānatā passatā arahatā sammāsambuddhena tayo dhammā sammadakkhātā. Tattha sabbeheva saṅgāyitabbaṃ…pe… atthāya hitāya sukhāya devamanussānaṃ.

\subsubsection{Catukkaṃ}

\paragraph{306.} ‘‘Atthi kho, āvuso, tena bhagavatā jānatā passatā arahatā sammāsambuddhena cattāro dhammā sammadakkhātā. Tattha sabbeheva saṅgāyitabbaṃ, na vivaditabbaṃ…pe… atthāya hitāya sukhāya devamanussānaṃ. Katame cattāro?

‘‘Cattāro satipaṭṭhānā. Idhāvuso, bhikkhu kāye kāyānupassī viharati ātāpī sampajāno satimā, vineyya loke abhijjhādomanassaṃ. Vedanāsu vedanānupassī…pe… citte cittānupassī…pe… dhammesu dhammānupassī viharati ātāpī sampajāno satimā vineyya loke abhijjhādomanassaṃ.

‘‘Cattāro sammappadhānā. Idhāvuso, bhikkhu anuppannānaṃ pāpakānaṃ akusalānaṃ dhammānaṃ anuppādāya chandaṃ janeti vāyamati vīriyaṃ ārabhati cittaṃ paggaṇhāti padahati. Uppannānaṃ pāpakānaṃ akusalānaṃ dhammānaṃ pahānāya chandaṃ janeti vāyamati vīriyaṃ ārabhati cittaṃ paggaṇhāti padahati. Anuppannānaṃ kusalānaṃ dhammānaṃ uppādāya chandaṃ janeti vāyamati vīriyaṃ ārabhati cittaṃ paggaṇhāti padahati. Uppannānaṃ kusalānaṃ dhammānaṃ ṭhitiyā asammosāya bhiyyobhāvāya vepullāya bhāvanāya pāripūriyā chandaṃ janeti vāyamati vīriyaṃ ārabhati cittaṃ paggaṇhāti padahati.

‘‘Cattāro iddhipādā. Idhāvuso, bhikkhu chandasamādhipadhānasaṅkhārasamannāgataṃ iddhipādaṃ bhāveti. Cittasamādhipadhānasaṅkhārasamannāgataṃ iddhipādaṃ bhāveti. Vīriyasamādhipadhānasaṅkhārasamannāgataṃ iddhipādaṃ bhāveti. Vīmaṃsāsamādhipadhānasaṅkhārasamannāgataṃ iddhipādaṃ bhāveti.

‘‘Cattāri jhānāni. Idhāvuso, bhikkhu vivicceva kāmehi vivicca akusalehi dhammehi savitakkaṃ savicāraṃ vivekajaṃ pītisukhaṃ paṭhamaṃ jhānaṃ\footnote{paṭhamajjhānaṃ (syā. kaṃ.)} upasampajja viharati. Vitakkavicārānaṃ vūpasamā ajjhattaṃ sampasādanaṃ cetaso ekodibhāvaṃ avitakkaṃ avicāraṃ samādhijaṃ pītisukhaṃ dutiyaṃ jhānaṃ\footnote{dutiyajjhānaṃ (syā. kaṃ.)} upasampajja viharati. Pītiyā ca virāgā upekkhako ca viharati sato ca sampajāno, sukhañca kāyena paṭisaṃvedeti, yaṃ taṃ ariyā ācikkhanti – ‘upekkhako satimā sukhavihārī’ti tatiyaṃ jhānaṃ\footnote{tatiyajjhānaṃ (syā. kaṃ.)} upasampajja viharati. Sukhassa ca pahānā dukkhassa ca pahānā, pubbeva somanassadomanassānaṃ atthaṅgamā, adukkhamasukhaṃ upekkhāsatipārisuddhiṃ catutthaṃ jhānaṃ\footnote{catutthajjhānaṃ (syā. kaṃ.)} upasampajja viharati.

\paragraph{307.} ‘‘Catasso samādhibhāvanā. Atthāvuso, samādhibhāvanā bhāvitā bahulīkatā diṭṭhadhammasukhavihārāya saṃvattati. Atthāvuso, samādhibhāvanā bhāvitā bahulīkatā ñāṇadassanapaṭilābhāya saṃvattati. Atthāvuso samādhibhāvanā bhāvitā bahulīkatā satisampajaññāya saṃvattati. Atthāvuso samādhibhāvanā bhāvitā bahulīkatā āsavānaṃ khayāya saṃvattati.

‘‘Katamā cāvuso, samādhibhāvanā bhāvitā bahulīkatā diṭṭhadhammasukhavihārāya saṃvattati? Idhāvuso, bhikkhu vivicceva kāmehi vivicca akusalehi dhammehi savitakkaṃ…pe… catutthaṃ jhānaṃ upasampajja viharati. Ayaṃ, āvuso , samādhibhāvanā bhāvitā bahulīkatā diṭṭhadhammasukhavihārāya saṃvattati.

‘‘Katamā cāvuso, samādhibhāvanā bhāvitā bahulīkatā ñāṇadassanapaṭilābhāya saṃvattati? Idhāvuso, bhikkhu ālokasaññaṃ manasi karoti, divāsaññaṃ adhiṭṭhāti yathā divā tathā rattiṃ, yathā rattiṃ tathā divā. Iti vivaṭena cetasā apariyonaddhena sappabhāsaṃ cittaṃ bhāveti. Ayaṃ, āvuso samādhibhāvanā bhāvitā bahulīkatā ñāṇadassanapaṭilābhāya saṃvattati.

‘‘Katamā cāvuso, samādhibhāvanā bhāvitā bahulīkatā satisampajaññāya saṃvattati? Idhāvuso, bhikkhuno viditā vedanā uppajjanti, viditā upaṭṭhahanti, viditā abbhatthaṃ gacchanti. Viditā saññā uppajjanti, viditā upaṭṭhahanti, viditā abbhatthaṃ gacchanti. Viditā vitakkā uppajjanti, viditā upaṭṭhahanti, viditā abbhatthaṃ gacchanti. Ayaṃ, āvuso, samādhibhāvanā bhāvitā bahulīkatā satisampajaññāya saṃvattati.

‘‘Katamā cāvuso, samādhibhāvanā bhāvitā bahulīkatā āsavānaṃ khayāya saṃvattati? Idhāvuso, bhikkhu pañcasu upādānakkhandhesu udayabbayānupassī viharati. Iti rūpaṃ, iti rūpassa samudayo, iti rūpassa atthaṅgamo. Iti vedanā…pe… iti saññā… iti saṅkhārā… iti viññāṇaṃ, iti viññāṇassa samudayo, iti viññāṇassa atthaṅgamo. Ayaṃ, āvuso, samādhibhāvanā bhāvitā bahulīkatā āsavānaṃ khayāya saṃvattati.

\paragraph{308.} ‘‘Catasso appamaññā. Idhāvuso, bhikkhu mettāsahagatena cetasā ekaṃ disaṃ pharitvā viharati. Tathā dutiyaṃ. Tathā tatiyaṃ. Tathā catutthaṃ. Iti uddhamadho tiriyaṃ sabbadhi sabbattatāya sabbāvantaṃ lokaṃ mettāsahagatena cetasā vipulena mahaggatena appamāṇena averena abyāpajjena\footnote{abyāpajjhena (sī. syā. kaṃ. pī.)} pharitvā viharati. Karuṇāsahagatena cetasā…pe… muditāsahagatena cetasā…pe… upekkhāsahagatena cetasā ekaṃ disaṃ pharitvā viharati. Tathā dutiyaṃ. Tathā tatiyaṃ. Tathā catutthaṃ. Iti uddhamadho tiriyaṃ sabbadhi sabbattatāya sabbāvantaṃ lokaṃ upekkhāsahagatena cetasā vipulena mahaggatena appamāṇena averena abyāpajjena pharitvā viharati.

‘‘Cattāro āruppā.\footnote{arūpā (syā. kaṃ. pī.)} Idhāvuso, bhikkhu sabbaso rūpasaññānaṃ samatikkamā paṭighasaññānaṃ atthaṅgamā nānattasaññānaṃ amanasikārā ‘ananto ākāso’ti ākāsānañcāyatanaṃ upasampajja viharati. Sabbaso ākāsānañcāyatanaṃ samatikkamma ‘anantaṃ viññāṇa’nti viññāṇañcāyatanaṃ upasampajja viharati. Sabbaso viññāṇañcāyatanaṃ samatikkamma ‘natthi kiñcī’ti ākiñcaññāyatanaṃ upasampajja viharati. Sabbaso ākiñcaññāyatanaṃ samatikkamma nevasaññānāsaññāyatanaṃ upasampajja viharati.

‘‘Cattāri apassenāni. Idhāvuso, bhikkhu saṅkhāyekaṃ paṭisevati, saṅkhāyekaṃ adhivāseti, saṅkhāyekaṃ parivajjeti, saṅkhāyekaṃ vinodeti.

\paragraph{309.} ‘‘Cattāro ariyavaṃsā. Idhāvuso, bhikkhu santuṭṭho hoti itarītarena cīvarena, itarītaracīvarasantuṭṭhiyā ca vaṇṇavādī, na ca cīvarahetu anesanaṃ appatirūpaṃ āpajjati; aladdhā ca cīvaraṃ na paritassati, laddhā ca cīvaraṃ agadhito\footnote{agathito (sī. pī.)} amucchito anajjhāpanno ādīnavadassāvī nissaraṇapañño paribhuñjati; tāya ca pana itarītaracīvarasantuṭṭhiyā nevattānukkaṃseti na paraṃ vambheti. Yo hi tattha dakkho analaso sampajāno paṭissato, ayaṃ vuccatāvuso – ‘bhikkhu porāṇe aggaññe ariyavaṃse ṭhito’.

‘‘Puna caparaṃ, āvuso, bhikkhu santuṭṭho hoti itarītarena piṇḍapātena, itarītarapiṇḍapātasantuṭṭhiyā ca vaṇṇavādī, na ca piṇḍapātahetu anesanaṃ appatirūpaṃ āpajjati; aladdhā ca piṇḍapātaṃ na paritassati, laddhā ca piṇḍapātaṃ agadhito amucchito anajjhāpanno ādīnavadassāvī nissaraṇapañño paribhuñjati; tāya ca pana itarītarapiṇḍapātasantuṭṭhiyā nevattānukkaṃseti na paraṃ vambheti. Yo hi tattha dakkho analaso sampajāno paṭissato , ayaṃ vuccatāvuso – ‘bhikkhu porāṇe aggaññe ariyavaṃse ṭhito’.

‘‘Puna caparaṃ, āvuso, bhikkhu santuṭṭho hoti itarītarena senāsanena, itarītarasenāsanasantuṭṭhiyā ca vaṇṇavādī, na ca senāsanahetu anesanaṃ appatirūpaṃ āpajjati; aladdhā ca senāsanaṃ na paritassati, laddhā ca senāsanaṃ agadhito amucchito anajjhāpanno ādīnavadassāvī nissaraṇapañño paribhuñjati; tāya ca pana itarītarasenāsanasantuṭṭhiyā nevattānukkaṃseti na paraṃ vambheti. Yo hi tattha dakkho analaso sampajāno paṭissato, ayaṃ vuccatāvuso – ‘bhikkhu porāṇe aggaññe ariyavaṃse ṭhito’.

‘‘Puna caparaṃ, āvuso, bhikkhu pahānārāmo hoti pahānarato, bhāvanārāmo hoti bhāvanārato; tāya ca pana pahānārāmatāya pahānaratiyā bhāvanārāmatāya bhāvanāratiyā nevattānukkaṃseti na paraṃ vambheti. Yo hi tattha dakkho analaso sampajāno paṭissato ayaṃ vuccatāvuso – ‘bhikkhu porāṇe aggaññe ariyavaṃse ṭhito’.

\paragraph{310.} ‘‘Cattāri padhānāni. Saṃvarapadhānaṃ pahānapadhānaṃ bhāvanāpadhānaṃ\footnote{bhāvanāppadhānaṃ (syā.)} anurakkhaṇāpadhānaṃ\footnote{anurakkhanāppadhānaṃ (syā.)}. Katamañcāvuso, saṃvarapadhānaṃ? Idhāvuso, bhikkhu cakkhunā rūpaṃ disvā na nimittaggāhī hoti nānubyañjanaggāhī. Yatvādhikaraṇamenaṃ cakkhundriyaṃ asaṃvutaṃ viharantaṃ abhijjhādomanassā pāpakā akusalā dhammā anvāssaveyyuṃ, tassa saṃvarāya paṭipajjati, rakkhati cakkhundriyaṃ, cakkhundriye saṃvaraṃ āpajjati. Sotena saddaṃ sutvā… ghānena gandhaṃ ghāyitvā… jivhāya rasaṃ sāyitvā… kāyena phoṭṭhabbaṃ phusitvā… manasā dhammaṃ viññāya na nimittaggāhī hoti nānubyañjanaggāhī. Yatvādhikaraṇamenaṃ manindriyaṃ asaṃvutaṃ viharantaṃ abhijjhādomanassā pāpakā akusalā dhammā anvāssaveyyuṃ, tassa saṃvarāya paṭipajjati, rakkhati manindriyaṃ, manindriye saṃvaraṃ āpajjati. Idaṃ vuccatāvuso, saṃvarapadhānaṃ.

‘‘Katamañcāvuso, pahānapadhānaṃ? Idhāvuso, bhikkhu uppannaṃ kāmavitakkaṃ nādhivāseti pajahati vinodeti byantiṃ karoti\footnote{byantī karoti (syā. kaṃ.)} anabhāvaṃ gameti. Uppannaṃ byāpādavitakkaṃ…pe… uppannaṃ vihiṃsāvitakkaṃ… uppannuppanne pāpake akusale dhamme nādhivāseti pajahati vinodeti byantiṃ karoti anabhāvaṃ gameti. Idaṃ vuccatāvuso, pahānapadhānaṃ.

‘‘Katamañcāvuso , bhāvanāpadhānaṃ? Idhāvuso, bhikkhu satisambojjhaṅgaṃ bhāveti vivekanissitaṃ virāganissitaṃ nirodhanissitaṃ vossaggapariṇāmiṃ. Dhammavicayasambojjhaṅgaṃ bhāveti… vīriyasambojjhaṅgaṃ bhāveti… pītisambojjhaṅgaṃ bhāveti… passaddhisambojjhaṅgaṃ bhāveti… samādhisambojjhaṅgaṃ bhāveti… upekkhāsambojjhaṅgaṃ bhāveti vivekanissitaṃ virāganissitaṃ nirodhanissitaṃ vossaggapariṇāmiṃ. Idaṃ vuccatāvuso, bhāvanāpadhānaṃ.

‘‘Katamañcāvuso, anurakkhaṇāpadhānaṃ? Idhāvuso, bhikkhu uppannaṃ bhadrakaṃ\footnote{bhaddakaṃ (syā. kaṃ. pī.)} samādhinimittaṃ anurakkhati – aṭṭhikasaññaṃ, puḷuvakasaññaṃ\footnote{puḷavakasaññaṃ (sī. pī.)}, vinīlakasaññaṃ, vicchiddakasaññaṃ, uddhumātakasaññaṃ. Idaṃ vuccatāvuso, anurakkhaṇāpadhānaṃ.

‘‘Cattāri ñāṇāni – dhamme ñāṇaṃ, anvaye ñāṇaṃ, pariye\footnote{paricce (sī. ka.), paricchede (syā. pī. ka.) ṭīkā oloketabbā} ñāṇaṃ, sammutiyā ñāṇaṃ\footnote{sammatiñāṇaṃ (syā. kaṃ.)}.

‘‘Aparānipi cattāri ñāṇāni – dukkhe ñāṇaṃ, dukkhasamudaye ñāṇaṃ, dukkhanirodhe ñāṇaṃ, dukkhanirodhagāminiyā paṭipadāya ñāṇaṃ.

\paragraph{311.} ‘‘Cattāri sotāpattiyaṅgāni – sappurisasaṃsevo, saddhammassavanaṃ, yonisomanasikāro, dhammānudhammappaṭipatti.

‘‘Cattāri sotāpannassa aṅgāni. Idhāvuso, ariyasāvako buddhe aveccappasādena samannāgato hoti – ‘itipi so bhagavā arahaṃ sammāsambuddho vijjācaraṇasampanno sugato lokavidū anuttaro purisadammasārathi satthā devamanussānaṃ buddho, bhagavā’ti. Dhamme aveccappasādena samannāgato hoti – ‘svākkhāto bhagavatā dhammo sandiṭṭhiko akāliko ehipassiko opaneyyiko\footnote{opanayiko (syā. kaṃ.)} paccattaṃ veditabbo viññūhī’ti. Saṅghe aveccappasādena samannāgato hoti – ‘suppaṭipanno bhagavato sāvakasaṅgho ujuppaṭipanno bhagavato sāvakasaṅgho ñāyappaṭipanno bhagavato sāvakasaṅgho sāmīcippaṭipanno bhagavato sāvakasaṅgho yadidaṃ cattāri purisayugāni aṭṭha purisapuggalā, esa bhagavato sāvakasaṅgho āhuneyyo pāhuneyyo dakkhiṇeyyo añjalikaraṇīyo anuttaraṃ puññakkhettaṃ lokassā’ti. Ariyakantehi sīlehi samannāgato hoti akhaṇḍehi acchiddehi asabalehi akammāsehi bhujissehi viññuppasatthehi aparāmaṭṭhehi samādhisaṃvattanikehi.

‘‘Cattāri sāmaññaphalāni – sotāpattiphalaṃ, sakadāgāmiphalaṃ, anāgāmiphalaṃ, arahattaphalaṃ.

‘‘Catasso dhātuyo – pathavīdhātu, āpodhātu, tejodhātu, vāyodhātu.

‘‘Cattāro āhārā – kabaḷīkāro āhāro oḷāriko vā sukhumo vā, phasso dutiyo, manosañcetanā tatiyā, viññāṇaṃ catutthaṃ.

‘‘Catasso viññāṇaṭṭhitiyo. Rūpūpāyaṃ vā, āvuso, viññāṇaṃ tiṭṭhamānaṃ tiṭṭhati rūpārammaṇaṃ\footnote{rūpāramaṇaṃ (?)} rūpappatiṭṭhaṃ nandūpasecanaṃ vuddhiṃ virūḷhiṃ vepullaṃ āpajjati; vedanūpāyaṃ vā āvuso…pe… saññūpāyaṃ vā, āvuso…pe… saṅkhārūpāyaṃ vā, āvuso, viññāṇaṃ tiṭṭhamānaṃ tiṭṭhati saṅkhārārammaṇaṃ saṅkhārappatiṭṭhaṃ nandūpasecanaṃ vuddhiṃ virūḷhiṃ vepullaṃ āpajjati.

‘‘Cattāri agatigamanāni – chandāgatiṃ gacchati, dosāgati gacchati, mohāgatiṃ gacchati, bhayāgatiṃ gacchati.

‘‘Cattāro taṇhuppādā – cīvarahetu vā, āvuso, bhikkhuno taṇhā uppajjamānā uppajjati; piṇḍapātahetu vā, āvuso, bhikkhuno taṇhā uppajjamānā uppajjati; senāsanahetu vā, āvuso, bhikkhuno taṇhā uppajjamānā uppajjati; itibhavābhavahetu vā, āvuso, bhikkhuno taṇhā uppajjamānā uppajjati.

‘‘Catasso paṭipadā – dukkhā paṭipadā dandhābhiññā, dukkhā paṭipadā khippābhiññā, sukhā paṭipadā dandhābhiññā, sukhā paṭipadā khippābhiññā.

‘‘Aparāpi catasso paṭipadā – akkhamā paṭipadā, khamā paṭipadā, damā paṭipadā, samā paṭipadā.

‘‘Cattāri dhammapadāni – anabhijjhā dhammapadaṃ, abyāpādo dhammapadaṃ, sammāsati dhammapadaṃ, sammāsamādhi dhammapadaṃ.

‘‘Cattāri dhammasamādānāni – atthāvuso, dhammasamādānaṃ paccuppannadukkhañceva āyatiñca dukkhavipākaṃ. Atthāvuso, dhammasamādānaṃ paccuppannadukkhaṃ āyatiṃ sukhavipākaṃ. Atthāvuso, dhammasamādānaṃ paccuppannasukhaṃ āyatiṃ dukkhavipākaṃ. Atthāvuso, dhammasamādānaṃ paccuppannasukhañceva āyatiñca sukhavipākaṃ.

‘‘Cattāro dhammakkhandhā – sīlakkhandho, samādhikkhandho, paññākkhandho, vimuttikkhandho.

‘‘Cattāri balāni – vīriyabalaṃ, satibalaṃ, samādhibalaṃ, paññābalaṃ.

‘‘Cattāri adhiṭṭhānāni – paññādhiṭṭhānaṃ, saccādhiṭṭhānaṃ, cāgādhiṭṭhānaṃ, upasamādhiṭṭhānaṃ.

\paragraph{312.} ‘‘Cattāri pañhabyākaraṇāni\footnote{cattāro pañhābyākaraṇā (sī. syā. kaṃ. pī.)} - ekaṃsabyākaraṇīyo pañho, paṭipucchābyākaraṇīyo pañho, vibhajjabyākaraṇīyo pañho, ṭhapanīyo pañho.

‘‘Cattāri kammāni – atthāvuso, kammaṃ kaṇhaṃ kaṇhavipākaṃ; atthāvuso, kammaṃ sukkaṃ sukkavipākaṃ; atthāvuso, kammaṃ kaṇhasukkaṃ kaṇhasukkavipākaṃ; atthāvuso, kammaṃ akaṇhaasukkaṃ akaṇhaasukkavipākaṃ kammakkhayāya saṃvattati.

‘‘Cattāro sacchikaraṇīyā dhammā – pubbenivāso satiyā sacchikaraṇīyo; sattānaṃ cutūpapāto cakkhunā sacchikaraṇīyo; aṭṭha vimokkhā kāyena sacchikaraṇīyā; āsavānaṃ khayo paññāya sacchikaraṇīyo.

‘‘Cattāro oghā – kāmogho, bhavogho, diṭṭhogho, avijjogho.

‘‘Cattāro yogā – kāmayogo, bhavayogo, diṭṭhiyogo, avijjāyogo.

‘‘Cattāro visaññogā – kāmayogavisaññogo, bhavayogavisaññogo, diṭṭhiyogavisaññogo, avijjāyogavisaññogo.

‘‘Cattāro ganthā – abhijjhā kāyagantho, byāpādo kāyagantho, sīlabbataparāmāso kāyagantho, idaṃsaccābhiniveso kāyagantho.

‘‘Cattāri upādānāni – kāmupādānaṃ\footnote{kāmūpādānaṃ (sī. pī.) evamitaresupi}, diṭṭhupādānaṃ, sīlabbatupādānaṃ, attavādupādānaṃ.

‘‘Catasso yoniyo – aṇḍajayoni, jalābujayoni, saṃsedajayoni, opapātikayoni.

‘‘Catasso gabbhāvakkantiyo. Idhāvuso, ekacco asampajāno mātukucchiṃ okkamati, asampajāno mātukucchismiṃ ṭhāti, asampajāno mātukucchimhā nikkhamati, ayaṃ paṭhamā gabbhāvakkanti. Puna caparaṃ, āvuso, idhekacco sampajāno mātukucchiṃ okkamati, asampajāno mātukucchismiṃ ṭhāti, asampajāno mātukucchimhā nikkhamati, ayaṃ dutiyā gabbhāvakkanti. Puna caparaṃ, āvuso, idhekacco sampajāno mātukucchiṃ okkamati, sampajāno mātukucchismiṃ ṭhāti, asampajāno mātukucchimhā nikkhamati, ayaṃ tatiyā gabbhāvakkanti. Puna caparaṃ, āvuso, idhekacco sampajāno mātukucchiṃ okkamati, sampajāno mātukucchismiṃ ṭhāti, sampajāno mātukucchimhā nikkhamati, ayaṃ catutthā gabbhāvakkanti.

‘‘Cattāro attabhāvapaṭilābhā. Atthāvuso, attabhāvapaṭilābho, yasmiṃ attabhāvapaṭilābhe attasañcetanāyeva kamati, no parasañcetanā. Atthāvuso, attabhāvapaṭilābho, yasmiṃ attabhāvapaṭilābhe parasañcetanāyeva kamati, no attasañcetanā. Atthāvuso, attabhāvapaṭilābho, yasmiṃ attabhāvapaṭilābhe attasañcetanā ceva kamati parasañcetanā ca. Atthāvuso, attabhāvapaṭilābho, yasmiṃ attabhāvapaṭilābhe neva attasañcetanā kamati, no parasañcetanā.

\paragraph{313.} ‘‘Catasso dakkhiṇāvisuddhiyo. Atthāvuso, dakkhiṇā dāyakato visujjhati no paṭiggāhakato. Atthāvuso, dakkhiṇā paṭiggāhakato visujjhati no dāyakato. Atthāvuso, dakkhiṇā neva dāyakato visujjhati no paṭiggāhakato. Atthāvuso, dakkhiṇā dāyakato ceva visujjhati paṭiggāhakato ca.

‘‘Cattāri saṅgahavatthūni – dānaṃ, peyyavajjaṃ\footnote{piyavajjaṃ (syā. kaṃ. ka.)}, atthacariyā, samānattatā.

‘‘Cattāro anariyavohārā – musāvādo, pisuṇāvācā, pharusāvācā, samphappalāpo.

‘‘Cattāro ariyavohārā – musāvādā veramaṇī\footnote{veramaṇi (ka.)}, pisuṇāya vācāya veramaṇī, pharusāya vācāya veramaṇī, samphappalāpā veramaṇī.

‘‘Aparepi cattāro anariyavohārā – adiṭṭhe diṭṭhavāditā, assute sutavāditā, amute mutavāditā, aviññāte viññātavāditā.

‘‘Aparepi cattāro ariyavohārā – adiṭṭhe adiṭṭhavāditā, assute assutavāditā, amute amutavāditā, aviññāte aviññātavāditā.

‘‘Aparepi cattāro anariyavohārā – diṭṭhe adiṭṭhavāditā, sute assutavāditā, mute amutavāditā, viññāte aviññātavāditā.

‘‘Aparepi cattāro ariyavohārā – diṭṭhe diṭṭhavāditā, sute sutavāditā, mute mutavāditā, viññāte viññātavāditā.

\paragraph{314.} ‘‘Cattāro puggalā. Idhāvuso, ekacco puggalo attantapo hoti attaparitāpanānuyogamanuyutto. Idhāvuso, ekacco puggalo parantapo hoti paraparitāpanānuyogamanuyutto. Idhāvuso, ekacco puggalo attantapo ca hoti attaparitāpanānuyogamanuyutto, parantapo ca paraparitāpanānuyogamanuyutto. Idhāvuso, ekacco puggalo neva attantapo hoti na attaparitāpanānuyogamanuyutto na parantapo na paraparitāpanānuyogamanuyutto. So anattantapo aparantapo diṭṭheva dhamme nicchāto nibbuto sītībhūto\footnote{sītibhūto (ka.)} sukhappaṭisaṃvedī brahmabhūtena attanā viharati.

‘‘Aparepi cattāro puggalā. Idhāvuso, ekacco puggalo attahitāya paṭipanno hoti no parahitāya. Idhāvuso, ekacco puggalo parahitāya paṭipanno hoti no attahitāya. Idhāvuso , ekacco puggalo neva attahitāya paṭipanno hoti no parahitāya. Idhāvuso, ekacco puggalo attahitāya ceva paṭipanno hoti parahitāya ca.

‘‘Aparepi cattāro puggalā – tamo tamaparāyano, tamo jotiparāyano, joti tamaparāyano, joti jotiparāyano.

‘‘Aparepi cattāro puggalā – samaṇamacalo, samaṇapadumo, samaṇapuṇḍarīko, samaṇesu samaṇasukhumālo.

‘‘Ime kho, āvuso, tena bhagavatā jānatā passatā arahatā sammāsambuddhena cattāro dhammā sammadakkhātā; tattha sabbeheva saṅgāyitabbaṃ…pe… atthāya hitāya sukhāya devamanussānaṃ.

\xsubsubsectionEnd{Paṭhamabhāṇavāro niṭṭhito.}

\subsubsection{Pañcakaṃ}

\paragraph{315.} ‘‘Atthi kho, āvuso, tena bhagavatā jānatā passatā arahatā sammāsambuddhena pañca dhammā sammadakkhātā. Tattha sabbeheva saṅgāyitabbaṃ…pe… atthāya hitāya sukhāya devamanussānaṃ. Katame pañca?

‘‘Pañcakkhandhā. Rūpakkhandho vedanākkhandho saññākkhandho saṅkhārakkhandho viññāṇakkhandho.

‘‘Pañcupādānakkhandhā. Rūpupādānakkhandho\footnote{rūpūpādānakkhandho (sī. syā. kaṃ. pī.) evamitaresupi} vedanupādānakkhandho saññupādānakkhandho saṅkhārupādānakkhandho viññāṇupādānakkhandho.

‘‘Pañca kāmaguṇā. Cakkhuviññeyyā rūpā iṭṭhā kantā manāpā piyarūpā kāmūpasañhitā rajanīyā , sotaviññeyyā saddā… ghānaviññeyyā gandhā… jivhāviññeyyā rasā… kāyaviññeyyā phoṭṭhabbā iṭṭhā kantā manāpā piyarūpā kāmūpasañhitā rajanīyā.

‘‘Pañca gatiyo – nirayo, tiracchānayoni, pettivisayo, manussā, devā.

‘‘Pañca macchariyāni – āvāsamacchariyaṃ, kulamacchariyaṃ, lābhamacchariyaṃ, vaṇṇamacchariyaṃ, dhammamacchariyaṃ.

‘‘Pañca nīvaraṇāni – kāmacchandanīvaraṇaṃ, byāpādanīvaraṇaṃ, thinamiddhanīvaraṇaṃ, uddhaccakukkuccanīvaraṇaṃ, vicikicchānīvaraṇaṃ.

‘‘Pañca orambhāgiyāni saññojanāni – sakkāyadiṭṭhi, vicikicchā, sīlabbataparāmāso, kāmacchando, byāpādo.

‘‘Pañca uddhambhāgiyāni saññojanāni – rūparāgo, arūparāgo, māno, uddhaccaṃ, avijjā.

‘‘Pañca sikkhāpadāni – pāṇātipātā veramaṇī, adinnādānā veramaṇī, kāmesumicchācārā veramaṇī, musāvādā veramaṇī, surāmerayamajjappamādaṭṭhānā veramaṇī.

\paragraph{316.} ‘‘Pañca abhabbaṭṭhānāni. Abhabbo, āvuso, khīṇāsavo bhikkhu sañcicca pāṇaṃ jīvitā voropetuṃ. Abhabbo khīṇāsavo bhikkhu adinnaṃ theyyasaṅkhātaṃ ādiyituṃ\footnote{ādātuṃ (syā. kaṃ. pī.)}. Abhabbo khīṇāsavo bhikkhu methunaṃ dhammaṃ paṭisevituṃ. Abhabbo khīṇāsavo bhikkhu sampajānamusā bhāsituṃ. Abhabbo khīṇāsavo bhikkhu sannidhikārakaṃ kāme paribhuñjituṃ, seyyathāpi pubbe āgārikabhūto.

‘‘Pañca byasanāni – ñātibyasanaṃ, bhogabyasanaṃ, rogabyasanaṃ, sīlabyasanaṃ, diṭṭhibyasanaṃ. Nāvuso, sattā ñātibyasanahetu vā bhogabyasanahetu vā rogabyasanahetu vā kāyassa bhedā paraṃ maraṇā apāyaṃ duggatiṃ vinipātaṃ nirayaṃ upapajjanti. Sīlabyasanahetu vā, āvuso, sattā diṭṭhibyasanahetu vā kāyassa bhedā paraṃ maraṇā apāyaṃ duggatiṃ vinipātaṃ nirayaṃ upapajjanti.

‘‘Pañca sampadā – ñātisampadā, bhogasampadā, ārogyasampadā, sīlasampadā, diṭṭhisampadā. Nāvuso, sattā ñātisampadāhetu vā bhogasampadāhetu vā ārogyasampadāhetu vā kāyassa bhedā paraṃ maraṇā sugatiṃ saggaṃ lokaṃ upapajjanti. Sīlasampadāhetu vā, āvuso, sattā diṭṭhisampadāhetu vā kāyassa bhedā paraṃ maraṇā sugatiṃ saggaṃ lokaṃ upapajjanti.

‘‘Pañca ādīnavā dussīlassa sīlavipattiyā. Idhāvuso , dussīlo sīlavipanno pamādādhikaraṇaṃ mahatiṃ bhogajāniṃ nigacchati, ayaṃ paṭhamo ādīnavo dussīlassa sīlavipattiyā. Puna caparaṃ, āvuso, dussīlassa sīlavipannassa pāpako kittisaddo abbhuggacchati, ayaṃ dutiyo ādīnavo dussīlassa sīlavipattiyā. Puna caparaṃ, āvuso, dussīlo sīlavipanno yaññadeva parisaṃ upasaṅkamati yadi khattiyaparisaṃ yadi brāhmaṇaparisaṃ yadi gahapatiparisaṃ yadi samaṇaparisaṃ, avisārado upasaṅkamati maṅkubhūto, ayaṃ tatiyo ādīnavo dussīlassa sīlavipattiyā. Puna caparaṃ, āvuso, dussīlo sīlavipanno sammūḷho kālaṃ karoti, ayaṃ catuttho ādīnavo dussīlassa sīlavipattiyā. Puna caparaṃ, āvuso, dussīlo sīlavipanno kāyassa bhedā paraṃ maraṇā apāyaṃ duggatiṃ vinipātaṃ nirayaṃ upapajjati, ayaṃ pañcamo ādīnavo dussīlassa sīlavipattiyā.

‘‘Pañca ānisaṃsā sīlavato sīlasampadāya. Idhāvuso, sīlavā sīlasampanno appamādādhikaraṇaṃ mahantaṃ bhogakkhandhaṃ adhigacchati, ayaṃ paṭhamo ānisaṃso sīlavato sīlasampadāya. Puna caparaṃ, āvuso, sīlavato sīlasampannassa kalyāṇo kittisaddo abbhuggacchati, ayaṃ dutiyo ānisaṃso sīlavato sīlasampadāya. Puna caparaṃ, āvuso, sīlavā sīlasampanno yaññadeva parisaṃ upasaṅkamati yadi khattiyaparisaṃ yadi brāhmaṇaparisaṃ yadi gahapatiparisaṃ yadi samaṇaparisaṃ, visārado upasaṅkamati amaṅkubhūto, ayaṃ tatiyo ānisaṃso sīlavato sīlasampadāya. Puna caparaṃ, āvuso, sīlavā sīlasampanno asammūḷho kālaṃ karoti, ayaṃ catuttho ānisaṃso sīlavato sīlasampadāya. Puna caparaṃ, āvuso, sīlavā sīlasampanno kāyassa bhedā paraṃ maraṇā sugatiṃ saggaṃ lokaṃ upapajjati, ayaṃ pañcamo ānisaṃso sīlavato sīlasampadāya.

‘‘Codakena , āvuso, bhikkhunā paraṃ codetukāmena pañca dhamme ajjhattaṃ upaṭṭhapetvā paro codetabbo. Kālena vakkhāmi no akālena, bhūtena vakkhāmi no abhūtena, saṇhena vakkhāmi no pharusena, atthasaṃhitena vakkhāmi no anatthasaṃhitena, mettacittena\footnote{mettācittena (katthaci)} vakkhāmi no dosantarenāti. Codakena, āvuso, bhikkhunā paraṃ codetukāmena ime pañca dhamme ajjhattaṃ upaṭṭhapetvā paro codetabbo.

\paragraph{317.} ‘‘Pañca padhāniyaṅgāni. Idhāvuso, bhikkhu saddho hoti, saddahati tathāgatassa bodhiṃ – ‘itipi so bhagavā arahaṃ sammāsambuddho vijjācaraṇasampanno sugato, lokavidū anuttaro purisadammasārathi satthā devamanussānaṃ buddho bhagavā’ti. Appābādho hoti appātaṅko, samavepākiniyā gahaṇiyā samannāgato nātisītāya nāccuṇhāya majjhimāya padhānakkhamāya. Asaṭho hoti amāyāvī, yathābhūtaṃ attānaṃ āvikattā satthari vā viññūsu vā sabrahmacārīsu. Āraddhavīriyo viharati akusalānaṃ dhammānaṃ pahānāya kusalānaṃ dhammānaṃ upasampadāya thāmavā daḷhaparakkamo anikkhittadhuro kusalesu dhammesu. Paññavā hoti udayatthagāminiyā paññāya samannāgato ariyāya nibbedhikāya sammādukkhakkhayagāminiyā.

\paragraph{318.} ‘‘Pañca suddhāvāsā – avihā, atappā, sudassā, sudassī, akaniṭṭhā.

‘‘Pañca anāgāmino – antarāparinibbāyī, upahaccaparinibbāyī, asaṅkhāraparinibbāyī, sasaṅkhāraparinibbāyī, uddhaṃsotoakaniṭṭhagāmī.

\paragraph{319.} ‘‘Pañca cetokhilā. Idhāvuso, bhikkhu satthari kaṅkhati vicikicchati nādhimuccati na sampasīdati. Yo so, āvuso, bhikkhu satthari kaṅkhati vicikicchati nādhimuccati na sampasīdati, tassa cittaṃ na namati ātappāya anuyogāya sātaccāya padhānāya, yassa cittaṃ na namati ātappāya anuyogāya sātaccāya padhānāya, ayaṃ paṭhamo cetokhilo. Puna caparaṃ, āvuso, bhikkhu dhamme kaṅkhati vicikicchati…pe… saṅghe kaṅkhati vicikicchati… sikkhāya kaṅkhati vicikicchati… sabrahmacārīsu kupito hoti anattamano āhatacitto khilajāto. Yo so, āvuso, bhikkhu sabrahmacārīsu kupito hoti anattamano āhatacitto khilajāto, tassa cittaṃ na namati ātappāya anuyogāya sātaccāya padhānāya, yassa cittaṃ na namati ātappāya anuyogāya sātaccāya padhānāya, ayaṃ pañcamo cetokhilo.

\paragraph{320.} ‘‘Pañca cetasovinibandhā. Idhāvuso, bhikkhu kāmesu avītarāgo hoti avigatacchando avigatapemo avigatapipāso avigatapariḷāho avigatataṇho. Yo so, āvuso, bhikkhu kāmesu avītarāgo hoti avigatacchando avigatapemo avigatapipāso avigatapariḷāho avigatataṇho, tassa cittaṃ na namati ātappāya anuyogāya sātaccāya padhānāya. Yassa cittaṃ na namati ātappāya anuyogāya sātaccāya padhānāya. Ayaṃ paṭhamo cetaso vinibandho. Puna caparaṃ, āvuso, bhikkhu kāye avītarāgo hoti…pe… rūpe avītarāgo hoti…pe… puna caparaṃ, āvuso, bhikkhu yāvadatthaṃ udarāvadehakaṃ bhuñjitvā seyyasukhaṃ passasukhaṃ middhasukhaṃ anuyutto viharati…pe… puna caparaṃ, āvuso, bhikkhu aññataraṃ devanikāyaṃ paṇidhāya brahmacariyaṃ carati – ‘imināhaṃ sīlena vā vatena vā tapena vā brahmacariyena vā devo vā bhavissāmi devaññataro vā’ti. Yo so, āvuso, bhikkhu aññataraṃ devanikāyaṃ paṇidhāya brahmacariyaṃ carati – ‘imināhaṃ sīlena vā vatena vā tapena vā brahmacariyena vā devo vā bhavissāmi devaññataro vā’ti, tassa cittaṃ na namati ātappāya anuyogāya sātaccāya padhānāya. Yassa cittaṃ na namati ātappāya anuyogāya sātaccāya padhānāya. Ayaṃ pañcamo cetaso vinibandho.

‘‘Pañcindriyāni – cakkhundriyaṃ, sotindriyaṃ, ghānindriyaṃ, jivhindriyaṃ, kāyindriyaṃ.

‘‘Aparānipi pañcindriyāni – sukhindriyaṃ, dukkhindriyaṃ, somanassindriyaṃ, domanassindriyaṃ, upekkhindriyaṃ.

‘‘Aparānipi pañcindriyāni – saddhindriyaṃ, vīriyindriyaṃ, satindriyaṃ, samādhindriyaṃ, paññindriyaṃ.

\paragraph{321.} ‘‘Pañca nissaraṇiyā\footnote{nissāraṇīyā (sī. syā. kaṃ. pī.) ṭīkā oloketabbā}dhātuyo. Idhāvuso, bhikkhuno kāme manasikaroto kāmesu cittaṃ na pakkhandati na pasīdati na santiṭṭhati na vimuccati. Nekkhammaṃ kho panassa manasikaroto nekkhamme cittaṃ pakkhandati pasīdati santiṭṭhati vimuccati. Tassa taṃ cittaṃ sugataṃ subhāvitaṃ suvuṭṭhitaṃ suvimuttaṃ visaṃyuttaṃ kāmehi. Ye ca kāmapaccayā uppajjanti āsavā vighātā pariḷāhā\footnote{vighātapariḷāhā (syā. kaṃ.)}, mutto so tehi, na so taṃ vedanaṃ vedeti. Idamakkhātaṃ kāmānaṃ nissaraṇaṃ.

‘‘Puna caparaṃ, āvuso, bhikkhuno byāpādaṃ manasikaroto byāpāde cittaṃ na pakkhandati na pasīdati na santiṭṭhati na vimuccati. Abyāpādaṃ kho panassa manasikaroto abyāpāde cittaṃ pakkhandati pasīdati santiṭṭhati vimuccati. Tassa taṃ cittaṃ sugataṃ subhāvitaṃ suvuṭṭhitaṃ suvimuttaṃ visaṃyuttaṃ byāpādena. Ye ca byāpādapaccayā uppajjanti āsavā vighātā pariḷāhā, mutto so tehi, na so taṃ vedanaṃ vedeti. Idamakkhātaṃ byāpādassa nissaraṇaṃ.

‘‘Puna caparaṃ, āvuso, bhikkhuno vihesaṃ manasikaroto vihesāya cittaṃ na pakkhandati na pasīdati na santiṭṭhati na vimuccati. Avihesaṃ kho panassa manasikaroto avihesāya cittaṃ pakkhandati pasīdati santiṭṭhati vimuccati. Tassa taṃ cittaṃ sugataṃ subhāvitaṃ suvuṭṭhitaṃ suvimuttaṃ visaṃyuttaṃ vihesāya. Ye ca vihesāpaccayā uppajjanti āsavā vighātā pariḷāhā, mutto so tehi, na so taṃ vedanaṃ vedeti. Idamakkhātaṃ vihesāya nissaraṇaṃ.

‘‘Puna caparaṃ, āvuso, bhikkhuno rūpe manasikaroto rūpesu cittaṃ na pakkhandati na pasīdati na santiṭṭhati na vimuccati. Arūpaṃ kho panassa manasikaroto arūpe cittaṃ pakkhandati pasīdati santiṭṭhati vimuccati. Tassa taṃ cittaṃ sugataṃ subhāvitaṃ suvuṭṭhitaṃ suvimuttaṃ visaṃyuttaṃ rūpehi. Ye ca rūpapaccayā uppajjanti āsavā vighātā pariḷāhā, mutto so tehi, na so taṃ vedanaṃ vedeti. Idamakkhātaṃ rūpānaṃ nissaraṇaṃ.

‘‘Puna caparaṃ, āvuso, bhikkhuno sakkāyaṃ manasikaroto sakkāye cittaṃ na pakkhandati na pasīdati na santiṭṭhati na vimuccati. Sakkāyanirodhaṃ kho panassa manasikaroto sakkāyanirodhe cittaṃ pakkhandati pasīdati santiṭṭhati vimuccati. Tassa taṃ cittaṃ sugataṃ subhāvitaṃ suvuṭṭhitaṃ suvimuttaṃ visaṃyuttaṃ sakkāyena. Ye ca sakkāyapaccayā uppajjanti āsavā vighātā pariḷāhā, mutto so tehi, na so taṃ vedanaṃ vedeti. Idamakkhātaṃ sakkāyassa nissaraṇaṃ.

\paragraph{322.} ‘‘Pañca vimuttāyatanāni. Idhāvuso, bhikkhuno satthā dhammaṃ deseti aññataro vā garuṭṭhāniyo sabrahmacārī. Yathā yathā, āvuso, bhikkhuno satthā dhammaṃ deseti aññataro vā garuṭṭhāniyo sabrahmacārī . Tathā tathā so tasmiṃ dhamme atthapaṭisaṃvedī ca hoti dhammapaṭisaṃvedī ca. Tassa atthapaṭisaṃvedino dhammapaṭisaṃvedino pāmojjaṃ jāyati, pamuditassa pīti jāyati, pītimanassa kāyo passambhati, passaddhakāyo sukhaṃ vedeti, sukhino cittaṃ samādhiyati. Idaṃ paṭhamaṃ vimuttāyatanaṃ.

‘‘Puna caparaṃ, āvuso, bhikkhuno na heva kho satthā dhammaṃ deseti aññataro vā garuṭṭhāniyo sabrahmacārī, api ca kho yathāsutaṃ yathāpariyattaṃ dhammaṃ vitthārena paresaṃ deseti…pe… api ca kho yathāsutaṃ yathāpariyattaṃ dhammaṃ vitthārena sajjhāyaṃ karoti…pe… api ca kho yathāsutaṃ yathāpariyattaṃ dhammaṃ cetasā anuvitakketi anuvicāreti manasānupekkhati…pe… api ca khvassa aññataraṃ samādhinimittaṃ suggahitaṃ hoti sumanasikataṃ sūpadhāritaṃ suppaṭividdhaṃ paññāya. Yathā yathā, āvuso, bhikkhuno aññataraṃ samādhinimittaṃ suggahitaṃ hoti sumanasikataṃ sūpadhāritaṃ suppaṭividdhaṃ paññāya tathā tathā so tasmiṃ dhamme atthapaṭisaṃvedī ca hoti dhammapaṭisaṃvedī ca. Tassa atthapaṭisaṃvedino dhammapaṭisaṃvedino pāmojjaṃ jāyati, pamuditassa pīti jāyati, pītimanassa kāyo passambhati, passaddhakāyo sukhaṃ vedeti , sukhino cittaṃ samādhiyati. Idaṃ pañcamaṃ vimuttāyatanaṃ.

‘‘Pañca vimuttiparipācanīyā saññā – aniccasaññā, anicce dukkhasaññā, dukkhe anattasaññā, pahānasaññā, virāgasaññā.

‘‘Ime kho, āvuso, tena bhagavatā jānatā passatā arahatā sammāsambuddhena pañca dhammā sammadakkhātā; tattha sabbeheva saṅgāyitabbaṃ…pe… atthāya hitāya sukhāya devamanussānaṃ\footnote{saṅgitiyapañcakaṃ niṭṭhitaṃ (syā. kaṃ.)}.

\subsubsection{Chakkaṃ}

\paragraph{323.} ‘‘Atthi kho, āvuso, tena bhagavatā jānatā passatā arahatā sammāsambuddhena cha dhammā sammadakkhātā; tattha sabbeheva saṅgāyitabbaṃ…pe… atthāya hitāya sukhāya devamanussānaṃ. Katame cha?

‘‘Cha ajjhattikāniāyatanāni – cakkhāyatanaṃ, sotāyatanaṃ, ghānāyatanaṃ, jivhāyatanaṃ, kāyāyatanaṃ, manāyatanaṃ.

‘‘Cha bāhirāni āyatanāni – rūpāyatanaṃ, saddāyatanaṃ, gandhāyatanaṃ, rasāyatanaṃ, phoṭṭhabbāyatanaṃ, dhammāyatanaṃ.

‘‘Cha viññāṇakāyā – cakkhuviññāṇaṃ, sotaviññāṇaṃ, ghānaviññāṇaṃ, jivhāviññāṇaṃ, kāyaviññāṇaṃ, manoviññāṇaṃ.

‘‘Cha phassakāyā – cakkhusamphasso, sotasamphasso, ghānasamphasso, jivhāsamphasso, kāyasamphasso, manosamphasso.

‘‘Cha vedanākāyā – cakkhusamphassajā vedanā, sotasamphassajā vedanā, ghānasamphassajā vedanā, jivhāsamphassajā vedanā, kāyasamphassajā vedanā, manosamphassajā vedanā.

‘‘Cha saññākāyā – rūpasaññā, saddasaññā, gandhasaññā, rasasaññā, phoṭṭhabbasaññā, dhammasaññā.

‘‘Cha sañcetanākāyā – rūpasañcetanā, saddasañcetanā, gandhasañcetanā, rasasañcetanā, phoṭṭhabbasañcetanā, dhammasañcetanā.

‘‘Cha taṇhākāyā – rūpataṇhā, saddataṇhā, gandhataṇhā, rasataṇhā, phoṭṭhabbataṇhā, dhammataṇhā.

\paragraph{324.} ‘‘Cha agāravā. Idhāvuso, bhikkhu satthari agāravo viharati appatisso; dhamme agāravo viharati appatisso; saṅghe agāravo viharati appatisso; sikkhāya agāravo viharati appatisso; appamāde agāravo viharati appatisso; paṭisanthāre\footnote{paṭisandhāre (ka.)} agāravo viharati appatisso.

‘‘Cha gāravā. Idhāvuso, bhikkhu satthari sagāravo viharati sappatisso; dhamme sagāravo viharati sappatisso; saṅghe sagāravo viharati sappatisso; sikkhāya sagāravo viharati sappatisso; appamāde sagāravo viharati sappatisso; paṭisanthāre sagāravo viharati sappatisso.

‘‘Cha somanassūpavicārā. Cakkhunā rūpaṃ disvā somanassaṭṭhāniyaṃ rūpaṃ upavicarati; sotena saddaṃ sutvā… ghānena gandhaṃ ghāyitvā… jivhāya rasaṃ sāyitvā… kāyena phoṭṭhabbaṃ phusitvā. Manasā dhammaṃ viññāya somanassaṭṭhāniyaṃ dhammaṃ upavicarati.

‘‘Cha domanassūpavicārā. Cakkhunā rūpaṃ disvā domanassaṭṭhāniyaṃ rūpaṃ upavicarati…pe… manasā dhammaṃ viññāya domanassaṭṭhāniyaṃ dhammaṃ upavicarati.

‘‘Cha upekkhūpavicārā. Cakkhunā rūpaṃ disvā upekkhāṭṭhāniyaṃ\footnote{upekkhāṭhāniyaṃ (ka.)} rūpaṃ upavicarati…pe… manasā dhammaṃ viññāya upekkhāṭṭhāniyaṃ dhammaṃ upavicarati.

‘‘Cha sāraṇīyā dhammā. Idhāvuso, bhikkhuno mettaṃ kāyakammaṃ paccupaṭṭhitaṃ hoti sabrahmacārīsu āvi\footnote{āvī (ka. sī. pī. ka.)} ceva raho ca. Ayampi dhammo sāraṇīyo piyakaraṇo garukaraṇo saṅgahāya avivādāya sāmaggiyā ekībhāvāya saṃvattati.

‘‘Puna caparaṃ, āvuso, bhikkhuno mettaṃ vacīkammaṃ paccupaṭṭhitaṃ hoti sabrahmacārīsu āvi ceva raho ca. Ayampi dhammo sāraṇīyo…pe… ekībhāvāya saṃvattati.

‘‘Puna caparaṃ, āvuso, bhikkhuno mettaṃ manokammaṃ paccupaṭṭhitaṃ hoti sabrahmacārīsu āvi ceva raho ca. Ayampi dhammo sāraṇīyo…pe… ekībhāvāya saṃvattati.

‘‘Puna caparaṃ, āvuso, bhikkhu ye te lābhā dhammikā dhammaladdhā antamaso pattapariyāpannamattampi, tathārūpehi lābhehi appaṭivibhattabhogī hoti sīlavantehi sabrahmacārīhi sādhāraṇabhogī. Ayampi dhammo sāraṇīyo…pe… ekībhāvāya saṃvattati.

‘‘Puna caparaṃ, āvuso, bhikkhu yāni tāni sīlāni akhaṇḍāni acchiddāni asabalāni akammāsāni bhujissāni viññuppasatthāni aparāmaṭṭhāni samādhisaṃvattanikāni, tathārūpesu sīlesu sīlasāmaññagato viharati sabrahmacārīhi āvi ceva raho ca. Ayampi dhammo sāraṇīyo…pe… ekībhāvāya saṃvattati.

‘‘Puna caparaṃ, āvuso, bhikkhu yāyaṃ diṭṭhi ariyā niyyānikā niyyāti takkarassa sammā dukkhakkhayāya, tathārūpāya diṭṭhiyā diṭṭhisāmaññagato viharati sabrahmacārīhi āvi ceva raho ca. Ayampi dhammo sāraṇīyo piyakaraṇo garukaraṇo saṅgahāya avivādāya sāmaggiyā ekībhāvāya saṃvattati.

\paragraph{325.} Cha vivādamūlāni. Idhāvuso, bhikkhu kodhano hoti upanāhī. Yo so, āvuso, bhikkhu kodhano hoti upanāhī, so sattharipi agāravo viharati appatisso, dhammepi agāravo viharati appatisso, saṅghepi agāravo viharati appatisso, sikkhāyapi na paripūrakārī\footnote{paripūrīkārī (syā. kaṃ.)} hoti. Yo so, āvuso, bhikkhu satthari agāravo viharati appatisso, dhamme agāravo viharati appatisso, saṅghe agāravo viharati appatisso, sikkhāya na paripūrakārī, so saṅghe vivādaṃ janeti. Yo hoti vivādo bahujanaahitāya bahujanaasukhāya anatthāya ahitāya dukkhāya devamanussānaṃ. Evarūpaṃ ce tumhe, āvuso, vivādamūlaṃ ajjhattaṃ vā bahiddhā vā samanupasseyyātha. Tatra tumhe, āvuso, tasseva pāpakassa vivādamūlassa pahānāya vāyameyyātha. Evarūpaṃ ce tumhe, āvuso, vivādamūlaṃ ajjhattaṃ vā bahiddhā vā na samanupasseyyātha. Tatra tumhe, āvuso, tasseva pāpakassa vivādamūlassa āyatiṃ anavassavāya paṭipajjeyyātha. Evametassa pāpakassa vivādamūlassa pahānaṃ hoti. Evametassa pāpakassa vivādamūlassa āyatiṃ anavassavo hoti.

‘‘Puna caparaṃ, āvuso, bhikkhu makkhī hoti paḷāsī…pe… issukī hoti maccharī…pe… saṭho hoti māyāvī… pāpiccho hoti micchādiṭṭhī… sandiṭṭhiparāmāsī hoti ādhānaggāhī duppaṭinissaggī…pe… yo so, āvuso, bhikkhu sandiṭṭhiparāmāsī hoti ādhānaggāhī duppaṭinissaggī, so sattharipi agāravo viharati appatisso, dhammepi agāravo viharati appatisso, saṅghepi agāravo viharati appatisso, sikkhāyapi na paripūrakārī hoti. Yo so, āvuso, bhikkhu satthari agāravo viharati appatisso, dhamme agāravo viharati appatisso, saṅghe agāravo viharati appatisso , sikkhāya na paripūrakārī, so saṅghe vivādaṃ janeti. Yo hoti vivādo bahujanaahitāya bahujanaasukhāya anatthāya ahitāya dukkhāya devamanussānaṃ. Evarūpaṃ ce tumhe, āvuso, vivādamūlaṃ ajjhattaṃ vā bahiddhā vā samanupasseyyātha. Tatra tumhe, āvuso, tasseva pāpakassa vivādamūlassa pahānāya vāyameyyātha. Evarūpaṃ ce tumhe, āvuso, vivādamūlaṃ ajjhattaṃ vā bahiddhā vā na samanupasseyyātha. Tatra tumhe, āvuso, tasseva pāpakassa vivādamūlassa āyatiṃ anavassavāya paṭipajjeyyātha. Evametassa pāpakassa vivādamūlassa pahānaṃ hoti. Evametassa pāpakassa vivādamūlassa āyatiṃ anavassavo hoti.

‘‘Cha dhātuyo – pathavīdhātu, āpodhātu, tejodhātu, vāyodhātu, ākāsadhātu, viññāṇadhātu.

\paragraph{326.} ‘‘Cha nissaraṇiyā dhātuyo. Idhāvuso, bhikkhu evaṃ vadeyya – ‘mettā hi kho me cetovimutti bhāvitā bahulīkatā yānīkatā vatthukatā anuṭṭhitā paricitā susamāraddhā, atha ca pana me byāpādo cittaṃ pariyādāya tiṭṭhatī’ti. So ‘mā hevaṃ’, tissa vacanīyo, ‘māyasmā evaṃ avaca, mā bhagavantaṃ abbhācikkhi, na hi sādhu bhagavato abbhakkhānaṃ, na hi bhagavā evaṃ vadeyya. Aṭṭhānametaṃ, āvuso, anavakāso, yaṃ mettāya cetovimuttiyā bhāvitāya bahulīkatāya yānīkatāya vatthukatāya anuṭṭhitāya paricitāya susamāraddhāya. Atha ca panassa byāpādo cittaṃ pariyādāya ṭhassati, netaṃ ṭhānaṃ vijjati. Nissaraṇaṃ hetaṃ, āvuso, byāpādassa, yadidaṃ mettā cetovimuttī’ti.

‘‘Idha panāvuso, bhikkhu evaṃ vadeyya – ‘karuṇā hi kho me cetovimutti bhāvitā bahulīkatā yānīkatā vatthukatā anuṭṭhitā paricitā susamāraddhā. Atha ca pana me vihesā cittaṃ pariyādāya tiṭṭhatī’ti, so ‘mā hevaṃ’ tissa vacanīyo ‘māyasmā evaṃ avaca, mā bhagavantaṃ abbhācikkhi, na hi sādhu bhagavato abbhakkhānaṃ, na hi bhagavā evaṃ vadeyya. Aṭṭhānametaṃ āvuso, anavakāso, yaṃ karuṇāya cetovimuttiyā bhāvitāya bahulīkatāya yānīkatāya vatthukatāya anuṭṭhitāya paricitāya susamāraddhāya, atha ca panassa vihesā cittaṃ pariyādāya ṭhassati, netaṃ ṭhānaṃ vijjati. Nissaraṇaṃ hetaṃ, āvuso, vihesāya, yadidaṃ karuṇā cetovimuttī’ti.

‘‘Idha panāvuso, bhikkhu evaṃ vadeyya – ‘muditā hi kho me cetovimutti bhāvitā bahulīkatā yānīkatā vatthukatā anuṭṭhitā paricitā susamāraddhā. Atha ca pana me arati cittaṃ pariyādāya tiṭṭhatī’ti, so ‘mā hevaṃ’ tissa vacanīyo ‘‘māyasmā evaṃ avaca, mā bhagavantaṃ abbhācikkhi, na hi sādhu bhagavato abbhakkhānaṃ, na hi bhagavā evaṃ vadeyya. Aṭṭhānametaṃ, āvuso, anavakāso, yaṃ muditāya cetovimuttiyā bhāvitāya bahulīkatāya yānīkatāya vatthukatāya anuṭṭhitāya paricitāya susamāraddhāya, atha ca panassa arati cittaṃ pariyādāya ṭhassati, netaṃ ṭhānaṃ vijjati. Nissaraṇaṃ hetaṃ, āvuso, aratiyā, yadidaṃ muditā cetovimuttī’ti.

‘‘Idha panāvuso, bhikkhu evaṃ vadeyya – ‘upekkhā hi kho me cetovimutti bhāvitā bahulīkatā yānīkatā vatthukatā anuṭṭhitā paricitā susamāraddhā. Atha ca pana me rāgo cittaṃ pariyādāya tiṭṭhatī’ti. So ‘mā hevaṃ’ tissa vacanīyo ‘māyasmā evaṃ avaca, mā bhagavantaṃ abbhācikkhi, na hi sādhu bhagavato abbhakkhānaṃ, na hi bhagavā evaṃ vadeyya. Aṭṭhānametaṃ, āvuso, anavakāso, yaṃ upekkhāya cetovimuttiyā bhāvitāya bahulīkatāya yānīkatāya vatthukatāya anuṭṭhitāya paricitāya susamāraddhāya, atha ca panassa rāgo cittaṃ pariyādāya ṭhassati netaṃ ṭhānaṃ vijjati. Nissaraṇaṃ hetaṃ, āvuso, rāgassa, yadidaṃ upekkhā cetovimuttī’ti.

‘‘Idha panāvuso, bhikkhu evaṃ vadeyya – ‘animittā hi kho me cetovimutti bhāvitā bahulīkatā yānīkatā vatthukatā anuṭṭhitā paricitā susamāraddhā. Atha ca pana me nimittānusāri viññāṇaṃ hotī’ti. So ‘mā hevaṃ’ tissa vacanīyo ‘māyasmā evaṃ avaca, mā bhagavantaṃ abbhācikkhi, na hi sādhu bhagavato abbhakkhānaṃ, na hi bhagavā evaṃ vadeyya. Aṭṭhānametaṃ, āvuso, anavakāso, yaṃ animittāya cetovimuttiyā bhāvitāya bahulīkatāya yānīkatāya vatthukatāya anuṭṭhitāya paricitāya susamāraddhāya, atha ca panassa nimittānusāri viññāṇaṃ bhavissati, netaṃ ṭhānaṃ vijjati. Nissaraṇaṃ hetaṃ, āvuso, sabbanimittānaṃ, yadidaṃ animittā cetovimuttī’ti.

‘‘Idha panāvuso, bhikkhu evaṃ vadeyya – ‘asmīti kho me vigataṃ\footnote{vighātaṃ (sī. pī.), vigate (syā. ka.)}, ayamahamasmīti na samanupassāmi, atha ca pana me vicikicchākathaṅkathāsallaṃ cittaṃ pariyādāya tiṭṭhatī’ti. So ‘mā hevaṃ’ tissa vacanīyo ‘māyasmā evaṃ avaca, mā bhagavantaṃ abbhācikkhi, na hi sādhu bhagavato abbhakkhānaṃ, na hi bhagavā evaṃ vadeyya. Aṭṭhānametaṃ, āvuso, anavakāso, yaṃ asmīti vigate\footnote{vighāte (sī. pī.)} ayamahamasmīti asamanupassato, atha ca panassa vicikicchākathaṅkathāsallaṃ cittaṃ pariyādāya ṭhassati, netaṃ ṭhānaṃ vijjati. Nissaraṇaṃ hetaṃ, āvuso, vicikicchākathaṅkathāsallassa, yadidaṃ asmimānasamugghāto’ti.

\paragraph{327.} ‘‘Cha anuttariyāni – dassanānuttariyaṃ, savanānuttariyaṃ, lābhānuttariyaṃ, sikkhānuttariyaṃ, pāricariyānuttariyaṃ, anussatānuttariyaṃ.

‘‘Cha anussatiṭṭhānāni – buddhānussati, dhammānussati, saṅghānussati, sīlānussati, cāgānussati, devatānussati.

\paragraph{328.} ‘‘Cha satatavihārā. Idhāvuso, bhikkhu cakkhunā rūpaṃ disvā neva sumano hoti na dummano, upekkhako\footnote{upekkhako ca (syā. ka.)} viharati sato sampajāno. Sotena saddaṃ sutvā…pe… manasā dhammaṃ viññāya neva sumano hoti na dummano, upekkhako viharati sato sampajāno.

\paragraph{329.}‘‘Chaḷābhijātiyo. Idhāvuso, ekacco kaṇhābhijātiko samāno kaṇhaṃ dhammaṃ abhijāyati. Idha panāvuso, ekacco kaṇhābhijātiko samāno sukkaṃ dhammaṃ abhijāyati. Idha panāvuso, ekacco kaṇhābhijātiko samāno akaṇhaṃ asukkaṃ nibbānaṃ abhijāyati. Idha panāvuso, ekacco sukkābhijātiko samāno sukkaṃ dhammaṃ abhijāyati. Idha panāvuso, ekacco sukkābhijātiko samāno kaṇhaṃ dhammaṃ abhijāyati. Idha panāvuso, ekacco sukkābhijātiko samāno akaṇhaṃ asukkaṃ nibbānaṃ abhijāyati.

‘‘Cha nibbedhabhāgiyā saññā\footnote{nibbedhabhāgiyasaññā (syā. kaṃ.)} – aniccasaññā anicce, dukkhasaññā dukkhe, anattasaññā, pahānasaññā, virāgasaññā, nirodhasaññā.

‘‘Ime kho, āvuso, tena bhagavatā jānatā passatā arahatā sammāsambuddhena cha dhammā sammadakkhātā; tattha sabbeheva saṅgāyitabbaṃ…pe… atthāya hitāya sukhāya devamanussānaṃ.

\subsubsection{Sattakaṃ}

\paragraph{330.} ‘‘Atthi kho, āvuso, tena bhagavatā jānatā passatā arahatā sammāsambuddhena satta dhammā sammadakkhātā; tattha sabbeheva saṅgāyitabbaṃ…pe… atthāya hitāya sukhāya devamanussānaṃ. Katame satta?

‘‘Satta ariyadhanāni – saddhādhanaṃ, sīladhanaṃ, hiridhanaṃ, ottappadhanaṃ, sutadhanaṃ, cāgadhanaṃ, paññādhanaṃ.

‘‘Satta bojjhaṅgā – satisambojjhaṅgo, dhammavicayasambojjhaṅgo , vīriyasambojjhaṅgo, pītisambojjhaṅgo, passaddhisambojjhaṅgo, samādhisambojjhaṅgo, upekkhāsambojjhaṅgo.

‘‘Satta samādhiparikkhārā – sammādiṭṭhi, sammāsaṅkappo, sammāvācā, sammākammanto, sammāājīvo, sammāvāyāmo, sammāsati.

‘‘Satta asaddhammā – idhāvuso, bhikkhu assaddho hoti, ahiriko hoti, anottappī hoti, appassuto hoti, kusīto hoti, muṭṭhassati hoti, duppañño hoti.

‘‘Satta saddhammā – idhāvuso, bhikkhu saddho hoti, hirimā hoti, ottappī hoti, bahussuto hoti, āraddhavīriyo hoti, upaṭṭhitassati hoti, paññavā hoti.

‘‘Satta sappurisadhammā – idhāvuso, bhikkhu dhammaññū ca hoti atthaññū ca attaññū ca mattaññū ca kālaññū ca parisaññū ca puggalaññū ca.

\paragraph{331.} ‘‘Satta niddasavatthūni. Idhāvuso, bhikkhu sikkhāsamādāne tibbacchando hoti, āyatiñca sikkhāsamādāne avigatapemo. Dhammanisantiyā tibbacchando hoti, āyatiñca dhammanisantiyā avigatapemo. Icchāvinaye tibbacchando hoti, āyatiñca icchāvinaye avigatapemo. Paṭisallāne tibbacchando hoti, āyatiñca paṭisallāne avigatapemo. Vīriyārambhe tibbacchando hoti, āyatiñca vīriyārambhe avigatapemo. Satinepakke tibbacchando hoti, āyatiñca satinepakke avigatapemo . Diṭṭhipaṭivedhe tibbacchando hoti, āyatiñca diṭṭhipaṭivedhe avigatapemo.

‘‘Satta saññā – aniccasaññā, anattasaññā, asubhasaññā, ādīnavasaññā, pahānasaññā, virāgasaññā, nirodhasaññā.

‘‘Sattabalāni – saddhābalaṃ, vīriyabalaṃ, hiribalaṃ, ottappabalaṃ, satibalaṃ, samādhibalaṃ, paññābalaṃ.

\paragraph{332.} ‘‘Satta viññāṇaṭṭhitiyo. Santāvuso, sattā nānattakāyā nānattasaññino, seyyathāpi manussā ekacce ca devā ekacce ca vinipātikā. Ayaṃ paṭhamā viññāṇaṭṭhiti.

‘‘Santāvuso, sattā nānattakāyā ekattasaññino seyyathāpi devā brahmakāyikā paṭhamābhinibbattā. Ayaṃ dutiyā viññāṇaṭṭhiti.

‘‘Santāvuso, sattā ekattakāyā nānattasaññino seyyathāpi devā ābhassarā. Ayaṃ tatiyā viññāṇaṭṭhiti.

‘‘Santāvuso, sattā ekattakāyā ekattasaññino seyyathāpi devā subhakiṇhā. Ayaṃ catutthī viññāṇaṭṭhiti.

‘‘Santāvuso, sattā sabbaso rūpasaññānaṃ samatikkamā paṭighasaññānaṃ atthaṅgamā nānattasaññānaṃ amanasikārā ‘ananto ākāso’ti ākāsānañcāyatanūpagā. Ayaṃ pañcamī viññāṇaṭṭhiti.

‘‘Santāvuso, sattā sabbaso ākāsānañcāyatanaṃ samatikkamma ‘anantaṃ viññāṇa’nti viññāṇañcāyatanūpagā. Ayaṃ chaṭṭhī viññāṇaṭṭhiti.

‘‘Santāvuso , sattā sabbaso viññāṇañcāyatanaṃ samatikkamma ‘natthi kiñcī’ti ākiñcaññāyatanūpagā. Ayaṃ sattamī viññāṇaṭṭhiti.

‘‘Satta puggalā dakkhiṇeyyā – ubhatobhāgavimutto , paññāvimutto, kāyasakkhi, diṭṭhippatto, saddhāvimutto, dhammānusārī, saddhānusārī.

‘‘Satta anusayā – kāmarāgānusayo, paṭighānusayo, diṭṭhānusayo, vicikicchānusayo, mānānusayo, bhavarāgānusayo, avijjānusayo.

‘‘Satta saññojanāni – anunayasaññojanaṃ\footnote{kāmasaññojanaṃ (syā. kaṃ.)}, paṭighasaññojanaṃ, diṭṭhisaññojanaṃ, vicikicchāsaññojanaṃ, mānasaññojanaṃ, bhavarāgasaññojanaṃ, avijjāsaññojanaṃ.

‘‘Satta adhikaraṇasamathā – uppannuppannānaṃ adhikaraṇānaṃ samathāya vūpasamāya sammukhāvinayo dātabbo, sativinayo dātabbo, amūḷhavinayo dātabbo, paṭiññāya kāretabbaṃ, yebhuyyasikā, tassapāpiyasikā, tiṇavatthārako.

‘‘Ime kho, āvuso, tena bhagavatā jānatā passatā arahatā sammāsambuddhena satta dhammā sammadakkhātā; tattha sabbeheva saṅgāyitabbaṃ…pe… atthāya hitāya sukhāya devamanussānaṃ.

\xsubsubsectionEnd{Dutiyabhāṇavāro niṭṭhito.}

\subsubsection{Aṭṭhakaṃ}

\paragraph{333.} ‘‘Atthi kho, āvuso, tena bhagavatā jānatā passatā arahatā sammāsambuddhena aṭṭha dhammā sammadakkhātā; tattha sabbeheva saṅgāyitabbaṃ…pe… atthāya hitāya sukhāya devamanussānaṃ. Katame aṭṭha?

‘‘Aṭṭha micchattā – micchādiṭṭhi, micchāsaṅkappo, micchāvācā, micchākammanto, micchāājīvo, micchāvāyāmo micchāsati, micchāsamādhi.

‘‘Aṭṭha sammattā – sammādiṭṭhi, sammāsaṅkappo, sammāvācā, sammākammanto, sammāājīvo, sammāvāyāmo, sammāsati, sammāsamādhi.

‘‘Aṭṭha puggalā dakkhiṇeyyā – sotāpanno, sotāpattiphalasacchikiriyāya paṭipanno; sakadāgāmī, sakadāgāmiphalasacchikiriyāya paṭipanno; anāgāmī, anāgāmiphalasacchikiriyāya paṭipanno; arahā, arahattaphalasacchikiriyāya paṭipanno.

\paragraph{334.} ‘‘Aṭṭha kusītavatthūni. Idhāvuso, bhikkhunā kammaṃ kātabbaṃ hoti. Tassa evaṃ hoti – ‘kammaṃ kho me kātabbaṃ bhavissati, kammaṃ kho pana me karontassa kāyo kilamissati, handāhaṃ nipajjāmī’ti! So nipajjati na vīriyaṃ ārabhati appattassa pattiyā anadhigatassa adhigamāya asacchikatassa sacchikiriyāya. Idaṃ paṭhamaṃ kusītavatthu.

‘‘Puna caparaṃ, āvuso, bhikkhunā kammaṃ kataṃ hoti. Tassa evaṃ hoti – ‘ahaṃ kho kammaṃ akāsiṃ, kammaṃ kho pana me karontassa kāyo kilanto, handāhaṃ nipajjāmī’ti! So nipajjati na vīriyaṃ ārabhati…pe… idaṃ dutiyaṃ kusītavatthu.

‘‘Puna caparaṃ, āvuso, bhikkhunā maggo gantabbo hoti. Tassa evaṃ hoti – ‘maggo kho me gantabbo bhavissati, maggaṃ kho pana me gacchantassa kāyo kilamissati, handāhaṃ nipajjāmī’ti! So nipajjati na vīriyaṃ ārabhati… idaṃ tatiyaṃ kusītavatthu.

‘‘Puna caparaṃ, āvuso, bhikkhunā maggo gato hoti. Tassa evaṃ hoti – ‘ahaṃ kho maggaṃ agamāsiṃ, maggaṃ kho pana me gacchantassa kāyo kilanto, handāhaṃ nipajjāmī’ti! So nipajjati na vīriyaṃ ārabhati… idaṃ catutthaṃ kusītavatthu.

‘‘Puna caparaṃ, āvuso, bhikkhu gāmaṃ vā nigamaṃ vā piṇḍāya caranto na labhati lūkhassa vā paṇītassa vā bhojanassa yāvadatthaṃ pāripūriṃ. Tassa evaṃ hoti – ‘ahaṃ kho gāmaṃ vā nigamaṃ vā piṇḍāya caranto nālatthaṃ lūkhassa vā paṇītassa vā bhojanassa yāvadatthaṃ pāripūriṃ, tassa me kāyo kilanto akammañño, handāhaṃ nipajjāmī’ti! So nipajjati na vīriyaṃ ārabhati… idaṃ pañcamaṃ kusītavatthu.

‘‘Puna caparaṃ, āvuso, bhikkhu gāmaṃ vā nigamaṃ vā piṇḍāya caranto labhati lūkhassa vā paṇītassa vā bhojanassa yāvadatthaṃ pāripūriṃ. Tassa evaṃ hoti – ‘ahaṃ kho gāmaṃ vā nigamaṃ vā piṇḍāya caranto alatthaṃ lūkhassa vā paṇītassa vā bhojanassa yāvadatthaṃ pāripūriṃ, tassa me kāyo garuko akammañño, māsācitaṃ maññe , handāhaṃ nipajjāmī’ti! So nipajjati na vīriyaṃ ārabhati… idaṃ chaṭṭhaṃ kusītavatthu.

‘‘Puna caparaṃ, āvuso, bhikkhuno uppanno hoti appamattako ābādho. Tassa evaṃ hoti – ‘uppanno kho me ayaṃ appamattako ābādho; atthi kappo nipajjituṃ, handāhaṃ nipajjāmī’ti! So nipajjati na vīriyaṃ ārabhati… idaṃ sattamaṃ kusītavatthu.

‘‘Puna caparaṃ, āvuso, bhikkhu gilānā vuṭṭhito\footnote{gilānavuṭṭhito (saddanīti) a. ni. 6.16 nakulapitusuttaṭīkā passitabbā} hoti aciravuṭṭhito gelaññā. Tassa evaṃ hoti – ‘ahaṃ kho gilānā vuṭṭhito aciravuṭṭhito gelaññā, tassa me kāyo dubbalo akammañño, handāhaṃ nipajjāmī’ti! So nipajjati na vīriyaṃ ārabhati appattassa pattiyā anadhigatassa adhigamāya asacchikatassa sacchikiriyāya. Idaṃ aṭṭhamaṃ kusītavatthu.

\paragraph{335.} ‘‘Aṭṭha ārambhavatthūni. Idhāvuso, bhikkhunā kammaṃ kātabbaṃ hoti. Tassa evaṃ hoti – ‘kammaṃ kho me kātabbaṃ bhavissati, kammaṃ kho pana me karontena na sukaraṃ buddhānaṃ sāsanaṃ manasi kātuṃ, handāhaṃ vīriyaṃ ārabhāmi appattassa pattiyā anadhigatassa adhigamāya, asacchikatassa sacchikiriyāyā’ti! So vīriyaṃ ārabhati appattassa pattiyā, anadhigatassa adhigamāya asacchikatassa sacchikiriyāya. Idaṃ paṭhamaṃ ārambhavatthu.

‘‘Puna caparaṃ, āvuso, bhikkhunā kammaṃ kataṃ hoti. Tassa evaṃ hoti – ‘ahaṃ kho kammaṃ akāsiṃ, kammaṃ kho panāhaṃ karonto nāsakkhiṃ buddhānaṃ sāsanaṃ manasi kātuṃ, handāhaṃ vīriyaṃ ārabhāmi…pe… so vīriyaṃ ārabhati… idaṃ dutiyaṃ ārambhavatthu.

‘‘Puna caparaṃ, āvuso, bhikkhunā maggo gantabbo hoti. Tassa evaṃ hoti – ‘maggo kho me gantabbo bhavissati, maggaṃ kho pana me gacchantena na sukaraṃ buddhānaṃ sāsanaṃ manasi kātuṃ. Handāhaṃ vīriyaṃ ārabhāmi…pe… so vīriyaṃ ārabhati… idaṃ tatiyaṃ ārambhavatthu.

‘‘Puna caparaṃ, āvuso, bhikkhunā maggo gato hoti. Tassa evaṃ hoti – ‘ahaṃ kho maggaṃ agamāsiṃ, maggaṃ kho panāhaṃ gacchanto nāsakkhiṃ buddhānaṃ sāsanaṃ manasi kātuṃ, handāhaṃ vīriyaṃ ārabhāmi…pe… so vīriyaṃ ārabhati… idaṃ catutthaṃ ārambhavatthu.

‘‘Puna caparaṃ, āvuso, bhikkhu gāmaṃ vā nigamaṃ vā piṇḍāya caranto na labhati lūkhassa vā paṇītassa vā bhojanassa yāvadatthaṃ pāripūriṃ . Tassa evaṃ hoti – ‘ahaṃ kho gāmaṃ vā nigamaṃ vā piṇḍāya caranto nālatthaṃ lūkhassa vā paṇītassa vā bhojanassa yāvadatthaṃ pāripūriṃ, tassa me kāyo lahuko kammañño, handāhaṃ vīriyaṃ ārabhāmi…pe… so vīriyaṃ ārabhati… idaṃ pañcamaṃ ārambhavatthu.

‘‘Puna caparaṃ, āvuso, bhikkhu gāmaṃ vā nigamaṃ vā piṇḍāya caranto labhati lūkhassa vā paṇītassa vā bhojanassa yāvadatthaṃ pāripūriṃ. Tassa evaṃ hoti – ‘ahaṃ kho gāmaṃ vā nigamaṃ vā piṇḍāya caranto alatthaṃ lūkhassa vā paṇītassa vā bhojanassa yāvadatthaṃ pāripūriṃ, tassa me kāyo balavā kammañño, handāhaṃ vīriyaṃ ārabhāmi…pe… so vīriyaṃ ārabhati… idaṃ chaṭṭhaṃ ārambhavatthu .

‘‘Puna caparaṃ, āvuso, bhikkhuno uppanno hoti appamattako ābādho. Tassa evaṃ hoti – ‘uppanno kho me ayaṃ appamattako ābādho, ṭhānaṃ kho panetaṃ vijjati yaṃ me ābādho pavaḍḍheyya, handāhaṃ vīriyaṃ ārabhāmi…pe… so vīriyaṃ ārabhati… idaṃ sattamaṃ ārambhavatthu.

‘‘Puna caparaṃ, āvuso, bhikkhu gilānā vuṭṭhito hoti aciravuṭṭhito gelaññā. Tassa evaṃ hoti – ‘ahaṃ kho gilānā vuṭṭhito aciravuṭṭhito gelaññā, ṭhānaṃ kho panetaṃ vijjati yaṃ me ābādho paccudāvatteyya, handāhaṃ vīriyaṃ ārabhāmi appattassa pattiyā anadhigatassa adhigamāya asacchikatassa sacchikiriyāyā’’ti! So vīriyaṃ ārabhati appattassa pattiyā anadhigatassa adhigamāya asacchikatassa sacchikiriyāya. Idaṃ aṭṭhamaṃ ārambhavatthu.

\paragraph{336.} ‘‘Aṭṭha dānavatthūni. Āsajja dānaṃ deti, bhayā dānaṃ deti, ‘adāsi me’ti dānaṃ deti, ‘dassati me’ti dānaṃ deti, ‘sāhu dāna’nti dānaṃ deti, ‘ahaṃ pacāmi, ime na pacanti, nārahāmi pacanto apacantānaṃ dānaṃ na dātu’nti dānaṃ deti, ‘idaṃ me dānaṃ dadato kalyāṇo kittisaddo abbhuggacchatī’ti dānaṃ deti. Cittālaṅkāra-cittaparikkhāratthaṃ dānaṃ deti.

\paragraph{337.} ‘‘Aṭṭha dānūpapattiyo. Idhāvuso, ekacco dānaṃ deti samaṇassa vā brāhmaṇassa vā annaṃ pānaṃ vatthaṃ yānaṃ mālāgandhavilepanaṃ seyyāvasathapadīpeyyaṃ. So yaṃ deti taṃ paccāsīsati\footnote{paccāsiṃsati (sī. syā. kaṃ. pī.)}. So passati khattiyamahāsālaṃ vā brāhmaṇamahāsālaṃ vā gahapatimahāsālaṃ vā pañcahi kāmaguṇehi samappitaṃ samaṅgībhūtaṃ paricārayamānaṃ. Tassa evaṃ hoti – ‘aho vatāhaṃ kāyassa bhedā paraṃ maraṇā khattiyamahāsālānaṃ vā brāhmaṇamahāsālānaṃ vā gahapatimahāsālānaṃ vā sahabyataṃ upapajjeyya’nti! So taṃ cittaṃ dahati, taṃ cittaṃ adhiṭṭhāti, taṃ cittaṃ bhāveti, tassa taṃ cittaṃ hīne vimuttaṃ uttari abhāvitaṃ tatrūpapattiyā saṃvattati . Tañca kho sīlavato vadāmi no dussīlassa. Ijjhatāvuso, sīlavato cetopaṇidhi visuddhattā.

‘‘Puna caparaṃ, āvuso, idhekacco dānaṃ deti samaṇassa vā brāhmaṇassa vā annaṃ pānaṃ…pe… seyyāvasathapadīpeyyaṃ. So yaṃ deti taṃ paccāsīsati. Tassa sutaṃ hoti – ‘cātumahārājikā\footnote{cātummahārājikā (sī. syā. pī.)} devā dīghāyukā vaṇṇavanto sukhabahulā’’ti. Tassa evaṃ hoti – ‘aho vatāhaṃ kāyassa bhedā paraṃ maraṇā cātumahārājikānaṃ devānaṃ sahabyataṃ upapajjeyya’’nti! So taṃ cittaṃ dahati, taṃ cittaṃ adhiṭṭhāti, taṃ cittaṃ bhāveti, tassa taṃ cittaṃ hīne vimuttaṃ uttari abhāvitaṃ tatrūpapattiyā saṃvattati. Tañca kho sīlavato vadāmi no dussīlassa. Ijjhatāvuso, sīlavato cetopaṇidhi visuddhattā.

‘‘Puna caparaṃ, āvuso, idhekacco dānaṃ deti samaṇassa vā brāhmaṇassa vā annaṃ pānaṃ…pe… seyyāvasathapadīpeyyaṃ. So yaṃ deti taṃ paccāsīsati. Tassa sutaṃ hoti – ‘tāvatiṃsā devā…pe… yāmā devā…pe… tusitā devā…pe… nimmānaratī devā…pe… paranimmitavasavattī devā dīghāyukā vaṇṇavanto sukhabahulā’ti. Tassa evaṃ hoti – ‘aho vatāhaṃ kāyassa bhedā paraṃ maraṇā paranimmitavasavattīnaṃ devānaṃ sahabyataṃ upapajjeyya’’nti! So taṃ cittaṃ dahati, taṃ cittaṃ adhiṭṭhāti, taṃ cittaṃ bhāveti, tassa taṃ cittaṃ hīne vimuttaṃ uttari abhāvitaṃ tatrūpapattiyā saṃvattati. Tañca kho sīlavato vadāmi no dussīlassa. Ijjhatāvuso, sīlavato cetopaṇidhi visuddhattā.

‘‘Puna caparaṃ, āvuso, idhekacco dānaṃ deti samaṇassa vā brāhmaṇassa vā annaṃ pānaṃ vatthaṃ yānaṃ mālāgandhavilepanaṃ seyyāvasathapadīpeyyaṃ. So yaṃ deti taṃ paccāsīsati. Tassa sutaṃ hoti – ‘brahmakāyikā devā dīghāyukā vaṇṇavanto sukhabahulā’ti. Tassa evaṃ hoti – ‘aho vatāhaṃ kāyassa bhedā paraṃ maraṇā brahmakāyikānaṃ devānaṃ sahabyataṃ upapajjeyya’nti! So taṃ cittaṃ dahati, taṃ cittaṃ adhiṭṭhāti, taṃ cittaṃ bhāveti, tassa taṃ cittaṃ hīne vimuttaṃ uttari abhāvitaṃ tatrūpapattiyā saṃvattati. Tañca kho sīlavato vadāmi no dussīlassa; vītarāgassa no sarāgassa. Ijjhatāvuso, sīlavato cetopaṇidhi vītarāgattā.

‘‘Aṭṭha parisā – khattiyaparisā, brāhmaṇaparisā, gahapatiparisā, samaṇaparisā, cātumahārājikaparisā, tāvatiṃsaparisā, māraparisā, brahmaparisā .

‘‘Aṭṭha lokadhammā – lābho ca, alābho ca, yaso ca, ayaso ca, nindā ca, pasaṃsā ca, sukhañca, dukkhañca.

\paragraph{338.} ‘‘Aṭṭha abhibhāyatanāni. Ajjhattaṃ rūpasaññī eko bahiddhā rūpāni passati parittāni suvaṇṇadubbaṇṇāni, ‘tāni abhibhuyya jānāmi passāmī’ti evaṃsaññī hoti. Idaṃ paṭhamaṃ abhibhāyatanaṃ.

‘‘Ajjhattaṃ rūpasaññī eko bahiddhā rūpāni passati appamāṇāni suvaṇṇadubbaṇṇāni, ‘tāni abhibhuyya jānāmi passāmī’ti – evaṃsaññī hoti. Idaṃ dutiyaṃ abhibhāyatanaṃ.

‘‘Ajjhattaṃ arūpasaññī eko bahiddhā rūpāni passati parittāni suvaṇṇadubbaṇṇāni, ‘tāni abhibhuyya jānāmi passāmī’ti evaṃsaññī hoti. Idaṃ tatiyaṃ abhibhāyatanaṃ.

‘‘Ajjhattaṃ arūpasaññī eko bahiddhā rūpāni passati appamāṇāni suvaṇṇadubbaṇṇāni, ‘tāni abhibhuyya jānāmi passāmī’ti evaṃsaññī hoti. Idaṃ catutthaṃ abhibhāyatanaṃ.

‘‘Ajjhattaṃ arūpasaññī eko bahiddhā rūpāni passati nīlāni nīlavaṇṇāni nīlanidassanāni nīlanibhāsāni. Seyyathāpi nāma umāpupphaṃ nīlaṃ nīlavaṇṇaṃ nīlanidassanaṃ nīlanibhāsaṃ, seyyathā vā pana taṃ vatthaṃ bārāṇaseyyakaṃ ubhatobhāgavimaṭṭhaṃ nīlaṃ nīlavaṇṇaṃ nīlanidassanaṃ nīlanibhāsaṃ. Evameva\footnote{evamevaṃ (ka.)} ajjhattaṃ arūpasaññī eko bahiddhā rūpāni passati nīlāni nīlavaṇṇāni nīlanidassanāni nīlanibhāsāni, ‘tāni abhibhuyya jānāmi passāmī’ti evaṃsaññī hoti. Idaṃ pañcamaṃ abhibhāyatanaṃ.

‘‘Ajjhattaṃ arūpasaññī eko bahiddhā rūpāni passati pītāni pītavaṇṇāni pītanidassanāni pītanibhāsāni. Seyyathāpi nāma kaṇikārapupphaṃ\footnote{kaṇṇikārapupphaṃ (syā. kaṃ.)} pītaṃ pītavaṇṇaṃ pītanidassanaṃ pītanibhāsaṃ, seyyathā vā pana taṃ vatthaṃ bārāṇaseyyakaṃ ubhatobhāgavimaṭṭhaṃ pītaṃ pītavaṇṇaṃ pītanidassanaṃ pītanibhāsaṃ. Evameva ajjhattaṃ arūpasaññī eko bahiddhā rūpāni passati pītāni pītavaṇṇāni pītanidassanāni pītanibhāsāni, ‘tāni abhibhuyya jānāmi passāmī’ti evaṃsaññī hoti. Idaṃ chaṭṭhaṃ abhibhāyatanaṃ.

‘‘Ajjhattaṃ arūpasaññī eko bahiddhā rūpāni passati lohitakāni lohitakavaṇṇāni lohitakanidassanāni lohitakanibhāsāni. Seyyathāpi nāma bandhujīvakapupphaṃ lohitakaṃ lohitakavaṇṇaṃ lohitakanidassanaṃ lohitakanibhāsaṃ, seyyathā vā pana taṃ vatthaṃ bārāṇaseyyakaṃ ubhatobhāgavimaṭṭhaṃ lohitakaṃ lohitakavaṇṇaṃ lohitakanidassanaṃ lohitakanibhāsaṃ. Evameva ajjhattaṃ arūpasaññī eko bahiddhā rūpāni passati lohitakāni lohitakavaṇṇāni lohitakanidassanāni lohitakanibhāsāni, ‘tāni abhibhuyya jānāmi passāmī’ti evaṃsaññī hoti. Idaṃ sattamaṃ abhibhāyatanaṃ.

‘‘Ajjhattaṃ arūpasaññī eko bahiddhā rūpāni passati odātāni odātavaṇṇāni odātanidassanāni odātanibhāsāni. Seyyathāpi nāma osadhitārakā odātā odātavaṇṇā odātanidassanā odātanibhāsā, seyyathā vā pana taṃ vatthaṃ bārāṇaseyyakaṃ ubhatobhāgavimaṭṭhaṃ odātaṃ odātavaṇṇaṃ odātanidassanaṃ odātanibhāsaṃ. Evameva ajjhattaṃ arūpasaññī eko bahiddhā rūpāni passati odātāni odātavaṇṇāni odātanidassanāni odātanibhāsāni , ‘tāni abhibhuyya jānāmi passāmī’ti evaṃsaññī hoti. Idaṃ aṭṭhamaṃ abhibhāyatanaṃ.

\paragraph{339.} ‘‘Aṭṭha vimokkhā. Rūpī rūpāni passati. Ayaṃ paṭhamo vimokkho.

‘‘Ajjhattaṃ arūpasaññī bahiddhā rūpāni passati. Ayaṃ dutiyo vimokkho.

‘‘Subhanteva adhimutto hoti. Ayaṃ tatiyo vimokkho.

‘‘Sabbaso rūpasaññānaṃ samatikkamā paṭighasaññānaṃ atthaṅgamā nānattasaññānaṃ amanasikārā ‘ananto ākāso’ti ākāsānañcāyatanaṃ upasampajja viharati. Ayaṃ catuttho vimokkho.

‘‘Sabbaso ākāsānañcāyatanaṃ samatikkamma ‘anantaṃ viññāṇa’nti viññāṇañcāyatanaṃ upasampajja viharati. Ayaṃ pañcamo vimokkho.

‘‘Sabbaso viññāṇañcāyatanaṃ samatikkamma ‘natthi kiñcī’ti ākiñcaññāyatanaṃ upasampajja viharati. Ayaṃ chaṭṭho vimokkho.

‘‘Sabbaso ākiñcaññāyatanaṃ samatikkamma nevasaññānāsaññāyatanaṃ upasampajja viharati. Ayaṃ sattamo vimokkho.

‘‘Sabbaso nevasaññānāsaññāyatanaṃ samatikkamma saññāvedayita nirodhaṃ upasampajja viharati. Ayaṃ aṭṭhamo vimokkho.

‘‘Ime kho, āvuso, tena bhagavatā jānatā passatā arahatā sammāsambuddhena aṭṭha dhammā sammadakkhātā; tattha sabbeheva saṅgāyitabbaṃ…pe… atthāya hitāya sukhāya devamanussānaṃ.

\subsubsection{Navakaṃ}

\paragraph{340.} ‘‘Atthi kho, āvuso, tena bhagavatā jānatā passatā arahatā sammāsambuddhena nava dhammā sammadakkhātā; tattha sabbeheva saṅgāyitabbaṃ…pe… atthāya hitāya sukhāya devamanussānaṃ. Katame nava?

‘‘Nava āghātavatthūni. ‘Anatthaṃ me acarī’ti āghātaṃ bandhati; ‘anatthaṃ me caratī’ti āghātaṃ bandhati; ‘anatthaṃ me carissatī’ti āghātaṃ bandhati; ‘piyassa me manāpassa anatthaṃ acarī’ti āghātaṃ bandhati…pe… anatthaṃ caratīti āghātaṃ bandhati…pe… anatthaṃ carissatīti āghātaṃ bandhati; ‘appiyassa me amanāpassa atthaṃ acarī’ti āghātaṃ bandhati…pe… atthaṃ caratīti āghātaṃ bandhati…pe… atthaṃ carissatīti āghātaṃ bandhati.

‘‘Nava āghātapaṭivinayā. ‘Anatthaṃ me acari\footnote{acarīti (syā. ka.) evaṃ ‘‘carati carissati’’ padesupi}, taṃ kutettha labbhā’ti āghātaṃ paṭivineti ; ‘anatthaṃ me carati, taṃ kutettha labbhā’ti āghātaṃ paṭivineti; ‘anatthaṃ me carissati, taṃ kutettha labbhā’ti āghātaṃ paṭivineti; ‘piyassa me manāpassa anatthaṃ acari…pe… anatthaṃ carati…pe… anatthaṃ carissati, taṃ kutettha labbhā’ti āghātaṃ paṭivineti; ‘appiyassa me amanāpassa atthaṃ acari…pe… atthaṃ carati…pe… atthaṃ carissati, taṃ kutettha labbhā’ti āghātaṃ paṭivineti.

\paragraph{341.} ‘‘Nava sattāvāsā. Santāvuso, sattā nānattakāyā nānattasaññino, seyyathāpi manussā ekacce ca devā ekacce ca vinipātikā. Ayaṃ paṭhamo sattāvāso.

‘‘Santāvuso, sattā nānattakāyā ekattasaññino, seyyathāpi devā brahmakāyikā paṭhamābhinibbattā. Ayaṃ dutiyo sattāvāso.

‘‘Santāvuso, sattā ekattakāyā nānattasaññino, seyyathāpi devā ābhassarā. Ayaṃ tatiyo sattāvāso.

‘‘Santāvuso , sattā ekattakāyā ekattasaññino, seyyathāpi devā subhakiṇhā. Ayaṃ catuttho sattāvāso.

‘‘Santāvuso, sattā asaññino appaṭisaṃvedino, seyyathāpi devā asaññasattā\footnote{asaññisattā (syā. kaṃ.)}. Ayaṃ pañcamo sattāvāso.

‘‘Santāvuso, sattā sabbaso rūpasaññānaṃ samatikkamā paṭighasaññānaṃ atthaṅgamā nānattasaññānaṃ amanasikārā ‘ananto ākāso’ti ākāsānañcāyatanūpagā. Ayaṃ chaṭṭho sattāvāso.

‘‘Santāvuso, sattā sabbaso ākāsānañcāyatanaṃ samatikkamma ‘anantaṃ viññāṇa’nti viññāṇañcāyatanūpagā. Ayaṃ sattamo sattāvāso.

‘‘Santāvuso, sattā sabbaso viññāṇañcāyatanaṃ samatikkamma ‘natthi kiñcī’ti ākiñcāññāyatanūpagā. Ayaṃ aṭṭhamo sattāvāso.

‘‘Santāvuso , sattā sabbaso ākiñcaññāyatanaṃ samatikkamma\footnote{samatikkamma santametaṃ paṇītametanti (syā. kaṃ.)} nevasaññānāsaññāyatanūpagā. Ayaṃ navamo sattāvāso.

\paragraph{342.} ‘‘Nava akkhaṇā asamayā brahmacariyavāsāya. Idhāvuso , tathāgato ca loke uppanno hoti arahaṃ sammāsambuddho, dhammo ca desiyati opasamiko parinibbāniko sambodhagāmī sugatappavedito. Ayañca puggalo nirayaṃ upapanno hoti. Ayaṃ paṭhamo akkhaṇo asamayo brahmacariyavāsāya.

‘‘Puna caparaṃ, āvuso, tathāgato ca loke uppanno hoti arahaṃ sammāsambuddho, dhammo ca desiyati opasamiko parinibbāniko sambodhagāmī sugatappavedito. Ayañca puggalo tiracchānayoniṃ upapanno hoti. Ayaṃ dutiyo akkhaṇo asamayo brahmacariyavāsāya.

‘‘Puna caparaṃ…pe… pettivisayaṃ upapanno hoti. Ayaṃ tatiyo akkhaṇo asamayo brahmacariyavāsāya.

‘‘Puna caparaṃ…pe… asurakāyaṃ upapanno hoti. Ayaṃ catuttho akkhaṇo asamayo brahmacariyavāsāya.

‘‘Puna caparaṃ…pe… aññataraṃ dīghāyukaṃ devanikāyaṃ upapanno hoti. Ayaṃ pañcamo akkhaṇo asamayo brahmacariyavāsāya.

‘‘Puna caparaṃ…pe… paccantimesu janapadesu paccājāto hoti milakkhesu\footnote{milakkhakesu (syā. kaṃ.) milakkhūsu (ka.)} aviññātāresu, yattha natthi gati bhikkhūnaṃ bhikkhunīnaṃ upāsakānaṃ upāsikānaṃ. Ayaṃ chaṭṭho akkhaṇo asamayo brahmacariyavāsāya.

‘‘Puna caparaṃ…pe… majjhimesu janapadesu paccājāto hoti. So ca hoti micchādiṭṭhiko viparītadassano – ‘natthi dinnaṃ, natthi yiṭṭhaṃ, natthi hutaṃ, natthi sukatadukkaṭānaṃ\footnote{sukaṭa dukkaṭānaṃ (sī. pī.)} kammānaṃ phalaṃ vipāko, natthi ayaṃ loko, natthi paro loko, natthi mātā, natthi pitā, natthi sattā opapātikā, natthi loke samaṇabrāhmaṇā sammaggatā sammāpaṭipannā ye imañca lokaṃ parañca lokaṃ sayaṃ abhiññā sacchikatvā pavedentī’ti. Ayaṃ sattamo akkhaṇo asamayo brahmacariyavāsāya.

‘‘Puna caparaṃ…pe… majjhimesu janapadesu paccājāto hoti. So ca hoti duppañño jaḷo eḷamūgo, nappaṭibalo subhāsitadubbhāsitānamatthamaññātuṃ. Ayaṃ aṭṭhamo akkhaṇo asamayo brahmacariyavāsāya.

‘‘Puna caparaṃ, āvuso, tathāgato ca loke na\footnote{katthaci nakāro na dissati} uppanno hoti arahaṃ sammāsambuddho, dhammo ca na desiyati opasamiko parinibbāniko sambodhagāmī sugatappavedito. Ayañca puggalo majjhimesu janapadesu paccājāto hoti, so ca hoti paññavā ajaḷo aneḷamūgo, paṭibalo subhāsita-dubbhāsitānamatthamaññātuṃ. Ayaṃ navamo akkhaṇo asamayo brahmacariyavāsāya.

\paragraph{343.} ‘‘Nava anupubbavihārā. Idhāvuso, bhikkhu vivicceva kāmehi vivicca akusalehi dhammehi savitakkaṃ savicāraṃ vivekajaṃ pītisukhaṃ paṭhamaṃ jhānaṃ upasampajja viharati. Vitakkavicārānaṃ vūpasamā…pe… dutiyaṃ jhānaṃ upasampajja viharati. Pītiyā ca virāgā…pe… tatiyaṃ jhānaṃ upasampajja viharati. Sukhassa ca pahānā…pe… catutthaṃ jhānaṃ upasampajja viharati. Sabbaso rūpasaññānaṃ samatikkamā…pe… ākāsānañcāyatanaṃ upasampajja viharati. Sabbaso ākāsānañcāyatanaṃ samatikkamma ‘anantaṃ viññāṇa’nti viññāṇañcāyatanaṃ upasampajja viharati. Sabbaso viññāṇañcāyatanaṃ samatikkamma ‘natthi kiñcī’ti ākiñcaññāyatanaṃ upasampajja viharati. Sabbaso ākiñcaññāyatanaṃ samatikkamma nevasaññānāsaññāyatanaṃ upasampajja viharati. Sabbaso nevasaññānāsaññāyatanaṃ samatikkamma saññāvedayitanirodhaṃ upasampajja viharati.

\paragraph{344.} ‘‘Nava anupubbanirodhā. Paṭhamaṃ jhānaṃ samāpannassa kāmasaññā niruddhā hoti. Dutiyaṃ jhānaṃ samāpannassa vitakkavicārā niruddhā honti. Tatiyaṃ jhānaṃ samāpannassa pīti niruddhā hoti. Catutthaṃ jhānaṃ samāpannassa assāsapassāssā niruddhā honti. Ākāsānañcāyatanaṃ samāpannassa rūpasaññā niruddhā hoti. Viññāṇañcāyatanaṃ samāpannassa ākāsānañcāyatanasaññā niruddhā hoti. Ākiñcaññāyatanaṃ samāpannassa viññāṇañcāyatanasaññā niruddhā hoti. Nevasaññānāsaññāyatanaṃ samāpannassa ākiñcaññāyatanasaññā niruddhā hoti. Saññāvedayitanirodhaṃ samāpannassa saññā ca vedanā ca niruddhā honti.

‘‘Ime kho, āvuso, tena bhagavatā jānatā passatā arahatā sammāsambuddhena nava dhammā sammadakkhātā. Tattha sabbeheva saṅgāyitabbaṃ…pe… atthāya hitāya sukhāya devamanussānaṃ.

\subsubsection{Dasakaṃ}

\paragraph{345.} ‘‘Atthi kho, āvuso, tena bhagavatā jānatā passatā arahatā sammāsambuddhena dasa dhammā sammadakkhātā. Tattha sabbeheva saṅgāyitabbaṃ…pe… atthāya hitāya sukhāya devamanussānaṃ. Katame dasa?

‘‘Dasa nāthakaraṇā dhammā. Idhāvuso, bhikkhu sīlavā hoti. Pātimokkhasaṃvarasaṃvuto viharati ācāragocarasampanno, aṇumattesu vajjesu bhayadassāvī samādāya sikkhati sikkhāpadesu. Yaṃpāvuso, bhikkhu sīlavā hoti, pātimokkhasaṃvarasaṃvuto viharati, ācāragocarasampanno, aṇumattesu vajjesu bhayadassāvī samādāya sikkhati sikkhāpadesu. Ayampi dhammo nāthakaraṇo.

‘‘Puna caparaṃ, āvuso, bhikkhu bahussuto hoti sutadharo sutasannicayo. Ye te dhammā ādikalyāṇā majjhekalyāṇā pariyosānakalyāṇā sātthā sabyañjanā\footnote{sātthaṃ sabyañjanaṃ (sī. syā. pī.)} kevalaparipuṇṇaṃ parisuddhaṃ brahmacariyaṃ abhivadanti, tathārūpāssa dhammā bahussutā honti\footnote{dhatā (ka. sī. syā. kaṃ.)} dhātā vacasā paricitā manasānupekkhitā diṭṭhiyā suppaṭividdhā, yaṃpāvuso, bhikkhu bahussuto hoti…pe… diṭṭhiyā suppaṭividdhā. Ayampi dhammo nāthakaraṇo.

‘‘Puna caparaṃ, āvuso, bhikkhu kalyāṇamitto hoti kalyāṇasahāyo kalyāṇasampavaṅko. Yaṃpāvuso, bhikkhu kalyāṇamitto hoti kalyāṇasahāyo kalyāṇasampavaṅko. Ayampi dhammo nāthakaraṇo.

‘‘Puna caparaṃ, āvuso, bhikkhu suvaco hoti sovacassakaraṇehi dhammehi samannāgato khamo padakkhiṇaggāhī anusāsaniṃ. Yaṃpāvuso, bhikkhu suvaco hoti…pe… padakkhiṇaggāhī anusāsaniṃ. Ayampi dhammo nāthakaraṇo.

‘‘Puna caparaṃ, āvuso, bhikkhu yāni tāni sabrahmacārīnaṃ uccāvacāni kiṃkaraṇīyāni, tattha dakkho hoti analaso tatrupāyāya vīmaṃsāya samannāgato, alaṃ kātuṃ alaṃ saṃvidhātuṃ. Yaṃpāvuso, bhikkhu yāni tāni sabrahmacārīnaṃ…pe… alaṃ saṃvidhātuṃ. Ayampi dhammo nāthakaraṇo.

‘‘Puna caparaṃ, āvuso, bhikkhu dhammakāmo hoti piyasamudāhāro, abhidhamme abhivinaye uḷārapāmojjo\footnote{uḷārapāmujjo (sī. pī.), oḷārapāmojjo (syā. kaṃ.)}. Yaṃpāvuso, bhikkhu dhammakāmo hoti…pe… uḷārapāmojjo\footnote{uḷārapāmujjo (sī. pī.), oḷārapāmojjo (syā. kaṃ.)}. Ayampi dhammo nāthakaraṇo.

‘‘Puna caparaṃ, āvuso , bhikkhu santuṭṭho hoti itarītarehi cīvarapiṇḍapātasenāsanagilānappaccayabhesajjaparikkhārehi . Yaṃpāvuso, bhikkhu santuṭṭho hoti…pe… parikkhārehi. Ayampi dhammo nāthakaraṇo.

‘‘Puna caparaṃ, āvuso, bhikkhu āraddhavīriyo viharati akusalānaṃ dhammānaṃ pahānāya kusalānaṃ dhammānaṃ upasampadāya, thāmavā daḷhaparakkamo anikkhittadhuro kusalesu dhammesu. Yaṃpāvuso, bhikkhu āraddhavīriyo viharati…pe… anikkhittadhuro kusalesu dhammesu. Ayampi dhammo nāthakaraṇo.

‘‘Puna caparaṃ, āvuso, bhikkhu satimā hoti paramena satinepakkena samannāgato cirakatampi cirabhāsitampi saritā anussaritā. Yaṃpāvuso, bhikkhu satimā hoti…pe… saritā anussaritā. Ayampi dhammo nāthakaraṇo.

‘‘Puna caparaṃ, āvuso, bhikkhu paññavā hoti, udayatthagāminiyā paññāya samannāgato ariyāya nibbedhikāya sammādukkhakkhayagāminiyā. Yaṃpāvuso, bhikkhu paññavā hoti…pe… sammādukkhakkhayagāminiyā. Ayampi dhammo nāthakaraṇo.

\paragraph{346.} Dasa kasiṇāyatanāni. Pathavīkasiṇameko sañjānāti, uddhaṃ adho tiriyaṃ advayaṃ appamāṇaṃ. Āpokasiṇameko sañjānāti…pe… tejokasiṇameko sañjānāti… vāyokasiṇameko sañjānāti… nīlakasiṇameko sañjānāti… pītakasiṇameko sañjānāti… lohitakasiṇameko sañjānāti… odātakasiṇameko sañjānāti… ākāsakasiṇameko sañjānāti… viññāṇakasiṇameko sañjānāti, uddhaṃ adho tiriyaṃ advayaṃ appamāṇaṃ.

\paragraph{347.} ‘‘Dasa akusalakammapathā – pāṇātipāto, adinnādānaṃ, kāmesumicchācāro, musāvādo, pisuṇā vācā, pharusā vācā, samphappalāpo, abhijjhā, byāpādo, micchādiṭṭhi.

‘‘Dasa kusalakammapathā – pāṇātipātā veramaṇī, adinnādānā veramaṇī, kāmesumicchācārā veramaṇī, musāvādā veramaṇī, pisuṇāya vācāya veramaṇī, pharusāya vācāya veramaṇī, samphappalāpā veramaṇī, anabhijjhā, abyāpādo, sammādiṭṭhi.

\paragraph{348.} ‘‘Dasa ariyavāsā. Idhāvuso, bhikkhu pañcaṅgavippahīno hoti, chaḷaṅgasamannāgato, ekārakkho, caturāpasseno, paṇunnapaccekasacco, samavayasaṭṭhesano, anāvilasaṅkappo, passaddhakāyasaṅkhāro, suvimuttacitto, suvimuttapañño.

‘‘Kathañcāvuso, bhikkhu pañcaṅgavippahīno hoti? Idhāvuso, bhikkhuno kāmacchando pahīno hoti, byāpādo pahīno hoti, thinamiddhaṃ pahīnaṃ hoti, uddhaccakukuccaṃ pahīnaṃ hoti, vicikicchā pahīnā hoti. Evaṃ kho, āvuso, bhikkhu pañcaṅgavippahīno hoti.

‘‘Kathañcāvuso, bhikkhu chaḷaṅgasamannāgato hoti? Idhāvuso, bhikkhu cakkhunā rūpaṃ disvā neva sumano hoti na dummano, upekkhako viharati sato sampajāno. Sotena saddaṃ sutvā…pe… manasā dhammaṃ viññāya neva sumano hoti na dummano, upekkhako viharati sato sampajāno. Evaṃ kho, āvuso, bhikkhu chaḷaṅgasamannāgato hoti.

‘‘Kathañcāvuso , bhikkhu ekārakkho hoti? Idhāvuso, bhikkhu satārakkhena cetasā samannāgato hoti. Evaṃ kho, āvuso, bhikkhu ekārakkho hoti .

‘‘Kathañcāvuso, bhikkhu caturāpasseno hoti? Idhāvuso, bhikkhu saṅkhāyekaṃ paṭisevati, saṅkhāyekaṃ adhivāseti, saṅkhāyekaṃ parivajjeti, saṅkhāyekaṃ vinodeti. Evaṃ kho, āvuso, bhikkhu caturāpasseno hoti.

‘‘Kathañcāvuso, bhikkhu paṇunnapaccekasacco hoti? Idhāvuso, bhikkhuno yāni tāni puthusamaṇabrāhmaṇānaṃ puthupaccekasaccāni, sabbāni tāni nunnāni honti paṇunnāni cattāni vantāni muttāni pahīnāni paṭinissaṭṭhāni. Evaṃ kho, āvuso, bhikkhu paṇunnapaccekasacco hoti.

‘‘Kathañcāvuso , bhikkhu samavayasaṭṭhesano hoti? Idhāvuso, bhikkhuno kāmesanā pahīnā hoti, bhavesanā pahīnā hoti, brahmacariyesanā paṭippassaddhā. Evaṃ kho, āvuso, bhikkhu samavayasaṭṭhesano hoti.

‘‘Kathañcāvuso, bhikkhu anāvilasaṅkappo hoti? Idhāvuso, bhikkhuno kāmasaṅkappo pahīno hoti, byāpādasaṅkappo pahīno hoti, vihiṃsāsaṅkappo pahīno hoti. Evaṃ kho, āvuso, bhikkhu anāvilasaṅkappo hoti.

‘‘Kathañcāvuso, bhikkhu passaddhakāyasaṅkhāro hoti ? Idhāvuso, bhikkhu sukhassa ca pahānā dukkhassa ca pahānā pubbeva somanassadomanassānaṃ atthaṅgamā adukkhamasukhaṃ upekkhāsatipārisuddhiṃ catutthaṃ jhānaṃ upasampajja viharati. Evaṃ kho, āvuso, bhikkhu passaddhakāyasaṅkhāro hoti.

‘‘Kathañcāvuso, bhikkhu suvimuttacitto hoti? Idhāvuso, bhikkhuno rāgā cittaṃ vimuttaṃ hoti, dosā cittaṃ vimuttaṃ hoti, mohā cittaṃ vimuttaṃ hoti. Evaṃ kho, āvuso, bhikkhu suvimuttacitto hoti.

‘‘Kathañcāvuso, bhikkhu suvimuttapañño hoti? Idhāvuso, bhikkhu ‘rāgo me pahīno ucchinnamūlo tālāvatthukato anabhāvaṃkato āyatiṃ anuppādadhammo’ti pajānāti. ‘Doso me pahīno ucchinnamūlo tālāvatthukato anabhāvaṃkato āyatiṃ anuppādadhammo’ti pajānāti. ‘Moho me pahīno ucchinnamūlo tālāvatthukato anabhāvaṃkato āyatiṃ anuppādadhammo’ti pajānāti. Evaṃ kho, āvuso, bhikkhu suvimuttapañño hoti.

‘‘Dasa asekkhā dhammā – asekkhā sammādiṭṭhi, asekkho sammāsaṅkappo, asekkhā sammāvācā, asekkho sammākammanto, asekkho sammāājīvo, asekkho sammāvāyāmo, asekkhā sammāsati, asekkho sammāsamādhi, asekkhaṃ sammāñāṇaṃ, asekkhā sammāvimutti.

‘‘Ime kho, āvuso, tena bhagavatā jānatā passatā arahatā sammāsambuddhena dasa dhammā sammadakkhātā. Tattha sabbeheva saṅgāyitabbaṃ na vivaditabbaṃ, yathayidaṃ brahmacariyaṃ addhaniyaṃ assa ciraṭṭhitikaṃ, tadassa bahujanahitāya bahujanasukhāya lokānukampāya atthāya hitāya sukhāya devamanussāna’’nti.

\paragraph{349.} Atha kho bhagavā uṭṭhahitvā āyasmantaṃ sāriputtaṃ āmantesi – ‘sādhu sādhu, sāriputta, sādhu kho tvaṃ, sāriputta, bhikkhūnaṃ saṅgītipariyāyaṃ abhāsī’ti. Idamavocāyasmā sāriputto, samanuñño satthā ahosi. Attamanā te bhikkhū āyasmato sāriputtassa bhāsitaṃ abhinandunti.

\xsectionEnd{Saṅgītisuttaṃ niṭṭhitaṃ dasamaṃ.}
