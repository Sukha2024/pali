\section{Lakkhaṇasuttaṃ}

\subsubsection{Dvattiṃsamahāpurisalakkhaṇāni}

\paragraph{198.} Evaṃ me sutaṃ – ekaṃ samayaṃ bhagavā sāvatthiyaṃ viharati jetavane anāthapiṇḍikassa ārāme. Tatra kho bhagavā bhikkhū āmantesi – ‘‘bhikkhavo’’ti. ‘‘Bhaddante’’ti\footnote{bhadanteti (sī. syā. pī.)} te bhikkhū bhagavato paccassosuṃ. Bhagavā etadavoca –

\paragraph{199.} ‘‘Dvattiṃsimāni, bhikkhave, mahāpurisassa mahāpurisalakkhaṇāni, yehi samannāgatassa mahāpurisassa dveva gatiyo bhavanti anaññā. Sace agāraṃ ajjhāvasati, rājā hoti cakkavattī dhammiko dhammarājā cāturanto vijitāvī janapadatthāvariyappatto sattaratanasamannāgato. Tassimāni satta ratanāni bhavanti; seyyathidaṃ, cakkaratanaṃ hatthiratanaṃ assaratanaṃ maṇiratanaṃ itthiratanaṃ gahapatiratanaṃ pariṇāyakaratanameva sattamaṃ. Parosahassaṃ kho panassa puttā bhavanti sūrā vīraṅgarūpā parasenappamaddanā. So imaṃ pathaviṃ sāgarapariyantaṃ adaṇḍena asatthena dhammena abhivijiya ajjhāvasati. Sace kho pana agārasmā anagāriyaṃ pabbajati, arahaṃ hoti sammāsambuddho loke vivaṭṭacchado\footnote{vivaṭacchado (syā. ka.), vivattacchado (sī. pī.)}.

\paragraph{200.} ‘‘Katamāni ca tāni, bhikkhave, dvattiṃsa mahāpurisassa mahāpurisalakkhaṇāni, yehi samannāgatassa mahāpurisassa dveva gatiyo bhavanti anaññā? Sace agāraṃ ajjhāvasati, rājā hoti cakkavattī…pe… sace kho pana agārasmā anagāriyaṃ pabbajati, arahaṃ hoti sammāsambuddho loke vivaṭṭacchado.

‘‘Idha, bhikkhave, mahāpuriso suppatiṭṭhitapādo hoti. Yampi, bhikkhave, mahāpuriso suppatiṭṭhitapādo hoti, idampi, bhikkhave, mahāpurisassa mahāpurisalakkhaṇaṃ bhavati.

‘‘Puna caparaṃ, bhikkhave, mahāpurisassa heṭṭhāpādatalesu cakkāni jātāni honti sahassārāni sanemikāni sanābhikāni sabbākāraparipūrāni\footnote{sabbākāraparipūrāni suvibhattantarāni (sī. pī.)}. Yampi , bhikkhave, mahāpurisassa heṭṭhāpādatalesu cakkāni jātāni honti sahassārāni sanemikāni sanābhikāni sabbākāraparipūrāni, idampi, bhikkhave, mahāpurisassa mahāpurisalakkhaṇaṃ bhavati.

‘‘Puna caparaṃ, bhikkhave, mahāpuriso āyatapaṇhi hoti…pe… dīghaṅguli hoti… mudutalunahatthapādo hoti… jālahatthapādo hoti… ussaṅkhapādo hoti… eṇijaṅgho hoti… ṭhitakova anonamanto ubhohi pāṇitalehi jaṇṇukāni parimasati parimajjati… kosohitavatthaguyho hoti… suvaṇṇavaṇṇo hoti kañcanasannibhattaco… sukhumacchavi hoti, sukhumattā chaviyā rajojallaṃ kāye na upalimpati… ekekalomo hoti, ekekāni lomāni lomakūpesu jātāni… uddhaggalomo hoti, uddhaggāni lomāni jātāni nīlāni añjanavaṇṇāni kuṇḍalāvaṭṭāni\footnote{kuṇḍalāvattāni (bahūsu)} dakkhiṇāvaṭṭakajātāni\footnote{dakkhiṇāvattakajātāni (sī. syā. pī.)} … brahmujugatto hoti… sattussado hoti… sīhapubbaddhakāyo hoti… citantaraṃso\footnote{pitantaraṃso (syā.)} hoti… nigrodhaparimaṇḍalo hoti, yāvatakvassa kāyo tāvatakvassa byāmo yāvatakvassa byāmo tāvatakvassa kāyo… samavaṭṭakkhandho hoti… rasaggasaggī hoti… sīhahanu hoti… cattālīsadanto hoti … samadanto hoti… aviraḷadanto hoti… susukkadāṭho hoti… pahūtajivho hoti… brahmassaro hoti karavīkabhāṇī… abhinīlanetto hoti… gopakhumo hoti… uṇṇā bhamukantare jātā hoti, odātā mudutūlasannibhā. Yampi, bhikkhave, mahāpurisassa uṇṇā bhamukantare jātā hoti, odātā mudutūlasannibhā, idampi, bhikkhave, mahāpurisassa mahāpurisalakkhaṇaṃ bhavati.

‘‘Puna caparaṃ, bhikkhave, mahāpuriso uṇhīsasīso hoti. Yampi, bhikkhave, mahāpuriso uṇhīsasīso hoti, idampi, bhikkhave, mahāpurisassa mahāpurisalakkhaṇaṃ bhavati.

‘‘Imāni kho tāni, bhikkhave, dvattiṃsa mahāpurisassa mahāpurisalakkhaṇāni, yehi samannāgatassa mahāpurisassa dveva gatiyo bhavanti anaññā. Sace agāraṃ ajjhāvasati, rājā hoti cakkavattī…pe… sace kho pana agārasmā anagāriyaṃ pabbajati, arahaṃ hoti sammāsambuddho loke vivaṭṭacchado.

‘‘Imāni kho, bhikkhave, dvattiṃsa mahāpurisassa mahāpurisalakkhaṇāni bāhirakāpi isayo dhārenti, no ca kho te jānanti – ‘imassa kammassa kaṭattā idaṃ lakkhaṇaṃ paṭilabhatī’ti.

\subsubsection{(1) Suppatiṭṭhitapādatālakkhaṇaṃ}

\paragraph{201.} ‘‘Yampi, bhikkhave, tathāgato purimaṃ jātiṃ purimaṃ bhavaṃ purimaṃ niketaṃ pubbe manussabhūto samāno daḷhasamādāno ahosi kusalesu dhammesu, avatthitasamādāno kāyasucarite vacīsucarite manosucarite dānasaṃvibhāge sīlasamādāne uposathupavāse matteyyatāya petteyyatāya sāmaññatāya brahmaññatāya kule jeṭṭhāpacāyitāya aññataraññataresu ca adhikusalesu dhammesu . So tassa kammassa kaṭattā upacitattā ussannattā vipulattā kāyassa bhedā paraṃ maraṇā sugatiṃ saggaṃ lokaṃ upapajjati. So tattha aññe deve dasahi ṭhānehi adhiggaṇhāti dibbena āyunā dibbena vaṇṇena dibbena sukhena dibbena yasena dibbena ādhipateyyena dibbehi rūpehi dibbehi saddehi dibbehi gandhehi dibbehi rasehi dibbehi phoṭṭhabbehi. So tato cuto itthattaṃ āgato samāno imaṃ mahāpurisalakkhaṇaṃ paṭilabhati. Suppatiṭṭhitapādo hoti. Samaṃ pādaṃ bhūmiyaṃ nikkhipati, samaṃ uddharati, samaṃ sabbāvantehi pādatalehi bhūmiṃ phusati.

\paragraph{202.} ‘‘So tena lakkhaṇena samannāgato sace agāraṃ ajjhāvasati, rājā hoti cakkavattī dhammiko dhammarājā cāturanto vijitāvī janapadatthāvariyappatto sattaratanasamannāgato. Tassimāni satta ratanāni bhavanti; seyyathidaṃ, cakkaratanaṃ hatthiratanaṃ assaratanaṃ maṇiratanaṃ itthiratanaṃ gahapatiratanaṃ pariṇāyakaratanameva sattamaṃ. Parosahassaṃ kho panassa puttā bhavanti sūrā vīraṅgarūpā parasenappamaddanā. So imaṃ pathaviṃ sāgarapariyantaṃ akhilamanimittamakaṇṭakaṃ iddhaṃ phītaṃ khemaṃ sivaṃ nirabbudaṃ adaṇḍena asatthena dhammena abhivijiya ajjhāvasati . Rājā samāno kiṃ labhati? Akkhambhiyo\footnote{avikkhambhiyo (sī. pī.)} hoti kenaci manussabhūtena paccatthikena paccāmittena. Rājā samāno idaṃ labhati. ‘‘Sace kho pana agārasmā anagāriyaṃ pabbajati, arahaṃ hoti sammāsambuddho loke vivaṭṭacchado. Buddho samāno kiṃ labhati? Akkhambhiyo hoti abbhantarehi vā bāhirehi vā paccatthikehi paccāmittehi rāgena vā dosena vā mohena vā samaṇena vā brāhmaṇena vā devena vā mārena vā brahmunā vā kenaci vā lokasmiṃ. Buddho samāno idaṃ labhati’’. Etamatthaṃ bhagavā avoca.

\paragraph{203.} Tatthetaṃ vuccati –

‘‘Sacce ca dhamme ca dame ca saṃyame,

Soceyyasīlālayuposathesu ca;

Dāne ahiṃsāya asāhase rato,

Daḷhaṃ samādāya samattamācari\footnote{samantamācari (syā. ka.)}.

‘‘So tena kammena divaṃ samakkami\footnote{apakkami (syā. ka.)},

Sukhañca khiḍḍāratiyo ca anvabhi\footnote{aṃnvabhi (ṭīkā)};

Tato cavitvā punarāgato idha,

Samehi pādehi phusī vasundharaṃ.

‘‘Byākaṃsu veyyañjanikā samāgatā,

Samappatiṭṭhassa na hoti khambhanā;

Gihissa vā pabbajitassa vā puna\footnote{pana (syā.)},

Taṃ lakkhaṇaṃ bhavati tadatthajotakaṃ.

‘‘Akkhambhiyo hoti agāramāvasaṃ,

Parābhibhū sattubhi nappamaddano;

Manussabhūtenidha hoti kenaci,

Akkhambhiyo tassa phalena kammuno.

‘‘Sace ca pabbajjamupeti tādiso,

Nekkhammachandābhirato vicakkhaṇo;

Aggo na so gacchati jātu khambhanaṃ,

Naruttamo esa hi tassa dhammatā’’ti.

\subsubsection{(2) Pādatalacakkalakkhaṇaṃ}

\paragraph{204.} ‘‘Yampi, bhikkhave, tathāgato purimaṃ jātiṃ purimaṃ bhavaṃ purimaṃ niketaṃ pubbe manussabhūto samāno bahujanassa sukhāvaho ahosi, ubbegauttāsabhayaṃ apanuditā, dhammikañca rakkhāvaraṇaguttiṃ saṃvidhātā, saparivārañca dānaṃ adāsi. So tassa kammassa kaṭattā upacitattā ussannattā vipulattā kāyassa bhedā paraṃ maraṇā sugatiṃ saggaṃ lokaṃ upapajjati…pe… so tato cuto itthattaṃ āgato samāno imaṃ mahāpurisalakkhaṇaṃ paṭilabhati. Heṭṭhāpādatalesu cakkāni jātāni honti sahassārāni sanemikāni sanābhikāni sabbākāraparipūrāni suvibhattantarāni.

‘‘So tena lakkhaṇena samannāgato sace agāraṃ ajjhāvasati, rājā hoti cakkavattī…pe… rājā samāno kiṃ labhati? Mahāparivāro hoti; mahāssa honti parivārā brāhmaṇagahapatikā negamajānapadā gaṇakamahāmattā anīkaṭṭhā dovārikā amaccā pārisajjā rājāno bhogiyā kumārā. Rājā samāno idaṃ labhati. Sace kho pana agārasmā anagāriyaṃ pabbajati, arahaṃ hoti sammāsambuddho loke vivaṭṭacchado. Buddho samāno kiṃ labhati? Mahāparivāro hoti; mahāssa honti parivārā bhikkhū bhikkhuniyo upāsakā upāsikāyo devā manussā asurā nāgā gandhabbā. Buddho samāno idaṃ labhati’’. Etamatthaṃ bhagavā avoca.

\paragraph{205.} Tatthetaṃ vuccati –

‘‘Pure puratthā purimāsu jātisu,

Manussabhūto bahunaṃ sukhāvaho;

Ubbhegauttāsabhayāpanūdano,

Guttīsu rakkhāvaraṇesu ussuko.

‘‘So tena kammena divaṃ samakkami,

Sukhañca khiḍḍāratiyo ca anvabhi;

Tato cavitvā punarāgato idha,

Cakkāni pādesu duvesu vindati.

‘‘Samantanemīni sahassarāni ca,

Byākaṃsu veyyañjanikā samāgatā;

Disvā kumāraṃ satapuññalakkhaṇaṃ,

Parivāravā hessati sattumaddano.

Tathā hī cakkāni samantanemini,

Sace na pabbajjamupeti tādiso;

Vatteti cakkaṃ pathaviṃ pasāsati,

Tassānuyantādha\footnote{tassānuyuttā idha (sī. pī.), tassānuyantā idha (syā. ka.)} bhavanti khattiyā.

‘‘Mahāyasaṃ saṃparivārayanti naṃ,

Sace ca pabbajjamupeti tādiso;

Nekkhammachandābhirato vicakkhaṇo,

Devāmanussāsurasakkarakkhasā\footnote{sattarakkhasā (ka.) sī. syāaṭṭhakathā oloketabbā}.

‘‘Gandhabbanāgā vihagā catuppadā,

Anuttaraṃ devamanussapūjitaṃ;

Mahāyasaṃ saṃparivārayanti na’’nti.

\subsubsection{(3-5) Āyatapaṇhitāditilakkhaṇaṃ}

\paragraph{206.} ‘‘Yampi, bhikkhave, tathāgato purimaṃ jātiṃ purimaṃ bhavaṃ purimaṃ niketaṃ pubbe manussabhūto samāno pāṇātipātaṃ pahāya pāṇātipātā paṭivirato ahosi nihitadaṇḍo nihitasattho lajjī dayāpanno, sabbapāṇabhūtahitānukampī vihāsi. So tassa kammassa kaṭattā upacitattā ussannattā vipulattā…pe… so tato cuto itthattaṃ āgato samāno imāni tīṇi mahāpurisalakkhaṇāni paṭilabhati. Āyatapaṇhi ca hoti, dīghaṅguli ca brahmujugatto ca.

‘‘So tehi lakkhaṇehi samannāgato sace agāraṃ ajjhāvasati, rājā hoti cakkavattī…pe… rājā samāno kiṃ labhati? Dīghāyuko hoti ciraṭṭhitiko, dīghamāyuṃ pāleti, na sakkā hoti antarā jīvitā voropetuṃ kenaci manussabhūtena paccatthikena paccāmittena . Rājā samāno idaṃ labhati… buddho samāno kiṃ labhati? Dīghāyuko hoti ciraṭṭhitiko, dīghamāyuṃ pāleti, na sakkā hoti antarā jīvitā voropetuṃ paccatthikehi paccāmittehi samaṇena vā brāhmaṇena vā devena vā mārena vā brahmunā vā kenaci vā lokasmiṃ. Buddho samāno idaṃ labhati’’. Etamatthaṃ bhagavā avoca.

\paragraph{207.} Tatthetaṃ vuccati –

‘‘Māraṇavadhabhayattano\footnote{maraṇavadhabhayattano (sī. pī. ka.), maraṇavadhabhayamattano (syā.)} viditvā,

Paṭivirato paraṃ māraṇāyahosi;

Tena sucaritena saggamagamā\footnote{tena so sucaritena saggamagamāsi (syā.)},

Sukataphalavipākamanubhosi.

‘‘Caviya punaridhāgato samāno,

Paṭilabhati idha tīṇi lakkhaṇāni;

Bhavati vipuladīghapāsaṇhiko,

Brahmāva suju subho sujātagatto.

‘‘Subhujo susu susaṇṭhito sujāto,

Mudutalunaṅguliyassa honti;

Dīghā tībhi purisavaraggalakkhaṇehi,

Cirayapanāya\footnote{cirayāpanāya (syā.)} kumāramādisanti.

‘‘Bhavati yadi gihī ciraṃ yapeti,

Cirataraṃ pabbajati yadi tato hi;

Yāpayati ca vasiddhibhāvanāya,

Iti dīghāyukatāya taṃ nimitta’’nti.

\subsubsection{(6) Sattussadatālakkhaṇaṃ}

\paragraph{208.} ‘‘Yampi , bhikkhave, tathāgato purimaṃ jātiṃ purimaṃ bhavaṃ purimaṃ niketaṃ pubbe manussabhūto samāno dātā ahosi paṇītānaṃ rasitānaṃ khādanīyānaṃ bhojanīyānaṃ sāyanīyānaṃ lehanīyānaṃ pānānaṃ. So tassa kammassa kaṭattā…pe… so tato cuto itthattaṃ āgato samāno imaṃ mahāpurisalakkhaṇaṃ paṭilabhati, sattussado hoti, sattassa ussadā honti; ubhosu hatthesu ussadā honti, ubhosu pādesu ussadā honti, ubhosu aṃsakūṭesu ussadā honti, khandhe ussado hoti.

‘‘So tena lakkhaṇena samannāgato sace agāraṃ ajjhāvasati, rājā hoti cakkavattī…pe… rājā samāno kiṃ labhati? Lābhī hoti paṇītānaṃ rasitānaṃ khādanīyānaṃ bhojanīyānaṃ sāyanīyānaṃ lehanīyānaṃ pānānaṃ. Rājā samāno idaṃ labhati… buddho samāno kiṃ labhati? Lābhī hoti paṇītānaṃ rasitānaṃ khādanīyānaṃ bhojanīyānaṃ sāyanīyānaṃ lehanīyānaṃ pānānaṃ. Buddho samāno idaṃ labhati’’. Etamatthaṃ bhagavā avoca.

\paragraph{209.} Tatthetaṃ vuccati –

‘‘Khajjabhojjamatha leyya sāyiyaṃ,

Uttamaggarasadāyako ahu;

Tena so sucaritena kammunā,

Nandane ciramabhippamodati.

‘‘Satta cussade idhādhigacchati,

Hatthapādamudutañca vindati;

Āhu byañjananimittakovidā,

Khajjabhojjarasalābhitāya naṃ.

‘‘Yaṃ gihissapi\footnote{na taṃ gihissāpi (syā.)} tadatthajotakaṃ,

Pabbajjampi ca tadādhigacchati;

Khajjabhojjarasalābhiruttamaṃ,

Āhu sabbagihibandhanacchida’’nti.

\subsubsection{(7-8) Karacaraṇamudujālatālakkhaṇāni}

\paragraph{210.} ‘‘Yampi , bhikkhave, tathāgato purimaṃ jātiṃ purimaṃ bhavaṃ purimaṃ niketaṃ pubbe manussabhūto samāno catūhi saṅgahavatthūhi janaṃ saṅgāhako ahosi – dānena peyyavajjena\footnote{piyavācena (syā. ka.)} atthacariyāya samānattatāya. So tassa kammassa kaṭattā…pe… so tato cuto itthattaṃ āgato samāno imāni dve mahāpurisalakkhaṇāni paṭilabhati. Mudutalunahatthapādo ca hoti jālahatthapādo ca.

‘‘So tehi lakkhaṇehi samannāgato sace agāraṃ ajjhāvasati, rājā hoti cakkavattī…pe… rājā samāno kiṃ labhati? Susaṅgahitaparijano hoti, susaṅgahitāssa honti brāhmaṇagahapatikā negamajānapadā gaṇakamahāmattā anīkaṭṭhā dovārikā amaccā pārisajjā rājāno bhogiyā kumārā. Rājā samāno idaṃ labhati… buddho samāno kiṃ labhati? Susaṅgahitaparijano hoti, susaṅgahitāssa honti bhikkhū bhikkhuniyo upāsakā upāsikāyo devā manussā asurā nāgā gandhabbā. Buddho samāno idaṃ labhati’’. Etamatthaṃ bhagavā avoca.

\paragraph{211.} Tatthetaṃ vuccati –

‘‘Dānampi catthacariyatañca\footnote{dānampi ca atthacariyatampi ca (sī. pī.)},

Piyavāditañca samānattatañca\footnote{piyavadatañca samānachandatañca (sī. pī.)};

Kariyacariyasusaṅgahaṃ bahūnaṃ,

Anavamatena guṇena yāti saggaṃ.

‘‘Caviya punaridhāgato samāno,

Karacaraṇamudutañca jālino ca;

Atirucirasuvaggudassaneyyaṃ,

Paṭilabhati daharo susu kumāro.

‘‘Bhavati parijanassavo vidheyyo,

Mahimaṃ āvasito susaṅgahito;

Piyavadū hitasukhataṃ jigīsamāno\footnote{jigiṃ samāno (sī. syā. pī.)},

Abhirucitāni guṇāni ācarati.

‘‘Yadi ca jahati sabbakāmabhogaṃ,

Kathayati dhammakathaṃ jino janassa;

Vacanapaṭikarassābhippasannā ,

Sutvāna dhammānudhammamācarantī’’ti.

\subsubsection{(9-10) Ussaṅkhapādauddhaggalomatālakkhaṇāni}

\paragraph{212.} ‘‘Yampi, bhikkhave, tathāgato purimaṃ jātiṃ purimaṃ bhavaṃ purimaṃ niketaṃ pubbe manussabhūto samāno\footnote{samāno bahuno janassa (sī. pī.)} atthūpasaṃhitaṃ dhammūpasaṃhitaṃ vācaṃ bhāsitā ahosi, bahujanaṃ nidaṃsesi, pāṇīnaṃ hitasukhāvaho dhammayāgī. So tassa kammassa kaṭattā…pe… so tato cuto itthattaṃ āgato samāno imāni dve mahāpurisalakkhaṇāni paṭilabhati. Ussaṅkhapādo ca hoti, uddhaggalomo ca.

‘‘So tehi lakkhaṇehi samannāgato, sace agāraṃ ajjhāvasati, rājā hoti cakkavattī…pe… rājā samāno kiṃ labhati? Aggo ca hoti seṭṭho ca pāmokkho ca uttamo ca pavaro ca kāmabhogīnaṃ. Rājā samāno idaṃ labhati… buddho samāno kiṃ labhati? Aggo ca hoti seṭṭho ca pāmokkho ca uttamo ca pavaro ca sabbasattānaṃ. Buddho samāno idaṃ labhati’’. Etamatthaṃ bhagavā avoca.

\paragraph{213.} Tatthetaṃ vuccati –

‘‘Atthadhammasahitaṃ\footnote{atthadhammasaṃhitaṃ (ka. sī. pī.), atthadhammupasaṃhitaṃ (ka.)} pure giraṃ,

Erayaṃ bahujanaṃ nidaṃsayi;

Pāṇinaṃ hitasukhāvaho ahu,

Dhammayāgamayajī\footnote{dhammayāgaṃ assaji (ka.)} amaccharī.

‘‘Tena so sucaritena kammunā,

Suggatiṃ vajati tattha modati;

Lakkhaṇāni ca duve idhāgato,

Uttamappamukhatāya\footnote{uttamasukhatāya (syā.), uttamapamukkhatāya (ka.)} vindati.

‘‘Ubbhamuppatitalomavā saso,

Pādagaṇṭhirahu sādhusaṇṭhitā;

Maṃsalohitācitā tacotthatā,

Uparicaraṇasobhanā\footnote{uparijānusobhanā (syā.), upari ca pana sobhanā (sī. pī.)} ahu.

‘‘Gehamāvasati ce tathāvidho,

Aggataṃ vajati kāmabhoginaṃ;

Tena uttaritaro na vijjati,

Jambudīpamabhibhuyya iriyati.

‘‘Pabbajampi ca anomanikkamo,

Aggataṃ vajati sabbapāṇinaṃ;

Tena uttaritaro na vijjati,

Sabbalokamabhibhuyya viharatī’’ti.

\subsubsection{(11) Eṇijaṅghalakkhaṇaṃ}

\paragraph{214.} ‘‘Yampi, bhikkhave, tathāgato purimaṃ jātiṃ purimaṃ bhavaṃ purimaṃ niketaṃ pubbe manussabhūto samāno sakkaccaṃ vācetā ahosi sippaṃ vā vijjaṃ vā caraṇaṃ vā kammaṃ vā – ‘kiṃ time khippaṃ vijāneyyuṃ, khippaṃ paṭipajjeyyuṃ, na ciraṃ kilisseyyu’’nti. So tassa kammassa kaṭattā…pe… so tato cuto itthattaṃ āgato samāno imaṃ mahāpurisalakkhaṇaṃ paṭilabhati. Eṇijaṅgho hoti.

‘‘So tena lakkhaṇena samannāgato sace agāraṃ ajjhāvasati, rājā hoti cakkavattī…pe… rājā samāno kiṃ labhati? Yāni tāni rājārahāni rājaṅgāni rājūpabhogāni rājānucchavikāni tāni khippaṃ paṭilabhati. Rājā samāno idaṃ labhati… buddho samāno kiṃ labhati? Yāni tāni samaṇārahāni samaṇaṅgāni samaṇūpabhogāni samaṇānucchavikāni, tāni khippaṃ paṭilabhati. Buddho samāno idaṃ labhati’’. Etamatthaṃ bhagavā avoca.

\paragraph{215.} Tatthetaṃ vuccati –

‘‘Sippesu vijjācaraṇesu kammesu\footnote{kammasu (sī. pī.)},

Kathaṃ vijāneyyuṃ\footnote{vijāneyya (sī. pī.), vijāneyyu (syā.)} lahunti icchati;

Yadūpaghātāya na hoti kassaci,

Vāceti khippaṃ na ciraṃ kilissati.

‘‘Taṃ kammaṃ katvā kusalaṃ sukhudrayaṃ\footnote{sukhindriyaṃ (ka.)},

Jaṅghā manuññā labhate susaṇṭhitā;

Vaṭṭā sujātā anupubbamuggatā,

Uddhaggalomā sukhumattacotthatā.

‘‘Eṇeyyajaṅghoti tamāhu puggalaṃ,

Sampattiyā khippamidhāhu\footnote{khippamidāhu (?)} lakkhaṇaṃ;

Gehānulomāni yadābhikaṅkhati,

Apabbajaṃ khippamidhādhigacchati\footnote{khippamidādhigacchati (?)}.

‘‘Sace ca pabbajjamupeti tādiso,

Nekkhammachandābhirato vicakkhaṇo;

Anucchavikassa yadānulomikaṃ,

Taṃ vindati khippamanomavikkamo\footnote{nikkamo (sī. syā. pī.)}’’ti.

\subsubsection{(12) Sukhumacchavilakkhaṇaṃ}

\paragraph{216.} ‘‘Yampi, bhikkhave, tathāgato purimaṃ jātiṃ purimaṃ bhavaṃ purimaṃ niketaṃ pubbe manussabhūto samāno samaṇaṃ vā brāhmaṇaṃ vā upasaṅkamitvā paripucchitā ahosi – ‘‘kiṃ, bhante, kusalaṃ, kiṃ akusalaṃ, kiṃ sāvajjaṃ, kiṃ anavajjaṃ, kiṃ sevitabbaṃ, kiṃ na sevitabbaṃ, kiṃ me karīyamānaṃ dīgharattaṃ ahitāya dukkhāya assa, kiṃ vā pana me karīyamānaṃ dīgharattaṃ hitāya sukhāya assā’’ti. So tassa kammassa kaṭattā…pe… so tato cuto itthattaṃ āgato samāno imaṃ mahāpurisalakkhaṇaṃ paṭilabhati. Sukhumacchavi hoti, sukhumattā chaviyā rajojallaṃ kāye na upalimpati.

‘‘So tena lakkhaṇena samannāgato sace agāraṃ ajjhāvasati, rājā hoti cakkavattī…pe… rājā samāno kiṃ labhati? Mahāpañño hoti, nāssa hoti koci paññāya sadiso vā seṭṭho vā kāmabhogīnaṃ. Rājā samāno idaṃ labhati… buddho samāno kiṃ labhati? Mahāpañño hoti puthupañño hāsapañño\footnote{hāsupañño (sī. pī.)} javanapañño tikkhapañño nibbedhikapañño, nāssa hoti koci paññāya sadiso vā seṭṭho vā sabbasattānaṃ. Buddho samāno idaṃ labhati’’. Etamatthaṃ bhagavā avoca.

\paragraph{217.} Tatthetaṃ vuccati –

‘‘Pure puratthā purimāsu jātisu,

Aññātukāmo paripucchitā ahu;

Sussūsitā pabbajitaṃ upāsitā,

Atthantaro atthakathaṃ nisāmayi.

‘‘Paññāpaṭilābhagatena\footnote{paññāpaṭilābhakatena (sī. pī.) ṭīkā oloketabbā} kammunā,

Manussabhūto sukhumacchavī ahu;

Byākaṃsu uppādanimittakovidā,

Sukhumāni atthāni avecca dakkhiti.

‘‘Sace na pabbajjamupeti tādiso,

Vatteti cakkaṃ pathaviṃ pasāsati;

Atthānusiṭṭhīsu pariggahesu ca,

Na tena seyyo sadiso ca vijjati.

‘‘Sace ca pabbajjamupeti tādiso,

Nekkhammachandābhirato vicakkhaṇo;

Paññāvisiṭṭhaṃ labhate anuttaraṃ,

Pappoti bodhiṃ varabhūrimedhaso’’ti.

\subsubsection{(13) Suvaṇṇavaṇṇalakkhaṇaṃ}

\paragraph{218.} ‘‘Yampi , bhikkhave, tathāgato purimaṃ jātiṃ purimaṃ bhavaṃ purimaṃ niketaṃ pubbe manussabhūto samāno akkodhano ahosi anupāyāsabahulo, bahumpi vutto samāno nābhisajji na kuppi na byāpajji na patitthīyi, na kopañca dosañca appaccayañca pātvākāsi. Dātā ca ahosi sukhumānaṃ mudukānaṃ attharaṇānaṃ pāvuraṇānaṃ\footnote{pāpuraṇānaṃ (sī. syā. pī.)} khomasukhumānaṃ kappāsikasukhumānaṃ koseyyasukhumānaṃ kambalasukhumānaṃ. So tassa kammassa kaṭattā upacitattā…pe… so tato cuto itthattaṃ āgato samāno imaṃ mahāpurisalakkhaṇaṃ paṭilabhati. Suvaṇṇavaṇṇo hoti kañcanasannibhattaco.

‘‘So tena lakkhaṇena samannāgato sace agāraṃ ajjhāvasati, rājā hoti cakkavattī…pe… rājā samāno kiṃ labhati? Lābhī hoti sukhumānaṃ mudukānaṃ attharaṇānaṃ pāvuraṇānaṃ khomasukhumānaṃ kappāsikasukhumānaṃ koseyyasukhumānaṃ kambalasukhumānaṃ. Rājā samāno idaṃ labhati… buddho samāno kiṃ labhati? Lābhī hoti sukhumānaṃ mudukānaṃ attharaṇānaṃ pāvuraṇānaṃ khomasukhumānaṃ kappāsikasukhumānaṃ koseyyasukhumānaṃ kambalasukhumānaṃ. Buddho samāno idaṃ labhati’’. Etamatthaṃ bhagavā avoca.

\paragraph{219.} Tatthetaṃ vuccati –

‘‘Akkodhañca adhiṭṭhahi adāsi\footnote{adāsi ca (sī. pī.)},

Dānañca vatthāni sukhumāni succhavīni;

Purimatarabhave ṭhito abhivissaji,

Mahimiva suro abhivassaṃ.

‘‘Taṃ katvāna ito cuto dibbaṃ,

Upapajji\footnote{upapajja (sī. pī.)} sukataphalavipākamanubhutvā;

Kanakatanusannibho idhābhibhavati,

Suravarataroriva indo.

‘‘Gehañcāvasati naro apabbajja,

Micchaṃ mahatimahiṃ anusāsati\footnote{pasāsati (syā.)};

Pasayha sahidha sattaratanaṃ\footnote{pasayha abhivasana-varataraṃ (sī. pī.)},

Paṭilabhati vimala\footnote{vipula (syā.), vipulaṃ (sī. pī.)} sukhumacchaviṃ suciñca.

‘‘Lābhī acchādanavatthamokkhapāvuraṇānaṃ,

Bhavati yadi anāgāriyataṃ upeti;

Sahito\footnote{suhita (syā.), sa hi (sī. pī.)} purimakataphalaṃ anubhavati,

Na bhavati katassa panāso’’ti.

\subsubsection{(14) Kosohitavatthaguyhalakkhaṇaṃ}

\paragraph{220.} Yampi, bhikkhave, tathāgato purimaṃ jātiṃ purimaṃ bhavaṃ purimaṃ niketaṃ pubbe manussabhūto samāno cirappanaṭṭhe sucirappavāsino ñātimitte suhajje sakhino samānetā ahosi. Mātarampi puttena samānetā ahosi, puttampi mātarā samānetā ahosi, pitarampi puttena samānetā ahosi, puttampi pitarā samānetā ahosi, bhātarampi bhātarā samānetā ahosi, bhātarampi bhaginiyā samānetā ahosi, bhaginimpi bhātarā samānetā ahosi, bhaginimpi bhaginiyā samānetā ahosi, samaṅgīkatvā\footnote{samaggiṃ katvā (sī. syā. pī.)} ca abbhanumoditā ahosi. So tassa kammassa kaṭattā…pe… so tato cuto itthattaṃ āgato samāno imaṃ mahāpurisalakkhaṇaṃ paṭilabhati – kosohitavatthaguyho hoti.

‘‘So tena lakkhaṇena samannāgato sace agāraṃ ajjhāvasati, rājā hoti cakkavattī…pe… rājā samāno kiṃ labhati? Pahūtaputto hoti, parosahassaṃ kho panassa puttā bhavanti sūrā vīraṅgarūpā parasenappamaddanā. Rājā samāno idaṃ labhati… buddho samāno kiṃ labhati? Pahūtaputto hoti, anekasahassaṃ kho panassa puttā bhavanti sūrā vīraṅgarūpā parasenappamaddanā. Buddho samāno idaṃ labhati’’. Etamatthaṃ bhagavā avoca.

\paragraph{221.} Tatthetaṃ vuccati –

‘‘Pure puratthā purimāsu jātisu,

Cirappanaṭṭhe sucirappavāsino;

Ñātī suhajje sakhino samānayi,

Samaṅgikatvā anumoditā ahu.

‘‘So tena\footnote{sa tena (ka.)} kammena divaṃ samakkami,

Sukhañca khiḍḍāratiyo ca anvabhi;

Tato cavitvā punarāgato idha,

Kosohitaṃ vindati vatthachādiyaṃ.

‘‘Pahūtaputto bhavatī tathāvidho,

Parosahassañca\footnote{parosahassassa (sī. pī.)} bhavanti atrajā;

Sūrā ca vīrā ca\footnote{sūrā ca vīraṅgarūpā (ka.)} amittatāpanā,

Gihissa pītiṃjananā piyaṃvadā.

‘‘Bahūtarā pabbajitassa iriyato,

Bhavanti puttā vacanānusārino;

Gihissa vā pabbajitassa vā puna,

Taṃ lakkhaṇaṃ jāyati tadatthajotaka’’nti.

\xsubsubsectionEnd{Paṭhamabhāṇavāro niṭṭhito.}

\subsubsection{(15-16) Parimaṇḍalaanonamajaṇṇuparimasanalakkhaṇāni}

\paragraph{222.} ‘‘Yampi , bhikkhave, tathāgato purimaṃ jātiṃ purimaṃ bhavaṃ purimaṃ niketaṃ pubbe manussabhūto samāno mahājanasaṅgahaṃ\footnote{mahājanasaṅgāhakaṃ (ka.)} samekkhamāno\footnote{samapekkhamāno (ka.)} samaṃ jānāti sāmaṃ jānāti, purisaṃ jānāti purisavisesaṃ jānāti – ‘ayamidamarahati ayamidamarahatī’ti tattha tattha purisavisesakaro ahosi. So tassa kammassa kaṭattā…pe… so tato cuto itthattaṃ āgato samāno imāni dve mahāpurisalakkhaṇāni paṭilabhati. Nigrodha parimaṇḍalo ca hoti, ṭhitakoyeva ca anonamanto ubhohi pāṇitalehi jaṇṇukāni parimasati parimajjati.

‘‘So tehi lakkhaṇehi samannāgato sace agāraṃ ajjhāvasati, rājā hoti cakkavattī…pe… rājā samāno kiṃ labhati ? Aḍḍho hoti mahaddhano mahābhogo pahūtajātarūparajato pahūtavittūpakaraṇo pahūtadhanadhañño paripuṇṇakosakoṭṭhāgāro. Rājā samāno idaṃ labhati…pe… buddho samāno kiṃ labhati? Aḍḍho hoti mahaddhano mahābhogo. Tassimāni dhanāni honti, seyyathidaṃ, saddhādhanaṃ sīladhanaṃ hiridhanaṃ ottappadhanaṃ sutadhanaṃ cāgadhanaṃ paññādhanaṃ. Buddho samāno idaṃ labhati’’. Etamatthaṃ bhagavā avoca.

\paragraph{223.} Tatthetaṃ vuccati –

‘‘Tuliya paṭivicaya cintayitvā,

Mahājanasaṅgahanaṃ\footnote{mahājanaṃ saṅgāhakaṃ (ka.)} samekkhamāno;

Ayamidamarahati tattha tattha,

Purisavisesakaro pure ahosi.

‘‘Mahiñca pana\footnote{samā ca pana (syā.), sa hi ca pana (sī. pī.)} ṭhito anonamanto,

Phusati karehi ubhohi jaṇṇukāni;

Mahiruhaparimaṇḍalo ahosi,

Sucaritakammavipākasesakena.

‘‘Bahuvividhanimittalakkhaṇaññū,

Atinipuṇā manujā byākariṃsu;

Bahuvividhā gihīnaṃ arahāni,

Paṭilabhati daharo susu kumāro.

‘‘Idha ca mahīpatissa kāmabhogī,

Gihipatirūpakā bahū bhavanti;

Yadi ca jahati sabbakāmabhogaṃ,

Labhati anuttaraṃ uttamadhanagga’’nti.

\subsubsection{(17-19) Sīhapubbaddhakāyāditilakkhaṇaṃ}

\paragraph{224.} ‘‘Yampi , bhikkhave, tathāgato purimaṃ jātiṃ purimaṃ bhavaṃ purimaṃ niketaṃ pubbe manussabhūto samāno bahujanassa atthakāmo ahosi hitakāmo phāsukāmo yogakkhemakāmo – ‘kintime saddhāya vaḍḍheyyuṃ, sīlena vaḍḍheyyuṃ, sutena vaḍḍheyyuṃ\footnote{sutena vaḍḍheyyuṃ, buddhiyā vaḍḍheyyuṃ (syā.)}, cāgena vaḍḍheyyuṃ, dhammena vaḍḍheyyuṃ, paññāya vaḍḍheyyuṃ, dhanadhaññena vaḍḍheyyuṃ, khettavatthunā vaḍḍheyyuṃ, dvipadacatuppadehi vaḍḍheyyuṃ, puttadārehi vaḍḍheyyuṃ, dāsakammakaraporisehi vaḍḍheyyuṃ, ñātīhi vaḍḍheyyuṃ, mittehi vaḍḍheyyuṃ, bandhavehi vaḍḍheyyu’nti. So tassa kammassa kaṭattā…pe… so tato cuto itthattaṃ āgato samāno imāni tīṇi mahāpurisalakkhaṇāni paṭilabhati. Sīhapubbaddhakāyo ca hoti citantaraṃso ca samavaṭṭakkhandho ca.

‘‘So tehi lakkhaṇehi samannāgato sace agāraṃ ajjhāvasati, rājā hoti cakkavattī…pe… rājā samāno kiṃ labhati? Aparihānadhammo hoti, na parihāyati dhanadhaññena khettavatthunā dvipadacatuppadehi puttadārehi dāsakammakaraporisehi ñātīhi mittehi bandhavehi, na parihāyati sabbasampattiyā. Rājā samāno idaṃ labhati… buddho samāno kiṃ labhati? Aparihānadhammo hoti, na parihāyati saddhāya sīlena sutena cāgena paññāya, na parihāyati sabbasampattiyā. Buddho samāno idaṃ labhati’’. Etamatthaṃ bhagavā avoca.

\paragraph{225.} Tatthetaṃ vuccati –

‘‘Saddhāya sīlena sutena buddhiyā,

Cāgena dhammena bahūhi sādhuhi;

Dhanena dhaññena ca khettavatthunā,

Puttehi dārehi catuppadehi ca.

‘‘Ñātīhi mittehi ca bandhavehi ca,

Balena vaṇṇena sukhena cūbhayaṃ;

Kathaṃ na hāyeyyuṃ pareti icchati,

Atthassa middhī ca\footnote{idaṃ samiddhañca (ka.), addhaṃ samiddhañca (syā.)} panābhikaṅkhati.

‘‘Sa sīhapubbaddhasusaṇṭhito ahu,

Samavaṭṭakkhandho ca citantaraṃso;

Pubbe suciṇṇena katena kammunā,

Ahāniyaṃ pubbanimittamassa taṃ.

‘‘Gihīpi dhaññena dhanena vaḍḍhati,

Puttehi dārehi catuppadehi ca;

Akiñcano pabbajito anuttaraṃ,

Pappoti bodhiṃ asahānadhammata’’nti\footnote{sambodhimahānadhammatanti (syā. ka.) ṭīkā oloketabbā}.

\subsubsection{(20) Rasaggasaggitālakkhaṇaṃ}

\paragraph{226.} ‘‘Yampi , bhikkhave, tathāgato purimaṃ jātiṃ purimaṃ bhavaṃ purimaṃ niketaṃ pubbe manussabhūto samāno sattānaṃ aviheṭhakajātiko ahosi pāṇinā vā leḍḍunā vā daṇḍena vā satthena vā. So tassa kammassa kaṭattā upacitattā…pe… so tato cuto itthattaṃ āgato samāno imaṃ mahāpurisalakkhaṇaṃ paṭilabhati, rasaggasaggī hoti, uddhaggāssa rasaharaṇīyo gīvāya jātā honti samābhivāhiniyo\footnote{samavāharasaharaṇiyo (syā.)}.

‘‘So tena lakkhaṇena samannāgato sace agāraṃ ajjhāvasati, rājā hoti cakkavattī…pe… rājā samāno kiṃ labhati? Appābādho hoti appātaṅko, samavepākiniyā gahaṇiyā samannāgato nātisītāya nāccuṇhāya. Rājā samāno idaṃ labhati… buddho samāno kiṃ labhati? Appābādho hoti appātaṅko samavepākiniyā gahaṇiyā samannāgato nātisītāya nāccuṇhāya majjhimāya padhānakkhamāya. Buddho samāno idaṃ labhati’’. Etamatthaṃ bhagavā avoca.

\paragraph{227.} Tatthetaṃ vuccati –

‘‘Na pāṇidaṇḍehi panātha leḍḍunā,

Satthena vā maraṇavadhena\footnote{māraṇavadhena (ka.)} vā pana;

Ubbādhanāya paritajjanāya vā,

Na heṭhayī janatamaheṭhako ahu.

‘‘Teneva so sugatimupecca modati,

Sukhapphalaṃ kariya sukhāni vindati;

Samojasā\footnote{sampajjasā (sī. pī.), pāmuñjasā (syā.), sāmañca sā (ka.)} rasaharaṇī susaṇṭhitā,

Idhāgato labhati rasaggasaggitaṃ.

‘‘Tenāhu naṃ atinipuṇā vicakkhaṇā,

Ayaṃ naro sukhabahulo bhavissati;

Gihissa vā pabbajitassa vā puna\footnote{pana (syā.)},

Taṃ lakkhaṇaṃ bhavati tadatthajotaka’’nti.

\subsubsection{(21-22) Abhinīlanettagopakhumalakkhaṇāni}

\paragraph{228.} ‘‘Yampi, bhikkhave, tathāgato purimaṃ jātiṃ purimaṃ bhavaṃ purimaṃ niketaṃ pubbe manussabhūto samāno na ca visaṭaṃ, na ca visāci\footnote{na ca visācitaṃ (sī. pī.), na ca visāvi (syā.)}, na ca pana viceyya pekkhitā, ujuṃ tathā pasaṭamujumano, piyacakkhunā bahujanaṃ udikkhitā ahosi. So tassa kammassa kaṭattā…pe… so tato cuto itthattaṃ āgato samāno imāni dve mahāpurisalakkhaṇāni paṭilabhati. Abhinīlanetto ca hoti gopakhumo ca.

‘‘So tehi lakkhaṇehi samannāgato, sace agāraṃ ajjhāvasati, rājā hoti cakkavattī…pe… rājā samāno kiṃ labhati? Piyadassano hoti bahuno janassa, piyo hoti manāpo brāhmaṇagahapatikānaṃ negamajānapadānaṃ gaṇakamahāmattānaṃ anīkaṭṭhānaṃ dovārikānaṃ amaccānaṃ pārisajjānaṃ rājūnaṃ bhogiyānaṃ kumārānaṃ. Rājā samāno idaṃ labhati…pe… buddho samāno kiṃ labhati? Piyadassano hoti bahuno janassa, piyo hoti manāpo bhikkhūnaṃ bhikkhunīnaṃ upāsakānaṃ upāsikānaṃ devānaṃ manussānaṃ asurānaṃ nāgānaṃ gandhabbānaṃ. Buddho samāno idaṃ labhati’’. Etamatthaṃ bhagavā avoca.

\paragraph{229.} Tatthetaṃ vuccati –

‘‘Na ca visaṭaṃ na ca visāci\footnote{visācitaṃ (sī. pī.), visāvi (syā.)}, na ca pana viceyyapekkhitā;

Ujuṃ tathā pasaṭamujumano, piyacakkhunā bahujanaṃ udikkhitā.

‘‘Sugatīsu so phalavipākaṃ,

Anubhavati tattha modati;

Idha ca pana bhavati gopakhumo,

Abhinīlanettanayano sudassano.

‘‘Abhiyogino ca nipuṇā,

Bahū pana nimittakovidā;

Sukhumanayanakusalā manujā,

Piyadassanoti abhiniddisanti naṃ.

‘‘Piyadassano gihīpi santo ca,

Bhavati bahujanapiyāyito;

Yadi ca na bhavati gihī samaṇo hoti,

Piyo bahūnaṃ sokanāsano’’ti.

\subsubsection{(23) Uṇhīsasīsalakkhaṇaṃ}

\paragraph{230.} ‘‘Yampi, bhikkhave, tathāgato purimaṃ jātiṃ purimaṃ bhavaṃ purimaṃ niketaṃ pubbe manussabhūto samāno bahujanapubbaṅgamo ahosi kusalesu dhammesu bahujanapāmokkho kāyasucarite vacīsucarite manosucarite dānasaṃvibhāge sīlasamādāne uposathupavāse matteyyatāya petteyyatāya sāmaññatāya brahmaññatāya kule jeṭṭhāpacāyitāya aññataraññataresu ca adhikusalesu dhammesu. So tassa kammassa kaṭattā…pe… so tato cuto itthattaṃ āgato samāno imaṃ mahāpurisalakkhaṇaṃ paṭilabhati – uṇhīsasīso hoti.

‘‘So tena lakkhaṇena samannāgato sace agāraṃ ajjhāvasati, rājā hoti cakkavattī…pe… rājā samāno kiṃ labhati? Mahāssa jano anvāyiko hoti, brāhmaṇagahapatikā negamajānapadā gaṇakamahāmattā anīkaṭṭhā dovārikā amaccā pārisajjā rājāno bhogiyā kumārā. Rājā samāno idaṃ labhati… buddho samāno kiṃ labhati? Mahāssa jano anvāyiko hoti, bhikkhū bhikkhuniyo upāsakā upāsikāyo devā manussā asurā nāgā gandhabbā. Buddho samāno idaṃ labhati’’. Etamatthaṃ bhagavā avoca.

\paragraph{231.} Tatthetaṃ vuccati –

‘‘Pubbaṅgamo sucaritesu ahu,

Dhammesu dhammacariyābhirato;

Anvāyiko bahujanassa ahu,

Saggesu vedayittha puññaphalaṃ.

‘‘Veditvā so sucaritassa phalaṃ,

Uṇhīsasīsattamidhajjhagamā;

Byākaṃsu byañjananimittadharā,

Pubbaṅgamo bahujanaṃ hessati.

‘‘Paṭibhogiyā manujesu idha,

Pubbeva tassa abhiharanti tadā;

Yadi khattiyo bhavati bhūmipati,

Paṭihārakaṃ bahujane labhati.

‘‘Atha cepi pabbajati so manujo,

Dhammesu hoti paguṇo visavī;

Tassānusāsaniguṇābhirato,

Anvāyiko bahujano bhavatī’’ti.

\subsubsection{(24-25) Ekekalomatāuṇṇālakkhaṇāni}

\paragraph{232.} ‘‘Yampi, bhikkhave, tathāgato purimaṃ jātiṃ purimaṃ bhavaṃ purimaṃ niketaṃ pubbe manussabhūto samāno musāvādaṃ pahāya musāvādā paṭivirato ahosi, saccavādī saccasandho theto paccayiko avisaṃvādako lokassa . So tassa kammassa kaṭattā upacitattā…pe… so tato cuto itthattaṃ āgato samāno imāni dve mahāpurisalakkhaṇāni paṭilabhati. Ekekalomo ca hoti, uṇṇā ca bhamukantare jātā hoti odātā mudutūlasannibhā.

‘‘So tehi lakkhaṇehi samannāgato, sace agāraṃ ajjhāvasati, rājā hoti cakkavattī…pe… rājā samāno kiṃ labhati? Mahāssa jano upavattati, brāhmaṇagahapatikā negamajānapadā gaṇakamahāmattā anīkaṭṭhā dovārikā amaccā pārisajjā rājāno bhogiyā kumārā. Rājā samāno idaṃ labhati… buddho samāno kiṃ labhati? Mahāssa jano upavattati, bhikkhū bhikkhuniyo upāsakā upāsikāyo devā manussā asurā nāgā gandhabbā. Buddho samāno idaṃ labhati’’. Etamatthaṃ bhagavā avoca.

\paragraph{233.} Tatthetaṃ vuccati –

‘‘Saccappaṭiñño purimāsu jātisu,

Advejjhavāco alikaṃ vivajjayi;

Na so visaṃvādayitāpi kassaci,

Bhūtena tacchena tathena bhāsayi\footnote{tosayi (sī. pī.)}.

‘‘Setā susukkā mudutūlasannibhā,

Uṇṇā sujātā\footnote{uṇṇāssa jātā (ka. sī.)} bhamukantare ahu;

Na lomakūpesu duve ajāyisuṃ,

Ekekalomūpacitaṅgavā ahu.

‘‘Taṃ lakkhaṇaññū bahavo samāgatā,

Byākaṃsu uppādanimittakovidā;

Uṇṇā ca lomā ca yathā susaṇṭhitā,

Upavattatī īdisakaṃ bahujjano.

‘‘Gihimpi santaṃ upavattatī jano,

Bahu puratthāpakatena kammunā;

Akiñcanaṃ pabbajitaṃ anuttaraṃ,

Buddhampi santaṃ upavattati jano’’ti.

\subsubsection{(26-27) Cattālīsaaviraḷadantalakkhaṇāni}

\paragraph{234.} ‘‘Yampi, bhikkhave tathāgato purimaṃ jātiṃ purimaṃ bhavaṃ purimaṃ niketaṃ pubbe manussabhūto samāno pisuṇaṃ vācaṃ pahāya pisuṇāya vācāya paṭivirato ahosi. Ito sutvā na amutra akkhātā imesaṃ bhedāya, amutra vā sutvā na imesaṃ akkhātā amūsaṃ bhedāya , iti bhinnānaṃ vā sandhātā, sahitānaṃ vā anuppadātā, samaggārāmo samaggarato samagganandī samaggakaraṇiṃ vācaṃ bhāsitā ahosi. So tassa kammassa kaṭattā…pe… so tato cuto itthattaṃ āgato samāno imāni dve mahāpurisalakkhaṇāni paṭilabhati. Cattālīsadanto ca hoti aviraḷadanto ca.

‘‘So tehi lakkhaṇehi samannāgato sace agāraṃ ajjhāvasati, rājā hoti cakkavattī…pe… rājā samāno kiṃ labhati? Abhejjapariso hoti, abhejjāssa honti parisā, brāhmaṇagahapatikā negamajānapadā gaṇakamahāmattā anīkaṭṭhā dovārikā amaccā pārisajjā rājāno bhogiyā kumārā. Rājā samāno idaṃ labhati … buddho samāno kiṃ labhati? Abhejjapariso hoti, abhejjāssa honti parisā, bhikkhū bhikkhuniyo upāsakā upāsikāyo devā manussā asurā nāgā gandhabbā. Buddho samāno idaṃ labhati’’. Etamatthaṃ bhagavā avoca.

\paragraph{235.} Tatthetaṃ vuccati –

‘‘Vebhūtiyaṃ sahitabhedakāriṃ,

Bhedappavaḍḍhanavivādakāriṃ;

Kalahappavaḍḍhanaākiccakāriṃ,

Sahitānaṃ bhedajananiṃ na bhaṇi.

‘‘Avivādavaḍḍhanakariṃ sugiraṃ,

Bhinnānusandhijananiṃ abhaṇi;

Kalahaṃ janassa panudī samaṅgī,

Sahitehi nandati pamodati ca.

‘‘Sugatīsu so phalavipākaṃ,

Anubhavati tattha modati;

Dantā idha honti aviraḷā sahitā,

Caturo dasassa mukhajā susaṇṭhitā.

‘‘Yadi khattiyo bhavati bhūmipati,

Avibhediyāssa parisā bhavati;

Samaṇo ca hoti virajo vimalo,

Parisāssa hoti anugatā acalā’’ti.

\subsubsection{(28-29) Pahūtajivhābrahmassaralakkhaṇāni}

\paragraph{236.} ‘‘Yampi , bhikkhave, tathāgato purimaṃ jātiṃ purimaṃ bhavaṃ purimaṃ niketaṃ pubbe manussabhūto samāno pharusaṃ vācaṃ pahāya pharusāya vācāya paṭivirato ahosi. Yā sā vācā nelā kaṇṇasukhā pemanīyā hadayaṅgamā porī bahujanakantā bahujanamanāpā, tathārūpiṃ vācaṃ bhāsitā ahosi. So tassa kammassa kaṭattā upacitattā…pe… so tato cuto itthattaṃ āgato samāno imāni dve mahāpurisalakkhaṇāni paṭilabhati. Pahūtajivho ca hoti brahmassaro ca karavīkabhāṇī.

‘‘So tehi lakkhaṇehi samannāgato sace agāraṃ ajjhāvasati, rājā hoti cakkavattī…pe… rājā samāno kiṃ labhati? Ādeyyavāco hoti, ādiyantissa vacanaṃ brāhmaṇagahapatikā negamajānapadā gaṇakamahāmattā anīkaṭṭhā dovārikā amaccā pārisajjā rājāno bhogiyā kumārā. Rājā samāno idaṃ labhati… buddho samāno kiṃ labhati? Ādeyyavāco hoti, ādiyantissa vacanaṃ bhikkhū bhikkhuniyo upāsakā upāsikāyo devā manussā asurā nāgā gandhabbā. Buddho samāno idaṃ labhati’’. Etamatthaṃ bhagavā avoca.

\paragraph{237.} Tatthetaṃ vuccati –

‘‘Akkosabhaṇḍanavihesakāriṃ,

Ubbādhikaṃ\footnote{ubbādhakaraṃ (syā.)} bahujanappamaddanaṃ;

Abāḷhaṃ giraṃ so na bhaṇi pharusaṃ,

Madhuraṃ bhaṇi susaṃhitaṃ\footnote{susahitaṃ (syā.)} sakhilaṃ.

‘‘Manaso piyā hadayagāminiyo,

Vācā so erayati kaṇṇasukhā;

Vācāsuciṇṇaphalamanubhavi,

Saggesu vedayatha\footnote{vedayati (?) ṭīkā oloketabbā} puññaphalaṃ.

‘‘Veditvā so sucaritassa phalaṃ,

Brahmassarattamidhamajjhagamā;

Jivhāssa hoti vipulā puthulā,

Ādeyyavākyavacano bhavati.

‘‘Gihinopi ijjhati yathā bhaṇato,

Atha ce pabbajati so manujo;

Ādiyantissa vacanaṃ janatā,

Bahuno bahuṃ subhaṇitaṃ bhaṇato’’ti.

\subsubsection{(30) Sīhahanulakkhaṇaṃ}

\paragraph{238.} ‘‘Yampi , bhikkhave, tathāgato purimaṃ jātiṃ purimaṃ bhavaṃ purimaṃ niketaṃ pubbe manussabhūto samāno samphappalāpaṃ pahāya samphappalāpā paṭivirato ahosi kālavādī bhūtavādī atthavādī dhammavādī vinayavādī, nidhānavatiṃ vācaṃ bhāsitā ahosi kālena sāpadesaṃ pariyantavatiṃ atthasaṃhitaṃ. So tassa kammassa kaṭattā…pe… so tato cuto itthattaṃ āgato samāno imaṃ mahāpurisalakkhaṇaṃ paṭilabhati, sīhahanu hoti.

‘‘So tena lakkhaṇena samannāgato sace agāraṃ ajjhāvasati, rājā hoti cakkavattī…pe… rājā samāno kiṃ labhati? Appadhaṃsiyo hoti kenaci manussabhūtena paccatthikena paccāmittena. Rājā samāno idaṃ labhati… buddho samāno kiṃ labhati? Appadhaṃsiyo hoti abbhantarehi vā bāhirehi vā paccatthikehi paccāmittehi, rāgena vā dosena vā mohena vā samaṇena vā brāhmaṇena vā devena vā mārena vā brahmunā vā kenaci vā lokasmiṃ. Buddho samāno idaṃ labhati’’. Etamatthaṃ bhagavā avoca.

\paragraph{239.} Tatthetaṃ vuccati –

‘‘Na samphappalāpaṃ na muddhataṃ\footnote{buddhatanti (ka.)},

Avikiṇṇavacanabyappatho ahosi;

Ahitamapi ca apanudi,

Hitamapi ca bahujanasukhañca abhaṇi.

‘‘Taṃ katvā ito cuto divamupapajji,

Sukataphalavipākamanubhosi;

Caviya punaridhāgato samāno,

Dvidugamavaratarahanuttamalattha.

‘‘Rājā hoti suduppadhaṃsiyo,

Manujindo manujādhipati mahānubhāvo;

Tidivapuravarasamo bhavati,

Suravarataroriva indo.

‘‘Gandhabbāsurayakkharakkhasebhi\footnote{surasakkarakkhasebhi (syā.)},

Surehi na hi bhavati suppadhaṃsiyo;

Tathatto yadi bhavati tathāvidho,

Idha disā ca paṭidisā ca vidisā cā’’ti.

\subsubsection{(31-32) Samadantasusukkadāṭhālakkhaṇāni}

\paragraph{240.} ‘‘Yampi, bhikkhave, tathāgato purimaṃ jātiṃ purimaṃ bhavaṃ purimaṃ niketaṃ pubbe manussabhūto samāno micchājīvaṃ pahāya sammāājīvena jīvikaṃ kappesi, tulākūṭa kaṃsakūṭa mānakūṭa ukkoṭana vañcana nikati sāciyoga chedana vadha bandhana viparāmosa ālopa sahasākārā\footnote{sāhasākārā (sī. syā. pī.)} paṭivirato ahosi. So tassa kammassa kaṭattā upacitattā ussannattā vipulattā kāyassa bhedā paraṃ maraṇā sugatiṃ saggaṃ lokaṃ upapajjati. So tattha aññe deve dasahi ṭhānehi adhigaṇhāti dibbena āyunā dibbena vaṇṇena dibbena sukhena dibbena yasena dibbena ādhipateyyena dibbehi rūpehi dibbehi saddehi dibbehi gandhehi dibbehi rasehi dibbehi phoṭṭhabbehi. So tato cuto itthattaṃ āgato samāno imāni dve mahāpurisalakkhaṇāni paṭilabhati, samadanto ca hoti susukkadāṭho ca.

‘‘So tehi lakkhaṇehi samannāgato sace agāraṃ ajjhāvasati, rājā hoti cakkavattī dhammiko dhammarājā cāturanto vijitāvī janapadatthāvariyappatto sattaratanasamannāgato. Tassimāni satta ratanāni bhavanti, seyyathidaṃ – cakkaratanaṃ hatthiratanaṃ assaratanaṃ maṇiratanaṃ itthiratanaṃ gahapatiratanaṃ pariṇāyakaratanameva sattamaṃ. Parosahassaṃ kho panassa puttā bhavanti sūrā vīraṅgarūpā parasenappamaddanā. So imaṃ pathaviṃ sāgarapariyantaṃ akhilamanimittamakaṇṭakaṃ iddhaṃ phītaṃ khemaṃ sivaṃ nirabbudaṃ adaṇḍena asatthena dhammena abhivijiya ajjhāvasati. Rājā samāno kiṃ labhati? Suciparivāro hoti sucissa honti parivārā brāhmaṇagahapatikā negamajānapadā gaṇakamahāmattā anīkaṭṭhā dovārikā amaccā pārisajjā rājāno bhogiyā kumārā. Rājā samāno idaṃ labhati.

‘‘Sace kho pana agārasmā anagāriyaṃ pabbajati, arahaṃ hoti sammāsambuddho loke vivaṭṭacchado. Buddho samāno kiṃ labhati? Suciparivāro hoti, sucissa honti parivārā, bhikkhū bhikkhuniyo upāsakā upāsikāyo devā manussā asurā nāgā gandhabbā. Buddho samāno idaṃ labhati’’. Etamatthaṃ bhagavā avoca.

\paragraph{241.} Tatthetaṃ vuccati –

‘‘Micchājīvañca avassaji samena vuttiṃ,

Sucinā so janayittha dhammikena;

Ahitamapi ca apanudi,

Hitamapi ca bahujanasukhañca acari.

‘‘Sagge vedayati naro sukhapphalāni,

Karitvā nipuṇebhi vidūhi sabbhi;

Vaṇṇitāni tidivapuravarasamo,

Abhiramati ratikhiḍḍāsamaṅgī.

‘‘Laddhānaṃ mānusakaṃ bhavaṃ tato,

Cavitvāna sukataphalavipākaṃ;

Sesakena paṭilabhati lapanajaṃ,

Samamapi sucisusukkaṃ\footnote{laddhāna manussakaṃ bhavaṃ tato caviya, puna sukataphalavipākasesakena; paṭilabhati lapanajaṃ samamapi, suci ca suvisuddhasusukkaṃ (syā.)}.

‘‘Taṃ veyyañjanikā samāgatā bahavo,

Byākaṃsu nipuṇasammatā manujā;

Sucijanaparivāragaṇo bhavati,

Dijasamasukkasucisobhanadanto.

‘‘Rañño hoti bahujano,

Suciparivāro mahatiṃ mahiṃ anusāsato;

Pasayha na ca janapadatudanaṃ,

Hitamapi ca bahujanasukhañca caranti.

‘‘Atha ce pabbajati bhavati vipāpo,

Samaṇo samitarajo vivaṭṭacchado;

Vigatadarathakilamatho,

Imamapi ca paramapi ca\footnote{imampi ca parampi ca (pī.), paraṃpi paramaṃpi ca (syā.)} passati lokaṃ.

‘‘Tassovādakarā bahugihī ca pabbajitā ca,

Asuciṃ garahitaṃ dhunanti pāpaṃ;

Sa hi sucibhi parivuto bhavati,

Malakhilakalikilese panudehī’’ti\footnote{tassovādakarā bahugihī ca, pabbajitā ca asucivigarahita; panudipāpassa hi sucibhiparivuto, bhavati malakhilakakilese panudeti (syā.)}.

Idamavoca bhagavā. Attamanā te bhikkhū bhagavato bhāsitaṃ abhinandunti.

\xsectionEnd{Lakkhaṇasuttaṃ niṭṭhitaṃ sattamaṃ.}
