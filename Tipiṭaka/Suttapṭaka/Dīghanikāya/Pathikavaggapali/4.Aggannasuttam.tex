\section{Aggaññasuttaṃ}

\subsubsection{Vāseṭṭhabhāradvājā}

\paragraph{111.} Evaṃ me sutaṃ – ekaṃ samayaṃ bhagavā sāvatthiyaṃ viharati pubbārāme migāramātupāsāde. Tena kho pana samayena vāseṭṭhabhāradvājā bhikkhūsu parivasanti bhikkhubhāvaṃ ākaṅkhamānā. Atha kho bhagavā sāyanhasamayaṃ paṭisallānā vuṭṭhito pāsādā orohitvā pāsādapacchāyāyaṃ\footnote{pāsādacchāyāyaṃ (ka.)} abbhokāse caṅkamati.

\paragraph{112.} Addasā kho vāseṭṭho bhagavantaṃ sāyanhasamayaṃ paṭisallānā vuṭṭhitaṃ pāsādā orohitvā pāsādapacchāyāyaṃ abbhokāse caṅkamantaṃ. Disvāna bhāradvājaṃ āmantesi – ‘‘ayaṃ, āvuso bhāradvāja, bhagavā sāyanhasamayaṃ paṭisallānā vuṭṭhito pāsādā orohitvā pāsādapacchāyāyaṃ abbhokāse caṅkamati. Āyāmāvuso bhāradvāja, yena bhagavā tenupasaṅkamissāma; appeva nāma labheyyāma bhagavato santikā\footnote{sammukhā (syā. ka.)} dhammiṃ kathaṃ savanāyā’’ti. ‘‘Evamāvuso’’ti kho bhāradvājo vāseṭṭhassa paccassosi.

\paragraph{113.} Atha kho vāseṭṭhabhāradvājā yena bhagavā tenupasaṅkamiṃsu; upasaṅkamitvā bhagavantaṃ abhivādetvā bhagavantaṃ caṅkamantaṃ anucaṅkamiṃsu. Atha kho bhagavā vāseṭṭhaṃ āmantesi – ‘‘tumhe khvattha, vāseṭṭha, brāhmaṇajaccā brāhmaṇakulīnā brāhmaṇakulā agārasmā anagāriyaṃ pabbajitā, kacci vo, vāseṭṭha, brāhmaṇā na akkosanti na paribhāsantī’’ti? ‘‘Taggha no, bhante, brāhmaṇā akkosanti paribhāsanti attarūpāya paribhāsāya paripuṇṇāya, no aparipuṇṇāyā’’ti. ‘‘Yathā kathaṃ pana vo, vāseṭṭha, brāhmaṇā akkosanti paribhāsanti attarūpāya paribhāsāya paripuṇṇāya, no aparipuṇṇāyā’’ti? ‘‘Brāhmaṇā, bhante, evamāhaṃsu – ‘brāhmaṇova seṭṭho vaṇṇo, hīnā aññe vaṇṇā\footnote{hīno añño vaṇṇo (sī. pī. ma. ni. 2 madhurasutta)}. Brāhmaṇova sukko vaṇṇo , kaṇhā aññe vaṇṇā\footnote{kaṇho añño vaṇṇo (sī. pī. ma. ni. 2 madhurasutta)}. Brāhmaṇāva sujjhanti, no abrāhmaṇā. Brāhmaṇāva\footnote{brāhmaṇā (syā.)} brahmuno puttā orasā mukhato jātā brahmajā brahmanimmitā brahmadāyādā. Te tumhe seṭṭhaṃ vaṇṇaṃ hitvā hīnamattha vaṇṇaṃ ajjhupagatā, yadidaṃ muṇḍake samaṇake ibbhe kaṇhe bandhupādāpacce. Tayidaṃ na sādhu, tayidaṃ nappatirūpaṃ, yaṃ tumhe seṭṭhaṃ vaṇṇaṃ hitvā hīnamattha vaṇṇaṃ ajjhupagatā yadidaṃ muṇḍake samaṇake ibbhe kaṇhe bandhupādāpacce’ti. Evaṃ kho no, bhante, brāhmaṇā akkosanti paribhāsanti attarūpāya paribhāsāya paripuṇṇāya, no aparipuṇṇāyā’’ti.

\paragraph{114.} ‘‘Taggha vo, vāseṭṭha, brāhmaṇā porāṇaṃ assarantā evamāhaṃsu – ‘brāhmaṇova seṭṭho vaṇṇo, hīnā aññe vaṇṇā; brāhmaṇova sukko vaṇṇo, kaṇhā aññe vaṇṇā; brāhmaṇāva sujjhanti, no abrāhmaṇā; brāhmaṇāva brahmuno puttā orasā mukhato jātā brahmajā brahmanimmitā brahmadāyādā’ti. Dissanti kho pana, vāseṭṭha, brāhmaṇānaṃ brāhmaṇiyo utuniyopi gabbhiniyopi vijāyamānāpi pāyamānāpi. Te ca brāhmaṇā yonijāva samānā evamāhaṃsu – ‘brāhmaṇova seṭṭho vaṇṇo, hīnā aññe vaṇṇā; brāhmaṇova sukko vaṇṇo, kaṇhā aññe vaṇṇā; brāhmaṇāva sujjhanti, no abrāhmaṇā; brāhmaṇāva brahmuno puttā orasā mukhato jātā brahmajā brahmanimmitā brahmadāyādā’ti. Te\footnote{te ca (syā. ka.)} brahmānañceva abbhācikkhanti, musā ca bhāsanti, bahuñca apuññaṃ pasavanti.

\subsubsection{Catuvaṇṇasuddhi}

\paragraph{115.} ‘‘Cattārome, vāseṭṭha, vaṇṇā – khattiyā, brāhmaṇā, vessā, suddā. Khattiyopi kho, vāseṭṭha, idhekacco pāṇātipātī hoti adinnādāyī kāmesumicchācārī musāvādī pisuṇavāco pharusavāco samphappalāpī abhijjhālu byāpannacitto micchādiṭṭhī. Iti kho, vāseṭṭha, yeme dhammā akusalā akusalasaṅkhātā sāvajjā sāvajjasaṅkhātā asevitabbā asevitabbasaṅkhātā naalamariyā naalamariyasaṅkhātā kaṇhā kaṇhavipākā viññugarahitā, khattiyepi te\footnote{kho vāseṭṭha (ka.)} idhekacce sandissanti. Brāhmaṇopi kho, vāseṭṭha…pe… vessopi kho, vāseṭṭha…pe… suddopi kho, vāseṭṭha, idhekacco pāṇātipātī hoti adinnādāyī kāmesumicchācārī musāvādī pisuṇavāco pharusavāco samphappalāpī abhijjhālu byāpannacitto micchādiṭṭhī. Iti kho, vāseṭṭha, yeme dhammā akusalā akusalasaṅkhātā…pe… kaṇhā kaṇhavipākā viññugarahitā; suddepi te idhekacce sandissanti.

‘‘Khattiyopi kho, vāseṭṭha, idhekacco pāṇātipātā paṭivirato hoti, adinnādānā paṭivirato, kāmesumicchācārā paṭivirato, musāvādā paṭivirato, pisuṇāya vācāya paṭivirato, pharusāya vācāya paṭivirato, samphappalāpā paṭivirato, anabhijjhālu abyāpannacitto, sammādiṭṭhī. Iti kho, vāseṭṭha, yeme dhammā kusalā kusalasaṅkhātā anavajjā anavajjasaṅkhātā sevitabbā sevitabbasaṅkhātā alamariyā alamariyasaṅkhātā sukkā sukkavipākā viññuppasatthā, khattiyepi te idhekacce sandissanti. Brāhmaṇopi kho, vāseṭṭha…pe… vessopi kho, vāseṭṭha…pe… suddopi kho, vāseṭṭha, idhekacco pāṇātipātā paṭivirato hoti…pe… anabhijjhālu , abyāpannacitto, sammādiṭṭhī. Iti kho, vāseṭṭha, yeme dhammā kusalā kusalasaṅkhātā anavajjā anavajjasaṅkhātā sevitabbā sevitabbasaṅkhātā alamariyā alamariyasaṅkhātā sukkā sukkavipākā viññuppasatthā; suddepi te idhekacce sandissanti.

\paragraph{116.} ‘‘Imesu kho, vāseṭṭha, catūsu vaṇṇesu evaṃ ubhayavokiṇṇesu vattamānesu kaṇhasukkesu dhammesu viññugarahitesu ceva viññuppasatthesu ca yadettha brāhmaṇā evamāhaṃsu – ‘brāhmaṇova seṭṭho vaṇṇo, hīnā aññe vaṇṇā; brāhmaṇova sukko vaṇṇo, kaṇhā aññe vaṇṇā; brāhmaṇāva sujjhanti, no abrāhmaṇā; brāhmaṇāva brahmuno puttā orasā mukhato jātā brahmajā brahmanimmitā brahmadāyādā’ti. Taṃ tesaṃ viññū nānujānanti. Taṃ kissa hetu? Imesañhi, vāseṭṭha, catunnaṃ vaṇṇānaṃ yo hoti bhikkhu arahaṃ khīṇāsavo vusitavā katakaraṇīyo ohitabhāro anuppattasadattho parikkhīṇabhavasaṃyojano sammadaññāvimutto, so nesaṃ aggamakkhāyati dhammeneva, no adhammena. Dhammo hi, vāseṭṭha, seṭṭho janetasmiṃ, diṭṭhe ceva dhamme abhisamparāyañca .

\paragraph{117.} ‘‘Tadamināpetaṃ, vāseṭṭha, pariyāyena veditabbaṃ, yathā dhammova seṭṭho janetasmiṃ, diṭṭhe ceva dhamme abhisamparāyañca.

‘‘Jānāti kho\footnote{kho pana (ka.)}, vāseṭṭha, rājā pasenadi kosalo – ‘samaṇo gotamo anantarā\footnote{anuttaro (bahūsu)} sakyakulā pabbajito’ti. Sakyā kho pana, vāseṭṭha, rañño pasenadissa kosalassa anuyuttā\footnote{anantarā anuyantā (syā.), anantarā anuyuttā (ka.)} bhavanti. Karonti kho, vāseṭṭha, sakyā raññe pasenadimhi kosale nipaccakāraṃ abhivādanaṃ paccuṭṭhānaṃ añjalikammaṃ sāmīcikammaṃ. Iti kho, vāseṭṭha, yaṃ karonti sakyā raññe pasenadimhi kosale nipaccakāraṃ abhivādanaṃ paccuṭṭhānaṃ añjalikammaṃ sāmīcikammaṃ, karoti taṃ rājā pasenadi kosalo tathāgate nipaccakāraṃ abhivādanaṃ paccuṭṭhānaṃ añjalikammaṃ sāmīcikammaṃ, na naṃ\footnote{nanu (bahūsu)} ‘sujāto samaṇo gotamo, dujjātohamasmi. Balavā samaṇo gotamo, dubbalohamasmi. Pāsādiko samaṇo gotamo, dubbaṇṇohamasmi. Mahesakkho samaṇo gotamo, appesakkhohamasmī’ti. Atha kho naṃ dhammaṃyeva sakkaronto dhammaṃ garuṃ karonto dhammaṃ mānento dhammaṃ pūjento dhammaṃ apacāyamāno evaṃ rājā pasenadi kosalo tathāgate nipaccakāraṃ karoti, abhivādanaṃ paccuṭṭhānaṃ añjalikammaṃ sāmīcikammaṃ. Imināpi kho etaṃ, vāseṭṭha, pariyāyena veditabbaṃ, yathā dhammova seṭṭho janetasmiṃ, diṭṭhe ceva dhamme abhisamparāyañca.

\paragraph{118.} ‘‘Tumhe khvattha, vāseṭṭha, nānājaccā nānānāmā nānāgottā nānākulā agārasmā anagāriyaṃ pabbajitā. ‘Ke tumhe’ti – puṭṭhā samānā ‘samaṇā sakyaputtiyāmhā’ti – paṭijānātha. Yassa kho panassa, vāseṭṭha, tathāgate saddhā niviṭṭhā mūlajātā patiṭṭhitā daḷhā asaṃhāriyā samaṇena vā brāhmaṇena vā devena vā mārena vā brahmunā vā kenaci vā lokasmiṃ, tassetaṃ kallaṃ vacanāya – ‘bhagavatomhi putto oraso mukhato jāto dhammajo dhammanimmito dhammadāyādo’ti. Taṃ kissa hetu? Tathāgatassa hetaṃ, vāseṭṭha, adhivacanaṃ ‘dhammakāyo’ itipi, ‘brahmakāyo’ itipi, ‘dhammabhūto’ itipi, ‘brahmabhūto’ itipi.

\paragraph{119.} ‘‘Hoti kho so, vāseṭṭha, samayo yaṃ kadāci karahaci dīghassa addhuno accayena ayaṃ loko saṃvaṭṭati. Saṃvaṭṭamāne loke yebhuyyena sattā ābhassarasaṃvattanikā honti. Te tattha honti manomayā pītibhakkhā sayaṃpabhā antalikkhacarā subhaṭṭhāyino ciraṃ dīghamaddhānaṃ tiṭṭhanti.

‘‘Hoti kho so, vāseṭṭha, samayo yaṃ kadāci karahaci dīghassa addhuno accayena ayaṃ loko vivaṭṭati. Vivaṭṭamāne loke yebhuyyena sattā ābhassarakāyā cavitvā itthattaṃ āgacchanti. Tedha honti manomayā pītibhakkhā sayaṃpabhā antalikkhacarā subhaṭṭhāyino ciraṃ dīghamaddhānaṃ tiṭṭhanti.

\subsubsection{Rasapathavipātubhāvo}

\paragraph{120.} ‘‘Ekodakībhūtaṃ kho pana, vāseṭṭha, tena samayena hoti andhakāro andhakāratimisā . Na candimasūriyā paññāyanti, na nakkhattāni tārakarūpāni paññāyanti, na rattindivā paññāyanti, na māsaḍḍhamāsā paññāyanti, na utusaṃvaccharā paññāyanti , na itthipumā paññāyanti, sattā sattātveva saṅkhyaṃ gacchanti. Atha kho tesaṃ, vāseṭṭha, sattānaṃ kadāci karahaci dīghassa addhuno accayena rasapathavī udakasmiṃ samatani\footnote{samatāni (bahūsu)}; seyyathāpi nāma payaso tattassa\footnote{payatattassa (syā.)} nibbāyamānassa upari santānakaṃ hoti, evameva pāturahosi. Sā ahosi vaṇṇasampannā gandhasampannā rasasampannā, seyyathāpi nāma sampannaṃ vā sappi sampannaṃ vā navanītaṃ evaṃvaṇṇā ahosi. Seyyathāpi nāma khuddamadhuṃ\footnote{khuddaṃ madhuṃ (ka. sī.)} aneḷakaṃ\footnote{anelakaṃ (sī. pī.)}, evamassādā ahosi. Atha kho, vāseṭṭha, aññataro satto lolajātiko – ‘ambho, kimevidaṃ bhavissatī’ti rasapathaviṃ aṅguliyā sāyi. Tassa rasapathaviṃ aṅguliyā sāyato acchādesi, taṇhā cassa okkami. Aññepi kho, vāseṭṭha, sattā tassa sattassa diṭṭhānugatiṃ āpajjamānā rasapathaviṃ aṅguliyā sāyiṃsu. Tesaṃ rasapathaviṃ aṅguliyā sāyataṃ acchādesi, taṇhā ca tesaṃ okkami.

\subsubsection{Candimasūriyādipātubhāvo}

\paragraph{121.} ‘‘Atha kho te, vāseṭṭha, sattā rasapathaviṃ hatthehi āluppakārakaṃ upakkamiṃsu paribhuñjituṃ. Yato kho te\footnote{yato kho (sī. syā. pī.)}, vāseṭṭha, sattā rasapathaviṃ hatthehi āluppakārakaṃ upakkamiṃsu paribhuñjituṃ. Atha tesaṃ sattānaṃ sayaṃpabhā antaradhāyi. Sayaṃpabhāya antarahitāya candimasūriyā pāturahesuṃ. Candimasūriyesu pātubhūtesu nakkhattāni tārakarūpāni pāturahesuṃ. Nakkhattesu tārakarūpesu pātubhūtesu rattindivā paññāyiṃsu. Rattindivesu paññāyamānesu māsaḍḍhamāsā paññāyiṃsu. Māsaḍḍhamāsesu paññāyamānesu utusaṃvaccharā paññāyiṃsu. Ettāvatā kho , vāseṭṭha, ayaṃ loko puna vivaṭṭo hoti.

\paragraph{122.} ‘‘Atha kho te, vāseṭṭha, sattā rasapathaviṃ paribhuñjantā taṃbhakkhā\footnote{tabbhakkhā (syā.)} tadāhārā ciraṃ dīghamaddhānaṃ aṭṭhaṃsu. Yathā yathā kho te, vāseṭṭha, sattā rasapathaviṃ paribhuñjantā taṃbhakkhā tadāhārā ciraṃ dīghamaddhānaṃ aṭṭhaṃsu, tathā tathā tesaṃ sattānaṃ (rasapathaviṃ paribhuñjantānaṃ)\footnote{( ) sī. syā. pī. potthakesu natthi} kharattañceva kāyasmiṃ okkami, vaṇṇavevaṇṇatā\footnote{vaṇṇavevajjatā (ṭīkā)} ca paññāyittha. Ekidaṃ sattā vaṇṇavanto honti, ekidaṃ sattā dubbaṇṇā. Tattha ye te sattā vaṇṇavanto, te dubbaṇṇe satte atimaññanti – ‘mayametehi vaṇṇavantatarā, amhehete dubbaṇṇatarā’ti. Tesaṃ vaṇṇātimānapaccayā mānātimānajātikānaṃ rasapathavī antaradhāyi. Rasāya pathaviyā antarahitāya sannipatiṃsu. Sannipatitvā anutthuniṃsu – ‘aho rasaṃ, aho rasa’nti! Tadetarahipi manussā kañcideva surasaṃ\footnote{sādhurasaṃ (sī. syā. pī.)} labhitvā evamāhaṃsu – ‘aho rasaṃ, aho rasa’nti! Tadeva porāṇaṃ aggaññaṃ akkharaṃ anusaranti, na tvevassa atthaṃ ājānanti.

\subsubsection{Bhūmipappaṭakapātubhāvo}

\paragraph{123.} ‘‘Atha kho tesaṃ, vāseṭṭha, sattānaṃ rasāya pathaviyā antarahitāya bhūmipappaṭako pāturahosi. Seyyathāpi nāma ahicchattako, evameva pāturahosi. So ahosi vaṇṇasampanno gandhasampanno rasasampanno, seyyathāpi nāma sampannaṃ vā sappi sampannaṃ vā navanītaṃ evaṃvaṇṇo ahosi. Seyyathāpi nāma khuddamadhuṃ aneḷakaṃ, evamassādo ahosi.

‘‘Atha kho te, vāseṭṭha, sattā bhūmipappaṭakaṃ upakkamiṃsu paribhuñjituṃ. Te taṃ paribhuñjantā taṃbhakkhā tadāhārā ciraṃ dīghamaddhānaṃ aṭṭhaṃsu. Yathā yathā kho te, vāseṭṭha, sattā bhūmipappaṭakaṃ paribhuñjantā taṃbhakkhā tadāhārā ciraṃ dīghamaddhānaṃ aṭṭhaṃsu, tathā tathā tesaṃ sattānaṃ bhiyyoso mattāya kharattañceva kāyasmiṃ okkami, vaṇṇavevaṇṇatā ca paññāyittha. Ekidaṃ sattā vaṇṇavanto honti, ekidaṃ sattā dubbaṇṇā. Tattha ye te sattā vaṇṇavanto, te dubbaṇṇe satte atimaññanti – ‘mayametehi vaṇṇavantatarā, amhehete dubbaṇṇatarā’ti. Tesaṃ vaṇṇātimānapaccayā mānātimānajātikānaṃ bhūmipappaṭako antaradhāyi.

\subsubsection{Padālatāpātubhāvo}

\paragraph{124.} ‘‘Bhūmipappaṭake antarahite padālatā\footnote{saddālatā (sī.)} pāturahosi, seyyathāpi nāma kalambukā\footnote{kalambakā (syā.)}, evameva pāturahosi. Sā ahosi vaṇṇasampannā gandhasampannā rasasampannā, seyyathāpi nāma sampannaṃ vā sappi sampannaṃ vā navanītaṃ evaṃvaṇṇā ahosi. Seyyathāpi nāma khuddamadhuṃ aneḷakaṃ, evamassādā ahosi.

‘‘Atha kho te, vāseṭṭha, sattā padālataṃ upakkamiṃsu paribhuñjituṃ. Te taṃ paribhuñjantā taṃbhakkhā tadāhārā ciraṃ dīghamaddhānaṃ aṭṭhaṃsu. Yathā yathā kho te, vāseṭṭha, sattā padālataṃ paribhuñjantā taṃbhakkhā tadāhārā ciraṃ dīghamaddhānaṃ aṭṭhaṃsu, tathā tathā tesaṃ sattānaṃ bhiyyosomattāya kharattañceva kāyasmiṃ okkami, vaṇṇavevaṇṇatā ca paññāyittha. Ekidaṃ sattā vaṇṇavanto honti, ekidaṃ sattā dubbaṇṇā. Tattha ye te sattā vaṇṇavanto, te dubbaṇṇe satte atimaññanti – ‘mayametehi vaṇṇavantatarā, amhehete dubbaṇṇatarā’ti. Tesaṃ vaṇṇātimānapaccayā mānātimānajātikānaṃ padālatā antaradhāyi.

‘‘Padālatāya antarahitāya sannipatiṃsu. Sannipatitvā anutthuniṃsu – ‘ahu vata no, ahāyi vata no padālatā’ti! Tadetarahipi manussā kenaci\footnote{kenacideva (sī. syā. pī.)} dukkhadhammena phuṭṭhā evamāhaṃsu – ‘ahu vata no, ahāyi vata no’ti! Tadeva porāṇaṃ aggaññaṃ akkharaṃ anusaranti, na tvevassa atthaṃ ājānanti.

\subsubsection{Akaṭṭhapākasālipātubhāvo}

\paragraph{125.} ‘‘Atha kho tesaṃ, vāseṭṭha, sattānaṃ padālatāya antarahitāya akaṭṭhapāko sāli pāturahosi akaṇo athuso suddho sugandho taṇḍulapphalo. Yaṃ taṃ sāyaṃ sāyamāsāya āharanti, pāto taṃ hoti pakkaṃ paṭivirūḷhaṃ. Yaṃ taṃ pāto pātarāsāya āharanti, sāyaṃ taṃ hoti pakkaṃ paṭivirūḷhaṃ; nāpadānaṃ paññāyati. Atha kho te, vāseṭṭha, sattā akaṭṭhapākaṃ sāliṃ paribhuñjantā taṃbhakkhā tadāhārā ciraṃ dīghamaddhānaṃ aṭṭhaṃsu.

\subsubsection{Itthipurisaliṅgapātubhāvo}

\paragraph{126.} ‘‘Yathā yathā kho te, vāseṭṭha, sattā akaṭṭhapākaṃ sāliṃ paribhuñjantā taṃbhakkhā tadāhārā ciraṃ dīghamaddhānaṃ aṭṭhaṃsu, tathā tathā tesaṃ sattānaṃ bhiyyosomattāya kharattañceva kāyasmiṃ okkami, vaṇṇavevaṇṇatā ca paññāyittha, itthiyā ca itthiliṅgaṃ pāturahosi purisassa ca purisaliṅgaṃ. Itthī ca purisaṃ ativelaṃ upanijjhāyati puriso ca itthiṃ. Tesaṃ ativelaṃ aññamaññaṃ upanijjhāyataṃ sārāgo udapādi, pariḷāho kāyasmiṃ okkami. Te pariḷāhapaccayā methunaṃ dhammaṃ paṭiseviṃsu.

‘‘Ye kho pana te, vāseṭṭha, tena samayena sattā passanti methunaṃ dhammaṃ paṭisevante, aññe paṃsuṃ khipanti, aññe seṭṭhiṃ khipanti , aññe gomayaṃ khipanti – ‘nassa asuci\footnote{vasali (syā.), vasalī (ka.)}, nassa asucī’ti. ‘Kathañhi nāma satto sattassa evarūpaṃ karissatī’ti! Tadetarahipi manussā ekaccesu janapadesu vadhuyā nibbuyhamānāya\footnote{nivayhamānāya, niggayhamānāya (ka.)} aññe paṃsuṃ khipanti, aññe seṭṭhiṃ khipanti, aññe gomayaṃ khipanti. Tadeva porāṇaṃ aggaññaṃ akkharaṃ anusaranti, na tvevassa atthaṃ ājānanti.

\subsubsection{Methunadhammasamācāro}

\paragraph{127.} ‘‘Adhammasammataṃ kho pana\footnote{adhammasammataṃ taṃ kho pana (syā.), adhammasammataṃ kho pana taṃ (?)}, vāseṭṭha, tena samayena hoti, tadetarahi dhammasammataṃ. Ye kho pana, vāseṭṭha, tena samayena sattā methunaṃ dhammaṃ paṭisevanti, te māsampi dvemāsampi na labhanti gāmaṃ vā nigamaṃ vā pavisituṃ. Yato kho te, vāseṭṭha, sattā tasmiṃ asaddhamme ativelaṃ pātabyataṃ āpajjiṃsu. Atha agārāni upakkamiṃsu kātuṃ tasseva asaddhammassa paṭicchādanatthaṃ. Atha kho, vāseṭṭha, aññatarassa sattassa alasajātikassa etadahosi – ‘ambho, kimevāhaṃ vihaññāmi sāliṃ āharanto sāyaṃ sāyamāsāya pāto pātarāsāya! Yaṃnūnāhaṃ sāliṃ āhareyyaṃ sakiṃdeva\footnote{sakiṃdeva (ka.)} sāyapātarāsāyā’ti !

‘‘Atha kho so, vāseṭṭha, satto sāliṃ āhāsi sakiṃdeva sāyapātarāsāya. Atha kho, vāseṭṭha, aññataro satto yena so satto tenupasaṅkami; upasaṅkamitvā taṃ sattaṃ etadavoca – ‘ehi, bho satta, sālāhāraṃ gamissāmā’ti. ‘Alaṃ, bho satta, āhato\footnote{āhaṭo (syā.)} me sāli sakiṃdeva sāyapātarāsāyā’ti. Atha kho so, vāseṭṭha, satto tassa sattassa diṭṭhānugatiṃ āpajjamāno sāliṃ āhāsi sakiṃdeva dvīhāya. ‘Evampi kira, bho, sādhū’ti.

‘‘Atha kho, vāseṭṭha, aññataro satto yena so satto tenupasaṅkami; upasaṅkamitvā taṃ sattaṃ etadavoca – ‘ehi, bho satta, sālāhāraṃ gamissāmā’ti. ‘Alaṃ, bho satta, āhato me sāli sakiṃdeva dvīhāyā’ti. Atha kho so, vāseṭṭha, satto tassa sattassa diṭṭhānugatiṃ āpajjamāno sāliṃ āhāsi sakiṃdeva catūhāya, ‘evampi kira, bho, sādhū’ti.

‘‘Atha kho, vāseṭṭha, aññataro satto yena so satto tenupasaṅkami; upasaṅkamitvā taṃ sattaṃ etadavoca – ‘ehi, bho satta, sālāhāraṃ gamissāmā’ti. ‘Alaṃ, bho satta, āhato me sāli sakideva catūhāyā’ti. Atha kho so, vāseṭṭha, satto tassa sattassa diṭṭhānugatiṃ āpajjamāno sāliṃ āhāsi sakideva aṭṭhāhāya, ‘evampi kira, bho, sādhū’ti.

‘‘Yato kho te, vāseṭṭha, sattā sannidhikārakaṃ sāliṃ upakkamiṃsu paribhuñjituṃ. Atha kaṇopi taṇḍulaṃ pariyonandhi, thusopi taṇḍulaṃ pariyonandhi; lūnampi nappaṭivirūḷhaṃ , apadānaṃ paññāyittha, saṇḍasaṇḍā sālayo aṭṭhaṃsu.

\subsubsection{Sālivibhāgo}

\paragraph{128.} ‘‘Atha kho te, vāseṭṭha, sattā sannipatiṃsu, sannipatitvā anutthuniṃsu – ‘pāpakā vata, bho, dhammā sattesu pātubhūtā. Mayañhi pubbe manomayā ahumhā pītibhakkhā sayaṃpabhā antalikkhacarā subhaṭṭhāyino, ciraṃ dīghamaddhānaṃ aṭṭhamhā. Tesaṃ no amhākaṃ kadāci karahaci dīghassa addhuno accayena rasapathavī udakasmiṃ samatani. Sā ahosi vaṇṇasampannā gandhasampannā rasasampannā. Te mayaṃ rasapathaviṃ hatthehi āluppakārakaṃ upakkamimha paribhuñjituṃ, tesaṃ no rasapathaviṃ hatthehi āluppakārakaṃ upakkamataṃ paribhuñjituṃ sayaṃpabhā antaradhāyi. Sayaṃpabhāya antarahitāya candimasūriyā pāturahesuṃ, candimasūriyesu pātubhūtesu nakkhattāni tārakarūpāni pāturahesuṃ, nakkhattesu tārakarūpesu pātubhūtesu rattindivā paññāyiṃsu, rattindivesu paññāyamānesu māsaḍḍhamāsā paññāyiṃsu. Māsaḍḍhamāsesu paññāyamānesu utusaṃvaccharā paññāyiṃsu. Te mayaṃ rasapathaviṃ paribhuñjantā taṃbhakkhā tadāhārā ciraṃ dīghamaddhānaṃ aṭṭhamhā. Tesaṃ no pāpakānaṃyeva akusalānaṃ dhammānaṃ pātubhāvā rasapathavī antaradhāyi. Rasapathaviyā antarahitāya bhūmipappaṭako pāturahosi. So ahosi vaṇṇasampanno gandhasampanno rasasampanno. Te mayaṃ bhūmipappaṭakaṃ upakkamimha paribhuñjituṃ. Te mayaṃ taṃ paribhuñjantā taṃbhakkhā tadāhārā ciraṃ dīghamaddhānaṃ aṭṭhamhā. Tesaṃ no pāpakānaṃyeva akusalānaṃ dhammānaṃ pātubhāvā bhūmipappaṭako antaradhāyi. Bhūmipappaṭake antarahite padālatā pāturahosi. Sā ahosi vaṇṇasampannā gandhasampannā rasasampannā. Te mayaṃ padālataṃ upakkamimha paribhuñjituṃ. Te mayaṃ taṃ paribhuñjantā taṃbhakkhā tadāhārā ciraṃ dīghamaddhānaṃ aṭṭhamhā. Tesaṃ no pāpakānaṃyeva akusalānaṃ dhammānaṃ pātubhāvā padālatā antaradhāyi. Padālatāya antarahitāya akaṭṭhapāko sāli pāturahosi akaṇo athuso suddho sugandho taṇḍulapphalo. Yaṃ taṃ sāyaṃ sāyamāsāya āharāma, pāto taṃ hoti pakkaṃ paṭivirūḷhaṃ. Yaṃ taṃ pāto pātarāsāya āharāma, sāyaṃ taṃ hoti pakkaṃ paṭivirūḷhaṃ. Nāpadānaṃ paññāyittha. Te mayaṃ akaṭṭhapākaṃ sāliṃ paribhuñjantā taṃbhakkhā tadāhārā ciraṃ dīghamaddhānaṃ aṭṭhamhā. Tesaṃ no pāpakānaṃyeva akusalānaṃ dhammānaṃ pātubhāvā kaṇopi taṇḍulaṃ pariyonandhi, thusopi taṇḍulaṃ pariyonandhi, lūnampi nappaṭivirūḷhaṃ, apadānaṃ paññāyittha, saṇḍasaṇḍā sālayo ṭhitā. Yaṃnūna mayaṃ sāliṃ vibhajeyyāma, mariyādaṃ ṭhapeyyāmā’ti! Atha kho te, vāseṭṭha, sattā sāliṃ vibhajiṃsu, mariyādaṃ ṭhapesuṃ.

\paragraph{129.} ‘‘Atha kho, vāseṭṭha, aññataro satto lolajātiko sakaṃ bhāgaṃ parirakkhanto aññataraṃ\footnote{aññassa (?)} bhāgaṃ adinnaṃ ādiyitvā paribhuñji. Tamenaṃ aggahesuṃ, gahetvā etadavocuṃ – ‘pāpakaṃ vata, bho satta, karosi, yatra hi nāma sakaṃ bhāgaṃ parirakkhanto aññataraṃ bhāgaṃ adinnaṃ ādiyitvā paribhuñjasi. Māssu, bho satta, punapi evarūpamakāsī’ti. ‘Evaṃ, bho’ti kho, vāseṭṭha, so satto tesaṃ sattānaṃ paccassosi. Dutiyampi kho, vāseṭṭha, so satto…pe… tatiyampi kho, vāseṭṭha, so satto sakaṃ bhāgaṃ parirakkhanto aññataraṃ bhāgaṃ adinnaṃ ādiyitvā paribhuñji. Tamenaṃ aggahesuṃ, gahetvā etadavocuṃ – ‘pāpakaṃ vata, bho satta, karosi, yatra hi nāma sakaṃ bhāgaṃ parirakkhanto aññataraṃ bhāgaṃ adinnaṃ ādiyitvā paribhuñjasi. Māssu, bho satta, punapi evarūpamakāsī’ti. Aññe pāṇinā pahariṃsu, aññe leḍḍunā pahariṃsu, aññe daṇḍena pahariṃsu. Tadagge kho, vāseṭṭha, adinnādānaṃ paññāyati, garahā paññāyati, musāvādo paññāyati, daṇḍādānaṃ paññāyati.

\subsubsection{Mahāsammatarājā}

\paragraph{130.} ‘‘Atha kho te, vāseṭṭha, sattā sannipatiṃsu, sannipatitvā anutthuniṃsu – ‘pāpakā vata bho dhammā sattesu pātubhūtā, yatra hi nāma adinnādānaṃ paññāyissati, garahā paññāyissati, musāvādo paññāyissati, daṇḍādānaṃ paññāyissati. Yaṃnūna mayaṃ ekaṃ sattaṃ sammanneyyāma, yo no sammā khīyitabbaṃ khīyeyya, sammā garahitabbaṃ garaheyya, sammā pabbājetabbaṃ pabbājeyya. Mayaṃ panassa sālīnaṃ bhāgaṃ anuppadassāmā’ti.

‘‘Atha kho te, vāseṭṭha, sattā yo nesaṃ satto abhirūpataro ca dassanīyataro ca pāsādikataro ca mahesakkhataro ca taṃ sattaṃ upasaṅkamitvā etadavocuṃ – ‘ehi, bho satta, sammā khīyitabbaṃ khīya, sammā garahitabbaṃ garaha, sammā pabbājetabbaṃ pabbājehi. Mayaṃ pana te sālīnaṃ bhāgaṃ anuppadassāmā’ti. ‘Evaṃ, bho’ti kho, vāseṭṭha, so satto tesaṃ sattānaṃ paṭissuṇitvā sammā khīyitabbaṃ khīyi, sammā garahitabbaṃ garahi, sammā pabbājetabbaṃ pabbājesi. Te panassa sālīnaṃ bhāgaṃ anuppadaṃsu.

\paragraph{131.} ‘‘Mahājanasammatoti kho, vāseṭṭha, ‘mahāsammato, mahāsammato’ tveva paṭhamaṃ akkharaṃ upanibbattaṃ. Khettānaṃ adhipatīti kho, vāseṭṭha, ‘khattiyo, khattiyo’ tveva dutiyaṃ akkharaṃ upanibbattaṃ. Dhammena pare rañjetīti kho, vāseṭṭha, ‘rājā, rājā’ tveva tatiyaṃ akkharaṃ upanibbattaṃ. Iti kho, vāseṭṭha, evametassa khattiyamaṇḍalassa porāṇena aggaññena akkharena abhinibbatti ahosi tesaṃyeva sattānaṃ, anaññesaṃ. Sadisānaṃyeva, no asadisānaṃ. Dhammeneva, no adhammena. Dhammo hi, vāseṭṭha, seṭṭho janetasmiṃ diṭṭhe ceva dhamme abhisamparāyañca.

\subsubsection{Brāhmaṇamaṇḍalaṃ}

\paragraph{132.} ‘‘Atha kho tesaṃ, vāseṭṭha, sattānaṃyeva\footnote{tesaṃ yeva kho vāseṭṭha sattānaṃ (sī. pī.)} ekaccānaṃ etadahosi – ‘pāpakā vata, bho, dhammā sattesu pātubhūtā, yatra hi nāma adinnādānaṃ paññāyissati, garahā paññāyissati, musāvādo paññāyissati, daṇḍādānaṃ paññāyissati, pabbājanaṃ paññāyissati. Yaṃnūna mayaṃ pāpake akusale dhamme vāheyyāmā’ti. Te pāpake akusale dhamme vāhesuṃ . Pāpake akusale dhamme vāhentīti kho, vāseṭṭha, ‘brāhmaṇā, brāhmaṇā’ tveva paṭhamaṃ akkharaṃ upanibbattaṃ. Te araññāyatane paṇṇakuṭiyo karitvā paṇṇakuṭīsu jhāyanti vītaṅgārā vītadhūmā pannamusalā sāyaṃ sāyamāsāya pāto pātarāsāya gāmanigamarājadhāniyo osaranti ghāsamesamānā\footnote{ghāsamesanā (sī. syā. pī.)}. Te ghāsaṃ paṭilabhitvā punadeva araññāyatane paṇṇakuṭīsu jhāyanti. Tamenaṃ manussā disvā evamāhaṃsu – ‘ime kho, bho, sattā araññāyatane paṇṇakuṭiyo karitvā paṇṇakuṭīsu jhāyanti, vītaṅgārā vītadhūmā pannamusalā sāyaṃ sāyamāsāya pāto pātarāsāya gāmanigamarājadhāniyo osaranti ghāsamesamānā. Te ghāsaṃ paṭilabhitvā punadeva araññāyatane paṇṇakuṭīsu jhāyantī’ti, jhāyantīti kho\footnote{paṇṇakuṭīsu jhāyanti jhāyantīti kho (sī. pī.), paṇṇakuṭīsu jhāyantīti kho (ka.)}, vāseṭṭha, ‘jhāyakā, jhāyakā’ tveva dutiyaṃ akkharaṃ upanibbattaṃ. Tesaṃyeva kho, vāseṭṭha, sattānaṃ ekacce sattā araññāyatane paṇṇakuṭīsu taṃ jhānaṃ anabhisambhuṇamānā\footnote{anabhisaṃbhūnamānā (katthaci)} gāmasāmantaṃ nigamasāmantaṃ osaritvā ganthe karontā acchanti. Tamenaṃ manussā disvā evamāhaṃsu – ‘ime kho, bho, sattā araññāyatane paṇṇakuṭīsu taṃ jhānaṃ anabhisambhuṇamānā gāmasāmantaṃ nigamasāmantaṃ osaritvā ganthe karontā acchanti, na dānime jhāyantī’ti. Na dānime\footnote{na dānime jhāyantī na dānime (sī. pī. ka.)} jhāyantīti kho, vāseṭṭha, ‘ajjhāyakā ajjhāyakā’ tveva tatiyaṃ akkharaṃ upanibbattaṃ. Hīnasammataṃ kho pana, vāseṭṭha, tena samayena hoti, tadetarahi seṭṭhasammataṃ. Iti kho, vāseṭṭha, evametassa brāhmaṇamaṇḍalassa porāṇena aggaññena akkharena abhinibbatti ahosi tesaṃyeva sattānaṃ , anaññesaṃ sadisānaṃyeva no asadisānaṃ dhammeneva , no adhammena. Dhammo hi, vāseṭṭha, seṭṭho janetasmiṃ diṭṭhe ceva dhamme abhisamparāyañca.

\subsubsection{Vessamaṇḍalaṃ}

\paragraph{133.} ‘‘Tesaṃyeva kho, vāseṭṭha, sattānaṃ ekacce sattā methunaṃ dhammaṃ samādāya visukammante\footnote{vissutakammante (sī. pī.), vissukammante (ka. sī.), visuṃ kammante (syā. ka.)} payojesuṃ. Methunaṃ dhammaṃ samādāya visukammante payojentīti kho, vāseṭṭha, ‘vessā, vessā’ tveva akkharaṃ upanibbattaṃ. Iti kho, vāseṭṭha, evametassa vessamaṇḍalassa porāṇena aggaññena akkharena abhinibbatti ahosi tesaññeva sattānaṃ anaññesaṃ sadisānaṃyeva , no asadisānaṃ, dhammeneva no adhammena. Dhammo hi, vāseṭṭha, seṭṭho janetasmiṃ diṭṭhe ceva dhamme abhisamparāyañca.

\subsubsection{Suddamaṇḍalaṃ}

\paragraph{134.} ‘‘Tesaññeva kho, vāseṭṭha, sattānaṃ ye te sattā avasesā te luddācārā khuddācārā ahesuṃ. Luddācārā khuddācārāti kho, vāseṭṭha, ‘suddā, suddā’ tveva akkharaṃ upanibbattaṃ. Iti kho, vāseṭṭha, evametassa suddamaṇḍalassa porāṇena aggaññena akkharena abhinibbatti ahosi tesaṃyeva sattānaṃ anaññesaṃ, sadisānaṃyeva no asadisānaṃ, dhammeneva, no adhammena. Dhammo hi, vāseṭṭha, seṭṭho janetasmiṃ diṭṭhe ceva dhamme abhisamparāyañca.

\paragraph{135.} ‘‘Ahu kho so, vāseṭṭha, samayo, yaṃ khattiyopi sakaṃ dhammaṃ garahamāno agārasmā anagāriyaṃ pabbajati – ‘samaṇo bhavissāmī’ti. Brāhmaṇopi kho, vāseṭṭha…pe… vessopi kho, vāseṭṭha…pe… suddopi kho, vāseṭṭha, sakaṃ dhammaṃ garahamāno agārasmā anagāriyaṃ pabbajati – ‘samaṇo bhavissāmī’ti. Imehi kho, vāseṭṭha, catūhi maṇḍalehi samaṇamaṇḍalassa abhinibbatti ahosi, tesaṃyeva sattānaṃ anaññesaṃ, sadisānaṃyeva no asadisānaṃ, dhammeneva no adhammena. Dhammo hi, vāseṭṭha, seṭṭho janetasmiṃ diṭṭhe ceva dhamme abhisamparāyañca.

\subsubsection{Duccaritādikathā}

\paragraph{136.} ‘‘Khattiyopi kho, vāseṭṭha, kāyena duccaritaṃ caritvā vācāya duccaritaṃ caritvā manasā duccaritaṃ caritvā micchādiṭṭhiko micchādiṭṭhikammasamādāno\footnote{idaṃ padaṃ sī. ipotthakesu natthi} micchādiṭṭhikammasamādānahetu kāyassa bhedā paraṃ maraṇā apāyaṃ duggatiṃ vinipātaṃ nirayaṃ upapajjati. Brāhmaṇopi kho, vāseṭṭha…pe… vessopi kho, vāseṭṭha… suddopi kho, vāseṭṭha… samaṇopi kho, vāseṭṭha, kāyena duccaritaṃ caritvā vācāya duccaritaṃ caritvā manasā duccaritaṃ caritvā micchādiṭṭhiko micchādiṭṭhikammasamādāno micchādiṭṭhikammasamādānahetu kāyassa bhedā paraṃ maraṇā apāyaṃ duggatiṃ vinipātaṃ nirayaṃ upapajjati.

‘‘Khattiyopi kho, vāseṭṭha, kāyena sucaritaṃ caritvā vācāya sucaritaṃ caritvā manasā sucaritaṃ caritvā sammādiṭṭhiko sammādiṭṭhikammasamādāno\footnote{idaṃ padaṃ sī. pī. potthakesu natthi} sammādiṭṭhikammasamādānahetu kāyassa bhedā paraṃ maraṇā sugatiṃ saggaṃ lokaṃ upapajjati. Brāhmaṇopi kho, vāseṭṭha…pe… vessopi kho, vāseṭṭha… suddopi kho, vāseṭṭha… samaṇopi kho, vāseṭṭha, kāyena sucaritaṃ caritvā vācāya sucaritaṃ caritvā manasā sucaritaṃ caritvā sammādiṭṭhiko sammādiṭṭhikammasamādāno sammādiṭṭhikammasamādānahetu kāyassa bhedā paraṃ maraṇā sugatiṃ saggaṃ lokaṃ upapajjati.

\paragraph{137.} ‘‘Khattiyopi kho, vāseṭṭha, kāyena dvayakārī, vācāya dvayakārī, manasā dvayakārī, vimissadiṭṭhiko vimissadiṭṭhikammasamādāno vimissadiṭṭhikammasamādānahetu\footnote{vimissadiṭṭhiko vimissakammasamādāno vimissakammasamādānahetu (syā.), vītimissadiṭṭhiko vītimissadiṭṭhikammasamādānahetu (sī. pī.)} kāyassa bhedā paraṃ maraṇā sukhadukkhappaṭisaṃvedī hoti. Brāhmaṇopi kho, vāseṭṭha …pe… vessopi kho, vāseṭṭha… suddopi kho, vāseṭṭha… samaṇopi kho, vāseṭṭha, kāyena dvayakārī , vācāya dvayakārī, manasā dvayakārī, vimissadiṭṭhiko vimissadiṭṭhikammasamādāno vimissadiṭṭhikammasamādānahetu kāyassa bhedā paraṃ maraṇā sukhadukkhappaṭisaṃvedī hoti.

\subsubsection{Bodhipakkhiyabhāvanā}

\paragraph{138.} ‘‘Khattiyopi kho, vāseṭṭha, kāyena saṃvuto vācāya saṃvuto manasā saṃvuto sattannaṃ bodhipakkhiyānaṃ dhammānaṃ bhāvanamanvāya diṭṭheva dhamme parinibbāyati\footnote{parinibbāti (ka.)}. Brāhmaṇopi kho, vāseṭṭha…pe… vessopi kho vāseṭṭha… suddopi kho, vāseṭṭha … samaṇopi kho, vāseṭṭha, kāyena saṃvuto vācāya saṃvuto manasā saṃvuto sattannaṃ bodhipakkhiyānaṃ dhammānaṃ bhāvanamanvāya diṭṭheva dhamme parinibbāyati.

\paragraph{139.} ‘‘Imesañhi, vāseṭṭha, catunnaṃ vaṇṇānaṃ yo hoti bhikkhu arahaṃ khīṇāsavo vusitavā katakaraṇīyo ohitabhāro anuppattasadattho parikkhīṇabhavasaṃyojano sammadaññā vimutto so nesaṃ aggamakkhāyati dhammeneva. No adhammena. Dhammo hi, vāseṭṭha, seṭṭho janetasmiṃ diṭṭhe ceva dhamme abhisamparāyañca.

\paragraph{140.} ‘‘Brahmunā pesā, vāseṭṭha, sanaṅkumārena gāthā bhāsitā –

‘Khattiyo seṭṭho janetasmiṃ, ye gottapaṭisārino;

Vijjācaraṇasampanno, so seṭṭho devamānuse’ti.

‘‘Sā kho panesā, vāseṭṭha, brahmunā sanaṅkumārena gāthā sugītā, no duggītā. Subhāsitā, no dubbhāsitā. Atthasaṃhitā, no anatthasaṃhitā. Anumatā mayā. Ahampi, vāseṭṭha, evaṃ vadāmi –

‘Khattiyo seṭṭho janetasmiṃ, ye gottapaṭisārino;

Vijjācaraṇasampanno, so seṭṭho devamānuse’ti.

Idamavoca bhagavā. Attamanā vāseṭṭhabhāradvājā bhagavato bhāsitaṃ abhinandunti.

\xsectionEnd{Aggaññasuttaṃ niṭṭhitaṃ catutthaṃ.}
