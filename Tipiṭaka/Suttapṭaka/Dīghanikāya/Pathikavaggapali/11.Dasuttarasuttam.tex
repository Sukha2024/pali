\section{Dasuttarasuttaṃ}

\paragraph{350.} Evaṃ me sutaṃ – ekaṃ samayaṃ bhagavā campāyaṃ viharati gaggarāya pokkharaṇiyā tīre mahatā bhikkhusaṅghena saddhiṃ pañcamattehi bhikkhusatehi. Tatra kho āyasmā sāriputto bhikkhū āmantesi – ‘‘āvuso bhikkhave’’ti! ‘‘Āvuso’’ti kho te bhikkhū āyasmato sāriputtassa paccassosuṃ. Āyasmā sāriputto etadavoca –

‘‘Dasuttaraṃ pavakkhāmi, dhammaṃ nibbānapattiyā;

Dukkhassantakiriyāya, sabbaganthappamocanaṃ’’.

\subsubsection{Eko dhammo}

\paragraph{351.} ‘‘Eko, āvuso, dhammo bahukāro, eko dhammo bhāvetabbo, eko dhammo pariññeyyo, eko dhammo pahātabbo, eko dhammo hānabhāgiyo, eko dhammo visesabhāgiyo, eko dhammo duppaṭivijjho, eko dhammo uppādetabbo, eko dhammo abhiññeyyo, eko dhammo sacchikātabbo.

(Ka) ‘‘katamo eko dhammo bahukāro? Appamādo kusalesu dhammesu. Ayaṃ eko dhammo bahukāro.

(Kha) ‘‘katamo eko dhammo bhāvetabbo? Kāyagatāsati sātasahagatā. Ayaṃ eko dhammo bhāvetabbo.

(Ga) ‘‘katamo eko dhammo pariññeyyo? Phasso sāsavo upādāniyo. Ayaṃ eko dhammo pariññeyyo.

(Gha) ‘‘katamo eko dhammo pahātabbo? Asmimāno. Ayaṃ eko dhammo pahātabbo.

(Ṅa) ‘‘katamo eko dhammo hānabhāgiyo? Ayoniso manasikāro. Ayaṃ eko dhammo hānabhāgiyo.

(Ca) ‘‘katamo eko dhammo visesabhāgiyo? Yoniso manasikāro. Ayaṃ eko dhammo visesabhāgiyo.

(Cha) ‘‘katamo eko dhammo duppaṭivijjho? Ānantariko cetosamādhi. Ayaṃ eko dhammo duppaṭivijjho.

(Ja) ‘‘katamo eko dhammo uppādetabbo? Akuppaṃ ñāṇaṃ. Ayaṃ eko dhammo uppādetabbo.

(Jha) ‘‘katamo eko dhammo abhiññeyyo? Sabbe sattā āhāraṭṭhitikā. Ayaṃ eko dhammo abhiññeyyo.

(Ña) ‘‘katamo eko dhammo sacchikātabbo? Akuppā cetovimutti. Ayaṃ eko dhammo sacchikātabbo.

‘‘Iti ime dasa dhammā bhūtā tacchā tathā avitathā anaññathā sammā tathāgatena abhisambuddhā.

\subsubsection{Dve dhammā}

\paragraph{352.} ‘‘Dve dhammā bahukārā, dve dhammā bhāvetabbā, dve dhammā pariññeyyā, dve dhammā pahātabbā , dve dhammā hānabhāgiyā, dve dhammā visesabhāgiyā, dve dhammā duppaṭivijjhā, dve dhammā uppādetabbā, dve dhammā abhiññeyyā, dve dhammā sacchikātabbā.

(Ka) ‘‘katame dve dhammā bahukārā? Sati ca sampajaññañca. Ime dve dhammā bahukārā.

(Kha) ‘‘katame dve dhammā bhāvetabbā? Samatho ca vipassanā ca. Ime dve dhammā bhāvetabbā.

(Ga) ‘‘katame dve dhammā pariññeyyā? Nāmañca rūpañca. Ime dve dhammā pariññeyyā.

(Gha) ‘‘katame dve dhammā pahātabbā? Avijjā ca bhavataṇhā ca. Ime dve dhammā pahātabbā.

(Ṅa) ‘‘katame dve dhammā hānabhāgiyā? Dovacassatā ca pāpamittatā ca. Ime dve dhammā hānabhāgiyā.

(Ca) ‘‘katame dve dhammā visesabhāgiyā? Sovacassatā ca kalyāṇamittatā ca. Ime dve dhammā visesabhāgiyā.

(Cha) ‘‘katame dve dhammā duppaṭivijjhā? Yo ca hetu yo ca paccayo sattānaṃ saṃkilesāya, yo ca hetu yo ca paccayo sattānaṃ visuddhiyā. Ime dve dhammā duppaṭivijjhā.

(Ja) ‘‘katame dve dhammā uppādetabbā? Dve ñāṇāni – khaye ñāṇaṃ, anuppāde ñāṇaṃ. Ime dve dhammā uppādetabbā.

(Jha) ‘‘katame dve dhammā abhiññeyyā? Dve dhātuyo – saṅkhatā ca dhātu asaṅkhatā ca dhātu. Ime dve dhammā abhiññeyyā.

(Ña) ‘‘katame dve dhammā sacchikātabbā? Vijjā ca vimutti ca. Ime dve dhammā sacchikātabbā.

‘‘Iti ime vīsati dhammā bhūtā tacchā tathā avitathā anaññathā sammā tathāgatena abhisambuddhā.

\subsubsection{Tayo dhammā}

\paragraph{353.} ‘‘Tayo dhammā bahukārā, tayo dhammā bhāvetabbā…pe… tayo dhammā sacchikātabbā.

(Ka) ‘‘katame tayo dhammā bahukārā? Sappurisasaṃsevo, saddhammassavanaṃ, dhammānudhammappaṭipatti. Ime tayo dhammā bahukārā.

(Kha) ‘‘katame tayo dhammā bhāvetabbā? Tayo samādhī – savitakko savicāro samādhi, avitakko vicāramatto samādhi, avitakko avicāro samādhi. Ime tayo dhammā bhāvetabbā.

(Ga) ‘‘katame tayo dhammā pariññeyyā? Tisso vedanā – sukhā vedanā, dukkhā vedanā, adukkhamasukhā vedanā. Ime tayo dhammā pariññeyyā.

(Gha) ‘‘katame tayo dhammā pahātabbā? Tisso taṇhā – kāmataṇhā, bhavataṇhā, vibhavataṇhā. Ime tayo dhammā pahātabbā.

(Ṅa) ‘‘katame tayo dhammā hānabhāgiyā? Tīṇi akusalamūlāni – lobho akusalamūlaṃ, doso akusalamūlaṃ, moho akusalamūlaṃ. Ime tayo dhammā hānabhāgiyā.

(Ca) ‘‘katame tayo dhammā visesabhāgiyā? Tīṇi kusalamūlāni – alobho kusalamūlaṃ, adoso kusalamūlaṃ, amoho kusalamūlaṃ. Ime tayo dhammā visesabhāgiyā.

(Cha) ‘‘katame tayo dhammā duppaṭivijjhā? Tisso nissaraṇiyā dhātuyo – kāmānametaṃ nissaraṇaṃ yadidaṃ nekkhammaṃ, rūpānametaṃ nissaraṇaṃ yadidaṃ arūpaṃ, yaṃ kho pana kiñci bhūtaṃ saṅkhataṃ paṭiccasamuppannaṃ, nirodho tassa nissaraṇaṃ. Ime tayo dhammā duppaṭivijjhā.

(Ja) ‘‘katame tayo dhammā uppādetabbā? Tīṇi ñāṇāni – atītaṃse ñāṇaṃ, anāgataṃse ñāṇaṃ, paccuppannaṃse ñāṇaṃ. Ime tayo dhammā uppādetabbā.

(Jha) ‘‘katame tayo dhammā abhiññeyyā? Tisso dhātuyo – kāmadhātu, rūpadhātu, arūpadhātu. Ime tayo dhammā abhiññeyyā.

(Ña) ‘‘katame tayo dhammā sacchikātabbā? Tisso vijjā – pubbenivāsānussatiñāṇaṃ vijjā, sattānaṃ cutūpapāte ñāṇaṃ vijjā, āsavānaṃ khaye ñāṇaṃ vijjā. Ime tayo dhammā sacchikātabbā.

‘‘Iti ime tiṃsa dhammā bhūtā tacchā tathā avitathā anaññathā sammā tathāgatena abhisambuddhā.

\subsubsection{Cattāro dhammā}

\paragraph{354.} ‘‘Cattāro dhammā bahukārā, cattāro dhammā bhāvetabbā…pe… cattāro dhammā sacchikātabbā.

(Ka) ‘‘katame cattāro dhammā bahukārā? Cattāri cakkāni – patirūpadesavāso, sappurisūpanissayo\footnote{sappurisupassayo (syā. kaṃ.)}, attasammāpaṇidhi, pubbe ca katapuññatā. Ime cattāro dhammā bahukārā.

(Kha) ‘‘katame cattāro dhammā bhāvetabbā? Cattāro satipaṭṭhānā – idhāvuso, bhikkhu kāye kāyānupassī viharati ātāpī sampajāno satimā vineyya loke abhijjhādomanassaṃ. Vedanāsu…pe… citte… dhammesu dhammānupassī viharati ātāpī sampajāno satimā vineyya loke abhijjhādomanassaṃ. Ime cattāro dhammā bhāvetabbā.

(Ga) ‘‘katame cattāro dhammā pariññeyyā? Cattāro āhārā – kabaḷīkāro\footnote{kavaḷīkāro (syā. kaṃ.)} āhāro oḷāriko vā sukhumo vā, phasso dutiyo, manosañcetanā tatiyā, viññāṇaṃ catutthaṃ. Ime cattāro dhammā pariññeyyā.

(Gha) ‘‘katame cattāro dhammā pahātabbā? Cattāro oghā – kāmogho, bhavogho, diṭṭhogho, avijjogho. Ime cattāro dhammā pahātabbā.

(Ṅa) ‘‘katame cattāro dhammā hānabhāgiyā? Cattāro yogā – kāmayogo, bhavayogo, diṭṭhiyogo, avijjāyogo. Ime cattāro dhammā hānabhāgiyā.

(Ca) ‘‘katame cattāro dhammā visesabhāgiyā? Cattāro visaññogā – kāmayogavisaṃyogo, bhavayogavisaṃyogo, diṭṭhiyogavisaṃyogo, avijjāyogavisaṃyogo. Ime cattāro dhammā visesabhāgiyā.

(Cha) ‘‘katame cattāro dhammā duppaṭivijjhā? Cattāro samādhī – hānabhāgiyo samādhi, ṭhitibhāgiyo samādhi, visesabhāgiyo samādhi, nibbedhabhāgiyo samādhi. Ime cattāro dhammā duppaṭivijjhā.

(Ja) ‘‘katame cattāro dhammā uppādetabbā? Cattāri ñāṇāni – dhamme ñāṇaṃ, anvaye ñāṇaṃ, pariye ñāṇaṃ, sammutiyā ñāṇaṃ. Ime cattāro dhammā uppādetabbā.

(Jha) ‘‘katame cattāro dhammā abhiññeyyā? Cattāri ariyasaccāni – dukkhaṃ ariyasaccaṃ, dukkhasamudayaṃ\footnote{dukkhasamudayo (syā. kaṃ.)} ariyasaccaṃ, dukkhanirodhaṃ\footnote{dukkhanirodho (syā. kaṃ.)} ariyasaccaṃ, dukkhanirodhagāminī paṭipadā ariyasaccaṃ. Ime cattāro dhammā abhiññeyyā.

(Ña) ‘‘katame cattāro dhammā sacchikātabbā? Cattāri sāmaññaphalāni – sotāpattiphalaṃ, sakadāgāmiphalaṃ, anāgāmiphalaṃ, arahattaphalaṃ . Ime cattāro dhammā sacchikātabbā.

‘‘Iti ime cattārīsadhammā bhūtā tacchā tathā avitathā anaññathā sammā tathāgatena abhisambuddhā.

\subsubsection{Pañca dhammā}

\paragraph{355.} ‘‘Pañca dhammā bahukārā…pe… pañca dhammā sacchikātabbā.

(Ka) ‘‘katame pañca dhammā bahukārā? Pañca padhāniyaṅgāni – idhāvuso, bhikkhu saddho hoti, saddahati tathāgatassa bodhiṃ – ‘itipi so bhagavā arahaṃ sammāsambuddho vijjācaraṇasampanno sugato lokavidū anuttaro purisadammasārathi satthā devamanussānaṃ buddho bhagavā’ti. Appābādho hoti appātaṅko samavepākiniyā gahaṇiyā samannāgato nātisītāya nāccuṇhāya majjhimāya padhānakkhamāya. Asaṭho hoti amāyāvī yathābhūtamattānaṃ āvīkattā satthari vā viññūsu vā sabrahmacārīsu. Āraddhavīriyo viharati akusalānaṃ dhammānaṃ pahānāya, kusalānaṃ dhammānaṃ upasampadāya, thāmavā daḷhaparakkamo anikkhittadhuro kusalesu dhammesu. Paññavā hoti udayatthagāminiyā paññāya samannāgato ariyāya nibbedhikāya sammā dukkhakkhayagāminiyā. Ime pañca dhammā bahukārā.

(Kha) ‘‘katame pañca dhammā bhāvetabbā? Pañcaṅgiko sammāsamādhi – pītipharaṇatā, sukhapharaṇatā, cetopharaṇatā , ālokapharaṇatā, paccavekkhaṇanimittaṃ\footnote{paccavekkhaṇānimittaṃ (syā. kaṃ.)}. Ime pañca dhammā bhāvetabbā.

(Ga) ‘‘katame pañca dhammā pariññeyyā? Pañcupādānakkhandhā\footnote{seyyathīdaṃ (sī. syā. kaṃ. pī.)} – rūpupādānakkhandho, vedanupādānakkhandho, saññupādānakkhandho, saṅkhārupādānakkhandho viññāṇupādānakkhandho. Ime pañca dhammā pariññeyyā.

(Gha) ‘‘katame pañca dhammā pahātabbā? Pañca nīvaraṇāni – kāmacchandanīvaraṇaṃ, byāpādanīvaraṇaṃ, thinamiddhanīvaraṇaṃ, uddhaccakukuccanīvaraṇaṃ, vicikicchānīvaraṇaṃ. Ime pañca dhammā pahātabbā.

(Ṅa) ‘‘katame pañca dhammā hānabhāgiyā? Pañca cetokhilā – idhāvuso, bhikkhu satthari kaṅkhati vicikicchati nādhimuccati na sampasīdati. Yo so, āvuso, bhikkhu satthari kaṅkhati vicikicchati nādhimuccati na sampasīdati, tassa cittaṃ na namati ātappāya anuyogāya sātaccāya padhānāya. Yassa cittaṃ na namati ātappāya anuyogāya sātaccāya padhānāya . Ayaṃ paṭhamo cetokhilo. Puna caparaṃ, āvuso, bhikkhu dhamme kaṅkhati vicikicchati…pe… saṅghe kaṅkhati vicikicchati…pe… sikkhāya kaṅkhati vicikicchati…pe… sabrahmacārīsu kupito hoti anattamano āhatacitto khilajāto, yo so, āvuso, bhikkhu sabrahmacārīsu kupito hoti anattamano āhatacitto khilajāto, tassa cittaṃ na namati ātappāya anuyogāya sātaccāya padhānāya. Yassa cittaṃ na namati ātappāya anuyogāya sātaccāya padhānāya. Ayaṃ pañcamo cetokhilo. Ime pañca dhammā hānabhāgiyā.

(Ca) ‘‘katame pañca dhammā visesabhāgiyā? Pañcindriyāni – saddhindriyaṃ, vīriyindriyaṃ, satindriyaṃ, samādhindriyaṃ, paññindriyaṃ. Ime pañca dhammā visesabhāgiyā.

(Cha) ‘‘katame pañca dhammā duppaṭivijjhā? Pañca nissaraṇiyā dhātuyo – idhāvuso, bhikkhuno kāme manasikaroto kāmesu cittaṃ na pakkhandati na pasīdati na santiṭṭhati na vimuccati. Nekkhammaṃ kho panassa manasikaroto nekkhamme cittaṃ pakkhandati pasīdati santiṭṭhati vimuccati. Tassa taṃ cittaṃ sugataṃ subhāvitaṃ suvuṭṭhitaṃ suvimuttaṃ visaṃyuttaṃ kāmehi. Ye ca kāmapaccayā uppajjanti āsavā vighātā pariḷāhā, mutto so tehi. Na so taṃ vedanaṃ vedeti. Idamakkhātaṃ kāmānaṃ nissaraṇaṃ.

‘‘Puna caparaṃ, āvuso, bhikkhuno byāpādaṃ manasikaroto byāpāde cittaṃ na pakkhandati na pasīdati na santiṭṭhati na vimuccati. Abyāpādaṃ kho panassa manasikaroto abyāpāde cittaṃ pakkhandati pasīdati santiṭṭhati vimuccati. Tassa taṃ cittaṃ sugataṃ subhāvitaṃ suvuṭṭhitaṃ suvimuttaṃ visaṃyuttaṃ byāpādena. Ye ca byāpādapaccayā uppajjanti āsavā vighātā pariḷāhā, mutto so tehi. Na so taṃ vedanaṃ vedeti. Idamakkhātaṃ byāpādassa nissaraṇaṃ.

‘‘Puna caparaṃ, āvuso, bhikkhuno vihesaṃ manasikaroto vihesāya cittaṃ na pakkhandati na pasīdati na santiṭṭhati na vimuccati. Avihesaṃ kho panassa manasikaroto avihesāya cittaṃ pakkhandati pasīdati santiṭṭhati vimuccati . Tassa taṃ cittaṃ sugataṃ subhāvitaṃ suvuṭṭhitaṃ suvimuttaṃ visaṃyuttaṃ vihesāya. Ye ca vihesāpaccayā uppajjanti āsavā vighātā pariḷāhā, mutto so tehi. Na so taṃ vedanaṃ vedeti. Idamakkhātaṃ vihesāya nissaraṇaṃ.

‘‘Puna caparaṃ, āvuso, bhikkhuno rūpe manasikaroto rūpesu cittaṃ na pakkhandati na pasīdati na santiṭṭhati na vimuccati. Arūpaṃ kho panassa manasikaroto arūpe cittaṃ pakkhandati pasīdati santiṭṭhati vimuccati. Tassa taṃ cittaṃ sugataṃ subhāvitaṃ suvuṭṭhitaṃ suvimuttaṃ visaṃyuttaṃ rūpehi. Ye ca rūpapaccayā uppajjanti āsavā vighātā pariḷāhā, mutto so tehi. Na so taṃ vedanaṃ vedeti. Idamakkhātaṃ rūpānaṃ nissaraṇaṃ.

‘‘Puna caparaṃ, āvuso, bhikkhuno sakkāyaṃ manasikaroto sakkāye cittaṃ na pakkhandati na pasīdati na santiṭṭhati na vimuccati. Sakkāyanirodhaṃ kho panassa manasikaroto sakkāyanirodhe cittaṃ pakkhandati pasīdati santiṭṭhati vimuccati. Tassa taṃ cittaṃ sugataṃ subhāvitaṃ suvuṭṭhitaṃ suvimuttaṃ visaṃyuttaṃ sakkāyena. Ye ca sakkāyapaccayā uppajjanti āsavā vighātā pariḷāhā, mutto so tehi. Na so taṃ vedanaṃ vedeti. Idamakkhātaṃ sakkāyassa nissaraṇaṃ. Ime pañca dhammā duppaṭivijjhā.

(Ja) ‘‘katame pañca dhammā uppādetabbā? Pañca ñāṇiko sammāsamādhi – ‘ayaṃ samādhi paccuppannasukho ceva āyatiñca sukhavipāko’ti paccattaṃyeva ñāṇaṃ uppajjati. ‘Ayaṃ samādhi ariyo nirāmiso’ti paccattaññeva ñāṇaṃ uppajjati. ‘Ayaṃ samādhi akāpurisasevito’ti paccattaṃyeva ñāṇaṃ uppajjati. ‘Ayaṃ samādhi santo paṇīto paṭippassaddhaladdho ekodibhāvādhigato, na sasaṅkhāraniggayhavāritagato’ti\footnote{na ca sasaṅkhāraniggayha vāritavatoti (sī. syā. kaṃ. pī.), na sasaṅkhāraniggayhavārivāvato (ka.), na sasaṅkhāraniggayhavāriyādhigato (?)} paccattaṃyeva ñāṇaṃ uppajjati. ‘So kho panāhaṃ imaṃ samādhiṃ satova samāpajjāmi sato vuṭṭhahāmī’ti paccattaṃyeva ñāṇaṃ uppajjati. Ime pañca dhammā uppādetabbā.

(Jha) ‘‘katame pañca dhammā abhiññeyyā? Pañca vimuttāyatanāni – idhāvuso, bhikkhuno satthā dhammaṃ deseti aññataro vā garuṭṭhāniyo sabrahmacārī. Yathā yathā, āvuso, bhikkhuno satthā dhammaṃ deseti, aññataro vā garuṭṭhāniyo sabrahmacārī, tathā tathā so\footnote{bhikkhu (syā. kaṃ.)} tasmiṃ dhamme atthappaṭisaṃvedī ca hoti dhammapaṭisaṃvedī ca. Tassa atthappaṭisaṃvedino dhammapaṭisaṃvedino pāmojjaṃ jāyati, pamuditassa pīti jāyati, pītimanassa kāyo passambhati, passaddhakāyo sukhaṃ vedeti, sukhino cittaṃ samādhiyati. Idaṃ paṭhamaṃ vimuttāyatanaṃ.

‘‘Puna caparaṃ, āvuso, bhikkhuno na heva kho satthā dhammaṃ deseti, aññataro vā garuṭṭhāniyo sabrahmacārī, api ca kho yathāsutaṃ yathāpariyattaṃ dhammaṃ vitthārena paresaṃ deseti yathā yathā, āvuso, bhikkhu yathāsutaṃ yathāpariyattaṃ dhammaṃ vitthārena paresaṃ deseti. Tathā tathā so tasmiṃ dhamme atthappaṭisaṃvedī ca hoti dhammapaṭisaṃvedī ca. Tassa atthappaṭisaṃvedino dhammapaṭisaṃvedino pāmojjaṃ jāyati, pamuditassa pīti jāyati, pītimanassa kāyo passambhati, passaddhakāyo sukhaṃ vedeti, sukhino cittaṃ samādhiyati. Idaṃ dutiyaṃ vimuttāyatanaṃ.

‘‘Puna caparaṃ, āvuso, bhikkhuno na heva kho satthā dhammaṃ deseti, aññataro vā garuṭṭhāniyo sabrahmacārī, nāpi yathāsutaṃ yathāpariyattaṃ dhammaṃ vitthārena paresaṃ deseti. Api ca kho, yathāsutaṃ yathāpariyattaṃ dhammaṃ vitthārena sajjhāyaṃ karoti. Yathā yathā, āvuso, bhikkhu yathāsutaṃ yathāpariyattaṃ dhammaṃ vitthārena sajjhāyaṃ karoti tathā tathā so tasmiṃ dhamme atthappaṭisaṃvedī ca hoti dhammapaṭisaṃvedī ca. Tassa atthappaṭisaṃvedino dhammapaṭisaṃvedino pāmojjaṃ jāyati, pamuditassa pīti jāyati, pītimanassa kāyo passambhati, passaddhakāyo sukhaṃ vedeti, sukhino cittaṃ samādhiyati. Idaṃ tatiyaṃ vimuttāyatanaṃ.

‘‘Puna caparaṃ, āvuso, bhikkhuno na heva kho satthā dhammaṃ deseti, aññataro vā garuṭṭhāniyo sabrahmacārī, nāpi yathāsutaṃ yathāpariyattaṃ dhammaṃ vitthārena paresaṃ deseti, nāpi yathāsutaṃ yathāpariyattaṃ dhammaṃ vitthārena sajjhāyaṃ karoti. Api ca kho, yathāsutaṃ yathāpariyattaṃ dhammaṃ cetasā anuvitakketi anuvicāreti manasānupekkhati. Yathā yathā , āvuso , bhikkhu yathāsutaṃ yathāpariyattaṃ dhammaṃ cetasā anuvitakketi anuvicāreti manasānupekkhati tathā tathā so tasmiṃ dhamme atthappaṭisaṃvedī ca hoti dhammapaṭisaṃvedī ca. Tassa atthappaṭisaṃvedino dhammapaṭisaṃvedino pāmojjaṃ jāyati, pamuditassa pīti jāyati, pītimanassa kāyo passambhati, passaddhakāyo sukhaṃ vedeti, sukhino cittaṃ samādhiyati. Idaṃ catutthaṃ vimuttāyatanaṃ.

‘‘Puna caparaṃ, āvuso, bhikkhuno na heva kho satthā dhammaṃ deseti, aññataro vā garuṭṭhāniyo sabrahmacārī, nāpi yathāsutaṃ yathāpariyattaṃ dhammaṃ vitthārena paresaṃ deseti, nāpi yathāsutaṃ yathāpariyattaṃ dhammaṃ vitthārena sajjhāyaṃ karoti, nāpi yathāsutaṃ yathāpariyattaṃ dhammaṃ cetasā anuvitakketi anuvicāreti manasānupekkhati; api ca khvassa aññataraṃ samādhinimittaṃ suggahitaṃ hoti sumanasikataṃ sūpadhāritaṃ suppaṭividdhaṃ paññāya. Yathā yathā, āvuso, bhikkhuno aññataraṃ samādhinimittaṃ suggahitaṃ hoti sumanasikataṃ sūpadhāritaṃ suppaṭividdhaṃ paññāya tathā tathā so tasmiṃ dhamme atthappaṭisaṃvedī ca hoti dhammappaṭisaṃvedī ca. Tassa atthappaṭisaṃvedino dhammappaṭisaṃvedino pāmojjaṃ jāyati, pamuditassa pīti jāyati, pītimanassa kāyo passambhati, passaddhakāyo sukhaṃ vedeti, sukhino cittaṃ samādhiyati. Idaṃ pañcamaṃ vimuttāyatanaṃ. Ime pañca dhammā abhiññeyyā.

(Ña) ‘‘katame pañca dhammā sacchikātabbā? Pañca dhammakkhandhā – sīlakkhandho , samādhikkhandho, paññākkhandho, vimuttikkhandho, vimuttiñāṇadassanakkhandho. Ime pañca dhammā sacchikātabbā.

‘‘Iti ime paññāsa dhammā bhūtā tacchā tathā avitathā anaññathā sammā tathāgatena abhisambuddhā.

\subsubsection{Cha dhammā}

\paragraph{356.} ‘‘Cha dhammā bahukārā…pe… cha dhammā sacchikātabbā.

(Ka) ‘‘katame cha dhammā bahukārā? Cha sāraṇīyā dhammā. Idhāvuso, bhikkhuno mettaṃ kāyakammaṃ paccupaṭṭhitaṃ hoti sabrahmacārīsu āvi ceva raho ca, ayampi dhammo sāraṇīyo piyakaraṇo garukaraṇo saṅgahāya avivādāya sāmaggiyā ekībhāvāya saṃvattati.

‘‘Puna caparaṃ, āvuso, bhikkhuno mettaṃ vacīkammaṃ…pe… ekībhāvāya saṃvattati.

‘‘Puna caparaṃ, āvuso, bhikkhuno mettaṃ manokammaṃ…pe… ekībhāvāya saṃvattati.

‘‘Puna caparaṃ, āvuso, bhikkhu ye te lābhā dhammikā dhammaladdhā antamaso pattapariyāpannamattampi, tathārūpehi lābhehi appaṭivibhattabhogī hoti sīlavantehi sabrahmacārīhi sādhāraṇabhogī, ayampi dhammo sāraṇīyo…pe… ekībhāvāya saṃvattati.

‘‘Puna caparaṃ, āvuso, bhikkhu, yāni tāni sīlāni akhaṇḍāni acchiddāni asabalāni akammāsāni bhujissāni viññuppasatthāni aparāmaṭṭhāni samādhisaṃvattanikāni, tathārūpesu sīlesu sīlasāmaññagato viharati sabrahmacārīhi āvi ceva raho ca, ayampi dhammo sāraṇīyo…pe… ekībhāvāya saṃvattati.

‘‘Puna caparaṃ, āvuso, bhikkhu yāyaṃ diṭṭhi ariyā niyyānikā niyyāti takkarassa sammā dukkhakkhayāya, tathārūpāya diṭṭhiyā diṭṭhi sāmaññagato viharati sabrahmacārīhi āvi ceva raho ca, ayampi dhammo sāraṇīyo piyakaraṇo garukaraṇo, saṅgahāya avivādāya sāmaggiyā ekībhāvāya saṃvattati. Ime cha dhammā bahukārā.

(Kha) ‘‘katame cha dhammā bhāvetabbā? Cha anussatiṭṭhānāni – buddhānussati, dhammānussati, saṅghānussati, sīlānussati, cāgānussati, devatānussati. Ime cha dhammā bhāvetabbā.

(Ga) ‘‘katame cha dhammā pariññeyyā? Cha ajjhattikāni āyatanāni – cakkhāyatanaṃ, sotāyatanaṃ, ghānāyatanaṃ, jivhāyatanaṃ, kāyāyatanaṃ, manāyatanaṃ. Ime cha dhammā pariññeyyā.

(Gha) ‘‘katame cha dhammā pahātabbā? Cha taṇhākāyā – rūpataṇhā, saddataṇhā, gandhataṇhā, rasataṇhā, phoṭṭhabbataṇhā, dhammataṇhā. Ime cha dhammā pahātabbā.

(Ṅa) ‘‘katame cha dhammā hānabhāgiyā? Cha agāravā – idhāvuso, bhikkhu satthari agāravo viharati appatisso. Dhamme…pe… saṅghe… sikkhāya… appamāde… paṭisanthāre agāravo viharati appatisso. Ime cha dhammā hānabhāgiyā.

(Ca) ‘‘katame cha dhammā visesabhāgiyā? Cha gāravā – idhāvuso, bhikkhu satthari sagāravo viharati sappatisso dhamme…pe… saṅghe… sikkhāya… appamāde… paṭisanthāre sagāravo viharati sappatisso. Ime cha dhammā visesabhāgiyā.

(Cha) ‘‘katame cha dhammā duppaṭivijjhā? Cha nissaraṇiyā dhātuyo – idhāvuso, bhikkhu evaṃ vadeyya – ‘mettā hi kho me, cetovimutti bhāvitā bahulīkatā yānīkatā vatthukatā anuṭṭhitā paricitā susamāraddhā, atha ca pana me byāpādo cittaṃ pariyādāya tiṭṭhatī’ti. So ‘mā hevaṃ’ tissa vacanīyo ‘māyasmā evaṃ avaca, mā bhagavantaṃ abbhācikkhi. Na hi sādhu bhagavato abbhakkhānaṃ, na hi bhagavā evaṃ vadeyya. Aṭṭhānametaṃ āvuso anavakāso yaṃ mettāya cetovimuttiyā bhāvitāya bahulīkatāya yānīkatāya vatthukatāya anuṭṭhitāya paricitāya susamāraddhāya. Atha ca panassa byāpādo cittaṃ pariyādāya ṭhassatīti, netaṃ ṭhānaṃ vijjati. Nissaraṇaṃ hetaṃ, āvuso, byāpādassa, yadidaṃ mettācetovimuttī’ti.

‘‘Idha panāvuso, bhikkhu evaṃ vadeyya – ‘karuṇā hi kho me cetovimutti bhāvitā bahulīkatā yānīkatā vatthukatā anuṭṭhitā paricitā susamāraddhā. Atha ca pana me vihesā cittaṃ pariyādāya tiṭṭhatī’ti. So – ‘mā hevaṃ’ tissa vacanīyo, ‘māyasmā evaṃ avaca, mā bhagavantaṃ abbhācikkhi…pe… nissaraṇaṃ hetaṃ, āvuso, vihesāya, yadidaṃ karuṇācetovimuttī’ti.

‘‘Idha panāvuso, bhikkhu evaṃ vadeyya – ‘muditā hi kho me cetovimutti bhāvitā…pe… atha ca pana me arati cittaṃ pariyādāya tiṭṭhatī’ti. So – ‘mā hevaṃ’ tissa vacanīyo ‘māyasmā evaṃ avaca…pe… nissaraṇaṃ hetaṃ, āvuso aratiyā, yadidaṃ muditācetovimuttī’ti.

‘‘Idha panāvuso, bhikkhu evaṃ vadeyya – ‘upekkhā hi kho me cetovimutti bhāvitā…pe… atha ca pana me rāgo cittaṃ pariyādāya tiṭṭhatī’ti. So – ‘mā hevaṃ’ tissa vacanīyo ‘māyasmā evaṃ avaca…pe… nissaraṇaṃ hetaṃ, āvuso, rāgassa yadidaṃ upekkhācetovimuttī’ti.

‘‘Idha panāvuso, bhikkhu evaṃ vadeyya – ‘animittā hi kho me cetovimutti bhāvitā…pe… atha ca pana me nimittānusāri viññāṇaṃ hotī’ti. So – ‘mā hevaṃ’ tissa vacanīyo ‘māyasmā evaṃ avaca…pe… nissaraṇaṃ hetaṃ, āvuso, sabbanimittānaṃ yadidaṃ animittā cetovimuttī’ti.

‘‘Idha panāvuso, bhikkhu evaṃ vadeyya – ‘asmīti kho me vigataṃ, ayamahamasmīti na samanupassāmi, atha ca pana me vicikicchākathaṃkathāsallaṃ cittaṃ pariyādāya tiṭṭhatī’ti. So – ‘mā hevaṃ’ tissa vacanīyo ‘māyasmā evaṃ avaca, mā bhagavantaṃ abbhācikkhi, na hi sādhu bhagavato abbhakkhānaṃ, na hi bhagavā evaṃ vadeyya. Aṭṭhānametaṃ, āvuso, anavakāso yaṃ asmīti vigate ayamahamasmīti asamanupassato. Atha ca panassa vicikicchākathaṃkathāsallaṃ cittaṃ pariyādāya ṭhassati, netaṃ ṭhānaṃ vijjati. Nissaraṇaṃ hetaṃ, āvuso, vicikicchākathaṃkathāsallassa, yadidaṃ asmimānasamugghāṭo’ti. Ime cha dhammā duppaṭivijjhā.

(Ja) ‘‘katame cha dhammā uppādetabbā? Cha satatavihārā. Idhāvuso, bhikkhu cakkhunā rūpaṃ disvā neva sumano hoti na dummano, upekkhako viharati sato sampajāno. Sotena saddaṃ sutvā…pe… ghānena gandhaṃ ghāyitvā… jivhāya rasaṃ sāyitvā… kāyena phoṭṭhabbaṃ phusitvā… manasā dhammaṃ viññāya neva sumano hoti na dummano, upekkhako viharati sato sampajāno. Ime cha dhammā uppādetabbā.

(Jha) ‘‘katame cha dhammā abhiññeyyā? Cha anuttariyāni – dassanānuttariyaṃ, savanānuttariyaṃ, lābhānuttariyaṃ, sikkhānuttariyaṃ, pāricariyānuttariyaṃ, anussatānuttariyaṃ. Ime cha dhammā abhiññeyyā.

(Ña) ‘‘katame cha dhammā sacchikātabbā? Cha abhiññā – idhāvuso, bhikkhu anekavihitaṃ iddhividhaṃ paccanubhoti – ekopi hutvā bahudhā hoti , bahudhāpi hutvā eko hoti. Āvibhāvaṃ tirobhāvaṃ. Tirokuṭṭaṃ tiropākāraṃ tiropabbataṃ asajjamāno gacchati seyyathāpi ākāse . Pathaviyāpi ummujjanimujjaṃ karoti seyyathāpi udake. Udakepi abhijjamāne gacchati seyyathāpi pathaviyaṃ. Ākāsepi pallaṅkena kamati seyyathāpi pakkhī sakuṇo. Imepi candimasūriye evaṃmahiddhike evaṃmahānubhāve pāṇinā parāmasati parimajjati. Yāva brahmalokāpi kāyena vasaṃ vatteti.

‘‘Dibbāya sotadhātuyā visuddhāya atikkantamānusikāya ubho sadde suṇāti dibbe ca mānuse ca, ye dūre santike ca.

‘‘Parasattānaṃ parapuggalānaṃ cetasā ceto paricca pajānāti\footnote{jānāti (syā. kaṃ.)}, sarāgaṃ vā cittaṃ sarāgaṃ cittanti pajānāti …pe… avimuttaṃ vā cittaṃ avimuttaṃ cittanti pajānāti.

‘‘So anekavihitaṃ pubbenivāsaṃ anussarati, seyyathidaṃ ekampi jātiṃ…pe… iti sākāraṃ sauddesaṃ anekavihitaṃ pubbenivāsaṃ anussarati.

‘‘Dibbena cakkhunā visuddhena atikkantamānusakena satte passati cavamāne upapajjamāne hīne paṇīte suvaṇṇe dubbaṇṇe sugate duggate yathākammūpage satte pajānāti …pe…

‘‘Āsavānaṃ khayā anāsavaṃ cetovimuttiṃ paññāvimuttiṃ diṭṭheva dhamme sayaṃ abhiññā sacchikatvā upasampajja viharati. Ime cha dhammā sacchikātabbā.

‘‘Iti ime saṭṭhi dhammā bhūtā tacchā tathā avitathā anaññathā sammā tathāgatena abhisambuddhā.

\subsubsection{Satta dhammā}

\paragraph{357.} ‘‘Satta dhammā bahukārā…pe… satta dhammā sacchikātabbā.

(Ka) ‘‘katame satta dhammā bahukārā? Satta ariyadhanāni – saddhādhanaṃ, sīladhanaṃ, hiridhanaṃ, ottappadhanaṃ, sutadhanaṃ, cāgadhanaṃ, paññādhanaṃ. Ime satta dhammā bahukārā.

(Kha) ‘‘katame satta dhammā bhāvetabbā? Satta sambojjhaṅgā – satisambojjhaṅgo, dhammavicayasambojjhaṅgo, vīriyasambojjhaṅgo, pītisambojjhaṅgo, passaddhisambojjhaṅgo, samādhisambojjhaṅgo, upekkhāsambojjhaṅgo . Ime satta dhammā bhāvetabbā.

(Ga) ‘‘katame satta dhammā pariññeyyā? Satta viññāṇaṭṭhitiyo – santāvuso, sattā nānattakāyā nānattasaññino, seyyathāpi manussā ekacce ca devā ekacce ca vinipātikā. Ayaṃ paṭhamā viññāṇaṭṭhiti.

‘‘Santāvuso , sattā nānattakāyā ekattasaññino, seyyathāpi devā brahmakāyikā paṭhamābhinibbattā. Ayaṃ dutiyā viññāṇaṭṭhiti.

‘‘Santāvuso, sattā ekattakāyā nānattasaññino, seyyathāpi devā ābhassarā. Ayaṃ tatiyā viññāṇaṭṭhiti.

‘‘Santāvuso, sattā ekattakāyā ekattasaññino, seyyathāpi devā subhakiṇhā. Ayaṃ catutthī viññāṇaṭṭhiti.

‘‘Santāvuso, sattā sabbaso rūpasaññānaṃ samatikkamā…pe… ‘ananto ākāso’ti ākāsānañcāyatanūpagā. Ayaṃ pañcamī viññāṇaṭṭhiti.

‘‘Santāvuso, sattā sabbaso ākāsānañcāyatanaṃ samatikkamma ‘anantaṃ viññāṇa’nti viññāṇañcāyatanūpagā. Ayaṃ chaṭṭhī viññāṇaṭṭhiti.

‘‘Santāvuso, sattā sabbaso viññāṇañcāyatanaṃ samatikkamma ‘natthi kiñcī’ti ākiñcaññāyatanūpagā. Ayaṃ sattamī viññāṇaṭṭhiti. Ime satta dhammā pariññeyyā.

(Gha) ‘‘katame satta dhammā pahātabbā? Sattānusayā – kāmarāgānusayo, paṭighānusayo, diṭṭhānusayo, vicikicchānusayo, mānānusayo, bhavarāgānusayo , avijjānusayo. Ime satta dhammā pahātabbā.

(Ṅa) ‘‘katame satta dhammā hānabhāgiyā? Satta asaddhammā – idhāvuso, bhikkhu assaddho hoti, ahiriko hoti, anottappī hoti, appassuto hoti, kusīto hoti, muṭṭhassati hoti, duppañño hoti. Ime satta dhammā hānabhāgiyā.

(Ca) ‘‘katame satta dhammā visesabhāgiyā? Satta saddhammā – idhāvuso, bhikkhu saddho hoti, hirimā\footnote{hiriko (syā. kaṃ.)} hoti, ottappī hoti, bahussuto hoti, āraddhavīriyo hoti, upaṭṭhitassati hoti, paññavā hoti. Ime satta dhammā visesabhāgiyā.

(Cha) ‘‘katame satta dhammā duppaṭivijjhā? Satta sappurisadhammā – idhāvuso, bhikkhu dhammaññū ca hoti atthaññū ca attaññū ca mattaññū ca kālaññū ca parisaññū ca puggalaññū ca. Ime satta dhammā duppaṭivijjhā.

(Ja) ‘‘katame satta dhammā uppādetabbā? Satta saññā – aniccasaññā, anattasaññā, asubhasaññā, ādīnavasaññā, pahānasaññā, virāgasaññā, nirodhasaññā. Ime satta dhammā uppādetabbā.

(Jha) ‘‘katame satta dhammā abhiññeyyā? Satta niddasavatthūni – idhāvuso, bhikkhu sikkhāsamādāne tibbacchando hoti, āyatiñca sikkhāsamādāne avigatapemo. Dhammanisantiyā tibbacchando hoti, āyatiñca dhammanisantiyā avigatapemo. Icchāvinaye tibbacchando hoti, āyatiñca icchāvinaye avigatapemo. Paṭisallāne tibbacchando hoti, āyatiñca paṭisallāne avigatapemo. Vīriyāramme tibbacchando hoti, āyatiñca vīriyāramme avigatapemo. Satinepakke tibbacchando hoti, āyatiñca satinepakke avigatapemo. Diṭṭhipaṭivedhe tibbacchando hoti, āyatiñca diṭṭhipaṭivedhe avigatapemo. Ime satta dhammā abhiññeyyā.

(Ña) ‘‘katame satta dhammā sacchikātabbā? Satta khīṇāsavabalāni – idhāvuso, khīṇāsavassa bhikkhuno aniccato sabbe saṅkhārā yathābhūtaṃ sammappaññāya sudiṭṭhā honti. Yaṃpāvuso, khīṇāsavassa bhikkhuno aniccato sabbe saṅkhārā yathābhūtaṃ sammappaññāya sudiṭṭhā honti, idampi khīṇāsavassa bhikkhuno balaṃ hoti, yaṃ balaṃ āgamma khīṇāsavo bhikkhu āsavānaṃ khayaṃ paṭijānāti – ‘khīṇā me āsavā’ti.

‘‘Puna caparaṃ, āvuso, khīṇāsavassa bhikkhuno aṅgārakāsūpamā kāmā yathābhūtaṃ sammappaññāya sudiṭṭhā honti. Yaṃpāvuso…pe… ‘khīṇā me āsavā’ti.

‘‘Puna caparaṃ, āvuso, khīṇāsavassa bhikkhuno vivekaninnaṃ cittaṃ hoti vivekapoṇaṃ vivekapabbhāraṃ vivekaṭṭhaṃ nekkhammābhirataṃ byantībhūtaṃ sabbaso āsavaṭṭhāniyehi dhammehi. Yaṃpāvuso…pe… ‘khīṇā me āsavā’ti.

‘‘Puna caparaṃ, āvuso, khīṇāsavassa bhikkhuno cattāro satipaṭṭhānā bhāvitā honti subhāvitā . Yaṃpāvuso…pe… ‘khīṇā me āsavā’ti.

‘‘Puna caparaṃ, āvuso, khīṇāsavassa bhikkhuno pañcindriyāni bhāvitāni honti subhāvitāni. Yaṃpāvuso…pe… ‘khīṇā me āsavā’ti.

‘‘Puna caparaṃ, āvuso, khīṇāsavassa bhikkhuno satta bojjhaṅgā bhāvitā honti subhāvitā. Yaṃpāvuso…pe… ‘khīṇā me āsavā’ti.

‘‘Puna caparaṃ, āvuso, khīṇāsavassa bhikkhuno ariyo aṭṭhaṅgiko maggo bhāvito hoti subhāvito. Yaṃpāvuso, khīṇāsavassa bhikkhuno ariyo aṭṭhaṅgiko maggo bhāvito hoti subhāvito, idampi khīṇāsavassa bhikkhuno balaṃ hoti, yaṃ balaṃ āgamma khīṇāsavo bhikkhu āsavānaṃ khayaṃ paṭijānāti – ‘khīṇā me āsavā’ti. Ime satta dhammā sacchikātabbā.

‘‘Itime sattati dhammā bhūtā tacchā tathā avitathā anaññathā sammā tathāgatena abhisambuddhā.

Paṭhamabhāṇavāro niṭṭhito.

\subsubsection{Aṭṭha dhammā}

\paragraph{358.} ‘‘Aṭṭha dhammā bahukārā…pe… aṭṭha dhammā sacchikātabbā.

(Ka) ‘‘katame aṭṭha dhammā bahukārā? Aṭṭha hetū aṭṭha paccayā ādibrahmacariyikāya paññāya appaṭiladdhāya paṭilābhāya paṭiladdhāya bhiyyobhāvāya vepullāya bhāvanāya pāripūriyā saṃvattanti. Katame aṭṭha? Idhāvuso, bhikkhu satthāraṃ\footnote{satthāraṃ vā (syā. ka.)} upanissāya viharati aññataraṃ vā garuṭṭhāniyaṃ sabrahmacāriṃ, yatthassa tibbaṃ hirottappaṃ paccupaṭṭhitaṃ hoti pemañca gāravo ca. Ayaṃ paṭhamo hetu paṭhamo paccayo ādibrahmacariyikāya paññāya appaṭiladdhāya paṭilābhāya . Paṭiladdhāya bhiyyobhāvāya vepullāya bhāvanāya pāripūriyā saṃvattati.

‘‘Taṃ kho pana satthāraṃ upanissāya viharati aññataraṃ vā garuṭṭhāniyaṃ sabrahmacāriṃ , yatthassa tibbaṃ hirottappaṃ paccupaṭṭhitaṃ hoti pemañca gāravo ca. Te kālena kālaṃ upasaṅkamitvā paripucchati paripañhati – ‘idaṃ, bhante, kathaṃ? Imassa ko attho’ti? Tassa te āyasmanto avivaṭañceva vivaranti, anuttānīkatañca uttānī\footnote{anuttānikatañca uttāniṃ (ka.)} karonti, anekavihitesu ca kaṅkhāṭṭhāniyesu dhammesu kaṅkhaṃ paṭivinodenti. Ayaṃ dutiyo hetu dutiyo paccayo ādibrahmacariyikāya paññāya appaṭiladdhāya paṭilābhāya, paṭiladdhāya bhiyyobhāvāya, vepullāya bhāvanāya pāripūriyā saṃvattati.

‘‘Taṃ kho pana dhammaṃ sutvā dvayena vūpakāsena sampādeti – kāyavūpakāsena ca cittavūpakāsena ca. Ayaṃ tatiyo hetu tatiyo paccayo ādibrahmacariyikāya paññāya appaṭiladdhāya paṭilābhāya, paṭiladdhāya bhiyyobhāvāya vepullāya bhāvanāya pāripūriyā saṃvattati.

‘‘Puna caparaṃ, āvuso, bhikkhu sīlavā hoti, pātimokkhasaṃvarasaṃvuto viharati ācāragocarasampanno, aṇumattesu vajjesu bhayadassāvī samādāya sikkhati sikkhāpadesu. Ayaṃ catuttho hetu catuttho paccayo ādibrahmacariyikāya paññāya appaṭiladdhāya paṭilābhāya, paṭiladdhāya bhiyyobhāvāya vepullāya bhāvanāya pāripūriyā saṃvattati.

‘‘Puna caparaṃ, āvuso, bhikkhu bahussuto hoti sutadharo sutasannicayo. Ye te dhammā ādikalyāṇā majjhekalyāṇā pariyosānakalyāṇā sātthā sabyañjanā kevalaparipuṇṇaṃ parisuddhaṃ brahmacariyaṃ abhivadanti, tathārūpāssa dhammā bahussutā honti dhātā vacasā paricitā manasānupekkhitā diṭṭhiyā suppaṭividdhā. Ayaṃ pañcamo hetu pañcamo paccayo ādibrahmacariyikāya paññāya appaṭiladdhāya paṭilābhāya, paṭiladdhāya bhiyyobhāvāya vepullāya bhāvanāya pāripūriyā saṃvattati.

‘‘Puna caparaṃ, āvuso, bhikkhu āraddhavīriyo viharati akusalānaṃ dhammānaṃ pahānāya, kusalānaṃ dhammānaṃ upasampadāya, thāmavā daḷhaparakkamo anikkhittadhuro kusalesu dhammesu. Ayaṃ chaṭṭho hetu chaṭṭho paccayo ādibrahmacariyikāya paññāya appaṭiladdhāya paṭilābhāya, paṭiladdhāya bhiyyobhāvāya vepullāya bhāvanāya pāripūriyā saṃvattati.

‘‘Puna caparaṃ, āvuso, bhikkhu satimā hoti paramena satinepakkena samannāgato. Cirakatampi cirabhāsitampi saritā anussaritā. Ayaṃ sattamo hetu sattamo paccayo ādibrahmacariyikāya paññāya appaṭiladdhāya paṭilābhāya, paṭiladdhāya bhiyyobhāvāya vepullāya bhāvanāya pāripūriyā saṃvattati.

‘‘Puna caparaṃ, āvuso, bhikkhu pañcasu upādānakkhandhesu, udayabbayānupassī viharati – ‘iti rūpaṃ iti rūpassa samudayo iti rūpassa atthaṅgamo; iti vedanā iti vedanāya samudayo iti vedanāya atthaṅgamo; iti saññā iti saññāya samudayo iti saññāya atthaṅgamo; iti saṅkhārā iti saṅkhārānaṃ samudayo iti saṅkhārānaṃ atthaṅgamo; iti viññāṇaṃ iti viññāṇassa samudayo iti viññāṇassa atthaṅgamo’ti. Ayaṃ aṭṭhamo hetu aṭṭhamo paccayo ādibrahmacariyikāya paññāya appaṭiladdhāya paṭilābhāya, paṭiladdhāya bhiyyobhāvāya vepullāya bhāvanāya pāripūriyā saṃvattati. Ime aṭṭha dhammā bahukārā.

(Kha) ‘‘katame aṭṭha dhammā bhāvetabbā? Ariyo aṭṭhaṅgiko maggo seyyathidaṃ – sammādiṭṭhi, sammāsaṅkappo, sammāvācā, sammākammanto, sammāājīvo, sammāvāyāmo, sammāsati, sammāsamādhi. Ime aṭṭha dhammā bhāvetabbā.

(Ga) ‘‘katame aṭṭha dhammā pariññeyyā? Aṭṭha lokadhammā – lābho ca, alābho ca, yaso ca, ayaso ca, nindā ca, pasaṃsā ca, sukhañca, dukkhañca. Ime aṭṭha dhammā pariññeyyā.

(Gha) ‘‘katame aṭṭha dhammā pahātabbā? Aṭṭha micchattā – micchādiṭṭhi, micchāsaṅkappo, micchāvācā, micchākammanto, micchāājīvo, micchāvāyāmo, micchāsati, micchāsamādhi. Ime aṭṭha dhammā pahātabbā.

(Ṅa) ‘‘katame aṭṭha dhammā hānabhāgiyā? Aṭṭha kusītavatthūni. Idhāvuso, bhikkhunā kammaṃ kātabbaṃ hoti, tassa evaṃ hoti – ‘kammaṃ kho me kātabbaṃ bhavissati, kammaṃ kho pana me karontassa kāyo kilamissati, handāhaṃ nipajjāmī’ti. So nipajjati, na vīriyaṃ ārabhati appattassa pattiyā anadhigatassa adhigamāya asacchikatassa sacchikiriyāya. Idaṃ paṭhamaṃ kusītavatthu.

‘‘Puna caparaṃ, āvuso, bhikkhunā kammaṃ kataṃ hoti . Tassa evaṃ hoti – ‘ahaṃ kho kammaṃ akāsiṃ, kammaṃ kho pana me karontassa kāyo kilanto, handāhaṃ nipajjāmī’ti. So nipajjati, na vīriyaṃ ārabhati…pe… idaṃ dutiyaṃ kusītavatthu.

‘‘Puna caparaṃ, āvuso, bhikkhunā maggo gantabbo hoti. Tassa evaṃ hoti – ‘maggo kho me gantabbo bhavissati, maggaṃ kho pana me gacchantassa kāyo kilamissati, handāhaṃ nipajjāmī’ti. So nipajjati, na vīriyaṃ ārabhati…pe… idaṃ tatiyaṃ kusītavatthu.

‘‘Puna caparaṃ, āvuso, bhikkhunā maggo gato hoti. Tassa evaṃ hoti – ‘ahaṃ kho maggaṃ agamāsiṃ, maggaṃ kho pana me gacchantassa kāyo kilanto, handāhaṃ nipajjāmī’ti. So nipajjati, na vīriyaṃ ārabhati…pe… idaṃ catutthaṃ kusītavatthu.

‘‘Puna caparaṃ, āvuso, bhikkhu gāmaṃ vā nigamaṃ vā piṇḍāya caranto na labhati lūkhassa vā paṇītassa vā bhojanassa yāvadatthaṃ pāripūriṃ. Tassa evaṃ hoti – ‘ahaṃ kho gāmaṃ vā nigamaṃ vā piṇḍāya caranto nālatthaṃ lūkhassa vā paṇītassa vā bhojanassa yāvadatthaṃ pāripūriṃ, tassa me kāyo kilanto akammañño, handāhaṃ nipajjāmī’ti…pe… idaṃ pañcamaṃ kusītavatthu.

‘‘Puna caparaṃ, āvuso, bhikkhu gāmaṃ vā nigamaṃ vā piṇḍāya caranto labhati lūkhassa vā paṇītassa vā bhojanassa yāvadatthaṃ pāripūriṃ. Tassa evaṃ hoti – ‘ahaṃ kho gāmaṃ vā nigamaṃ vā piṇḍāya caranto alatthaṃ lūkhassa vā paṇītassa vā bhojanassa yāvadatthaṃ pāripūriṃ , tassa me kāyo garuko akammañño, māsācitaṃ maññe, handāhaṃ nipajjāmī’ti. So nipajjati…pe… idaṃ chaṭṭhaṃ kusītavatthu.

‘‘Puna caparaṃ, āvuso, bhikkhuno uppanno hoti appamattako ābādho, tassa evaṃ hoti – ‘uppanno kho me ayaṃ appamattako ābādho atthi kappo nipajjituṃ, handāhaṃ nipajjāmī’ti. So nipajjati…pe… idaṃ sattamaṃ kusītavatthu.

‘‘Puna caparaṃ, āvuso, bhikkhu gilānāvuṭṭhito hoti aciravuṭṭhito gelaññā. Tassa evaṃ hoti – ‘ahaṃ kho gilānāvuṭṭhito aciravuṭṭhito gelaññā. Tassa me kāyo dubbalo akammañño, handāhaṃ nipajjāmī’ti. So nipajjati…pe… idaṃ aṭṭhamaṃ kusītavatthu. Ime aṭṭha dhammā hānabhāgiyā.

(Ca) ‘‘katame aṭṭha dhammā visesabhāgiyā? Aṭṭha ārambhavatthūni. Idhāvuso, bhikkhunā kammaṃ kātabbaṃ hoti, tassa evaṃ hoti – ‘kammaṃ kho me kātabbaṃ bhavissati, kammaṃ kho pana me karontena na sukaraṃ buddhānaṃ sāsanaṃ manasikātuṃ, handāhaṃ vīriyaṃ ārabhāmi appattassa pattiyā anadhigatassa adhigamāya asacchikatassa sacchikiriyāyā’ti. So vīriyaṃ ārabhati appattassa pattiyā anadhigatassa adhigamāya asacchikatassa sacchikiriyāya. Idaṃ paṭhamaṃ ārambhavatthu.

‘‘Puna caparaṃ, āvuso, bhikkhunā kammaṃ kataṃ hoti. Tassa evaṃ hoti – ‘ahaṃ kho kammaṃ akāsiṃ, kammaṃ kho panāhaṃ karonto nāsakkhiṃ buddhānaṃ sāsanaṃ manasikātuṃ, handāhaṃ vīriyaṃ ārabhāmi…pe… idaṃ dutiyaṃ ārambhavatthu.

‘‘Puna caparaṃ, āvuso, bhikkhunā maggo gantabbo hoti. Tassa evaṃ hoti – ‘maggo kho me gantabbo bhavissati, maggaṃ kho pana me gacchantena na sukaraṃ buddhānaṃ sāsanaṃ manasikātuṃ, handāhaṃ vīriyaṃ ārabhāmi…pe… idaṃ tatiyaṃ ārambhavatthu.

‘‘Puna caparaṃ, āvuso, bhikkhunā maggo gato hoti. Tassa evaṃ hoti – ‘ahaṃ kho maggaṃ agamāsiṃ, maggaṃ kho panāhaṃ gacchanto nāsakkhiṃ buddhānaṃ sāsanaṃ manasikātuṃ, handāhaṃ vīriyaṃ ārabhāmi…pe… idaṃ catutthaṃ ārambhavatthu.

‘‘Puna caparaṃ, āvuso, bhikkhu gāmaṃ vā nigamaṃ vā piṇḍāya caranto na labhati lūkhassa vā paṇītassa vā bhojanassa yāvadatthaṃ pāripūriṃ. Tassa evaṃ hoti – ‘ahaṃ kho gāmaṃ vā nigamaṃ vā piṇḍāya caranto nālatthaṃ lūkhassa vā paṇītassa vā bhojanassa yāvadatthaṃ pāripūriṃ , tassa me kāyo lahuko kammañño, handāhaṃ vīriyaṃ ārabhāmi…pe… idaṃ pañcamaṃ ārambhavatthu.

‘‘Puna caparaṃ, āvuso, bhikkhu gāmaṃ vā nigamaṃ vā piṇḍāya caranto labhati lūkhassa vā paṇītassa vā bhojanassa yāvadatthaṃ pāripūriṃ. Tassa evaṃ hoti – ‘ahaṃ kho gāmaṃ vā nigamaṃ vā piṇḍāya caranto alatthaṃ lūkhassa vā paṇītassa vā bhojanassa yāvadatthaṃ pāripūriṃ. Tassa me kāyo balavā kammañño, handāhaṃ vīriyaṃ ārabhāmi…pe… idaṃ chaṭṭhaṃ ārambhavatthu.

‘‘Puna caparaṃ, āvuso, bhikkhuno uppanno hoti appamattako ābādho. Tassa evaṃ hoti – ‘uppanno kho me ayaṃ appamattako ābādho ṭhānaṃ kho panetaṃ vijjati, yaṃ me ābādho pavaḍḍheyya, handāhaṃ vīriyaṃ ārabhāmi…pe… idaṃ sattamaṃ ārambhavatthu.

‘‘Puna caparaṃ, āvuso, bhikkhu gilānā vuṭṭhito hoti aciravuṭṭhito gelaññā. Tassa evaṃ hoti – ‘ahaṃ kho gilānā vuṭṭhito aciravuṭṭhito gelaññā, ṭhānaṃ kho panetaṃ vijjati, yaṃ me ābādho paccudāvatteyya, handāhaṃ vīriyaṃ ārabhāmi appattassa pattiyā anadhigatassa adhigamāya asacchikatassa sacchikiriyāyā’ti. So vīriyaṃ ārabhati appattassa pattiyā anadhigatassa adhigamāya asacchikatassa sacchikiriyāya. Idaṃ aṭṭhamaṃ ārambhavatthu. Ime aṭṭha dhammā visesabhāgiyā.

(Cha) ‘‘katame aṭṭha dhammā duppaṭivijjhā? Aṭṭha akkhaṇā asamayā brahmacariyavāsāya. Idhāvuso, tathāgato ca loke uppanno hoti arahaṃ sammāsambuddho, dhammo ca desiyati opasamiko parinibbāniko sambodhagāmī sugatappavedito. Ayañca puggalo nirayaṃ upapanno hoti. Ayaṃ paṭhamo akkhaṇo asamayo brahmacariyavāsāya.

‘‘Puna caparaṃ, āvuso, tathāgato ca loke uppanno hoti arahaṃ sammāsambuddho, dhammo ca desiyati opasamiko parinibbāniko sambodhagāmī sugatappavedito, ayañca puggalo tiracchānayoniṃ upapanno hoti. Ayaṃ dutiyo akkhaṇo asamayo brahmacariyavāsāya.

‘‘Puna caparaṃ…pe… pettivisayaṃ upapanno hoti. Ayaṃ tatiyo akkhaṇo asamayo brahmacariyavāsāya.

‘‘Puna caparaṃ…pe… aññataraṃ dīghāyukaṃ devanikāyaṃ upapanno hoti. Ayaṃ catuttho akkhaṇo asamayo brahmacariyavāsāya.

‘‘Puna caparaṃ…pe… paccantimesu janapadesu paccājāto hoti milakkhesu aviññātāresu, yattha natthi gati bhikkhūnaṃ bhikkhunīnaṃ upāsakānaṃ upāsikānaṃ. Ayaṃ pañcamo akkhaṇo asamayo brahmacariyavāsāya.

‘‘Puna caparaṃ…pe… ayañca puggalo majjhimesu janapadesu paccājāto hoti, so ca hoti micchādiṭṭhiko viparītadassano – ‘natthi dinnaṃ, natthi yiṭṭhaṃ, natthi hutaṃ, natthi sukatadukkaṭānaṃ kammānaṃ phalaṃ vipāko, natthi ayaṃ loko, natthi paro loko, natthi mātā, natthi pitā, natthi sattā opapātikā, natthi loke samaṇabrāhmaṇā sammaggatā sammāpaṭipannā ye imañca lokaṃ parañca lokaṃ sayaṃ abhiññā sacchikatvā pavedentī’ti. Ayaṃ chaṭṭho akkhaṇo asamayo brahmacariyavāsāya.

‘‘Puna caparaṃ…pe… ayañca puggalo majjhimesu janapadesu paccājāto hoti , so ca hoti duppañño jaḷo eḷamūgo, nappaṭibalo subhāsitadubbhāsitānamatthamaññātuṃ. Ayaṃ sattamo akkhaṇo asamayo brahmacariyavāsāya.

‘‘Puna caparaṃ…pe… ayañca puggalo majjhimesu janapadesu paccājāto hoti, so ca hoti paññavā ajaḷo aneḷamūgo, paṭibalo subhāsitadubbhāsitānamatthamaññātuṃ. Ayaṃ aṭṭhamo akkhaṇo asamayo brahmacariyavāsāya. Ime aṭṭha dhammā duppaṭivijjhā.

(Ja) ‘‘katame aṭṭha dhammā uppādetabbā? Aṭṭha mahāpurisavitakkā – appicchassāyaṃ dhammo, nāyaṃ dhammo mahicchassa. Santuṭṭhassāyaṃ dhammo, nāyaṃ dhammo asantuṭṭhassa. Pavivittassāyaṃ dhammo, nāyaṃ dhammo saṅgaṇikārāmassa. Āraddhavīriyassāyaṃ dhammo, nāyaṃ dhammo kusītassa. Upaṭṭhitasatissāyaṃ dhammo, nāyaṃ dhammo muṭṭhassatissa. Samāhitassāyaṃ dhammo, nāyaṃ dhammo asamāhitassa . Paññavato\footnote{paññāvato (sī. pī.)} ayaṃ dhammo, nāyaṃ dhammo duppaññassa. Nippapañcassāyaṃ dhammo, nāyaṃ dhammo papañcārāmassāti\footnote{nippapañcārāmassa ayaṃ dhammo nippapañcaratino, nāyaṃ dhammo papañcārāmassa papañcaratinoti (sī. syā. pī.) aṅguttarepi tatheva dissati. aṭṭhakathāṭīkā pana oloketabbā} ime aṭṭha dhammā uppādetabbā.

(Jha) ‘‘katame aṭṭha dhammā abhiññeyyā? Aṭṭha abhibhāyatanāni – ajjhattaṃ rūpasaññī eko bahiddhā rūpāni passati parittāni suvaṇṇadubbaṇṇāni , ‘tāni abhibhuyya jānāmi passāmī’ti – evaṃsaññī hoti. Idaṃ paṭhamaṃ abhibhāyatanaṃ.

‘‘Ajjhattaṃ rūpasaññī eko bahiddhā rūpāni passati appamāṇāni suvaṇṇadubbaṇṇāni, ‘tāni abhibhuyya jānāmi passāmī’ti – evaṃsaññī hoti. Idaṃ dutiyaṃ abhibhāyatanaṃ.

‘‘Ajjhattaṃ arūpasaññī eko bahiddhā rūpāni passati parittāni suvaṇṇadubbaṇṇāni, ‘tāni abhibhuyya jānāmi passāmī’ti evaṃsaññī hoti. Idaṃ tatiyaṃ abhibhāyatanaṃ.

‘‘Ajjhattaṃ arūpasaññī eko bahiddhā rūpāni passati appamāṇāni suvaṇṇadubbaṇṇāni, ‘tāni abhibhuyya jānāmi passāmī’ti evaṃsaññī hoti. Idaṃ catutthaṃ abhibhāyatanaṃ.

‘‘Ajjhattaṃ arūpasaññī eko bahiddhā rūpāni passati nīlāni nīlavaṇṇāni nīlanidassanāni nīlanibhāsāni. Seyyathāpi nāma umāpupphaṃ nīlaṃ nīlavaṇṇaṃ nīlanidassanaṃ nīlanibhāsaṃ. Seyyathā vā pana taṃ vatthaṃ bārāṇaseyyakaṃ ubhatobhāgavimaṭṭhaṃ nīlaṃ nīlavaṇṇaṃ nīlanidassanaṃ nīlanibhāsaṃ, evameva ajjhattaṃ arūpasaññī eko bahiddhā rūpāni passati nīlāni nīlavaṇṇāni nīlanidassanāni nīlanibhāsāni, ‘tāni abhibhuyya jānāmi passāmī’ti evaṃsaññī hoti. Idaṃ pañcamaṃ abhibhāyatanaṃ.

‘‘Ajjhattaṃ arūpasaññī eko bahiddhā rūpāni passati pītāni pītavaṇṇāni pītanidassanāni pītanibhāsāni. Seyyathāpi nāma kaṇikārapupphaṃ pītaṃ pītavaṇṇaṃ pītanidassanaṃ pītanibhāsaṃ. Seyyathā vā pana taṃ vatthaṃ bārāṇaseyyakaṃ ubhatobhāgavimaṭṭhaṃ pītaṃ pītavaṇṇaṃ pītanidassanaṃ pītanibhāsaṃ , evameva ajjhattaṃ arūpasaññī eko bahiddhā rūpāni passati pītāni pītavaṇṇāni pītanidassanāni pītanibhāsāni, ‘tāni abhibhuyya jānāmi passāmī’ti evaṃsaññī hoti. Idaṃ chaṭṭhaṃ abhibhāyatanaṃ.

‘‘Ajjhattaṃ arūpasaññī eko bahiddhā rūpāni passati lohitakāni lohitakavaṇṇāni lohitakanidassanāni lohitakanibhāsāni. Seyyathāpi nāma bandhujīvakapupphaṃ lohitakaṃ lohitakavaṇṇaṃ lohitakanidassanaṃ lohitakanibhāsaṃ, seyyathā vā pana taṃ vatthaṃ bārāṇaseyyakaṃ ubhatobhāgavimaṭṭhaṃ lohitakaṃ lohitakavaṇṇaṃ lohitakanidassanaṃ lohitakanibhāsaṃ, evameva ajjhattaṃ arūpasaññī eko bahiddhā rūpāni passati lohitakāni lohitakavaṇṇāni lohitakanidassanāni lohitakanibhāsāni, ‘tāni abhibhuyya jānāmi passāmī’ti evaṃsaññī hoti. Idaṃ sattamaṃ abhibhāyatanaṃ.

‘‘Ajjhattaṃ arūpasaññī eko bahiddhā rūpāni passati odātāni odātavaṇṇāni odātanidassanāni odātanibhāsāni. Seyyathāpi nāma osadhitārakā odātā odātavaṇṇā odātanidassanā odātanibhāsā, seyyathā vā pana taṃ vatthaṃ bārāṇaseyyakaṃ ubhatobhāgavimaṭṭhaṃ odātaṃ odātavaṇṇaṃ odātanidassanaṃ odātanibhāsaṃ, evameva ajjhattaṃ arūpasaññī eko bahiddhā rūpāni passati odātāni odātavaṇṇāni odātanidassanāni odātanibhāsāni, ‘tāni abhibhuyya jānāmi passāmī’ti evaṃsaññī hoti. Idaṃ aṭṭhamaṃ abhibhāyatanaṃ. Ime aṭṭha dhammā abhiññeyyā.

(Ña) ‘‘katame aṭṭha dhammā sacchikātabbā? Aṭṭha vimokkhā – rūpī rūpāni passati. Ayaṃ paṭhamo vimokkho.

‘‘Ajjhattaṃ arūpasaññī eko bahiddhā rūpāni passati. Ayaṃ dutiyo vimokkho.

‘‘Subhanteva adhimutto hoti. Ayaṃ tatiyo vimokkho.

‘‘Sabbaso rūpasaññānaṃ samatikkamā paṭighasaññānaṃ atthaṅgamā nānattasaññānaṃ amanasikārā ‘ananto ākāso’ti ākāsānañcāyatanaṃ upasampajja viharati. Ayaṃ catuttho vimokkho.

‘‘Sabbaso ākāsānañcāyatanaṃ samatikkamma ‘anantaṃ viññāṇa’nti viññāṇañcāyatanaṃ upasampajja viharati. Ayaṃ pañcamo vimokkho.

‘‘Sabbaso viññāṇañcāyatanaṃ samatikkamma ‘natthi kiñcī’ti ākiñcaññāyatanaṃ upasampajja viharati. Ayaṃ chaṭṭho vimokkho.

‘‘Sabbaso ākiñcaññāyatanaṃ samatikkamma nevasaññānāsaññāyatanaṃ upasampajja viharati. Ayaṃ sattamo vimokkho.

‘‘Sabbaso nevasaññānāsaññāyatanaṃ samatikkamma saññāvedayitanirodhaṃ upasampajja viharati. Ayaṃ aṭṭhamo vimokkho. Ime aṭṭha dhammā sacchikātabbā.

‘‘Iti ime asīti dhammā bhūtā tacchā tathā avitathā anaññathā sammā tathāgatena abhisambuddhā.

\subsubsection{Nava dhammā}

\paragraph{359.} ‘‘Nava dhammā bahukārā…pe… nava dhammā sacchikātabbā.

(Ka) ‘‘katame nava dhammā bahukārā? Nava yonisomanasikāramūlakā dhammā, yonisomanasikaroto pāmojjaṃ jāyati, pamuditassa pīti jāyati, pītimanassa kāyo passambhati, passaddhakāyo sukhaṃ vedeti, sukhino cittaṃ samādhiyati, samāhite citte yathābhūtaṃ jānāti passati, yathābhūtaṃ jānaṃ passaṃ nibbindati, nibbindaṃ virajjati, virāgā vimuccati. Ime nava dhammā bahukārā.

(Kha) ‘‘katame nava dhammā bhāvetabbā? Nava pārisuddhipadhāniyaṅgāni – sīlavisuddhi pārisuddhipadhāniyaṅgaṃ, cittavisuddhi pārisuddhipadhāniyaṅgaṃ, diṭṭhivisuddhi pārisuddhipadhāniyaṅgaṃ, kaṅkhāvitaraṇavisuddhi pārisuddhipadhāniyaṅgaṃ, maggāmaggañāṇadassana – visuddhi pārisuddhipadhāniyaṅgaṃ, paṭipadāñāṇadassanavisuddhi pārisuddhipadhāniyaṅgaṃ, ñāṇadassanavisuddhi pārisuddhipadhāniyaṅgaṃ, paññāvisuddhi pārisuddhipadhāniyaṅgaṃ, vimuttivisuddhi pārisuddhipadhāniyaṅgaṃ. Ime nava dhammā bhāvetabbā.

(Ga) ‘‘katame nava dhammā pariññeyyā? Nava sattāvāsā – santāvuso, sattā nānattakāyā nānattasaññino, seyyathāpi manussā ekacce ca devā ekacce ca vinipātikā. Ayaṃ paṭhamo sattāvāso.

‘‘Santāvuso , sattā nānattakāyā ekattasaññino, seyyathāpi devā brahmakāyikā paṭhamābhinibbattā. Ayaṃ dutiyo sattāvāso.

‘‘Santāvuso, sattā ekattakāyā nānattasaññino, seyyathāpi devā ābhassarā. Ayaṃ tatiyo sattāvāso.

‘‘Santāvuso, sattā ekattakāyā ekattasaññino, seyyathāpi devā subhakiṇhā. Ayaṃ catuttho sattāvāso.

‘‘Santāvuso, sattā asaññino appaṭisaṃvedino, seyyathāpi devā asaññasattā. Ayaṃ pañcamo sattāvāso.

‘‘Santāvuso, sattā sabbaso rūpasaññānaṃ samatikkamā paṭighasaññānaṃ atthaṅgamā nānattasaññānaṃ amanasikārā ‘ananto ākāso’ti ākāsānañcāyatanūpagā. Ayaṃ chaṭṭho sattāvāso.

‘‘Santāvuso, sattā sabbaso ākāsānañcāyatanaṃ samatikkamma ‘anantaṃ viññāṇa’nti viññāṇañcāyatanūpagā. Ayaṃ sattamo sattāvāso.

‘‘Santāvuso, sattā sabbaso viññāṇañcāyatanaṃ samatikkamma ‘natthi kiñcī’ti ākiñcaññāyatanūpagā. Ayaṃ aṭṭhamo sattāvāso.

‘‘Santāvuso, sattā sabbaso ākiñcaññāyatanaṃ samatikkamma nevasaññānāsaññāyatanūpagā. Ayaṃ navamo sattāvāso. Ime nava dhammā pariññeyyā.

(Gha) ‘‘katame nava dhammā pahātabbā? Nava taṇhāmūlakā dhammā – taṇhaṃ paṭicca pariyesanā, pariyesanaṃ paṭicca lābho, lābhaṃ paṭicca vinicchayo, vinicchayaṃ paṭicca chandarāgo , chandarāgaṃ paṭicca ajjhosānaṃ, ajjhosānaṃ paṭicca pariggaho, pariggahaṃ paṭicca macchariyaṃ, macchariyaṃ paṭicca ārakkho, ārakkhādhikaraṇaṃ\footnote{ārakkhādhikaraṇaṃ paṭicca (syā. pī. ka.)} daṇḍādānasatthādānakalahaviggahavivādatuvaṃtuvaṃpesuññamusāvādā aneke pāpakā akusalā dhammā sambhavanti. Ime nava dhammā pahātabbā.

(Ṅa) ‘‘katame nava dhammā hānabhāgiyā? Nava āghātavatthūni – ‘anatthaṃ me acarī’ti āghātaṃ bandhati, ‘anatthaṃ me caratī’ti āghātaṃ bandhati, ‘anatthaṃ me carissatī’ti āghātaṃ bandhati; ‘piyassa me manāpassa anatthaṃ acarī’ti āghātaṃ bandhati…pe… ‘anatthaṃ caratī’ti āghātaṃ bandhati…pe… ‘anatthaṃ carissatī’ti āghātaṃ bandhati; ‘appiyassa me amanāpassa atthaṃ acarī’ti āghātaṃ bandhati…pe… ‘atthaṃ caratī’ti āghātaṃ bandhati…pe… ‘atthaṃ carissatī’ti āghātaṃ bandhati. Ime nava dhammā hānabhāgiyā.

(Ca) ‘‘katame nava dhammā visesabhāgiyā? Nava āghātapaṭivinayā – ‘anatthaṃ me acari, taṃ kutettha labbhā’ti āghātaṃ paṭivineti; ‘anatthaṃ me carati, taṃ kutettha labbhā’ti āghātaṃ paṭivineti; ‘anatthaṃ me carissati, taṃ kutettha labbhā’ti āghātaṃ paṭivineti; ‘piyassa me manāpassa anatthaṃ acari…pe… anatthaṃ carati…pe… anatthaṃ carissati, taṃ kutettha labbhā’ti āghātaṃ paṭivineti; ‘appiyassa me amanāpassa atthaṃ acari…pe… atthaṃ carati…pe… atthaṃ carissati, taṃ kutettha labbhā’ti āghātaṃ paṭivineti. Ime nava dhammā visesabhāgiyā.

(Cha) ‘‘katame nava dhammā duppaṭivijjhā? Nava nānattā – dhātunānattaṃ paṭicca uppajjati phassanānattaṃ, phassanānattaṃ paṭicca uppajjati vedanānānattaṃ, vedanānānattaṃ paṭicca uppajjati saññānānattaṃ, saññānānattaṃ paṭicca uppajjati saṅkappanānattaṃ, saṅkappanānattaṃ paṭicca uppajjati chandanānattaṃ, chandanānattaṃ paṭicca uppajjati pariḷāhanānattaṃ, pariḷāhanānattaṃ paṭicca uppajjati pariyesanānānattaṃ, pariyesanānānattaṃ paṭicca uppajjati lābhanānattaṃ. Ime nava dhammā duppaṭivijjhā.

(Ja) ‘‘katame nava dhammā uppādetabbā? Nava saññā – asubhasaññā, maraṇasaññā, āhārepaṭikūlasaññā , sabbalokeanabhiratisaññā\footnote{anabhiratasaññā (syā. ka.)}, aniccasaññā, anicce dukkhasaññā, dukkhe anattasaññā, pahānasaññā, virāgasaññā. Ime nava dhammā uppādetabbā.

(Jha) ‘‘katame nava dhammā abhiññeyyā? Nava anupubbavihārā – idhāvuso, bhikkhu vivicceva kāmehi vivicca akusalehi dhammehi savitakkaṃ savicāraṃ vivekajaṃ pītisukhaṃ paṭhamaṃ jhānaṃ upasampajja viharati. Vitakkavicārānaṃ vūpasamā…pe… dutiyaṃ jhānaṃ upasampajja viharati. Pītiyā ca virāgā …pe… tatiyaṃ jhānaṃ upasampajja viharati. Sukhassa ca pahānā…pe… catutthaṃ jhānaṃ upasampajja viharati. Sabbaso rūpasaññānaṃ samatikkamā…pe… ākāsānañcāyatanaṃ upasampajja viharati. Sabbaso ākāsānañcāyatanaṃ samatikkamma ‘anantaṃ viññāṇa’nti viññāṇañcāyatanaṃ upasampajja viharati. Sabbaso viññāṇañcāyatanaṃ samatikkamma ‘natthi kiñcī’ti ākiñcaññāyatanaṃ upasampajja viharati. Sabbaso ākiñcaññāyatanaṃ samatikkamma nevasaññānāsaññāyatanaṃ upasampajja viharati. Sabbaso nevasaññānāsaññāyatanaṃ samatikkamma saññāvedayitanirodhaṃ upasampajja viharati. Ime nava dhammā abhiññeyyā.

(Ña) ‘‘katame nava dhammā sacchikātabbā? Nava anupubbanirodhā – paṭhamaṃ jhānaṃ samāpannassa kāmasaññā niruddhā hoti, dutiyaṃ jhānaṃ samāpannassa vitakkavicārā niruddhā honti, tatiyaṃ jhānaṃ samāpannassa pīti niruddhā hoti, catutthaṃ jhānaṃ samāpannassa assāsapassāssā niruddhā honti, ākāsānañcāyatanaṃ samāpannassa rūpasaññā niruddhā hoti, viññāṇañcāyatanaṃ samāpannassa ākāsānañcāyatanasaññā niruddhā hoti, ākiñcaññāyatanaṃ samāpannassa viññāṇañcāyatanasaññā niruddhā hoti, nevasaññānāsaññāyatanaṃ samāpannassa ākiñcaññāyatanasaññā niruddhā hoti, saññāvedayitanirodhaṃ samāpannassa saññā ca vedanā ca niruddhā honti. Ime nava dhammā sacchikātabbā.

‘‘Iti ime navuti dhammā bhūtā tacchā tathā avitathā anaññathā sammā tathāgatena abhisambuddhā.

\subsubsection{Dasa dhammā}

\paragraph{360.} ‘‘Dasa dhammā bahukārā…pe… dasa dhammā sacchikātabbā.

(Ka) ‘‘katame dasa dhammā bahukārā? Dasa nāthakaraṇādhammā – idhāvuso, bhikkhu sīlavā hoti, pātimokkhasaṃvarasaṃvuto viharati ācāragocarasampanno, aṇumattesu vajjesu bhayadassāvī samādāya sikkhati sikkhāpadesu, yaṃpāvuso, bhikkhu sīlavā hoti…pe… sikkhati sikkhāpadesu. Ayampi dhammo nāthakaraṇo.

‘‘Puna caparaṃ, āvuso, bhikkhu bahussuto …pe… diṭṭhiyā suppaṭividdhā, yaṃpāvuso, bhikkhu bahussuto…pe… ayampi dhammo nāthakaraṇo.

‘‘Puna caparaṃ, āvuso, bhikkhu kalyāṇamitto hoti kalyāṇasahāyo kalyāṇasampavaṅko. Yaṃpāvuso, bhikkhu…pe… kalyāṇasampavaṅko. Ayampi dhammo nāthakaraṇo.

‘‘Puna caparaṃ, āvuso, bhikkhu suvaco hoti sovacassakaraṇehi dhammehi samannāgato, khamo padakkhiṇaggāhī anusāsaniṃ. Yaṃpāvuso, bhikkhu…pe… anusāsaniṃ. Ayampi dhammo nāthakaraṇo.

‘‘Puna caparaṃ, āvuso, bhikkhu yāni tāni sabrahmacārīnaṃ uccāvacāni kiṃkaraṇīyāni tattha dakkho hoti analaso tatrupāyāya vīmaṃsāya samannāgato, alaṃ kātuṃ, alaṃ saṃvidhātuṃ. Yaṃpāvuso, bhikkhu…pe… alaṃ saṃvidhātuṃ. Ayampi dhammo nāthakaraṇo.

‘‘Puna caparaṃ, āvuso, bhikkhu dhammakāmo hoti piyasamudāhāro abhidhamme abhivinaye uḷārapāmojjo. Yaṃpāvuso, bhikkhu…pe… uḷārapāmojjo. Ayampi dhammo nāthakaraṇo.

‘‘Puna caparaṃ, āvuso, bhikkhu santuṭṭho hoti itarītarehi cīvarapiṇḍapātasenāsanagilānappaccayabhesajjaparikkhārehi. Yaṃpāvuso, bhikkhu …pe… ayampi dhammo nāthakaraṇo.

‘‘Puna caparaṃ, āvuso, bhikkhu āraddhavīriyo viharati…pe… kusalesu dhammesu. Yaṃpāvuso, bhikkhu…pe… ayampi dhammo nāthakaraṇo.

‘‘Puna caparaṃ, āvuso, bhikkhu satimā hoti, paramena satinepakkena samannāgato, cirakatampi cirabhāsitampi saritā anussaritā. Yaṃpāvuso, bhikkhu…pe… ayampi dhammo nāthakaraṇo.

‘‘Puna caparaṃ, āvuso, bhikkhu paññavā hoti udayatthagāminiyā paññāya samannāgato, ariyāya nibbedhikāya sammā dukkhakkhayagāminiyā. Yaṃpāvuso, bhikkhu…pe… ayampi dhammo nāthakaraṇo. Ime dasa dhammā bahukārā.

(Kha) ‘‘katame dasa dhammā bhāvetabbā? Dasa kasiṇāyatanāni – pathavīkasiṇameko sañjānāti uddhaṃ adho tiriyaṃ advayaṃ appamāṇaṃ. Āpokasiṇameko sañjānāti…pe… tejokasiṇameko sañjānāti… vāyokasiṇameko sañjānāti… nīlakasiṇameko sañjānāti… pītakasiṇameko sañjānāti… lohitakasiṇameko sañjānāti… odātakasiṇameko sañjānāti… ākāsakasiṇameko sañjānāti… viññāṇakasiṇameko sañjānāti uddhaṃ adho tiriyaṃ advayaṃ appamāṇaṃ . Ime dasa dhammā bhāvetabbā.

(Ga) ‘‘katame dasa dhammā pariññeyyā? Dasāyatanāni – cakkhāyatanaṃ, rūpāyatanaṃ, sotāyatanaṃ, saddāyatanaṃ, ghānāyatanaṃ, gandhāyatanaṃ, jivhāyatanaṃ, rasāyatanaṃ, kāyāyatanaṃ, phoṭṭhabbāyatanaṃ. Ime dasa dhammā pariññeyyā.

(Gha) ‘‘katame dasa dhammā pahātabbā? Dasa micchattā – micchādiṭṭhi, micchāsaṅkappo, micchāvācā, micchākammanto, micchāājīvo, micchāvāyāmo, micchāsati, micchāsamādhi, micchāñāṇaṃ, micchāvimutti. Ime dasa dhammā pahātabbā.

(Ṅa) ‘‘katame dasa dhammā hānabhāgiyā? Dasa akusalakammapathā – pāṇātipāto, adinnādānaṃ, kāmesumicchācāro, musāvādo, pisuṇā vācā, pharusā vācā, samphappalāpo, abhijjhā, byāpādo, micchādiṭṭhi. Ime dasa dhammā hānabhāgiyā.

(Ca) ‘‘katame dasa dhammā visesabhāgiyā? Dasa kusalakammapathā – pāṇātipātā veramaṇī, adinnādānā veramaṇī, kāmesumicchācārā veramaṇī, musāvādā veramaṇī, pisuṇāya vācāya veramaṇī, pharusāya vācāya veramaṇī, samphappalāpā veramaṇī, anabhijjhā, abyāpādo, sammādiṭṭhi. Ime dasa dhammā visesabhāgiyā.

(Cha) ‘‘katame dasa dhammā duppaṭivijjhā? Dasa ariyavāsā – idhāvuso , bhikkhu pañcaṅgavippahīno hoti, chaḷaṅgasamannāgato, ekārakkho, caturāpasseno, paṇunnapaccekasacco, samavayasaṭṭhesano, anāvilasaṅkappo, passaddhakāyasaṅkhāro, suvimuttacitto, suvimuttapañño.

‘‘Kathañcāvuso , bhikkhu pañcaṅgavippahīno hoti? Idhāvuso, bhikkhuno kāmacchando pahīno hoti, byāpādo pahīno hoti, thinamiddhaṃ pahīnaṃ hoti, uddhaccakukkuccaṃ pahīnaṃ hoti, vicikicchā pahīnā hoti. Evaṃ kho, āvuso, bhikkhu pañcaṅgavippahīno hoti.

‘‘Kathañcāvuso, bhikkhu chaḷaṅgasamannāgato hoti? Idhāvuso, bhikkhu cakkhunā rūpaṃ disvā neva sumano hoti na dummano, upekkhako viharati sato sampajāno. Sotena saddaṃ sutvā…pe… ghānena gandhaṃ ghāyitvā… jivhāya rasaṃ sāyitvā… kāyena phoṭṭhabbaṃ phusitvā… manasā dhammaṃ viññāya neva sumano hoti na dummano, upekkhako viharati sato sampajāno. Evaṃ kho, āvuso, bhikkhu chaḷaṅgasamannāgato hoti.

‘‘Kathañcāvuso, bhikkhu ekārakkho hoti? Idhāvuso, bhikkhu satārakkhena cetasā samannāgato hoti. Evaṃ kho, āvuso, bhikkhu ekārakkho hoti.

‘‘Kathañcāvuso, bhikkhu caturāpasseno hoti? Idhāvuso, bhikkhu saṅkhāyekaṃ paṭisevati, saṅkhāyekaṃ adhivāseti, saṅkhāyekaṃ parivajjeti, saṅkhāyekaṃ vinodeti. Evaṃ kho, āvuso, bhikkhu caturāpasseno hoti.

‘‘Kathañcāvuso, bhikkhu paṇunnapaccekasacco hoti? Idhāvuso, bhikkhuno yāni tāni puthusamaṇabrāhmaṇānaṃ puthupaccekasaccāni, sabbāni tāni nunnāni honti paṇunnāni cattāni vantāni muttāni pahīnāni paṭinissaṭṭhāni. Evaṃ kho, āvuso, bhikkhu paṇunnapaccekasacco hoti.

‘‘Kathañcāvuso, bhikkhu samavayasaṭṭhesano hoti? Idhāvuso, bhikkhuno kāmesanā pahīnā hoti, bhavesanā pahīnā hoti, brahmacariyesanā paṭippassaddhā. Evaṃ kho, āvuso, bhikkhu samavayasaṭṭhesano hoti.

‘‘Kathañcāvuso , bhikkhu anāvilasaṅkappā hoti? Idhāvuso, bhikkhuno kāmasaṅkappo pahīno hoti, byāpādasaṅkappo pahīno hoti, vihiṃsāsaṅkappo pahīno hoti. Evaṃ kho, āvuso, bhikkhu anāvilasaṅkappo hoti.

‘‘Kathañcāvuso, bhikkhu passaddhakāyasaṅkhāro hoti? Idhāvuso, bhikkhu sukhassa ca pahānā dukkhassa ca pahānā pubbeva somanassadomanassānaṃ atthaṅgamā adukkhamasukhaṃ upekkhāsatipārisuddhiṃ catutthaṃ jhānaṃ upasampajja viharati. Evaṃ kho, āvuso, bhikkhu passaddhakāyasaṅkhāro hoti.

‘‘Kathañcāvuso, bhikkhu suvimuttacitto hoti? Idhāvuso, bhikkhuno rāgā cittaṃ vimuttaṃ hoti, dosā cittaṃ vimuttaṃ hoti, mohā cittaṃ vimuttaṃ hoti. Evaṃ kho, āvuso, bhikkhu suvimuttacitto hoti.

‘‘Kathañcāvuso, bhikkhu suvimuttapañño hoti? Idhāvuso, bhikkhu ‘rāgo me pahīno ucchinnamūlo tālāvatthukato anabhāvaṃkato āyatiṃ anuppādadhammo’ti pajānāti. ‘Doso me pahīno…pe… āyatiṃ anuppādadhammo’ti pajānāti. ‘Moho me pahīno …pe… āyatiṃ anuppādadhammo’ti pajānāti. Evaṃ kho, āvuso, bhikkhu suvimuttapañño hoti. Ime dasa dhammā duppaṭivijjhā.

(Ja) ‘‘katame dasa dhammā uppādetabbā? Dasa saññā – asubhasaññā, maraṇasaññā, āhārepaṭikūlasaññā, sabbalokeanabhiratisaññā, aniccasaññā, anicce dukkhasaññā, dukkhe anattasaññā, pahānasaññā, virāgasaññā, nirodhasaññā. Ime dasa dhammā uppādetabbā.

(Jha) ‘‘katame dasa dhammā abhiññeyyā? Dasa nijjaravatthūni – sammādiṭṭhissa micchādiṭṭhi nijjiṇṇā hoti. Ye ca micchādiṭṭhipaccayā aneke pāpakā akusalā dhammā sambhavanti, te cassa nijjiṇṇā honti. Sammāsaṅkappassa micchāsaṅkappo…pe… sammāvācassa micchāvācā… sammākammantassa micchākammanto… sammāājīvassa micchāājīvo… sammāvāyāmassa micchāvāyāmo… sammāsatissa micchāsati… sammāsamādhissa micchāsamādhi… sammāñāṇassa micchāñāṇaṃ nijjiṇṇaṃ hoti. Sammāvimuttissa micchāvimutti nijjiṇṇā hoti. Ye ca micchāvimuttipaccayā aneke pāpakā akusalā dhammā sambhavanti, te cassa nijjiṇṇā honti. Ime dasa dhammā abhiññeyyā.

(Ña) ‘‘katame dasa dhammā sacchikātabbā? Dasa asekkhā dhammā – asekkhā sammādiṭṭhi, asekkho sammāsaṅkappo, asekkhā sammāvācā, asekkho sammākammanto, asekkho sammāājīvo, asekkho sammāvāyāmo, asekkhā sammāsati, asekkho sammāsamādhi, asekkhaṃ sammāñāṇaṃ, asekkhā sammāvimutti. Ime dasa dhammā sacchikātabbā.

‘‘Iti ime satadhammā bhūtā tacchā tathā avitathā anaññathā sammā tathāgatena abhisambuddhā’’ti. Idamavocāyasmā sāriputto. Attamanā te bhikkhū āyasmato sāriputtassa bhāsitaṃ abhinandunti.

\xsectionEnd{Dasuttarasuttaṃ niṭṭhitaṃ ekādasamaṃ. \\ Pāthikavaggo\footnote{pāṭikavaggo (sī. syā. pī.)} niṭṭhito.}

\paragraph{}
Tassuddānaṃ –

Pāthiko ca\footnote{pāṭikañca (syā. kaṃ.)} udumbaraṃ\footnote{pāṭikodumbarīceva (sī. pī.)}, cakkavatti aggaññakaṃ;

Sampasādanapāsādaṃ\footnote{sampasādañca pāsādaṃ (sī. syā. kaṃ. pī.)}, mahāpurisalakkhaṇaṃ.

Siṅgālāṭānāṭiyakaṃ , saṅgīti ca dasuttaraṃ;

Ekādasahi suttehi, pāthikavaggoti vuccati.

\xsectionEnd{Pāthikavaggapāḷi niṭṭhitā. \\ Tīhi vaggehi paṭimaṇḍito sakalo \\ Dīghanikāyo samatto.}
