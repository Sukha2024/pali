\section{Mahālisuttaṃ}

\subsubsection{Brāhmaṇadūtavatthu}

\paragraph{359.} Evaṃ me sutaṃ – uekaṃ samayaṃ bhagavā vesāliyaṃ viharati mahāvane kūṭāgārasālāyaṃ. Tena kho pana samayena sambahulā kosalakā ca brāhmaṇadūtā māgadhakā ca brāhmaṇadūtā vesāliyaṃ paṭivasanti kenacideva karaṇīyena. Assosuṃ kho te kosalakā ca brāhmaṇadūtā māgadhakā ca brāhmaṇadūtā – ‘‘samaṇo khalu, bho, gotamo sakyaputto sakyakulā pabbajito vesāliyaṃ viharati mahāvane kūṭāgārasālāyaṃ. Taṃ kho pana bhavantaṃ gotamaṃ evaṃ kalyāṇo kittisaddo abbhuggato – ‘itipi so bhagavā arahaṃ sammāsambuddho vijjācaraṇasampanno sugato lokavidū anuttaro purisadammasārathi satthā devamanussānaṃ buddho bhagavā’. So imaṃ lokaṃ sadevakaṃ samārakaṃ sabrahmakaṃ sassamaṇabrāhmaṇiṃ pajaṃ sadevamanussaṃ sayaṃ abhiññā sacchikatvā pavedeti. So dhammaṃ deseti ādikalyāṇaṃ majjhekalyāṇaṃ pariyosānakalyāṇaṃ sātthaṃ sabyañjanaṃ kevalaparipuṇṇaṃ parisuddhaṃ brahmacariyaṃ pakāseti. Sādhu kho pana tathārūpānaṃ arahataṃ dassanaṃ hotī’’ti.

\paragraph{360.} Atha kho te kosalakā ca brāhmaṇadūtā māgadhakā ca brāhmaṇadūtā yena mahāvanaṃ kūṭāgārasālā tenupasaṅkamiṃsu. Tena kho pana samayena āyasmā nāgito bhagavato upaṭṭhāko hoti. Atha kho te kosalakā ca brāhmaṇadūtā māgadhakā ca brāhmaṇadūtā yenāyasmā nāgito tenupasaṅkamiṃsu. Upasaṅkamitvā āyasmantaṃ nāgitaṃ etadavocuṃ – ‘‘kahaṃ nu kho, bho nāgita, etarahi so bhavaṃ gotamo viharati? Dassanakāmā hi mayaṃ taṃ bhavantaṃ gotama’’nti. ‘‘Akālo kho, āvuso, bhagavantaṃ dassanāya, paṭisallīno bhagavā’’ti. Atha kho te kosalakā ca brāhmaṇadūtā māgadhakā ca brāhmaṇadūtā tattheva ekamantaṃ nisīdiṃsu – ‘‘disvāva mayaṃ taṃ bhavantaṃ gotamaṃ gamissāmā’’ti.

\subsubsection{Oṭṭhaddhalicchavīvatthu}

\paragraph{361.} Oṭṭhaddhopi licchavī mahatiyā licchavīparisāya saddhiṃ yena mahāvanaṃ kūṭāgārasālā yenāyasmā nāgito tenupasaṅkami; upasaṅkamitvā āyasmantaṃ nāgitaṃ abhivādetvā ekamantaṃ aṭṭhāsi. Ekamantaṃ ṭhito kho oṭṭhaddhopi licchavī āyasmantaṃ nāgitaṃ etadavoca – ‘‘kahaṃ nu kho, bhante nāgita, etarahi so bhagavā viharati arahaṃ sammāsambuddho, dassanakāmā hi mayaṃ taṃ bhagavantaṃ arahantaṃ sammāsambuddha’’nti. ‘‘Akālo kho, mahāli, bhagavantaṃ dassanāya, paṭisallīno bhagavā’’ti. Oṭṭhaddhopi licchavī tattheva ekamantaṃ nisīdi – ‘‘disvāva ahaṃ taṃ bhagavantaṃ gamissāmi arahantaṃ sammāsambuddha’’nti.

\paragraph{362.} Atha kho sīho samaṇuddeso yenāyasmā nāgito tenupasaṅkami; upasaṅkamitvā āyasmantaṃ nāgitaṃ abhivādetvā ekamantaṃ aṭṭhāsi. Ekamantaṃ ṭhito kho sīho samaṇuddeso āyasmantaṃ nāgitaṃ etadavoca – ‘‘ete, bhante kassapa, sambahulā kosalakā ca brāhmaṇadūtā māgadhakā ca brāhmaṇadūtā idhūpasaṅkantā bhagavantaṃ dassanāya; oṭṭhaddhopi licchavī mahatiyā licchavīparisāya saddhiṃ idhūpasaṅkanto bhagavantaṃ dassanāya, sādhu, bhante kassapa, labhataṃ esā janatā bhagavantaṃ dassanāyā’’ti. ‘‘Tena hi, sīha, tvaññeva bhagavato ārocehī’’ti. ‘‘Evaṃ, bhante’’ti kho sīho samaṇuddeso āyasmato nāgitassa paṭissutvā yena bhagavā tenupasaṅkami; upasaṅkamitvā bhagavantaṃ abhivādetvā ekamantaṃ aṭṭhāsi. Ekamantaṃ ṭhito kho sīho samaṇuddeso bhagavantaṃ etadavoca – ‘‘ete, bhante, sambahulā kosalakā ca brāhmaṇadūtā māgadhakā ca brāhmaṇadūtā idhūpasaṅkantā bhagavantaṃ dassanāya, oṭṭhaddhopi licchavī mahatiyā licchavīparisāya saddhiṃ idhūpasaṅkanto bhagavantaṃ dassanāya. Sādhu, bhante, labhataṃ esā janatā bhagavantaṃ dassanāyā’’ti. ‘‘Tena hi, sīha, vihārapacchāyāyaṃ āsanaṃ paññapehī’’ti. ‘‘Evaṃ, bhante’’ti kho sīho samaṇuddeso bhagavato paṭissutvā vihārapacchāyāyaṃ āsanaṃ paññapesi.

\paragraph{363.} Atha kho bhagavā vihārā nikkhamma vihārapacchāyāyaṃ paññatte āsane nisīdi. Atha kho te kosalakā ca brāhmaṇadūtā māgadhakā ca brāhmaṇadūtā yena bhagavā tenupasaṅkamiṃsu; upasaṅkamitvā bhagavatā saddhiṃ sammodiṃsu. Sammodanīyaṃ kathaṃ sāraṇīyaṃ vītisāretvā ekamantaṃ nisīdiṃsu. Oṭṭhaddhopi licchavī mahatiyā licchavīparisāya saddhiṃ yena bhagavā tenupasaṅkami; upasaṅkamitvā bhagavantaṃ abhivādetvā ekamantaṃ nisīdi.

\paragraph{364.} Ekamantaṃ nisinno kho oṭṭhaddho licchavī bhagavantaṃ etadavoca – ‘‘purimāni, bhante, divasāni purimatarāni sunakkhatto licchaviputto yenāhaṃ tenupasaṅkami; upasaṅkamitvā maṃ etadavoca – ‘yadagge ahaṃ, mahāli, bhagavantaṃ upanissāya viharāmi, na ciraṃ tīṇi vassāni, dibbāni hi kho rūpāni passāmi piyarūpāni kāmūpasaṃhitāni rajanīyāni, no ca kho dibbāni saddāni suṇāmi piyarūpāni kāmūpasaṃhitāni rajanīyānī’ti. Santāneva nu kho, bhante, sunakkhatto licchaviputto dibbāni saddāni nāssosi piyarūpāni kāmūpasaṃhitāni rajanīyāni, udāhu asantānī’’ti?

\subsubsection{Ekaṃsabhāvitasamādhi}

\paragraph{365.} ‘‘Santāneva kho, mahāli, sunakkhatto licchaviputto dibbāni saddāni nāssosi piyarūpāni kāmūpasaṃhitāni rajanīyāni, no asantānī’’ti. ‘‘Ko nu kho, bhante, hetu, ko paccayo, yena santāneva sunakkhatto licchaviputto dibbāni saddāni nāssosi piyarūpāni kāmūpasaṃhitāni rajanīyāni, no asantānī’’ti?

\paragraph{366.} ‘‘Idha, mahāli, bhikkhuno puratthimāya disāya ekaṃsabhāvito samādhi hoti dibbānaṃ rūpānaṃ dassanāya piyarūpānaṃ kāmūpasaṃhitānaṃ rajanīyānaṃ, no ca kho dibbānaṃ saddānaṃ savanāya piyarūpānaṃ kāmūpasaṃhitānaṃ rajanīyānaṃ. So puratthimāya disāya ekaṃsabhāvite samādhimhi dibbānaṃ rūpānaṃ dassanāya piyarūpānaṃ kāmūpasaṃhitānaṃ rajanīyānaṃ, no ca kho dibbānaṃ saddānaṃ savanāya piyarūpānaṃ kāmūpasaṃhitānaṃ rajanīyānaṃ. Puratthimāya disāya dibbāni rūpāni passati piyarūpāni kāmūpasaṃhitāni rajanīyāni, no ca kho dibbāni saddāni suṇāti piyarūpāni kāmūpasaṃhitāni rajanīyāni. Taṃ kissa hetu? Evañhetaṃ, mahāli, hoti bhikkhuno puratthimāya disāya ekaṃsabhāvite samādhimhi dibbānaṃ rūpānaṃ dassanāya piyarūpānaṃ kāmūpasaṃhitānaṃ rajanīyānaṃ, no ca kho dibbānaṃ saddānaṃ savanāya piyarūpānaṃ kāmūpasaṃhitānaṃ rajanīyānaṃ.

\paragraph{367.} ‘‘Puna caparaṃ, mahāli, bhikkhuno dakkhiṇāya disāya…pe… pacchimāya disāya … uttarāya disāya… uddhamadho tiriyaṃ ekaṃsabhāvito samādhi hoti dibbānaṃ rūpānaṃ dassanāya piyarūpānaṃ kāmūpasaṃhitānaṃ rajanīyānaṃ, no ca kho dibbānaṃ saddānaṃ savanāya piyarūpānaṃ kāmūpasaṃhitānaṃ rajanīyānaṃ. So uddhamadho tiriyaṃ ekaṃsabhāvite samādhimhi dibbānaṃ rūpānaṃ dassanāya piyarūpānaṃ kāmūpasaṃhitānaṃ rajanīyānaṃ, no ca kho dibbānaṃ saddānaṃ savanāya piyarūpānaṃ kāmūpasaṃhitānaṃ rajanīyānaṃ. Uddhamadho tiriyaṃ dibbāni rūpāni passati piyarūpāni kāmūpasaṃhitāni rajanīyāni, no ca kho dibbāni saddāni suṇāti piyarūpāni kāmūpasaṃhitāni rajanīyāni. Taṃ kissa hetu? Evañhetaṃ, mahāli, hoti bhikkhuno uddhamadho tiriyaṃ ekaṃsabhāvite samādhimhi dibbānaṃ rūpānaṃ dassanāya piyarūpānaṃ kāmūpasaṃhitānaṃ rajanīyānaṃ, no ca kho dibbānaṃ saddānaṃ savanāya piyarūpānaṃ kāmūpasaṃhitānaṃ rajanīyānaṃ.

\paragraph{368.} ‘‘Idha, mahāli, bhikkhuno puratthimāya disāya ekaṃsabhāvito samādhi hoti dibbānaṃ saddānaṃ savanāya piyarūpānaṃ kāmūpasaṃhitānaṃ rajanīyānaṃ, no ca kho dibbānaṃ rūpānaṃ dassanāya piyarūpānaṃ kāmūpasaṃhitānaṃ rajanīyānaṃ. So puratthimāya disāya ekaṃsabhāvite samādhimhi dibbānaṃ saddānaṃ savanāya piyarūpānaṃ kāmūpasaṃhitānaṃ rajanīyānaṃ, no ca kho dibbānaṃ rūpānaṃ dassanāya piyarūpānaṃ kāmūpasaṃhitānaṃ rajanīyānaṃ. Puratthimāya disāya dibbāni saddāni suṇāti piyarūpāni kāmūpasaṃhitāni rajanīyāni, no ca kho dibbāni rūpāni passati piyarūpāni kāmūpasaṃhitāni rajanīyāni. Taṃ kissa hetu? Evañhetaṃ, mahāli, hoti bhikkhuno puratthimāya disāya ekaṃsabhāvite samādhimhi dibbānaṃ saddānaṃ savanāya piyarūpānaṃ kāmūpasaṃhitānaṃ rajanīyānaṃ, no ca kho dibbānaṃ rūpānaṃ dassanāya piyarūpānaṃ kāmūpasaṃhitānaṃ rajanīyānaṃ.

\paragraph{369.} ‘‘Puna caparaṃ, mahāli, bhikkhuno dakkhiṇāya disāya…pe… pacchimāya disāya… uttarāya disāya… uddhamadho tiriyaṃ ekaṃsabhāvito samādhi hoti dibbānaṃ saddānaṃ savanāya piyarūpānaṃ kāmūpasaṃhitānaṃ rajanīyānaṃ, no ca kho dibbānaṃ rūpānaṃ dassanāya piyarūpānaṃ kāmūpasaṃhitānaṃ rajanīyānaṃ. So uddhamadho tiriyaṃ ekaṃsabhāvite samādhimhi dibbānaṃ saddānaṃ savanāya piyarūpānaṃ kāmūpasaṃhitānaṃ rajanīyānaṃ, no ca kho dibbānaṃ rūpānaṃ dassanāya piyarūpānaṃ kāmūpasaṃhitānaṃ rajanīyānaṃ. Uddhamadho tiriyaṃ dibbāni saddāni suṇāti piyarūpāni kāmūpasaṃhitāni rajanīyāni, no ca kho dibbāni rūpāni passati piyarūpāni kāmūpasaṃhitāni rajanīyāni. Taṃ kissa hetu? Evañhetaṃ, mahāli, hoti bhikkhuno uddhamadho tiriyaṃ ekaṃsabhāvite samādhimhi dibbānaṃ saddānaṃ savanāya piyarūpānaṃ kāmūpasaṃhitānaṃ rajanīyānaṃ, no ca kho dibbānaṃ rūpānaṃ dassanāya piyarūpānaṃ kāmūpasaṃhitānaṃ rajanīyānaṃ.

\paragraph{370.} ‘‘Idha, mahāli, bhikkhuno puratthimāya disāya ubhayaṃsabhāvito samādhi hoti dibbānañca rūpānaṃ dassanāya piyarūpānaṃ kāmūpasaṃhitānaṃ rajanīyānaṃ dibbānañca saddānaṃ savanāya piyarūpānaṃ kāmūpasaṃhitānaṃ rajanīyānaṃ. So puratthimāya disāya ubhayaṃsabhāvite samādhimhi dibbānañca rūpānaṃ dassanāya piyarūpānaṃ kāmūpasaṃhitānaṃ rajanīyānaṃ, dibbānañca saddānaṃ savanāya piyarūpānaṃ kāmūpasaṃhitānaṃ rajanīyānaṃ. Puratthimāya disāya dibbāni ca rūpāni passati piyarūpāni kāmūpasaṃhitāni rajanīyāni, dibbāni ca saddāni suṇāti piyarūpāni kāmūpasaṃhitāni rajanīyāni. Taṃ kissa hetu? Evañhetaṃ, mahāli, hoti bhikkhuno puratthimāya disāya ubhayaṃsabhāvite samādhimhi dibbānañca rūpānaṃ dassanāya piyarūpānaṃ kāmūpasaṃhitānaṃ rajanīyānaṃ dibbānañca saddānaṃ savanāya piyarūpānaṃ kāmūpasaṃhitānaṃ rajanīyānaṃ.

\paragraph{371.} ‘‘Puna caparaṃ, mahāli, bhikkhuno dakkhiṇāya disāya…pe… pacchimāya disāya… uttarāya disāya… uddhamadho tiriyaṃ ubhayaṃsabhāvito samādhi hoti dibbānañca rūpānaṃ dassanāya piyarūpānaṃ kāmūpasaṃhitānaṃ rajanīyānaṃ, dibbānañca saddānaṃ savanāya piyarūpānaṃ kāmūpasaṃhitānaṃ rajanīyānaṃ. So uddhamadho tiriyaṃ ubhayaṃsabhāvite samādhimhi dibbānañca rūpānaṃ dassanāya piyarūpānaṃ kāmūpasaṃhitānaṃ rajanīyānaṃ dibbānañca saddānaṃ savanāya piyarūpānaṃ kāmūpasaṃhitānaṃ rajanīyānaṃ. Uddhamadho tiriyaṃ dibbāni ca rūpāni passati piyarūpāni kāmūpasaṃhitāni rajanīyāni, dibbāni ca saddāni suṇāti piyarūpāni kāmūpasaṃhitāni rajanīyāni. Taṃ kissa hetu? Evañhetaṃ, mahāli, hoti bhikkhuno uddhamadho tiriyaṃ ubhayaṃsabhāvite samādhimhi dibbānañca rūpānaṃ dassanāya piyarūpānaṃ kāmūpasaṃhitānaṃ rajanīyānaṃ, dibbānañca saddānaṃ savanāya piyarūpānaṃ kāmūpasaṃhitānaṃ rajanīyānaṃ. Ayaṃ kho mahāli, hetu, ayaṃ paccayo, yena santāneva sunakkhatto licchaviputto dibbāni saddāni nāssosi piyarūpāni kāmūpasaṃhitāni rajanīyāni, no asantānī’’ti.

\paragraph{372.} ‘‘Etāsaṃ nūna, bhante, samādhibhāvanānaṃ sacchikiriyāhetu bhikkhū bhagavati brahmacariyaṃ carantī’’ti. ‘‘Na kho, mahāli, etāsaṃ samādhibhāvanānaṃ sacchikiriyāhetu bhikkhū mayi brahmacariyaṃ caranti. Atthi kho, mahāli, aññeva dhammā uttaritarā ca paṇītatarā ca, yesaṃ sacchikiriyāhetu bhikkhū mayi brahmacariyaṃ carantī’’ti.

\subsubsection{Catuariyaphalaṃ}

\paragraph{373.} ‘‘Katame pana te, bhante, dhammā uttaritarā ca paṇītatarā ca, yesaṃ sacchikiriyāhetu bhikkhū bhagavati brahmacariyaṃ carantī’’ti? ‘‘Idha, mahāli, bhikkhu tiṇṇaṃ saṃyojanānaṃ parikkhayā sotāpanno hoti avinipātadhammo niyato sambodhiparāyaṇo. Ayampi kho, mahāli, dhammo uttaritaro ca paṇītataro ca, yassa sacchikiriyāhetu bhikkhū mayi brahmacariyaṃ caranti. ‘‘Puna caparaṃ, mahāli, bhikkhu tiṇṇaṃ saṃyojanānaṃ parikkhayā rāgadosamohānaṃ tanuttā sakadāgāmī hoti, sakideva\footnote{sakiṃdeva (ka.)} imaṃ lokaṃ āgantvā dukkhassantaṃ karoti. Ayampi kho, mahāli, dhammo uttaritaro ca paṇītataro ca, yassa sacchikiriyāhetu bhikkhū mayi brahmacariyaṃ caranti. ‘‘Puna caparaṃ, mahāli, bhikkhu pañcannaṃ orambhāgiyānaṃ saṃyojanānaṃ parikkhayā opapātiko hoti, tattha parinibbāyī, anāvattidhammo tasmā lokā. Ayampi kho, mahāli, dhammo uttaritaro ca paṇītataro ca, yassa sacchikiriyāhetu bhikkhū mayi brahmacariyaṃ caranti. ‘‘Puna caparaṃ, mahāli, bhikkhu āsavānaṃ khayā anāsavaṃ cetovimuttiṃ paññāvimuttiṃ diṭṭheva dhamme sayaṃ abhiññā sacchikatvā upasampajja viharati. Ayampi kho, mahāli, dhammo uttaritaro ca paṇītataro ca, yassa sacchikiriyāhetu bhikkhū mayi brahmacariyaṃ caranti. Ime kho te, mahāli, dhammā uttaritarā ca paṇītatarā ca, yesaṃ sacchikiriyāhetu bhikkhū mayi brahmacariyaṃ carantī’’ti.

\subsubsection{Ariyaaṭṭhaṅgikamaggo}

\paragraph{374.} ‘‘Atthi pana, bhante, maggo atthi paṭipadā etesaṃ dhammānaṃ sacchikiriyāyā’’ti? ‘‘Atthi kho, mahāli, maggo atthi paṭipadā etesaṃ dhammānaṃ sacchikiriyāyā’’ti.

\paragraph{375.} ‘‘Katamo pana, bhante, maggo katamā paṭipadā etesaṃ dhammānaṃ sacchikiriyāyā’’ti? ‘‘Ayameva ariyo aṭṭhaṅgiko maggo. Seyyathidaṃ – sammādiṭṭhi sammāsaṅkappo sammāvācā sammākammanto sammāājīvo sammāvāyāmo sammāsati sammāsamādhi. Ayaṃ kho, mahāli, maggo ayaṃ paṭipadā etesaṃ dhammānaṃ sacchikiriyāya.

\subsubsection{Dvepabbajitavatthu}

\paragraph{376.} ‘‘Ekamidāhaṃ, mahāli, samayaṃ kosambiyaṃ viharāmi ghositārāme. Atha kho dve pabbajitā – muṇḍiyo ca paribbājako jāliyo ca dārupattikantevāsī yenāhaṃ tenupasaṅkamiṃsu. Upasaṅkamitvā mayā saddhiṃ sammodiṃsu. Sammodanīyaṃ kathaṃ sāraṇīyaṃ vītisāretvā ekamantaṃ aṭṭhaṃsu. Ekamantaṃ ṭhitā kho te dve pabbajitā maṃ etadavocuṃ – ‘kiṃ nu kho, āvuso gotama, taṃ jīvaṃ taṃ sarīraṃ, udāhu aññaṃ jīvaṃ aññaṃ sarīra’nti?

\paragraph{377.} ‘‘‘Tena hāvuso, suṇātha sādhukaṃ manasi karotha bhāsissāmī’’ti. ‘Evamāvuso’ti kho te dve pabbajitā mama paccassosuṃ. Ahaṃ etadavocaṃ – idhāvuso tathāgato loke uppajjati arahaṃ sammāsambuddho…pe… (yathā 190-212 anucchedesu evaṃ vitthāretabbaṃ). Evaṃ kho, āvuso, bhikkhu sīlasampanno hoti…pe… paṭhamaṃ jhānaṃ upasampajja viharati. Yo kho, āvuso, bhikkhu evaṃ jānāti evaṃ passati, kallaṃ nu kho tassetaṃ vacanāya – ‘taṃ jīvaṃ taṃ sarīra’nti vā ‘aññaṃ jīvaṃ aññaṃ sarīra’nti vāti? Yo so, āvuso, bhikkhu evaṃ jānāti evaṃ passati, kallaṃ tassetaṃ vacanāya – ‘taṃ jīvaṃ taṃ sarīra’nti vā, ‘aññaṃ jīvaṃ aññaṃ sarīra’nti vāti. Ahaṃ kho panetaṃ, āvuso, evaṃ jānāmi evaṃ passāmi. Atha ca panāhaṃ na vadāmi – ‘taṃ jīvaṃ taṃ sarīra’nti vā ‘aññaṃ jīvaṃ aññaṃ sarīra’nti vā…pe… dutiyaṃ jhānaṃ…pe… tatiyaṃ jhānaṃ…pe… catutthaṃ jhānaṃ upasampajja viharati. Yo kho, āvuso, bhikkhu evaṃ jānāti evaṃ passati, kallaṃ nu kho tassetaṃ vacanāya – ‘taṃ jīvaṃ taṃ sarīra’nti vā ‘aññaṃ jīvaṃ aññaṃ sarīra’nti vāti? Yo so, āvuso, bhikkhu evaṃ jānāti evaṃ passati, kallaṃ tassetaṃ vacanāya – ‘taṃ jīvaṃ taṃ sarīra’nti vā ‘aññaṃ jīvaṃ aññaṃ sarīra’nti vāti. Ahaṃ kho panetaṃ, āvuso, evaṃ jānāmi evaṃ passāmi. Atha ca panāhaṃ na vadāmi – ‘taṃ jīvaṃ taṃ sarīra’nti vā ‘aññaṃ jīvaṃ aññaṃ sarīra’nti vā…pe… ñāṇadassanāya cittaṃ abhinīharati abhininnāmeti…pe… yo kho, āvuso, bhikkhu evaṃ jānāti evaṃ passati, kallaṃ nu kho tassetaṃ vacanāya – ‘taṃ jīvaṃ taṃ sarīra’nti vā ‘aññaṃ jīvaṃ aññaṃ sarīra’nti vāti? Yo so, āvuso, bhikkhu evaṃ jānāti evaṃ passati, kallaṃ\footnote{na kallaṃ (sī. syā. kaṃ. ka.)} tassetaṃ vacanāya – ‘taṃ jīvaṃ taṃ sarīra’’nti vā ‘aññaṃ jīvaṃ aññaṃ sarīra’nti vāti. Ahaṃ kho panetaṃ, āvuso, evaṃ jānāmi evaṃ passāmi. Atha ca panāhaṃ na vadāmi – ‘taṃ jīvaṃ taṃ sarīra’nti vā ‘aññaṃ jīvaṃ aññaṃ sarīra’nti vā…pe… nāparaṃ itthattāyāti pajānāti. Yo kho, āvuso, bhikkhu evaṃ jānāti evaṃ passati, kallaṃ nu kho tassetaṃ vacanāya – ‘taṃ jīvaṃ taṃ sarīra’nti vā ‘aññaṃ jīvaṃ aññaṃ sarīra’nti vāti? Yo so, āvuso, bhikkhu evaṃ jānāti evaṃ passati na kallaṃ tassetaṃ vacanāya – ‘taṃ jīvaṃ taṃ sarīra’nti vā ‘aññaṃ jīvaṃ aññaṃ sarīra’nti vāti. Ahaṃ kho panetaṃ, āvuso, evaṃ jānāmi evaṃ passāmi. Atha ca panāhaṃ na vadāmi – ‘taṃ jīvaṃ taṃ sarīra’nti vā ‘aññaṃ jīvaṃ aññaṃ sarīra’nti vā’’ti. Idamavoca bhagavā. Attamano oṭṭhaddho licchavī bhagavato bhāsitaṃ abhinandīti.

\xsectionEnd{Mahālisuttaṃ niṭṭhitaṃ chaṭṭhaṃ.}
