\section{Mahāsīhanādasuttaṃ}

\subsubsection{Acelakassapavatthu}

\paragraph{381.} Evaṃ me sutaṃ – ekaṃ samayaṃ bhagavā uruññāyaṃ\footnote{ujuññāyaṃ (sī. syā. kaṃ. pī.)} viharati kaṇṇakatthale migadāye. Atha kho acelo kassapo yena bhagavā tenupasaṅkami; upasaṅkamitvā bhagavatā saddhiṃ sammodi. Sammodanīyaṃ kathaṃ sāraṇīyaṃ vītisāretvā ekamantaṃ aṭṭhāsi. Ekamantaṃ ṭhito kho acelo kassapo bhagavantaṃ etadavoca – ‘‘sutaṃ metaṃ, bho gotama – ‘samaṇo gotamo sabbaṃ tapaṃ garahati, sabbaṃ tapassiṃ lūkhājīviṃ ekaṃsena upakkosati upavadatī’ti. Ye te, bho gotama, evamāhaṃsu – ‘samaṇo gotamo sabbaṃ tapaṃ garahati, sabbaṃ tapassiṃ lūkhājīviṃ ekaṃsena upakkosati upavadatī’ti, kacci te bhoto gotamassa vuttavādino, na ca bhavantaṃ gotamaṃ abhūtena abbhācikkhanti, dhammassa cānudhammaṃ byākaronti, na ca koci sahadhammiko vādānuvādo gārayhaṃ ṭhānaṃ āgacchati? Anabbhakkhātukāmā hi mayaṃ bhavantaṃ gotama’’nti.

\paragraph{382.} ‘‘Ye te, kassapa, evamāhaṃsu – ‘samaṇo gotamo sabbaṃ tapaṃ garahati, sabbaṃ tapassiṃ lūkhājīviṃ ekaṃsena upakkosati upavadatī’ti, na me te vuttavādino, abbhācikkhanti ca pana maṃ te asatā abhūtena. Idhāhaṃ, kassapa, ekaccaṃ tapassiṃ lūkhājīviṃ passāmi dibbena cakkhunā visuddhena atikkantamānusakena kāyassa bhedā paraṃ maraṇā apāyaṃ duggatiṃ vinipātaṃ nirayaṃ upapannaṃ. Idha panāhaṃ, kassapa, ekaccaṃ tapassiṃ lūkhājīviṃ passāmi dibbena cakkhunā visuddhena atikkantamānusakena kāyassa bhedā paraṃ maraṇā sugatiṃ saggaṃ lokaṃ upapannaṃ.

\paragraph{383.} ‘‘Idhāhaṃ, kassapa, ekaccaṃ tapassiṃ appadukkhavihāriṃ passāmi dibbena cakkhunā visuddhena atikkantamānusakena kāyassa bhedā paraṃ maraṇā apāyaṃ duggatiṃ vinipātaṃ nirayaṃ upapannaṃ. Idha panāhaṃ, kassapa, ekaccaṃ tapassiṃ appadukkhavihāriṃ passāmi dibbena cakkhunā visuddhena atikkantamānusakena kāyassa bhedā paraṃ maraṇā sugatiṃ saggaṃ lokaṃ upapannaṃ. Yohaṃ, kassapa, imesaṃ tapassīnaṃ evaṃ āgatiñca gatiñca cutiñca upapattiñca yathābhūtaṃ pajānāmi, sohaṃ kiṃ sabbaṃ tapaṃ garahissāmi, sabbaṃ vā tapassiṃ lūkhājīviṃ ekaṃsena upakkosissāmi upavadissāmi?

\paragraph{384.} ‘‘Santi, kassapa, eke samaṇabrāhmaṇā paṇḍitā nipuṇā kataparappavādā vālavedhirūpā. Te bhindantā maññe caranti paññāgatena diṭṭhigatāni. Tehipi me saddhiṃ ekaccesu ṭhānesu sameti, ekaccesu ṭhānesu na sameti. Yaṃ te ekaccaṃ vadanti ‘sādhū’ti, mayampi taṃ ekaccaṃ vadema ‘sādhū’ti. Yaṃ te ekaccaṃ vadanti ‘na sādhū’ti, mayampi taṃ ekaccaṃ vadema ‘na sādhū’ti. Yaṃ te ekaccaṃ vadanti ‘sādhū’ti, mayaṃ taṃ ekaccaṃ vadema ‘na sādhū’ti. Yaṃ te ekaccaṃ vadanti ‘na sādhū’ti, mayaṃ taṃ ekaccaṃ vadema ‘sādhū’ti. ‘‘Yaṃ mayaṃ ekaccaṃ vadema ‘sādhū’ti, parepi taṃ ekaccaṃ vadanti ‘sādhū’ti. Yaṃ mayaṃ ekaccaṃ vadema ‘na sādhū’ti, parepi taṃ ekaccaṃ vadanti ‘na sādhū’ti. Yaṃ mayaṃ ekaccaṃ vadema ‘na sādhū’ti, pare taṃ ekaccaṃ vadanti ‘sādhū’ti. Yaṃ mayaṃ ekaccaṃ vadema ‘sādhū’ti, pare taṃ ekaccaṃ vadanti ‘na sādhū’ti.

\subsubsection{Samanuyuñjāpanakathā}

\paragraph{385.} ‘‘Tyāhaṃ upasaṅkamitvā evaṃ vadāmi – yesu no, āvuso, ṭhānesu na sameti, tiṭṭhantu tāni ṭhānāni. Yesu ṭhānesu sameti, tattha viññū samanuyuñjantaṃ samanugāhantaṃ samanubhāsantaṃ satthārā vā satthāraṃ saṅghena vā saṅghaṃ – ‘ye imesaṃ bhavataṃ dhammā akusalā akusalasaṅkhātā, sāvajjā sāvajjasaṅkhātā, asevitabbā asevitabbasaṅkhātā, na alamariyā na alamariyasaṅkhātā, kaṇhā kaṇhasaṅkhātā. Ko ime dhamme anavasesaṃ pahāya vattati, samaṇo vā gotamo, pare vā pana bhonto gaṇācariyā’ti?

\paragraph{386.} ‘‘Ṭhānaṃ kho panetaṃ, kassapa, vijjati, yaṃ viññū samanuyuñjantā samanugāhantā samanubhāsantā evaṃ vadeyyuṃ – ‘ye imesaṃ bhavataṃ dhammā akusalā akusalasaṅkhātā, sāvajjā sāvajjasaṅkhātā, asevitabbā asevitabbasaṅkhātā, na alamariyā na alamariyasaṅkhātā, kaṇhā kaṇhasaṅkhātā. Samaṇo gotamo ime dhamme anavasesaṃ pahāya vattati, yaṃ vā pana bhonto pare gaṇācariyā’ti. Itiha, kassapa, viññū samanuyuñjantā samanugāhantā samanubhāsantā amheva tattha yebhuyyena pasaṃseyyuṃ.

\paragraph{387.} ‘‘Aparampi no, kassapa, viññū samanuyuñjantaṃ samanugāhantaṃ samanubhāsantaṃ satthārā vā satthāraṃ saṅghena vā saṅghaṃ – ‘ye imesaṃ bhavataṃ dhammā kusalā kusalasaṅkhātā, anavajjā anavajjasaṅkhātā, sevitabbā sevitabbasaṅkhātā, alamariyā alamariyasaṅkhātā, sukkā sukkasaṅkhātā. Ko ime dhamme anavasesaṃ samādāya vattati, samaṇo vā gotamo, pare vā pana bhonto gaṇācariyā’ ti?

\paragraph{388.} ‘‘Ṭhānaṃ kho panetaṃ, kassapa, vijjati, yaṃ viññū samanuyuñjantā samanugāhantā samanubhāsantā evaṃ vadeyyuṃ – ‘ye imesaṃ bhavataṃ dhammā kusalā kusalasaṅkhātā, anavajjā anavajjasaṅkhātā, sevitabbā sevitabbasaṅkhātā, alamariyā alamariyasaṅkhātā, sukkā sukkasaṅkhātā. Samaṇo gotamo ime dhamme anavasesaṃ samādāya vattati, yaṃ vā pana bhonto pare gaṇācariyā’ti. Itiha, kassapa, viññū samanuyuñjantā samanugāhantā samanubhāsantā amheva tattha yebhuyyena pasaṃseyyuṃ.

\paragraph{389.} ‘‘Aparampi no, kassapa, viññū samanuyuñjantaṃ samanugāhantaṃ samanubhāsantaṃ satthārā vā satthāraṃ saṅghena vā saṅghaṃ – ‘ye imesaṃ bhavataṃ dhammā akusalā akusalasaṅkhātā, sāvajjā sāvajjasaṅkhātā, asevitabbā asevitabbasaṅkhātā, na alamariyā na alamariyasaṅkhātā, kaṇhā kaṇhasaṅkhātā. Ko ime dhamme anavasesaṃ pahāya vattati, gotamasāvakasaṅgho vā, pare vā pana bhonto gaṇācariyasāvakasaṅghā’ti?

\paragraph{390.} ‘‘Ṭhānaṃ kho panetaṃ, kassapa, vijjati, yaṃ viññū samanuyuñjantā samanugāhantā samanubhāsantā evaṃ vadeyyuṃ – ‘ye imesaṃ bhavataṃ dhammā akusalā akusalasaṅkhātā, sāvajjā sāvajjasaṅkhātā, asevitabbā asevitabbasaṅkhātā, na alamariyā na alamariyasaṅkhātā, kaṇhā kaṇhasaṅkhātā. Gotamasāvakasaṅgho ime dhamme anavasesaṃ pahāya vattati, yaṃ vā pana bhonto pare gaṇācariyasāvakasaṅghā’ti. Itiha, kassapa, viññū samanuyuñjantā samanugāhantā samanubhāsantā amheva tattha yebhuyyena pasaṃseyyuṃ.

\paragraph{391.} ‘‘Aparampi no, kassapa, viññū samanuyuñjantaṃ samanugāhantaṃ samanubhāsantaṃ satthārā vā satthāraṃ saṅghena vā saṅghaṃ. ‘Ye imesaṃ bhavataṃ dhammā kusalā kusalasaṅkhātā, anavajjā anavajjasaṅkhātā, sevitabbā sevitabbasaṅkhātā, alamariyā alamariyasaṅkhātā, sukkā sukkasaṅkhātā. Ko ime dhamme anavasesaṃ samādāya vattati, gotamasāvakasaṅgho vā, pare vā pana bhonto gaṇācariyasāvakasaṅghā’ti?

\paragraph{392.} ‘‘Ṭhānaṃ kho panetaṃ, kassapa, vijjati, yaṃ viññū samanuyuñjantā samanugāhantā samanubhāsantā evaṃ vadeyyuṃ – ‘ye imesaṃ bhavataṃ dhammā kusalā kusalasaṅkhātā, anavajjā anavajjasaṅkhātā, sevitabbā sevitabbasaṅkhātā, alamariyā alamariyasaṅkhātā, sukkā sukkasaṅkhātā. Gotamasāvakasaṅgho ime dhamme anavasesaṃ samādāya vattati, yaṃ vā pana bhonto pare gaṇācariyasāvakasaṅghā’ti. Itiha, kassapa, viññū samanuyuñjantā samanugāhantā samanubhāsantā amheva tattha yebhuyyena pasaṃseyyuṃ.

\subsubsection{Ariyo aṭṭhaṅgiko maggo}

\paragraph{393.} ‘‘Atthi, kassapa, maggo atthi paṭipadā, yathāpaṭipanno sāmaṃyeva ñassati sāmaṃ dakkhati\footnote{dakkhiti (sī.)} – ‘samaṇova gotamo kālavādī bhūtavādī atthavādī dhammavādī vinayavādī’ti. Katamo ca, kassapa, maggo, katamā ca paṭipadā, yathāpaṭipanno sāmaṃyeva ñassati sāmaṃ dakkhati – ‘samaṇova gotamo kālavādī bhūtavādī atthavādī dhammavādī vinayavādī’ti? Ayameva ariyo aṭṭhaṅgiko maggo. Seyyathidaṃ – sammādiṭṭhi sammāsaṅkappo sammāvācā sammākammanto sammāājīvo sammāvāyāmo sammāsati sammāsamādhi. Ayaṃ kho, kassapa, maggo, ayaṃ paṭipadā, yathāpaṭipanno sāmaṃyeva ñassati sāmaṃ dakkhati ‘samaṇova gotamo kālavādī bhūtavādī atthavādī dhammavādī vinayavādī’’’ti.

\subsubsection{Tapopakkamakathā}

\paragraph{394.} Evaṃ vutte, acelo kassapo bhagavantaṃ etadavoca – ‘‘imepi kho, āvuso gotama, tapopakkamā etesaṃ samaṇabrāhmaṇānaṃ sāmaññasaṅkhātā ca brahmaññasaṅkhātā ca. Acelako hoti, muttācāro, hatthāpalekhano, na ehibhaddantiko, na tiṭṭhabhaddantiko, nābhihaṭaṃ, na uddissakataṃ, na nimantanaṃ sādiyati. So na kumbhimukhā paṭiggaṇhāti, na kaḷopimukhā paṭiggaṇhāti, na eḷakamantaraṃ, na daṇḍamantaraṃ, na musalamantaraṃ, na dvinnaṃ bhuñjamānānaṃ, na gabbhiniyā, na pāyamānāya, na purisantaragatāya, na saṅkittīsu, na yattha sā upaṭṭhito hoti, na yattha makkhikā saṇḍasaṇḍacārinī, na macchaṃ, na maṃsaṃ, na suraṃ, na merayaṃ, na thusodakaṃ pivati. So ekāgāriko vā hoti ekālopiko, dvāgāriko vā hoti dvālopiko…pe… sattāgāriko vā hoti sattālopiko; ekissāpi dattiyā yāpeti, dvīhipi dattīhi yāpeti… sattahipi dattīhi yāpeti; ekāhikampi āhāraṃ āhāreti, dvīhikampi āhāraṃ āhāreti… sattāhikampi āhāraṃ āhāreti. Iti evarūpaṃ addhamāsikampi pariyāyabhattabhojanānuyogamanuyutto viharati.

\paragraph{395.} ‘‘Imepi kho, āvuso gotama, tapopakkamā etesaṃ samaṇabrāhmaṇānaṃ sāmaññasaṅkhātā ca brahmaññasaṅkhātā ca. Sākabhakkho vā hoti, sāmākabhakkho vā hoti, nīvārabhakkho vā hoti, daddulabhakkho vā hoti, haṭabhakkho vā hoti, kaṇabhakkho vā hoti, ācāmabhakkho vā hoti, piññākabhakkho vā hoti, tiṇabhakkho vā hoti, gomayabhakkho vā hoti, vanamūlaphalāhāro yāpeti pavattaphalabhojī.

\paragraph{396.} ‘‘Imepi kho, āvuso gotama, tapopakkamā etesaṃ samaṇabrāhmaṇānaṃ sāmaññasaṅkhātā ca brahmaññasaṅkhātā ca. Sāṇānipi dhāreti, masāṇānipi dhāreti, chavadussānipi dhāreti, paṃsukūlānipi dhāreti, tirīṭānipi dhāreti, ajinampi dhāreti, ajinakkhipampi dhāreti, kusacīrampi dhāreti, vākacīrampi dhāreti, phalakacīrampi dhāreti, kesakambalampi dhāreti, vāḷakambalampi dhāreti, ulūkapakkhikampi dhāreti, kesamassulocakopi hoti kesamassulocanānuyogamanuyutto, ubbhaṭṭhakopi\footnote{ubbhaṭṭhikopi (ka.)} hoti āsanapaṭikkhitto, ukkuṭikopi hoti ukkuṭikappadhānamanuyutto, kaṇṭakāpassayikopi hoti kaṇṭakāpassaye seyyaṃ kappeti, phalakaseyyampi kappeti, thaṇḍilaseyyampi kappeti, ekapassayikopi hoti rajojalladharo, abbhokāsikopi hoti yathāsanthatiko, vekaṭikopi hoti vikaṭabhojanānuyogamanuyutto, apānakopi hoti apānakattamanuyutto, sāyatatiyakampi udakorohanānuyogamanuyutto viharatī’’ti.

\subsubsection{Tapopakkamaniratthakathā}

\paragraph{397.} ‘‘Acelako cepi, kassapa, hoti, muttācāro, hatthāpalekhano…pe… iti evarūpaṃ addhamāsikampi pariyāyabhattabhojanānuyogamanuyutto viharati. Tassa cāyaṃ sīlasampadā cittasampadā paññāsampadā abhāvitā hoti asacchikatā. Atha kho so ārakāva sāmaññā ārakāva brahmaññā. Yato kho, kassapa, bhikkhu averaṃ abyāpajjaṃ mettacittaṃ bhāveti, āsavānañca khayā anāsavaṃ cetovimuttiṃ paññāvimuttiṃ diṭṭheva dhamme sayaṃ abhiññā sacchikatvā upasampajja viharati. Ayaṃ vuccati, kassapa, bhikkhu samaṇo itipi brāhmaṇo itipi. ‘‘Sākabhakkho cepi, kassapa, hoti, sāmākabhakkho…pe… vanamūlaphalāhāro yāpeti pavattaphalabhojī. Tassa cāyaṃ sīlasampadā cittasampadā paññāsampadā abhāvitā hoti asacchikatā. Atha kho so ārakāva sāmaññā ārakāva brahmaññā. Yato kho, kassapa, bhikkhu averaṃ abyāpajjaṃ mettacittaṃ bhāveti, āsavānañca khayā anāsavaṃ cetovimuttiṃ paññāvimuttiṃ diṭṭheva dhamme sayaṃ abhiññā sacchikatvā upasampajja viharati. Ayaṃ vuccati, kassapa, bhikkhu samaṇo itipi brāhmaṇo itipi. ‘‘Sāṇāni cepi, kassapa, dhāreti, masāṇānipi dhāreti…pe… sāyatatiyakampi udakorohanānuyogamanuyutto viharati. Tassa cāyaṃ sīlasampadā cittasampadā paññāsampadā abhāvitā hoti asacchikatā. Atha kho so ārakāva sāmaññā ārakāva brahmaññā. Yato kho, kassapa, bhikkhu averaṃ abyāpajjaṃ mettacittaṃ bhāveti, āsavānañca khayā anāsavaṃ cetovimuttiṃ paññāvimuttiṃ diṭṭheva dhamme sayaṃ abhiññā sacchikatvā upasampajja viharati. Ayaṃ vuccati, kassapa, bhikkhu samaṇo itipi brāhmaṇo itipī’’ti.

\paragraph{398.} Evaṃ vutte, acelo kassapo bhagavantaṃ etadavoca – ‘‘dukkaraṃ, bho gotama, sāmaññaṃ dukkaraṃ brahmañña’’nti. ‘‘Pakati kho esā, kassapa, lokasmiṃ ‘dukkaraṃ sāmaññaṃ dukkaraṃ brahmañña’nti. Acelako cepi, kassapa, hoti, muttācāro, hatthāpalekhano…pe… iti evarūpaṃ addhamāsikampi pariyāyabhattabhojanānuyogamanuyutto viharati. Imāya ca, kassapa, mattāya iminā tapopakkamena sāmaññaṃ vā abhavissa brahmaññaṃ vā dukkaraṃ sudukkaraṃ, netaṃ abhavissa kallaṃ vacanāya – ‘dukkaraṃ sāmaññaṃ dukkaraṃ brahmañña’nti. ‘‘Sakkā ca panetaṃ abhavissa kātuṃ gahapatinā vā gahapatiputtena vā antamaso kumbhadāsiyāpi – ‘handāhaṃ acelako homi, muttācāro, hatthāpalekhano…pe… iti evarūpaṃ addhamāsikampi pariyāyabhattabhojanānuyogamanuyutto viharāmī’ti. ‘‘Yasmā ca kho, kassapa, aññatreva imāya mattāya aññatra iminā tapopakkamena sāmaññaṃ vā hoti brahmaññaṃ vā dukkaraṃ sudukkaraṃ, tasmā etaṃ kallaṃ vacanāya – ‘dukkaraṃ sāmaññaṃ dukkaraṃ brahmañña’nti. Yato kho, kassapa, bhikkhu averaṃ abyāpajjaṃ mettacittaṃ bhāveti, āsavānañca khayā anāsavaṃ cetovimuttiṃ paññāvimuttiṃ diṭṭheva dhamme sayaṃ abhiññā sacchikatvā upasampajja viharati. Ayaṃ vuccati, kassapa, bhikkhu samaṇo itipi brāhmaṇo itipi. ‘‘Sākabhakkho cepi, kassapa, hoti, sāmākabhakkho…pe… vanamūlaphalāhāro yāpeti pavattaphalabhojī. Imāya ca, kassapa, mattāya iminā tapopakkamena sāmaññaṃ vā abhavissa brahmaññaṃ vā dukkaraṃ sudukkaraṃ, netaṃ abhavissa kallaṃ vacanāya – ‘dukkaraṃ sāmaññaṃ dukkaraṃ brahmañña’nti. ‘‘Sakkā ca panetaṃ abhavissa kātuṃ gahapatinā vā gahapatiputtena vā antamaso kumbhadāsiyāpi – ‘handāhaṃ sākabhakkho vā homi, sāmākabhakkho vā…pe… vanamūlaphalāhāro yāpemi pavattaphalabhojī’ti. ‘‘Yasmā ca kho, kassapa, aññatreva imāya mattāya aññatra iminā tapopakkamena sāmaññaṃ vā hoti brahmaññaṃ vā dukkaraṃ sudukkaraṃ, tasmā etaṃ kallaṃ vacanāya – ‘dukkaraṃ sāmaññaṃ dukkaraṃ brahmañña’nti. Yato kho, kassapa, bhikkhu averaṃ abyāpajjaṃ mettacittaṃ bhāveti, āsavānañca khayā anāsavaṃ cetovimuttiṃ paññāvimuttiṃ diṭṭheva dhamme sayaṃ abhiññā sacchikatvā upasampajja viharati. Ayaṃ vuccati, kassapa, bhikkhu samaṇo itipi brāhmaṇo itipi. ‘‘Sāṇāni cepi, kassapa, dhāreti, masāṇānipi dhāreti…pe… sāyatatiyakampi udakorohanānuyogamanuyutto viharati. Imāya ca, kassapa, mattāya iminā tapopakkamena sāmaññaṃ vā abhavissa brahmaññaṃ vā dukkaraṃ sudukkaraṃ, netaṃ abhavissa kallaṃ vacanāya – ‘dukkaraṃ sāmaññaṃ dukkaraṃ brahmañña’nti. ‘‘Sakkā ca panetaṃ abhavissa kātuṃ gahapatinā vā gahapatiputtena vā antamaso kumbhadāsiyāpi – ‘handāhaṃ sāṇānipi dhāremi, masāṇānipi dhāremi…pe… sāyatatiyakampi udakorohanānuyogamanuyutto viharāmī’ti. ‘‘Yasmā ca kho, kassapa, aññatreva imāya mattāya aññatra iminā tapopakkamena sāmaññaṃ vā hoti brahmaññaṃ vā dukkaraṃ sudukkaraṃ, tasmā etaṃ kallaṃ vacanāya – ‘dukkaraṃ sāmaññaṃ dukkaraṃ brahmañña’nti. Yato kho, kassapa, bhikkhu averaṃ abyāpajjaṃ mettacittaṃ bhāveti, āsavānañca khayā anāsavaṃ cetovimuttiṃ paññāvimuttiṃ diṭṭheva dhamme sayaṃ abhiññā sacchikatvā upasampajja viharati. Ayaṃ vuccati, kassapa, bhikkhu samaṇo itipi brāhmaṇo itipī’’ti.

\paragraph{399.} Evaṃ vutte, acelo kassapo bhagavantaṃ etadavoca – ‘‘dujjāno, bho gotama, samaṇo, dujjāno brāhmaṇo’’ti. ‘‘Pakati kho esā, kassapa, lokasmiṃ ‘dujjāno samaṇo dujjāno brāhmaṇo’ti. Acelako cepi, kassapa, hoti, muttācāro, hatthāpalekhano…pe… iti evarūpaṃ addhamāsikampi pariyāyabhattabhojanānuyogamanuyutto viharati. Imāya ca, kassapa, mattāya iminā tapopakkamena samaṇo vā abhavissa brāhmaṇo vā dujjāno sudujjāno, netaṃ abhavissa kallaṃ vacanāya – ‘dujjāno samaṇo dujjāno brāhmaṇo’ti. ‘‘Sakkā ca paneso abhavissa ñātuṃ gahapatinā vā gahapatiputtena vā antamaso kumbhadāsiyāpi – ‘ayaṃ acelako hoti, muttācāro, hatthāpalekhano…pe… iti evarūpaṃ addhamāsikampi pariyāyabhattabhojanānuyogamanuyutto viharatī’ti. ‘‘Yasmā ca kho, kassapa, aññatreva imāya mattāya aññatra iminā tapopakkamena samaṇo vā hoti brāhmaṇo vā dujjāno sudujjāno, tasmā etaṃ kallaṃ vacanāya – ‘dujjāno samaṇo dujjāno brāhmaṇo’ti. Yato kho\footnote{yato ca kho (ka.)}, kassapa, bhikkhu averaṃ abyāpajjaṃ mettacittaṃ bhāveti, āsavānañca khayā anāsavaṃ cetovimuttiṃ paññāvimuttiṃ diṭṭheva dhamme sayaṃ abhiññā sacchikatvā upasampajja viharati. Ayaṃ vuccati, kassapa, bhikkhu samaṇo itipi brāhmaṇo itipi. ‘‘Sākabhakkho cepi, kassapa, hoti sāmākabhakkho…pe… vanamūlaphalāhāro yāpeti pavattaphalabhojī. Imāya ca, kassapa, mattāya iminā tapopakkamena samaṇo vā abhavissa brāhmaṇo vā dujjāno sudujjāno, netaṃ abhavissa kallaṃ vacanāya – ‘dujjāno samaṇo dujjāno brāhmaṇo’ti. ‘‘Sakkā ca paneso abhavissa ñātuṃ gahapatinā vā gahapatiputtena vā antamaso kumbhadāsiyāpi – ‘ayaṃ sākabhakkho vā hoti sāmākabhakkho…pe… vanamūlaphalāhāro yāpeti pavattaphalabhojī’ti. ‘‘Yasmā ca kho, kassapa, aññatreva imāya mattāya aññatra iminā tapopakkamena samaṇo vā hoti brāhmaṇo vā dujjāno sudujjāno, tasmā etaṃ kallaṃ vacanāya – ‘dujjāno samaṇo dujjāno brāhmaṇo’ti. Yato kho, kassapa, bhikkhu averaṃ abyāpajjaṃ mettacittaṃ bhāveti, āsavānañca khayā anāsavaṃ cetovimuttiṃ paññāvimuttiṃ diṭṭheva dhamme sayaṃ abhiññā sacchikatvā upasampajja viharati. Ayaṃ vuccati, kassapa, bhikkhu samaṇo itipi brāhmaṇo itipi. ‘‘Sāṇāni cepi, kassapa, dhāreti, masāṇānipi dhāreti…pe… sāyatatiyakampi udakorohanānuyogamanuyutto viharati. Imāya ca, kassapa, mattāya iminā tapopakkamena samaṇo vā abhavissa brāhmaṇo vā dujjāno sudujjāno, netaṃ abhavissa kallaṃ vacanāya – ‘dujjāno samaṇo dujjāno brāhmaṇo’ti. ‘‘Sakkā ca paneso abhavissa ñātuṃ gahapatinā vā gahapatiputtena vā antamaso kumbhadāsiyāpi – ‘ayaṃ sāṇānipi dhāreti, masāṇānipi dhāreti…pe… sāyatatiyakampi udakorohanānuyogamanuyutto viharatī’ti. ‘‘Yasmā ca kho, kassapa, aññatreva imāya mattāya aññatra iminā tapopakkamena samaṇo vā hoti brāhmaṇo vā dujjāno sudujjāno, tasmā etaṃ kallaṃ vacanāya – ‘dujjāno samaṇo dujjāno brāhmaṇo’ti. Yato kho, kassapa, bhikkhu averaṃ abyāpajjaṃ mettacittaṃ bhāveti, āsavānañca khayā anāsavaṃ cetovimuttiṃ paññāvimuttiṃ diṭṭheva dhamme sayaṃ abhiññā sacchikatvā upasampajja viharati. Ayaṃ vuccati, kassapa, bhikkhu samaṇo itipi brāhmaṇo itipī’’ti.

\subsubsection{Sīlasamādhipaññāsampadā}

\paragraph{400.} Evaṃ vutte, acelo kassapo bhagavantaṃ etadavoca – ‘‘katamā pana sā, bho gotama, sīlasampadā, katamā cittasampadā, katamā paññāsampadā’’ti? ‘‘Idha, kassapa, tathāgato loke uppajjati arahaṃ, sammāsambuddho…pe… (yathā 190-193 anucchedesu, evaṃ vitthāretabbaṃ) bhayadassāvī samādāya sikkhati sikkhāpadesu, kāyakammavacīkammena samannāgato kusalena parisuddhājīvo sīlasampanno indriyesu guttadvāro satisampajaññena samannāgato santuṭṭho.

\paragraph{401.} ‘‘Kathañca, kassapa, bhikkhu sīlasampanno hoti? Idha, kassapa, bhikkhu pāṇātipātaṃ pahāya pāṇātipātā paṭivirato hoti nihitadaṇḍo nihitasattho lajjī dayāpanno, sabbapāṇabhūtahitānukampī viharati. Idampissa hoti sīlasampadāya …pe… (yathā 194 yāva 210 anucchedesu) ‘‘Yathā vā paneke bhonto samaṇabrāhmaṇā saddhādeyyāni bhojanāni bhuñjitvā te evarūpāya tiracchānavijjāya micchājīvena jīvitaṃ kappenti. Seyyathidaṃ – santikammaṃ paṇidhikammaṃ…pe… (yathā 211 anucchede) osadhīnaṃ patimokkho iti vā iti, evarūpāya tiracchānavijjāya micchājīvā paṭivirato hoti. Idampissa hoti sīlasampadāya. ‘‘Sa kho so\footnote{ayaṃ kho (ka.)}, kassapa, bhikkhu evaṃ sīlasampanno na kutoci bhayaṃ samanupassati, yadidaṃ sīlasaṃvarato. Seyyathāpi, kassapa, rājā khattiyo muddhāvasitto nihatapaccāmitto na kutoci bhayaṃ samanupassati, yadidaṃ paccatthikato. Evameva kho, kassapa, bhikkhu evaṃ sīlasampanno na kutoci bhayaṃ samanupassati, yadidaṃ sīlasaṃvarato. So iminā ariyena sīlakkhandhena samannāgato ajjhattaṃ anavajjasukhaṃ paṭisaṃvedeti. Evaṃ kho, kassapa, bhikkhu sīlasampanno hoti. Ayaṃ kho, kassapa, sīlasampadā…pe… paṭhamaṃ jhānaṃ upasampajja viharati. Idampissa hoti cittasampadāya…pe… dutiyaṃ jhānaṃ…pe… tatiyaṃ jhānaṃ…pe… catutthaṃ jhānaṃ upasampajja viharati. Idampissa hoti cittasampadāya. Ayaṃ kho, kassapa, cittasampadā. ‘‘So evaṃ samāhite citte…pe… ñāṇadassanāya cittaṃ abhinīharati abhininnāmeti…pe… idampissa hoti paññāsampadāya…pe… nāparaṃ itthattāyāti pajānāti…pe… idampissa hoti paññāsampadāya. Ayaṃ kho, kassapa, paññāsampadā. ‘‘Imāya ca, kassapa, sīlasampadāya cittasampadāya paññāsampadāya aññā sīlasampadā cittasampadā paññāsampadā uttaritarā vā paṇītatarā vā natthi. 
\subsubsection{Sīhanādakathā}

\paragraph{402.} ‘‘Santi, kassapa, eke samaṇabrāhmaṇā sīlavādā. Te anekapariyāyena sīlassa vaṇṇaṃ bhāsanti. Yāvatā, kassapa, ariyaṃ paramaṃ sīlaṃ, nāhaṃ tattha attano samasamaṃ samanupassāmi, kuto bhiyyo! Atha kho ahameva tattha bhiyyo, yadidaṃ adhisīlaṃ. ‘‘Santi, kassapa, eke samaṇabrāhmaṇā tapojigucchāvādā. Te anekapariyāyena tapojigucchāya vaṇṇaṃ bhāsanti. Yāvatā, kassapa, ariyā paramā tapojigucchā, nāhaṃ tattha attano samasamaṃ samanupassāmi, kuto bhiyyo! Atha kho ahameva tattha bhiyyo, yadidaṃ adhijegucchaṃ. ‘‘Santi, kassapa, eke samaṇabrāhmaṇā paññāvādā. Te anekapariyāyena paññāya vaṇṇaṃ bhāsanti. Yāvatā, kassapa, ariyā paramā paññā, nāhaṃ tattha attano samasamaṃ samanupassāmi, kuto bhiyyo! Atha kho ahameva tattha bhiyyo, yadidaṃ adhipaññaṃ. ‘‘Santi, kassapa, eke samaṇabrāhmaṇā vimuttivādā. Te anekapariyāyena vimuttiyā vaṇṇaṃ bhāsanti. Yāvatā, kassapa, ariyā paramā vimutti, nāhaṃ tattha attano samasamaṃ samanupassāmi, kuto bhiyyo! Atha kho ahameva tattha bhiyyo, yadidaṃ adhivimutti.

\paragraph{403.} ‘‘Ṭhānaṃ kho panetaṃ, kassapa, vijjati, yaṃ aññatitthiyā paribbājakā evaṃ vadeyyuṃ – ‘sīhanādaṃ kho samaṇo gotamo nadati, tañca kho suññāgāre nadati, no parisāsū’ti. Te – ‘mā heva’ntissu vacanīyā. ‘Sīhanādañca samaṇo gotamo nadati, parisāsu ca nadatī’ti evamassu, kassapa, vacanīyā. ‘‘Ṭhānaṃ kho panetaṃ, kassapa, vijjati, yaṃ aññatitthiyā paribbājakā evaṃ vadeyyuṃ – ‘sīhanādañca samaṇo gotamo nadati, parisāsu ca nadati, no ca kho visārado nadatī’ti. Te – ‘mā heva’ntissu vacanīyā. ‘Sīhanādañca samaṇo gotamo nadati, parisāsu ca nadati, visārado ca nadatī’’ti evamassu, kassapa, vacanīyā. ‘‘Ṭhānaṃ kho panetaṃ, kassapa, vijjati, yaṃ aññatitthiyā paribbājakā evaṃ vadeyyuṃ – ‘sīhanādañca samaṇo gotamo nadati, parisāsu ca nadati, visārado ca nadati, no ca kho naṃ pañhaṃ pucchanti…pe… pañhañca naṃ pucchanti; no ca kho nesaṃ pañhaṃ puṭṭho byākaroti…pe… pañhañca nesaṃ puṭṭho byākaroti; no ca kho pañhassa veyyākaraṇena cittaṃ ārādheti…pe… pañhassa ca veyyākaraṇena cittaṃ ārādheti; no ca kho sotabbaṃ maññanti…pe… sotabbañcassa maññanti; no ca kho sutvā pasīdanti…pe… sutvā cassa pasīdanti; no ca kho pasannākāraṃ karonti…pe… pasannākārañca karonti; no ca kho tathattāya paṭipajjanti…pe… tathattāya ca paṭipajjanti; no ca kho paṭipannā ārādhentī’ti. Te – ‘mā heva’ntissu vacanīyā. ‘Sīhanādañca samaṇo gotamo nadati, parisāsu ca nadati, visārado ca nadati, pañhañca naṃ pucchanti, pañhañca nesaṃ puṭṭho byākaroti, pañhassa ca veyyākaraṇena cittaṃ ārādheti, sotabbañcassa maññanti, sutvā cassa pasīdanti, pasannākārañca karonti, tathattāya ca paṭipajjanti, paṭipannā ca ārādhentī’ti evamassu, kassapa, vacanīyā.

\subsubsection{Titthiyaparivāsakathā}

\paragraph{404.} ‘‘Ekamidāhaṃ, kassapa, samayaṃ rājagahe viharāmi gijjhakūṭe pabbate. Tatra maṃ aññataro tapabrahmacārī nigrodho nāma adhijegucche pañhaṃ apucchi. Tassāhaṃ adhijegucche pañhaṃ puṭṭho byākāsiṃ. Byākate ca pana me attamano ahosi paraṃ viya mattāyā’’ti. ‘‘Ko hi, bhante, bhagavato dhammaṃ sutvā na attamano assa paraṃ viya mattāya? Ahampi hi, bhante, bhagavato dhammaṃ sutvā attamano paraṃ viya mattāya. Abhikkantaṃ, bhante, abhikkantaṃ, bhante. Seyyathāpi, bhante, nikkujjitaṃ vā ukkujjeyya, paṭicchannaṃ vā vivareyya, mūḷhassa vā maggaṃ ācikkheyya, andhakāre vā telapajjotaṃ dhāreyya – ‘cakkhumanto rūpāni dakkhantī’ti; evamevaṃ bhagavatā anekapariyāyena dhammo pakāsito. Esāhaṃ, bhante, bhagavantaṃ saraṇaṃ gacchāmi, dhammañca bhikkhusaṅghañca. Labheyyāhaṃ, bhante, bhagavato santike pabbajjaṃ, labheyyaṃ upasampada’’nti.

\paragraph{405.} ‘‘Yo kho, kassapa, aññatitthiyapubbo imasmiṃ dhammavinaye ākaṅkhati pabbajjaṃ, ākaṅkhati upasampadaṃ, so cattāro māse parivasati, catunnaṃ māsānaṃ accayena āraddhacittā bhikkhū pabbājenti, upasampādenti bhikkhubhāvāya. Api ca mettha puggalavemattatā viditā’’ti. ‘‘Sace, bhante, aññatitthiyapubbā imasmiṃ dhammavinaye ākaṅkhanti pabbajjaṃ, ākaṅkhanti upasampadaṃ, cattāro māse parivasanti, catunnaṃ māsānaṃ accayena āraddhacittā bhikkhū pabbājenti, upasampādenti bhikkhubhāvāya. Ahaṃ cattāri vassāni parivasissāmi, catunnaṃ vassānaṃ accayena āraddhacittā bhikkhū pabbājentu, upasampādentu bhikkhubhāvāyā’’ti. Alattha kho acelo kassapo bhagavato santike pabbajjaṃ, alattha upasampadaṃ. Acirūpasampanno kho panāyasmā kassapo eko vūpakaṭṭho appamatto ātāpī pahitatto viharanto na cirasseva – yassatthāya kulaputtā sammadeva agārasmā anagāriyaṃ pabbajanti, tadanuttaraṃ – brahmacariyapariyosānaṃ diṭṭheva dhamme sayaṃ abhiññā sacchikatvā upasampajja vihāsi. ‘Khīṇā jāti, vusitaṃ brahmacariyaṃ, kataṃ karaṇīyaṃ, nāparaṃ itthattāyā’ti – abbhaññāsi. Aññataro kho panāyasmā kassapo arahataṃ ahosīti.

\xsectionEnd{Mahāsīhanādasuttaṃ niṭṭhitaṃ aṭṭhamaṃ.}
