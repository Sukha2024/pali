\section{Kevaṭṭasuttaṃ}

\subsubsection{Kevaṭṭagahapatiputtavatthu}

\paragraph{481.} Evaṃ me sutaṃ – ekaṃ samayaṃ bhagavā nāḷandāyaṃ viharati pāvārikambavane. Atha kho kevaṭṭo gahapatiputto yena bhagavā tenupasaṅkami; upasaṅkamitvā bhagavantaṃ abhivādetvā ekamantaṃ nisīdi. Ekamantaṃ nisinno kho kevaṭṭo gahapatiputto bhagavantaṃ etadavoca – ‘‘ayaṃ, bhante, nāḷandā iddhā ceva phītā ca bahujanā ākiṇṇamanussā bhagavati abhippasannā. Sādhu, bhante, bhagavā ekaṃ bhikkhuṃ samādisatu, yo uttarimanussadhammā, iddhipāṭihāriyaṃ karissati; evāyaṃ nāḷandā bhiyyoso mattāya bhagavati abhippasīdissatī’’ti. Evaṃ vutte, bhagavā kevaṭṭaṃ gahapatiputtaṃ etadavoca – ‘‘na kho ahaṃ, kevaṭṭa, bhikkhūnaṃ evaṃ dhammaṃ desemi – etha tumhe, bhikkhave, gihīnaṃ odātavasanānaṃ uttarimanussadhammā iddhipāṭihāriyaṃ karothā’’ti.

\paragraph{482.} Dutiyampi kho kevaṭṭo gahapatiputto bhagavantaṃ etadavoca – ‘‘nāhaṃ, bhante, bhagavantaṃ dhaṃsemi; api ca, evaṃ vadāmi – ‘ayaṃ, bhante, nāḷandā iddhā ceva phītā ca bahujanā ākiṇṇamanussā bhagavati abhippasannā. Sādhu, bhante, bhagavā ekaṃ bhikkhuṃ samādisatu, yo uttarimanussadhammā iddhipāṭihāriyaṃ karissati; evāyaṃ nāḷandā bhiyyoso mattāya bhagavati abhippasīdissatī’’’ti. Dutiyampi kho bhagavā kevaṭṭaṃ gahapatiputtaṃ etadavoca – ‘‘na kho ahaṃ, kevaṭṭa, bhikkhūnaṃ evaṃ dhammaṃ desemi – etha tumhe, bhikkhave, gihīnaṃ odātavasanānaṃ uttarimanussadhammā iddhipāṭihāriyaṃ karothā’’’ti. Tatiyampi kho kevaṭṭo gahapatiputto bhagavantaṃ etadavoca – ‘‘nāhaṃ, bhante, bhagavantaṃ dhaṃsemi; api ca, evaṃ vadāmi – ‘ayaṃ, bhante, nāḷandā iddhā ceva phītā ca bahujanā ākiṇṇamanussā bhagavati abhippasannā. Sādhu, bhante, bhagavā ekaṃ bhikkhuṃ samādisatu, yo uttarimanussadhammā iddhipāṭihāriyaṃ karissati. Evāyaṃ nāḷandā bhiyyoso mattāya bhagavati abhippasīdissatī’ti.

\subsubsection{Iddhipāṭihāriyaṃ}

\paragraph{483.} ‘‘Tīṇi kho imāni, kevaṭṭa, pāṭihāriyāni mayā sayaṃ abhiññā sacchikatvā paveditāni. Katamāni tīṇi? Iddhipāṭihāriyaṃ, ādesanāpāṭihāriyaṃ, anusāsanīpāṭihāriyaṃ.

\paragraph{484.} ‘‘Katamañca, kevaṭṭa, iddhipāṭihāriyaṃ? Idha, kevaṭṭa, bhikkhu anekavihitaṃ iddhividhaṃ paccanubhoti. Ekopi hutvā bahudhā hoti, bahudhāpi hutvā eko hoti; āvibhāvaṃ tirobhāvaṃ tirokuṭṭaṃ tiropākāraṃ tiropabbataṃ asajjamāno gacchati seyyathāpi ākāse; pathaviyāpi ummujjanimujjaṃ karoti seyyathāpi udake; udakepi abhijjamāne gacchati seyyathāpi pathaviyaṃ; ākāsepi pallaṅkena kamati seyyathāpi pakkhī sakuṇo; imepi candimasūriye evaṃ mahiddhike evaṃ mahānubhāve pāṇinā parāmasati parimajjati; yāva brahmalokāpi kāyena vasaṃ vatteti. ‘‘Tamenaṃ aññataro saddho pasanno passati taṃ bhikkhuṃ anekavihitaṃ iddhividhaṃ paccanubhontaṃ – ekopi hutvā bahudhā hontaṃ, bahudhāpi hutvā eko hontaṃ; āvibhāvaṃ tirobhāvaṃ; tirokuṭṭaṃ tiropākāraṃ tiropabbataṃ asajjamānaṃ gacchantaṃ seyyathāpi ākāse; pathaviyāpi ummujjanimujjaṃ karontaṃ seyyathāpi udake; udakepi abhijjamāne gacchantaṃ seyyathāpi pathaviyaṃ; ākāsepi pallaṅkena kamantaṃ seyyathāpi pakkhī sakuṇo; imepi candimasūriye evaṃ mahiddhike evaṃ mahānubhāve pāṇinā parāmasantaṃ parimajjantaṃ yāva brahmalokāpi kāyena vasaṃ vattentaṃ. ‘‘Tamenaṃ so saddho pasanno aññatarassa assaddhassa appasannassa āroceti – ‘acchariyaṃ vata, bho, abbhutaṃ vata, bho, samaṇassa mahiddhikatā mahānubhāvatā. Amāhaṃ bhikkhuṃ addasaṃ anekavihitaṃ iddhividhaṃ paccanubhontaṃ – ekopi hutvā bahudhā hontaṃ, bahudhāpi hutvā eko hontaṃ…pe… yāva brahmalokāpi kāyena vasaṃ vattenta’nti. ‘‘Tamenaṃ so assaddho appasanno taṃ saddhaṃ pasannaṃ evaṃ vadeyya – ‘atthi kho, bho, gandhārī nāma vijjā. Tāya so bhikkhu anekavihitaṃ iddhividhaṃ paccanubhoti – ekopi hutvā bahudhā hoti, bahudhāpi hutvā eko hoti…pe… yāva brahmalokāpi kāyena vasaṃ vattetī’ti. ‘‘Taṃ kiṃ maññasi, kevaṭṭa, api nu so assaddho appasanno taṃ saddhaṃ pasannaṃ evaṃ vadeyyā’’ti? ‘‘Vadeyya, bhante’’ti. ‘‘Imaṃ kho ahaṃ, kevaṭṭa, iddhipāṭihāriye ādīnavaṃ sampassamāno iddhipāṭihāriyena aṭṭīyāmi harāyāmi jigucchāmi’’.

\subsubsection{Ādesanāpāṭihāriyaṃ}

\paragraph{485.} ‘‘Katamañca, kevaṭṭa, ādesanāpāṭihāriyaṃ? Idha, kevaṭṭa, bhikkhu parasattānaṃ parapuggalānaṃ cittampi ādisati, cetasikampi ādisati, vitakkitampi ādisati, vicāritampi ādisati – ‘evampi te mano, itthampi te mano, itipi te citta’nti. ‘‘Tamenaṃ aññataro saddho pasanno passati taṃ bhikkhuṃ parasattānaṃ parapuggalānaṃ cittampi ādisantaṃ, cetasikampi ādisantaṃ, vitakkitampi ādisantaṃ, vicāritampi ādisantaṃ – ‘evampi te mano, itthampi te mano, itipi te citta’nti. Tamenaṃ so saddho pasanno aññatarassa assaddhassa appasannassa āroceti – ‘acchariyaṃ vata, bho, abbhutaṃ vata, bho, samaṇassa mahiddhikatā mahānubhāvatā. Amāhaṃ bhikkhuṃ addasaṃ parasattānaṃ parapuggalānaṃ cittampi ādisantaṃ, cetasikampi ādisantaṃ, vitakkitampi ādisantaṃ, vicāritampi ādisantaṃ – ‘‘evampi te mano, itthampi te mano, itipi te citta’’’nti. ‘‘Tamenaṃ so assaddho appasanno taṃ saddhaṃ pasannaṃ evaṃ vadeyya – ‘atthi kho, bho, maṇikā nāma vijjā; tāya so bhikkhu parasattānaṃ parapuggalānaṃ cittampi ādisati, cetasikampi ādisati, vitakkitampi ādisati, vicāritampi ādisati – ‘evampi te mano, itthampi te mano, itipi te citta’’’nti. ‘‘Taṃ kiṃ maññasi, kevaṭṭa, api nu so assaddho appasanno taṃ saddhaṃ pasannaṃ evaṃ vadeyyā’’ti? ‘‘Vadeyya, bhante’’ti. ‘‘Imaṃ kho ahaṃ, kevaṭṭa, ādesanāpāṭihāriye ādīnavaṃ sampassamāno ādesanāpāṭihāriyena aṭṭīyāmi harāyāmi jigucchāmi’’.

\subsubsection{Anusāsanīpāṭihāriyaṃ}

\paragraph{486.} ‘‘Katamañca, kevaṭṭa, anusāsanīpāṭihāriyaṃ? Idha, kevaṭṭa, bhikkhu evamanusāsati – ‘evaṃ vitakketha, mā evaṃ vitakkayittha, evaṃ manasikarotha, mā evaṃ manasākattha, idaṃ pajahatha, idaṃ upasampajja viharathā’ti. Idaṃ vuccati, kevaṭṭa, anusāsanīpāṭihāriyaṃ. ‘‘Puna caparaṃ, kevaṭṭa, idha tathāgato loke uppajjati arahaṃ sammāsambuddho … pe… (yathā 190-212 anucchedesu evaṃ vitthāretabbaṃ). Evaṃ kho, kevaṭṭa, bhikkhu sīlasampanno hoti…pe… paṭhamaṃ jhānaṃ upasampajja viharati. Idampi vuccati, kevaṭṭa, anusāsanīpāṭihāriyaṃ…pe… dutiyaṃ jhānaṃ…pe… tatiyaṃ jhānaṃ…pe… catutthaṃ jhānaṃ upasampajja viharati. Idampi vuccati, kevaṭṭa, anusāsanīpāṭihāriyaṃ…pe… ñāṇadassanāya cittaṃ abhinīharati abhininnāmeti…pe… idampi vuccati, kevaṭṭa, anusāsanīpāṭihāriyaṃ…pe… nāparaṃ itthattāyāti pajānāti…pe… idampi vuccati, kevaṭṭa, anusāsanīpāṭihāriyaṃ. ‘‘Imāni kho, kevaṭṭa, tīṇi pāṭihāriyāni mayā sayaṃ abhiññā sacchikatvā paveditāni’’.

\subsubsection{Bhūtanirodhesakabhikkhuvatthu}

\paragraph{487.} ‘‘Bhūtapubbaṃ, kevaṭṭa, imasmiññeva bhikkhusaṅghe aññatarassa bhikkhuno evaṃ cetaso parivitakko udapādi – ‘kattha nu kho ime cattāro mahābhūtā aparisesā nirujjhanti, seyyathidaṃ – pathavīdhātu āpodhātu tejodhātu vāyodhātū’ti?

\paragraph{488.} ‘‘Atha kho so, kevaṭṭa, bhikkhu tathārūpaṃ samādhiṃ samāpajji, yathāsamāhite citte devayāniyo maggo pāturahosi. Atha kho so, kevaṭṭa, bhikkhu yena cātumahārājikā devā tenupasaṅkami; upasaṅkamitvā cātumahārājike deve etadavoca – ‘kattha nu kho, āvuso, ime cattāro mahābhūtā aparisesā nirujjhanti, seyyathidaṃ – pathavīdhātu āpodhātu tejodhātu vāyodhātū’ti? ‘‘Evaṃ vutte, kevaṭṭa, cātumahārājikā devā taṃ bhikkhuṃ etadavocuṃ – ‘mayampi kho, bhikkhu, na jānāma, yatthime cattāro mahābhūtā aparisesā nirujjhanti, seyyathidaṃ – pathavīdhātu āpodhātu tejodhātu vāyodhātūti\footnote{vāyodhātu. atthi kho (pī. evamuparipi)}. Atthi kho\footnote{vāyodhātu. atthi kho (pī. evamuparipi)}, bhikkhu, cattāro mahārājāno amhehi abhikkantatarā ca paṇītatarā ca. Te kho etaṃ jāneyyuṃ, yatthime cattāro mahābhūtā aparisesā nirujjhanti, seyyathidaṃ – pathavīdhātu āpodhātu tejodhātu vāyodhātū’ti.

\paragraph{489.} ‘‘Atha kho so, kevaṭṭa, bhikkhu yena cattāro mahārājāno tenupasaṅkami; upasaṅkamitvā cattāro mahārāje etadavoca – ‘kattha nu kho, āvuso, ime cattāro mahābhūtā aparisesā nirujjhanti, seyyathidaṃ – pathavīdhātu āpodhātu tejodhātu vāyodhātū’ti? Evaṃ vutte, kevaṭṭa, cattāro mahārājāno taṃ bhikkhuṃ etadavocuṃ – ‘mayampi kho, bhikkhu, na jānāma, yatthime cattāro mahābhūtā aparisesā nirujjhanti, seyyathidaṃ – pathavīdhātu, āpodhātu tejodhātu vāyodhātūti. Atthi kho, bhikkhu, tāvatiṃsā nāma devā amhehi abhikkantatarā ca paṇītatarā ca. Te kho etaṃ jāneyyuṃ, yatthime cattāro mahābhūtā aparisesā nirujjhanti, seyyathidaṃ – pathavīdhātu āpodhātu tejodhātu vāyodhātū’ti.

\paragraph{490.} ‘‘Atha kho so, kevaṭṭa, bhikkhu yena tāvatiṃsā devā tenupasaṅkami; upasaṅkamitvā tāvatiṃse deve etadavoca – ‘kattha nu kho, āvuso, ime cattāro mahābhūtā aparisesā nirujjhanti, seyyathidaṃ – pathavīdhātu āpodhātu tejodhātu vāyodhātū’ti? Evaṃ vutte, kevaṭṭa, tāvatiṃsā devā taṃ bhikkhuṃ etadavocuṃ – ‘mayampi kho, bhikkhu, na jānāma, yatthime cattāro mahābhūtā aparisesā nirujjhanti, seyyathidaṃ – pathavīdhātu āpodhātu tejodhātu vāyodhātūti. Atthi kho, bhikkhu, sakko nāma devānamindo amhehi abhikkantataro ca paṇītataro ca. So kho etaṃ jāneyya, yatthime cattāro mahābhūtā aparisesā nirujjhanti, seyyathidaṃ – pathavīdhātu āpodhātu tejodhātu vāyodhātū’ti.

\paragraph{491.} ‘‘Atha kho so, kevaṭṭa, bhikkhu yena sakko devānamindo tenupasaṅkami; upasaṅkamitvā sakkaṃ devānamindaṃ etadavoca – ‘kattha nu kho, āvuso, ime cattāro mahābhūtā aparisesā nirujjhanti, seyyathidaṃ – pathavīdhātu āpodhātu tejodhātu vāyodhātū’ti? Evaṃ vutte, kevaṭṭa, sakko devānamindo taṃ bhikkhuṃ etadavoca – ‘ahampi kho, bhikkhu, na jānāmi, yatthime cattāro mahābhūtā aparisesā nirujjhanti, seyyathidaṃ – pathavīdhātu āpodhātu tejodhātu vāyodhātūti. Atthi kho, bhikkhu, yāmā nāma devā…pe… suyāmo nāma devaputto… tusitā nāma devā… santussito nāma devaputto… nimmānaratī nāma devā … sunimmito nāma devaputto… paranimmitavasavattī nāma devā… vasavattī nāma devaputto amhehi abhikkantataro ca paṇītataro ca. So kho etaṃ jāneyya, yatthime cattāro mahābhūtā aparisesā nirujjhanti, seyyathidaṃ – pathavīdhātu āpodhātu tejodhātu vāyodhātū’ti.

\paragraph{492.} ‘‘Atha kho so, kevaṭṭa, bhikkhu yena vasavattī devaputto tenupasaṅkami; upasaṅkamitvā vasavattiṃ devaputtaṃ etadavoca – ‘kattha nu kho, āvuso, ime cattāro mahābhūtā aparisesā nirujjhanti, seyyathidaṃ – pathavīdhātu āpodhātu tejodhātu vāyodhātū’ti? Evaṃ vutte, kevaṭṭa, vasavattī devaputto taṃ bhikkhuṃ etadavoca – ‘ahampi kho, bhikkhu, na jānāmi yatthime cattāro mahābhūtā aparisesā nirujjhanti, seyyathidaṃ – pathavīdhātu āpodhātu tejodhātu vāyodhātūti. Atthi kho, bhikkhu, brahmakāyikā nāma devā amhehi abhikkantatarā ca paṇītatarā ca. Te kho etaṃ jāneyyuṃ, yatthime cattāro mahābhūtā aparisesā nirujjhanti, seyyathidaṃ – pathavīdhātu āpodhātu tejodhātu vāyodhātū’ti.

\paragraph{493.} ‘‘Atha kho so, kevaṭṭa, bhikkhu tathārūpaṃ samādhiṃ samāpajji, yathāsamāhite citte brahmayāniyo maggo pāturahosi. Atha kho so, kevaṭṭa, bhikkhu yena brahmakāyikā devā tenupasaṅkami; upasaṅkamitvā brahmakāyike deve etadavoca – ‘kattha nu kho, āvuso, ime cattāro mahābhūtā aparisesā nirujjhanti, seyyathidaṃ – pathavīdhātu āpodhātu tejodhātu vāyodhātū’ti? Evaṃ vutte, kevaṭṭa, brahmakāyikā devā taṃ bhikkhuṃ etadavocuṃ – ‘mayampi kho, bhikkhu, na jānāma, yatthime cattāro mahābhūtā aparisesā nirujjhanti, seyyathidaṃ – pathavīdhātu āpodhātu tejodhātu vāyodhātūti. Atthi kho, bhikkhu, brahmā mahābrahmā abhibhū anabhibhūto aññadatthudaso vasavattī issaro kattā nimmātā seṭṭho sajitā vasī pitā bhūtabhabyānaṃ amhehi abhikkantataro ca paṇītataro ca. So kho etaṃ jāneyya, yatthime cattāro mahābhūtā aparisesā nirujjhanti, seyyathidaṃ – pathavīdhātu āpodhātu tejodhātu vāyodhātū’’ti. ‘‘‘Kahaṃ panāvuso, etarahi so mahābrahmā’ti? ‘Mayampi kho, bhikkhu, na jānāma, yattha vā brahmā yena vā brahmā yahiṃ vā brahmā; api ca, bhikkhu, yathā nimittā dissanti, āloko sañjāyati, obhāso pātubhavati, brahmā pātubhavissati, brahmuno hetaṃ pubbanimittaṃ pātubhāvāya, yadidaṃ āloko sañjāyati, obhāso pātubhavatī’ti. Atha kho so, kevaṭṭa, mahābrahmā nacirasseva pāturahosi.

\paragraph{494.} ‘‘Atha kho so, kevaṭṭa, bhikkhu yena so mahābrahmā tenupasaṅkami; upasaṅkamitvā taṃ mahābrahmānaṃ etadavoca – ‘kattha nu kho, āvuso, ime cattāro mahābhūtā aparisesā nirujjhanti, seyyathidaṃ – pathavīdhātu āpodhātu tejodhātu vāyodhātū’’ti? Evaṃ vutte, kevaṭṭa, so mahābrahmā taṃ bhikkhuṃ etadavoca – ‘ahamasmi, bhikkhu, brahmā mahābrahmā abhibhū anabhibhūto aññadatthudaso vasavattī issaro kattā nimmātā seṭṭho sajitā vasī pitā bhūtabhabyāna’nti. ‘‘Dutiyampi kho so, kevaṭṭa, bhikkhu taṃ mahābrahmānaṃ etadavoca – ‘na khohaṃ taṃ, āvuso, evaṃ pucchāmi – ‘‘tvamasi brahmā mahābrahmā abhibhū anabhibhūto aññadatthudaso vasavattī issaro kattā nimmātā seṭṭho sajitā vasī pitā bhūtabhabyāna’’nti. Evañca kho ahaṃ taṃ, āvuso, pucchāmi – ‘‘kattha nu kho, āvuso, ime cattāro mahābhūtā aparisesā nirujjhanti, seyyathidaṃ – pathavīdhātu āpodhātu tejodhātu vāyodhātū’’’ti? ‘‘Dutiyampi kho so, kevaṭṭa, mahābrahmā taṃ bhikkhuṃ etadavoca – ‘ahamasmi, bhikkhu, brahmā mahābrahmā abhibhū anabhibhūto aññadatthudaso vasavattī issaro kattā nimmātā seṭṭho sajitā vasī pitā bhūtabhabyāna’nti. Tatiyampi kho so, kevaṭṭa, bhikkhu taṃ mahābrahmānaṃ etadavoca – ‘na khohaṃ taṃ, āvuso, evaṃ pucchāmi – ‘‘tvamasi brahmā mahābrahmā abhibhū anabhibhūto aññadatthudaso vasavattī issaro kattā nimmātā seṭṭho sajitā vasī pitā bhūtabhabyāna’’nti. Evañca kho ahaṃ taṃ, āvuso, pucchāmi – ‘‘kattha nu kho, āvuso, ime cattāro mahābhūtā aparisesā nirujjhanti, seyyathidaṃ – pathavīdhātu āpodhātu tejodhātu vāyodhātū’’’ti?

\paragraph{495.} ‘‘Atha kho so, kevaṭṭa, mahābrahmā taṃ bhikkhuṃ bāhāyaṃ gahetvā ekamantaṃ apanetvā taṃ bhikkhuṃ etadavoca – ‘ime kho maṃ, bhikkhu, brahmakāyikā devā evaṃ jānanti, ‘‘natthi kiñci brahmuno aññātaṃ, natthi kiñci brahmuno adiṭṭhaṃ, natthi kiñci brahmuno aviditaṃ, natthi kiñci brahmuno asacchikata’’nti. Tasmāhaṃ tesaṃ sammukhā na byākāsiṃ. Ahampi kho, bhikkhu, na jānāmi yatthime cattāro mahābhūtā aparisesā nirujjhanti, seyyathidaṃ – pathavīdhātu āpodhātu tejodhātu vāyodhātūti. Tasmātiha, bhikkhu, tuyhevetaṃ dukkaṭaṃ, tuyhevetaṃ aparaddhaṃ, yaṃ tvaṃ taṃ bhagavantaṃ atidhāvitvā bahiddhā pariyeṭṭhiṃ āpajjasi imassa pañhassa veyyākaraṇāya. Gaccha tvaṃ, bhikkhu, tameva bhagavantaṃ upasaṅkamitvā imaṃ pañhaṃ puccha, yathā ca te bhagavā byākaroti, tathā naṃ dhāreyyāsī’ti.

\paragraph{496.} ‘‘Atha kho so, kevaṭṭa, bhikkhu – seyyathāpi nāma balavā puriso samiñjitaṃ vā bāhaṃ pasāreyya, pasāritaṃ vā bāhaṃ samiñjeyya evameva brahmaloke antarahito mama purato pāturahosi. Atha kho so, kevaṭṭa, bhikkhu maṃ abhivādetvā ekamantaṃ nisīdi, ekamantaṃ nisinno kho, kevaṭṭa, so bhikkhu maṃ etadavoca – ‘kattha nu kho, bhante, ime cattāro mahābhūtā aparisesā nirujjhanti, seyyathidaṃ – pathavīdhātu āpodhātu tejodhātu vāyodhātū’ti?

\subsubsection{Tīradassisakuṇupamā}

\paragraph{497.} ‘‘Evaṃ vutte, ahaṃ, kevaṭṭa, taṃ bhikkhuṃ etadavocaṃ – ‘bhūtapubbaṃ, bhikkhu, sāmuddikā vāṇijā tīradassiṃ sakuṇaṃ gahetvā nāvāya samuddaṃ ajjhogāhanti. Te atīradakkhiniyā nāvāya tīradassiṃ sakuṇaṃ muñcanti. So gacchateva puratthimaṃ disaṃ, gacchati dakkhiṇaṃ disaṃ, gacchati pacchimaṃ disaṃ, gacchati uttaraṃ disaṃ, gacchati uddhaṃ disaṃ, gacchati anudisaṃ. Sace so samantā tīraṃ passati, tathāgatakova\footnote{tathāpakkantova (syā.)} hoti. Sace pana so samantā tīraṃ na passati, tameva nāvaṃ paccāgacchati. Evameva kho tvaṃ, bhikkhu, yato yāva brahmalokā pariyesamāno imassa pañhassa veyyākaraṇaṃ nājjhagā, atha mamaññeva santike paccāgato. Na kho eso, bhikkhu, pañho evaṃ pucchitabbo – ‘kattha nu kho, bhante, ime cattāro mahābhūtā aparisesā nirujjhanti, seyyathidaṃ – pathavīdhātu āpodhātu tejodhātu vāyodhātū’ti?

\paragraph{498.} ‘‘Evañca kho eso, bhikkhu, pañho pucchitabbo –
\begin{verse}
  \small
  ‘Kattha āpo ca pathavī, tejo vāyo na gādhati;\\
  Kattha dīghañca rassañca, aṇuṃ thūlaṃ subhāsubhaṃ;\\
  Kattha nāmañca rūpañca, asesaṃ uparujjhatī’ti.\\
\end{verse}

\paragraph{499.} ‘‘Tatra veyyākaraṇaṃ bhavati –
\begin{verse}
  \small
  ‘Viññāṇaṃ anidassanaṃ, anantaṃ sabbatopabhaṃ;\\
  Ettha āpo ca pathavī, tejo vāyo na gādhati.\\[0.5cm]
  
  Ettha dīghañca rassañca, aṇuṃ thūlaṃ subhāsubhaṃ;\\
  Ettha nāmañca rūpañca, asesaṃ uparujjhati;\\
  Viññāṇassa nirodhena, etthetaṃ uparujjhatī’ti.\\
\end{verse}

\paragraph{500.} Idamavoca bhagavā. Attamano kevaṭṭo gahapatiputto bhagavato bhāsitaṃ abhinandīti.

\xsectionEnd{Kevaṭṭasuttaṃ niṭṭhitaṃ ekādasamaṃ.}
