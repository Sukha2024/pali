\section{Tevijjasuttaṃ}

\paragraph{518.} Evaṃ me sutaṃ – ekaṃ samayaṃ bhagavā kosalesu cārikaṃ caramāno mahatā bhikkhusaṅghena saddhiṃ pañcamattehi bhikkhusatehi yena manasākaṭaṃ nāma kosalānaṃ brāhmaṇagāmo tadavasari. Tatra sudaṃ bhagavā manasākaṭe viharati uttarena manasākaṭassa aciravatiyā nadiyā tīre ambavane.

\paragraph{519.} Tena kho pana samayena sambahulā abhiññātā abhiññātā brāhmaṇamahāsālā manasākaṭe paṭivasanti, seyyathidaṃ – caṅkī brāhmaṇo tārukkho brāhmaṇo pokkharasāti brāhmaṇo jāṇusoṇi brāhmaṇo todeyyo brāhmaṇo aññe ca abhiññātā abhiññātā brāhmaṇamahāsālā.

\paragraph{520.} Atha kho vāseṭṭhabhāradvājānaṃ māṇavānaṃ jaṅghavihāraṃ anucaṅkamantānaṃ anuvicarantānaṃ maggāmagge kathā udapādi. Atha kho vāseṭṭho māṇavo evamāha – ‘‘ayameva ujumaggo, ayamañjasāyano niyyāniko niyyāti takkarassa brahmasahabyatāya, yvāyaṃ akkhāto brāhmaṇena pokkharasātinā’’ti. Bhāradvājopi māṇavo evamāha – ‘‘ayameva ujumaggo, ayamañjasāyano niyyāniko, niyyāti takkarassa brahmasahabyatāya, yvāyaṃ akkhāto brāhmaṇena tārukkhenā’’ti. Neva kho asakkhi vāseṭṭho māṇavo bhāradvājaṃ māṇavaṃ saññāpetuṃ, na pana asakkhi bhāradvājo māṇavopi vāseṭṭhaṃ māṇavaṃ saññāpetuṃ.

\paragraph{521.} Atha kho vāseṭṭho māṇavo bhāradvājaṃ māṇavaṃ āmantesi – ‘‘ayaṃ kho, bhāradvāja, samaṇo gotamo sakyaputto sakyakulā pabbajito manasākaṭe viharati uttarena manasākaṭassa aciravatiyā nadiyā tīre ambavane. Taṃ kho pana bhavantaṃ gotamaṃ evaṃ kalyāṇo kittisaddo abbhuggato – ‘‘itipi so bhagavā arahaṃ sammāsambuddho vijjācaraṇasampanno sugato lokavidū anuttaro purisadammasārathi satthā devamanussānaṃ buddho bhagavā’’ti. Āyāma, bho bhāradvāja, yena samaṇo gotamo tenupasaṅkamissāma; upasaṅkamitvā etamatthaṃ samaṇaṃ gotamaṃ pucchissāma. Yathā no samaṇo gotamo byākarissati, tathā naṃ dhāressāmā’’ti. ‘‘Evaṃ, bho’’ti kho bhāradvājo māṇavo vāseṭṭhassa māṇavassa paccassosi.

\subsubsection{Maggāmaggakathā}

\paragraph{522.} Atha kho vāseṭṭhabhāradvājā māṇavā yena bhagavā tenupasaṅkamiṃsu; upasaṅkamitvā bhagavatā saddhiṃ sammodiṃsu. Sammodanīyaṃ kathaṃ sāraṇīyaṃ vītisāretvā ekamantaṃ nisīdiṃsu. Ekamantaṃ nisinno kho vāseṭṭho māṇavo bhagavantaṃ etadavoca – ‘‘idha, bho gotama, amhākaṃ jaṅghavihāraṃ anucaṅkamantānaṃ anuvicarantānaṃ maggāmagge kathā udapādi. Ahaṃ evaṃ vadāmi – ‘ayameva ujumaggo, ayamañjasāyano niyyāniko niyyāti takkarassa brahmasahabyatāya, yvāyaṃ akkhāto brāhmaṇena pokkharasātinā’ti. Bhāradvājo māṇavo evamāha – ‘ayameva ujumaggo ayamañjasāyano niyyāniko niyyāti takkarassa brahmasahabyatāya, yvāyaṃ akkhāto brāhmaṇena tārukkhenā’ti. Ettha, bho gotama, attheva viggaho, atthi vivādo, atthi nānāvādo’’ti.

\paragraph{523.} ‘‘Iti kira, vāseṭṭha, tvaṃ evaṃ vadesi – ‘‘ayameva ujumaggo, ayamañjasāyano niyyāniko niyyāti takkarassa brahmasahabyatāya, yvāyaṃ akkhāto brāhmaṇena pokkharasātinā’’ti. Bhāradvājo māṇavo evamāha – ‘‘ayameva ujumaggo ayamañjasāyano niyyāniko niyyāti takkarassa brahmasahabyatāya, yvāyaṃ akkhāto brāhmaṇena tārukkhenā’’ti. Atha kismiṃ pana vo, vāseṭṭha, viggaho, kismiṃ vivādo, kismiṃ nānāvādo’’ti?

\paragraph{524.} ‘‘Maggāmagge, bho gotama. Kiñcāpi, bho gotama, brāhmaṇā nānāmagge paññapenti, addhariyā brāhmaṇā tittiriyā brāhmaṇā chandokā brāhmaṇā bavhārijjhā brāhmaṇā, atha kho sabbāni tāni niyyānikā niyyanti takkarassa brahmasahabyatāya. ‘‘Seyyathāpi, bho gotama, gāmassa vā nigamassa vā avidūre bahūni cepi nānāmaggāni bhavanti, atha kho sabbāni tāni gāmasamosaraṇāni bhavanti; evameva kho, bho gotama, kiñcāpi brāhmaṇā nānāmagge paññapenti, addhariyā brāhmaṇā tittiriyā brāhmaṇā chandokā brāhmaṇā bavhārijjhā brāhmaṇā, atha kho sabbāni tāni niyyānikā niyyanti takkarassa brahmasahabyatāyā’’ti.

\subsubsection{Vāseṭṭhamāṇavānuyogo}

\paragraph{525.} ‘‘Niyyantīti vāseṭṭha vadesi’’? ‘‘Niyyantīti, bho gotama, vadāmi’’. ‘‘Niyyantīti, vāseṭṭha, vadesi’’? ‘‘Niyyantīti, bho gotama, vadāmi’’. ‘‘Niyyantīti, vāseṭṭha, vadesi’’? ‘‘Niyyantī’’ti, bho gotama, vadāmi’’. ‘‘Kiṃ pana, vāseṭṭha, atthi koci tevijjānaṃ brāhmaṇānaṃ ekabrāhmaṇopi, yena brahmā sakkhidiṭṭho’’ti? ‘‘No hidaṃ, bho gotama’’. ‘‘Kiṃ pana, vāseṭṭha, atthi koci tevijjānaṃ brāhmaṇānaṃ ekācariyopi, yena brahmā sakkhidiṭṭho’’ti? ‘‘No hidaṃ, bho gotama’’. ‘‘Kiṃ pana, vāseṭṭha, atthi koci tevijjānaṃ brāhmaṇānaṃ ekācariyapācariyopi, yena brahmā sakkhidiṭṭho’’ti? ‘‘No hidaṃ, bho gotama’’. ‘‘Kiṃ pana, vāseṭṭha, atthi koci tevijjānaṃ brāhmaṇānaṃ yāva sattamā ācariyāmahayugā\footnote{sattamācariyamahayugā (syā.)} yena brahmā sakkhidiṭṭho’’ti? ‘‘No hidaṃ, bho gotama’’.

\paragraph{526.} ‘‘Kiṃ pana, vāseṭṭha, yepi tevijjānaṃ brāhmaṇānaṃ pubbakā isayo mantānaṃ kattāro mantānaṃ pavattāro, yesamidaṃ etarahi tevijjā brāhmaṇā porāṇaṃ mantapadaṃ gītaṃ pavuttaṃ samihitaṃ\footnote{samīhitaṃ (syā.)}, tadanugāyanti, tadanubhāsanti, bhāsitamanubhāsanti, vācitamanuvācenti, seyyathidaṃ – aṭṭhako vāmako vāmadevo vessāmitto yamataggi aṅgīraso bhāradvājo vāseṭṭho kassapo bhagu. Tepi evamāhaṃsu – ‘mayametaṃ jānāma, mayametaṃ passāma, yattha vā brahmā, yena vā brahmā, yahiṃ vā brahmā’’’ti? ‘‘No hidaṃ, bho gotama’’.

\paragraph{527.} ‘‘Iti kira, vāseṭṭha, natthi koci tevijjānaṃ brāhmaṇānaṃ ekabrāhmaṇopi, yena brahmā sakkhidiṭṭho. Natthi koci tevijjānaṃ brāhmaṇānaṃ ekācariyopi, yena brahmā sakkhidiṭṭho. Natthi koci tevijjānaṃ brāhmaṇānaṃ ekācariyapācariyopi, yena brahmā sakkhidiṭṭho. Natthi koci tevijjānaṃ brāhmaṇānaṃ yāva sattamā ācariyāmahayugā yena brahmā sakkhidiṭṭho. Yepi kira tevijjānaṃ brāhmaṇānaṃ pubbakā isayo mantānaṃ kattāro mantānaṃ pavattāro, yesamidaṃ etarahi tevijjā brāhmaṇā porāṇaṃ mantapadaṃ gītaṃ pavuttaṃ samihitaṃ, tadanugāyanti, tadanubhāsanti, bhāsitamanubhāsanti, vācitamanuvācenti, seyyathidaṃ – aṭṭhako vāmako vāmadevo vessāmitto yamataggi aṅgīraso bhāradvājo vāseṭṭho kassapo bhagu, tepi na evamāhaṃsu – ‘mayametaṃ jānāma, mayametaṃ passāma, yattha vā brahmā, yena vā brahmā, yahiṃ vā brahmā’ti. Teva tevijjā brāhmaṇā evamāhaṃsu – ‘yaṃ na jānāma, yaṃ na passāma, tassa sahabyatāya maggaṃ desema. Ayameva ujumaggo ayamañjasāyano niyyāniko, niyyāti takkarassa brahmasahabyatāyā’’’ti.

\paragraph{528.} ‘‘Taṃ kiṃ maññasi, vāseṭṭha, nanu evaṃ sante tevijjānaṃ brāhmaṇānaṃ appāṭihīrakataṃ bhāsitaṃ sampajjatī’’ti? ‘‘Addhā kho, bho gotama, evaṃ sante tevijjānaṃ brāhmaṇānaṃ appāṭihīrakataṃ bhāsitaṃ sampajjatī’’ti. ‘‘Sādhu, vāseṭṭha, te vata\footnote{teva (ka.)}, vāseṭṭha, tevijjā brāhmaṇā yaṃ na jānanti, yaṃ na passanti, tassa sahabyatāya maggaṃ desessanti. ‘Ayameva ujumaggo, ayamañjasāyano niyyāniko, niyyāti takkarassa brahmasahabyatāyā’ti, netaṃ ṭhānaṃ vijjati.

\paragraph{529.} ‘‘Seyyathāpi, vāseṭṭha, andhaveṇi paramparasaṃsattā purimopi na passati, majjhimopi na passati, pacchimopi na passati. Evameva kho, vāseṭṭha, andhaveṇūpamaṃ maññe tevijjānaṃ brāhmaṇānaṃ bhāsitaṃ, purimopi na passati, majjhimopi na passati, pacchimopi na passati. Tesamidaṃ tevijjānaṃ brāhmaṇānaṃ bhāsitaṃ hassakaññeva sampajjati, nāmakaññeva sampajjati, rittakaññeva sampajjati, tucchakaññeva sampajjati.

\paragraph{530.} ‘‘Taṃ kiṃ maññasi, vāseṭṭha, passanti tevijjā brāhmaṇā candimasūriye, aññe cāpi bahujanā, yato ca candimasūriyā uggacchanti, yattha ca ogacchanti, āyācanti thomayanti pañjalikā namassamānā anuparivattantī’’ti? ‘‘Evaṃ, bho gotama, passanti tevijjā brāhmaṇā candimasūriye, aññe cāpi bahujanā, yato ca candimasūriyā uggacchanti, yattha ca ogacchanti, āyācanti thomayanti pañjalikā namassamānā anuparivattantī’’ti.

\paragraph{531.} ‘‘Taṃ kiṃ maññasi, vāseṭṭha, yaṃ passanti tevijjā brāhmaṇā candimasūriye, aññe cāpi bahujanā, yato ca candimasūriyā uggacchanti, yattha ca ogacchanti, āyācanti thomayanti pañjalikā namassamānā anuparivattanti, pahonti tevijjā brāhmaṇā candimasūriyānaṃ sahabyatāya maggaṃ desetuṃ – ‘‘ayameva ujumaggo, ayamañjasāyano niyyāniko, niyyāti takkarassa candimasūriyānaṃ sahabyatāyā’’ti? ‘‘No hidaṃ, bho gotama’’. ‘‘Iti kira, vāseṭṭha, yaṃ passanti tevijjā brāhmaṇā candimasūriye, aññe cāpi bahujanā, yato ca candimasūriyā uggacchanti, yattha ca ogacchanti, āyācanti thomayanti pañjalikā namassamānā anuparivattanti, tesampi nappahonti candimasūriyānaṃ sahabyatāya maggaṃ desetuṃ – ‘‘ayameva ujumaggo, ayamañjasāyano niyyāniko, niyyāti takkarassa candimasūriyānaṃ sahabyatāyā’’ti.

\paragraph{532.} ‘‘Iti pana\footnote{kiṃ pana (sī. syā. pī.)} na kira tevijjehi brāhmaṇehi brahmā sakkhidiṭṭho. Napi kira tevijjānaṃ brāhmaṇānaṃ ācariyehi brahmā sakkhidiṭṭho. Napi kira tevijjānaṃ brāhmaṇānaṃ ācariyapācariyehi brahmā sakkhidiṭṭho. Napi kira tevijjānaṃ brāhmaṇānaṃ yāva sattamā\footnote{sattamehi (?)} ācariyāmahayugehi brahmā sakkhidiṭṭho. Yepi kira tevijjānaṃ brāhmaṇānaṃ pubbakā isayo mantānaṃ kattāro mantānaṃ pavattāro, yesamidaṃ etarahi tevijjā brāhmaṇā porāṇaṃ mantapadaṃ gītaṃ pavuttaṃ samihitaṃ, tadanugāyanti, tadanubhāsanti, bhāsitamanubhāsanti, vācitamanuvācenti, seyyathidaṃ – aṭṭhako vāmako vāmadevo vessāmitto yamataggi aṅgīraso bhāradvājo vāseṭṭho kassapo bhagu, tepi na evamāhaṃsu – ‘‘mayametaṃ jānāma, mayametaṃ passāma, yattha vā brahmā, yena vā brahmā, yahiṃ vā brahmā’’ti. Teva tevijjā brāhmaṇā evamāhaṃsu – ‘‘yaṃ na jānāma, yaṃ na passāma, tassa sahabyatāya maggaṃ desema – ayameva ujumaggo ayamañjasāyano niyyāniko niyyāti takkarassa brahmasahabyatāyā’’ti.

\paragraph{533.} ‘‘Taṃ kiṃ maññasi, vāseṭṭha, nanu evaṃ sante tevijjānaṃ brāhmaṇānaṃ appāṭihīrakataṃ bhāsitaṃ sampajjatī’’ti? ‘‘Addhā kho, bho gotama, evaṃ sante tevijjānaṃ brāhmaṇānaṃ appāṭihīrakataṃ bhāsitaṃ sampajjatī’’ti. ‘‘Sādhu, vāseṭṭha, te vata, vāseṭṭha, tevijjā brāhmaṇā yaṃ na jānanti, yaṃ na passanti, tassa sahabyatāya maggaṃ desessanti – ‘‘ayameva ujumaggo, ayamañjasāyano niyyāniko, niyyāti takkarassa brahmasahabyatāyā’’ti, netaṃ ṭhānaṃ vijjati.

\subsubsection{Janapadakalyāṇīupamā}

\paragraph{534.} ‘‘Seyyathāpi, vāseṭṭha, puriso evaṃ vadeyya – ‘‘ahaṃ yā imasmiṃ janapade janapadakalyāṇī, taṃ icchāmi, taṃ kāmemī’’ti. Tamenaṃ evaṃ vadeyyuṃ – ‘‘ambho purisa, yaṃ tvaṃ janapadakalyāṇiṃ icchasi kāmesi, jānāsi taṃ janapadakalyāṇiṃ – khattiyī vā brāhmaṇī vā vessī vā suddī vā’’ti? Iti puṭṭho ‘‘no’’ti vadeyya. ‘‘Tamenaṃ evaṃ vadeyyuṃ – ‘‘ambho purisa, yaṃ tvaṃ janapadakalyāṇiṃ icchasi kāmesi, jānāsi taṃ janapadakalyāṇiṃ – evaṃnāmā evaṃgottāti vā, dīghā vā rassā vā majjhimā vā kāḷī vā sāmā vā maṅguracchavī vāti, amukasmiṃ gāme vā nigame vā nagare vā’’ti? Iti puṭṭho ‘no’ti vadeyya. Tamenaṃ evaṃ vadeyyuṃ – ‘‘ambho purisa, yaṃ tvaṃ na jānāsi na passasi, taṃ tvaṃ icchasi kāmesī’’ti? Iti puṭṭho ‘‘āmā’’ti vadeyya.

\paragraph{535.} ‘‘Taṃ kiṃ maññasi, vāseṭṭha, nanu evaṃ sante tassa purisassa appāṭihīrakataṃ bhāsitaṃ sampajjatī’’ti? ‘‘Addhā kho, bho gotama, evaṃ sante tassa purisassa appāṭihīrakataṃ bhāsitaṃ sampajjatī’’ti.

\paragraph{536.} ‘‘Evameva kho, vāseṭṭha, na kira tevijjehi brāhmaṇehi brahmā sakkhidiṭṭho, napi kira tevijjānaṃ brāhmaṇānaṃ ācariyehi brahmā sakkhidiṭṭho, napi kira tevijjānaṃ brāhmaṇānaṃ ācariyapācariyehi brahmā sakkhidiṭṭho. Napi kira tevijjānaṃ brāhmaṇānaṃ yāva sattamā ācariyāmahayugehi brahmā sakkhidiṭṭho. Yepi kira tevijjānaṃ brāhmaṇānaṃ pubbakā isayo mantānaṃ kattāro mantānaṃ pavattāro, yesamidaṃ etarahi tevijjā brāhmaṇā porāṇaṃ mantapadaṃ gītaṃ pavuttaṃ samihitaṃ, tadanugāyanti, tadanubhāsanti, bhāsitamanubhāsanti, vācitamanuvācenti, seyyathidaṃ – aṭṭhako vāmako vāmadevo vessāmitto yamataggi aṅgīraso bhāradvājo vāseṭṭho kassapo bhagu, tepi na evamāhaṃsu – ‘‘mayametaṃ jānāma, mayametaṃ passāma, yattha vā brahmā, yena vā brahmā, yahiṃ vā brahmā’’ti. Teva tevijjā brāhmaṇā evamāhaṃsu – ‘‘yaṃ na jānāma, yaṃ na passāma, tassa sahabyatāya maggaṃ desema – ayameva ujumaggo ayamañjasāyano niyyāniko niyyāti takkarassa brahmasahabyatāyā’’ti.

\paragraph{537.} ‘‘Taṃ kiṃ maññasi, vāseṭṭha, nanu evaṃ sante tevijjānaṃ brāhmaṇānaṃ appāṭihīrakataṃ bhāsitaṃ sampajjatī’’ti? ‘‘Addhā kho, bho gotama, evaṃ sante tevijjānaṃ brāhmaṇānaṃ appāṭihīrakataṃ bhāsitaṃ sampajjatī’’ti. ‘‘Sādhu, vāseṭṭha, te vata, vāseṭṭha, tevijjā brāhmaṇā yaṃ na jānanti, yaṃ na passanti, tassa sahabyatāya maggaṃ desessanti – ayameva ujumaggo ayamañjasāyano niyyāniko niyyāti takkarassa brahmasahabyatāyāti netaṃ ṭhānaṃ vijjati.

\subsubsection{Nisseṇīupamā}

\paragraph{538.} ‘‘Seyyathāpi, vāseṭṭha, puriso cātumahāpathe nisseṇiṃ kareyya – pāsādassa ārohaṇāya. Tamenaṃ evaṃ vadeyyuṃ – ‘‘ambho purisa, yassa tvaṃ\footnote{yaṃ tvaṃ (syā.)} pāsādassa ārohaṇāya nisseṇiṃ karosi, jānāsi taṃ pāsādaṃ – puratthimāya vā disāya dakkhiṇāya vā disāya pacchimāya vā disāya uttarāya vā disāya ucco vā nīco vā majjhimo vā’’ti? Iti puṭṭho ‘‘no’’ti vadeyya. ‘‘Tamenaṃ evaṃ vadeyyuṃ – ‘‘ambho purisa, yaṃ tvaṃ na jānāsi, na passasi, tassa tvaṃ pāsādassa ārohaṇāya nisseṇiṃ karosī’’ti? Iti puṭṭho ‘‘āmā’’ti vadeyya.

\paragraph{539.} ‘‘Taṃ kiṃ maññasi, vāseṭṭha, nanu evaṃ sante tassa purisassa appāṭihīrakataṃ bhāsitaṃ sampajjatī’’ti? ‘‘Addhā kho, bho gotama, evaṃ sante tassa purisassa appāṭihīrakataṃ bhāsitaṃ sampajjatī’’ti.

\paragraph{540.} ‘‘Evameva kho, vāseṭṭha, na kira tevijjehi brāhmaṇehi brahmā sakkhidiṭṭho, napi kira tevijjānaṃ brāhmaṇānaṃ ācariyehi brahmā sakkhidiṭṭho, napi kira tevijjānaṃ brāhmaṇānaṃ ācariyapācariyehi brahmā sakkhidiṭṭho, napi kira tevijjānaṃ brāhmaṇānaṃ yāva sattamā ācariyāmahayugehi brahmā sakkhidiṭṭho. Yepi kira tevijjānaṃ brāhmaṇānaṃ pubbakā isayo mantānaṃ kattāro mantānaṃ pavattāro, yesamidaṃ etarahi tevijjā brāhmaṇā porāṇaṃ mantapadaṃ gītaṃ pavuttaṃ samihitaṃ, tadanugāyanti, tadanubhāsanti, bhāsitamanubhāsanti, vācitamanuvācenti, seyyathidaṃ – aṭṭhako vāmako vāmadevo vessāmitto yamataggi aṅgīraso bhāradvājo vāseṭṭho kassapo bhagu, tepi na evamāhaṃsu – mayametaṃ jānāma, mayametaṃ passāma, yattha vā brahmā, yena vā brahmā, yahiṃ vā brahmāti. Teva tevijjā brāhmaṇā evamāhaṃsu – ‘‘yaṃ na jānāma, yaṃ na passāma, tassa sahabyatāya maggaṃ desema, ayameva ujumaggo ayamañjasāyano niyyāniko niyyāti takkarassa brahmasahabyatāyā’’ti.

\paragraph{541.} ‘‘Taṃ kiṃ maññasi, vāseṭṭha, nanu evaṃ sante tevijjānaṃ brāhmaṇānaṃ appāṭihīrakataṃ bhāsitaṃ sampajjatī’’ti? ‘‘Addhā kho, bho gotama, evaṃ sante tevijjānaṃ brāhmaṇānaṃ appāṭihīrakataṃ bhāsitaṃ sampajjatī’’ti. ‘‘Sādhu, vāseṭṭha. Te vata, vāseṭṭha, tevijjā brāhmaṇā yaṃ na jānanti, yaṃ na passanti, tassa sahabyatāya maggaṃ desessanti. Ayameva ujumaggo ayamañjasāyano niyyāniko niyyāti takkarassa brahmasabyatāyāti, netaṃ ṭhānaṃ vijjati.

\subsubsection{Aciravatīnadīupamā}

\paragraph{542.} ‘‘Seyyathāpi, vāseṭṭha, ayaṃ aciravatī nadī pūrā udakassa samatittikā kākapeyyā. Atha puriso āgaccheyya pāratthiko pāragavesī pāragāmī pāraṃ taritukāmo. So orime tīre ṭhito pārimaṃ tīraṃ avheyya – ‘‘ehi pārāpāraṃ, ehi pārāpāra’’nti.

\paragraph{543.} ‘‘Taṃ kiṃ maññasi, vāseṭṭha, api nu tassa purisassa avhāyanahetu vā āyācanahetu vā patthanahetu vā abhinandanahetu vā aciravatiyā nadiyā pārimaṃ tīraṃ orimaṃ tīraṃ āgaccheyyā’’ti? ‘‘No hidaṃ, bho gotama’’.

\paragraph{544.} ‘‘Evameva kho, vāseṭṭha, tevijjā brāhmaṇā ye dhammā brāhmaṇakārakā te dhamme pahāya vattamānā, ye dhammā abrāhmaṇakārakā te dhamme samādāya vattamānā evamāhaṃsu – ‘‘indamavhayāma, somamavhayāma, varuṇamavhayāma, īsānamavhayāma, pajāpatimavhayāma, brahmamavhayāma, mahiddhimavhayāma, yamamavhayāmā’’ti. ‘‘Te vata, vāseṭṭha, tevijjā brāhmaṇā ye dhammā brāhmaṇakārakā te dhamme pahāya vattamānā, ye dhammā abrāhmaṇakārakā te dhamme samādāya vattamānā avhāyanahetu vā āyācanahetu vā patthanahetu vā abhinandanahetu vā kāyassa bhedā paraṃ maraṇā brahmānaṃ sahabyūpagā bhavissantī’’ti, netaṃ ṭhānaṃ vijjati.

\paragraph{545.} ‘‘Seyyathāpi, vāseṭṭha, ayaṃ aciravatī nadī pūrā udakassa samatittikā kākapeyyā. Atha puriso āgaccheyya pāratthiko pāragavesī pāragāmī pāraṃ taritukāmo. So orime tīre daḷhāya anduyā pacchābāhaṃ gāḷhabandhanaṃ baddho. ‘‘Taṃ kiṃ maññasi, vāseṭṭha, api nu so puriso aciravatiyā nadiyā orimā tīrā pārimaṃ tīraṃ gaccheyyā’’ti? ‘‘No hidaṃ, bho gotama’’.

\paragraph{546.} ‘‘Evameva kho, vāseṭṭha, pañcime kāmaguṇā ariyassa vinaye andūtipi vuccanti, bandhanantipi vuccanti. Katame pañca? Cakkhuviññeyyā rūpā iṭṭhā kantā manāpā piyarūpā kāmūpasaṃhitā rajanīyā. Sotaviññeyyā saddā…pe… ghānaviññeyyā gandhā… jivhāviññeyyā rasā… kāyaviññeyyā phoṭṭhabbā iṭṭhā kantā manāpā piyarūpā kāmūpasaṃhitā rajanīyā. ‘‘Ime kho, vāseṭṭha, pañca kāmaguṇā ariyassa vinaye andūtipi vuccanti, bandhanantipi vuccanti. Ime kho vāseṭṭha pañca kāmaguṇe tevijjā brāhmaṇā gadhitā mucchitā ajjhopannā anādīnavadassāvino anissaraṇapaññā paribhuñjanti. Te vata, vāseṭṭha, tevijjā brāhmaṇā ye dhammā brāhmaṇakārakā, te dhamme pahāya vattamānā, ye dhammā abrāhmaṇakārakā, te dhamme samādāya vattamānā pañca kāmaguṇe gadhitā mucchitā ajjhopannā anādīnavadassāvino anissaraṇapaññā paribhuñjantā kāmandubandhanabaddhā kāyassa bhedā paraṃ maraṇā brahmānaṃ sahabyūpagā bhavissantī’’ti, netaṃ ṭhānaṃ vijjati.

\paragraph{547.} ‘‘Seyyathāpi, vāseṭṭha, ayaṃ aciravatī nadī pūrā udakassa samatittikā kākapeyyā. Atha puriso āgaccheyya pāratthiko pāragavesī pāragāmī pāraṃ taritukāmo. So orime tīre sasīsaṃ pārupitvā nipajjeyya. ‘‘Taṃ kiṃ maññasi, vāseṭṭha, api nu so puriso aciravatiyā nadiyā orimā tīrā pārimaṃ tīraṃ gaccheyyā’’ti? ‘‘No hidaṃ, bho gotama’’.

\paragraph{548.} ‘‘Evameva kho, vāseṭṭha, pañcime nīvaraṇā ariyassa vinaye āvaraṇātipi vuccanti, nīvaraṇātipi vuccanti, onāhanātipi vuccanti, pariyonāhanātipi vuccanti. Katame pañca? Kāmacchandanīvaraṇaṃ, byāpādanīvaraṇaṃ, thinamiddhanīvaraṇaṃ, uddhaccakukkuccanīvaraṇaṃ, vicikicchānīvaraṇaṃ. Ime kho, vāseṭṭha, pañca nīvaraṇā ariyassa vinaye āvaraṇātipi vuccanti, nīvaraṇātipi vuccanti, onāhanātipi vuccanti, pariyonāhanātipi vuccanti.

\paragraph{549.} ‘‘Imehi kho, vāseṭṭha, pañcahi nīvaraṇehi tevijjā brāhmaṇā āvuṭā nivuṭā onaddhā\footnote{ophuṭā (sī. ka.), ophutā (syā.)} pariyonaddhā. Te vata, vāseṭṭha, tevijjā brāhmaṇā ye dhammā brāhmaṇakārakā te dhamme pahāya vattamānā, ye dhammā abrāhmaṇakārakā te dhamme samādāya vattamānā pañcahi nīvaraṇehi āvuṭā nivuṭā onaddhā pariyonaddhā\footnote{pariyonaddhā, te (syā. ka.)} kāyassa bhedā paraṃ maraṇā brahmānaṃ sahabyūpagā bhavissantī’’ti, netaṃ ṭhānaṃ vijjati.

\subsubsection{Saṃsandanakathā}

\paragraph{550.} ‘‘Taṃ kiṃ maññasi, vāseṭṭha, kinti te sutaṃ brāhmaṇānaṃ vuddhānaṃ mahallakānaṃ ācariyapācariyānaṃ bhāsamānānaṃ, sapariggaho vā brahmā apariggaho vā’’ti? ‘‘Apariggaho, bho gotama’’. ‘‘Saveracitto vā averacitto vā’’ti? ‘‘Averacitto, bho gotama’’. ‘‘Sabyāpajjacitto vā abyāpajjacitto vā’’ti? ‘‘Abyāpajjacitto, bho gotama’’. ‘‘Saṃkiliṭṭhacitto vā asaṃkiliṭṭhacitto vā’’ti? ‘‘Asaṃkiliṭṭhacitto, bho gotama’’. ‘‘Vasavattī vā avasavattī vā’’ti? ‘‘Vasavattī, bho gotama’’. ‘‘Taṃ kiṃ maññasi, vāseṭṭha, sapariggahā vā tevijjā brāhmaṇā apariggahā vā’’ti? ‘‘Sapariggahā, bho gotama’’. ‘‘Saveracittā vā averacittā vā’’ti? ‘‘Saveracittā, bho gotama’’. ‘‘Sabyāpajjacittā vā abyāpajjacittā vā’’ti? ‘‘Sabyāpajjacittā, bho gotama’’. ‘‘Saṃkiliṭṭhacittā vā asaṃkiliṭṭhacittā vā’’ti? ‘‘Saṃkiliṭṭhacittā, bho gotama’’. ‘‘Vasavattī vā avasavattī vā’’ti? ‘‘Avasavattī, bho gotama’’.

\paragraph{551.} ‘‘Iti kira, vāseṭṭha, sapariggahā tevijjā brāhmaṇā apariggaho brahmā. Api nu kho sapariggahānaṃ tevijjānaṃ brāhmaṇānaṃ apariggahena brahmunā saddhiṃ saṃsandati sametī’’ti? ‘‘No hidaṃ, bho gotama’’. ‘‘Sādhu, vāseṭṭha, te vata, vāseṭṭha, sapariggahā tevijjā brāhmaṇā kāyassa bhedā paraṃ maraṇā apariggahassa brahmuno sahabyūpagā bhavissantī’’ti, netaṃ ṭhānaṃ vijjati. ‘‘Iti kira, vāseṭṭha, saveracittā tevijjā brāhmaṇā, averacitto brahmā…pe… sabyāpajjacittā tevijjā brāhmaṇā abyāpajjacitto brahmā… saṃkiliṭṭhacittā tevijjā brāhmaṇā asaṃkiliṭṭhacitto brahmā… avasavattī tevijjā brāhmaṇā vasavattī brahmā, api nu kho avasavattīnaṃ tevijjānaṃ brāhmaṇānaṃ vasavattinā brahmunā saddhiṃ saṃsandati sametī’’ti? ‘‘No hidaṃ, bho gotama’’. ‘‘Sādhu, vāseṭṭha, te vata, vāseṭṭha, avasavattī tevijjā brāhmaṇā kāyassa bhedā paraṃ maraṇā vasavattissa brahmuno sahabyūpagā bhavissantī’’ti, netaṃ ṭhānaṃ vijjati.

\paragraph{552.} ‘‘Idha kho pana te, vāseṭṭha, tevijjā brāhmaṇā āsīditvā\footnote{ādisitvā (ka.)} saṃsīdanti, saṃsīditvā visāraṃ\footnote{visādaṃ (sī. pī.), visattaṃ (syā.)} pāpuṇanti, sukkhataraṃ\footnote{sukkhataraṇaṃ (ka.)} maññe taranti. Tasmā idaṃ tevijjānaṃ brāhmaṇānaṃ tevijjāiriṇantipi vuccati, tevijjāvivanantipi vuccati, tevijjābyasanantipi vuccatī’’ti.

\paragraph{553.} Evaṃ vutte, vāseṭṭho māṇavo bhagavantaṃ etadavoca – ‘‘sutaṃ metaṃ, bho gotama, samaṇo gotamo brahmānaṃ sahabyatāya maggaṃ jānātī’’ti. ‘‘Taṃ kiṃ maññasi, vāseṭṭha. Āsanne ito manasākaṭaṃ, na ito dūre manasākaṭa’’nti? ‘‘Evaṃ, bho gotama, āsanne ito manasākaṭaṃ, na ito dūre manasākaṭa’’nti.

\paragraph{554.} ‘‘Taṃ kiṃ maññasi, vāseṭṭha, idhassa puriso manasākaṭe jātasaṃvaddho. Tamenaṃ manasākaṭato tāvadeva avasaṭaṃ manasākaṭassa maggaṃ puccheyyuṃ. Siyā nu kho, vāseṭṭha, tassa purisassa manasākaṭe jātasaṃvaddhassa manasākaṭassa maggaṃ puṭṭhassa dandhāyitattaṃ vā vitthāyitattaṃ vā’’ti? ‘‘No hidaṃ, bho gotama’’. ‘‘Taṃ kissa hetu’’? ‘‘Amu hi, bho gotama, puriso manasākaṭe jātasaṃvaddho, tassa sabbāneva manasākaṭassa maggāni suviditānī’’ti. ‘‘Siyā kho, vāseṭṭha, tassa purisassa manasākaṭe jātasaṃvaddhassa manasākaṭassa maggaṃ puṭṭhassa dandhāyitattaṃ vā vitthāyitattaṃ vā, na tveva tathāgatassa brahmaloke vā brahmalokagāminiyā vā paṭipadāya puṭṭhassa dandhāyitattaṃ vā vitthāyitattaṃ vā. Brahmānaṃ cāhaṃ, vāseṭṭha, pajānāmi brahmalokañca brahmalokagāminiñca paṭipadaṃ, yathā paṭipanno ca brahmalokaṃ upapanno, tañca pajānāmī’’ti.

\paragraph{555.} Evaṃ vutte, vāseṭṭho māṇavo bhagavantaṃ etadavoca – ‘‘sutaṃ metaṃ, bho gotama, samaṇo gotamo brahmānaṃ sahabyatāya maggaṃ desetī’’ti. ‘‘Sādhu no bhavaṃ gotamo brahmānaṃ sahabyatāya maggaṃ desetu ullumpatu bhavaṃ gotamo brāhmaṇiṃ paja’’nti. ‘‘Tena hi, vāseṭṭha, suṇāhi; sādhukaṃ manasi karohi; bhāsissāmī’’ti. ‘‘Evaṃ bho’’ti kho vāseṭṭho māṇavo bhagavato paccassosi.

\subsubsection{Brahmalokamaggadesanā}

\paragraph{556.} Bhagavā etadavoca – ‘‘idha, vāseṭṭha, tathāgato loke uppajjati arahaṃ, sammāsambuddho…pe… (yathā 190-212 anucchedesu evaṃ vitthāretabbaṃ). Evaṃ kho, vāseṭṭha, bhikkhu sīlasampanno hoti…pe… tassime pañca nīvaraṇe pahīne attani samanupassato pāmojjaṃ jāyati, pamuditassa pīti jāyati, pītimanassa kāyo passambhati, passaddhakāyo sukhaṃ vedeti, sukhino cittaṃ samādhiyati. ‘‘So mettāsahagatena cetasā ekaṃ disaṃ pharitvā viharati. Tathā dutiyaṃ. Tathā tatiyaṃ. Tathā catutthaṃ. Iti uddhamadho tiriyaṃ sabbadhi sabbattatāya sabbāvantaṃ lokaṃ mettāsahagatena cetasā vipulena mahaggatena appamāṇena averena abyāpajjena pharitvā viharati. ‘‘Seyyathāpi, vāseṭṭha, balavā saṅkhadhamo appakasireneva catuddisā viññāpeyya; evameva kho, vāseṭṭha, evaṃ bhāvitāya mettāya cetovimuttiyā yaṃ pamāṇakataṃ kammaṃ na taṃ tatrāvasissati, na taṃ tatrāvatiṭṭhati. Ayampi kho, vāseṭṭha, brahmānaṃ sahabyatāya maggo. ‘‘Puna caparaṃ, vāseṭṭha, bhikkhu karuṇāsahagatena cetasā…pe… muditāsahagatena cetasā…pe… upekkhāsahagatena cetasā ekaṃ disaṃ pharitvā viharati. Tathā dutiyaṃ. Tathā tatiyaṃ. Tathā catutthaṃ. Iti uddhamadho tiriyaṃ sabbadhi sabbattatāya sabbāvantaṃ lokaṃ upekkhāsahagatena cetasā vipulena mahaggatena appamāṇena averena abyāpajjena pharitvā viharati. ‘‘Seyyathāpi, vāseṭṭha, balavā saṅkhadhamo appakasireneva catuddisā viññāpeyya. Evameva kho, vāseṭṭha, evaṃ bhāvitāya upekkhāya cetovimuttiyā yaṃ pamāṇakataṃ kammaṃ na taṃ tatrāvasissati, na taṃ tatrāvatiṭṭhati. Ayaṃ kho, vāseṭṭha, brahmānaṃ sahabyatāya maggo.

\paragraph{557.} ‘‘Taṃ kiṃ maññasi, vāseṭṭha, evaṃvihārī bhikkhu sapariggaho vā apariggaho vā’’ti? ‘‘Apariggaho, bho gotama’’. ‘‘Saveracitto vā averacitto vā’’ti? ‘‘Averacitto, bho gotama’’. ‘‘Sabyāpajjacitto vā abyāpajjacitto vā’’ti? ‘‘Abyāpajjacitto, bho gotama’’. ‘‘Saṃkiliṭṭhacitto vā asaṃkiliṭṭhacitto vā’’ti? ‘‘Asaṃkiliṭṭhacitto, bho gotama’’. ‘‘Vasavattī vā avasavattī vā’’ti? ‘‘Vasavattī, bho gotama’’. ‘‘Iti kira, vāseṭṭha, apariggaho bhikkhu, apariggaho brahmā. Api nu kho apariggahassa bhikkhuno apariggahena brahmunā saddhiṃ saṃsandati sametī’’ti? ‘‘Evaṃ, bho gotama’’. ‘‘Sādhu, vāseṭṭha, so vata vāseṭṭha apariggaho bhikkhu kāyassa bhedā paraṃ maraṇā apariggahassa brahmuno sahabyūpago bhavissatī’’ti, ṭhānametaṃ vijjati.

\paragraph{558.} ‘‘Iti kira, vāseṭṭha, averacitto bhikkhu, averacitto brahmā…pe… abyāpajjacitto bhikkhu, abyāpajjacitto brahmā… asaṃkiliṭṭhacitto bhikkhu, asaṃkiliṭṭhacitto brahmā… vasavattī bhikkhu, vasavattī brahmā, api nu kho vasavattissa bhikkhuno vasavattinā brahmunā saddhiṃ saṃsandati sametī’’ti? ‘‘Evaṃ, bho gotama’’. ‘‘Sādhu, vāseṭṭha, so vata, vāseṭṭha, vasavattī bhikkhu kāyassa bhedā paraṃ maraṇā vasavattissa brahmuno sahabyūpago bhavissatīti, ṭhānametaṃ vijjatī’’ti.

\paragraph{559.} Evaṃ vutte, vāseṭṭhabhāradvājā māṇavā bhagavantaṃ etadavocuṃ – ‘‘abhikkantaṃ, bho gotama, abhikkantaṃ, bho gotama! Seyyathāpi, bho gotama, nikkujjitaṃ vā ukkujjeyya, paṭicchannaṃ vā vivareyya, mūḷhassa vā maggaṃ ācikkheyya, andhakāre vā telapajjotaṃ dhāreyya ‘cakkhumanto rūpāni dakkhantī’ti. Evamevaṃ bhotā gotamena anekapariyāyena dhammo pakāsito. Ete mayaṃ bhavantaṃ gotamaṃ saraṇaṃ gacchāma, dhammañca bhikkhusaṅghañca. Upāsake no bhavaṃ gotamo dhāretu ajjatagge pāṇupete saraṇaṃ gate’’ti.

\xsectionEnd{Tevijjasuttaṃ niṭṭhitaṃ terasamaṃ. \\ Sīlakkhandhavaggo niṭṭhito.}

\paragraph{}
Tassuddānaṃ –
\begin{verse}
  Brahmāsāmaññaambaṭṭha,\\
  Soṇakūṭamahālijālinī;\\
  Sīhapoṭṭhapādasubho kevaṭṭo,\\
  Lohiccatevijjā terasāti.\\
\end{verse}

\xsectionEnd{Sīlakkhandhavaggapāḷi niṭṭhitā.}
