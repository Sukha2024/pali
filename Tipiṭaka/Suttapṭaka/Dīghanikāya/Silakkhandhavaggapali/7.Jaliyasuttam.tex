\section{Jāliyasuttaṃ}

\subsubsection{Dvepabbajitavatthu}

\paragraph{378.} Evaṃ me sutaṃ – ekaṃ samayaṃ bhagavā kosambiyaṃ viharati ghositārāme. Tena kho pana samayena dve pabbajitā – muṇḍiyo ca paribbājako jāliyo ca dārupattikantevāsī yena bhagavā tenupasaṅkamiṃsu; upasaṅkamitvā bhagavatā saddhiṃ sammodiṃsu. Sammodanīyaṃ kathaṃ sāraṇīyaṃ vītisāretvā ekamantaṃ aṭṭhaṃsu. Ekamantaṃ ṭhitā kho te dve pabbajitā bhagavantaṃ etadavocuṃ – ‘‘kiṃ nu kho, āvuso gotama, taṃ jīvaṃ taṃ sarīraṃ, udāhu aññaṃ jīvaṃ aññaṃ sarīra’’nti?

\paragraph{379.} ‘‘Tena hāvuso, suṇātha sādhukaṃ manasi karotha; bhāsissāmī’’ti. ‘‘Evamāvuso’’ti kho te dve pabbajitā bhagavato paccassosuṃ. Bhagavā etadavoca – ‘‘idhāvuso, tathāgato loke uppajjati arahaṃ, sammāsambuddho…pe… (yathā 190-212 anucchedesu evaṃ vitthāretabbaṃ). Evaṃ kho, āvuso, bhikkhu sīlasampanno hoti…pe… paṭhamaṃ jhānaṃ upasampajja viharati. Yo kho, āvuso, bhikkhu evaṃ jānāti evaṃ passati, kallaṃ nu kho tassetaṃ vacanāya – ‘taṃ jīvaṃ taṃ sarīra’nti vā ‘aññaṃ jīvaṃ aññaṃ sarīra’nti vāti. Yo so, āvuso, bhikkhu evaṃ jānāti evaṃ passati, kallaṃ tassetaṃ vacanāya – ‘taṃ jīvaṃ taṃ sarīra’nti vā ‘aññaṃ jīvaṃ aññaṃ sarīra’nti vāti. Ahaṃ kho panetaṃ, āvuso, evaṃ jānāmi evaṃ passāmi. Atha ca panāhaṃ na vadāmi – ‘taṃ jīvaṃ taṃ sarīra’nti vā ‘aññaṃ jīvaṃ aññaṃ sarīra’nti vā…pe… dutiyaṃ jhānaṃ…pe… tatiyaṃ jhānaṃ…pe… catutthaṃ jhānaṃ upasampajja viharati. Yo kho, āvuso, bhikkhu evaṃ jānāti evaṃ passati, kallaṃ nu kho tassetaṃ vacanāya – ‘taṃ jīvaṃ taṃ sarīra’nti vā ‘aññaṃ jīvaṃ aññaṃ sarīra’nti vāti? Yo so, āvuso, bhikkhu evaṃ jānāti evaṃ passati kallaṃ, tassetaṃ vacanāya – ‘taṃ jīvaṃ taṃ sarīra’nti vā ‘aññaṃ jīvaṃ aññaṃ sarīra’nti vāti. Ahaṃ kho panetaṃ, āvuso, evaṃ jānāmi evaṃ passāmi. Atha ca panāhaṃ na vadāmi – ‘taṃ jīvaṃ taṃ sarīra’nti vā ‘aññaṃ jīvaṃ aññaṃ sarīra’nti vā…pe… ñāṇadassanāya cittaṃ abhinīharati abhininnāmeti…pe… yo kho, āvuso, bhikkhu evaṃ jānāti evaṃ passati, kallaṃ nu kho tassetaṃ vacanāya – ‘taṃ jīvaṃ taṃ sarīra’nti vā ‘aññaṃ jīvaṃ aññaṃ sarīra’nti vāti. Yo so, āvuso, bhikkhu evaṃ jānāti evaṃ passati kallaṃ tassetaṃ vacanāya – ‘taṃ jīvaṃ taṃ sarīra’nti vā ‘aññaṃ jīvaṃ aññaṃ sarīra’nti vāti. Ahaṃ kho panetaṃ, āvuso, evaṃ jānāmi evaṃ passāmi. Atha ca panāhaṃ na vadāmi – ‘taṃ jīvaṃ taṃ sarīra’nti vā ‘aññaṃ jīvaṃ aññaṃ sarīra’nti vā…pe….

\paragraph{380.} …Pe… nāparaṃ itthattāyāti pajānāti. Yo kho, āvuso, bhikkhu evaṃ jānāti evaṃ passati, kallaṃ nu kho tassetaṃ vacanāya – ‘taṃ jīvaṃ taṃ sarīra’nti vā ‘aññaṃ jīvaṃ aññaṃ sarīra’nti vāti? Yo so, āvuso, bhikkhu evaṃ jānāti evaṃ passati, na kallaṃ tassetaṃ vacanāya – ‘taṃ jīvaṃ taṃ sarīra’nti vā ‘aññaṃ jīvaṃ aññaṃ sarīra’nti vāti. Ahaṃ kho panetaṃ, āvuso, evaṃ jānāmi evaṃ passāmi. Atha ca panāhaṃ na vadāmi – ‘taṃ jīvaṃ taṃ sarīra’nti vā ‘aññaṃ jīvaṃ aññaṃ sarīra’nti vā’’ti. Idamavoca bhagavā. Attamanā te dve pabbajitā bhagavato bhāsitaṃ abhinandunti.

\xsectionEnd{Jāliyasuttaṃ niṭṭhitaṃ sattamaṃ.}
