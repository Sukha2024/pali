\newcommand{\xchapter}[2]{
  \setcounter{chapter}{#1}
  \setcounter{section}{0}
  \chapter*{#2}
  \chaptermark{#2}
  \addcontentsline{toc}{chapter}{#1. #2}
}

\newcommand{\xchapterEnd}[1]{
  \begin{center}
    \chapter*{#1}
  \end{center}
  \vspace{7mm}
}

\newcommand{\xsectionEnd}[1]{
  \begin{center}
    \section*{#1}
  \end{center}
  \vspace{7mm}
}

\newcommand{\xsubsectionEnd}[1]{
  \begin{center}
    \subsection*{#1}
  \end{center}
  \vspace{7mm}
}

\newcommand{\xsubsubsectionEnd}[1]{
  \begin{center}
    \subsubsection{#1}
  \end{center}
  \vspace{7mm}
}

\newcommand{\xxsubsubsectionEnd}[1]{
  \begin{center}
    \subsubsection[x]{#1}
  \end{center}
  \vspace{7mm}
}

%%%%%%%%%%%%%%%%%%%%%%%%%%%%%%%%%%%%%%%%%%%%%%%%%%%%%%%

\documentclass[b5paper,11pt,twoside,draft]{book}
\usepackage{fontspec}
\usepackage[a5paper]{geometry}
\usepackage{polyglossia}
\setdefaultlanguage{pali}

%\usepackage{fancyhdr}
\usepackage{hyphenat} %hyphenation at - : \hyp{}
%Hyperlinks:
\usepackage{hyperref}
\hypersetup{pdftex,colorlinks=true,allcolors=blue,pdfpagemode=UseOutlines}
\usepackage{hypcap}

\overfullrule=1mm

%\pretolerance=1500
%\setlength{\emergencystretch}{2mm}
%\setlength{\parskip}{0pt}
%%%%%%%%%%%%%%%

\title{Tipiṭaka \\ \vspace{2 mm} \large Suttapiṭaka}
\begin{document}
\maketitle

\thispagestyle{empty}
\cleardoublepage


\vspace*{\fill}
\textit{Namo tassa bhagavato arahato sammāsambuddhassa}

\textit{Namo tassa bhagavato arahato sammāsambuddhassa}

\textit{Namo tassa bhagavato arahato sammāsambuddhassa}
\vspace*{\fill}

\tableofcontents

%Remove text number from Part pages:
\makeatletter
\renewcommand\part{
  \if@openright
  \cleardoublepage
  \else
  \clearpage
  \fi
  \thispagestyle{empty}
  \if@twocolumn
  \onecolumn
  \@tempswatrue
  \else
  \@tempswafalse
  \fi
  \null\vfil
  \secdef\@part\@spart}
\makeatother

\part*{Majjhimanikāyo}

%\pagestyle{myheadings}

%\xchapter{2}{Mahāvaggapāḷi}
\part{Mūlapaṇṇāsapāḷi}

\xchapter{1}{Mūlapariyāyavaggo}

\section{Mūlapariyāyasuttaṃ}

\paragraph{1.} Evaṃ me sutaṃ – ekaṃ samayaṃ bhagavā ukkaṭṭhāyaṃ viharati subhagavane sālarājamūle. Tatra kho bhagavā bhikkhū āmantesi – ‘‘bhikkhavo’’ti. ‘‘Bhadante’’ti te bhikkhū bhagavato paccassosuṃ. Bhagavā etadavoca – ‘‘sabbadhammamūlapariyāyaṃ vo, bhikkhave, desessāmi. Taṃ suṇātha, sādhukaṃ manasi karotha, bhāsissāmī’’ti. ‘‘Evaṃ, bhante’’ti kho te bhikkhū bhagavato paccassosuṃ. Bhagavā etadavoca –

\paragraph{2.} ‘‘Idha, bhikkhave, assutavā puthujjano ariyānaṃ adassāvī ariyadhammassa akovido ariyadhamme avinīto, sappurisānaṃ adassāvī sappurisadhammassa akovido sappurisadhamme avinīto – pathaviṃ\footnote{paṭhaviṃ (sī. syā. kaṃ. pī.)} pathavito sañjānāti; pathaviṃ pathavito saññatvā pathaviṃ maññati, pathaviyā maññati, pathavito maññati, pathaviṃ meti maññati , pathaviṃ abhinandati. Taṃ kissa hetu? ‘Apariññātaṃ tassā’ti vadāmi.

\paragraph{3.} ‘‘Āpaṃ āpato sañjānāti; āpaṃ āpato saññatvā āpaṃ maññati, āpasmiṃ maññati, āpato maññati, āpaṃ meti maññati, āpaṃ abhinandati. Taṃ kissa hetu? ‘Apariññātaṃ tassā’ti vadāmi.

\paragraph{4.} ‘‘Tejaṃ tejato sañjānāti; tejaṃ tejato saññatvā tejaṃ maññati, tejasmiṃ maññati, tejato maññati, tejaṃ meti maññati, tejaṃ abhinandati. Taṃ kissa hetu? ‘Apariññātaṃ tassā’ti vadāmi.

\paragraph{5.} ‘‘Vāyaṃ vāyato sañjānāti; vāyaṃ vāyato saññatvā vāyaṃ maññati, vāyasmiṃ maññati, vāyato maññati, vāyaṃ meti maññati, vāyaṃ abhinandati. Taṃ kissa hetu? ‘Apariññātaṃ tassā’ti vadāmi.

\paragraph{6.} . ‘‘Bhūte bhūtato sañjānāti; bhūte bhūtato saññatvā bhūte maññati, bhūtesu maññati, bhūtato maññati, bhūte meti maññati, bhūte abhinandati. Taṃ kissa hetu? ‘Apariññātaṃ tassā’ti vadāmi.

\paragraph{7.} ‘‘Deve devato sañjānāti; deve devato saññatvā deve maññati, devesu maññati, devato maññati, deve meti maññati, deve abhinandati. Taṃ kissa hetu? ‘Apariññātaṃ tassā’ti vadāmi.

\paragraph{8.} ‘‘Pajāpatiṃ pajāpatito sañjānāti; pajāpatiṃ pajāpatito saññatvā pajāpatiṃ maññati, pajāpatismiṃ maññati, pajāpatito maññati, pajāpatiṃ meti maññati, pajāpatiṃ abhinandati. Taṃ kissa hetu? ‘Apariññātaṃ tassā’ti vadāmi.

\paragraph{9.} ‘‘Brahmaṃ brahmato sañjānāti; brahmaṃ brahmato saññatvā brahmaṃ maññati , brahmasmiṃ maññati, brahmato maññati, brahmaṃ meti maññati, brahmaṃ abhinandati. Taṃ kissa hetu? ‘Apariññātaṃ tassā’ti vadāmi.

\paragraph{10.} ‘‘Ābhassare ābhassarato sañjānāti; ābhassare ābhassarato saññatvā ābhassare maññati, ābhassaresu maññati, ābhassarato maññati, ābhassare meti maññati, ābhassare abhinandati. Taṃ kissa hetu? ‘Apariññātaṃ tassā’ti vadāmi.

\paragraph{11.} ‘‘Subhakiṇhe subhakiṇhato sañjānāti; subhakiṇhe subhakiṇhato saññatvā subhakiṇhe maññati, subhakiṇhesu maññati, subhakiṇhato maññati, subhakiṇhe meti maññati, subhakiṇhe abhinandati. Taṃ kissa hetu? ‘Apariññātaṃ tassā’ti vadāmi.

\paragraph{12.} ‘‘Vehapphale vehapphalato sañjānāti; vehapphale vehapphalato saññatvā vehapphale maññati, vehapphalesu maññati, vehapphalato maññati, vehapphale meti maññati, vehapphale abhinandati. Taṃ kissa hetu? ‘Apariññātaṃ tassā’ti vadāmi.

\paragraph{13.} ‘‘Abhibhuṃ abhibhūto sañjānāti; abhibhuṃ abhibhūto saññatvā abhibhuṃ maññati, abhibhusmiṃ maññati, abhibhūto maññati, abhibhuṃ meti maññati, abhibhuṃ abhinandati. Taṃ kissa hetu? ‘Apariññātaṃ tassā’ti vadāmi.

\paragraph{14.} . ‘‘Ākāsānañcāyatanaṃ ākāsānañcāyatanato sañjānāti; ākāsānañcāyatanaṃ ākāsānañcāyatanato saññatvā ākāsānañcāyatanaṃ maññati, ākāsānañcāyatanasmiṃ maññati, ākāsānañcāyatanato maññati, ākāsānañcāyatanaṃ meti maññati, ākāsānañcāyatanaṃ abhinandati. Taṃ kissa hetu? ‘Apariññātaṃ tassā’ti vadāmi.

\paragraph{15.} ‘‘Viññāṇañcāyatanaṃ viññāṇañcāyatanato sañjānāti; viññāṇañcāyatanaṃ viññāṇañcāyatanato saññatvā viññāṇañcāyatanaṃ maññati, viññāṇañcāyatanasmiṃ maññati, viññāṇañcāyatanato maññati, viññāṇañcāyatanaṃ meti maññati, viññāṇañcāyatanaṃ abhinandati. Taṃ kissa hetu? ‘Apariññātaṃ tassā’ti vadāmi.

\paragraph{16.} ‘‘Ākiñcaññāyatanaṃ ākiñcaññāyatanato sañjānāti; ākiñcaññāyatanaṃ ākiñcaññāyatanato saññatvā ākiñcaññāyatanaṃ maññati, ākiñcaññāyatanasmiṃ maññati, ākiñcaññāyatanato maññati, ākiñcaññāyatanaṃ meti maññati, ākiñcaññāyatanaṃ abhinandati. Taṃ kissa hetu? ‘Apariññātaṃ tassā’ti vadāmi.

\paragraph{17.} ‘‘Nevasaññānāsaññāyatanaṃ nevasaññānāsaññāyatanato sañjānāti; nevasaññānāsaññāyatanaṃ nevasaññānāsaññāyatanato saññatvā nevasaññānāsaññāyatanaṃ maññati, nevasaññānāsaññāyatanasmiṃ maññati, nevasaññānāsaññāyatanato maññati, nevasaññānāsaññāyatanaṃ meti maññati, nevasaññānāsaññāyatanaṃ abhinandati. Taṃ kissa hetu? ‘Apariññātaṃ tassā’ti vadāmi.

\paragraph{18.} . ‘‘Diṭṭhaṃ diṭṭhato sañjānāti; diṭṭhaṃ diṭṭhato saññatvā diṭṭhaṃ maññati, diṭṭhasmiṃ maññati, diṭṭhato maññati, diṭṭhaṃ meti maññati, diṭṭhaṃ abhinandati. Taṃ kissa hetu? ‘Apariññātaṃ tassā’ti vadāmi.

\paragraph{19.} ‘‘Sutaṃ sutato sañjānāti; sutaṃ sutato saññatvā sutaṃ maññati, sutasmiṃ maññati, sutato maññati, sutaṃ meti maññati, sutaṃ abhinandati. Taṃ kissa hetu? ‘Apariññātaṃ tassā’ti vadāmi.

\paragraph{20.} ‘‘Mutaṃ mutato sañjānāti; mutaṃ mutato saññatvā mutaṃ maññati, mutasmiṃ maññati, mutato maññati, mutaṃ meti maññati, mutaṃ abhinandati. Taṃ kissa hetu? ‘Apariññātaṃ tassā’ti vadāmi.

\paragraph{21.} ‘‘Viññātaṃ viññātato sañjānāti; viññātaṃ viññātato saññatvā viññātaṃ maññati, viññātasmiṃ maññati, viññātato maññati, viññātaṃ meti maññati, viññātaṃ abhinandati. Taṃ kissa hetu? ‘Apariññātaṃ tassā’ti vadāmi.

\paragraph{22.} . ‘‘Ekattaṃ ekattato sañjānāti; ekattaṃ ekattato saññatvā ekattaṃ maññati, ekattasmiṃ maññati, ekattato maññati, ekattaṃ meti maññati, ekattaṃ abhinandati. Taṃ kissa hetu? ‘Apariññātaṃ tassā’ti vadāmi.

\paragraph{23.} ‘‘Nānattaṃ nānattato sañjānāti; nānattaṃ nānattato saññatvā nānattaṃ maññati, nānattasmiṃ maññati, nānattato maññati, nānattaṃ meti maññati, nānattaṃ abhinandati. Taṃ kissa hetu? ‘Apariññātaṃ tassā’ti vadāmi.

\paragraph{24.} ‘‘Sabbaṃ sabbato sañjānāti; sabbaṃ sabbato saññatvā sabbaṃ maññati, sabbasmiṃ maññati, sabbato maññati, sabbaṃ meti maññati, sabbaṃ abhinandati. Taṃ kissa hetu? ‘Apariññātaṃ tassā’ti vadāmi.

\paragraph{25.} ‘‘Nibbānaṃ nibbānato sañjānāti; nibbānaṃ nibbānato saññatvā nibbānaṃ maññati, nibbānasmiṃ maññati , nibbānato maññati, nibbānaṃ meti maññati, nibbānaṃ abhinandati. Taṃ kissa hetu? ‘Apariññātaṃ tassā’ti vadāmi.

\xsubsubsectionEnd{Puthujjanavasena paṭhamanayabhūmiparicchedo niṭṭhito.}

\paragraph{26.} . ‘‘Yopi so, bhikkhave, bhikkhu sekkho\footnote{sekho (sī. syā. kaṃ. pī.)} appattamānaso anuttaraṃ yogakkhemaṃ patthayamāno viharati, sopi pathaviṃ pathavito abhijānāti; pathaviṃ pathavito abhiññāya\footnote{abhiññatvā (ka.)} pathaviṃ mā maññi\footnote{vā maññati}, pathaviyā mā maññi, pathavito mā maññi, pathaviṃ meti mā maññi, pathaviṃ mābhinandi\footnote{vā abhinandati (sī.) ṭīkā oloketabbā}. Taṃ kissa hetu? ‘Pariññeyyaṃ tassā’ti vadāmi.

\paragraph{27.} ‘‘Āpaṃ…pe… tejaṃ… vāyaṃ… bhūte… deve… pajāpatiṃ… brahmaṃ… ābhassare… subhakiṇhe… vehapphale… abhibhuṃ… ākāsānañcāyatanaṃ… viññāṇañcāyatanaṃ… ākiñcaññāyatanaṃ… nevasaññānāsaññāyatanaṃ… diṭṭhaṃ… sutaṃ… mutaṃ… viññātaṃ… ekattaṃ… nānattaṃ… sabbaṃ… nibbānaṃ nibbānato abhijānāti; nibbānaṃ nibbānato abhiññāya nibbānaṃ mā maññi, nibbānasmiṃ mā maññi, nibbānato mā maññi, nibbānaṃ meti mā maññi, nibbānaṃ mābhinandi. Taṃ kissa hetu? ‘Pariññeyyaṃ tassā’ti vadāmi.

\xsubsubsectionEnd{Sekkhavasena\protect\footnote{satthāravasena (sī.), satthuvasena (syā. ka.)} dutiyanayabhūmiparicchedo niṭṭhito.}

\paragraph{28.} . ‘‘Yopi so, bhikkhave, bhikkhu arahaṃ khīṇāsavo vusitavā katakaraṇīyo ohitabhāro anuppattasadattho parikkhīṇabhavasaṃyojano sammadaññā vimutto, sopi pathaviṃ pathavito abhijānāti; pathaviṃ pathavito abhiññāya pathaviṃ na maññati, pathaviyā na maññati, pathavito na maññati, pathaviṃ meti na maññati, pathaviṃ nābhinandati. Taṃ kissa hetu? ‘Pariññātaṃ tassā’ti vadāmi.

\paragraph{29.} ‘‘Āpaṃ…pe… tejaṃ… vāyaṃ… bhūte… deve… pajāpatiṃ… brahmaṃ… ābhassare… subhakiṇhe… vehapphale… abhibhuṃ… ākāsānañcāyatanaṃ… viññāṇañcāyatanaṃ… ākiñcaññāyatanaṃ… nevasaññānāsaññāyatanaṃ… diṭṭhaṃ… sutaṃ… mutaṃ… viññātaṃ… ekattaṃ… nānattaṃ… sabbaṃ… nibbānaṃ nibbānato abhijānāti; nibbānaṃ nibbānato abhiññāya nibbānaṃ na maññati, nibbānasmiṃ na maññati, nibbānato na maññati, nibbānaṃ meti na maññati, nibbānaṃ nābhinandati. Taṃ kissa hetu? ‘Pariññātaṃ tassā’ti vadāmi.

\xsubsubsectionEnd{Khīṇāsavavasena tatiyanayabhūmiparicchedo niṭṭhito.}

\paragraph{30.} . ‘‘Yopi so, bhikkhave, bhikkhu arahaṃ khīṇāsavo vusitavā katakaraṇīyo ohitabhāro anuppattasadattho parikkhīṇabhavasaṃyojano sammadaññā vimutto, sopi pathaviṃ pathavito abhijānāti; pathaviṃ pathavito abhiññāya pathaviṃ na maññati, pathaviyā na maññati, pathavito na maññati, pathaviṃ meti na maññati, pathaviṃ nābhinandati. Taṃ kissa hetu? Khayā rāgassa, vītarāgattā.

\paragraph{31.} ‘‘Āpaṃ…pe… tejaṃ… vāyaṃ… bhūte… deve… pajāpatiṃ… brahmaṃ… ābhassare… subhakiṇhe… vehapphale… abhibhuṃ… ākāsānañcāyatanaṃ… viññāṇañcāyatanaṃ… ākiñcaññāyatanaṃ … nevasaññānāsaññāyatanaṃ … diṭṭhaṃ… sutaṃ… mutaṃ… viññātaṃ… ekattaṃ… nānattaṃ… sabbaṃ… nibbānaṃ nibbānato abhijānāti; nibbānaṃ nibbānato abhiññāya nibbānaṃ na maññati, nibbānasmiṃ na maññati, nibbānato na maññati, nibbānaṃ meti na maññati, nibbānaṃ nābhinandati. Taṃ kissa hetu? Khayā rāgassa, vītarāgattā.

\xsubsubsectionEnd{Khīṇāsavavasena catutthanayabhūmiparicchedo niṭṭhito.}

\paragraph{32.} ‘‘Yopi so, bhikkhave, bhikkhu arahaṃ khīṇāsavo vusitavā katakaraṇīyo ohitabhāro anuppattasadattho parikkhīṇabhavasaṃyojano sammadaññā vimutto, sopi pathaviṃ pathavito abhijānāti; pathaviṃ pathavito abhiññāya pathaviṃ na maññati, pathaviyā na maññati, pathavito na maññati, pathaviṃ meti na maññati, pathaviṃ nābhinandati. Taṃ kissa hetu? Khayā dosassa, vītadosattā.

\paragraph{33.} ‘‘Āpaṃ…pe… tejaṃ… vāyaṃ… bhūte… deve… pajāpatiṃ… brahmaṃ… ābhassare… subhakiṇhe… vehapphale… abhibhuṃ… ākāsānañcāyatanaṃ… viññāṇañcāyatanaṃ… ākiñcaññāyatanaṃ… nevasaññānāsaññāyatanaṃ… diṭṭhaṃ… sutaṃ… mutaṃ… viññātaṃ… ekattaṃ… nānattaṃ… sabbaṃ… nibbānaṃ nibbānato abhijānāti; nibbānaṃ nibbānato abhiññāya nibbānaṃ na maññati, nibbānasmiṃ na maññati, nibbānato na maññati, nibbānaṃ meti na maññati, nibbānaṃ nābhinandati. Taṃ kissa hetu? Khayā dosassa, vītadosattā.

\xsubsubsectionEnd{Khīṇāsavavasena pañcamanayabhūmiparicchedo niṭṭhito.}

\paragraph{34.} ‘‘Yopi so, bhikkhave, bhikkhu arahaṃ khīṇāsavo vusitavā katakaraṇīyo ohitabhāro anuppattasadattho parikkhīṇabhavasaṃyojano sammadaññā vimutto, sopi pathaviṃ pathavito abhijānāti; pathaviṃ pathavito abhiññāya pathaviṃ na maññati, pathaviyā na maññati, pathavito na maññati, pathaviṃ meti na maññati, pathaviṃ nābhinandati. Taṃ kissa hetu? Khayā mohassa, vītamohattā.

\paragraph{35.} ‘‘Āpaṃ…pe… tejaṃ… vāyaṃ… bhūte… deve… pajāpatiṃ… brahmaṃ… ābhassare… subhakiṇhe… vehapphale… abhibhuṃ… ākāsānañcāyatanaṃ… viññāṇañcāyatanaṃ… ākiñcaññāyatanaṃ … nevasaññānāsaññāyatanaṃ… diṭṭhaṃ… sutaṃ… mutaṃ… viññātaṃ… ekattaṃ… nānattaṃ… sabbaṃ… nibbānaṃ nibbānato abhijānāti; nibbānaṃ nibbānato abhiññāya nibbānaṃ na maññati, nibbānasmiṃ na maññati, nibbānato na maññati, nibbānaṃ meti na maññati, nibbānaṃ nābhinandati. Taṃ kissa hetu? Khayā mohassa, vītamohattā.

\xsubsubsectionEnd{Khīṇāsavavasena chaṭṭhanayabhūmiparicchedo niṭṭhito.}

\paragraph{36.} ‘‘Tathāgatopi, bhikkhave, arahaṃ sammāsambuddho pathaviṃ pathavito abhijānāti; pathaviṃ pathavito abhiññāya pathaviṃ na maññati, pathaviyā na maññati, pathavito na maññati, pathaviṃ meti na maññati, pathaviṃ nābhinandati . Taṃ kissa hetu? ‘Pariññātantaṃ tathāgatassā’ti vadāmi.

\paragraph{37.} ‘‘Āpaṃ…pe… tejaṃ… vāyaṃ… bhūte… deve… pajāpatiṃ… brahmaṃ… ābhassare… subhakiṇhe… vehapphale… abhibhuṃ… ākāsānañcāyatanaṃ… viññāṇañcāyatanaṃ … ākiñcaññāyatanaṃ… nevasaññānāsaññāyatanaṃ… diṭṭhaṃ… sutaṃ… mutaṃ… viññātaṃ… ekattaṃ… nānattaṃ… sabbaṃ… nibbānaṃ nibbānato abhijānāti; nibbānaṃ nibbānato abhiññāya nibbānaṃ na maññati, nibbānasmiṃ na maññati, nibbānato na maññati, nibbānaṃ meti na maññati, nibbānaṃ nābhinandati. Taṃ kissa hetu? ‘Pariññātantaṃ tathāgatassā’ti vadāmi.

\xsubsubsectionEnd{Tathāgatavasena sattamanayabhūmiparicchedo niṭṭhito.}

\paragraph{38.} ‘‘Tathāgatopi , bhikkhave, arahaṃ sammāsambuddho pathaviṃ pathavito abhijānāti; pathaviṃ pathavito abhiññāya pathaviṃ na maññati, pathaviyā na maññati, pathavito na maññati, pathaviṃ meti na maññati, pathaviṃ nābhinandati. Taṃ kissa hetu? ‘Nandī\footnote{nandi (sī. syā.)} dukkhassa mūla’nti – iti viditvā ‘bhavā jāti bhūtassa jarāmaraṇa’nti. Tasmātiha, bhikkhave, ‘tathāgato sabbaso taṇhānaṃ khayā virāgā nirodhā cāgā paṭinissaggā anuttaraṃ sammāsambodhiṃ abhisambuddho’ti vadāmi.

\paragraph{39.} ‘‘Āpaṃ …pe… tejaṃ… vāyaṃ… bhūte… deve… pajāpatiṃ… brahmaṃ… ābhassare… subhakiṇhe… vehapphale… abhibhuṃ… ākāsānañcāyatanaṃ… viññāṇañcāyatanaṃ… ākiñcaññāyatanaṃ… nevasaññānāsaññāyatanaṃ… diṭṭhaṃ… sutaṃ… mutaṃ… viññātaṃ… ekattaṃ… nānattaṃ… sabbaṃ… nibbānaṃ nibbānato abhijānāti; nibbānaṃ nibbānato abhiññāya nibbānaṃ na maññati, nibbānasmiṃ na maññati, nibbānato na maññati, nibbānaṃ meti na maññati, nibbānaṃ nābhinandati. Taṃ kissa hetu? ‘Nandī dukkhassa mūla’nti – iti viditvā ‘bhavā jāti bhūtassa jarāmaraṇa’nti. Tasmātiha, bhikkhave, ‘tathāgato sabbaso taṇhānaṃ khayā virāgā nirodhā cāgā paṭinissaggā anuttaraṃ sammāsambodhiṃ abhisambuddho’ti vadāmī’’ti.

\xsubsubsectionEnd{Tathāgatavasena aṭṭhamanayabhūmiparicchedo niṭṭhito.}

\paragraph{40.} Idamavoca bhagavā. Na te bhikkhū\footnote{na attamanā tebhikkhū (syā.), te bhikkhū (pī. ka.)} bhagavato bhāsitaṃ abhinandunti.

\xsectionEnd{Mūlapariyāyasuttaṃ niṭṭhitaṃ paṭhamaṃ.}



%\clearpage
%\section{Mahānidānasuttaṃ}

\subsubsection{Paṭiccasamuppādo}

\paragraph{95.} Evaṃ me sutaṃ – ekaṃ samayaṃ bhagavā kurūsu viharati kammāsadhammaṃ nāma\footnote{kammāsadammaṃ nāma (syā.)} kurūnaṃ nigamo. Atha kho āyasmā ānando yena bhagavā tenupasaṅkami, upasaṅkamitvā bhagavantaṃ abhivādetvā ekamantaṃ nisīdi. Ekamantaṃ nisinno kho āyasmā ānando bhagavantaṃ etadavoca – ‘‘acchariyaṃ, bhante, abbhutaṃ, bhante! Yāva gambhīro cāyaṃ, bhante, paṭiccasamuppādo gambhīrāvabhāso ca, atha ca pana me uttānakuttānako viya khāyatī’’ti. ‘‘Mā hevaṃ, ānanda, avaca, mā hevaṃ, ānanda, avaca. Gambhīro cāyaṃ, ānanda, paṭiccasamuppādo gambhīrāvabhāso ca. Etassa, ānanda, dhammassa ananubodhā appaṭivedhā evamayaṃ pajā tantākulakajātā kulagaṇṭhikajātā\footnote{gulāguṇṭhikajātā (sī. pī.), guṇagaṇṭhikajātā (syā.)} muñjapabbajabhūtā apāyaṃ duggatiṃ vinipātaṃ saṃsāraṃ nātivattati.

\paragraph{96.} ‘‘‘Atthi idappaccayā jarāmaraṇa’nti iti puṭṭhena satā, ānanda, atthītissa vacanīyaṃ. ‘Kiṃpaccayā jarāmaraṇa’nti iti ce vadeyya, ‘jātipaccayā jarāmaraṇa’nti iccassa vacanīyaṃ.

‘‘‘Atthi idappaccayā jātī’ti iti puṭṭhena satā, ānanda, atthītissa vacanīyaṃ. ‘Kiṃpaccayā jātī’ti iti ce vadeyya, ‘bhavapaccayā jātī’ti iccassa vacanīyaṃ.

‘‘‘Atthi idappaccayā bhavo’ti iti puṭṭhena satā, ānanda, atthītissa vacanīyaṃ . ‘Kiṃpaccayā bhavo’ti iti ce vadeyya, ‘upādānapaccayā bhavo’ti iccassa vacanīyaṃ.

‘‘‘Atthi idappaccayā upādāna’nti iti puṭṭhena satā, ānanda, atthītissa vacanīyaṃ. ‘Kiṃpaccayā upādāna’nti iti ce vadeyya, ‘taṇhāpaccayā upādāna’nti iccassa vacanīyaṃ.

‘‘‘Atthi idappaccayā taṇhā’ti iti puṭṭhena satā, ānanda, atthītissa vacanīyaṃ. ‘Kiṃpaccayā taṇhā’ti iti ce vadeyya, ‘vedanāpaccayā taṇhā’ti iccassa vacanīyaṃ.

‘‘‘Atthi idappaccayā vedanā’ti iti puṭṭhena satā, ānanda, atthītissa vacanīyaṃ. ‘Kiṃpaccayā vedanā’ti iti ce vadeyya, ‘phassapaccayā vedanā’ti iccassa vacanīyaṃ.

‘‘‘Atthi idappaccayā phasso’ti iti puṭṭhena satā, ānanda, atthītissa vacanīyaṃ. ‘Kiṃpaccayā phasso’ti iti ce vadeyya, ‘nāmarūpapaccayā phasso’ti iccassa vacanīyaṃ.

‘‘‘Atthi idappaccayā nāmarūpa’nti iti puṭṭhena satā, ānanda, atthītissa vacanīyaṃ. ‘Kiṃpaccayā nāmarūpa’nti iti ce vadeyya, ‘viññāṇapaccayā nāmarūpa’nti iccassa vacanīyaṃ.

‘‘‘Atthi idappaccayā viññāṇa’nti iti puṭṭhena satā, ānanda, atthītissa vacanīyaṃ. ‘Kiṃpaccayā viññāṇa’nti iti ce vadeyya, ‘nāmarūpapaccayā viññāṇa’nti iccassa vacanīyaṃ.

\paragraph{97.} ‘‘Iti kho, ānanda, nāmarūpapaccayā viññāṇaṃ, viññāṇapaccayā nāmarūpaṃ, nāmarūpapaccayā phasso, phassapaccayā vedanā, vedanāpaccayā taṇhā, taṇhāpaccayā upādānaṃ, upādānapaccayā bhavo, bhavapaccayā jāti , jātipaccayā jarāmaraṇaṃ sokaparidevadukkhadomanassupāyāsā sambhavanti. Evametassa kevalassa dukkhakkhandhassa samudayo hoti.

\paragraph{98.} ‘‘‘Jātipaccayā jarāmaraṇa’nti iti kho panetaṃ vuttaṃ, tadānanda, imināpetaṃ pariyāyena veditabbaṃ, yathā jātipaccayā jarāmaraṇaṃ. Jāti ca hi, ānanda, nābhavissa sabbena sabbaṃ sabbathā sabbaṃ kassaci kimhici, seyyathidaṃ – devānaṃ vā devattāya, gandhabbānaṃ vā gandhabbattāya, yakkhānaṃ vā yakkhattāya, bhūtānaṃ vā bhūtattāya, manussānaṃ vā manussattāya, catuppadānaṃ vā catuppadattāya, pakkhīnaṃ vā pakkhittāya, sarīsapānaṃ vā sarīsapattāya\footnote{siriṃsapānaṃ siriṃsapattāya (sī. syā.)}, tesaṃ tesañca hi, ānanda, sattānaṃ tadattāya jāti nābhavissa. Sabbaso jātiyā asati jātinirodhā api nu kho jarāmaraṇaṃ paññāyethā’’ti? ‘‘No hetaṃ, bhante’’. ‘‘Tasmātihānanda, eseva hetu etaṃ nidānaṃ esa samudayo esa paccayo jarāmaraṇassa, yadidaṃ jāti’’.

\paragraph{99.} ‘‘‘Bhavapaccayā jātī’ti iti kho panetaṃ vuttaṃ, tadānanda, imināpetaṃ pariyāyena veditabbaṃ, yathā bhavapaccayā jāti. Bhavo ca hi, ānanda, nābhavissa sabbena sabbaṃ sabbathā sabbaṃ kassaci kimhici, seyyathidaṃ – kāmabhavo vā rūpabhavo vā arūpabhavo vā, sabbaso bhave asati bhavanirodhā api nu kho jāti paññāyethā’’ti? ‘‘No hetaṃ, bhante’’. ‘‘Tasmātihānanda, eseva hetu etaṃ nidānaṃ esa samudayo esa paccayo jātiyā, yadidaṃ bhavo’’.

\paragraph{100.} ‘‘‘Upādānapaccayā bhavo’ti iti kho panetaṃ vuttaṃ, tadānanda, imināpetaṃ pariyāyena veditabbaṃ, yathā upādānapaccayā bhavo. Upādānañca hi, ānanda, nābhavissa sabbena sabbaṃ sabbathā sabbaṃ kassaci kimhici , seyyathidaṃ – kāmupādānaṃ vā diṭṭhupādānaṃ vā sīlabbatupādānaṃ vā attavādupādānaṃ vā, sabbaso upādāne asati upādānanirodhā api nu kho bhavo paññāyethā’’ti? ‘‘No hetaṃ, bhante’’. ‘‘Tasmātihānanda, eseva hetu etaṃ nidānaṃ esa samudayo esa paccayo bhavassa, yadidaṃ upādānaṃ’’.

\paragraph{101.} ‘‘‘Taṇhāpaccayā upādāna’nti iti kho panetaṃ vuttaṃ tadānanda, imināpetaṃ pariyāyena veditabbaṃ, yathā taṇhāpaccayā upādānaṃ. Taṇhā ca hi, ānanda, nābhavissa sabbena sabbaṃ sabbathā sabbaṃ kassaci kimhici, seyyathidaṃ – rūpataṇhā saddataṇhā gandhataṇhā rasataṇhā phoṭṭhabbataṇhā dhammataṇhā, sabbaso taṇhāya asati taṇhānirodhā api nu kho upādānaṃ paññāyethā’’ti? ‘‘No hetaṃ, bhante’’. ‘‘Tasmātihānanda, eseva hetu etaṃ nidānaṃ esa samudayo esa paccayo upādānassa, yadidaṃ taṇhā’’.

\paragraph{102.} ‘‘‘Vedanāpaccayā taṇhā’ti iti kho panetaṃ vuttaṃ, tadānanda, imināpetaṃ pariyāyena veditabbaṃ, yathā vedanāpaccayā taṇhā. Vedanā ca hi, ānanda, nābhavissa sabbena sabbaṃ sabbathā sabbaṃ kassaci kimhici, seyyathidaṃ – cakkhusamphassajā vedanā sotasamphassajā vedanā ghānasamphassajā vedanā jivhāsamphassajā vedanā kāyasamphassajā vedanā manosamphassajā vedanā, sabbaso vedanāya asati vedanānirodhā api nu kho taṇhā paññāyethā’’ti ? ‘‘No hetaṃ, bhante’’. ‘‘Tasmātihānanda, eseva hetu etaṃ nidānaṃ esa samudayo esa paccayo taṇhāya, yadidaṃ vedanā’’.

\paragraph{103.} ‘‘Iti kho panetaṃ, ānanda, vedanaṃ paṭicca taṇhā, taṇhaṃ paṭicca pariyesanā, pariyesanaṃ paṭicca lābho, lābhaṃ paṭicca vinicchayo, vinicchayaṃ paṭicca chandarāgo, chandarāgaṃ paṭicca ajjhosānaṃ, ajjhosānaṃ paṭicca pariggaho, pariggahaṃ paṭicca macchariyaṃ, macchariyaṃ paṭicca ārakkho. Ārakkhādhikaraṇaṃ daṇḍādānasatthādānakalahaviggahavivādatuvaṃtuvaṃpesuññamusāvādā aneke pāpakā akusalā dhammā sambhavanti.

\paragraph{104.} ‘‘‘Ārakkhādhikaraṇaṃ\footnote{ārakkhaṃ paṭicca ārakkhādhikaraṇaṃ (syā.)} daṇḍādānasatthādānakalahaviggahavivādatuvaṃtuvaṃpesuññamusāvādā aneke pāpakā akusalā dhammā sambhavantī’ti iti kho panetaṃ vuttaṃ, tadānanda, imināpetaṃ pariyāyena veditabbaṃ, yathā ārakkhādhikaraṇaṃ daṇḍādānasatthādānakalahaviggahavivādatuvaṃtuvaṃpesuññamusāvādā aneke pāpakā akusalā dhammā sambhavanti. Ārakkho ca hi, ānanda, nābhavissa sabbena sabbaṃ sabbathā sabbaṃ kassaci kimhici, sabbaso ārakkhe asati ārakkhanirodhā api nu kho daṇḍādānasatthādānakalahaviggahavivādatuvaṃtuvaṃpesuññamusāvādā aneke pāpakā akusalā dhammā sambhaveyyu’’nti? ‘‘No hetaṃ, bhante’’. ‘‘Tasmātihānanda, eseva hetu etaṃ nidānaṃ esa samudayo esa paccayo daṇḍādānasatthādānakalahaviggahavivādatuvaṃtuvaṃpesuññamusāvādānaṃ anekesaṃ pāpakānaṃ akusalānaṃ dhammānaṃ sambhavāya yadidaṃ ārakkho.

\paragraph{105.} ‘‘‘Macchariyaṃ paṭicca ārakkho’ti iti kho panetaṃ vuttaṃ, tadānanda, imināpetaṃ pariyāyena veditabbaṃ, yathā macchariyaṃ paṭicca ārakkho. Macchariyañca hi, ānanda, nābhavissa sabbena sabbaṃ sabbathā sabbaṃ kassaci kimhici , sabbaso macchariye asati macchariyanirodhā api nu kho ārakkho paññāyethā’’ti? ‘‘No hetaṃ, bhante’’. ‘‘Tasmātihānanda, eseva hetu etaṃ nidānaṃ esa samudayo esa paccayo ārakkhassa, yadidaṃ macchariyaṃ’’.

\paragraph{106.} ‘‘‘Pariggahaṃ paṭicca macchariya’nti iti kho panetaṃ vuttaṃ, tadānanda, imināpetaṃ pariyāyena veditabbaṃ, yathā pariggahaṃ paṭicca macchariyaṃ. Pariggaho ca hi, ānanda, nābhavissa sabbena sabbaṃ sabbathā sabbaṃ kassaci kimhici, sabbaso pariggahe asati pariggahanirodhā api nu kho macchariyaṃ paññāyethā’’ti? ‘‘No hetaṃ, bhante’’. ‘‘Tasmātihānanda, eseva hetu etaṃ nidānaṃ esa samudayo esa paccayo macchariyassa, yadidaṃ pariggaho’’.

\paragraph{107.} ‘‘‘Ajjhosānaṃ paṭicca pariggaho’ti iti kho panetaṃ vuttaṃ, tadānanda, imināpetaṃ pariyāyena veditabbaṃ, yathā ajjhosānaṃ paṭicca pariggaho. Ajjhosānañca hi, ānanda, nābhavissa sabbena sabbaṃ sabbathā sabbaṃ kassaci kimhici, sabbaso ajjhosāne asati ajjhosānanirodhā api nu kho pariggaho paññāyethā’’ti ? ‘‘No hetaṃ, bhante’’. ‘‘Tasmātihānanda, eseva hetu etaṃ nidānaṃ esa samudayo esa paccayo pariggahassa – yadidaṃ ajjhosānaṃ’’.

\paragraph{108.} ‘‘‘Chandarāgaṃ paṭicca ajjhosāna’nti iti kho panetaṃ vuttaṃ, tadānanda, imināpetaṃ pariyāyena veditabbaṃ, yathā chandarāgaṃ paṭicca ajjhosānaṃ. Chandarāgo ca hi, ānanda, nābhavissa sabbena sabbaṃ sabbathā sabbaṃ kassaci kimhici, sabbaso chandarāge asati chandarāganirodhā api nu kho ajjhosānaṃ paññāyethā’’ti? ‘‘No hetaṃ, bhante’’. ‘‘Tasmātihānanda, eseva hetu etaṃ nidānaṃ esa samudayo esa paccayo ajjhosānassa, yadidaṃ chandarāgo’’.

\paragraph{109.} ‘‘‘Vinicchayaṃ paṭicca chandarāgo’ti iti kho panetaṃ vuttaṃ, tadānanda, imināpetaṃ pariyāyena veditabbaṃ, yathā vinicchayaṃ paṭicca chandarāgo. Vinicchayo ca hi, ānanda, nābhavissa sabbena sabbaṃ sabbathā sabbaṃ kassaci kimhici, sabbaso vinicchaye asati vinicchayanirodhā api nu kho chandarāgo paññāyethā’’ti? ‘‘No hetaṃ , bhante’’. ‘‘Tasmātihānanda, eseva hetu etaṃ nidānaṃ esa samudayo esa paccayo chandarāgassa, yadidaṃ vinicchayo’’.

\paragraph{110.} ‘‘‘Lābhaṃ paṭicca vinicchayo’ti iti kho panetaṃ vuttaṃ, tadānanda, imināpetaṃ pariyāyena veditabbaṃ, yathā lābhaṃ paṭicca vinicchayo. Lābho ca hi, ānanda, nābhavissa sabbena sabbaṃ sabbathā sabbaṃ kassaci kimhici, sabbaso lābhe asati lābhanirodhā api nu kho vinicchayo paññāyethā’’ti? ‘‘No hetaṃ, bhante’’. ‘‘Tasmātihānanda eseva hetu etaṃ nidānaṃ esa samudayo esa paccayo vinicchayassa, yadidaṃ lābho’’.

\paragraph{111.} ‘‘‘Pariyesanaṃ paṭicca lābho’ti iti kho panetaṃ vuttaṃ, tadānanda, imināpetaṃ pariyāyena veditabbaṃ, yathā pariyesanaṃ paṭicca lābho. Pariyesanā ca hi, ānanda, nābhavissa sabbena sabbaṃ sabbathā sabbaṃ kassaci kimhici, sabbaso pariyesanāya asati pariyesanānirodhā api nu kho lābho paññāyethā’’ti? ‘‘No hetaṃ, bhante’’. ‘‘Tasmātihānanda, eseva hetu etaṃ nidānaṃ esa samudayo esa paccayo lābhassa, yadidaṃ pariyesanā’’.

\paragraph{112.} ‘‘‘Taṇhaṃ paṭicca pariyesanā’ti iti kho panetaṃ vuttaṃ, tadānanda, imināpetaṃ pariyāyena veditabbaṃ, yathā taṇhaṃ paṭicca pariyesanā. Taṇhā ca hi, ānanda, nābhavissa sabbena sabbaṃ sabbathā sabbaṃ kassaci kimhici, seyyathidaṃ – kāmataṇhā bhavataṇhā vibhavataṇhā, sabbaso taṇhāya asati taṇhānirodhā api nu kho pariyesanā paññāyethā’’ti? ‘‘No hetaṃ, bhante’’. ‘‘Tasmātihānanda, eseva hetu etaṃ nidānaṃ esa samudayo esa paccayo pariyesanāya, yadidaṃ taṇhā. Iti kho, ānanda, ime dve dhammā\footnote{ime dhammā (ka.)} dvayena vedanāya ekasamosaraṇā bhavanti’’.

\paragraph{113.} ‘‘‘Phassapaccayā vedanā’ti iti kho panetaṃ vuttaṃ, tadānanda, imināpetaṃ pariyāyena veditabbaṃ, yathā ‘phassapaccayā vedanā. Phasso ca hi, ānanda, nābhavissa sabbena sabbaṃ sabbathā sabbaṃ kassaci kimhici, seyyathidaṃ – cakkhusamphasso sotasamphasso ghānasamphasso jivhāsamphasso kāyasamphasso manosamphasso, sabbaso phasse asati phassanirodhā api nu kho vedanā paññāyethā’’ti? ‘‘No hetaṃ, bhante’’. ‘‘Tasmātihānanda , eseva hetu etaṃ nidānaṃ esa samudayo esa paccayo vedanāya, yadidaṃ phasso’’.

\paragraph{114.} ‘‘‘Nāmarūpapaccayā phasso’ti iti kho panetaṃ vuttaṃ, tadānanda, imināpetaṃ pariyāyena veditabbaṃ, yathā nāmarūpapaccayā phasso. Yehi, ānanda, ākārehi yehi liṅgehi yehi nimittehi yehi uddesehi nāmakāyassa paññatti hoti, tesu ākāresu tesu liṅgesu tesu nimittesu tesu uddesesu asati api nu kho rūpakāye adhivacanasamphasso paññāyethā’’ti? ‘‘No hetaṃ, bhante’’. ‘‘Yehi, ānanda, ākārehi yehi liṅgehi yehi nimittehi yehi uddesehi rūpakāyassa paññatti hoti, tesu ākāresu…pe… tesu uddesesu asati api nu kho nāmakāye paṭighasamphasso paññāyethā’’ti? ‘‘No hetaṃ, bhante’’. ‘‘Yehi, ānanda, ākārehi…pe… yehi uddesehi nāmakāyassa ca rūpakāyassa ca paññatti hoti , tesu ākāresu…pe… tesu uddesesu asati api nu kho adhivacanasamphasso vā paṭighasamphasso vā paññāyethā’’ti? ‘‘No hetaṃ, bhante’’. ‘‘Yehi, ānanda, ākārehi…pe… yehi uddesehi nāmarūpassa paññatti hoti, tesu ākāresu …pe… tesu uddesesu asati api nu kho phasso paññāyethā’’ti? ‘‘No hetaṃ, bhante’’. ‘‘Tasmātihānanda, eseva hetu etaṃ nidānaṃ esa samudayo esa paccayo phassassa, yadidaṃ nāmarūpaṃ’’.

\paragraph{115.} ‘‘‘Viññāṇapaccayā nāmarūpa’nti iti kho panetaṃ vuttaṃ, tadānanda, imināpetaṃ pariyāyena veditabbaṃ, yathā viññāṇapaccayā nāmarūpaṃ. Viññāṇañca hi, ānanda, mātukucchismiṃ na okkamissatha, api nu kho nāmarūpaṃ mātukucchismiṃ samuccissathā’’ti? ‘‘No hetaṃ, bhante’’. ‘‘Viññāṇañca hi, ānanda, mātukucchismiṃ okkamitvā vokkamissatha, api nu kho nāmarūpaṃ itthattāya abhinibbattissathā’’ti? ‘‘No hetaṃ, bhante’’. ‘‘Viññāṇañca hi, ānanda, daharasseva sato vocchijjissatha kumārakassa vā kumārikāya vā, api nu kho nāmarūpaṃ vuddhiṃ virūḷhiṃ vepullaṃ āpajjissathā’’ti? ‘‘No hetaṃ, bhante’’. ‘‘Tasmātihānanda, eseva hetu etaṃ nidānaṃ esa samudayo esa paccayo nāmarūpassa – yadidaṃ viññāṇaṃ’’.

\paragraph{116.} ‘‘‘Nāmarūpapaccayā viññāṇa’nti iti kho panetaṃ vuttaṃ, tadānanda, imināpetaṃ pariyāyena veditabbaṃ, yathā nāmarūpapaccayā viññāṇaṃ. Viññāṇañca hi, ānanda, nāmarūpe patiṭṭhaṃ na labhissatha, api nu kho āyatiṃ jātijarāmaraṇaṃ dukkhasamudayasambhavo\footnote{jātijarāmaraṇadukkhasamudayasambhavo (sī. syā. pī.)} paññāyethā’’ti? ‘‘No hetaṃ, bhante’’. ‘‘Tasmātihānanda, eseva hetu etaṃ nidānaṃ esa samudayo esa paccayo viññāṇassa yadidaṃ nāmarūpaṃ. Ettāvatā kho, ānanda, jāyetha vā jīyetha\footnote{jiyyetha (ka.)} vā mīyetha\footnote{miyyetha (ka.)} vā cavetha vā upapajjetha vā. Ettāvatā adhivacanapatho, ettāvatā niruttipatho, ettāvatā paññattipatho, ettāvatā paññāvacaraṃ, ettāvatā vaṭṭaṃ vattati itthattaṃ paññāpanāya yadidaṃ nāmarūpaṃ saha viññāṇena aññamaññapaccayatā pavattati.

\subsubsection{Attapaññatti}

\paragraph{117.} ‘‘Kittāvatā ca, ānanda, attānaṃ paññapento paññapeti? Rūpiṃ vā hi, ānanda, parittaṃ attānaṃ paññapento paññapeti – ‘‘rūpī me paritto attā’’ti. Rūpiṃ vā hi , ānanda, anantaṃ attānaṃ paññapento paññapeti – ‘rūpī me ananto attā’ti. Arūpiṃ vā hi, ānanda, parittaṃ attānaṃ paññapento paññapeti – ‘arūpī me paritto attā’ti. Arūpiṃ vā hi, ānanda, anantaṃ attānaṃ paññapento paññapeti – ‘arūpī me ananto attā’ti.

\paragraph{118.} ‘‘Tatrānanda, yo so rūpiṃ parittaṃ attānaṃ paññapento paññapeti. Etarahi vā so rūpiṃ parittaṃ attānaṃ paññapento paññapeti, tattha bhāviṃ vā so rūpiṃ parittaṃ attānaṃ paññapento paññapeti, ‘atathaṃ vā pana santaṃ tathattāya upakappessāmī’ti iti vā panassa hoti. Evaṃ santaṃ kho, ānanda, rūpiṃ\footnote{rūpī (ka.)} parittattānudiṭṭhi anusetīti iccālaṃ vacanāya.

‘‘Tatrānanda, yo so rūpiṃ anantaṃ attānaṃ paññapento paññapeti. Etarahi vā so rūpiṃ anantaṃ attānaṃ paññapento paññapeti, tattha bhāviṃ vā so rūpiṃ anantaṃ attānaṃ paññapento paññapeti, ‘atathaṃ vā pana santaṃ tathattāya upakappessāmī’ti iti vā panassa hoti. Evaṃ santaṃ kho, ānanda, rūpiṃ\footnote{rūpī (ka.)} anantattānudiṭṭhi anusetīti iccālaṃ vacanāya.

‘‘Tatrānanda, yo so arūpiṃ parittaṃ attānaṃ paññapento paññapeti. Etarahi vā so arūpiṃ parittaṃ attānaṃ paññapento paññapeti, tattha bhāviṃ vā so arūpiṃ parittaṃ attānaṃ paññapento paññapeti, ‘atathaṃ vā pana santaṃ tathattāya upakappessāmī’ti iti vā panassa hoti. Evaṃ santaṃ kho, ānanda, arūpiṃ\footnote{arūpī (ka.)} parittattānudiṭṭhi anusetīti iccālaṃ vacanāya.

‘‘Tatrānanda, yo so arūpiṃ anantaṃ attānaṃ paññapento paññapeti. Etarahi vā so arūpiṃ anantaṃ attānaṃ paññapento paññapeti, tattha bhāviṃ vā so arūpiṃ anantaṃ attānaṃ paññapento paññapeti, ‘atathaṃ vā pana santaṃ tathattāya upakappessāmī’ti iti vā panassa hoti. Evaṃ santaṃ kho, ānanda, arūpiṃ\footnote{arūpī (ka.)} anantattānudiṭṭhi anusetīti iccālaṃ vacanāya. Ettāvatā kho, ānanda, attānaṃ paññapento paññapeti.

\subsubsection{Naattapaññatti}

\paragraph{119.} ‘‘Kittāvatā ca, ānanda, attānaṃ na paññapento na paññapeti? Rūpiṃ vā hi, ānanda, parittaṃ attānaṃ na paññapento na paññapeti – ‘rūpī me paritto attā’ti. Rūpiṃ vā hi, ānanda, anantaṃ attānaṃ na paññapento na paññapeti – ‘rūpī me ananto attā’ti. Arūpiṃ vā hi, ānanda, parittaṃ attānaṃ na paññapento na paññapeti – ‘arūpī me paritto attā’ti. Arūpiṃ vā hi, ānanda, anantaṃ attānaṃ na paññapento na paññapeti – ‘arūpī me ananto attā’ti.

\paragraph{120.} ‘‘Tatrānanda, yo so rūpiṃ parittaṃ attānaṃ na paññapento na paññapeti. Etarahi vā so rūpiṃ parittaṃ attānaṃ na paññapento na paññapeti, tattha bhāviṃ vā so rūpiṃ parittaṃ attānaṃ na paññapento na paññapeti, ‘atathaṃ vā pana santaṃ tathattāya upakappessāmī’ti iti vā panassa na hoti. Evaṃ santaṃ kho, ānanda, rūpiṃ parittattānudiṭṭhi nānusetīti iccālaṃ vacanāya.

‘‘Tatrānanda , yo so rūpiṃ anantaṃ attānaṃ na paññapento na paññapeti. Etarahi vā so rūpiṃ anantaṃ attānaṃ na paññapento na paññapeti, tattha bhāviṃ vā so rūpiṃ anantaṃ attānaṃ na paññapento na paññapeti, ‘atathaṃ vā pana santaṃ tathattāya upakappessāmī’ti iti vā panassa na hoti. Evaṃ santaṃ kho, ānanda, rūpiṃ anantattānudiṭṭhi nānusetīti iccālaṃ vacanāya.

‘‘Tatrānanda, yo so arūpiṃ parittaṃ attānaṃ na paññapento na paññapeti. Etarahi vā so arūpiṃ parittaṃ attānaṃ na paññapento na paññapeti, tattha bhāviṃ vā so arūpiṃ parittaṃ attānaṃ na paññapento na paññapeti, ‘atathaṃ vā pana santaṃ tathattāya upakappessāmī’ti iti vā panassa na hoti. Evaṃ santaṃ kho, ānanda, arūpiṃ parittattānudiṭṭhi nānusetīti iccālaṃ vacanāya.

‘‘Tatrānanda, yo so arūpiṃ anantaṃ attānaṃ na paññapento na paññapeti. Etarahi vā so arūpiṃ anantaṃ attānaṃ na paññapento na paññapeti, tattha bhāviṃ vā so arūpiṃ anantaṃ attānaṃ na paññapento na paññapeti, ‘atathaṃ vā pana santaṃ tathattāya upakappessāmī’ti iti vā panassa na hoti. Evaṃ santaṃ kho, ānanda, arūpiṃ anantattānudiṭṭhi nānusetīti iccālaṃ vacanāya. Ettāvatā kho, ānanda, attānaṃ na paññapento na paññapeti.

\subsubsection{Attasamanupassanā}

\paragraph{121.} ‘‘Kittāvatā ca, ānanda, attānaṃ samanupassamāno samanupassati? Vedanaṃ vā hi, ānanda, attānaṃ samanupassamāno samanupassati – ‘vedanā me attā’ti. ‘Na heva kho me vedanā attā, appaṭisaṃvedano me attā’ti iti vā hi, ānanda, attānaṃ samanupassamāno samanupassati. ‘Na heva kho me vedanā attā, nopi appaṭisaṃvedano me attā, attā me vediyati, vedanādhammo hi me attā’ti iti vā hi, ānanda, attānaṃ samanupassamāno samanupassati.

\paragraph{122.} ‘‘Tatrānanda, yo so evamāha – ‘vedanā me attā’ti, so evamassa vacanīyo – ‘tisso kho imā, āvuso, vedanā – sukhā vedanā dukkhā vedanā adukkhamasukhā vedanā. Imāsaṃ kho tvaṃ tissannaṃ vedanānaṃ katamaṃ attato samanupassasī’ti? Yasmiṃ, ānanda, samaye sukhaṃ vedanaṃ vedeti, neva tasmiṃ samaye dukkhaṃ vedanaṃ vedeti, na adukkhamasukhaṃ vedanaṃ vedeti; sukhaṃyeva tasmiṃ samaye vedanaṃ vedeti. Yasmiṃ, ānanda, samaye dukkhaṃ vedanaṃ vedeti, neva tasmiṃ samaye sukhaṃ vedanaṃ vedeti, na adukkhamasukhaṃ vedanaṃ vedeti; dukkhaṃyeva tasmiṃ samaye vedanaṃ vedeti. Yasmiṃ, ānanda, samaye adukkhamasukhaṃ vedanaṃ vedeti, neva tasmiṃ samaye sukhaṃ vedanaṃ vedeti, na dukkhaṃ vedanaṃ vedeti; adukkhamasukhaṃyeva tasmiṃ samaye vedanaṃ vedeti.

\paragraph{123.} ‘‘Sukhāpi kho, ānanda, vedanā aniccā saṅkhatā paṭiccasamuppannā khayadhammā vayadhammā virāgadhammā nirodhadhammā. Dukkhāpi kho, ānanda, vedanā aniccā saṅkhatā paṭiccasamuppannā khayadhammā vayadhammā virāgadhammā nirodhadhammā. Adukkhamasukhāpi kho, ānanda, vedanā aniccā saṅkhatā paṭiccasamuppannā khayadhammā vayadhammā virāgadhammā nirodhadhammā. Tassa sukhaṃ vedanaṃ vediyamānassa ‘eso me attā’ti hoti. Tassāyeva sukhāya vedanāya nirodhā ‘byagā\footnote{byaggā (sī. ka.)} me attā’ti hoti. Dukkhaṃ vedanaṃ vediyamānassa ‘eso me attā’ti hoti. Tassāyeva dukkhāya vedanāya nirodhā ‘byagā me attā’ti hoti. Adukkhamasukhaṃ vedanaṃ vediyamānassa ‘eso me attā’ti hoti. Tassāyeva adukkhamasukhāya vedanāya nirodhā ‘byagā me attā’ti hoti. Iti so diṭṭheva dhamme aniccasukhadukkhavokiṇṇaṃ uppādavayadhammaṃ attānaṃ samanupassamāno samanupassati, yo so evamāha – ‘vedanā me attā’ti. Tasmātihānanda, etena petaṃ nakkhamati – ‘vedanā me attā’ti samanupassituṃ.

\paragraph{124.} ‘‘Tatrānanda , yo so evamāha – ‘na heva kho me vedanā attā, appaṭisaṃvedano me attā’ti, so evamassa vacanīyo – ‘yattha panāvuso, sabbaso vedayitaṃ natthi api nu kho, tattha ‘‘ayamahamasmī’’ti siyā’’’ti ? ‘‘No hetaṃ, bhante’’. ‘‘Tasmātihānanda, etena petaṃ nakkhamati – ‘na heva kho me vedanā attā, appaṭisaṃvedano me attā’ti samanupassituṃ.

\paragraph{125.} ‘‘Tatrānanda , yo so evamāha – ‘na heva kho me vedanā attā, nopi appaṭisaṃvedano me attā, attā me vediyati, vedanādhammo hi me attā’ti. So evamassa vacanīyo – vedanā ca hi, āvuso, sabbena sabbaṃ sabbathā sabbaṃ aparisesā nirujjheyyuṃ. Sabbaso vedanāya asati vedanānirodhā api nu kho tattha ‘ayamahamasmī’ti siyā’’ti? ‘No hetaṃ, bhante’’. ‘‘Tasmātihānanda, etena petaṃ nakkhamati – ‘‘na heva kho me vedanā attā, nopi appaṭisaṃvedano me attā, attā me vediyati, vedanādhammo hi me attā’ti samanupassituṃ.

\paragraph{126.} ‘‘Yato kho, ānanda, bhikkhu neva vedanaṃ attānaṃ samanupassati, nopi appaṭisaṃvedanaṃ attānaṃ samanupassati, nopi ‘attā me vediyati, vedanādhammo hi me attā’ti samanupassati. So evaṃ na samanupassanto na ca kiñci loke upādiyati, anupādiyaṃ na paritassati, aparitassaṃ\footnote{aparitassanaṃ (ka.)} paccattaññeva parinibbāyati, ‘khīṇā jāti, vusitaṃ brahmacariyaṃ, kataṃ karaṇīyaṃ, nāparaṃ itthattāyā’ti pajānāti. Evaṃ vimuttacittaṃ kho, ānanda, bhikkhuṃ yo evaṃ vadeyya – ‘hoti tathāgato paraṃ maraṇā itissa\footnote{iti sā (aṭṭhakathāyaṃ pāṭhantaraṃ)} diṭṭhī’ti, tadakallaṃ. ‘Na hoti tathāgato paraṃ maraṇā itissa diṭṭhī’ti, tadakallaṃ. ‘Hoti ca na ca hoti tathāgato paraṃ maraṇā itissa diṭṭhī’ti, tadakallaṃ. ‘Neva hoti na na hoti tathāgato paraṃ maraṇā itissa diṭṭhī’ti, tadakallaṃ. Taṃ kissa hetu? Yāvatā, ānanda, adhivacanaṃ yāvatā adhivacanapatho, yāvatā nirutti yāvatā niruttipatho, yāvatā paññatti yāvatā paññattipatho, yāvatā paññā yāvatā paññāvacaraṃ, yāvatā vaṭṭaṃ\footnote{yāvatā vaṭṭaṃ vaṭṭati (ka. sī.)}, yāvatā vaṭṭati\footnote{yāvatā vaṭṭaṃ vaṭṭati (ka. sī.)}, tadabhiññāvimutto bhikkhu, tadabhiññāvimuttaṃ bhikkhuṃ ‘na jānāti na passati itissa diṭṭhī’ti, tadakallaṃ.

\subsubsection{Satta viññāṇaṭṭhiti}

\paragraph{127.} ‘‘Satta kho, ānanda\footnote{satta kho imā ānanda (ka. sī. syā.)}, viññāṇaṭṭhitiyo, dve āyatanāni. Katamā satta? Santānanda, sattā nānattakāyā nānattasaññino, seyyathāpi manussā , ekacce ca devā, ekacce ca vinipātikā. Ayaṃ paṭhamā viññāṇaṭṭhiti. Santānanda, sattā nānattakāyā ekattasaññino, seyyathāpi devā brahmakāyikā paṭhamābhinibbattā. Ayaṃ dutiyā viññāṇaṭṭhiti. Santānanda, sattā ekattakāyā nānattasaññino, seyyathāpi devā ābhassarā. Ayaṃ tatiyā viññāṇaṭṭhiti. Santānanda, sattā ekattakāyā ekattasaññino, seyyathāpi devā subhakiṇhā. Ayaṃ catutthī viññāṇaṭṭhiti. Santānanda, sattā sabbaso rūpasaññānaṃ samatikkamā paṭighasaññānaṃ atthaṅgamā nānattasaññānaṃ amanasikārā ‘ananto ākāso’ti ākāsānañcāyatanūpagā. Ayaṃ pañcamī viññāṇaṭṭhiti . Santānanda, sattā sabbaso ākāsānañcāyatanaṃ samatikkamma ‘anantaṃ viññāṇa’nti viññāṇañcāyatanūpagā. Ayaṃ chaṭṭhī viññāṇaṭṭhiti. Santānanda, sattā sabbaso viññāṇañcāyatanaṃ samatikkamma ‘natthi kiñcī’ti ākiñcaññāyatanūpagā. Ayaṃ sattamī viññāṇaṭṭhiti. Asaññasattāyatanaṃ nevasaññānāsaññāyatanameva dutiyaṃ.

\paragraph{128.} ‘‘Tatrānanda, yāyaṃ paṭhamā viññāṇaṭṭhiti nānattakāyā nānattasaññino, seyyathāpi manussā, ekacce ca devā, ekacce ca vinipātikā. Yo nu kho, ānanda, tañca pajānāti, tassā ca samudayaṃ pajānāti, tassā ca atthaṅgamaṃ pajānāti, tassā ca assādaṃ pajānāti, tassā ca ādīnavaṃ pajānāti, tassā ca nissaraṇaṃ pajānāti, kallaṃ nu tena tadabhinanditu’’nti? ‘‘No hetaṃ, bhante’’…pe… ‘‘tatrānanda, yamidaṃ asaññasattāyatanaṃ. Yo nu kho, ānanda, tañca pajānāti, tassa ca samudayaṃ pajānāti, tassa ca atthaṅgamaṃ pajānāti, tassa ca assādaṃ pajānāti, tassa ca ādīnavaṃ pajānāti, tassa ca nissaraṇaṃ pajānāti, kallaṃ nu tena tadabhinanditu’’nti? ‘‘No hetaṃ, bhante’’. ‘‘Tatrānanda, yamidaṃ nevasaññānāsaññāyatanaṃ. Yo nu kho, ānanda, tañca pajānāti, tassa ca samudayaṃ pajānāti, tassa ca atthaṅgamaṃ pajānāti, tassa ca assādaṃ pajānāti, tassa ca ādīnavaṃ pajānāti, tassa ca nissaraṇaṃ pajānāti, kallaṃ nu tena tadabhinanditu’’nti? ‘‘No hetaṃ, bhante’’. Yato kho, ānanda, bhikkhu imāsañca sattannaṃ viññāṇaṭṭhitīnaṃ imesañca dvinnaṃ āyatanānaṃ samudayañca atthaṅgamañca assādañca ādīnavañca nissaraṇañca yathābhūtaṃ viditvā anupādā vimutto hoti, ayaṃ vuccatānanda, bhikkhu paññāvimutto.

\subsubsection{Aṭṭha vimokkhā}

\paragraph{129.} ‘‘Aṭṭha kho ime, ānanda, vimokkhā. Katame aṭṭha? Rūpī rūpāni passati ayaṃ paṭhamo vimokkho. Ajjhattaṃ arūpasaññī bahiddhā rūpāni passati, ayaṃ dutiyo vimokkho. Subhanteva adhimutto hoti, ayaṃ tatiyo vimokkho. Sabbaso rūpasaññānaṃ samatikkamā paṭighasaññānaṃ atthaṅgamā nānattasaññānaṃ amanasikārā ‘ananto ākāso’ti ākāsānañcāyatanaṃ upasampajja viharati, ayaṃ catuttho vimokkho. Sabbaso ākāsānañcāyatanaṃ samatikkamma ‘anantaṃ viññāṇa’nti viññāṇañcāyatanaṃ upasampajja viharati, ayaṃ pañcamo vimokkho. Sabbaso viññāṇañcāyatanaṃ samatikkamma ‘natthi kiñcī’ti ākiñcaññāyatanaṃ upasampajja viharati, ayaṃ chaṭṭho vimokkho. Sabbaso ākiñcaññāyatanaṃ samatikkamma ‘nevasaññānāsaññā’yatanaṃ upasampajja viharati, ayaṃ sattamo vimokkho. Sabbaso nevasaññānāsaññāyatanaṃ samatikkamma saññāvedayitanirodhaṃ upasampajja viharati, ayaṃ aṭṭhamo vimokkho. Ime kho, ānanda, aṭṭha vimokkhā.

\paragraph{130.} ‘‘Yato kho, ānanda, bhikkhu ime aṭṭha vimokkhe anulomampi samāpajjati, paṭilomampi samāpajjati, anulomapaṭilomampi samāpajjati, yatthicchakaṃ yadicchakaṃ yāvaticchakaṃ samāpajjatipi vuṭṭhātipi. Āsavānañca khayā anāsavaṃ cetovimuttiṃ paññāvimuttiṃ diṭṭheva dhamme sayaṃ abhiññā sacchikatvā upasampajja viharati, ayaṃ vuccatānanda, bhikkhu ubhatobhāgavimutto. Imāya ca ānanda ubhatobhāgavimuttiyā aññā ubhatobhāgavimutti uttaritarā vā paṇītatarā vā natthī’’ti. Idamavoca bhagavā. Attamano āyasmā ānando bhagavato bhāsitaṃ abhinandīti.

\xsectionEnd{Mahānidānasuttaṃ niṭṭhitaṃ dutiyaṃ.}


%\clearpage
%\section{Mahāparinibbānasuttaṃ}

\paragraph{131.} Evaṃ me sutaṃ – ekaṃ samayaṃ bhagavā rājagahe viharati gijjhakūṭe pabbate. Tena kho pana samayena rājā māgadho ajātasattu vedehiputto vajjī abhiyātukāmo hoti. So evamāha – ‘‘ahaṃ hime vajjī evaṃmahiddhike evaṃmahānubhāve ucchecchāmi\footnote{ucchejjāmi (syā. pī.), ucchijjāmi (ka.)} vajjī, vināsessāmi vajjī, anayabyasanaṃ āpādessāmi vajjī’’ti\footnote{āpādessāmi vajjīti (sabbattha) a. ni. 7.22 passitabbaṃ}.

\paragraph{132.} Atha kho rājā māgadho ajātasattu vedehiputto vassakāraṃ brāhmaṇaṃ magadhamahāmattaṃ āmantesi – ‘‘ehi tvaṃ, brāhmaṇa, yena bhagavā tenupasaṅkama; upasaṅkamitvā mama vacanena bhagavato pāde sirasā vandāhi, appābādhaṃ appātaṅkaṃ lahuṭṭhānaṃ balaṃ phāsuvihāraṃ puccha – ‘rājā, bhante, māgadho ajātasattu vedehiputto bhagavato pāde sirasā vandati, appābādhaṃ appātaṅkaṃ lahuṭṭhānaṃ balaṃ phāsuvihāraṃ pucchatī’ti. Evañca vadehi – ‘rājā, bhante, māgadho ajātasattu vedehiputto vajjī abhiyātukāmo. So evamāha – ‘‘ahaṃ hime vajjī evaṃmahiddhike evaṃmahānubhāve ucchecchāmi vajjī, vināsessāmi vajjī, anayabyasanaṃ āpādessāmī’’’ti. Yathā te bhagavā byākaroti, taṃ sādhukaṃ uggahetvā mama āroceyyāsi. Na hi tathāgatā vitathaṃ bhaṇantī’’ti.

\subsubsection{Vassakārabrāhmaṇo}

\paragraph{133.} ‘‘Evaṃ, bho’’ti kho vassakāro brāhmaṇo magadhamahāmatto rañño māgadhassa ajātasattussa vedehiputtassa paṭissutvā bhaddāni bhaddāni yānāni yojetvā bhaddaṃ bhaddaṃ yānaṃ abhiruhitvā bhaddehi bhaddehi yānehi rājagahamhā niyyāsi, yena gijjhakūṭo pabbato tena pāyāsi. Yāvatikā yānassa bhūmi, yānena gantvā, yānā paccorohitvā pattikova yena bhagavā tenupasaṅkami; upasaṅkamitvā bhagavatā saddhiṃ sammodi. Sammodanīyaṃ kathaṃ sāraṇīyaṃ vītisāretvā ekamantaṃ nisīdi. Ekamantaṃ nisinno kho vassakāro brāhmaṇo magadhamahāmatto bhagavantaṃ etadavoca – ‘‘rājā, bho gotama, māgadho ajātasattu vedehiputto bhoto gotamassa pāde sirasā vandati, appābādhaṃ appātaṅkaṃ lahuṭṭhānaṃ balaṃ phāsuvihāraṃ pucchati. Rājā\footnote{evañca vadeti rājā (ka.)}, bho gotama, māgadho ajātasattu vedehiputto vajjī abhiyātukāmo. So evamāha – ‘ahaṃ hime vajjī evaṃmahiddhike evaṃmahānubhāve ucchecchāmi vajjī, vināsessāmi vajjī, anayabyasanaṃ āpādessāmī’’’ti.

\subsubsection{Rājaaparihāniyadhammā}

\paragraph{134.} Tena kho pana samayena āyasmā ānando bhagavato piṭṭhito ṭhito hoti bhagavantaṃ bījayamāno\footnote{vījayamāno (sī.), vījiyamāno (syā.)}. Atha kho bhagavā āyasmantaṃ ānandaṃ āmantesi – ‘‘kinti te, ānanda, sutaṃ, ‘vajjī abhiṇhaṃ sannipātā sannipātabahulā’ti? ‘‘Sutaṃ metaṃ, bhante – ‘vajjī abhiṇhaṃ sannipātā sannipātabahulā’’ti. ‘‘Yāvakīvañca, ānanda, vajjī abhiṇhaṃ sannipātā sannipātabahulā bhavissanti, vuddhiyeva, ānanda, vajjīnaṃ pāṭikaṅkhā, no parihāni.

‘‘Kinti te, ānanda, sutaṃ , ‘vajjī samaggā sannipatanti, samaggā vuṭṭhahanti, samaggā vajjikaraṇīyāni karontī’ti? ‘‘Sutaṃ metaṃ, bhante – ‘vajjī samaggā sannipatanti, samaggā vuṭṭhahanti, samaggā vajjikaraṇīyāni karontī’’ti. ‘‘Yāvakīvañca, ānanda, vajjī samaggā sannipatissanti, samaggā vuṭṭhahissanti, samaggā vajjikaraṇīyāni karissanti, vuddhiyeva, ānanda, vajjīnaṃ pāṭikaṅkhā, no parihāni.

‘‘Kinti te, ānanda, sutaṃ, ‘vajjī apaññattaṃ na paññapenti, paññattaṃ na samucchindanti, yathāpaññatte porāṇe vajjidhamme samādāya vattantī’’’ti? ‘‘Sutaṃ metaṃ, bhante – ‘vajjī apaññattaṃ na paññapenti, paññattaṃ na samucchindanti, yathāpaññatte porāṇe vajjidhamme samādāya vattantī’’’ti. ‘‘Yāvakīvañca, ānanda, ‘‘vajjī apaññattaṃ na paññapessanti, paññattaṃ na samucchindissanti, yathāpaññatte porāṇe vajjidhamme samādāya vattissanti, vuddhiyeva, ānanda, vajjīnaṃ pāṭikaṅkhā, no parihāni.

‘‘Kinti te, ānanda, sutaṃ, ‘vajjī ye te vajjīnaṃ vajjimahallakā, te sakkaronti garuṃ karonti\footnote{garukaronti (sī. syā. pī.)} mānenti pūjenti, tesañca sotabbaṃ maññantī’’’ti? ‘‘Sutaṃ metaṃ, bhante – ‘vajjī ye te vajjīnaṃ vajjimahallakā, te sakkaronti garuṃ karonti mānenti pūjenti, tesañca sotabbaṃ maññantī’’’ti. ‘‘Yāvakīvañca, ānanda, vajjī ye te vajjīnaṃ vajjimahallakā , te sakkarissanti garuṃ karissanti mānessanti pūjessanti, tesañca sotabbaṃ maññissanti, vuddhiyeva, ānanda, vajjīnaṃ pāṭikaṅkhā, no parihāni.

‘‘Kinti te, ānanda, sutaṃ, ‘vajjī yā tā kulitthiyo kulakumāriyo, tā na okkassa pasayha vāsentī’’’ti? ‘‘Sutaṃ metaṃ, bhante – ‘vajjī yā tā kulitthiyo kulakumāriyo tā na okkassa pasayha vāsentī’’’ti. ‘‘Yāvakīvañca, ānanda, vajjī yā tā kulitthiyo kulakumāriyo, tā na okkassa pasayha vāsessanti, vuddhiyeva, ānanda, vajjīnaṃ pāṭikaṅkhā, no parihāni.

‘‘Kinti te, ānanda, sutaṃ, ‘vajjī yāni tāni

Vajjīnaṃ vajjicetiyāni abbhantarāni ceva bāhirāni ca, tāni sakkaronti garuṃ karonti mānenti pūjenti, tesañca dinnapubbaṃ katapubbaṃ dhammikaṃ baliṃ no parihāpentī’’’ti? ‘‘Sutaṃ metaṃ, bhante – ‘vajjī yāni tāni vajjīnaṃ vajjicetiyāni abbhantarāni ceva bāhirāni ca, tāni sakkaronti garuṃ karonti mānenti pūjenti tesañca dinnapubbaṃ katapubbaṃ dhammikaṃ baliṃ no parihāpentī’’’ti. ‘‘Yāvakīvañca, ānanda, vajjī yāni tāni vajjīnaṃ vajjicetiyāni abbhantarāni ceva bāhirāni ca, tāni sakkarissanti garuṃ karissanti mānessanti pūjessanti, tesañca dinnapubbaṃ katapubbaṃ dhammikaṃ baliṃ no parihāpessanti, vuddhiyeva, ānanda, vajjīnaṃ pāṭikaṅkhā, no parihāni.

‘‘Kinti te, ānanda, sutaṃ, ‘vajjīnaṃ arahantesu dhammikā rakkhāvaraṇagutti susaṃvihitā, kinti anāgatā ca arahanto vijitaṃ āgaccheyyuṃ, āgatā ca arahanto vijite phāsu vihareyyu’’’nti? ‘‘Sutaṃ metaṃ, bhante ‘vajjīnaṃ arahantesu dhammikā rakkhāvaraṇagutti susaṃvihitā kinti anāgatā ca arahanto vijitaṃ āgaccheyyuṃ, āgatā ca arahanto vijite phāsu vihareyyu’’’nti. ‘‘Yāvakīvañca, ānanda, vajjīnaṃ arahantesu dhammikā rakkhāvaraṇagutti susaṃvihitā bhavissati, kinti anāgatā ca arahanto vijitaṃ āgaccheyyuṃ, āgatā ca arahanto vijite phāsu vihareyyunti. Vuddhiyeva, ānanda, vajjīnaṃ pāṭikaṅkhā, no parihānī’’ti.

\paragraph{135.} Atha kho bhagavā vassakāraṃ brāhmaṇaṃ magadhamahāmattaṃ āmantesi – ‘‘ekamidāhaṃ, brāhmaṇa, samayaṃ vesāliyaṃ viharāmi sārandade\footnote{sānandare (ka.)} cetiye. Tatrāhaṃ vajjīnaṃ ime satta aparihāniye dhamme desesiṃ. Yāvakīvañca, brāhmaṇa, ime satta aparihāniyā dhammā vajjīsu ṭhassanti, imesu ca sattasu aparihāniyesu dhammesu vajjī sandississanti, vuddhiyeva, brāhmaṇa, vajjīnaṃ pāṭikaṅkhā, no parihānī’’ti.

Evaṃ vutte, vassakāro brāhmaṇo magadhamahāmatto bhagavantaṃ etadavoca – ‘‘ekamekenapi, bho gotama, aparihāniyena dhammena samannāgatānaṃ vajjīnaṃ vuddhiyeva pāṭikaṅkhā, no parihāni . Ko pana vādo sattahi aparihāniyehi dhammehi. Akaraṇīyāva\footnote{akaraṇīyā ca (syā. ka.)}, bho gotama, vajjī\footnote{vajjīnaṃ (ka.)} raññā māgadhena ajātasattunā vedehiputtena yadidaṃ yuddhassa, aññatra upalāpanāya aññatra mithubhedā. Handa ca dāni mayaṃ, bho gotama, gacchāma , bahukiccā mayaṃ bahukaraṇīyā’’ti. ‘‘Yassadāni tvaṃ, brāhmaṇa, kālaṃ maññasī’’ti. Atha kho vassakāro brāhmaṇo magadhamahāmatto bhagavato bhāsitaṃ abhinanditvā anumoditvā uṭṭhāyāsanā pakkāmi.

\subsubsection{Bhikkhuaparihāniyadhammā}

\paragraph{136.} Atha kho bhagavā acirapakkante vassakāre brāhmaṇe magadhamahāmatte āyasmantaṃ ānandaṃ āmantesi – ‘‘gaccha tvaṃ, ānanda, yāvatikā bhikkhū rājagahaṃ upanissāya viharanti, te sabbe upaṭṭhānasālāyaṃ sannipātehī’’ti. ‘‘Evaṃ, bhante’’ti kho āyasmā ānando bhagavato paṭissutvā yāvatikā bhikkhū rājagahaṃ upanissāya viharanti, te sabbe upaṭṭhānasālāyaṃ sannipātetvā yena bhagavā tenupasaṅkami; upasaṅkamitvā bhagavantaṃ abhivādetvā ekamantaṃ aṭṭhāsi. Ekamantaṃ ṭhito kho āyasmā ānando bhagavantaṃ etadavoca – ‘‘sannipatito, bhante, bhikkhusaṅgho, yassadāni, bhante, bhagavā kālaṃ maññatī’’ti.

Atha kho bhagavā uṭṭhāyāsanā yena upaṭṭhānasālā tenupasaṅkami; upasaṅkamitvā paññatte āsane nisīdi. Nisajja kho bhagavā bhikkhū āmantesi – ‘‘satta vo, bhikkhave, aparihāniye dhamme desessāmi, taṃ suṇātha, sādhukaṃ manasikarotha, bhāsissāmī’’ti. ‘‘Evaṃ, bhante’’ti kho te bhikkhū bhagavato paccassosuṃ. Bhagavā etadavoca –

‘‘Yāvakīvañca , bhikkhave, bhikkhū abhiṇhaṃ sannipātā sannipātabahulā bhavissanti, vuddhiyeva, bhikkhave, bhikkhūnaṃ pāṭikaṅkhā, no parihāni.

‘‘Yāvakīvañca, bhikkhave, bhikkhū samaggā sannipatissanti, samaggā vuṭṭhahissanti, samaggā saṅghakaraṇīyāni karissanti , vuddhiyeva, bhikkhave, bhikkhūnaṃ pāṭikaṅkhā, no parihāni.

‘‘Yāvakīvañca, bhikkhave, bhikkhū apaññattaṃ na paññapessanti, paññattaṃ na samucchindissanti, yathāpaññattesu sikkhāpadesu samādāya vattissanti, vuddhiyeva, bhikkhave, bhikkhūnaṃ pāṭikaṅkhā, no parihāni.

‘‘Yāvakīvañca, bhikkhave, bhikkhū ye te bhikkhū therā rattaññū cirapabbajitā saṅghapitaro saṅghapariṇāyakā, te sakkarissanti garuṃ karissanti mānessanti pūjessanti, tesañca sotabbaṃ maññissanti, vuddhiyeva, bhikkhave, bhikkhūnaṃ pāṭikaṅkhā, no parihāni.

‘‘Yāvakīvañca, bhikkhave, bhikkhū uppannāya taṇhāya ponobbhavikāya na vasaṃ gacchissanti, vuddhiyeva, bhikkhave, bhikkhūnaṃ pāṭikaṅkhā, no parihāni.

‘‘Yāvakīvañca, bhikkhave, bhikkhū āraññakesu senāsanesu sāpekkhā bhavissanti, vuddhiyeva, bhikkhave, bhikkhūnaṃ pāṭikaṅkhā, no parihāni.

‘‘Yāvakīvañca, bhikkhave, bhikkhū paccattaññeva satiṃ upaṭṭhapessanti – ‘kinti anāgatā ca pesalā sabrahmacārī āgaccheyyuṃ, āgatā ca pesalā sabrahmacārī phāsu\footnote{phāsuṃ (sī. syā. pī.)} vihareyyu’nti. Vuddhiyeva, bhikkhave, bhikkhūnaṃ pāṭikaṅkhā, no parihāni.

‘‘Yāvakīvañca, bhikkhave, ime satta aparihāniyā dhammā bhikkhūsu ṭhassanti, imesu ca sattasu aparihāniyesu dhammesu bhikkhū sandississanti, vuddhiyeva, bhikkhave, bhikkhūnaṃ pāṭikaṅkhā, no parihāni.

\paragraph{137.} ‘‘Aparepi vo, bhikkhave, satta aparihāniye dhamme desessāmi, taṃ suṇātha, sādhukaṃ manasikarotha, bhāsissāmī’’ti. ‘‘Evaṃ, bhante’’ti kho te bhikkhū bhagavato paccassosuṃ. Bhagavā etadavoca –

‘‘Yāvakīvañca, bhikkhave, bhikkhū na kammārāmā bhavissanti na kammaratā na kammārāmatamanuyuttā, vuddhiyeva, bhikkhave, bhikkhūnaṃ pāṭikaṅkhā, no parihāni.

‘‘Yāvakīvañca, bhikkhave, bhikkhū na bhassārāmā bhavissanti na bhassaratā na bhassārāmatamanuyuttā, vuddhiyeva, bhikkhave, bhikkhūnaṃ pāṭikaṅkhā, no parihāni.

‘‘Yāvakīvañca, bhikkhave, bhikkhū na niddārāmā bhavissanti na niddāratā na niddārāmatamanuyuttā, vuddhiyeva, bhikkhave, bhikkhūnaṃ pāṭikaṅkhā, no parihāni.

‘‘Yāvakīvañca, bhikkhave, bhikkhū na saṅgaṇikārāmā bhavissanti na saṅgaṇikaratā na saṅgaṇikārāmatamanuyuttā, vuddhiyeva, bhikkhave, bhikkhūnaṃ pāṭikaṅkhā, no parihāni.

‘‘Yāvakīvañca, bhikkhave, bhikkhū na pāpicchā bhavissanti na pāpikānaṃ icchānaṃ vasaṃ gatā, vuddhiyeva, bhikkhave, bhikkhūnaṃ pāṭikaṅkhā, no parihāni.

‘‘Yāvakīvañca, bhikkhave, bhikkhū na pāpamittā bhavissanti na pāpasahāyā na pāpasampavaṅkā, vuddhiyeva, bhikkhave, bhikkhūnaṃ pāṭikaṅkhā, no parihāni.

‘‘Yāvakīvañca, bhikkhave, bhikkhū na oramattakena visesādhigamena antarāvosānaṃ āpajjissanti, vuddhiyeva, bhikkhave, bhikkhūnaṃ pāṭikaṅkhā, no parihāni.

‘‘Yāvakīvañca, bhikkhave, ime satta aparihāniyā dhammā bhikkhūsu ṭhassanti, imesu ca sattasu aparihāniyesu dhammesu bhikkhū sandississanti, vuddhiyeva, bhikkhave, bhikkhūnaṃ pāṭikaṅkhā, no parihāni.

\paragraph{138.} ‘‘Aparepi vo, bhikkhave, satta aparihāniye dhamme desessāmi…pe… ‘‘yāvakīvañca, bhikkhave, bhikkhū saddhā bhavissanti…pe… hirimanā bhavissanti… ottappī bhavissanti… bahussutā bhavissanti… āraddhavīriyā bhavissanti… upaṭṭhitassatī bhavissanti… paññavanto bhavissanti, vuddhiyeva, bhikkhave, bhikkhūnaṃ pāṭikaṅkhā, no parihāni. Yāvakīvañca, bhikkhave, ime satta aparihāniyā dhammā bhikkhūsu ṭhassanti, imesu ca sattasu aparihāniyesu dhammesu bhikkhū sandississanti, vuddhiyeva, bhikkhave, bhikkhūnaṃ pāṭikaṅkhā, no parihāni.

\paragraph{139.} ‘‘Aparepi vo, bhikkhave, satta aparihāniye dhamme desessāmi, taṃ suṇātha, sādhukaṃ manasikarotha, bhāsissāmī’’ti. ‘‘Evaṃ, bhante’’ti kho te bhikkhū bhagavato paccassosuṃ. Bhagavā etadavoca –

‘‘Yāvakīvañca, bhikkhave, bhikkhu satisambojjhaṅgaṃ bhāvessanti…pe… dhammavicayasambojjhaṅgaṃ bhāvessanti… vīriyasambojjhaṅgaṃ bhāvessanti… pītisambojjhaṅgaṃ bhāvessanti… passaddhisambojjhaṅgaṃ bhāvessanti… samādhisambojjhaṅgaṃ bhāvessanti… upekkhāsambojjhaṅgaṃ bhāvessanti, vuddhiyeva , bhikkhave, bhikkhūnaṃ pāṭikaṅkhā, no parihāni.

‘‘Yāvakīvañca, bhikkhave, ime satta aparihāniyā dhammā bhikkhūsu ṭhassanti, imesu ca sattasu aparihāniyesu dhammesu bhikkhū sandississanti, vuddhiyeva, bhikkhave, bhikkhūnaṃ pāṭikaṅkhā no parihāni.

\paragraph{140.} ‘‘Aparepi vo, bhikkhave, satta aparihāniye dhamme desessāmi, taṃ suṇātha, sādhukaṃ manasikarotha, bhāsissāmī’’ti. ‘‘Evaṃ, bhante’’ti kho te bhikkhū bhagavato paccassosuṃ. Bhagavā etadavoca –

‘‘Yāvakīvañca, bhikkhave, bhikkhū aniccasaññaṃ bhāvessanti…pe… anattasaññaṃ bhāvessanti… asubhasaññaṃ bhāvessanti… ādīnavasaññaṃ bhāvessanti… pahānasaññaṃ bhāvessanti… virāgasaññaṃ bhāvessanti… nirodhasaññaṃ bhāvessanti, vuddhiyeva, bhikkhave, bhikkhūnaṃ pāṭikaṅkhā, no parihāni.

‘‘Yāvakīvañca , bhikkhave, ime satta aparihāniyā dhammā bhikkhūsu ṭhassanti, imesu ca sattasu aparihāniyesu dhammesu bhikkhū sandississanti, vuddhiyeva, bhikkhave, bhikkhūnaṃ pāṭikaṅkhā, no parihāni.

\paragraph{141.} ‘‘Cha, vo bhikkhave, aparihāniye dhamme desessāmi, taṃ suṇātha, sādhukaṃ manasikarotha, bhāsissāmī’’ti. ‘‘Evaṃ, bhante’’ti kho te bhikkhū bhagavato paccassosuṃ. Bhagavā etadavoca –

‘‘Yāvakīvañca , bhikkhave, bhikkhū mettaṃ kāyakammaṃ paccupaṭṭhāpessanti sabrahmacārīsu āvi ceva raho ca, vuddhiyeva, bhikkhave, bhikkhūnaṃ pāṭikaṅkhā, no parihāni.

‘‘Yāvakīvañca, bhikkhave, bhikkhū mettaṃ vacīkammaṃ paccupaṭṭhāpessanti …pe… mettaṃ manokammaṃ paccupaṭṭhāpessanti sabrahmacārīsu āvi ceva raho ca, vuddhiyeva, bhikkhave, bhikkhūnaṃ pāṭikaṅkhā, no parihāni.

‘‘Yāvakīvañca, bhikkhave, bhikkhū, ye te lābhā dhammikā dhammaladdhā antamaso pattapariyāpannamattampi tathārūpehi lābhehi appaṭivibhattabhogī bhavissanti sīlavantehi sabrahmacārīhi sādhāraṇabhogī, vuddhiyeva, bhikkhave, bhikkhūnaṃ pāṭikaṅkhā, no parihāni.

‘‘Yāvakīvañca, bhikkhave, bhikkhū yāni kāni sīlāni akhaṇḍāni acchiddāni asabalāni akammāsāni bhujissāni viññūpasatthāni\footnote{viññuppasatthāni (sī.)} aparāmaṭṭhāni samādhisaṃvattanikāni tathārūpesu sīlesu sīlasāmaññagatā viharissanti sabrahmacārīhi āvi ceva raho ca, vuddhiyeva, bhikkhave, bhikkhūnaṃ pāṭikaṅkhā, no parihāni.

‘‘Yāvakīvañca, bhikkhave, bhikkhū yāyaṃ diṭṭhi ariyā niyyānikā, niyyāti takkarassa sammā dukkhakkhayāya, tathārūpāya diṭṭhiyā diṭṭhisāmaññagatā viharissanti sabrahmacārīhi āvi ceva raho ca, vuddhiyeva, bhikkhave, bhikkhūnaṃ pāṭikaṅkhā, no parihāni.

‘‘Yāvakīvañca , bhikkhave, ime cha aparihāniyā dhammā bhikkhūsu ṭhassanti, imesu ca chasu aparihāniyesu dhammesu bhikkhū sandississanti, vuddhiyeva, bhikkhave, bhikkhūnaṃ pāṭikaṅkhā, no parihānī’’ti.

\paragraph{142.} Tatra sudaṃ bhagavā rājagahe viharanto gijjhakūṭe pabbate etadeva bahulaṃ bhikkhūnaṃ dhammiṃ kathaṃ karoti – ‘‘iti sīlaṃ, iti samādhi, iti paññā. Sīlaparibhāvito samādhi mahapphalo hoti mahānisaṃso. Samādhiparibhāvitā paññā mahapphalā hoti mahānisaṃsā. Paññāparibhāvitaṃ cittaṃ sammadeva āsavehi vimuccati, seyyathidaṃ – kāmāsavā, bhavāsavā, avijjāsavā’’ti.

\paragraph{143.} Atha kho bhagavā rājagahe yathābhirantaṃ viharitvā āyasmantaṃ ānandaṃ āmantesi – ‘‘āyāmānanda, yena ambalaṭṭhikā tenupasaṅkamissāmā’’ti. ‘‘Evaṃ, bhante’’ti kho āyasmā ānando bhagavato paccassosi. Atha kho bhagavā mahatā bhikkhusaṅghena saddhiṃ yena ambalaṭṭhikā tadavasari. Tatra sudaṃ bhagavā ambalaṭṭhikāyaṃ viharati rājāgārake. Tatrāpi sudaṃ bhagavā ambalaṭṭhikāyaṃ viharanto rājāgārake etadeva bahulaṃ bhikkhūnaṃ dhammiṃ kathaṃ karoti – ‘‘iti sīlaṃ iti samādhi iti paññā. Sīlaparibhāvito samādhi mahapphalo hoti mahānisaṃso. Samādhiparibhāvitā paññā mahapphalā hoti mahānisaṃsā. Paññāparibhāvitaṃ cittaṃ sammadeva āsavehi vimuccati, seyyathidaṃ – kāmāsavā, bhavāsavā, avijjāsavā’’ti.

\paragraph{144.} Atha kho bhagavā ambalaṭṭhikāyaṃ yathābhirantaṃ viharitvā āyasmantaṃ ānandaṃ āmantesi – ‘‘āyāmānanda, yena nāḷandā tenupasaṅkamissāmā’’ti. ‘‘Evaṃ, bhante’’ti kho āyasmā ānando bhagavato paccassosi. Atha kho bhagavā mahatā bhikkhusaṅghena saddhiṃ yena nāḷandā tadavasari, tatra sudaṃ bhagavā nāḷandāyaṃ viharati pāvārikambavane .

\subsubsection{Sāriputtasīhanādo}

\paragraph{145.} Atha kho āyasmā sāriputto yena bhagavā tenupasaṅkami; upasaṅkamitvā bhagavantaṃ abhivādetvā ekamantaṃ nisīdi. Ekamantaṃ nisinno kho āyasmā sāriputto bhagavantaṃ etadavoca – ‘‘evaṃ pasanno ahaṃ, bhante, bhagavati; na cāhu na ca bhavissati na cetarahi vijjati añño samaṇo vā brāhmaṇo vā bhagavatā bhiyyobhiññataro yadidaṃ sambodhiya’’nti. ‘‘Uḷārā kho te ayaṃ, sāriputta, āsabhī vācā\footnote{āsabhivācā (syā.)} bhāsitā, ekaṃso gahito, sīhanādo nadito – ‘evaṃpasanno ahaṃ, bhante, bhagavati; na cāhu na ca bhavissati na cetarahi vijjati añño samaṇo vā brāhmaṇo vā bhagavatā bhiyyobhiññataro yadidaṃ sambodhiya’nti.

‘‘Kiṃ te\footnote{kiṃ nu (syā. pī. ka.)}, sāriputta, ye te ahesuṃ atītamaddhānaṃ arahanto sammāsambuddhā, sabbe te bhagavanto cetasā ceto paricca viditā – ‘evaṃsīlā te bhagavanto ahesuṃ itipi, evaṃdhammā evaṃpaññā evaṃvihārī evaṃvimuttā te bhagavanto ahesuṃ itipī’’’ti? ‘‘No hetaṃ, bhante’’.

‘‘Kiṃ pana te\footnote{kiṃ pana (syā. pī. ka.)}, sāriputta, ye te bhavissanti anāgatamaddhānaṃ arahanto sammāsambuddhā, sabbe te bhagavanto cetasā ceto paricca viditā – ‘evaṃsīlā te bhagavanto bhavissanti itipi, evaṃdhammā evaṃpaññā evaṃvihārī evaṃvimuttā te bhagavanto bhavissanti itipī’’’ti? ‘‘No hetaṃ, bhante’’.

‘‘Kiṃ pana te, sāriputta, ahaṃ etarahi arahaṃ sammāsambuddho cetasā ceto paricca vidito – ‘‘evaṃsīlo bhagavā itipi , evaṃdhammo evaṃpañño evaṃvihārī evaṃvimutto bhagavā itipī’’’ti? ‘‘No hetaṃ, bhante’’.

‘‘Ettha ca hi te, sāriputta, atītānāgatapaccuppannesu arahantesu sammāsambuddhesu cetopariyañāṇaṃ\footnote{cetopariññāyañāṇaṃ (syā.), cetasā cetopariyāyañāṇaṃ (ka.)} natthi. Atha kiñcarahi te ayaṃ, sāriputta, uḷārā āsabhī vācā bhāsitā, ekaṃso gahito, sīhanādo nadito – ‘evaṃpasanno ahaṃ, bhante, bhagavati; na cāhu na ca bhavissati na cetarahi vijjati añño samaṇo vā brāhmaṇo vā bhagavatā bhiyyobhiññataro yadidaṃ sambodhiya’’’nti?

\paragraph{146.} ‘‘Na kho me, bhante, atītānāgatapaccuppannesu arahantesu sammāsambuddhesu cetopariyañāṇaṃ atthi, api ca me dhammanvayo vidito. Seyyathāpi, bhante, rañño paccantimaṃ nagaraṃ daḷhuddhāpaṃ daḷhapākāratoraṇaṃ ekadvāraṃ, tatrassa dovāriko paṇḍito viyatto medhāvī aññātānaṃ nivāretā ñātānaṃ pavesetā. So tassa nagarassa samantā anupariyāyapathaṃ\footnote{anucariyāyapathaṃ (syā.)} anukkamamāno na passeyya pākārasandhiṃ vā pākāravivaraṃ vā, antamaso biḷāranikkhamanamattampi. Tassa evamassa\footnote{na passeyya tassa evamassa (syā.)} – ‘ye kho keci oḷārikā pāṇā imaṃ nagaraṃ pavisanti vā nikkhamanti vā, sabbe te imināva dvārena pavisanti vā nikkhamanti vā’ti. Evameva kho me, bhante, dhammanvayo vidito – ‘ye te, bhante, ahesuṃ atītamaddhānaṃ arahanto sammāsambuddhā , sabbe te bhagavanto pañca nīvaraṇe pahāya cetaso upakkilese paññāya dubbalīkaraṇe catūsu satipaṭṭhānesu supatiṭṭhitacittā sattabojjhaṅge yathābhūtaṃ bhāvetvā anuttaraṃ sammāsambodhiṃ abhisambujjhiṃsu. Yepi te, bhante, bhavissanti anāgatamaddhānaṃ arahanto sammāsambuddhā , sabbe te bhagavanto pañca nīvaraṇe pahāya cetaso upakkilese paññāya dubbalīkaraṇe catūsu satipaṭṭhānesu supatiṭṭhitacittā satta bojjhaṅge yathābhūtaṃ bhāvetvā anuttaraṃ sammāsambodhiṃ abhisambujjhissanti. Bhagavāpi, bhante, etarahi arahaṃ sammāsambuddho pañca nīvaraṇe pahāya cetaso upakkilese paññāya dubbalīkaraṇe catūsu satipaṭṭhānesu supatiṭṭhitacitto satta bojjhaṅge yathābhūtaṃ bhāvetvā anuttaraṃ sammāsambodhiṃ abhisambuddho’’’ti.

\paragraph{147.} Tatrapi sudaṃ bhagavā nāḷandāyaṃ viharanto pāvārikambavane etadeva bahulaṃ bhikkhūnaṃ dhammiṃ kathaṃ karoti – ‘‘iti sīlaṃ, iti samādhi, iti paññā. Sīlaparibhāvito samādhi mahapphalo hoti mahānisaṃso. Samādhiparibhāvitā paññā mahapphalā hoti mahānisaṃsā. Paññāparibhāvitaṃ cittaṃ sammadeva āsavehi vimuccati, seyyathidaṃ – kāmāsavā, bhavāsavā, avijjāsavā’’ti.

\subsubsection{Dussīlaādīnavā}

\paragraph{148.} Atha kho bhagavā nāḷandāyaṃ yathābhirantaṃ viharitvā āyasmantaṃ ānandaṃ āmantesi – ‘‘āyāmānanda, yena pāṭaligāmo tenupasaṅkamissāmā’’ti. ‘‘Evaṃ, bhante’’ti kho āyasmā ānando bhagavato paccassosi . Atha kho bhagavā mahatā bhikkhusaṅghena saddhiṃ yena pāṭaligāmo tadavasari. Assosuṃ kho pāṭaligāmikā upāsakā – ‘‘bhagavā kira pāṭaligāmaṃ anuppatto’’ti. Atha kho pāṭaligāmikā upāsakā yena bhagavā tenupasaṅkamiṃsu; upasaṅkamitvā bhagavantaṃ abhivādetvā ekamantaṃ nisīdiṃsu. Ekamantaṃ nisinnā kho pāṭaligāmikā upāsakā bhagavantaṃ etadavocuṃ – ‘‘adhivāsetu no, bhante, bhagavā āvasathāgāra’’nti. Adhivāsesi bhagavā tuṇhībhāvena. Atha kho pāṭaligāmikā upāsakā bhagavato adhivāsanaṃ viditvā uṭṭhāyāsanā bhagavantaṃ abhivādetvā padakkhiṇaṃ katvā yena āvasathāgāraṃ tenupasaṅkamiṃsu; upasaṅkamitvā sabbasanthariṃ\footnote{sabbasantharitaṃ satthataṃ (syā.), sabbasanthariṃ santhataṃ (ka.)} āvasathāgāraṃ santharitvā āsanāni paññapetvā udakamaṇikaṃ patiṭṭhāpetvā telapadīpaṃ āropetvā yena bhagavā tenupasaṅkamiṃsu, upasaṅkamitvā bhagavantaṃ abhivādetvā ekamantaṃ aṭṭhaṃsu. Ekamantaṃ ṭhitā kho pāṭaligāmikā upāsakā bhagavantaṃ etadavocuṃ – ‘‘sabbasantharisanthataṃ\footnote{sabbasanthariṃ santhataṃ (sī. syā. pī. ka.)}, bhante, āvasathāgāraṃ, āsanāni paññattāni, udakamaṇiko patiṭṭhāpito, telapadīpo āropito; yassadāni, bhante, bhagavā kālaṃ maññatī’’ti. Atha kho bhagavā sāyanhasamayaṃ\footnote{idaṃ padaṃ vinayamahāvagga na dissati}. Nivāsetvā pattacīvaramādāya saddhiṃ bhikkhusaṅghena yena āvasathāgāraṃ tenupasaṅkami; upasaṅkamitvā pāde pakkhāletvā āvasathāgāraṃ pavisitvā majjhimaṃ thambhaṃ nissāya puratthābhimukho\footnote{puratthimābhimukho (ka.)} nisīdi. Bhikkhusaṅghopi kho pāde pakkhāletvā āvasathāgāraṃ pavisitvā pacchimaṃ bhittiṃ nissāya puratthābhimukho nisīdi bhagavantameva purakkhatvā. Pāṭaligāmikāpi kho upāsakā pāde pakkhāletvā āvasathāgāraṃ pavisitvā puratthimaṃ bhittiṃ nissāya pacchimābhimukhā nisīdiṃsu bhagavantameva purakkhatvā.

\paragraph{149.} Atha kho bhagavā pāṭaligāmike upāsake āmantesi – ‘‘pañcime, gahapatayo, ādīnavā dussīlassa sīlavipattiyā. Katame pañca? Idha, gahapatayo, dussīlo sīlavipanno pamādādhikaraṇaṃ mahatiṃ bhogajāniṃ nigacchati. Ayaṃ paṭhamo ādīnavo dussīlassa sīlavipattiyā.

‘‘Puna caparaṃ, gahapatayo, dussīlassa sīlavipannassa pāpako kittisaddo abbhuggacchati. Ayaṃ dutiyo ādīnavo dussīlassa sīlavipattiyā.

‘‘Puna caparaṃ, gahapatayo, dussīlo sīlavipanno yaññadeva parisaṃ upasaṅkamati – yadi khattiyaparisaṃ yadi brāhmaṇaparisaṃ yadi gahapatiparisaṃ yadi samaṇaparisaṃ – avisārado upasaṅkamati maṅkubhūto. Ayaṃ tatiyo ādīnavo dussīlassa sīlavipattiyā.

‘‘Puna caparaṃ, gahapatayo, dussīlo sīlavipanno sammūḷho kālaṅkaroti. Ayaṃ catuttho ādīnavo dussīlassa sīlavipattiyā.

‘‘Puna caparaṃ, gahapatayo, dussīlo sīlavipanno kāyassa bhedā paraṃ maraṇā apāyaṃ duggatiṃ vinipātaṃ nirayaṃ upapajjati. Ayaṃ pañcamo ādīnavo dussīlassa sīlavipattiyā. Ime kho, gahapatayo, pañca ādīnavā dussīlassa sīlavipattiyā.

\subsubsection{Sīlavanttaānisaṃsā}

\paragraph{150.} ‘‘Pañcime , gahapatayo, ānisaṃsā sīlavato sīlasampadāya. Katame pañca? Idha, gahapatayo, sīlavā sīlasampanno appamādādhikaraṇaṃ mahantaṃ bhogakkhandhaṃ adhigacchati. Ayaṃ paṭhamo ānisaṃso sīlavato sīlasampadāya.

‘‘Puna caparaṃ, gahapatayo, sīlavato sīlasampannassa kalyāṇo kittisaddo abbhuggacchati. Ayaṃ dutiyo ānisaṃso sīlavato sīlasampadāya.

‘‘Puna caparaṃ, gahapatayo, sīlavā sīlasampanno yaññadeva parisaṃ upasaṅkamati – yadi khattiyaparisaṃ yadi brāhmaṇaparisaṃ yadi gahapatiparisaṃ yadi samaṇaparisaṃ visārado upasaṅkamati amaṅkubhūto. Ayaṃ tatiyo ānisaṃso sīlavato sīlasampadāya.

‘‘Puna caparaṃ, gahapatayo, sīlavā sīlasampanno asammūḷho kālaṅkaroti. Ayaṃ catuttho ānisaṃso sīlavato sīlasampadāya.

‘‘Puna caparaṃ, gahapatayo, sīlavā sīlasampanno kāyassa bhedā paraṃ maraṇā sugatiṃ saggaṃ lokaṃ upapajjati. Ayaṃ pañcamo ānisaṃso sīlavato sīlasampadāya. Ime kho, gahapatayo, pañca ānisaṃsā sīlavato sīlasampadāyā’’ti.

\paragraph{151.} Atha kho bhagavā pāṭaligāmike upāsake bahudeva rattiṃ dhammiyā kathāya sandassetvā samādapetvā samuttejetvā sampahaṃsetvā uyyojesi – ‘‘abhikkantā kho, gahapatayo, ratti, yassadāni tumhe kālaṃ maññathā’’ti. ‘‘Evaṃ, bhante’’ti kho pāṭaligāmikā upāsakā bhagavato paṭissutvā uṭṭhāyāsanā bhagavantaṃ abhivādetvā padakkhiṇaṃ katvā pakkamiṃsu. Atha kho bhagavā acirapakkantesu pāṭaligāmikesu upāsakesu suññāgāraṃ pāvisi.

\subsubsection{Pāṭaliputtanagaramāpanaṃ}

\paragraph{152.} Tena kho pana samayena sunidhavassakārā\footnote{sunīdhavassakārā (syā. ka.)} magadhamahāmattā pāṭaligāme nagaraṃ māpenti vajjīnaṃ paṭibāhāya. Tena samayena sambahulā devatāyo sahasseva\footnote{sahassasseva (sī. pī. ka.), sahassaseva (ṭīkāyaṃ pāṭhantaraṃ), sahassasahasseva (udānaṭṭhakathā)} pāṭaligāme vatthūni pariggaṇhanti. Yasmiṃ padese mahesakkhā devatā vatthūni pariggaṇhanti, mahesakkhānaṃ tattha raññaṃ rājamahāmattānaṃ cittāni namanti nivesanāni māpetuṃ. Yasmiṃ padese majjhimā devatā vatthūni pariggaṇhanti, majjhimānaṃ tattha raññaṃ rājamahāmattānaṃ cittāni namanti nivesanāni māpetuṃ. Yasmiṃ padese nīcā devatā vatthūni pariggaṇhanti, nīcānaṃ tattha raññaṃ rājamahāmattānaṃ cittāni namanti nivesanāni māpetuṃ. Addasā kho bhagavā dibbena cakkhunā visuddhena atikkantamānusakena tā devatāyo sahasseva pāṭaligāme vatthūni pariggaṇhantiyo. Atha kho bhagavā rattiyā paccūsasamayaṃ paccuṭṭhāya āyasmantaṃ ānandaṃ āmantesi – ‘‘ke nu kho\footnote{ko nu kho (sī. syā. pī. ka.)}, ānanda, pāṭaligāme nagaraṃ māpentī’’ti\footnote{māpetīti (sī. syā. pī. ka.)}? ‘‘Sunidhavassakārā, bhante, magadhamahāmattā pāṭaligāme nagaraṃ māpenti vajjīnaṃ paṭibāhāyā’’ti. ‘‘Seyyathāpi, ānanda, devehi tāvatiṃsehi saddhiṃ mantetvā, evameva kho, ānanda, sunidhavassakārā magadhamahāmattā pāṭaligāme nagaraṃ māpenti vajjīnaṃ paṭibāhāya. Idhāhaṃ, ānanda, addasaṃ dibbena cakkhunā visuddhena atikkantamānusakena sambahulā devatāyo sahasseva pāṭaligāme vatthūni pariggaṇhantiyo. Yasmiṃ , ānanda, padese mahesakkhā devatā vatthūni pariggaṇhanti, mahesakkhānaṃ tattha raññaṃ rājamahāmattānaṃ cittāni namanti nivesanāni māpetuṃ. Yasmiṃ padese majjhimā devatā vatthūni pariggaṇhanti, majjhimānaṃ tattha raññaṃ rājamahāmattānaṃ cittāni namanti nivesanāni māpetuṃ. Yasmiṃ padese nīcā devatā vatthūni pariggaṇhanti, nīcānaṃ tattha raññaṃ rājamahāmattānaṃ cittāni namanti nivesanāni māpetuṃ. Yāvatā, ānanda, ariyaṃ āyatanaṃ yāvatā vaṇippatho idaṃ agganagaraṃ bhavissati pāṭaliputtaṃ puṭabhedanaṃ . Pāṭaliputtassa kho, ānanda, tayo antarāyā bhavissanti – aggito vā udakato vā mithubhedā vā’’ti.

\paragraph{153.} Atha kho sunidhavassakārā magadhamahāmattā yena bhagavā tenupasaṅkamiṃsu; upasaṅkamitvā bhagavatā saddhiṃ sammodiṃsu, sammodanīyaṃ kathaṃ sāraṇīyaṃ vītisāretvā ekamantaṃ aṭṭhaṃsu, ekamantaṃ ṭhitā kho sunidhavassakārā magadhamahāmattā bhagavantaṃ etadavocuṃ – ‘‘adhivāsetu no bhavaṃ gotamo ajjatanāya bhattaṃ saddhiṃ bhikkhusaṅghenā’’ti. Adhivāsesi bhagavā tuṇhībhāvena. Atha kho sunidhavassakārā magadhamahāmattā bhagavato adhivāsanaṃ viditvā yena sako āvasatho tenupasaṅkamiṃsu; upasaṅkamitvā sake āvasathe paṇītaṃ khādanīyaṃ bhojanīyaṃ paṭiyādāpetvā bhagavato kālaṃ ārocāpesuṃ – ‘‘kālo, bho gotama, niṭṭhitaṃ bhatta’’nti.

Atha kho bhagavā pubbaṇhasamayaṃ nivāsetvā pattacīvaramādāya saddhiṃ bhikkhusaṅghena yena sunidhavassakārānaṃ magadhamahāmattānaṃ āvasatho tenupasaṅkami; upasaṅkamitvā paññatte āsane nisīdi. Atha kho sunidhavassakārā magadhamahāmattā buddhappamukhaṃ bhikkhusaṅghaṃ paṇītena khādanīyena bhojanīyena sahatthā santappesuṃ sampavāresuṃ. Atha kho sunidhavassakārā magadhamahāmattā bhagavantaṃ bhuttāviṃ onītapattapāṇiṃ aññataraṃ nīcaṃ āsanaṃ gahetvā ekamantaṃ nisīdiṃsu. Ekamantaṃ nisinne kho sunidhavassakāre magadhamahāmatte bhagavā imāhi gāthāhi anumodi –

‘‘Yasmiṃ padese kappeti, vāsaṃ paṇḍitajātiyo;

Sīlavantettha bhojetvā, saññate brahmacārayo\footnote{brahmacārino (syā.)}.

‘‘Yā tattha devatā āsuṃ, tāsaṃ dakkhiṇamādise;

Tā pūjitā pūjayanti\footnote{pūjitā pūjayanti naṃ (ka.)}, mānitā mānayanti naṃ.

‘‘Tato naṃ anukampanti, mātā puttaṃva orasaṃ;

Devatānukampito poso, sadā bhadrāni passatī’’ti.

Atha kho bhagavā sunidhavassakāre magadhamahāmatte imāhi gāthāhi anumoditvā uṭṭhāyāsanā pakkāmi.

\paragraph{154.} Tena kho pana samayena sunidhavassakārā magadhamahāmattā bhagavantaṃ piṭṭhito piṭṭhito anubandhā honti – ‘‘yenajja samaṇo gotamo dvārena nikkhamissati, taṃ gotamadvāraṃ nāma bhavissati. Yena titthena gaṅgaṃ nadiṃ tarissati, taṃ gotamatitthaṃ nāma bhavissatī’’ti. Atha kho bhagavā yena dvārena nikkhami , taṃ gotamadvāraṃ nāma ahosi. Atha kho bhagavā yena gaṅgā nadī tenupasaṅkami. Tena kho pana samayena gaṅgā nadī pūrā hoti samatittikā kākapeyyā. Appekacce manussā nāvaṃ pariyesanti, appekacce uḷumpaṃ pariyesanti, appekacce kullaṃ bandhanti apārā\footnote{pārā (sī. syā. ka.), orā (vi. mahāvagga)}, pāraṃ gantukāmā. Atha kho bhagavā – seyyathāpi nāma balavā puriso samiñjitaṃ vā bāhaṃ pasāreyya, pasāritaṃ vā bāhaṃ samiñjeyya, evameva – gaṅgāya nadiyā orimatīre antarahito pārimatīre paccuṭṭhāsi saddhiṃ bhikkhusaṅghena. Addasā kho bhagavā te manusse appekacce nāvaṃ pariyesante appekacce uḷumpaṃ pariyesante appekacce kullaṃ bandhante apārā pāraṃ gantukāme. Atha kho bhagavā etamatthaṃ viditvā tāyaṃ velāyaṃ imaṃ udānaṃ udānesi –

‘‘Ye taranti aṇṇavaṃ saraṃ, setuṃ katvāna visajja pallalāni;

Kullañhi jano bandhati\footnote{kullaṃ jano ca bandhati (syā.), kullaṃ hi jano pabandhati (sī. pī. ka.)}, tiṇṇā\footnote{nitiṇṇā, na tiṇṇā (ka.)} medhāvino janā’’ti.

\xsubsubsectionEnd{Paṭhamabhāṇavāro.}

\subsubsection{Ariyasaccakathā}

\paragraph{155.} Atha kho bhagavā āyasmantaṃ ānandaṃ āmantesi – ‘‘āyāmānanda, yena koṭigāmo tenupasaṅkamissāmā’’ti. ‘‘Evaṃ, bhante’’ti kho āyasmā ānando bhagavato paccassosi. Atha kho bhagavā mahatā bhikkhusaṅghena saddhiṃ yena koṭigāmo tadavasari. Tatra sudaṃ bhagavā koṭigāme viharati. Tatra kho bhagavā bhikkhū āmantesi –

‘‘Catunnaṃ , bhikkhave, ariyasaccānaṃ ananubodhā appaṭivedhā evamidaṃ dīghamaddhānaṃ sandhāvitaṃ saṃsaritaṃ mamañceva tumhākañca. Katamesaṃ catunnaṃ? Dukkhassa, bhikkhave, ariyasaccassa ananubodhā appaṭivedhā evamidaṃ dīghamaddhānaṃ sandhāvitaṃ saṃsaritaṃ mamañceva tumhākañca. Dukkhasamudayassa, bhikkhave, ariyasaccassa ananubodhā appaṭivedhā evamidaṃ dīghamaddhānaṃ sandhāvitaṃ saṃsaritaṃ mamañceva tumhākañca. Dukkhanirodhassa, bhikkhave, ariyasaccassa ananubodhā appaṭivedhā evamidaṃ dīghamaddhānaṃ sandhāvitaṃ saṃsaritaṃ mamañceva tumhākañca. Dukkhanirodhagāminiyā paṭipadāya, bhikkhave, ariyasaccassa ananubodhā appaṭivedhā evamidaṃ dīghamaddhānaṃ sandhāvitaṃ saṃsaritaṃ mamañceva tumhākañca. Tayidaṃ, bhikkhave, dukkhaṃ ariyasaccaṃ anubuddhaṃ paṭividdhaṃ, dukkhasamudayaṃ\footnote{dukkhasamudayo (syā.)} ariyasaccaṃ anubuddhaṃ paṭividdhaṃ, dukkhanirodhaṃ\footnote{dukkhanirodho (syā.)} ariyasaccaṃ anubuddhaṃ paṭividdhaṃ, dukkhanirodhagāminī paṭipadā ariyasaccaṃ anubuddhaṃ paṭividdhaṃ, ucchinnā bhavataṇhā, khīṇā bhavanetti, natthidāni punabbhavo’’ti. Idamavoca bhagavā. Idaṃ vatvāna sugato athāparaṃ etadavoca satthā –

‘‘Catunnaṃ ariyasaccānaṃ, yathābhūtaṃ adassanā;

Saṃsitaṃ dīghamaddhānaṃ, tāsu tāsveva jātisu.

Tāni etāni diṭṭhāni, bhavanetti samūhatā;

Ucchinnaṃ mūlaṃ dukkhassa, natthi dāni punabbhavo’’ti.

Tatrapi sudaṃ bhagavā koṭigāme viharanto etadeva bahulaṃ bhikkhūnaṃ dhammiṃ kathaṃ karoti – ‘‘iti sīlaṃ, iti samādhi, iti paññā. Sīlaparibhāvito samādhi mahapphalo hoti mahānisaṃso. Samādhiparibhāvitā paññā mahapphalā hoti mahānisaṃsā. Paññāparibhāvitaṃ cittaṃ sammadeva āsavehi vimuccati, seyyathidaṃ – kāmāsavā, bhavāsavā, avijjāsavā’’ti.

\subsubsection{Anāvattidhammasambodhiparāyaṇā}

\paragraph{156.} Atha kho bhagavā koṭigāme yathābhirantaṃ viharitvā āyasmantaṃ ānandaṃ āmantesi – ‘‘āyāmānanda, yena nātikā\footnote{nādikā (syā. pī.)} tenupaṅkamissāmā’’ti. ‘‘Evaṃ, bhante’’ti kho āyasmā ānando bhagavato paccassosi. Atha kho bhagavā mahatā bhikkhusaṅghena saddhiṃ yena nātikā tadavasari. Tatrapi sudaṃ bhagavā nātike viharati giñjakāvasathe. Atha kho āyasmā ānando yena bhagavā tenupasaṅkami; upasaṅkamitvā bhagavantaṃ abhivādetvā ekamantaṃ nisīdi. Ekamantaṃ nisinno kho āyasmā ānando bhagavantaṃ etadavoca – ‘‘sāḷho nāma, bhante, bhikkhu nātike kālaṅkato, tassa kā gati, ko abhisamparāyo? Nandā nāma, bhante, bhikkhunī nātike kālaṅkatā, tassā kā gati, ko abhisamparāyo? Sudatto nāma, bhante, upāsako nātike kālaṅkato, tassa kā gati, ko abhisamparāyo? Sujātā nāma, bhante, upāsikā nātike kālaṅkatā, tassā kā gati , ko abhisamparāyo? Kukkuṭo\footnote{kakudho (syā.)} nāma, bhante, upāsako nātike kālaṅkato, tassa kā gati, ko abhisamparāyo? Kāḷimbo\footnote{kāliṅgo (pī.), kāraḷimbo (syā.)} nāma, bhante, upāsako…pe… nikaṭo nāma, bhante, upāsako… kaṭissaho\footnote{kaṭissabho (sī. pī.)} nāma, bhante, upāsako… tuṭṭho nāma, bhante, upāsako… santuṭṭho nāma, bhante, upāsako… bhaddo\footnote{bhaṭo (syā.)} nāma, bhante, upāsako… subhaddo\footnote{subhaṭo (syā.)} nāma, bhante, upāsako nātike kālaṅkato, tassa kā gati, ko abhisamparāyo’’ti?

\paragraph{157.} ‘‘Sāḷho, ānanda, bhikkhu āsavānaṃ khayā anāsavaṃ cetovimuttiṃ paññāvimuttiṃ diṭṭheva dhamme sayaṃ abhiññā sacchikatvā upasampajja vihāsi. Nandā, ānanda, bhikkhunī pañcannaṃ orambhāgiyānaṃ saṃyojanānaṃ parikkhayā opapātikā tattha parinibbāyinī anāvattidhammā tasmā lokā. Sudatto, ānanda, upāsako tiṇṇaṃ saṃyojanānaṃ parikkhayā rāgadosamohānaṃ tanuttā sakadāgāmī sakideva imaṃ lokaṃ āgantvā dukkhassantaṃ karissati. Sujātā, ānanda, upāsikā tiṇṇaṃ saṃyojanānaṃ parikkhayā sotāpannā avinipātadhammā niyatā sambodhiparāyaṇā\footnote{parāyanā (sī. syā. pī. ka.)}. Kukkuṭo, ānanda, upāsako pañcannaṃ orambhāgiyānaṃ saṃyojanānaṃ parikkhayā opapātiko tattha parinibbāyī anāvattidhammo tasmā lokā. Kāḷimbo, ānanda, upāsako…pe… nikaṭo, ānanda, upāsako… kaṭissaho , ānanda, upāsako… tuṭṭho, ānanda, upāsako … santuṭṭho, ānanda, upāsako… bhaddo, ānanda, upāsako… subhaddo, ānanda, upāsako pañcannaṃ orambhāgiyānaṃ saṃyojanānaṃ parikkhayā opapātiko tattha parinibbāyī anāvattidhammo tasmā lokā . Paropaññāsaṃ, ānanda, nātike upāsakā kālaṅkatā, pañcannaṃ orambhāgiyānaṃ saṃyojanānaṃ parikkhayā opapātikā tattha parinibbāyino anāvattidhammā tasmā lokā. Sādhikā navuti\footnote{chādhikā navuti (syā.)}, ānanda, nātike upāsakā kālaṅkatā tiṇṇaṃ saṃyojanānaṃ parikkhayā rāgadosamohānaṃ tanuttā sakadāgāmino sakideva imaṃ lokaṃ āgantvā dukkhassantaṃ karissanti. Sātirekāni\footnote{dasātirekāni (syā.)}, ānanda, pañcasatāni nātike upāsakā kālaṅkatā, tiṇṇaṃ saṃyojanānaṃ parikkhayā sotāpannā avinipātadhammā niyatā sambodhiparāyaṇā.

\subsubsection{Dhammādāsadhammapariyāyā}

\paragraph{158.} ‘‘Anacchariyaṃ kho panetaṃ, ānanda, yaṃ manussabhūto kālaṅkareyya. Tasmiṃyeva\footnote{tasmiṃ tasmiṃ ce (sī. pī.), tasmiṃ tasmiṃ kho (syā.)} kālaṅkate tathāgataṃ upasaṅkamitvā etamatthaṃ pucchissatha, vihesā hesā, ānanda, tathāgatassa. Tasmātihānanda, dhammādāsaṃ nāma dhammapariyāyaṃ desessāmi, yena samannāgato ariyasāvako ākaṅkhamāno attanāva attānaṃ byākareyya – ‘khīṇanirayomhi khīṇatiracchānayoni khīṇapettivisayo khīṇāpāyaduggativinipāto, sotāpannohamasmi avinipātadhammo niyato sambodhiparāyaṇo’ti.

\paragraph{159.} ‘‘Katamo ca so, ānanda, dhammādāso dhammapariyāyo, yena samannāgato ariyasāvako ākaṅkhamāno attanāva attānaṃ byākareyya – ‘khīṇanirayomhi khīṇatiracchānayoni khīṇapettivisayo khīṇāpāyaduggativinipāto, sotāpannohamasmi avinipātadhammo niyato sambodhiparāyaṇo’ti?

‘‘Idhānanda , ariyasāvako buddhe aveccappasādena samannāgato hoti – ‘itipi so bhagavā arahaṃ sammāsambuddho vijjācaraṇasampanno sugato lokavidū anuttaro purisadammasārathi satthā devamanussānaṃ buddho bhagavā’ti.

‘‘Dhamme aveccappasādena samannāgato hoti – ‘svākkhāto bhagavatā dhammo sandiṭṭhiko akāliko ehipassiko opaneyyiko paccattaṃ veditabbo viññūhī’ti.

‘‘Saṅghe aveccappasādena samannāgato hoti – ‘suppaṭipanno bhagavato sāvakasaṅgho, ujuppaṭipanno bhagavato sāvakasaṅgho, ñāyappaṭipanno bhagavato sāvakasaṅgho, sāmīcippaṭipanno bhagavato sāvakasaṅgho yadidaṃ cattāri purisayugāni aṭṭha purisapuggalā, esa bhagavato sāvakasaṅgho āhuneyyo pāhuneyyo dakkhiṇeyyo añjalikaraṇīyo anuttaraṃ puññakkhettaṃ lokassā’ti.

‘‘Ariyakantehi sīlehi samannāgato hoti akhaṇḍehi acchiddehi asabalehi akammāsehi bhujissehi viññūpasatthehi aparāmaṭṭhehi samādhisaṃvattanikehi.

‘‘Ayaṃ kho so, ānanda, dhammādāso dhammapariyāyo, yena samannāgato ariyasāvako ākaṅkhamāno attanāva attānaṃ byākareyya – ‘khīṇanirayomhi khīṇatiracchānayoni khīṇapettivisayo khīṇāpāyaduggativinipāto, sotāpannohamasmi avinipātadhammo niyato sambodhiparāyaṇo’’’ti.

Tatrapi sudaṃ bhagavā nātike viharanto giñjakāvasathe etadeva bahulaṃ bhikkhūnaṃ dhammiṃ kathaṃ karoti –

‘‘Iti sīlaṃ iti samādhi iti paññā. Sīlaparibhāvito samādhi mahapphalo hoti mahānisaṃso. Samādhiparibhāvitā paññā mahapphalā hoti mahānisaṃsā. Paññāparibhāvitaṃ cittaṃ sammadeva āsavehi vimuccati, seyyathidaṃ – kāmāsavā, bhavāsavā, avijjāsavā’’ti.

\paragraph{160.} Atha kho bhagavā nātike yathābhirantaṃ viharitvā āyasmantaṃ ānandaṃ āmantesi – ‘‘āyāmānanda, yena vesālī tenupasaṅkamissāmā’’ti. ‘‘Evaṃ, bhante’’ti kho āyasmā ānando bhagavato paccassosi. Atha kho bhagavā mahatā bhikkhusaṅghena saddhiṃ yena vesālī tadavasari. Tatra sudaṃ bhagavā vesāliyaṃ viharati ambapālivane. Tatra kho bhagavā bhikkhū āmantesi –

‘‘Sato, bhikkhave, bhikkhu vihareyya sampajāno, ayaṃ vo amhākaṃ anusāsanī. Kathañca, bhikkhave, bhikkhu sato hoti? Idha, bhikkhave, bhikkhu kāye kāyānupassī viharati ātāpī sampajāno satimā vineyya loke abhijjhādomanassaṃ. Vedanāsu vedanānupassī…pe… citte cittānupassī…pe… dhammesu dhammānupassī viharati ātāpī sampajāno satimā vineyya loke abhijjhādomanassaṃ. Evaṃ kho, bhikkhave, bhikkhu sato hoti.

‘‘Kathañca , bhikkhave, bhikkhu sampajāno hoti? Idha, bhikkhave, bhikkhu abhikkante paṭikkante sampajānakārī hoti, ālokite vilokite sampajānakārī hoti, samiñjite pasārite sampajānakārī hoti, saṅghāṭipattacīvaradhāraṇe sampajānakārī hoti, asite pīte khāyite sāyite sampajānakārī hoti, uccārapassāvakamme sampajānakārī hoti, gate ṭhite nisinne sutte jāgarite bhāsite tuṇhībhāve sampajānakārī hoti. Evaṃ kho, bhikkhave, bhikkhu sampajāno hoti. Sato, bhikkhave, bhikkhu vihareyya sampajāno, ayaṃ vo amhākaṃ anusāsanī’’ti.

\subsubsection{Ambapālīgaṇikā}

\paragraph{161.} Assosi kho ambapālī gaṇikā – ‘‘bhagavā kira vesāliṃ anuppatto vesāliyaṃ viharati mayhaṃ ambavane’’ti. Atha kho ambapālī gaṇikā bhaddāni bhaddāni yānāni yojāpetvā bhaddaṃ bhaddaṃ yānaṃ abhiruhitvā bhaddehi bhaddehi yānehi vesāliyā niyyāsi. Yena sako ārāmo tena pāyāsi. Yāvatikā yānassa bhūmi, yānena gantvā, yānā paccorohitvā pattikāva yena bhagavā tenupasaṅkami; upasaṅkamitvā bhagavantaṃ abhivādetvā ekamantaṃ nisīdi. Ekamantaṃ nisinnaṃ kho ambapāliṃ gaṇikaṃ bhagavā dhammiyā kathāya sandassesi samādapesi samuttejesi sampahaṃsesi. Atha kho ambapālī gaṇikā bhagavatā dhammiyā kathāya sandassitā samādapitā samuttejitā sampahaṃsitā bhagavantaṃ etadavoca – ‘‘adhivāsetu me, bhante, bhagavā svātanāya bhattaṃ saddhiṃ bhikkhusaṅghenā’’ti. Adhivāsesi bhagavā tuṇhībhāvena. Atha kho ambapālī gaṇikā bhagavato adhivāsanaṃ viditvā uṭṭhāyāsanā bhagavantaṃ abhivādetvā padakkhiṇaṃ katvā pakkāmi.

Assosuṃ kho vesālikā licchavī – ‘‘bhagavā kira vesāliṃ anuppatto vesāliyaṃ viharati ambapālivane’’ti. Atha kho te licchavī bhaddāni bhaddāni yānāni yojāpetvā bhaddaṃ bhaddaṃ yānaṃ abhiruhitvā bhaddehi bhaddehi yānehi vesāliyā niyyiṃsu. Tatra ekacce licchavī nīlā honti nīlavaṇṇā nīlavatthā nīlālaṅkārā, ekacce licchavī pītā honti pītavaṇṇā pītavatthā pītālaṅkārā, ekacce licchavī lohitā honti lohitavaṇṇā lohitavatthā lohitālaṅkārā, ekacce licchavī odātā honti odātavaṇṇā odātavatthā odātālaṅkārā. Atha kho ambapālī gaṇikā daharānaṃ daharānaṃ licchavīnaṃ akkhena akkhaṃ cakkena cakkaṃ yugena yugaṃ paṭivaṭṭesi\footnote{parivattesi (vi. mahāvagga)}. Atha kho te licchavī ambapāliṃ gaṇikaṃ etadavocuṃ – ‘‘kiṃ, je ambapāli , daharānaṃ daharānaṃ licchavīnaṃ akkhena akkhaṃ cakkena cakkaṃ yugena yugaṃ paṭivaṭṭesī’’ti? ‘‘Tathā hi pana me, ayyaputtā, bhagavā nimantito svātanāya bhattaṃ saddhiṃ bhikkhusaṅghenā’’ti. ‘‘Dehi, je ambapāli, etaṃ\footnote{ekaṃ (ka.)} bhattaṃ satasahassenā’’ti. ‘‘Sacepi me, ayyaputtā, vesāliṃ sāhāraṃ dassatha\footnote{dajjeyyātha (vi. mahāvagga)}, evamahaṃ taṃ\footnote{evampi mahantaṃ (syā.), evaṃ mahantaṃ (sī. pī.)} bhattaṃ na dassāmī’’ti\footnote{neva dajjāhaṃ taṃ bhattanti (vi. mahāvagga)}. Atha kho te licchavī aṅguliṃ phoṭesuṃ – ‘‘jitamha\footnote{jitamhā (bahūsu)} vata bho ambakāya, jitamha vata bho ambakāyā’’ti\footnote{‘‘jitamhā vata bho ambapālikāya vañcitamhā vata bho ambapālikāyā’’ti (syā.)}.

Atha kho te licchavī yena ambapālivanaṃ tena pāyiṃsu. Addasā kho bhagavā te licchavī dūratova āgacchante. Disvāna bhikkhū āmantesi – ‘‘yesaṃ\footnote{yehi (vi. mahāvagga)}, bhikkhave, bhikkhūnaṃ devā tāvatiṃsā adiṭṭhapubbā, oloketha, bhikkhave, licchaviparisaṃ; apaloketha, bhikkhave , licchaviparisaṃ; upasaṃharatha, bhikkhave, licchaviparisaṃ – tāvatiṃsasadisa’’nti. Atha kho te licchavī yāvatikā yānassa bhūmi, yānena gantvā, yānā paccorohitvā pattikāva yena bhagavā tenupasaṅkamiṃsu; upasaṅkamitvā bhagavantaṃ abhivādetvā ekamantaṃ nisīdiṃsu. Ekamantaṃ nisinne kho te licchavī bhagavā dhammiyā kathāya sandassesi samādapesi samuttejesi sampahaṃsesi. Atha kho te licchavī bhagavatā dhammiyā kathāya sandassitā samādapitā samuttejitā sampahaṃsitā bhagavantaṃ etadavocuṃ – ‘‘adhivāsetu no, bhante, bhagavā svātanāya bhattaṃ saddhiṃ bhikkhusaṅghenā’’ti. Atha kho bhagavā te licchavī etadavoca – ‘‘adhivutthaṃ\footnote{adhivāsitaṃ (syā.)} kho me, licchavī, svātanāya ambapāliyā gaṇikāya bhatta’’nti. Atha kho te licchavī aṅguliṃ phoṭesuṃ – ‘‘jitamha vata bho ambakāya, jitamha vata bho ambakāyā’’ti. Atha kho te licchavī bhagavato bhāsitaṃ abhinanditvā anumoditvā uṭṭhāyāsanā bhagavantaṃ abhivādetvā padakkhiṇaṃ katvā pakkamiṃsu.

\paragraph{162.} Atha kho ambapālī gaṇikā tassā rattiyā accayena sake ārāme paṇītaṃ khādanīyaṃ bhojanīyaṃ paṭiyādāpetvā bhagavato kālaṃ ārocāpesi – ‘‘kālo, bhante, niṭṭhitaṃ bhatta’’nti. Atha kho bhagavā pubbaṇhasamayaṃ nivāsetvā pattacīvaramādāya saddhiṃ bhikkhusaṅghena yena ambapāliyā gaṇikāya nivesanaṃ tenupasaṅkami; upasaṅkamitvā paññatte āsane nisīdi. Atha kho ambapālī gaṇikā buddhappamukhaṃ bhikkhusaṅghaṃ paṇītena khādanīyena bhojanīyena sahatthā santappesi sampavāresi. Atha kho ambapālī gaṇikā bhagavantaṃ bhuttāviṃ onītapattapāṇiṃ aññataraṃ nīcaṃ āsanaṃ gahetvā ekamantaṃ nisīdi. Ekamantaṃ nisinnā kho ambapālī gaṇikā bhagavantaṃ etadavoca – ‘‘imāhaṃ, bhante, ārāmaṃ buddhappamukhassa bhikkhusaṅghassa dammī’’ti. Paṭiggahesi bhagavā ārāmaṃ. Atha kho bhagavā ambapāliṃ gaṇikaṃ dhammiyā kathāya sandassetvā samādapetvā samuttejetvā sampahaṃsetvā uṭṭhāyāsanā pakkāmi. Tatrapi sudaṃ bhagavā vesāliyaṃ viharanto ambapālivane etadeva bahulaṃ bhikkhūnaṃ dhammiṃ kathaṃ karoti – ‘‘iti sīlaṃ, iti samādhi, iti paññā. Sīlaparibhāvito samādhi mahapphalo hoti mahānisaṃso. Samādhiparibhāvitā paññā mahapphalā hoti mahānisaṃsā. Paññāparibhāvitaṃ cittaṃ sammadeva āsavehi vimuccati, seyyathidaṃ – kāmāsavā, bhavāsavā, avijjāsavā’’ti.

\subsubsection{Veḷuvagāmavassūpagamanaṃ}

\paragraph{163.} Atha kho bhagavā ambapālivane yathābhirantaṃ viharitvā āyasmantaṃ ānandaṃ āmantesi – ‘‘āyāmānanda, yena veḷuvagāmako\footnote{beḷuvagāmako (sī. pī.)} tenupasaṅkamissāmā’’ti. ‘‘Evaṃ, bhante’’ti kho āyasmā ānando bhagavato paccassosi. Atha kho bhagavā mahatā bhikkhusaṅghena saddhiṃ yena veḷuvagāmako tadavasari. Tatra sudaṃ bhagavā veḷuvagāmake viharati. Tatra kho bhagavā bhikkhū āmantesi – ‘‘etha tumhe, bhikkhave, samantā vesāliṃ yathāmittaṃ yathāsandiṭṭhaṃ yathāsambhattaṃ vassaṃ upetha\footnote{upagacchatha (syā.)}. Ahaṃ pana idheva veḷuvagāmake vassaṃ upagacchāmī’’ti. ‘‘Evaṃ, bhante’’ti kho te bhikkhū bhagavato paṭissutvā samantā vesāliṃ yathāmittaṃ yathāsandiṭṭhaṃ yathāsambhattaṃ vassaṃ upagacchiṃsu. Bhagavā pana tattheva veḷuvagāmake vassaṃ upagacchi.

\paragraph{164.} Atha kho bhagavato vassūpagatassa kharo ābādho uppajji, bāḷhā vedanā vattanti māraṇantikā. Tā sudaṃ bhagavā sato sampajāno adhivāsesi avihaññamāno. Atha kho bhagavato etadahosi – ‘‘na kho metaṃ patirūpaṃ, yvāhaṃ anāmantetvā upaṭṭhāke anapaloketvā bhikkhusaṅghaṃ parinibbāyeyyaṃ. Yaṃnūnāhaṃ imaṃ ābādhaṃ vīriyena paṭipaṇāmetvā jīvitasaṅkhāraṃ adhiṭṭhāya vihareyya’’nti. Atha kho bhagavā taṃ ābādhaṃ vīriyena paṭipaṇāmetvā jīvitasaṅkhāraṃ adhiṭṭhāya vihāsi. Atha kho bhagavato so ābādho paṭipassambhi. Atha kho bhagavā gilānā vuṭṭhito\footnote{gilānavuṭṭhito (saddanīti)} aciravuṭṭhito gelaññā vihārā nikkhamma vihārapacchāyāyaṃ paññatte āsane nisīdi. Atha kho āyasmā ānando yena bhagavā tenupasaṅkami; upasaṅkamitvā bhagavantaṃ abhivādetvā ekamantaṃ nisīdi. Ekamantaṃ nisinno kho āyasmā ānando bhagavantaṃ etadavoca – ‘‘diṭṭho me, bhante, bhagavato phāsu; diṭṭhaṃ me, bhante, bhagavato khamanīyaṃ, api ca me, bhante, madhurakajāto viya kāyo. Disāpi me na pakkhāyanti; dhammāpi maṃ na paṭibhanti bhagavato gelaññena, api ca me, bhante, ahosi kācideva assāsamattā – ‘na tāva bhagavā parinibbāyissati, na yāva bhagavā bhikkhusaṅghaṃ ārabbha kiñcideva udāharatī’’’ti.

\paragraph{165.} ‘‘Kiṃ panānanda, bhikkhusaṅgho mayi paccāsīsati\footnote{paccāsiṃsati (sī. syā.)}? Desito, ānanda, mayā dhammo anantaraṃ abāhiraṃ karitvā. Natthānanda, tathāgatassa dhammesu ācariyamuṭṭhi. Yassa nūna, ānanda, evamassa – ‘ahaṃ bhikkhusaṅghaṃ pariharissāmī’ti vā ‘mamuddesiko bhikkhusaṅgho’ti vā, so nūna, ānanda, bhikkhusaṅghaṃ ārabbha kiñcideva udāhareyya. Tathāgatassa kho, ānanda, na evaṃ hoti – ‘ahaṃ bhikkhusaṅghaṃ pariharissāmī’ti vā ‘mamuddesiko bhikkhusaṅgho’ti vā. Sakiṃ\footnote{kiṃ (sī. pī.)}, ānanda, tathāgato bhikkhusaṅghaṃ ārabbha kiñcideva udāharissati. Ahaṃ kho panānanda, etarahi jiṇṇo vuddho mahallako addhagato vayoanuppatto. Āsītiko me vayo vattati. Seyyathāpi, ānanda, jajjarasakaṭaṃ veṭhamissakena\footnote{veḷumissakena (syā.), veghamissakena (pī.), vedhamissakena, vekhamissakena (ka.)} yāpeti, evameva kho, ānanda, veṭhamissakena maññe tathāgatassa kāyo yāpeti. Yasmiṃ, ānanda, samaye tathāgato sabbanimittānaṃ amanasikārā ekaccānaṃ vedanānaṃ nirodhā animittaṃ cetosamādhiṃ upasampajja viharati, phāsutaro, ānanda, tasmiṃ samaye tathāgatassa kāyo hoti. Tasmātihānanda, attadīpā viharatha attasaraṇā anaññasaraṇā, dhammadīpā dhammasaraṇā anaññasaraṇā. Kathañcānanda, bhikkhu attadīpo viharati attasaraṇo anaññasaraṇo, dhammadīpo dhammasaraṇo anaññasaraṇo? Idhānanda, bhikkhu kāye kāyānupassī viharati atāpī sampajāno satimā, vineyya loke abhijjhādomanassaṃ. Vedanāsu…pe… citte…pe… dhammesu dhammānupassī viharati ātāpī sampajāno satimā, vineyya loke abhijjhādomanassaṃ. Evaṃ kho, ānanda, bhikkhu attadīpo viharati attasaraṇo anaññasaraṇo, dhammadīpo dhammasaraṇo anaññasaraṇo . Ye hi keci, ānanda, etarahi vā mama vā accayena attadīpā viharissanti attasaraṇā anaññasaraṇā, dhammadīpā dhammasaraṇā anaññasaraṇā, tamatagge me te, ānanda, bhikkhū bhavissanti ye keci sikkhākāmā’’ti.

\xsubsubsectionEnd{Dutiyabhāṇavāro.}

\subsubsection{Nimittobhāsakathā}

\paragraph{166.} Atha kho bhagavā pubbaṇhasamayaṃ nivāsetvā pattacīvaramādāya vesāliṃ piṇḍāya pāvisi. Vesāliyaṃ piṇḍāya caritvā pacchābhattaṃ piṇḍapātapaṭikkanto āyasmantaṃ ānandaṃ āmantesi – ‘‘gaṇhāhi, ānanda, nisīdanaṃ, yena cāpālaṃ cetiyaṃ\footnote{pāvālaṃ (cetiyaṃ (syā.)} tenupasaṅkamissāma divā vihārāyā’’ti. ‘‘Evaṃ, bhante’’ti kho āyasmā ānando bhagavato paṭissutvā nisīdanaṃ ādāya bhagavantaṃ piṭṭhito piṭṭhito anubandhi. Atha kho bhagavā yena cāpālaṃ cetiyaṃ tenupasaṅkami; upasaṅkamitvā paññatte āsane nisīdi. Āyasmāpi kho ānando bhagavantaṃ abhivādetvā ekamantaṃ nisīdi.

\paragraph{167.} Ekamantaṃ nisinnaṃ kho āyasmantaṃ ānandaṃ bhagavā etadavoca – ‘‘ramaṇīyā, ānanda, vesālī, ramaṇīyaṃ udenaṃ cetiyaṃ, ramaṇīyaṃ gotamakaṃ cetiyaṃ, ramaṇīyaṃ sattambaṃ\footnote{sattambakaṃ (pī.)} cetiyaṃ, ramaṇīyaṃ bahuputtaṃ cetiyaṃ, ramaṇīyaṃ sārandadaṃ cetiyaṃ, ramaṇīyaṃ cāpālaṃ cetiyaṃ. Yassa kassaci, ānanda, cattāro iddhipādā bhāvitā bahulīkatā yānīkatā vatthukatā anuṭṭhitā paricitā susamāraddhā, so ākaṅkhamāno kappaṃ vā tiṭṭheyya kappāvasesaṃ vā. Tathāgatassa kho, ānanda, cattāro iddhipādā bhāvitā bahulīkatā yānīkatā vatthukatā anuṭṭhitā paricitā susamāraddhā, so ākaṅkhamāno\footnote{ākaṅkhamāno (?)}, ānanda, tathāgato kappaṃ vā tiṭṭheyya kappāvasesaṃ vā’’ti. Evampi kho āyasmā ānando bhagavatā oḷārike nimitte kayiramāne oḷārike obhāse kayiramāne nāsakkhi paṭivijjhituṃ; na bhagavantaṃ yāci – ‘‘tiṭṭhatu, bhante, bhagavā kappaṃ, tiṭṭhatu sugato kappaṃ bahujanahitāya bahujanasukhāya lokānukampāya atthāya hitāya sukhāya devamanussāna’’nti, yathā taṃ mārena pariyuṭṭhitacitto. Dutiyampi kho bhagavā…pe… tatiyampi kho bhagavā āyasmantaṃ ānandaṃ āmantesi – ‘‘ramaṇīyā, ānanda, vesālī, ramaṇīyaṃ udenaṃ cetiyaṃ, ramaṇīyaṃ gotamakaṃ cetiyaṃ, ramaṇīyaṃ sattambaṃ cetiyaṃ, ramaṇīyaṃ bahuputtaṃ cetiyaṃ, ramaṇīyaṃ sārandadaṃ cetiyaṃ, ramaṇīyaṃ cāpālaṃ cetiyaṃ. Yassa kassaci, ānanda, cattāro iddhipādā bhāvitā bahulīkatā yānīkatā vatthukatā anuṭṭhitā paricitā susamāraddhā, so ākaṅkhamāno kappaṃ vā tiṭṭheyya kappāvasesaṃ vā. Tathāgatassa kho, ānanda, cattāro iddhipādā bhāvitā bahulīkatā yānīkatā vatthukatā anuṭṭhitā paricitā susamāraddhā, so ākaṅkhamāno, ānanda, tathāgato kappaṃ vā tiṭṭheyya kappāvasesaṃ vā’’ti. Evampi kho āyasmā ānando bhagavatā oḷārike nimitte kayiramāne oḷārike obhāse kayiramāne nāsakkhi paṭivijjhituṃ ; na bhagavantaṃ yāci – ‘‘tiṭṭhatu , bhante, bhagavā kappaṃ, tiṭṭhatu sugato kappaṃ bahujanahitāya bahujanasukhāya lokānukampāya atthāya hitāya sukhāya devamanussāna’’nti, yathā taṃ mārena pariyuṭṭhitacitto. Atha kho bhagavā āyasmantaṃ ānandaṃ āmantesi – ‘‘gaccha tvaṃ, ānanda, yassadāni kālaṃ maññasī’’ti. ‘‘Evaṃ, bhante’’ti kho āyasmā ānando bhagavato paṭissutvā uṭṭhāyāsanā bhagavantaṃ abhivādetvā padakkhiṇaṃ katvā avidūre aññatarasmiṃ rukkhamūle nisīdi.

\subsubsection{Mārayācanakathā}

\paragraph{168.} Atha kho māro pāpimā acirapakkante āyasmante ānande yena bhagavā tenupasaṅkami; upasaṅkamitvā ekamantaṃ aṭṭhāsi. Ekamantaṃ ṭhito kho māro pāpimā bhagavantaṃ etadavoca – ‘‘parinibbātudāni, bhante, bhagavā, parinibbātu sugato, parinibbānakālo dāni, bhante, bhagavato. Bhāsitā kho panesā, bhante, bhagavatā vācā – ‘na tāvāhaṃ, pāpima, parinibbāyissāmi, yāva me bhikkhū na sāvakā bhavissanti viyattā vinītā visāradā bahussutā dhammadharā dhammānudhammappaṭipannā sāmīcippaṭipannā anudhammacārino, sakaṃ ācariyakaṃ uggahetvā ācikkhissanti desessanti paññapessanti paṭṭhapessanti vivarissanti vibhajissanti uttānī\footnote{uttāniṃ (ka.), uttāni (sī. pī.)} karissanti, uppannaṃ parappavādaṃ sahadhammena suniggahitaṃ niggahetvā sappāṭihāriyaṃ dhammaṃ desessantī’ti . Etarahi kho pana, bhante, bhikkhū bhagavato sāvakā viyattā vinītā visāradā bahussutā dhammadharā dhammānudhammappaṭipannā sāmīcippaṭipannā anudhammacārino, sakaṃ ācariyakaṃ uggahetvā ācikkhanti desenti paññapenti paṭṭhapenti vivaranti vibhajanti uttānīkaronti, uppannaṃ parappavādaṃ sahadhammena suniggahitaṃ niggahetvā sappāṭihāriyaṃ dhammaṃ desenti. Parinibbātudāni, bhante, bhagavā, parinibbātu sugato, parinibbānakālodāni, bhante, bhagavato.

‘‘Bhāsitā kho panesā, bhante, bhagavatā vācā – ‘na tāvāhaṃ, pāpima, parinibbāyissāmi, yāva me bhikkhuniyo na sāvikā bhavissanti viyattā vinītā visāradā bahussutā dhammadharā dhammānudhammappaṭipannā sāmīcippaṭipannā anudhammacāriniyo, sakaṃ ācariyakaṃ uggahetvā ācikkhissanti desessanti paññapessanti paṭṭhapessanti vivarissanti vibhajissanti uttānīkarissanti, uppannaṃ parappavādaṃ sahadhammena suniggahitaṃ niggahetvā sappāṭihāriyaṃ dhammaṃ desessantī’ti . Etarahi kho pana, bhante, bhikkhuniyo bhagavato sāvikā viyattā vinītā visāradā bahussutā dhammadharā dhammānudhammappaṭipannā sāmīcippaṭipannā anudhammacāriniyo , sakaṃ ācariyakaṃ uggahetvā ācikkhanti desenti paññapenti paṭṭhapenti vivaranti vibhajanti uttānīkaronti, uppannaṃ parappavādaṃ sahadhammena suniggahitaṃ niggahetvā sappāṭihāriyaṃ dhammaṃ desenti. Parinibbātudāni, bhante, bhagavā, parinibbātu sugato, parinibbānakālodāni, bhante, bhagavato.

‘‘Bhāsitā kho panesā, bhante, bhagavatā vācā – ‘na tāvāhaṃ, pāpima, parinibbāyissāmi, yāva me upāsakā na sāvakā bhavissanti viyattā vinītā visāradā bahussutā dhammadharā dhammānudhammappaṭipannā sāmīcippaṭipannā anudhammacārino, sakaṃ ācariyakaṃ uggahetvā ācikkhissanti desessanti paññapessanti paṭṭhapessanti vivarissanti vibhajissanti uttānīkarissanti, uppannaṃ parappavādaṃ sahadhammena suniggahitaṃ niggahetvā sappāṭihāriyaṃ dhammaṃ desessantī’ti. Etarahi kho pana, bhante, upāsakā bhagavato sāvakā viyattā vinītā visāradā bahussutā dhammadharā dhammānudhammappaṭipannā sāmīcippaṭipannā anudhammacārino, sakaṃ ācariyakaṃ uggahetvā ācikkhanti desenti paññapenti paṭṭhapenti vivaranti vibhajanti uttānīkaronti, uppannaṃ parappavādaṃ sahadhammena suniggahitaṃ niggahetvā sappāṭihāriyaṃ dhammaṃ desenti. Parinibbātudāni , bhante, bhagavā, parinibbātu sugato, parinibbānakālodāni , bhante, bhagavato.

‘‘Bhāsitā kho panesā, bhante, bhagavatā vācā – ‘na tāvāhaṃ, pāpima parinibbāyissāmi, yāva me upāsikā na sāvikā bhavissanti viyattā vinītā visāradā bahussutā dhammadharā dhammānudhammappaṭipannā sāmīcippaṭipannā anudhammacāriniyo, sakaṃ ācariyakaṃ uggahetvā ācikkhissanti desessanti paññapessanti paṭṭhapessanti vivarissanti vibhajissanti uttānīkarissanti, uppannaṃ parappavādaṃ sahadhammena suniggahitaṃ niggahetvā sappāṭihāriyaṃ dhammaṃ desessantī’ti. Etarahi kho pana, bhante, upāsikā bhagavato sāvikā viyattā vinītā visāradā bahussutā dhammadharā dhammānudhammappaṭipannā sāmīcippaṭipannā anudhammacāriniyo, sakaṃ ācariyakaṃ uggahetvā ācikkhanti desenti paññapenti paṭṭhapenti vivaranti vibhajanti uttānīkaronti, uppannaṃ parappavādaṃ sahadhammena suniggahitaṃ niggahetvā sappāṭihāriyaṃ dhammaṃ desenti. Parinibbātudāni, bhante, bhagavā, parinibbātu sugato, parinibbānakālodāni, bhante, bhagavato.

‘‘Bhāsitā kho panesā, bhante, bhagavatā vācā – ‘na tāvāhaṃ, pāpima, parinibbāyissāmi , yāva me idaṃ brahmacariyaṃ na iddhaṃ ceva bhavissati phītañca vitthārikaṃ bāhujaññaṃ puthubhūtaṃ yāva devamanussehi suppakāsita’nti. Etarahi kho pana, bhante, bhagavato brahmacariyaṃ iddhaṃ ceva phītañca vitthārikaṃ bāhujaññaṃ puthubhūtaṃ, yāva devamanussehi suppakāsitaṃ. Parinibbātudāni, bhante, bhagavā, parinibbātu sugato, parinibbānakālodāni, bhante, bhagavato’’ti .

Evaṃ vutte bhagavā māraṃ pāpimantaṃ etadavoca – ‘‘appossukko tvaṃ, pāpima, hohi, na ciraṃ tathāgatassa parinibbānaṃ bhavissati. Ito tiṇṇaṃ māsānaṃ accayena tathāgato parinibbāyissatī’’ti.

\subsubsection{Āyusaṅkhāraossajjanaṃ}

\paragraph{169.} Atha kho bhagavā cāpāle cetiye sato sampajāno āyusaṅkhāraṃ ossaji. Ossaṭṭhe ca bhagavatā āyusaṅkhāre mahābhūmicālo ahosi bhiṃsanako salomahaṃso\footnote{lomahaṃso (syā.)}, devadundubhiyo\footnote{devadudrabhiyo (ka.)} ca phaliṃsu . Atha kho bhagavā etamatthaṃ viditvā tāyaṃ velāyaṃ imaṃ udānaṃ udānesi –

‘‘Tulamatulañca sambhavaṃ, bhavasaṅkhāramavassaji muni;

Ajjhattarato samāhito, abhindi kavacamivattasambhava’’nti.

\subsubsection{Mahābhūmicālahetu}

\paragraph{170.} Atha kho āyasmato ānandassa etadahosi – ‘‘acchariyaṃ vata bho, abbhutaṃ vata bho, mahā vatāyaṃ bhūmicālo; sumahā vatāyaṃ bhūmicālo bhiṃsanako salomahaṃso; devadundubhiyo ca phaliṃsu. Ko nu kho hetu ko paccayo mahato bhūmicālassa pātubhāvāyā’’ti?

Atha kho āyasmā ānando yena bhagavā tenupasaṅkami, upasaṅkamitvā bhagavantaṃ abhivādetvā ekamantaṃ nisīdi, ekamantaṃ nisinno kho āyasmā ānando bhagavantaṃ etadavoca – ‘‘acchariyaṃ, bhante, abbhutaṃ, bhante, mahā vatāyaṃ, bhante, bhūmicālo; sumahā vatāyaṃ , bhante, bhūmicālo bhiṃsanako salomahaṃso; devadundubhiyo ca phaliṃsu. Ko nu kho, bhante , hetu ko paccayo mahato bhūmicālassa pātubhāvāyā’’ti?

\paragraph{171.} ‘‘Aṭṭha kho ime, ānanda, hetū, aṭṭha paccayā mahato bhūmicālassa pātubhāvāya. Katame aṭṭha? Ayaṃ, ānanda, mahāpathavī udake patiṭṭhitā, udakaṃ vāte patiṭṭhitaṃ, vāto ākāsaṭṭho. Hoti kho so, ānanda, samayo, yaṃ mahāvātā vāyanti. Mahāvātā vāyantā udakaṃ kampenti. Udakaṃ kampitaṃ pathaviṃ kampeti. Ayaṃ paṭhamo hetu paṭhamo paccayo mahato bhūmicālassa pātubhāvāya.

‘‘Puna caparaṃ, ānanda, samaṇo vā hoti brāhmaṇo vā iddhimā cetovasippatto, devo vā mahiddhiko mahānubhāvo, tassa parittā pathavīsaññā bhāvitā hoti, appamāṇā āposaññā. So imaṃ pathaviṃ kampeti saṅkampeti sampakampeti sampavedheti. Ayaṃ dutiyo hetu dutiyo paccayo mahato bhūmicālassa pātubhāvāya.

‘‘Puna caparaṃ, ānanda, yadā bodhisatto tusitakāyā cavitvā sato sampajāno mātukucchiṃ okkamati, tadāyaṃ pathavī kampati saṅkampati sampakampati sampavedhati. Ayaṃ tatiyo hetu tatiyo paccayo mahato bhūmicālassa pātubhāvāya.

‘‘Puna caparaṃ, ānanda, yadā bodhisatto sato sampajāno mātukucchismā nikkhamati, tadāyaṃ pathavī kampati saṅkampati sampakampati sampavedhati. Ayaṃ catuttho hetu catuttho paccayo mahato bhūmicālassa pātubhāvāya.

‘‘Puna caparaṃ, ānanda, yadā tathāgato anuttaraṃ sammāsambodhiṃ abhisambujjhati, tadāyaṃ pathavī kampati saṅkampati sampakampati sampavedhati. Ayaṃ pañcamo hetu pañcamo paccayo mahato bhūmicālassa pātubhāvāya.

‘‘Puna caparaṃ, ānanda, yadā tathāgato anuttaraṃ dhammacakkaṃ pavatteti, tadāyaṃ pathavī kampati saṅkampati sampakampati sampavedhati. Ayaṃ chaṭṭho hetu chaṭṭho paccayo mahato bhūmicālassa pātubhāvāya.

‘‘Puna caparaṃ, ānanda, yadā tathāgato sato sampajāno āyusaṅkhāraṃ ossajjati, tadāyaṃ pathavī kampati saṅkampati sampakampati sampavedhati. Ayaṃ sattamo hetu sattamo paccayo mahato bhūmicālassa pātubhāvāya.

‘‘Puna caparaṃ, ānanda, yadā tathāgato anupādisesāya nibbānadhātuyā parinibbāyati, tadāyaṃ pathavī kampati saṅkampati sampakampati sampavedhati. Ayaṃ aṭṭhamo hetu aṭṭhamo paccayo mahato bhūmicālassa pātubhāvāya. Ime kho, ānanda, aṭṭha hetū, aṭṭha paccayā mahato bhūmicālassa pātubhāvāyā’’ti.

\subsubsection{Aṭṭha parisā}

\paragraph{172.} ‘‘Aṭṭha kho imā, ānanda, parisā. Katamā aṭṭha? Khattiyaparisā, brāhmaṇaparisā, gahapatiparisā, samaṇaparisā, cātumahārājikaparisā\footnote{cātummahārājikaparisā (sī. syā. kaṃ. pī.)}, tāvatiṃsaparisā, māraparisā, brahmaparisā. Abhijānāmi kho panāhaṃ, ānanda , anekasataṃ khattiyaparisaṃ upasaṅkamitā. Tatrapi mayā sannisinnapubbaṃ ceva sallapitapubbañca sākacchā ca samāpajjitapubbā . Tattha yādisako tesaṃ vaṇṇo hoti, tādisako mayhaṃ vaṇṇo hoti. Yādisako tesaṃ saro hoti, tādisako mayhaṃ saro hoti. Dhammiyā kathāya sandassemi samādapemi samuttejemi sampahaṃsemi. Bhāsamānañca maṃ na jānanti – ‘ko nu kho ayaṃ bhāsati devo vā manusso vā’ti? Dhammiyā kathāya sandassetvā samādapetvā samuttejetvā sampahaṃsetvā antaradhāyāmi. Antarahitañca maṃ na jānanti – ‘ko nu kho ayaṃ antarahito devo vā manusso vā’ti? Abhijānāmi kho panāhaṃ, ānanda, anekasataṃ brāhmaṇaparisaṃ…pe… gahapatiparisaṃ… samaṇaparisaṃ… cātumahārājikaparisaṃ… tāvatiṃsaparisaṃ… māraparisaṃ… brahmaparisaṃ upasaṅkamitā. Tatrapi mayā sannisinnapubbaṃ ceva sallapitapubbañca sākacchā ca samāpajjitapubbā. Tattha yādisako tesaṃ vaṇṇo hoti, tādisako mayhaṃ vaṇṇo hoti. Yādisako tesaṃ saro hoti, tādisako mayhaṃ saro hoti. Dhammiyā kathāya sandassemi samādapemi samuttejemi sampahaṃsemi. Bhāsamānañca maṃ na jānanti – ‘ko nu kho ayaṃ bhāsati devo vā manusso vā’ti? Dhammiyā kathāya sandassetvā samādapetvā samuttejetvā sampahaṃsetvā antaradhāyāmi. Antarahitañca maṃ na jānanti – ‘ko nu kho ayaṃ antarahito devo vā manusso vā’ti? Imā kho, ānanda, aṭṭha parisā.

\subsubsection{Aṭṭha abhibhāyatanāni}

\paragraph{173.} ‘‘Aṭṭha kho imāni, ānanda, abhibhāyatanāni. Katamāni aṭṭha ? Ajjhattaṃ rūpasaññī eko bahiddhā rūpāni passati parittāni suvaṇṇadubbaṇṇāni. ‘Tāni abhibhuyya jānāmi passāmī’ti evaṃsaññī hoti. Idaṃ paṭhamaṃ abhibhāyatanaṃ.

‘‘Ajjhattaṃ rūpasaññī eko bahiddhā rūpāni passati appamāṇāni suvaṇṇadubbaṇṇāni. ‘Tāni abhibhuyya jānāmi passāmī’ti evaṃsaññī hoti. Idaṃ dutiyaṃ abhibhāyatanaṃ.

‘‘Ajjhattaṃ arūpasaññī eko bahiddhā rūpāni passati parittāni suvaṇṇadubbaṇṇāni. ‘Tāni abhibhuyya jānāmi passāmī’ti evaṃsaññī hoti. Idaṃ tatiyaṃ abhibhāyatanaṃ.

‘‘Ajjhattaṃ arūpasaññī eko bahiddhā rūpāni passati appamāṇāni suvaṇṇadubbaṇṇāni. ‘Tāni abhibhuyya jānāmi passāmī’ti evaṃsaññī hoti. Idaṃ catutthaṃ abhibhāyatanaṃ.

‘‘Ajjhattaṃ arūpasaññī eko bahiddhā rūpāni passati nīlāni nīlavaṇṇāni nīlanidassanāni nīlanibhāsāni. Seyyathāpi nāma umāpupphaṃ nīlaṃ nīlavaṇṇaṃ nīlanidassanaṃ nīlanibhāsaṃ. Seyyathā vā pana taṃ vatthaṃ bārāṇaseyyakaṃ ubhatobhāgavimaṭṭhaṃ nīlaṃ nīlavaṇṇaṃ nīlanidassanaṃ nīlanibhāsaṃ. Evameva ajjhattaṃ arūpasaññī eko bahiddhā rūpāni passati nīlāni nīlavaṇṇāni nīlanidassanāni nīlanibhāsāni. ‘Tāni abhibhuyya jānāmi passāmī’ti evaṃsaññī hoti. Idaṃ pañcamaṃ abhibhāyatanaṃ.

‘‘Ajjhattaṃ arūpasaññī eko bahiddhā rūpāni passati pītāni pītavaṇṇāni pītanidassanāni pītanibhāsāni. Seyyathāpi nāma kaṇikārapupphaṃ pītaṃ pītavaṇṇaṃ pītanidassanaṃ pītanibhāsaṃ. Seyyathā vā pana taṃ vatthaṃ bārāṇaseyyakaṃ ubhatobhāgavimaṭṭhaṃ pītaṃ pītavaṇṇaṃ pītanidassanaṃ pītanibhāsaṃ. Evameva ajjhattaṃ arūpasaññī eko bahiddhā rūpāni passati pītāni pītavaṇṇāni pītanidassanāni pītanibhāsāni. ‘Tāni abhibhuyya jānāmi passāmī’ti evaṃsaññī hoti. Idaṃ chaṭṭhaṃ abhibhāyatanaṃ.

‘‘Ajjhattaṃ arūpasaññī eko bahiddhā rūpāni passati lohitakāni lohitakavaṇṇāni lohitakanidassanāni lohitakanibhāsāni. Seyyathāpi nāma bandhujīvakapupphaṃ lohitakaṃ lohitakavaṇṇaṃ lohitakanidassanaṃ lohitakanibhāsaṃ. Seyyathā vā pana taṃ vatthaṃ bārāṇaseyyakaṃ ubhatobhāgavimaṭṭhaṃ lohitakaṃ lohitakavaṇṇaṃ lohitakanidassanaṃ lohitakanibhāsaṃ. Evameva ajjhattaṃ arūpasaññī eko bahiddhā rūpāni passati lohitakāni lohitakavaṇṇāni lohitakanidassanāni lohitakanibhāsāni. ‘Tāni abhibhuyya jānāmi passāmī’ti evaṃsaññī hoti. Idaṃ sattamaṃ abhibhāyatanaṃ.

‘‘Ajjhattaṃ arūpasaññī eko bahiddhā rūpāni passati odātāni odātavaṇṇāni odātanidassanāni odātanibhāsāni. Seyyathāpi nāma osadhitārakā odātā odātavaṇṇā odātanidassanā odātanibhāsā. Seyyathā vā pana taṃ vatthaṃ bārāṇaseyyakaṃ ubhatobhāgavimaṭṭhaṃ odātaṃ odātavaṇṇaṃ odātanidassanaṃ odātanibhāsaṃ. Evameva ajjhattaṃ arūpasaññī eko bahiddhā rūpāni passati odātāni odātavaṇṇāni odātanidassanāni odātanibhāsāni. ‘Tāni abhibhuyya jānāmi passāmī’ti evaṃsaññī hoti. Idaṃ aṭṭhamaṃ abhibhāyatanaṃ . Imāni kho, ānanda, aṭṭha abhibhāyatanāni.

\subsubsection{Aṭṭha vimokkhā}

\paragraph{174.} ‘‘Aṭṭha kho ime, ānanda, vimokkhā. Katame aṭṭha? Rūpī rūpāni passati, ayaṃ paṭhamo vimokkho. Ajjhattaṃ arūpasaññī bahiddhā rūpāni passati, ayaṃ dutiyo vimokkho. Subhanteva adhimutto hoti, ayaṃ tatiyo vimokkho. Sabbaso rūpasaññānaṃ samatikkamā paṭighasaññānaṃ atthaṅgamā nānattasaññānaṃ amanasikārā ‘ananto ākāso’ti ākāsānañcāyatanaṃ upasampajja viharati, ayaṃ catuttho vimokkho. Sabbaso ākāsānañcāyatanaṃ samatikkamma ‘anantaṃ viññāṇa’nti viññāṇañcāyatanaṃ upasampajja viharati, ayaṃ pañcamo vimokkho. Sabbaso viññāṇañcāyatanaṃ samatikkamma ‘natthi kiñcī’ti ākiñcaññāyatanaṃ upasampajja viharati, ayaṃ chaṭṭho vimokkho. Sabbaso ākiñcaññāyatanaṃ samatikkamma nevasaññānāsaññāyatanaṃ upasampajja viharati. Ayaṃ sattamo vimokkho. Sabbaso nevasaññānāsaññāyatanaṃ samatikkamma saññāvedayitanirodhaṃ upasampajja viharati, ayaṃ aṭṭhamo vimokkho. Ime kho, ānanda, aṭṭha vimokkhā.

\paragraph{175.} ‘‘Ekamidāhaṃ , ānanda, samayaṃ uruvelāyaṃ viharāmi najjā nerañjarāya tīre ajapālanigrodhe paṭhamābhisambuddho. Atha kho, ānanda, māro pāpimā yenāhaṃ tenupasaṅkami; upasaṅkamitvā ekamantaṃ aṭṭhāsi. Ekamantaṃ ṭhito kho, ānanda, māro pāpimā maṃ etadavoca – ‘parinibbātudāni, bhante, bhagavā; parinibbātu sugato, parinibbānakālodāni, bhante, bhagavato’ti. Evaṃ vutte ahaṃ, ānanda, māraṃ pāpimantaṃ etadavocaṃ –

‘‘‘Na tāvāhaṃ, pāpima, parinibbāyissāmi, yāva me bhikkhū na sāvakā bhavissanti viyattā vinītā visāradā bahussutā dhammadharā dhammānudhammappaṭipannā sāmīcippaṭipannā anudhammacārino, sakaṃ ācariyakaṃ uggahetvā ācikkhissanti desessanti paññapessanti paṭṭhapessanti vivarissanti vibhajissanti uttānīkarissanti, uppannaṃ parappavādaṃ sahadhammena suniggahitaṃ niggahetvā sappāṭihāriyaṃ dhammaṃ desessanti.

‘‘‘Na tāvāhaṃ, pāpima, parinibbāyissāmi, yāva me bhikkhuniyo na sāvikā bhavissanti viyattā vinītā visāradā bahussutā dhammadharā dhammānudhammappaṭipannā sāmīcippaṭipannā anudhammacāriniyo, sakaṃ ācariyakaṃ uggahetvā ācikkhissanti desessanti paññapessanti paṭṭhapessanti vivarissanti vibhajissanti uttānīkarissanti, uppannaṃ parappavādaṃ sahadhammena suniggahitaṃ niggahetvā sappāṭihāriyaṃ dhammaṃ desessanti.

‘‘‘Na tāvāhaṃ, pāpima, parinibbāyissāmi, yāva me upāsakā na sāvakā bhavissanti viyattā vinītā visāradā bahussutā dhammadharā dhammānudhammappaṭipannā sāmīcippaṭipannā anudhammacārino, sakaṃ ācariyakaṃ uggahetvā ācikkhissanti desessanti paññapessanti paṭṭhapessanti vivarissanti vibhajissanti uttānīkarissanti, uppannaṃ parappavādaṃ sahadhammena suniggahitaṃ niggahetvā sappāṭihāriyaṃ dhammaṃ desessanti.

‘‘‘Na tāvāhaṃ, pāpima, parinibbāyissāmi, yāva me upāsikā na sāvikā bhavissanti viyattā vinītā visāradā bahussutā dhammadharā dhammānudhammappaṭipannā sāmīcippaṭipannā anudhammacāriniyo, sakaṃ ācariyakaṃ uggahetvā ācikkhissanti desessanti paññapessanti paṭṭhapessanti vivarissanti vibhajissanti uttānīkarissanti, uppannaṃ parappavādaṃ sahadhammena suniggahitaṃ niggahetvā sappāṭihāriyaṃ dhammaṃ desessanti.

‘‘‘Na tāvāhaṃ, pāpima, parinibbāyissāmi, yāva me idaṃ brahmacariyaṃ na iddhañceva bhavissati phītañca vitthārikaṃ bāhujaññaṃ puthubhūtaṃ yāva devamanussehi suppakāsita’nti.

\paragraph{176.} ‘‘Idāneva kho, ānanda, ajja cāpāle cetiye māro pāpimā yenāhaṃ tenupasaṅkami; upasaṅkamitvā ekamantaṃ aṭṭhāsi. Ekamantaṃ ṭhito kho, ānanda, māro pāpimā maṃ etadavoca – ‘parinibbātudāni, bhante, bhagavā, parinibbātu sugato, parinibbānakālodāni, bhante, bhagavato. Bhāsitā kho panesā, bhante, bhagavatā vācā – ‘‘na tāvāhaṃ, pāpima , parinibbāyissāmi , yāva me bhikkhū na sāvakā bhavissanti…pe… yāva me bhikkhuniyo na sāvikā bhavissanti…pe… yāva me upāsakā na sāvakā bhavissanti…pe… yāva me upāsikā na sāvikā bhavissanti…pe… yāva me idaṃ brahmacariyaṃ na iddhañceva bhavissati phītañca vitthārikaṃ bāhujaññaṃ puthubhūtaṃ, yāva devamanussehi suppakāsita’’nti. Etarahi kho pana, bhante, bhagavato brahmacariyaṃ iddhañceva phītañca vitthārikaṃ bāhujaññaṃ puthubhūtaṃ, yāva devamanussehi suppakāsitaṃ. Parinibbātudāni, bhante, bhagavā, parinibbātu sugato, parinibbānakālodāni, bhante, bhagavato’ti.

\paragraph{177.} ‘‘Evaṃ vutte, ahaṃ, ānanda, māraṃ pāpimantaṃ etadavocaṃ – ‘appossukko tvaṃ, pāpima, hohi, naciraṃ tathāgatassa parinibbānaṃ bhavissati. Ito tiṇṇaṃ māsānaṃ accayena tathāgato parinibbāyissatī’ti. Idāneva kho, ānanda, ajja cāpāle cetiye tathāgatena satena sampajānena āyusaṅkhāro ossaṭṭho’’ti.

\subsubsection{Ānandayācanakathā}

\paragraph{178.} Evaṃ vutte āyasmā ānando bhagavantaṃ etadavoca – ‘‘tiṭṭhatu, bhante, bhagavā kappaṃ, tiṭṭhatu sugato kappaṃ bahujanahitāya bahujanasukhāya lokānukampāya atthāya hitāya sukhāya devamanussāna’’nti.

‘‘Alaṃdāni, ānanda. Mā tathāgataṃ yāci, akālodāni, ānanda, tathāgataṃ yācanāyā’’ti. Dutiyampi kho āyasmā ānando…pe… tatiyampi kho āyasmā ānando bhagavantaṃ etadavoca – ‘‘tiṭṭhatu, bhante, bhagavā kappaṃ, tiṭṭhatu sugato kappaṃ bahujanahitāya bahujanasukhāya lokānukampāya atthāya hitāya sukhāya devamanussāna’’nti.

‘‘Saddahasi tvaṃ, ānanda, tathāgatassa bodhi’’nti? ‘‘Evaṃ, bhante’’. ‘‘Atha kiñcarahi tvaṃ, ānanda, tathāgataṃ yāvatatiyakaṃ abhinippīḷesī’’ti? ‘‘Sammukhā metaṃ, bhante, bhagavato sutaṃ sammukhā paṭiggahitaṃ – ‘yassa kassaci, ānanda, cattāro iddhipādā bhāvitā bahulīkatā yānīkatā vatthukatā anuṭṭhitā paricitā susamāraddhā, so ākaṅkhamāno kappaṃ vā tiṭṭheyya kappāvasesaṃ vā. Tathāgatassa kho, ānanda, cattāro iddhipādā bhāvitā bahulīkatā yānīkatā vatthukatā anuṭṭhitā paricitā susamāraddhā. So ākaṅkhamāno, ānanda, tathāgato kappaṃ vā tiṭṭheyya kappāvasesaṃ vā’’’ti. ‘‘Saddahasi tvaṃ, ānandā’’ti? ‘‘Evaṃ, bhante’’. ‘‘Tasmātihānanda, tuyhevetaṃ dukkaṭaṃ, tuyhevetaṃ aparaddhaṃ, yaṃ tvaṃ tathāgatena evaṃ oḷārike nimitte kayiramāne oḷārike obhāse kayiramāne nāsakkhi paṭivijjhituṃ, na tathāgataṃ yāci – ‘tiṭṭhatu, bhante, bhagavā kappaṃ, tiṭṭhatu sugato kappaṃ bahujanahitāya bahujanasukhāya lokānukampāya atthāya hitāya sukhāya devamanussāna’’nti. Sace tvaṃ, ānanda, tathāgataṃ yāceyyāsi, dveva te vācā tathāgato paṭikkhipeyya, atha tatiyakaṃ adhivāseyya. Tasmātihānanda, tuyhevetaṃ dukkaṭaṃ, tuyhevetaṃ aparaddhaṃ.

\paragraph{179.} ‘‘Ekamidāhaṃ, ānanda, samayaṃ rājagahe viharāmi gijjhakūṭe pabbate. Tatrāpi kho tāhaṃ, ānanda, āmantesiṃ – ‘ramaṇīyaṃ, ānanda, rājagahaṃ, ramaṇīyo, ānanda, gijjhakūṭo pabbato. Yassa kassaci, ānanda, cattāro iddhipādā bhāvitā bahulīkatā yānīkatā vatthukatā anuṭṭhitā paricitā susamāraddhā, so ākaṅkhamāno kappaṃ vā tiṭṭheyya kappāvasesaṃ vā. Tathāgatassa kho, ānanda, cattāro iddhipādā bhāvitā bahulīkatā yānīkatā vatthukatā anuṭṭhitā paricitā susamāraddhā, so ākaṅkhamāno, ānanda, tathāgato kappaṃ vā tiṭṭheyya kappāvasesaṃ vā’ti. Evampi kho tvaṃ, ānanda, tathāgatena oḷārike nimitte kayiramāne oḷārike obhāse kayiramāne nāsakkhi paṭivijjhituṃ, na tathāgataṃ yāci – ‘tiṭṭhatu, bhante, bhagavā kappaṃ, tiṭṭhatu sugato kappaṃ bahujanahitāya bahujanasukhāya lokānukampāya atthāya hitāya sukhāya devamanussāna’nti. Sace tvaṃ, ānanda, tathāgataṃ yāceyyāsi, dve te vācā tathāgato paṭikkhipeyya, atha tatiyakaṃ adhivāseyya. Tasmātihānanda, tuyhevetaṃ dukkaṭaṃ, tuyhevetaṃ aparaddhaṃ.

\paragraph{180.} ‘‘Ekamidāhaṃ, ānanda, samayaṃ tattheva rājagahe viharāmi gotamanigrodhe…pe… tattheva rājagahe viharāmi corapapāte… tattheva rājagahe viharāmi vebhārapasse sattapaṇṇiguhāyaṃ… tattheva rājagahe viharāmi isigilipasse kāḷasilāyaṃ… tattheva rājagahe viharāmi sītavane sappasoṇḍikapabbhāre… tattheva rājagahe viharāmi tapodārāme… tattheva rājagahe viharāmi veḷuvane kalandakanivāpe… tattheva rājagahe viharāmi jīvakambavane… tattheva rājagahe viharāmi maddakucchismiṃ migadāye tatrāpi kho tāhaṃ, ānanda, āmantesiṃ – ‘ramaṇīyaṃ, ānanda, rājagahaṃ, ramaṇīyo gijjhakūṭo pabbato, ramaṇīyo gotamanigrodho, ramaṇīyo corapapāto, ramaṇīyā vebhārapasse sattapaṇṇiguhā, ramaṇīyā isigilipasse kāḷasilā, ramaṇīyo sītavane sappasoṇḍikapabbhāro , ramaṇīyo tapodārāmo, ramaṇīyo veḷuvane kalandakanivāpo, ramaṇīyaṃ jīvakambavanaṃ, ramaṇīyo maddakucchismiṃ migadāyo. Yassa kassaci, ānanda, cattāro iddhipādā bhāvitā bahulīkatā yānīkatā vatthukatā anuṭṭhitā paricitā susamāraddhā…pe… ākaṅkhamāno, ānanda, tathāgato kappaṃ vā tiṭṭheyya kappāvasesaṃ vā’ti. Evampi kho tvaṃ, ānanda, tathāgatena oḷārike nimitte kayiramāne oḷārike obhāse kayiramāne nāsakkhi paṭivijjhituṃ, na tathāgataṃ yāci – ‘tiṭṭhatu, bhante, bhagavā kappaṃ, tiṭṭhatu sugato kappaṃ bahujanahitāya bahujanasukhāya lokānukampāya atthāya hitāya sukhāya devamanussāna’nti. Sace tvaṃ, ānanda, tathāgataṃ yāceyyāsi, dveva te vācā tathāgato paṭikkhipeyya, atha tatiyakaṃ adhivāseyya. Tasmātihānanda, tuyhevetaṃ dukkaṭaṃ, tuyhevetaṃ aparaddhaṃ.

\paragraph{181.} ‘‘Ekamidāhaṃ, ānanda, samayaṃ idheva vesāliyaṃ viharāmi udene cetiye. Tatrāpi kho tāhaṃ, ānanda, āmantesiṃ – ‘ramaṇīyā, ānanda, vesālī, ramaṇīyaṃ udenaṃ cetiyaṃ. Yassa kassaci, ānanda, cattāro iddhipādā bhāvitā bahulīkatā yānīkatā vatthukatā anuṭṭhitā paricitā susamāraddhā, so ākaṅkhamāno kappaṃ vā tiṭṭheyya kappāvasesaṃ vā. Tathāgatassa kho, ānanda, cattāro iddhipādā bhāvitā bahulīkatā yānīkatā vatthukatā anuṭṭhitā paricitā susamāraddhā, so ākaṅkhamāno, ānanda, tathāgato kappaṃ vā tiṭṭheyya kappāvasesaṃ vā’ti. Evampi kho tvaṃ, ānanda, tathāgatena oḷārike nimitte kayiramāne oḷārike obhāse kayiramāne nāsakkhi paṭivijjhituṃ, na tathāgataṃ yāci – ‘tiṭṭhatu, bhante, bhagavā kappaṃ, tiṭṭhatu sugato kappaṃ bahujanahitāya bahujanasukhāya lokānukampāya atthāya hitāya sukhāya devamanussāna’nti. Sace tvaṃ, ānanda, tathāgataṃ yāceyyāsi, dveva te vācā tathāgato paṭikkhipeyya, atha tatiyakaṃ adhivāseyya, tasmātihānanda, tuyhevetaṃ dukkaṭaṃ, tuyhevetaṃ aparaddhaṃ.

\paragraph{182.} ‘‘Ekamidāhaṃ , ānanda, samayaṃ idheva vesāliyaṃ viharāmi gotamake cetiye …pe… idheva vesāliyaṃ viharāmi sattambe cetiye… idheva vesāliyaṃ viharāmi bahuputte cetiye… idheva vesāliyaṃ viharāmi sārandade cetiye… idāneva kho tāhaṃ, ānanda, ajja cāpāle cetiye āmantesiṃ – ‘ramaṇīyā, ānanda, vesālī, ramaṇīyaṃ udenaṃ cetiyaṃ, ramaṇīyaṃ gotamakaṃ cetiyaṃ, ramaṇīyaṃ sattambaṃ cetiyaṃ, ramaṇīyaṃ bahuputtaṃ cetiyaṃ, ramaṇīyaṃ sārandadaṃ cetiyaṃ, ramaṇīyaṃ cāpālaṃ cetiyaṃ. Yassa kassaci, ānanda, cattāro iddhipādā bhāvitā bahulīkatā yānīkatā vatthukatā anuṭṭhitā paricitā susamāraddhā, so ākaṅkhamāno kappaṃ vā tiṭṭheyya kappāvasesaṃ vā. Tathāgatassa kho, ānanda, cattāro iddhipādā bhāvitā bahulīkatā yānīkatā vatthukatā anuṭṭhitā paricitā susamāraddhā, so ākaṅkhamāno, ānanda, tathāgato kappaṃ vā tiṭṭheyya kappāvasesaṃ vā’ti. Evampi kho tvaṃ, ānanda, tathāgatena oḷārike nimitte kayiramāne oḷārike obhāse kayiramāne nāsakkhi paṭivijjhituṃ, na tathāgataṃ yāci – ‘tiṭṭhatu bhagavā kappaṃ, tiṭṭhatu sugato kappaṃ bahujanahitāya bahujanasukhāya lokānukampāya atthāya hitāya sukhāya devamanussāna’nti. Sace tvaṃ, ānanda, tathāgataṃ yāceyyāsi, dveva te vācā tathāgato paṭikkhipeyya, atha tatiyakaṃ adhivāseyya. Tasmātihānanda, tuyhevetaṃ dukkaṭaṃ, tuyhevetaṃ aparaddhaṃ.

\paragraph{183.} ‘‘Nanu etaṃ\footnote{evaṃ (syā. pī.)}, ānanda, mayā paṭikacceva\footnote{paṭigacceva (sī. pī.)} akkhātaṃ – ‘sabbeheva piyehi manāpehi nānābhāvo vinābhāvo aññathābhāvo. Taṃ kutettha, ānanda, labbhā, yaṃ taṃ jātaṃ bhūtaṃ saṅkhataṃ palokadhammaṃ, taṃ vata mā palujjīti netaṃ ṭhānaṃ vijjati’. Yaṃ kho panetaṃ, ānanda, tathāgatena cattaṃ vantaṃ muttaṃ pahīnaṃ paṭinissaṭṭhaṃ ossaṭṭho āyusaṅkhāro, ekaṃsena vācā bhāsitā – ‘na ciraṃ tathāgatassa parinibbānaṃ bhavissati. Ito tiṇṇaṃ māsānaṃ accayena tathāgato parinibbāyissatī’ti. Tañca\footnote{taṃ vacanaṃ (sī.)} tathāgato jīvitahetu puna paccāvamissatīti\footnote{paccāgamissatīti (syā. ka.)} netaṃ ṭhānaṃ vijjati. Āyāmānanda, yena mahāvanaṃ kūṭāgārasālā tenupasaṅkamissāmā’’ti. ‘‘Evaṃ, bhante’’ti kho āyasmā ānando bhagavato paccassosi.

Atha kho bhagavā āyasmatā ānandena saddhiṃ yena mahāvanaṃ kūṭāgārasālā tenupasaṅkami; upasaṅkamitvā āyasmantaṃ ānandaṃ āmantesi – ‘‘gaccha tvaṃ, ānanda, yāvatikā bhikkhū vesāliṃ upanissāya viharanti, te sabbe upaṭṭhānasālāyaṃ sannipātehī’’ti. ‘‘Evaṃ, bhante’’ti kho āyasmā ānando bhagavato paṭissutvā yāvatikā bhikkhū vesāliṃ upanissāya viharanti, te sabbe upaṭṭhānasālāyaṃ sannipātetvā yena bhagavā tenupasaṅkami; upasaṅkamitvā bhagavantaṃ abhivādetvā ekamantaṃ aṭṭhāsi. Ekamantaṃ ṭhito kho āyasmā ānando bhagavantaṃ etadavoca – ‘‘sannipatito, bhante, bhikkhusaṅgho, yassadāni, bhante, bhagavā kālaṃ maññatī’’ti.

\paragraph{184.} Atha kho bhagavā yenupaṭṭhānasālā tenupasaṅkami; upasaṅkamitvā paññatte āsane nisīdi. Nisajja kho bhagavā bhikkhū āmantesi – ‘‘tasmātiha, bhikkhave, ye te mayā dhammā abhiññā desitā, te vo sādhukaṃ uggahetvā āsevitabbā bhāvetabbā bahulīkātabbā, yathayidaṃ brahmacariyaṃ addhaniyaṃ assa ciraṭṭhitikaṃ, tadassa bahujanahitāya bahujanasukhāya lokānukampāya atthāya hitāya sukhāya devamanussānaṃ. Katame ca te, bhikkhave, dhammā mayā abhiññā desitā, ye vo sādhukaṃ uggahetvā āsevitabbā bhāvetabbā bahulīkātabbā, yathayidaṃ brahmacariyaṃ addhaniyaṃ assa ciraṭṭhitikaṃ, tadassa bahujanahitāya bahujanasukhāya lokānukampāya atthāya hitāya sukhāya devamanussānaṃ. Seyyathidaṃ – cattāro satipaṭṭhānā cattāro sammappadhānā cattāro iddhipādā pañcindriyāni pañca balāni satta bojjhaṅgā ariyo aṭṭhaṅgiko maggo. Ime kho te, bhikkhave, dhammā mayā abhiññā desitā, ye vo sādhukaṃ uggahetvā āsevitabbā bhāvetabbā bahulīkātabbā, yathayidaṃ brahmacariyaṃ addhaniyaṃ assa ciraṭṭhitikaṃ, tadassa bahujanahitāya bahujanasukhāya lokānukampāya atthāya hitāya sukhāya devamanussāna’’nti.

\paragraph{185.} Atha kho bhagavā bhikkhū āmantesi – ‘‘handadāni, bhikkhave, āmantayāmi vo, vayadhammā saṅkhārā, appamādena sampādetha. Naciraṃ tathāgatassa parinibbānaṃ bhavissati. Ito tiṇṇaṃ māsānaṃ accayena tathāgato parinibbāyissatī’’ti. Idamavoca bhagavā, idaṃ vatvāna sugato athāparaṃ etadavoca satthā\footnote{ito paraṃ syāmapotthake evaṃpi pāṭho dissati –§daharāpi ca ye vuddhā, ye bālā ye ca paṇḍitā.§aḍḍhāceva daliddā ca, sabbe maccuparāyanā.§yathāpi kumbhakārassa, kataṃ mattikabhājanaṃ.§khuddakañca mahantañca, yañca pakkaṃ yañca āmakaṃ.§sabbaṃ bhedapariyantaṃ, evaṃ maccāna jīvitaṃ.§athāparaṃ etadavoca satthā}. –

‘‘Paripakko vayo mayhaṃ, parittaṃ mama jīvitaṃ;

Pahāya vo gamissāmi, kataṃ me saraṇamattano.

‘‘Appamattā satīmanto, susīlā hotha bhikkhavo;

Susamāhitasaṅkappā, sacittamanurakkhatha.

‘‘Yo imasmiṃ dhammavinaye, appamatto vihassati;

Pahāya jātisaṃsāraṃ, dukkhassantaṃ karissatī’’ti\footnote{viharissati (syā.), vihessati (sī.)}.

\xsubsubsectionEnd{Tatiyo bhāṇavāro.}

\subsubsection{Nāgāpalokitaṃ}

\paragraph{186.} Atha kho bhagavā pubbaṇhasamayaṃ nivāsetvā pattacīvaramādāya vesāliṃ piṇḍāya pāvisi. Vesāliyaṃ piṇḍāya caritvā pacchābhattaṃ piṇḍapātappaṭikkanto nāgāpalokitaṃ vesāliṃ apaloketvā āyasmantaṃ ānandaṃ āmantesi – ‘‘idaṃ pacchimakaṃ, ānanda, tathāgatassa vesāliyā dassanaṃ bhavissati. Āyāmānanda, yena bhaṇḍagāmo\footnote{bhaṇḍugāmo (ka.)} tenupasaṅkamissāmā’’ti. ‘‘Evaṃ, bhante’’ti kho āyasmā ānando bhagavato paccassosi.

Atha kho bhagavā mahatā bhikkhusaṅghena saddhiṃ yena bhaṇḍagāmo tadavasari. Tatra sudaṃ bhagavā bhaṇḍagāme viharati. Tatra kho bhagavā bhikkhū āmantesi – ‘‘catunnaṃ, bhikkhave, dhammānaṃ ananubodhā appaṭivedhā evamidaṃ dīghamaddhānaṃ sandhāvitaṃ saṃsaritaṃ mamañceva tumhākañca. Katamesaṃ catunnaṃ? Ariyassa, bhikkhave, sīlassa ananubodhā appaṭivedhā evamidaṃ dīghamaddhānaṃ sandhāvitaṃ saṃsaritaṃ mamaṃ ceva tumhākañca. Ariyassa, bhikkhave, samādhissa ananubodhā appaṭivedhā evamidaṃ dīghamaddhānaṃ sandhāvitaṃ saṃsaritaṃ mamaṃ ceva tumhākañca. Ariyāya, bhikkhave, paññāya ananubodhā appaṭivedhā evamidaṃ dīghamaddhānaṃ sandhāvitaṃ saṃsaritaṃ mamaṃ ceva tumhākañca. Ariyāya, bhikkhave, vimuttiyā ananubodhā appaṭivedhā evamidaṃ dīghamaddhānaṃ sandhāvitaṃ saṃsaritaṃ mamaṃ ceva tumhākañca. Tayidaṃ, bhikkhave, ariyaṃ sīlaṃ anubuddhaṃ paṭividdhaṃ, ariyo samādhi anubuddho paṭividdho, ariyā paññā anubuddhā paṭividdhā, ariyā vimutti anubuddhā paṭividdhā, ucchinnā bhavataṇhā, khīṇā bhavanetti, natthi dāni punabbhavo’’ti. Idamavoca bhagavā, idaṃ vatvāna sugato athāparaṃ etadavoca satthā –

‘‘Sīlaṃ samādhi paññā ca, vimutti ca anuttarā;

Anubuddhā ime dhammā, gotamena yasassinā.

‘‘Iti buddho abhiññāya, dhammamakkhāsi bhikkhunaṃ;

Dukkhassantakaro satthā, cakkhumā parinibbuto’’ti.

Tatrāpi sudaṃ bhagavā bhaṇḍagāme viharanto etadeva bahulaṃ bhikkhūnaṃ dhammiṃ kathaṃ karoti – ‘‘iti sīlaṃ, iti samādhi, iti paññā. Sīlaparibhāvito samādhi mahapphalo hoti mahānisaṃso. Samādhiparibhāvitā paññā mahapphalā hoti mahānisaṃsā. Paññāparibhāvitaṃ cittaṃ sammadeva āsavehi vimuccati, seyyathidaṃ – kāmāsavā, bhavāsavā, avijjāsavā’’ti.

\subsubsection{Catumahāpadesakathā}

\paragraph{187.} Atha kho bhagavā bhaṇḍagāme yathābhirantaṃ viharitvā āyasmantaṃ ānandaṃ āmantesi – ‘‘āyāmānanda, yena hatthigāmo, yena ambagāmo, yena jambugāmo, yena bhoganagaraṃ tenupasaṅkamissāmā’’ti. ‘‘Evaṃ, bhante’’ti kho āyasmā ānando bhagavato paccassosi. Atha kho bhagavā mahatā bhikkhusaṅghena saddhiṃ yena bhoganagaraṃ tadavasari. Tatra sudaṃ bhagavā bhoganagare viharati ānande\footnote{sānandare (ka.)} cetiye. Tatra kho bhagavā bhikkhū āmantesi – ‘‘cattārome, bhikkhave, mahāpadese desessāmi, taṃ suṇātha, sādhukaṃ manasikarotha, bhāsissāmī’’ti. ‘‘Evaṃ , bhante’’ti kho te bhikkhū bhagavato paccassosuṃ. Bhagavā etadavoca –

\paragraph{188.} ‘‘Idha, bhikkhave, bhikkhu evaṃ vadeyya – ‘sammukhā metaṃ, āvuso, bhagavato sutaṃ sammukhā paṭiggahitaṃ, ayaṃ dhammo ayaṃ vinayo idaṃ satthusāsana’nti. Tassa, bhikkhave, bhikkhuno bhāsitaṃ neva abhinanditabbaṃ nappaṭikkositabbaṃ. Anabhinanditvā appaṭikkositvā tāni padabyañjanāni sādhukaṃ uggahetvā sutte osāretabbāni\footnote{otāretabbāni}, vinaye sandassetabbāni. Tāni ce sutte osāriyamānāni\footnote{otāriyamānāni} vinaye sandassiyamānāni na ceva sutte osaranti\footnote{otaranti (sī. pī. a. ni. 4.180}, na ca vinaye sandissanti, niṭṭhamettha gantabbaṃ – ‘addhā, idaṃ na ceva tassa bhagavato vacanaṃ; imassa ca bhikkhuno duggahita’nti. Itihetaṃ, bhikkhave, chaḍḍeyyātha. Tāni ce sutte osāriyamānāni vinaye sandassiyamānāni sutte ceva osaranti, vinaye ca sandissanti, niṭṭhamettha gantabbaṃ – ‘addhā, idaṃ tassa bhagavato vacanaṃ; imassa ca bhikkhuno suggahita’nti. Idaṃ, bhikkhave, paṭhamaṃ mahāpadesaṃ dhāreyyātha.

‘‘Idha pana, bhikkhave, bhikkhu evaṃ vadeyya – ‘amukasmiṃ nāma āvāse saṅgho viharati sathero sapāmokkho. Tassa me saṅghassa sammukhā sutaṃ sammukhā paṭiggahitaṃ, ayaṃ dhammo ayaṃ vinayo idaṃ satthusāsana’nti. Tassa, bhikkhave, bhikkhuno bhāsitaṃ neva abhinanditabbaṃ nappaṭikkositabbaṃ. Anabhinanditvā appaṭikkositvā tāni padabyañjanāni sādhukaṃ uggahetvā sutte osāretabbāni, vinaye sandassetabbāni. Tāni ce sutte osāriyamānāni vinaye sandassiyamānāni na ceva sutte osaranti, na ca vinaye sandissanti, niṭṭhamettha gantabbaṃ – ‘addhā, idaṃ na ceva tassa bhagavato vacanaṃ; tassa ca saṅghassa duggahita’nti. Itihetaṃ, bhikkhave, chaḍḍeyyātha. Tāni ce sutte osāriyamānāni vinaye sandassiyamānāni sutte ceva osaranti vinaye ca sandissanti, niṭṭhamettha gantabbaṃ – ‘addhā , idaṃ tassa bhagavato vacanaṃ; tassa ca saṅghassa suggahita’nti. Idaṃ, bhikkhave, dutiyaṃ mahāpadesaṃ dhāreyyātha.

‘‘Idha pana, bhikkhave, bhikkhu evaṃ vadeyya – ‘amukasmiṃ nāma āvāse sambahulā therā bhikkhū viharanti bahussutā āgatāgamā dhammadharā vinayadharā mātikādharā. Tesaṃ me therānaṃ sammukhā sutaṃ sammukhā paṭiggahitaṃ – ayaṃ dhammo ayaṃ vinayo idaṃ satthusāsana’nti. Tassa, bhikkhave, bhikkhuno bhāsitaṃ neva abhinanditabbaṃ…pe… na ca vinaye sandissanti, niṭṭhamettha gantabbaṃ – ‘addhā, idaṃ na ceva tassa bhagavato vacanaṃ; tesañca therānaṃ duggahita’nti. Itihetaṃ, bhikkhave, chaḍḍeyyātha. Tāni ce sutte osāriyamānāni…pe… vinaye ca sandissanti, niṭṭhamettha gantabbaṃ – ‘addhā, idaṃ tassa bhagavato vacanaṃ; tesañca therānaṃ suggahita’nti. Idaṃ, bhikkhave, tatiyaṃ mahāpadesaṃ dhāreyyātha.

‘‘Idha pana, bhikkhave, bhikkhu evaṃ vadeyya – ‘amukasmiṃ nāma āvāse eko thero bhikkhu viharati bahussuto āgatāgamo dhammadharo vinayadharo mātikādharo. Tassa me therassa sammukhā sutaṃ sammukhā paṭiggahitaṃ – ayaṃ dhammo ayaṃ vinayo idaṃ satthusāsana’nti. Tassa, bhikkhave, bhikkhuno bhāsitaṃ neva abhinanditabbaṃ nappaṭikkositabbaṃ. Anabhinanditvā appaṭikkositvā tāni padabyañjanāni sādhukaṃ uggahetvā sutte osāritabbāni, vinaye sandassetabbāni. Tāni ce sutte osāriyamānāni vinaye sandassiyamānāni na ceva sutte osaranti, na ca vinaye sandissanti, niṭṭhamettha gantabbaṃ – ‘addhā, idaṃ na ceva tassa bhagavato vacanaṃ; tassa ca therassa duggahita’nti. Itihetaṃ, bhikkhave, chaḍḍeyyātha. Tāni ca sutte osāriyamānāni vinaye sandassiyamānāni sutte ceva osaranti, vinaye ca sandissanti , niṭṭhamettha gantabbaṃ – ‘addhā , idaṃ tassa bhagavato vacanaṃ; tassa ca therassa suggahita’nti. Idaṃ, bhikkhave, catutthaṃ mahāpadesaṃ dhāreyyātha. Ime kho, bhikkhave, cattāro mahāpadese dhāreyyāthā’’ti.

Tatrapi sudaṃ bhagavā bhoganagare viharanto ānande cetiye etadeva bahulaṃ bhikkhūnaṃ dhammiṃ kathaṃ karoti – ‘‘iti sīlaṃ, iti samādhi, iti paññā. Sīlaparibhāvito samādhi mahapphalo hoti mahānisaṃso . Samādhiparibhāvitā paññā mahapphalā hoti mahānisaṃsā. Paññāparibhāvitaṃ cittaṃ sammadeva āsavehi vimuccati, seyyathidaṃ – kāmāsavā, bhavāsavā, avijjāsavā’’ti.

\subsubsection{Kammāraputtacundavatthu}

\paragraph{189.} Atha kho bhagavā bhoganagare yathābhirantaṃ viharitvā āyasmantaṃ ānandaṃ āmantesi – ‘‘āyāmānanda, yena pāvā tenupasaṅkamissāmā’’ti. ‘‘Evaṃ, bhante’’ti kho āyasmā ānando bhagavato paccassosi. Atha kho bhagavā mahatā bhikkhusaṅghena saddhiṃ yena pāvā tadavasari. Tatra sudaṃ bhagavā pāvāyaṃ viharati cundassa kammāraputtassa ambavane. Assosi kho cundo kammāraputto – ‘‘bhagavā kira pāvaṃ anuppatto, pāvāyaṃ viharati mayhaṃ ambavane’’ti. Atha kho cundo kammāraputto yena bhagavā tenupasaṅkami; upasaṅkamitvā bhagavantaṃ abhivādetvā ekamantaṃ nisīdi. Ekamantaṃ nisinnaṃ kho cundaṃ kammāraputtaṃ bhagavā dhammiyā kathāya sandassesi samādapesi samuttejesi sampahaṃsesi. Atha kho cundo kammāraputto bhagavatā dhammiyā kathāya sandassito samādapito samuttejito sampahaṃsito bhagavantaṃ etadavoca – ‘‘adhivāsetu me, bhante, bhagavā svātanāya bhattaṃ saddhiṃ bhikkhusaṅghenā’’ti. Adhivāsesi bhagavā tuṇhībhāvena. Atha kho cundo kammāraputto bhagavato adhivāsanaṃ viditvā uṭṭhāyāsanā bhagavantaṃ abhivādetvā padakkhiṇaṃ katvā pakkāmi.

Atha kho cundo kammāraputto tassā rattiyā accayena sake nivesane paṇītaṃ khādanīyaṃ bhojanīyaṃ paṭiyādāpetvā pahūtañca sūkaramaddavaṃ bhagavato kālaṃ ārocāpesi – ‘‘kālo, bhante, niṭṭhitaṃ bhatta’’nti. Atha kho bhagavā pubbaṇhasamayaṃ nivāsetvā pattacīvaramādāya saddhiṃ bhikkhusaṅghena yena cundassa kammāraputtassa nivesanaṃ tenupasaṅkami; upasaṅkamitvā paññatte āsane nisīdi. Nisajja kho bhagavā cundaṃ kammāraputtaṃ āmantesi – ‘‘yaṃ te, cunda, sūkaramaddavaṃ paṭiyattaṃ, tena maṃ parivisa. Yaṃ panaññaṃ khādanīyaṃ bhojanīyaṃ paṭiyattaṃ, tena bhikkhusaṅghaṃ parivisā’’ti. ‘‘Evaṃ, bhante’’ti kho cundo kammāraputto bhagavato paṭissutvā yaṃ ahosi sūkaramaddavaṃ paṭiyattaṃ, tena bhagavantaṃ parivisi. Yaṃ panaññaṃ khādanīyaṃ bhojanīyaṃ paṭiyattaṃ , tena bhikkhusaṅghaṃ parivisi. Atha kho bhagavā cundaṃ kammāraputtaṃ āmantesi – ‘‘yaṃ te, cunda, sūkaramaddavaṃ avasiṭṭhaṃ, taṃ sobbhe nikhaṇāhi. Nāhaṃ taṃ, cunda, passāmi sadevake loke samārake sabrahmake sassamaṇabrāhmaṇiyā pajāya sadevamanussāya, yassa taṃ paribhuttaṃ sammā pariṇāmaṃ gaccheyya aññatra tathāgatassā’’ti. ‘‘Evaṃ, bhante’’ti kho cundo kammāraputto bhagavato paṭissutvā yaṃ ahosi sūkaramaddavaṃ avasiṭṭhaṃ, taṃ sobbhe nikhaṇitvā yena bhagavā tenupasaṅkami; upasaṅkamitvā bhagavantaṃ abhivādetvā ekamantaṃ nisīdi. Ekamantaṃ nisinnaṃ kho cundaṃ kammāraputtaṃ bhagavā dhammiyā kathāya sandassetvā samādapetvā samuttejetvā sampahaṃsetvā uṭṭhāyāsanā pakkāmi.

\paragraph{190.} Atha kho bhagavato cundassa kammāraputtassa bhattaṃ bhuttāvissa kharo ābādho uppajji, lohitapakkhandikā pabāḷhā vedanā vattanti māraṇantikā. Tā sudaṃ bhagavā sato sampajāno adhivāsesi avihaññamāno. Atha kho bhagavā āyasmantaṃ ānandaṃ āmantesi – ‘‘āyāmānanda, yena kusinārā tenupasaṅkamissāmā’’ti. ‘‘Evaṃ, bhante’’ti kho āyasmā ānando bhagavato paccassosi.

Cundassa bhattaṃ bhuñjitvā, kammārassāti me sutaṃ;

Ābādhaṃ samphusī dhīro, pabāḷhaṃ māraṇantikaṃ.

Bhuttassa ca sūkaramaddavena,

Byādhippabāḷho udapādi satthuno;

Virecamāno\footnote{viriccamāno (sī. syā. ka.), viriñcamāno (?)} bhagavā avoca,

Gacchāmahaṃ kusināraṃ nagaranti.

\subsubsection{Pānīyāharaṇaṃ}

\paragraph{191.} Atha kho bhagavā maggā okkamma yena aññataraṃ rukkhamūlaṃ tenupasaṅkami; upasaṅkamitvā āyasmantaṃ ānandaṃ āmantesi – ‘‘iṅgha me tvaṃ, ānanda, catugguṇaṃ saṅghāṭiṃ paññapehi, kilantosmi, ānanda, nisīdissāmī’’ti. ‘‘Evaṃ, bhante’’ti kho āyasmā ānando bhagavato paṭissutvā catugguṇaṃ saṅghāṭiṃ paññapesi. Nisīdi bhagavā paññatte āsane. Nisajja kho bhagavā āyasmantaṃ ānandaṃ āmantesi – ‘‘iṅgha me tvaṃ, ānanda, pānīyaṃ āhara, pipāsitosmi, ānanda, pivissāmī’’ti. Evaṃ vutte āyasmā ānando bhagavantaṃ etadavoca – ‘‘idāni, bhante, pañcamattāni sakaṭasatāni atikkantāni, taṃ cakkacchinnaṃ udakaṃ parittaṃ luḷitaṃ āvilaṃ sandati. Ayaṃ, bhante, kakudhā\footnote{kakuthā (sī. pī.)} nadī avidūre acchodakā sātodakā sītodakā setodakā\footnote{setakā (sī.)} suppatitthā ramaṇīyā. Ettha bhagavā pānīyañca pivissati, gattāni ca sītī\footnote{sītaṃ (sī. pī. ka.)} karissatī’’ti.

Dutiyampi kho bhagavā āyasmantaṃ ānandaṃ āmantesi – ‘‘iṅgha me tvaṃ, ānanda, pānīyaṃ āhara, pipāsitosmi, ānanda, pivissāmī’’ti. Dutiyampi kho āyasmā ānando bhagavantaṃ etadavoca – ‘‘idāni, bhante, pañcamattāni sakaṭasatāni atikkantāni, taṃ cakkacchinnaṃ udakaṃ parittaṃ luḷitaṃ āvilaṃ sandati. Ayaṃ, bhante, kakudhā nadī avidūre acchodakā sātodakā sītodakā setodakā suppatitthā ramaṇīyā. Ettha bhagavā pānīyañca pivissati, gattāni ca sītīkarissatī’’ti.

Tatiyampi kho bhagavā āyasmantaṃ ānandaṃ āmantesi – ‘‘iṅgha me tvaṃ, ānanda, pānīyaṃ āhara, pipāsitosmi, ānanda, pivissāmī’’ti. ‘‘Evaṃ, bhante’’ti kho āyasmā ānando bhagavato paṭissutvā pattaṃ gahetvā yena sā nadikā tenupasaṅkami. Atha kho sā nadikā cakkacchinnā parittā luḷitā āvilā sandamānā, āyasmante ānande upasaṅkamante acchā vippasannā anāvilā sandittha\footnote{sandati (syā.)}. Atha kho āyasmato ānandassa etadahosi – ‘‘acchariyaṃ vata, bho, abbhutaṃ vata, bho, tathāgatassa mahiddhikatā mahānubhāvatā. Ayañhi sā nadikā cakkacchinnā parittā luḷitā āvilā sandamānā mayi upasaṅkamante acchā vippasannā anāvilā sandatī’’ti. Pattena pānīyaṃ ādāya yena bhagavā tenupasaṅkami; upasaṅkamitvā bhagavantaṃ etadavoca – ‘‘acchariyaṃ, bhante, abbhutaṃ, bhante, tathāgatassa mahiddhikatā mahānubhāvatā. Idāni sā bhante nadikā cakkacchinnā parittā luḷitā āvilā sandamānā mayi upasaṅkamante acchā vippasannā anāvilā sandittha. Pivatu bhagavā pānīyaṃ pivatu sugato pānīya’’nti. Atha kho bhagavā pānīyaṃ apāyi.

\subsubsection{Pukkusamallaputtavatthu}

\paragraph{192.} Tena rokho pana samayena pukkuso mallaputto āḷārassa kālāmassa sāvako kusinārāya pāvaṃ addhānamaggappaṭippanno hoti. Addasā kho pukkuso mallaputto bhagavantaṃ aññatarasmiṃ rukkhamūle nisinnaṃ. Disvā yena bhagavā tenupasaṅkami; upasaṅkamitvā bhagavantaṃ abhivādetvā ekamantaṃ nisīdi. Ekamantaṃ nisinno kho pukkuso mallaputto bhagavantaṃ etadavoca – ‘‘acchariyaṃ, bhante, abbhutaṃ, bhante, santena vata, bhante, pabbajitā vihārena viharanti. Bhūtapubbaṃ, bhante , āḷāro kālāmo addhānamaggappaṭippanno maggā okkamma avidūre aññatarasmiṃ rukkhamūle divāvihāraṃ nisīdi. Atha kho, bhante, pañcamattāni sakaṭasatāni āḷāraṃ kālāmaṃ nissāya nissāya atikkamiṃsu. Atha kho, bhante, aññataro puriso tassa sakaṭasatthassa\footnote{sakaṭasatassa (ka.)} piṭṭhito piṭṭhito āgacchanto yena āḷāro kālāmo tenupasaṅkami; upasaṅkamitvā āḷāraṃ kālāmaṃ etadavoca – ‘api, bhante, pañcamattāni sakaṭasatāni atikkantāni addasā’ti? ‘Na kho ahaṃ, āvuso, addasa’nti. ‘Kiṃ pana, bhante, saddaṃ assosī’ti? ‘Na kho ahaṃ, āvuso, saddaṃ assosi’nti. ‘Kiṃ pana, bhante, sutto ahosī’ti? ‘Na kho ahaṃ, āvuso, sutto ahosi’nti. ‘Kiṃ pana, bhante, saññī ahosī’ti? ‘Evamāvuso’ti. ‘So tvaṃ, bhante, saññī samāno jāgaro pañcamattāni sakaṭasatāni nissāya nissāya atikkantāni neva addasa, na pana saddaṃ assosi; apisu\footnote{api hi (sī. syā. pī.)} te, bhante, saṅghāṭi rajena okiṇṇā’ti? ‘Evamāvuso’ti. Atha kho, bhante, tassa purisassa etadahosi – ‘acchariyaṃ vata bho, abbhutaṃ vata bho, santena vata bho pabbajitā vihārena viharanti. Yatra hi nāma saññī samāno jāgaro pañcamattāni sakaṭasatāni nissāya nissāya atikkantāni neva dakkhati, na pana saddaṃ sossatī’ti! Āḷāre kālāme uḷāraṃ pasādaṃ pavedetvā pakkāmī’’ti.

\paragraph{193.} ‘‘Taṃ kiṃ maññasi, pukkusa, katamaṃ nu kho dukkarataraṃ vā durabhisambhavataraṃ vā – yo vā saññī samāno jāgaro pañcamattāni sakaṭasatāni nissāya nissāya atikkantāni neva passeyya, na pana saddaṃ suṇeyya; yo vā saññī samāno jāgaro deve vassante deve gaḷagaḷāyante vijjullatāsu\footnote{vijjutāsu (sī. syā. pī.)} niccharantīsu asaniyā phalantiyā neva passeyya, na pana saddaṃ suṇeyyā’’ti? ‘‘Kiñhi, bhante, karissanti pañca vā sakaṭasatāni cha vā sakaṭasatāni satta vā sakaṭasatāni aṭṭha vā sakaṭasatāni nava vā sakaṭasatāni\footnote{nava vā sakaṭasatāni dasa vā sakaṭasatāni (sī.)}, sakaṭasahassaṃ vā sakaṭasatasahassaṃ vā. Atha kho etadeva dukkarataraṃ ceva durabhisambhavatarañca yo saññī samāno jāgaro deve vassante deve gaḷagaḷāyante vijjullatāsu niccharantīsu asaniyā phalantiyā neva passeyya, na pana saddaṃ suṇeyyā’’ti.

‘‘Ekamidāhaṃ, pukkusa, samayaṃ ātumāyaṃ viharāmi bhusāgāre. Tena kho pana samayena deve vassante deve gaḷagaḷāyante vijjullatāsu niccharantīsu asaniyā phalantiyā avidūre bhusāgārassa dve kassakā bhātaro hatā cattāro ca balibaddā\footnote{balibaddā (sī. pī.)}. Atha kho, pukkusa, ātumāya mahājanakāyo nikkhamitvā yena te dve kassakā bhātaro hatā cattāro ca balibaddā tenupasaṅkami. Tena kho panāhaṃ, pukkusa, samayena bhusāgārā nikkhamitvā bhusāgāradvāre abbhokāse caṅkamāmi. Atha kho, pukkusa, aññataro puriso tamhā mahājanakāyā yenāhaṃ tenupasaṅkami; upasaṅkamitvā maṃ abhivādetvā ekamantaṃ aṭṭhāsi. Ekamantaṃ ṭhitaṃ kho ahaṃ, pukkusa, taṃ purisaṃ etadavocaṃ – ‘kiṃ nu kho eso, āvuso, mahājanakāyo sannipatito’ti? ‘Idāni , bhante, deve vassante deve gaḷagaḷāyante vijjullatāsu niccharantīsu asaniyā phalantiyā dve kassakā bhātaro hatā cattāro ca balibaddā. Ettheso mahājanakāyo sannipatito. Tvaṃ pana, bhante, kva ahosī’ti? ‘Idheva kho ahaṃ, āvuso, ahosi’nti. ‘Kiṃ pana, bhante, addasā’ti? ‘Na kho ahaṃ, āvuso, addasa’nti. ‘Kiṃ pana, bhante, saddaṃ assosī’ti? ‘Na kho ahaṃ, āvuso, saddaṃ assosi’nti. ‘Kiṃ pana, bhante, sutto ahosī’ti? ‘Na kho ahaṃ, āvuso, sutto ahosi’nti. ‘Kiṃ pana, bhante, saññī ahosī’ti? ‘Evamāvuso’ti. ‘So tvaṃ, bhante, saññī samāno jāgaro deve vassante deve gaḷagaḷāyante vijjullatāsu niccharantīsu asaniyā phalantiyā neva addasa, na pana saddaṃ assosī’ti? ‘‘Evamāvuso’’ti?

‘‘Atha kho, pukkusa, purisassa etadahosi – ‘acchariyaṃ vata bho, abbhutaṃ vata bho, santena vata bho pabbajitā vihārena viharanti. Yatra hi nāma saññī samāno jāgaro deve vassante deve gaḷagaḷāyante vijjullatāsu niccharantīsu asaniyā phalantiyā neva dakkhati, na pana saddaṃ sossatī’ti\footnote{suṇissati (syā.)}. Mayi uḷāraṃ pasādaṃ pavedetvā maṃ abhivādetvā padakkhiṇaṃ katvā pakkāmī’’ti.

Evaṃ vutte pukkuso mallaputto bhagavantaṃ etadavoca – ‘‘esāhaṃ, bhante, yo me āḷāre kālāme pasādo taṃ mahāvāte vā ophuṇāmi sīghasotāya\footnote{siṅghasotāya (ka.)} vā nadiyā pavāhemi. Abhikkantaṃ, bhante, abhikkantaṃ, bhante! Seyyathāpi, bhante, nikkujjitaṃ vā ukkujjeyya, paṭicchannaṃ vā vivareyya, mūḷhassa vā maggaṃ ācikkheyya, andhakāre vā telapajjotaṃ dhāreyya ‘cakkhumanto rūpāni dakkhantī’ti; evamevaṃ bhagavatā anekapariyāyena dhammo pakāsito. Esāhaṃ, bhante, bhagavantaṃ saraṇaṃ gacchāmi dhammañca bhikkhusaṅghañca. Upāsakaṃ maṃ bhagavā dhāretu ajjatagge pāṇupetaṃ saraṇaṃ gata’’nti.

\paragraph{194.} Atha kho pukkuso mallaputto aññataraṃ purisaṃ āmantesi – ‘‘iṅgha me tvaṃ, bhaṇe, siṅgīvaṇṇaṃ yugamaṭṭhaṃ dhāraṇīyaṃ āharā’’ti. ‘‘Evaṃ, bhante’’ti kho so puriso pukkusassa mallaputtassa paṭissutvā taṃ siṅgīvaṇṇaṃ yugamaṭṭhaṃ dhāraṇīyaṃ āhari\footnote{āharasi (ka.)}. Atha kho pukkuso mallaputto taṃ siṅgīvaṇṇaṃ yugamaṭṭhaṃ dhāraṇīyaṃ bhagavato upanāmesi – ‘‘idaṃ, bhante, siṅgīvaṇṇaṃ yugamaṭṭhaṃ dhāraṇīyaṃ, taṃ me bhagavā paṭiggaṇhātu anukampaṃ upādāyā’’ti. ‘‘Tena hi, pukkusa, ekena maṃ acchādehi, ekena ānanda’’nti. ‘‘Evaṃ, bhante’’ti kho pukkuso mallaputto bhagavato paṭissutvā ekena bhagavantaṃ acchādeti, ekena āyasmantaṃ ānandaṃ. Atha kho bhagavā pukkusaṃ mallaputtaṃ dhammiyā kathāya sandassesi samādapesi samuttejesi sampahaṃsesi. Atha kho pukkuso mallaputto bhagavatā dhammiyā kathāya sandassito samādapito samuttejito sampahaṃsito uṭṭhāyāsanā bhagavantaṃ abhivādetvā padakkhiṇaṃ katvā pakkāmi.

\paragraph{195.} Atha kho āyasmā ānando acirapakkante pukkuse mallaputte taṃ siṅgīvaṇṇaṃ yugamaṭṭhaṃ dhāraṇīyaṃ bhagavato kāyaṃ upanāmesi. Taṃ bhagavato kāyaṃ upanāmitaṃ hataccikaṃ viya\footnote{vītaccikaṃviya (sī. pī.)} khāyati. Atha kho āyasmā ānando bhagavantaṃ etadavoca – ‘‘acchariyaṃ, bhante, abbhutaṃ, bhante, yāva parisuddho, bhante, tathāgatassa chavivaṇṇo pariyodāto. Idaṃ, bhante, siṅgīvaṇṇaṃ yugamaṭṭhaṃ dhāraṇīyaṃ bhagavato kāyaṃ upanāmitaṃ hataccikaṃ viya khāyatī’’ti. ‘‘Evametaṃ, ānanda, evametaṃ, ānanda dvīsu kālesu ativiya tathāgatassa kāyo parisuddho hoti chavivaṇṇo pariyodāto. Katamesu dvīsu? Yañca, ānanda, rattiṃ tathāgato anuttaraṃ sammāsambodhiṃ abhisambujjhati, yañca rattiṃ anupādisesāya nibbānadhātuyā parinibbāyati. Imesu kho, ānanda, dvīsu kālesu ativiya tathāgatassa kāyo parisuddho hoti chavivaṇṇo pariyodāto. ‘‘Ajja kho, panānanda, rattiyā pacchime yāme kusinārāyaṃ upavattane mallānaṃ sālavane antarena\footnote{antare (syā.)} yamakasālānaṃ tathāgatassa parinibbānaṃ bhavissati\footnote{bhavissatīti (ka.)}. Āyāmānanda, yena kakudhā nadī tenupasaṅkamissāmā’’ti. ‘‘Evaṃ, bhante’’ti kho āyasmā ānando bhagavato paccassosi.

Siṅgīvaṇṇaṃ yugamaṭṭhaṃ, pukkuso abhihārayi;

Tena acchādito satthā, hemavaṇṇo asobhathāti.

\paragraph{196.} Atha kho bhagavā mahatā bhikkhusaṅghena saddhiṃ yena kakudhā nadī tenupasaṅkami ; upasaṅkamitvā kakudhaṃ nadiṃ ajjhogāhetvā nhatvā ca pivitvā ca paccuttaritvā yena ambavanaṃ tenupasaṅkami. Upasaṅkamitvā āyasmantaṃ cundakaṃ āmantesi – ‘‘iṅgha me tvaṃ, cundaka, catugguṇaṃ saṅghāṭiṃ paññapehi, kilantosmi, cundaka, nipajjissāmī’’ti.

‘‘Evaṃ, bhante’’ti kho āyasmā cundako bhagavato paṭissutvā catugguṇaṃ saṅghāṭiṃ paññapesi. Atha kho bhagavā dakkhiṇena passena sīhaseyyaṃ kappesi pāde pādaṃ accādhāya sato sampajāno uṭṭhānasaññaṃ manasikaritvā. Āyasmā pana cundako tattheva bhagavato purato nisīdi.

Gantvāna buddho nadikaṃ kakudhaṃ,

Acchodakaṃ sātudakaṃ vippasannaṃ;

Ogāhi satthā akilantarūpo\footnote{sukilantarūpo (sī. pī.)},

Tathāgato appaṭimo ca\footnote{appaṭimodha (pī.)} loke.

Nhatvā ca pivitvā cudatāri satthā\footnote{pivitvā cundakena, pivitvā ca uttari (ka.)},

Purakkhato bhikkhugaṇassa majjhe;

Vattā\footnote{satthā (sī. syā. pī.)} pavattā bhagavā idha dhamme,

Upāgami ambavanaṃ mahesi.

Āmantayi cundakaṃ nāma bhikkhuṃ,

Catugguṇaṃ santhara me nipajjaṃ;

So codito bhāvitattena cundo,

Catugguṇaṃ santhari khippameva.

Nipajji satthā akilantarūpo,

Cundopi tattha pamukhe\footnote{samukhe (ka.)} nisīdīti.

\paragraph{197.} Atha kho bhagavā āyasmantaṃ ānandaṃ āmantesi – ‘‘siyā kho\footnote{yo kho (ka.)}, panānanda, cundassa kammāraputtassa koci vippaṭisāraṃ uppādeyya – ‘tassa te, āvuso cunda, alābhā tassa te dulladdhaṃ, yassa te tathāgato pacchimaṃ piṇḍapātaṃ paribhuñjitvā parinibbuto’ti. Cundassa, ānanda, kammāraputtassa evaṃ vippaṭisāro paṭivinetabbo – ‘tassa te, āvuso cunda, lābhā tassa te suladdhaṃ, yassa te tathāgato pacchimaṃ piṇḍapātaṃ paribhuñjitvā parinibbuto. Sammukhā metaṃ, āvuso cunda, bhagavato sutaṃ sammukhā paṭiggahitaṃ – dve me piṇḍapātā samasamaphalā\footnote{samā samaphalā (ka.)} samavipākā\footnote{samasamavipākā (sī. syā. pī.)}, ativiya aññehi piṇḍapātehi mahapphalatarā ca mahānisaṃsatarā ca. Katame dve? Yañca piṇḍapātaṃ paribhuñjitvā tathāgato anuttaraṃ sammāsambodhiṃ abhisambujjhati, yañca piṇḍapātaṃ paribhuñjitvā tathāgato anupādisesāya nibbānadhātuyā parinibbāyati. Ime dve piṇḍapātā samasamaphalā samavipākā , ativiya aññehi piṇḍapātehi mahapphalatarā ca mahānisaṃsatarā ca. Āyusaṃvattanikaṃ āyasmatā cundena kammāraputtena kammaṃ upacitaṃ, vaṇṇasaṃvattanikaṃ āyasmatā cundena kammāraputtena kammaṃ upacitaṃ, sukhasaṃvattanikaṃ āyasmatā cundena kammāraputtena kammaṃ upacitaṃ, yasasaṃvattanikaṃ āyasmatā cundena kammāraputtena kammaṃ upacitaṃ, saggasaṃvattanikaṃ āyasmatā cundena kammāraputtena kammaṃ upacitaṃ, ādhipateyyasaṃvattanikaṃ āyasmatā cundena kammāraputtena kammaṃ upacita’nti. Cundassa, ānanda, kammāraputtassa evaṃ vippaṭisāro paṭivinetabbo’’ti. Atha kho bhagavā etamatthaṃ viditvā tāyaṃ velāyaṃ imaṃ udānaṃ udānesi –

‘‘Dadato puññaṃ pavaḍḍhati,

Saṃyamato veraṃ na cīyati;

Kusalo ca jahāti pāpakaṃ,

Rāgadosamohakkhayā sanibbuto’’ti.

\xsubsubsectionEnd{Catuttho bhāṇavāro.}

\subsubsection{Yamakasālā}

\paragraph{198.} Atha kho bhagavā āyasmantaṃ ānandaṃ āmantesi – ‘‘āyāmānanda, yena hiraññavatiyā nadiyā pārimaṃ tīraṃ, yena kusinārā upavattanaṃ mallānaṃ sālavanaṃ tenupasaṅkamissāmā’’ti . ‘‘Evaṃ, bhante’’ti kho āyasmā ānando bhagavato paccassosi. Atha kho bhagavā mahatā bhikkhusaṅghena saddhiṃ yena hiraññavatiyā nadiyā pārimaṃ tīraṃ, yena kusinārā upavattanaṃ mallānaṃ sālavanaṃ tenupasaṅkami. Upasaṅkamitvā āyasmantaṃ ānandaṃ āmantesi – ‘‘iṅgha me tvaṃ, ānanda, antarena yamakasālānaṃ uttarasīsakaṃ mañcakaṃ paññapehi, kilantosmi, ānanda, nipajjissāmī’’ti. ‘‘Evaṃ, bhante’’ti kho āyasmā ānando bhagavato paṭissutvā antarena yamakasālānaṃ uttarasīsakaṃ mañcakaṃ paññapesi. Atha kho bhagavā dakkhiṇena passena sīhaseyyaṃ kappesi pāde pādaṃ accādhāya sato sampajāno.

Tena kho pana samayena yamakasālā sabbaphāliphullā honti akālapupphehi. Te tathāgatassa sarīraṃ okiranti ajjhokiranti abhippakiranti tathāgatassa pūjāya. Dibbānipi mandāravapupphāni antalikkhā papatanti, tāni tathāgatassa sarīraṃ okiranti ajjhokiranti abhippakiranti tathāgatassa pūjāya. Dibbānipi candanacuṇṇāni antalikkhā papatanti, tāni tathāgatassa sarīraṃ okiranti ajjhokiranti abhippakiranti tathāgatassa pūjāya. Dibbānipi tūriyāni antalikkhe vajjanti tathāgatassa pūjāya. Dibbānipi saṅgītāni antalikkhe vattanti tathāgatassa pūjāya.

\paragraph{199.} Atha kho bhagavā āyasmantaṃ ānandaṃ āmantesi – ‘‘sabbaphāliphullā kho, ānanda, yamakasālā akālapupphehi. Te tathāgatassa sarīraṃ okiranti ajjhokiranti abhippakiranti tathāgatassa pūjāya. Dibbānipi mandāravapupphāni antalikkhā papatanti, tāni tathāgatassa sarīraṃ okiranti ajjhokiranti abhippakiranti tathāgatassa pūjāya. Dibbānipi candanacuṇṇāni antalikkhā papatanti, tāni tathāgatassa sarīraṃ okiranti ajjhokiranti abhippakiranti tathāgatassa pūjāya. Dibbānipi tūriyāni antalikkhe vajjanti tathāgatassa pūjāya. Dibbānipi saṅgītāni antalikkhe vattanti tathāgatassa pūjāya. Na kho, ānanda, ettāvatā tathāgato sakkato vā hoti garukato vā mānito vā pūjito vā apacito vā. Yo kho, ānanda, bhikkhu vā bhikkhunī vā upāsako vā upāsikā vā dhammānudhammappaṭipanno viharati sāmīcippaṭipanno anudhammacārī, so tathāgataṃ sakkaroti garuṃ karoti māneti pūjeti apaciyati\footnote{idaṃ padaṃ sīsyāipotthakesu na dissati}, paramāya pūjāya. Tasmātihānanda, dhammānudhammappaṭipannā viharissāma sāmīcippaṭipannā anudhammacārinoti. Evañhi vo, ānanda, sikkhitabba’’nti.

\subsubsection{Upavāṇatthero}

\paragraph{200.} Tena kho pana samayena āyasmā upavāṇo bhagavato purato ṭhito hoti bhagavantaṃ bījayamāno. Atha kho bhagavā āyasmantaṃ upavāṇaṃ apasāresi – ‘‘apehi, bhikkhu, mā me purato aṭṭhāsī’’ti. Atha kho āyasmato ānandassa etadahosi – ‘‘ayaṃ kho āyasmā upavāṇo dīgharattaṃ bhagavato upaṭṭhāko santikāvacaro samīpacārī. Atha ca pana bhagavā pacchime kāle āyasmantaṃ upavāṇaṃ apasāreti – ‘apehi bhikkhu, mā me purato aṭṭhāsī’ti. Ko nu kho hetu, ko paccayo, yaṃ bhagavā āyasmantaṃ upavāṇaṃ apasāreti – ‘apehi, bhikkhu, mā me purato aṭṭhāsī’ti? Atha kho āyasmā ānando bhagavantaṃ etadavoca – ‘ayaṃ, bhante, āyasmā upavāṇo dīgharattaṃ bhagavato upaṭṭhāko santikāvacaro samīpacārī. Atha ca pana bhagavā pacchime kāle āyasmantaṃ upavāṇaṃ apasāreti – ‘‘apehi, bhikkhu, mā me purato aṭṭhāsī’’ti. Ko nu kho, bhante, hetu, ko paccayo, yaṃ bhagavā āyasmantaṃ upavāṇaṃ apasāreti – ‘‘apehi, bhikkhu, mā me purato aṭṭhāsī’’ti? ‘‘Yebhuyyena, ānanda, dasasu lokadhātūsu devatā sannipatitā tathāgataṃ dassanāya. Yāvatā, ānanda, kusinārā upavattanaṃ mallānaṃ sālavanaṃ samantato dvādasa yojanāni, natthi so padeso vālaggakoṭinitudanamattopi mahesakkhāhi devatāhi apphuṭo. Devatā, ānanda, ujjhāyanti – ‘dūrā ca vatamha āgatā tathāgataṃ dassanāya. Kadāci karahaci tathāgatā loke uppajjanti arahanto sammāsambuddhā. Ajjeva rattiyā pacchime yāme tathāgatassa parinibbānaṃ bhavissati. Ayañca mahesakkho bhikkhu bhagavato purato ṭhito ovārento, na mayaṃ labhāma pacchime kāle tathāgataṃ dassanāyā’’’ti.

\paragraph{201.} ‘‘Kathaṃbhūtā pana, bhante, bhagavā devatā manasikarotī’’ti\footnote{manasi karontīti (syā. ka.)}? ‘‘Santānanda, devatā ākāse pathavīsaññiniyo kese pakiriya kandanti, bāhā paggayha kandanti, chinnapātaṃ papatanti\footnote{chinnaṃpādaṃviya papatanti (syā.)}, āvaṭṭanti, vivaṭṭanti – ‘atikhippaṃ bhagavā parinibbāyissati, atikhippaṃ sugato parinibbāyissati, atikhippaṃ cakkhuṃ\footnote{cakkhumā (syā. ka.)} loke antaradhaṃāyissatī’ti.

‘‘Santānanda, devatā pathaviyaṃ pathavīsaññiniyo kese pakiriya kandanti, bāhā paggayha kandanti, chinnapātaṃ papatanti, āvaṭṭanti, vivaṭṭanti – ‘atikhippaṃ bhagavā parinibbāyissati, atikhippaṃ sugato parinibbāyissati, atikhippaṃ cakkhuṃ loke antaradhāyissatī’’’ti.

‘‘Yā pana tā devatā vītarāgā, tā satā sampajānā adhivāsenti – ‘aniccā saṅkhārā, taṃ kutettha labbhā’ti.

\subsubsection{Catusaṃvejanīyaṭṭhānāni}

\paragraph{202.} ‘‘Pubbe , bhante, disāsu vassaṃ vuṭṭhā\footnote{vassaṃvutthā (sī. syā. kaṃ. pī.)} bhikkhū āgacchanti tathāgataṃ dassanāya. Te mayaṃ labhāma manobhāvanīye bhikkhū dassanāya, labhāma payirupāsanāya. Bhagavato pana mayaṃ, bhante, accayena na labhissāma manobhāvanīye bhikkhū dassanāya, na labhissāma payirupāsanāyā’’ti.

‘‘Cattārimāni, ānanda, saddhassa kulaputtassa dassanīyāni saṃvejanīyāni ṭhānāni. Katamāni cattāri? ‘Idha tathāgato jāto’ti, ānanda, saddhassa kulaputtassa dassanīyaṃ saṃvejanīyaṃ ṭhānaṃ. ‘Idha tathāgato anuttaraṃ sammāsambodhiṃ abhisambuddho’ti, ānanda, saddhassa kulaputtassa dassanīyaṃ saṃvejanīyaṃ ṭhānaṃ. ‘Idha tathāgatena anuttaraṃ dhammacakkaṃ pavattita’nti, ānanda, saddhassa kulaputtassa dassanīyaṃ saṃvejanīyaṃ ṭhānaṃ. ‘Idha tathāgato anupādisesāya nibbānadhātuyā parinibbuto’ti, ānanda, saddhassa kulaputtassa dassanīyaṃ saṃvejanīyaṃ ṭhānaṃ. Imāni kho , ānanda, cattāri saddhassa kulaputtassa dassanīyāni saṃvejanīyāni ṭhānāni.

‘‘Āgamissanti kho, ānanda, saddhā bhikkhū bhikkhuniyo upāsakā upāsikāyo – ‘idha tathāgato jāto’tipi, ‘idha tathāgato anuttaraṃ sammāsambodhiṃ abhisambuddho’tipi, ‘idha tathāgatena anuttaraṃ dhammacakkaṃ pavattita’ntipi, ‘idha tathāgato anupādisesāya nibbānadhātuyā parinibbuto’tipi. Ye hi keci, ānanda, cetiyacārikaṃ āhiṇḍantā pasannacittā kālaṅkarissanti, sabbe te kāyassa bhedā paraṃ maraṇā sugatiṃ saggaṃ lokaṃ upapajjissantī’’ti.

\subsubsection{Ānandapucchākathā}

\paragraph{203.} ‘‘Kathaṃ mayaṃ, bhante, mātugāme paṭipajjāmā’’ti? ‘‘Adassanaṃ, ānandā’’ti. ‘‘Dassane, bhagavā, sati kathaṃ paṭipajjitabba’’nti? ‘‘Anālāpo, ānandā’’ti . ‘‘Ālapantena pana, bhante, kathaṃ paṭipajjitabba’’nti? ‘‘Sati, ānanda, upaṭṭhāpetabbā’’ti.

\paragraph{204.} ‘‘Kathaṃ mayaṃ, bhante, tathāgatassa sarīre paṭipajjāmā’’ti? ‘‘Abyāvaṭā tumhe, ānanda, hotha tathāgatassa sarīrapūjāya. Iṅgha tumhe, ānanda, sāratthe ghaṭatha anuyuñjatha\footnote{sadatthe anuyuñjatha (sī. syā.), sadatthaṃ anuyuñjatha (pī.), sāratthe anuyuñjatha (ka.)}, sāratthe appamattā ātāpino pahitattā viharatha. Santānanda, khattiyapaṇḍitāpi brāhmaṇapaṇḍitāpi gahapatipaṇḍitāpi tathāgate abhippasannā, te tathāgatassa sarīrapūjaṃ karissantī’’ti.

\paragraph{205.} ‘‘Kathaṃ pana, bhante, tathāgatassa sarīre paṭipajjitabba’’nti? ‘‘Yathā kho, ānanda, rañño cakkavattissa sarīre paṭipajjanti, evaṃ tathāgatassa sarīre paṭipajjitabba’’nti. ‘‘Kathaṃ pana, bhante, rañño cakkavattissa sarīre paṭipajjantī’’ti? ‘‘Rañño, ānanda, cakkavattissa sarīraṃ ahatena vatthena veṭhenti, ahatena vatthena veṭhetvā vihatena kappāsena veṭhenti, vihatena kappāsena veṭhetvā ahatena vatthena veṭhenti. Etenupāyena pañcahi yugasatehi rañño cakkavattissa sarīraṃ\footnote{sarīre (syā. ka.)} veṭhetvā āyasāya teladoṇiyā pakkhipitvā aññissā āyasāya doṇiyā paṭikujjitvā sabbagandhānaṃ citakaṃ karitvā rañño cakkavattissa sarīraṃ jhāpenti. Cātumahāpathe\footnote{cātummahāpathe (sī. syā. kaṃ. pī.)} rañño cakkavattissa thūpaṃ karonti . Evaṃ kho, ānanda, rañño cakkavattissa sarīre paṭipajjanti. Yathā kho, ānanda, rañño cakkavattissa sarīre paṭipajjanti, evaṃ tathāgatassa sarīre paṭipajjitabbaṃ. Cātumahāpathe tathāgatassa thūpo kātabbo. Tattha ye mālaṃ vā gandhaṃ vā cuṇṇakaṃ\footnote{vaṇṇakaṃ (sī. pī.)} vā āropessanti vā abhivādessanti vā cittaṃ vā pasādessanti tesaṃ taṃ bhavissati dīgharattaṃ hitāya sukhāya.

\subsubsection{Thūpārahapuggalo}

\paragraph{206.} ‘‘Cattārome, ānanda, thūpārahā. Katame cattāro? Tathāgato arahaṃ sammāsambuddho thūpāraho, paccekasambuddho thūpāraho, tathāgatassa sāvako thūpāraho, rājā cakkavattī\footnote{cakkavatti (syā. ka.)} thūpārahoti.

‘‘Kiñcānanda , atthavasaṃ paṭicca tathāgato arahaṃ sammāsambuddho thūpāraho? ‘Ayaṃ tassa bhagavato arahato sammāsambuddhassa thūpo’ti, ānanda, bahujanā cittaṃ pasādenti. Te tattha cittaṃ pasādetvā kāyassa bhedā paraṃ maraṇā sugatiṃ saggaṃ lokaṃ upapajjanti. Idaṃ kho, ānanda, atthavasaṃ paṭicca tathāgato arahaṃ sammāsambuddho thūpāraho.

‘‘Kiñcānanda, atthavasaṃ paṭicca paccekasambuddho thūpāraho? ‘Ayaṃ tassa bhagavato paccekasambuddhassa thūpo’ti, ānanda, bahujanā cittaṃ pasādenti. Te tattha cittaṃ pasādetvā kāyassa bhedā paraṃ maraṇā sugatiṃ saggaṃ lokaṃ upapajjanti. Idaṃ kho, ānanda, atthavasaṃ paṭicca paccekasambuddho thūpāraho.

‘‘Kiñcānanda, atthavasaṃ paṭicca tathāgatassa sāvako thūpāraho? ‘Ayaṃ tassa bhagavato arahato sammāsambuddhassa sāvakassa thūpo’ti ānanda, bahujanā cittaṃ pasādenti. Te tattha cittaṃ pasādetvā kāyassa bhedā paraṃ maraṇā sugatiṃ saggaṃ lokaṃ upapajjanti. Idaṃ kho, ānanda, atthavasaṃ paṭicca tathāgatassa sāvako thūpāraho.

‘‘Kiñcānanda, atthavasaṃ paṭicca rājā cakkavattī thūpāraho? ‘Ayaṃ tassa dhammikassa dhammarañño thūpo’ti, ānanda, bahujanā cittaṃ pasādenti. Te tattha cittaṃ pasādetvā kāyassa bhedā paraṃ maraṇā sugatiṃ saggaṃ lokaṃ upapajjanti. Idaṃ kho, ānanda, atthavasaṃ paṭicca rājā cakkavattī thūpāraho. Ime kho, ānanda cattāro thūpārahā’’ti.

\subsubsection{Ānandaacchariyadhammo}

\paragraph{207.} Atha kho āyasmā ānando vihāraṃ pavisitvā kapisīsaṃ ālambitvā rodamāno aṭṭhāsi – ‘‘ahañca vatamhi sekho sakaraṇīyo, satthu ca me parinibbānaṃ bhavissati, yo mama anukampako’’ti. Atha kho bhagavā bhikkhū āmantesi – ‘‘kahaṃ nu kho, bhikkhave, ānando’’ti? ‘‘Eso, bhante, āyasmā ānando vihāraṃ pavisitvā kapisīsaṃ ālambitvā rodamāno ṭhito – ‘ahañca vatamhi sekho sakaraṇīyo, satthu ca me parinibbānaṃ bhavissati, yo mama anukampako’’’ti. Atha kho bhagavā aññataraṃ bhikkhuṃ āmantesi – ‘‘ehi tvaṃ, bhikkhu, mama vacanena ānandaṃ āmantehi – ‘satthā taṃ, āvuso ānanda, āmantetī’’’ti. ‘‘Evaṃ , bhante’’ti kho so bhikkhu bhagavato paṭissutvā yenāyasmā ānando tenupasaṅkami; upasaṅkamitvā āyasmantaṃ ānandaṃ etadavoca – ‘‘satthā taṃ, āvuso ānanda, āmantetī’’ti. ‘‘Evamāvuso’’ti kho āyasmā ānando tassa bhikkhuno paṭissutvā yena bhagavā tenupasaṅkami; upasaṅkamitvā bhagavantaṃ abhivādetvā ekamantaṃ nisīdi. Ekamantaṃ nisinnaṃ kho āyasmantaṃ ānandaṃ bhagavā etadavoca – ‘‘alaṃ, ānanda, mā soci mā paridevi, nanu etaṃ, ānanda, mayā paṭikacceva akkhātaṃ – ‘sabbeheva piyehi manāpehi nānābhāvo vinābhāvo aññathābhāvo’; taṃ kutettha, ānanda, labbhā. Yaṃ taṃ jātaṃ bhūtaṃ saṅkhataṃ palokadhammaṃ, taṃ vata tathāgatassāpi sarīraṃ mā palujjī’ti netaṃ ṭhānaṃ vijjati. Dīgharattaṃ kho te, ānanda, tathāgato paccupaṭṭhito mettena kāyakammena hitena sukhena advayena appamāṇena, mettena vacīkammena hitena sukhena advayena appamāṇena, mettena manokammena hitena sukhena advayena appamāṇena. Katapuññosi tvaṃ, ānanda, padhānamanuyuñja, khippaṃ hohisi anāsavo’’ti.

\paragraph{208.} Atha kho bhagavā bhikkhū āmantesi – ‘‘yepi te, bhikkhave, ahesuṃ atītamaddhānaṃ arahanto sammāsambuddhā, tesampi bhagavantānaṃ etapparamāyeva upaṭṭhākā ahesuṃ, seyyathāpi mayhaṃ ānando. Yepi te, bhikkhave, bhavissanti anāgatamaddhānaṃ arahanto sammāsambuddhā, tesampi bhagavantānaṃ etapparamāyeva upaṭṭhākā bhavissanti, seyyathāpi mayhaṃ ānando. Paṇḍito, bhikkhave, ānando; medhāvī, bhikkhave, ānando. Jānāti ‘ayaṃ kālo tathāgataṃ dassanāya upasaṅkamituṃ bhikkhūnaṃ, ayaṃ kālo bhikkhunīnaṃ, ayaṃ kālo upāsakānaṃ , ayaṃ kālo upāsikānaṃ, ayaṃ kālo rañño rājamahāmattānaṃ titthiyānaṃ titthiyasāvakāna’nti.

\paragraph{209.} ‘‘Cattārome, bhikkhave, acchariyā abbhutā dhammā\footnote{abbhutadhammā (syā. ka.)} ānande. Katame cattāro? Sace, bhikkhave, bhikkhuparisā ānandaṃ dassanāya upasaṅkamati, dassanena sā attamanā hoti. Tatra ce ānando dhammaṃ bhāsati, bhāsitenapi sā attamanā hoti. Atittāva, bhikkhave, bhikkhuparisā hoti, atha kho ānando tuṇhī hoti. Sace, bhikkhave, bhikkhunīparisā ānandaṃ dassanāya upasaṅkamati, dassanena sā attamanā hoti. Tatra ce ānando dhammaṃ bhāsati, bhāsitenapi sā attamanā hoti. Atittāva, bhikkhave, bhikkhunīparisā hoti, atha kho ānando tuṇhī hoti. Sace, bhikkhave, upāsakaparisā ānandaṃ dassanāya upasaṅkamati, dassanena sā attamanā hoti. Tatra ce ānando dhammaṃ bhāsati, bhāsitenapi sā attamanā hoti. Atittāva, bhikkhave, upāsakaparisā hoti, atha kho ānando tuṇhī hoti. Sace, bhikkhave, upāsikāparisā ānandaṃ dassanāya upasaṅkamati, dassanena sā attamanā hoti. Tatra ce, ānando, dhammaṃ bhāsati, bhāsitenapi sā attamanā hoti. Atittāva, bhikkhave, upāsikāparisā hoti, atha kho ānando tuṇhī hoti. Ime kho, bhikkhave, cattāro acchariyā abbhutā dhammā ānande.

‘‘Cattārome, bhikkhave, acchariyā abbhutā dhammā raññe cakkavattimhi. Katame cattāro ? Sace, bhikkhave, khattiyaparisā rājānaṃ cakkavattiṃ dassanāya upasaṅkamati, dassanena sā attamanā hoti. Tatra ce rājā cakkavattī bhāsati, bhāsitenapi sā attamanā hoti. Atittāva, bhikkhave, khattiyaparisā hoti. Atha kho rājā cakkavattī tuṇhī hoti. Sace bhikkhave, brāhmaṇaparisā…pe… gahapatiparisā…pe… samaṇaparisā rājānaṃ cakkavattiṃ dassanāya upasaṅkamati, dassanena sā attamanā hoti. Tatra ce rājā cakkavattī bhāsati, bhāsitenapi sā attamanā hoti. Atittāva, bhikkhave, samaṇaparisā hoti, atha kho rājā cakkavattī tuṇhī hoti. Evameva kho, bhikkhave, cattārome acchariyā abbhutā dhammā ānande. Sace, bhikkhave, bhikkhuparisā ānandaṃ dassanāya upasaṅkamati, dassanena sā attamanā hoti. Tatra ce ānando dhammaṃ bhāsati, bhāsitenapi sā attamanā hoti. Atittāva, bhikkhave, bhikkhuparisā hoti. Atha kho ānando tuṇhī hoti. Sace, bhikkhave bhikkhunīparisā…pe… upāsakaparisā…pe… upāsikāparisā ānandaṃ dassanāya upasaṅkamati, dassanena sā attamanā hoti. Tatra ce ānando dhammaṃ bhāsati, bhāsitenapi sā attamanā hoti. Atittāva, bhikkhave, upāsikāparisā hoti. Atha kho ānando tuṇhī hoti. Ime kho, bhikkhave, cattāro acchariyā abbhutā dhammā ānande’’ti.

\subsubsection{Mahāsudassanasuttadesanā}

\paragraph{210.} Evaṃ vutte āyasmā ānando bhagavantaṃ etadavoca – ‘‘mā, bhante, bhagavā imasmiṃ khuddakanagarake ujjaṅgalanagarake sākhānagarake parinibbāyi. Santi, bhante, aññāni mahānagarāni, seyyathidaṃ – campā rājagahaṃ sāvatthī sāketaṃ kosambī bārāṇasī; ettha bhagavā parinibbāyatu. Ettha bahū khattiyamahāsālā, brāhmaṇamahāsālā gahapatimahāsālā tathāgate abhippasannā. Te tathāgatassa sarīrapūjaṃ karissantī’’ti ‘‘māhevaṃ, ānanda, avaca; māhevaṃ, ānanda, avaca – ‘khuddakanagarakaṃ ujjaṅgalanagarakaṃ sākhānagaraka’nti.

‘‘Bhūtapubbaṃ, ānanda, rājā mahāsudassano nāma ahosi cakkavattī dhammiko dhammarājā cāturanto vijitāvī janappadatthāvariyappatto sattaratanasamannāgato. Rañño, ānanda, mahāsudassanassa ayaṃ kusinārā kusāvatī nāma rājadhānī ahosi, puratthimena ca pacchimena ca dvādasayojanāni āyāmena; uttarena ca dakkhiṇena ca sattayojanāni vitthārena. Kusāvatī, ānanda, rājadhānī iddhā ceva ahosi phītā ca bahujanā ca ākiṇṇamanussā ca subhikkhā ca. Seyyathāpi, ānanda, devānaṃ āḷakamandā nāma rājadhānī iddhā ceva hoti phītā ca bahujanā ca ākiṇṇayakkhā ca subhikkhā ca; evameva kho, ānanda, kusāvatī rājadhānī iddhā ceva ahosi phītā ca bahujanā ca ākiṇṇamanussā ca subhikkhā ca. Kusāvatī, ānanda, rājadhānī dasahi saddehi avivittā ahosi divā ceva rattiñca, seyyathidaṃ – hatthisaddena assasaddena rathasaddena bherisaddena mudiṅgasaddena vīṇāsaddena gītasaddena saṅkhasaddena sammasaddena pāṇitāḷasaddena ‘asnātha pivatha khādathā’ti dasamena saddena.

‘‘Gaccha tvaṃ, ānanda, kusināraṃ pavisitvā kosinārakānaṃ mallānaṃ ārocehi – ‘ajja kho, vāseṭṭhā, rattiyā pacchime yāme tathāgatassa parinibbānaṃ bhavissati. Abhikkamatha vāseṭṭhā, abhikkamatha vāseṭṭhā. Mā pacchā vippaṭisārino ahuvattha – amhākañca no gāmakkhette tathāgatassa parinibbānaṃ ahosi, na mayaṃ labhimhā pacchime kāle tathāgataṃ dassanāyā’’’ti. ‘‘Evaṃ, bhante’’ti kho āyasmā ānando bhagavato paṭissutvā nivāsetvā pattacīvaramādāya attadutiyo kusināraṃ pāvisi.

\subsubsection{Mallānaṃ vandanā}

\paragraph{211.} Tena kho pana samayena kosinārakā mallā sandhāgāre\footnote{santhāgāre (sī. syā. pī.)} sannipatitā honti kenacideva karaṇīyena. Atha kho āyasmā ānando yena kosinārakānaṃ mallānaṃ sandhāgāraṃ tenupasaṅkami; upasaṅkamitvā kosinārakānaṃ mallānaṃ ārocesi – ‘‘ajja kho, vāseṭṭhā, rattiyā pacchime yāme tathāgatassa parinibbānaṃ bhavissati. Abhikkamatha vāseṭṭhā abhikkamatha vāseṭṭhā. Mā pacchā vippaṭisārino ahuvattha – ‘amhākañca no gāmakkhette tathāgatassa parinibbānaṃ ahosi, na mayaṃ labhimhā pacchime kāle tathāgataṃ dassanāyā’’’ti. Idamāyasmato ānandassa vacanaṃ sutvā mallā ca mallaputtā ca mallasuṇisā ca mallapajāpatiyo ca aghāvino dummanā cetodukkhasamappitā appekacce kese pakiriya kandanti, bāhā paggayha kandanti, chinnapātaṃ papatanti, āvaṭṭanti vivaṭṭanti – ‘atikhippaṃ bhagavā parinibbāyissati, atikhippaṃ sugato parinibbāyissati, atikhippaṃ cakkhuṃ loke antaradhāyissatī’ti. Atha kho mallā ca mallaputtā ca mallasuṇisā ca mallapajāpatiyo ca aghāvino dummanā cetodukkhasamappitā yena upavattanaṃ mallānaṃ sālavanaṃ yenāyasmā ānando tenupasaṅkamiṃsu. Atha kho āyasmato ānandassa etadahosi – ‘‘sace kho ahaṃ kosinārake malle ekamekaṃ bhagavantaṃ vandāpessāmi, avandito bhagavā kosinārakehi mallehi bhavissati, athāyaṃ ratti vibhāyissati. Yaṃnūnāhaṃ kosinārake malle kulaparivattaso kulaparivattaso ṭhapetvā bhagavantaṃ vandāpeyyaṃ – ‘itthannāmo, bhante, mallo saputto sabhariyo sapariso sāmacco bhagavato pāde sirasā vandatī’ti. Atha kho āyasmā ānando kosinārake malle kulaparivattaso kulaparivattaso ṭhapetvā bhagavantaṃ vandāpesi – ‘itthannāmo, bhante, mallo saputto sabhariyo sapariso sāmacco bhagavato pāde sirasā vandatī’’’ti. Atha kho āyasmā ānando etena upāyena paṭhameneva yāmena kosinārake malle bhagavantaṃ vandāpesi.

\subsubsection{Subhaddaparibbājakavatthu}

\paragraph{212.} Tena kho pana samayena subhaddo nāma paribbājako kusinārāyaṃ paṭivasati. Assosi kho subhaddo paribbājako – ‘‘ajja kira rattiyā pacchime yāme samaṇassa gotamassa parinibbānaṃ bhavissatī’’ti. Atha kho subhaddassa paribbājakassa etadahosi – ‘‘sutaṃ kho pana metaṃ paribbājakānaṃ vuḍḍhānaṃ mahallakānaṃ ācariyapācariyānaṃ bhāsamānānaṃ – ‘kadāci karahaci tathāgatā loke uppajjanti arahanto sammāsambuddhā’ti. Ajjeva rattiyā pacchime yāme samaṇassa gotamassa parinibbānaṃ bhavissati. Atthi ca me ayaṃ kaṅkhādhammo uppanno, evaṃ pasanno ahaṃ samaṇe gotame, ‘pahoti me samaṇo gotamo tathā dhammaṃ desetuṃ, yathāhaṃ imaṃ kaṅkhādhammaṃ pajaheyya’’’nti. Atha kho subhaddo paribbājako yena upavattanaṃ mallānaṃ sālavanaṃ, yenāyasmā ānando tenupasaṅkami; upasaṅkamitvā āyasmantaṃ ānandaṃ etadavoca – ‘‘sutaṃ metaṃ, bho ānanda, paribbājakānaṃ vuḍḍhānaṃ mahallakānaṃ ācariyapācariyānaṃ bhāsamānānaṃ – ‘kadāci karahaci tathāgatā loke uppajjanti arahanto sammāsambuddhā’ti. Ajjeva rattiyā pacchime yāme samaṇassa gotamassa parinibbānaṃ bhavissati. Atthi ca me ayaṃ kaṅkhādhammo uppanno – evaṃ pasanno ahaṃ samaṇe gotame ‘pahoti me samaṇo gotamo tathā dhammaṃ desetuṃ, yathāhaṃ imaṃ kaṅkhādhammaṃ pajaheyya’nti. Sādhāhaṃ, bho ānanda, labheyyaṃ samaṇaṃ gotamaṃ dassanāyā’’ti. Evaṃ vutte āyasmā ānando subhaddaṃ paribbājakaṃ etadavoca – ‘‘alaṃ, āvuso subhadda, mā tathāgataṃ viheṭhesi, kilanto bhagavā’’ti. Dutiyampi kho subhaddo paribbājako…pe… tatiyampi kho subhaddo paribbājako āyasmantaṃ ānandaṃ etadavoca – ‘‘sutaṃ metaṃ, bho ānanda, paribbājakānaṃ vuḍḍhānaṃ mahallakānaṃ ācariyapācariyānaṃ bhāsamānānaṃ – ‘kadāci karahaci tathāgatā loke uppajjanti arahanto sammāsambuddhā’ti. Ajjeva rattiyā pacchime yāme samaṇassa gotamassa parinibbānaṃ bhavissati. Atthi ca me ayaṃ kaṅkhādhammo uppanno – evaṃ pasanno ahaṃ samaṇe gotame, ‘pahoti me samaṇo gotamo tathā dhammaṃ desetuṃ, yathāhaṃ imaṃ kaṅkhādhammaṃ pajaheyya’nti. Sādhāhaṃ, bho ānanda, labheyyaṃ samaṇaṃ gotamaṃ dassanāyā’’ti. Tatiyampi kho āyasmā ānando subhaddaṃ paribbājakaṃ etadavoca – ‘‘alaṃ, āvuso subhadda, mā tathāgataṃ viheṭhesi, kilanto bhagavā’’ti.

\paragraph{213.} Assosi kho bhagavā āyasmato ānandassa subhaddena paribbājakena saddhiṃ imaṃ kathāsallāpaṃ. Atha kho bhagavā āyasmantaṃ ānandaṃ āmantesi – ‘‘alaṃ, ānanda, mā subhaddaṃ vāresi, labhataṃ, ānanda, subhaddo tathāgataṃ dassanāya. Yaṃ kiñci maṃ subhaddo pucchissati, sabbaṃ taṃ aññāpekkhova pucchissati, no vihesāpekkho. Yaṃ cassāhaṃ puṭṭho byākarissāmi, taṃ khippameva ājānissatī’’ti. Atha kho āyasmā ānando subhaddaṃ paribbājakaṃ etadavoca – ‘‘gacchāvuso subhadda, karoti te bhagavā okāsa’’nti. Atha kho subhaddo paribbājako yena bhagavā tenupasaṅkami; upasaṅkamitvā bhagavatā saddhiṃ sammodi, sammodanīyaṃ kathaṃ sāraṇīyaṃ vītisāretvā ekamantaṃ nisīdi. Ekamantaṃ nisinno kho subhaddo paribbājako bhagavantaṃ etadavoca – ‘‘yeme, bho gotama, samaṇabrāhmaṇā saṅghino gaṇino gaṇācariyā ñātā yasassino titthakarā sādhusammatā bahujanassa, seyyathidaṃ – pūraṇo kassapo, makkhali gosālo, ajito kesakambalo, pakudho kaccāyano, sañcayo belaṭṭhaputto, nigaṇṭho nāṭaputto, sabbete sakāya paṭiññāya abbhaññiṃsu, sabbeva na abbhaññiṃsu , udāhu ekacce abbhaññiṃsu, ekacce na abbhaññiṃsū’’ti? ‘‘Alaṃ, subhadda, tiṭṭhatetaṃ – ‘sabbete sakāya paṭiññāya abbhaññiṃsu, sabbeva na abbhaññiṃsu, udāhu ekacce abbhaññiṃsu, ekacce na abbhaññiṃsū’ti. Dhammaṃ te, subhadda, desessāmi; taṃ suṇāhi sādhukaṃ manasikarohi, bhāsissāmī’’ti. ‘‘Evaṃ, bhante’’ti kho subhaddo paribbājako bhagavato paccassosi. Bhagavā etadavoca –

\paragraph{214.} ‘‘Yasmiṃ kho, subhadda, dhammavinaye ariyo aṭṭhaṅgiko maggo na upalabbhati, samaṇopi tattha na upalabbhati. Dutiyopi tattha samaṇo na upalabbhati. Tatiyopi tattha samaṇo na upalabbhati. Catutthopi tattha samaṇo na upalabbhati. Yasmiñca kho, subhadda, dhammavinaye ariyo aṭṭhaṅgiko maggo upalabbhati, samaṇopi tattha upalabbhati, dutiyopi tattha samaṇo upalabbhati, tatiyopi tattha samaṇo upalabbhati, catutthopi tattha samaṇo upalabbhati. Imasmiṃ kho, subhadda, dhammavinaye ariyo aṭṭhaṅgiko maggo upalabbhati, idheva, subhadda, samaṇo, idha dutiyo samaṇo, idha tatiyo samaṇo, idha catuttho samaṇo, suññā parappavādā samaṇebhi aññehi\footnote{aññe (pī.)}. Ime ca\footnote{idheva (ka.)}, subhadda, bhikkhū sammā vihareyyuṃ, asuñño loko arahantehi assāti.

‘‘Ekūnatiṃso vayasā subhadda,

Yaṃ pabbajiṃ kiṃkusalānuesī;

Vassāni paññāsa samādhikāni,

Yato ahaṃ pabbajito subhadda.

Ñāyassa dhammassa padesavattī,

Ito bahiddhā samaṇopi natthi.

‘‘Dutiyopi samaṇo natthi. Tatiyopi samaṇo natthi. Catutthopi samaṇo natthi. Suññā parappavādā samaṇebhi aññehi. Ime ca, subhadda, bhikkhū sammā vihareyyuṃ, asuñño loko arahantehi assā’’ti.

\paragraph{215.} Evaṃ vutte subhaddo paribbājako bhagavantaṃ etadavoca – ‘‘abhikkantaṃ, bhante, abhikkantaṃ, bhante. Seyyathāpi, bhante, nikkujjitaṃ vā ukkujjeyya, paṭicchannaṃ vā vivareyya, mūḷhassa vā maggaṃ ācikkheyya, andhakāre vā telapajjotaṃ dhāreyya, ‘cakkhumanto rūpāni dakkhantī’ti, evamevaṃ bhagavatā anekapariyāyena dhammo pakāsito. Esāhaṃ, bhante, bhagavantaṃ saraṇaṃ gacchāmi dhammañca bhikkhusaṅghañca. Labheyyāhaṃ, bhante, bhagavato santike pabbajjaṃ, labheyyaṃ upasampada’’nti. ‘‘Yo kho, subhadda, aññatitthiyapubbo imasmiṃ dhammavinaye ākaṅkhati pabbajjaṃ, ākaṅkhati upasampadaṃ, so cattāro māse parivasati. Catunnaṃ māsānaṃ accayena āraddhacittā bhikkhū pabbājenti upasampādenti bhikkhubhāvāya. Api ca mettha puggalavemattatā viditā’’ti. ‘‘Sace, bhante, aññatitthiyapubbā imasmiṃ dhammavinaye ākaṅkhantā pabbajjaṃ ākaṅkhantā upasampadaṃ cattāro māse parivasanti, catunnaṃ māsānaṃ accayena āraddhacittā bhikkhū pabbājenti upasampādenti bhikkhubhāvāya. Ahaṃ cattāri vassāni parivasissāmi, catunnaṃ vassānaṃ accayena āraddhacittā bhikkhū pabbājentu upasampādentu bhikkhubhāvāyā’’ti.

Atha kho bhagavā āyasmantaṃ ānandaṃ āmantesi – ‘‘tenahānanda, subhaddaṃ pabbājehī’’ti. ‘‘Evaṃ, bhante’’ti kho āyasmā ānando bhagavato paccassosi. Atha kho subhaddo paribbājako āyasmantaṃ ānandaṃ etadavoca – ‘‘lābhā vo, āvuso ānanda; suladdhaṃ vo, āvuso ānanda, ye ettha satthu\footnote{satthārā (syā.)} sammukhā antevāsikābhisekena abhisittā’’ti. Alattha kho subhaddo paribbājako bhagavato santike pabbajjaṃ, alattha upasampadaṃ. Acirūpasampanno kho panāyasmā subhaddo eko vūpakaṭṭho appamatto ātāpī pahitatto viharanto nacirasseva – ‘yassatthāya kulaputtā sammadeva agārasmā anagāriyaṃ pabbajanti’ tadanuttaraṃ brahmacariyapariyosānaṃ diṭṭheva dhamme sayaṃ abhiññā sacchikatvā upasampajja vihāsi. ‘Khīṇā jāti, vusitaṃ brahmacariyaṃ, kataṃ karaṇīyaṃ, nāparaṃ itthattāyā’ti abbhaññāsi. Aññataro kho panāyasmā subhaddo arahataṃ ahosi. So bhagavato pacchimo sakkhisāvako ahosīti.

\xsubsubsectionEnd{Pañcamo bhāṇavāro.}

\subsubsection{Tathāgatapacchimavācā}

\paragraph{216.} Atha kho bhagavā āyasmantaṃ ānandaṃ āmantesi – ‘‘siyā kho panānanda, tumhākaṃ evamassa – ‘atītasatthukaṃ pāvacanaṃ, natthi no satthā’ti. Na kho panetaṃ, ānanda, evaṃ daṭṭhabbaṃ. Yo vo, ānanda, mayā dhammo ca vinayo ca desito paññatto, so vo mamaccayena satthā. Yathā kho panānanda, etarahi bhikkhū aññamaññaṃ āvusovādena samudācaranti, na kho mamaccayena evaṃ samudācaritabbaṃ. Theratarena, ānanda, bhikkhunā navakataro bhikkhu nāmena vā gottena vā āvusovādena vā samudācaritabbo. Navakatarena bhikkhunā therataro bhikkhu ‘bhante’ti vā ‘āyasmā’ti vā samudācaritabbo. Ākaṅkhamāno, ānanda, saṅgho mamaccayena khuddānukhuddakāni sikkhāpadāni samūhanatu. Channassa, ānanda, bhikkhuno mamaccayena brahmadaṇḍo dātabbo’’ti. ‘‘Katamo pana, bhante, brahmadaṇḍo’’ti? ‘‘Channo, ānanda, bhikkhu yaṃ iccheyya, taṃ vadeyya. So bhikkhūhi neva vattabbo, na ovaditabbo, na anusāsitabbo’’ti.

\paragraph{217.} Atha kho bhagavā bhikkhū āmantesi – ‘‘siyā kho pana, bhikkhave, ekabhikkhussāpi kaṅkhā vā vimati vā buddhe vā dhamme vā saṅghe vā magge vā paṭipadāya vā, pucchatha, bhikkhave, mā pacchā vippaṭisārino ahuvattha – ‘sammukhībhūto no satthā ahosi , na mayaṃ sakkhimhā bhagavantaṃ sammukhā paṭipucchitu’’’ nti. Evaṃ vutte te bhikkhū tuṇhī ahesuṃ. Dutiyampi kho bhagavā…pe… tatiyampi kho bhagavā bhikkhū āmantesi – ‘‘siyā kho pana, bhikkhave, ekabhikkhussāpi kaṅkhā vā vimati vā buddhe vā dhamme vā saṅghe vā magge vā paṭipadāya vā, pucchatha, bhikkhave, mā pacchā vippaṭisārino ahuvattha – ‘sammukhībhūto no satthā ahosi , na mayaṃ sakkhimhā bhagavantaṃ sammukhā paṭipucchitu’’’ nti. Tatiyampi kho te bhikkhū tuṇhī ahesuṃ. Atha kho bhagavā bhikkhū āmantesi – ‘‘siyā kho pana, bhikkhave, satthugāravenapi na puccheyyātha. Sahāyakopi, bhikkhave, sahāyakassa ārocetū’’ti. Evaṃ vutte te bhikkhū tuṇhī ahesuṃ. Atha kho āyasmā ānando bhagavantaṃ etadavoca – ‘‘acchariyaṃ, bhante, abbhutaṃ, bhante, evaṃ pasanno ahaṃ, bhante, imasmiṃ bhikkhusaṅghe, ‘natthi ekabhikkhussāpi kaṅkhā vā vimati vā buddhe vā dhamme vā saṅghe vā magge vā paṭipadāya vā’’’ti. ‘‘Pasādā kho tvaṃ, ānanda, vadesi, ñāṇameva hettha, ānanda, tathāgatassa. Natthi imasmiṃ bhikkhusaṅghe ekabhikkhussāpi kaṅkhā vā vimati vā buddhe vā dhamme vā saṅghe vā magge vā paṭipadāya vā. Imesañhi, ānanda, pañcannaṃ bhikkhusatānaṃ yo pacchimako bhikkhu, so sotāpanno avinipātadhammo niyato sambodhiparāyaṇo’’ti.

\paragraph{218.} Atha kho bhagavā bhikkhū āmantesi – ‘‘handa dāni, bhikkhave, āmantayāmi vo, vayadhammā saṅkhārā appamādena sampādethā’’ti. Ayaṃ tathāgatassa pacchimā vācā.

\subsubsection{Parinibbutakathā}

\paragraph{219.} Atha kho bhagavā paṭhamaṃ jhānaṃ samāpajji, paṭhamajjhānā vuṭṭhahitvā dutiyaṃ jhānaṃ samāpajji, dutiyajjhānā vuṭṭhahitvā tatiyaṃ jhānaṃ samāpajji, tatiyajjhānā vuṭṭhahitvā catutthaṃ jhānaṃ samāpajji. Catutthajjhānā vuṭṭhahitvā ākāsānañcāyatanaṃ samāpajji, ākāsānañcāyatanasamāpattiyā vuṭṭhahitvā viññāṇañcāyatanaṃ samāpajji, viññāṇañcāyatanasamāpattiyā vuṭṭhahitvā ākiñcaññāyatanaṃ samāpajji, ākiñcaññāyatanasamāpattiyā vuṭṭhahitvā nevasaññānāsaññāyatanaṃ samāpajji, nevasaññānāsaññāyatanasamāpattiyā vuṭṭhahitvā saññāvedayitanirodhaṃ samāpajji.

Atha kho āyasmā ānando āyasmantaṃ anuruddhaṃ etadavoca – ‘‘parinibbuto, bhante anuruddha , bhagavā’’ti. ‘‘Nāvuso ānanda, bhagavā parinibbuto, saññāvedayitanirodhaṃ samāpanno’’ti.

Atha kho bhagavā saññāvedayitanirodhasamāpattiyā vuṭṭhahitvā nevasaññānāsaññāyatanaṃ samāpajji, nevasaññānāsaññāyatanasamāpattiyā vuṭṭhahitvā ākiñcaññāyatanaṃ samāpajji, ākiñcaññāyatanasamāpattiyā vuṭṭhahitvā viññāṇañcāyatanaṃ samāpajji, viññāṇañcāyatanasamāpattiyā vuṭṭhahitvā ākāsānañcāyatanaṃ samāpajji, ākāsānañcāyatanasamāpattiyā vuṭṭhahitvā catutthaṃ jhānaṃ samāpajji, catutthajjhānā vuṭṭhahitvā tatiyaṃ jhānaṃ samāpajji, tatiyajjhānā vuṭṭhahitvā dutiyaṃ jhānaṃ samāpajji, dutiyajjhānā vuṭṭhahitvā paṭhamaṃ jhānaṃ samāpajji, paṭhamajjhānā vuṭṭhahitvā dutiyaṃ jhānaṃ samāpajji, dutiyajjhānā vuṭṭhahitvā tatiyaṃ jhānaṃ samāpajji, tatiyajjhānā vuṭṭhahitvā catutthaṃ jhānaṃ samāpajji, catutthajjhānā vuṭṭhahitvā samanantarā bhagavā parinibbāyi.

\paragraph{220.} Parinibbute bhagavati saha parinibbānā mahābhūmicālo ahosi bhiṃsanako salomahaṃso. Devadundubhiyo ca phaliṃsu. Parinibbute bhagavati saha parinibbānā brahmāsahampati imaṃ gāthaṃ abhāsi –

‘‘Sabbeva nikkhipissanti, bhūtā loke samussayaṃ;

Yattha etādiso satthā, loke appaṭipuggalo;

Tathāgato balappatto, sambuddho parinibbuto’’ti.

\paragraph{221.} Parinibbute bhagavati saha parinibbānā sakko devānamindo imaṃ gāthaṃ abhāsi –

‘‘Aniccā vata saṅkhārā, uppādavayadhammino;

Uppajjitvā nirujjhanti, tesaṃ vūpasamo sukho’’ti.

\paragraph{222.} Parinibbute bhagavati saha parinibbānā āyasmā anuruddho imā gāthāyo abhāsi –

‘‘Nāhu assāsapassāso, ṭhitacittassa tādino;

Anejo santimārabbha, yaṃ kālamakarī muni.

‘‘Asallīnena cittena, vedanaṃ ajjhavāsayi;

Pajjotasseva nibbānaṃ, vimokkho cetaso ahū’’ti.

\paragraph{223.} Parinibbute bhagavati saha parinibbānā āyasmā ānando imaṃ gāthaṃ abhāsi –

‘‘Tadāsi yaṃ bhiṃsanakaṃ, tadāsi lomahaṃsanaṃ;

Sabbākāravarūpete, sambuddhe parinibbute’’ti.

\paragraph{224.} Parinibbute bhagavati ye te tattha bhikkhū avītarāgā appekacce bāhā paggayha kandanti, chinnapātaṃ papatanti, āvaṭṭanti vivaṭṭanti, ‘‘atikhippaṃ bhagavā parinibbuto , atikhippaṃ sugato parinibbuto, atikhippaṃ cakkhuṃ loke antarahito’’ti. Ye pana te bhikkhū vītarāgā, te satā sampajānā adhivāsenti – ‘‘aniccā saṅkhārā, taṃ kutettha labbhā’’ti.

\paragraph{225.} Atha kho āyasmā anuruddho bhikkhū āmantesi – ‘‘alaṃ, āvuso, mā socittha mā paridevittha. Nanu etaṃ, āvuso, bhagavatā paṭikacceva akkhātaṃ – ‘sabbeheva piyehi manāpehi nānābhāvo vinābhāvo aññathābhāvo’. Taṃ kutettha, āvuso, labbhā. ‘Yaṃ taṃ jātaṃ bhūtaṃ saṅkhataṃ palokadhammaṃ, taṃ vata mā palujjī’ti, netaṃ ṭhānaṃ vijjati . Devatā, āvuso, ujjhāyantī’’ti. ‘‘Kathaṃbhūtā pana, bhante, āyasmā anuruddho devatā manasi karotī’’ti\footnote{bhante anuruddha devatā manasi karontīti (syā. ka.)}?

‘‘Santāvuso ānanda, devatā ākāse pathavīsaññiniyo kese pakiriya kandanti, bāhā paggayha kandanti, chinnapātaṃ papatanti, āvaṭṭanti, vivaṭṭanti – ‘atikhippaṃ bhagavā parinibbuto, atikhippaṃ sugato parinibbuto, atikhippaṃ cakkhuṃ loke antarahito’ti. Santāvuso ānanda, devatā pathaviyā pathavīsaññiniyo kese pakiriya kandanti, bāhā paggayha kandanti, chinnapātaṃ papatanti, āvaṭṭanti, vivaṭṭanti – ‘atikhippaṃ bhagavā parinibbuto , atikhippaṃ sugato parinibbuto, atikhippaṃ cakkhuṃ loke antarahito’ti. Yā pana tā devatā vītarāgā, tā satā sampajānā adhivāsenti – ‘aniccā saṅkhārā, taṃ kutettha labbhā’ti. Atha kho āyasmā ca anuruddho āyasmā ca ānando taṃ rattāvasesaṃ dhammiyā kathāya vītināmesuṃ.

\paragraph{226.} Atha kho āyasmā anuruddho āyasmantaṃ ānandaṃ āmantesi – ‘‘gacchāvuso ānanda, kusināraṃ pavisitvā kosinārakānaṃ mallānaṃ ārocehi – ‘parinibbuto, vāseṭṭhā, bhagavā, yassadāni kālaṃ maññathā’’’ti. ‘‘Evaṃ, bhante’’ti kho āyasmā ānando āyasmato anuruddhassa paṭissutvā pubbaṇhasamayaṃ nivāsetvā pattacīvaramādāya attadutiyo kusināraṃ pāvisi. Tena kho pana samayena kosinārakā mallā sandhāgāre sannipatitā honti teneva karaṇīyena. Atha kho āyasmā ānando yena kosinārakānaṃ mallānaṃ sandhāgāraṃ tenupasaṅkami; upasaṅkamitvā kosinārakānaṃ mallānaṃ ārocesi – ‘parinibbuto, vāseṭṭhā, bhagavā, yassadāni kālaṃ maññathā’ti. Idamāyasmato ānandassa vacanaṃ sutvā mallā ca mallaputtā ca mallasuṇisā ca mallapajāpatiyo ca aghāvino dummanā cetodukkhasamappitā appekacce kese pakiriya kandanti, bāhā paggayha kandanti, chinnapātaṃ papatanti, āvaṭṭanti, vivaṭṭanti – ‘‘atikhippaṃ bhagavā parinibbuto, atikhippaṃ sugato parinibbuto, atikhippaṃ cakkhuṃ loke antarahito’’ti.

\subsubsection{Buddhasarīrapūjā}

\paragraph{227.} Atha kho kosinārakā mallā purise āṇāpesuṃ – ‘‘tena hi, bhaṇe, kusinārāyaṃ gandhamālañca sabbañca tāḷāvacaraṃ sannipātethā’’ti. Atha kho kosinārakā mallā gandhamālañca sabbañca tāḷāvacaraṃ pañca ca dussayugasatāni ādāya yena upavattanaṃ mallānaṃ sālavanaṃ, yena bhagavato sarīraṃ tenupasaṅkamiṃsu; upasaṅkamitvā bhagavato sarīraṃ naccehi gītehi vāditehi mālehi gandhehi sakkarontā garuṃ karontā mānentā pūjentā celavitānāni karontā maṇḍalamāḷe paṭiyādentā ekadivasaṃ vītināmesuṃ.

Atha kho kosinārakānaṃ mallānaṃ etadahosi – ‘‘ativikālo kho ajja bhagavato sarīraṃ jhāpetuṃ, sve dāni mayaṃ bhagavato sarīraṃ jhāpessāmā’’ti. Atha kho kosinārakā mallā bhagavato sarīraṃ naccehi gītehi vāditehi mālehi gandhehi sakkarontā garuṃ karontā mānentā pūjentā celavitānāni karontā maṇḍalamāḷe paṭiyādentā dutiyampi divasaṃ vītināmesuṃ, tatiyampi divasaṃ vītināmesuṃ, catutthampi divasaṃ vītināmesuṃ, pañcamampi divasaṃ vītināmesuṃ, chaṭṭhampi divasaṃ vītināmesuṃ.

Atha kho sattamaṃ divasaṃ kosinārakānaṃ mallānaṃ etadahosi – ‘‘mayaṃ bhagavato sarīraṃ naccehi gītehi vāditehi mālehi gandhehi sakkarontā garuṃ karontā mānentā pūjentā dakkhiṇena dakkhiṇaṃ nagarassa haritvā bāhirena bāhiraṃ dakkhiṇato nagarassa bhagavato sarīraṃ jhāpessāmā’’ti.

\paragraph{228.} Tena kho pana samayena aṭṭha mallapāmokkhā sīsaṃnhātā ahatāni vatthāni nivatthā ‘‘mayaṃ bhagavato sarīraṃ uccāressāmā’’ti na sakkonti uccāretuṃ. Atha kho kosinārakā mallā āyasmantaṃ anuruddhaṃ etadavocuṃ – ‘‘ko nu kho, bhante anuruddha, hetu ko paccayo, yenime aṭṭha mallapāmokkhā sīsaṃnhātā ahatāni vatthāni nivatthā ‘mayaṃ bhagavato sarīraṃ uccāressāmā’ti na sakkonti uccāretu’’nti? ‘‘Aññathā kho, vāseṭṭhā, tumhākaṃ adhippāyo, aññathā devatānaṃ adhippāyo’’ti. ‘‘Kathaṃ pana, bhante, devatānaṃ adhippāyo’’ti? ‘‘Tumhākaṃ kho, vāseṭṭhā, adhippāyo – ‘mayaṃ bhagavato sarīraṃ naccehi gītehi vāditehi mālehi gandhehi sakkarontā garuṃ karontā mānentā pūjentā dakkhiṇena dakkhiṇaṃ nagarassa haritvā bāhirena bāhiraṃ dakkhiṇato nagarassa bhagavato sarīraṃ jhāpessāmā’ti; devatānaṃ kho, vāseṭṭhā, adhippāyo – ‘mayaṃ bhagavato sarīraṃ dibbehi naccehi gītehi vāditehi gandhehi sakkarontā garuṃ karontā mānentā pūjentā uttarena uttaraṃ nagarassa haritvā uttarena dvārena nagaraṃ pavesetvā majjhena majjhaṃ nagarassa haritvā puratthimena dvārena nikkhamitvā puratthimato nagarassa makuṭabandhanaṃ nāma mallānaṃ cetiyaṃ ettha bhagavato sarīraṃ jhāpessāmā’ti. ‘‘Yathā, bhante, devatānaṃ adhippāyo, tathā hotū’’ti.

\paragraph{229.} Tena kho pana samayena kusinārā yāva sandhisamalasaṃkaṭīrā jaṇṇumattena odhinā mandāravapupphehi santhatā\footnote{saṇṭhitā (syā.)} hoti. Atha kho devatā ca kosinārakā ca mallā bhagavato sarīraṃ dibbehi ca mānusakehi ca naccehi gītehi vāditehi mālehi gandhehi sakkarontā garuṃ karontā mānentā pūjentā uttarena uttaraṃ nagarassa haritvā uttarena dvārena nagaraṃ pavesetvā majjhena majjhaṃ nagarassa haritvā puratthimena dvārena nikkhamitvā puratthimato nagarassa makuṭabandhanaṃ nāma mallānaṃ cetiyaṃ ettha ca bhagavato sarīraṃ nikkhipiṃsu.

\paragraph{230.} Atha kho kosinārakā mallā āyasmantaṃ ānandaṃ etadavocuṃ – ‘‘kathaṃ mayaṃ, bhante ānanda, tathāgatassa sarīre paṭipajjāmā’’ti? ‘‘Yathā kho, vāseṭṭhā, rañño cakkavattissa sarīre paṭipajjanti, evaṃ tathāgatassa sarīre paṭipajjitabba’’nti. ‘‘Kathaṃ pana, bhante ānanda, rañño cakkavattissa sarīre paṭipajjantī’’ti? ‘‘Rañño, vāseṭṭhā, cakkavattissa sarīraṃ ahatena vatthena veṭhenti, ahatena vatthena veṭhetvā vihatena kappāsena veṭhenti, vihatena kappāsena veṭhetvā ahatena vatthena veṭhenti. Etena upāyena pañcahi yugasatehi rañño cakkavattissa sarīraṃ veṭhetvā āyasāya teladoṇiyā pakkhipitvā aññissā āyasāya doṇiyā paṭikujjitvā sabbagandhānaṃ citakaṃ karitvā rañño cakkavattissa sarīraṃ jhāpenti. Cātumahāpathe rañño cakkavattissa thūpaṃ karonti . Evaṃ kho, vāseṭṭhā, rañño cakkavattissa sarīre paṭipajjanti. Yathā kho, vāseṭṭhā, rañño cakkavattissa sarīre paṭipajjanti, evaṃ tathāgatassa sarīre paṭipajjitabbaṃ. Cātumahāpathe tathāgatassa thūpo kātabbo. Tattha ye mālaṃ vā gandhaṃ vā cuṇṇakaṃ vā āropessanti vā abhivādessanti vā cittaṃ vā pasādessanti, tesaṃ taṃ bhavissati dīgharattaṃ hitāya sukhāyā’’ti. Atha kho kosinārakā mallā purise āṇāpesuṃ – ‘‘tena hi, bhaṇe, mallānaṃ vihataṃ kappāsaṃ sannipātethā’’ti.

Atha kho kosinārakā mallā bhagavato sarīraṃ ahatena vatthena veṭhetvā vihatena kappāsena veṭhesuṃ, vihatena kappāsena veṭhetvā ahatena vatthena veṭhesuṃ. Etena upāyena pañcahi yugasatehi bhagavato sarīraṃ veṭhetvā āyasāya teladoṇiyā pakkhipitvā aññissā āyasāya doṇiyā paṭikujjitvā sabbagandhānaṃ citakaṃ karitvā bhagavato sarīraṃ citakaṃ āropesuṃ.

\subsubsection{Mahākassapattheravatthu}

\paragraph{231.} Tena kho pana samayena āyasmā mahākassapo pāvāya kusināraṃ addhānamaggappaṭippanno hoti mahatā bhikkhusaṅghena saddhiṃ pañcamattehi bhikkhusatehi. Atha kho āyasmā mahākassapo maggā okkamma aññatarasmiṃ rukkhamūle nisīdi. Tena kho pana samayena aññataro ājīvako kusinārāya mandāravapupphaṃ gahetvā pāvaṃ addhānamaggappaṭippanno hoti. Addasā kho āyasmā mahākassapo taṃ ājīvakaṃ dūratova āgacchantaṃ, disvā taṃ ājīvakaṃ etadavoca – ‘‘apāvuso, amhākaṃ satthāraṃ jānāsī’’ti? ‘‘Āmāvuso, jānāmi, ajja sattāhaparinibbuto samaṇo gotamo. Tato me idaṃ mandāravapupphaṃ gahita’’nti. Tattha ye te bhikkhū avītarāgā appekacce bāhā paggayha kandanti, chinnapātaṃ papatanti, āvaṭṭanti, vivaṭṭanti – ‘‘atikhippaṃ bhagavā parinibbuto, atikhippaṃ sugato parinibbuto, atikhippaṃ cakkhuṃ loke antarahito’’ti. Ye pana te bhikkhū vītarāgā, te satā sampajānā adhivāsenti – ‘‘aniccā saṅkhārā, taṃ kutettha labbhā’’ti.

\paragraph{232.} Tena kho pana samayena subhaddo nāma vuddhapabbajito tassaṃ parisāyaṃ nisinno hoti. Atha kho subhaddo vuddhapabbajito te bhikkhū etadavoca – ‘‘alaṃ, āvuso, mā socittha, mā paridevittha, sumuttā mayaṃ tena mahāsamaṇena. Upaddutā ca homa – ‘idaṃ vo kappati, idaṃ vo na kappatī’ti. Idāni pana mayaṃ yaṃ icchissāma, taṃ karissāma, yaṃ na icchissāma, na taṃ karissāmā’’ti. Atha kho āyasmā mahākassapo bhikkhū āmantesi – ‘‘alaṃ, āvuso, mā socittha, mā paridevittha. Nanu etaṃ , āvuso, bhagavatā paṭikacceva akkhātaṃ – ‘sabbeheva piyehi manāpehi nānābhāvo vinābhāvo aññathābhāvo’. Taṃ kutettha, āvuso, labbhā. ‘Yaṃ taṃ jātaṃ bhūtaṃ saṅkhataṃ palokadhammaṃ, taṃ tathāgatassāpi sarīraṃ mā palujjī’ti, netaṃ ṭhānaṃ vijjatī’’ti.

\paragraph{233.} Tena kho pana samayena cattāro mallapāmokkhā sīsaṃnhātā ahatāni vatthāni nivatthā – ‘‘mayaṃ bhagavato citakaṃ āḷimpessāmā’’ti na sakkonti āḷimpetuṃ. Atha kho kosinārakā mallā āyasmantaṃ anuruddhaṃ etadavocuṃ – ‘‘ko nu kho, bhante anuruddha, hetu ko paccayo, yenime cattāro mallapāmokkhā sīsaṃnhātā ahatāni vatthāni nivatthā – ‘mayaṃ bhagavato citakaṃ āḷimpessāmā’ti na sakkonti āḷimpetu’’nti? ‘‘Aññathā kho, vāseṭṭhā, devatānaṃ adhippāyo’’ti. ‘‘Kathaṃ pana, bhante, devatānaṃ adhippāyo’’ti? ‘‘Devatānaṃ kho, vāseṭṭhā, adhippāyo – ‘ayaṃ āyasmā mahākassapo pāvāya kusināraṃ addhānamaggappaṭippanno mahatā bhikkhusaṅghena saddhiṃ pañcamattehi bhikkhusatehi. Na tāva bhagavato citako pajjalissati, yāvāyasmā mahākassapo bhagavato pāde sirasā na vandissatī’’’ti. ‘‘Yathā, bhante, devatānaṃ adhippāyo, tathā hotū’’ti.

\paragraph{234.} Atha kho āyasmā mahākassapo yena kusinārā makuṭabandhanaṃ nāma mallānaṃ cetiyaṃ, yena bhagavato citako tenupasaṅkami; upasaṅkamitvā ekaṃsaṃ cīvaraṃ katvā añjaliṃ paṇāmetvā tikkhattuṃ citakaṃ padakkhiṇaṃ katvā bhagavato pāde sirasā vandi. Tānipi kho pañcabhikkhusatāni ekaṃsaṃ cīvaraṃ katvā añjaliṃ paṇāmetvā tikkhattuṃ citakaṃ padakkhiṇaṃ katvā bhagavato pāde sirasā vandiṃsu. Vandite ca panāyasmatā mahākassapena tehi ca pañcahi bhikkhusatehi sayameva bhagavato citako pajjali.

\paragraph{235.} Jhāyamānassa kho pana bhagavato sarīrassa yaṃ ahosi chavīti vā cammanti vā maṃsanti vā nhārūti vā lasikāti vā, tassa neva chārikā paññāyittha, na masi; sarīrāneva avasissiṃsu. Seyyathāpi nāma sappissa vā telassa vā jhāyamānassa neva chārikā paññāyati, na masi; evameva bhagavato sarīrassa jhāyamānassa yaṃ ahosi chavīti vā cammanti vā maṃsanti vā nhārūti vā lasikāti vā, tassa neva chārikā paññāyittha, na masi; sarīrāneva avasissiṃsu. Tesañca pañcannaṃ dussayugasatānaṃ dveva dussāni na ḍayhiṃsu yañca sabbaabbhantarimaṃ yañca bāhiraṃ. Daḍḍhe ca kho pana bhagavato sarīre antalikkhā udakadhārā pātubhavitvā bhagavato citakaṃ nibbāpesi. Udakasālatopi\footnote{udakaṃ sālatopi (sī. syā. kaṃ.)} abbhunnamitvā bhagavato citakaṃ nibbāpesi. Kosinārakāpi mallā sabbagandhodakena bhagavato citakaṃ nibbāpesuṃ. Atha kho kosinārakā mallā bhagavato sarīrāni sattāhaṃ sandhāgāre sattipañjaraṃ karitvā dhanupākāraṃ parikkhipāpetvā\footnote{parikkhipitvā (syā.)} naccehi gītehi vāditehi mālehi gandhehi sakkariṃsu garuṃ kariṃsu mānesuṃ pūjesuṃ.

\subsubsection{Sarīradhātuvibhājanaṃ}

\paragraph{236.} Assosi kho rājā māgadho ajātasattu vedehiputto – ‘‘bhagavā kira kusinārāyaṃ parinibbuto’’ti. Atha kho rājā māgadho ajātasattu vedehiputto kosinārakānaṃ mallānaṃ dūtaṃ pāhesi – ‘‘bhagavāpi khattiyo ahampi khattiyo, ahampi arahāmi bhagavato sarīrānaṃ bhāgaṃ, ahampi bhagavato sarīrānaṃ thūpañca mahañca karissāmī’’ti.

Assosuṃ kho vesālikā licchavī – ‘‘bhagavā kira kusinārāyaṃ parinibbuto’’ti. Atha kho vesālikā licchavī kosinārakānaṃ mallānaṃ dūtaṃ pāhesuṃ – ‘‘bhagavāpi khattiyo mayampi khattiyā, mayampi arahāma bhagavato sarīrānaṃ bhāgaṃ, mayampi bhagavato sarīrānaṃ thūpañca mahañca karissāmā’’ti.

Assosuṃ kho kapilavatthuvāsī sakyā – ‘‘bhagavā kira kusinārāyaṃ parinibbuto’’ti. Atha kho kapilavatthuvāsī sakyā kosinārakānaṃ mallānaṃ dūtaṃ pāhesuṃ – ‘‘bhagavā amhākaṃ ñātiseṭṭho , mayampi arahāma bhagavato sarīrānaṃ bhāgaṃ, mayampi bhagavato sarīrānaṃ thūpañca mahañca karissāmā’’ti.

Assosuṃ kho allakappakā bulayo\footnote{thūlayo (syā.)} – ‘‘bhagavā kira kusinārāyaṃ parinibbuto’’ti. Atha kho allakappakā bulayo kosinārakānaṃ mallānaṃ dūtaṃ pāhesuṃ – ‘‘bhagavāpi khattiyo mayampi khattiyā, mayampi arahāma bhagavato sarīrānaṃ bhāgaṃ, mayampi bhagavato sarīrānaṃ thūpañca mahañca karissāmā’’ti .

Assosuṃ kho rāmagāmakā koḷiyā – ‘‘bhagavā kira kusinārāyaṃ parinibbuto’’ti. Atha kho rāmagāmakā koḷiyā kosinārakānaṃ mallānaṃ dūtaṃ pāhesuṃ – ‘‘bhagavāpi khattiyo mayampi khattiyā, mayampi arahāma bhagavato sarīrānaṃ bhāgaṃ, mayampi bhagavato sarīrānaṃ thūpañca mahañca karissāmā’’ti.

Assosi kho veṭṭhadīpako brāhmaṇo – ‘‘bhagavā kira kusinārāyaṃ parinibbuto’’ti. Atha kho veṭṭhadīpako brāhmaṇo kosinārakānaṃ mallānaṃ dūtaṃ pāhesi – ‘‘bhagavāpi khattiyo ahaṃ pismi brāhmaṇo, ahampi arahāmi bhagavato sarīrānaṃ bhāgaṃ, ahampi bhagavato sarīrānaṃ thūpañca mahañca karissāmī’’ti.

Assosuṃ kho pāveyyakā mallā – ‘‘bhagavā kira kusinārāyaṃ parinibbuto’’ti. Atha kho pāveyyakā mallā kosinārakānaṃ mallānaṃ dūtaṃ pāhesuṃ – ‘‘bhagavāpi khattiyo mayampi khattiyā, mayampi arahāma bhagavato sarīrānaṃ bhāgaṃ, mayampi bhagavato sarīrānaṃ thūpañca mahañca karissāmā’’ti.

Evaṃ vutte kosinārakā mallā te saṅghe gaṇe etadavocuṃ – ‘‘bhagavā amhākaṃ gāmakkhette parinibbuto, na mayaṃ dassāma bhagavato sarīrānaṃ bhāga’’nti.

\paragraph{237.} Evaṃ vutte doṇo brāhmaṇo te saṅghe gaṇe etadavoca –

‘‘Suṇantu bhonto mama ekavācaṃ,

Amhāka\footnote{chandānurakkhaṇatthaṃ niggahītalopo}; Buddho ahu khantivādo;

Na hi sādhu yaṃ uttamapuggalassa,

Sarīrabhāge siyā sampahāro.

Sabbeva bhonto sahitā samaggā,

Sammodamānā karomaṭṭhabhāge;

Vitthārikā hontu disāsu thūpā,

Bahū janā cakkhumato pasannā’’ti.

\paragraph{238.} ‘‘Tena hi, brāhmaṇa, tvaññeva bhagavato sarīrāni aṭṭhadhā samaṃ savibhattaṃ vibhajāhī’’ti. ‘‘Evaṃ, bho’’ti kho doṇo brāhmaṇo tesaṃ saṅghānaṃ gaṇānaṃ paṭissutvā bhagavato sarīrāni aṭṭhadhā samaṃ suvibhattaṃ vibhajitvā te saṅghe gaṇe etadavoca – ‘‘imaṃ me bhonto tumbaṃ dadantu ahampi tumbassa thūpañca mahañca karissāmī’’ti. Adaṃsu kho te doṇassa brāhmaṇassa tumbaṃ.

Assosuṃ kho pippalivaniyā\footnote{pipphalivaniyā (syā.)} moriyā – ‘‘bhagavā kira kusinārāyaṃ parinibbuto’’ti. Atha kho pippalivaniyā moriyā kosinārakānaṃ mallānaṃ dūtaṃ pāhesuṃ – ‘‘bhagavāpi khattiyo mayampi khattiyā, mayampi arahāma bhagavato sarīrānaṃ bhāgaṃ, mayampi bhagavato sarīrānaṃ thūpañca mahañca karissāmā’’ti. ‘‘Natthi bhagavato sarīrānaṃ bhāgo, vibhattāni bhagavato sarīrāni. Ito aṅgāraṃ harathā’’ti. Te tato aṅgāraṃ hariṃsu\footnote{āhariṃsu (syā. ka.)}.

\subsubsection{Dhātuthūpapūjā}

\paragraph{239.} Atha kho rājā māgadho ajātasattu vedehiputto rājagahe bhagavato sarīrānaṃ thūpañca mahañca akāsi. Vesālikāpi licchavī vesāliyaṃ bhagavato sarīrānaṃ thūpañca mahañca akaṃsu. Kapilavatthuvāsīpi sakyā kapilavatthusmiṃ bhagavato sarīrānaṃ thūpañca mahañca akaṃsu. Allakappakāpi bulayo allakappe bhagavato sarīrānaṃ thūpañca mahañca akaṃsu. Rāmagāmakāpi koḷiyā rāmagāme bhagavato sarīrānaṃ thūpañca mahañca akaṃsu. Veṭṭhadīpakopi brāhmaṇo veṭṭhadīpe bhagavato sarīrānaṃ thūpañca mahañca akāsi. Pāveyyakāpi mallā pāvāyaṃ bhagavato sarīrānaṃ thūpañca mahañca akaṃsu. Kosinārakāpi mallā kusinārāyaṃ bhagavato sarīrānaṃ thūpañca mahañca akaṃsu. Doṇopi brāhmaṇo tumbassa thūpañca mahañca akāsi. Pippalivaniyāpi moriyā pippalivane aṅgārānaṃ thūpañca mahañca akaṃsu. Iti aṭṭha sarīrathūpā navamo tumbathūpo dasamo aṅgārathūpo. Evametaṃ bhūtapubbanti.

\paragraph{240.} Aṭṭhadoṇaṃ cakkhumato sarīraṃ, sattadoṇaṃ jambudīpe mahenti.

Ekañca doṇaṃ purisavaruttamassa, rāmagāme nāgarājā maheti.

Ekāhi dāṭhā tidivehi pūjitā, ekā pana gandhārapure mahīyati;

Kāliṅgarañño vijite punekaṃ, ekaṃ pana nāgarājā maheti.

Tasseva tejena ayaṃ vasundharā,

Āyāgaseṭṭhehi mahī alaṅkatā;

Evaṃ imaṃ cakkhumato sarīraṃ,

Susakkataṃ sakkatasakkatehi.

Devindanāgindanarindapūjito ,

Manussindaseṭṭhehi tatheva pūjito;

Taṃ vandatha\footnote{taṃ taṃ vandatha (syā.)} pañjalikā labhitvā,

Buddho have kappasatehi dullabhoti.

Cattālīsa samā dantā, kesā lomā ca sabbaso;

Devā hariṃsu ekekaṃ, cakkavāḷaparamparāti.

\xsectionEnd{Mahāparinibbānasuttaṃ niṭṭhitaṃ tatiyaṃ.}





%\input{Majjhimapannasapali/Majjhimapannasapali.tex}

%\input{Uparipannasapali/Uparipannasapali.tex}



% \newpage
%\section{1. Brahmajālasuttaṃ}

%\subsection{Paribbājakakathā}

%\paragraph{1.}





\end{document}
