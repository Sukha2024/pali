\section{Ākaṅkheyyasuttaṃ}

\paragraph{1.} Evaṃ me sutaṃ – ekaṃ samayaṃ bhagavā sāvatthiyaṃ viharati jetavane anāthapiṇḍikassa ārāme. Tatra kho bhagavā bhikkhū āmantesi – ‘‘bhikkhavo’’ti. ‘‘Bhadante’’ti te bhikkhū bhagavato paccassosuṃ. Bhagavā etadavoca –

\paragraph{2.} ‘‘Sampannasīlā, bhikkhave, viharatha sampannapātimokkhā; pātimokkhasaṃvarasaṃvutā viharatha ācāragocarasampannā aṇumattesu vajjesu bhayadassāvino; samādāya sikkhatha sikkhāpadesu.

\paragraph{3.} ‘‘Ākaṅkheyya ce, bhikkhave, bhikkhu – ‘sabrahmacārīnaṃ piyo ca assaṃ manāpo ca garu ca bhāvanīyo cā’ti\footnote{manāpo garubhāvaniyo cāti (sī.)}, sīlesvevassa paripūrakārī ajjhattaṃ cetosamathamanuyutto anirākatajjhāno vipassanāya samannāgato brūhetā suññāgārānaṃ.

\paragraph{4.} ‘‘Ākaṅkheyya ce, bhikkhave, bhikkhu – ‘lābhī assaṃ cīvarapiṇḍapātasenāsanagilānappaccayabhesajjaparikkhārāna’nti, sīlesvevassa paripūrakārī ajjhattaṃ cetosamathamanuyutto anirākatajjhāno vipassanāya samannāgato brūhetā suññāgārānaṃ.

\paragraph{5.} ‘‘Ākaṅkheyya ce, bhikkhave, bhikkhu – ‘yesāhaṃ cīvarapiṇḍapātasenāsana gilānappaccayabhesajjaparikkhāraṃ paribhuñjāmi tesaṃ te kārā mahapphalā assu mahānisaṃsā’ti, sīlesvevassa paripūrakārī ajjhattaṃ cetosamathamanuyutto anirākatajjhāno vipassanāya samannāgato brūhetā suññāgārānaṃ.

\paragraph{6.} ‘‘Ākaṅkheyya ce, bhikkhave, bhikkhu – ‘ye maṃ\footnote{ye me (sī. syā.)} ñātī sālohitā petā kālaṅkatā\footnote{kālakatā (sī. syā. pī.)} pasannacittā anussaranti tesaṃ taṃ mahapphalaṃ assa mahānisaṃsa’nti, sīlesvevassa paripūrakārī ajjhattaṃ cetosamathamanuyutto anirākatajjhāno vipassanāya samannāgato brūhetā suññāgārānaṃ.

\paragraph{7.} ‘‘Ākaṅkheyya ce, bhikkhave, bhikkhu – ‘aratiratisaho assaṃ, na ca maṃ arati saheyya, uppannaṃ aratiṃ abhibhuyya abhibhuyya vihareyya’nti, sīlesvevassa paripūrakārī…pe… brūhetā suññāgārānaṃ.

\paragraph{8.} ‘‘Ākaṅkheyya ce, bhikkhave, bhikkhu – ‘bhayabheravasaho assaṃ, na ca maṃ bhayabheravaṃ saheyya, uppannaṃ bhayabheravaṃ abhibhuyya abhibhuyya vihareyya’nti, sīlesvevassa paripūrakārī…pe… brūhetā suññāgārānaṃ.

\paragraph{9.} ‘‘Ākaṅkheyya ce, bhikkhave, bhikkhu – ‘catunnaṃ jhānānaṃ ābhicetasikānaṃ diṭṭhadhammasukhavihārānaṃ nikāmalābhī assaṃ akicchalābhī akasiralābhī’ti, sīlesvevassa paripūrakārī…pe… brūhetā suññāgārānaṃ.

\paragraph{10.} ‘‘Ākaṅkheyya ce, bhikkhave, bhikkhu – ‘ye te santā vimokkhā atikkamma rūpe āruppā, te kāyena phusitvā vihareyya’nti, sīlesvevassa paripūrakārī…pe… brūhetā suññāgārānaṃ.

\paragraph{11.} ‘‘Ākaṅkheyya ce, bhikkhave, bhikkhu – ‘tiṇṇaṃ saṃyojanānaṃ parikkhayā sotāpanno assaṃ avinipātadhammo niyato sambodhiparāyaṇo’ti, sīlesvevassa paripūrakārī…pe… brūhetā suññāgārānaṃ.

\paragraph{12.} ‘‘Ākaṅkheyya ce, bhikkhave, bhikkhu – ‘tiṇṇaṃ saṃyojanānaṃ parikkhayā rāgadosamohānaṃ tanuttā sakadāgāmī assaṃ sakideva imaṃ lokaṃ āgantvā dukkhassantaṃ kareyya’nti, sīlesvevassa paripūrakārī…pe… brūhetā suññāgārānaṃ.

\paragraph{13.} ‘‘Ākaṅkheyya ce, bhikkhave, bhikkhu – ‘pañcannaṃ orambhāgiyānaṃ saṃyojanānaṃ parikkhayā opapātiko assaṃ tattha parinibbāyī anāvattidhammo tasmā lokā’ti, sīlesvevassa paripūrakārī…pe… brūhetā suññāgārānaṃ.

\paragraph{14.} ‘‘Ākaṅkheyya ce, bhikkhave, bhikkhu – ‘anekavihitaṃ iddhividhaṃ paccanubhaveyyaṃ – ekopi hutvā bahudhā assaṃ, bahudhāpi hutvā eko assaṃ; āvibhāvaṃ tirobhāvaṃ; tirokuṭṭaṃ tiropākāraṃ tiropabbataṃ asajjamāno gaccheyyaṃ, seyyathāpi ākāse; pathaviyāpi ummujjanimujjaṃ kareyyaṃ, seyyathāpi udake; udakepi abhijjamāne gaccheyyaṃ, seyyathāpi pathaviyaṃ; ākāsepi pallaṅkena kameyyaṃ, seyyathāpi pakkhī sakuṇo; imepi candimasūriye evaṃmahiddhike evaṃmahānubhāve pāṇinā parāmaseyyaṃ parimajjeyyaṃ; yāva brahmalokāpi kāyena vasaṃ vatteyya’nti, sīlesvevassa paripūrakārī…pe… brūhetā suññāgārānaṃ.

\paragraph{15.} ‘‘Ākaṅkheyya ce, bhikkhave, bhikkhu – ‘dibbāya sotadhātuyā visuddhāya atikkantamānusikāya ubho sadde suṇeyyaṃ – dibbe ca mānuse ca ye dūre santike cā’ti, sīlesvevassa paripūrakārī…pe… brūhetā suññāgārānaṃ.

\paragraph{16.} ‘‘Ākaṅkheyya ce, bhikkhave, bhikkhu – ‘parasattānaṃ parapuggalānaṃ cetasā ceto paricca pajāneyyaṃ – sarāgaṃ vā cittaṃ sarāgaṃ cittanti pajāneyyaṃ, vītarāgaṃ vā cittaṃ vītarāgaṃ cittanti pajāneyyaṃ; sadosaṃ vā cittaṃ sadosaṃ cittanti pajāneyyaṃ, vītadosaṃ vā cittaṃ vītadosaṃ cittanti pajāneyyaṃ; samohaṃ vā cittaṃ samohaṃ cittanti pajāneyyaṃ, vītamohaṃ vā cittaṃ vītamohaṃ cittanti pajāneyyaṃ; saṃkhittaṃ vā cittaṃ saṃkhittaṃ cittanti pajāneyyaṃ, vikkhittaṃ vā cittaṃ vikkhittaṃ cittanti pajāneyyaṃ; mahaggataṃ vā cittaṃ mahaggataṃ cittanti pajāneyyaṃ, amahaggataṃ vā cittaṃ amahaggataṃ cittanti pajāneyyaṃ; sauttaraṃ vā cittaṃ sauttaraṃ cittanti pajāneyyaṃ, anuttaraṃ vā cittaṃ anuttaraṃ cittanti pajāneyyaṃ; samāhitaṃ vā cittaṃ samāhitaṃ cittanti pajāneyyaṃ, asamāhitaṃ vā cittaṃ asamāhitaṃ cittanti pajāneyyaṃ; vimuttaṃ vā cittaṃ vimuttaṃ cittanti pajāneyyaṃ, avimuttaṃ vā cittaṃ avimuttaṃ cittanti pajāneyya’nti, sīlesvevassa paripūrakārī…pe… brūhetā suññāgārānaṃ.

\paragraph{17.} ‘‘Ākaṅkheyya ce, bhikkhave, bhikkhu – ‘anekavihitaṃ pubbenivāsaṃ anussareyyaṃ, seyyathidaṃ – ekampi jātiṃ dvepi jātiyo tissopi jātiyo catassopi jātiyo pañcapi jātiyo dasapi jātiyo vīsampi jātiyo tiṃsampi jātiyo cattālīsampi jātiyo paññāsampi jātiyo jātisatampi jātisahassampi jāti satasahassampi anekepi saṃvaṭṭakappe anekepi vivaṭṭakappe anekepi saṃvaṭṭavivaṭṭakappe – amutrāsiṃ evaṃnāmo evaṃgotto evaṃvaṇṇo evamāhāro evaṃsukhadukkhappaṭisaṃvedī evamāyupariyanto, so tato cuto amutra udapādiṃ; tatrāpāsiṃ evaṃnāmo evaṃgotto evaṃvaṇṇo evamāhāro evaṃsukhadukkhappaṭisaṃvedī evamāyupariyanto, so tato cuto idhūpapannoti. Iti sākāraṃ sauddesaṃ anekavihitaṃ pubbenivāsaṃ anussareyya’nti, sīlesvevassa paripūrakārī…pe… brūhetā suññāgārānaṃ.

\paragraph{18.} ‘‘Ākaṅkheyya ce, bhikkhave, bhikkhu – ‘dibbena cakkhunā visuddhena atikkantamānusakena satte passeyyaṃ cavamāne upapajjamāne hīne paṇīte suvaṇṇe dubbaṇṇe sugate duggate yathākammūpage satte pajāneyyaṃ – ime vata bhonto sattā kāyaduccaritena samannāgatā vacīduccaritena samannāgatā manoduccaritena samannāgatā ariyānaṃ upavādakā micchādiṭṭhikā micchādiṭṭhikammasamādānā, te kāyassa bhedā paraṃ maraṇā apāyaṃ duggatiṃ vinipātaṃ nirayaṃ upapannā; ime vā pana bhonto sattā kāyasucaritena samannāgatā vacīsucaritena samannāgatā manosucaritena samannāgatā ariyānaṃ anupavādakā sammādiṭṭhikā sammādiṭṭhikammasamādānā, te kāyassa bhedā paraṃ maraṇā sugatiṃ saggaṃ lokaṃ upapannāti, iti dibbena cakkhunā visuddhena atikkantamānusakena satte passeyyaṃ cavamāne upapajjamāne hīne paṇīte suvaṇṇe dubbaṇṇe sugate duggate yathākammūpage satte pajāneyya’nti, sīlesvevassa paripūrakārī ajjhattaṃ cetosamathamanuyutto anirākatajjhāno vipassanāya samannāgato brūhetā suññāgārānaṃ.

\paragraph{19.} ‘‘Ākaṅkheyya ce, bhikkhave, bhikkhu – ‘āsavānaṃ khayā anāsavaṃ cetovimuttiṃ paññāvimuttiṃ diṭṭhevadhamme sayaṃ abhiññā sacchikatvā upasampajja vihareyya’nti, sīlesvevassa paripūrakārī ajjhattaṃ cetosamathamanuyutto anirākatajjhāno vipassanāya samannāgato brūhetā suññāgārānaṃ.

\paragraph{20.} ‘‘Sampannasīlā, bhikkhave, viharatha sampannapātimokkhā; pātimokkhasaṃvarasaṃvutā viharatha ācāragocarasampannā aṇumattesu vajjesu bhayadassāvino; samādāya sikkhatha sikkhāpadesū’’ti – iti yaṃ taṃ vuttaṃ idametaṃ paṭicca vutta’’nti.

\paragraph{21.} Idamavoca bhagavā. Attamanā te bhikkhū bhagavato bhāsitaṃ abhinandunti.

\xsectionEnd{Ākaṅkheyyasuttaṃ niṭṭhitaṃ chaṭṭhaṃ.}
