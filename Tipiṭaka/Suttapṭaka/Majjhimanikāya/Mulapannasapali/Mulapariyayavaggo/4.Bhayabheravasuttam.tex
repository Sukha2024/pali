\section{Bhayabheravasuttaṃ}

\paragraph{1.} Evaṃ me sutaṃ – ekaṃ samayaṃ bhagavā sāvatthiyaṃ viharati jetavane anāthapiṇḍikassa ārāme. Atha kho jāṇussoṇi brāhmaṇo yena bhagavā tenupasaṅkami; upasaṅkamitvā bhagavatā saddhiṃ sammodi. Sammodanīyaṃ kathaṃ sāraṇīyaṃ\footnote{sārāṇīyaṃ (sī. syā. pī.)} vītisāretvā ekamantaṃ nisīdi. Ekamantaṃ nisinno kho jāṇussoṇi brāhmaṇo bhagavantaṃ etadavoca – ‘‘yeme, bho gotama, kulaputtā bhavantaṃ gotamaṃ uddissa saddhā agārasmā anagāriyaṃ pabbajitā, bhavaṃ tesaṃ gotamo pubbaṅgamo, bhavaṃ tesaṃ gotamo bahukāro, bhavaṃ tesaṃ gotamo samādapetā\footnote{samādāpetā (?)}; bhoto ca pana gotamassa sā janatā diṭṭhānugatiṃ āpajjatī’’ti. ‘‘Evametaṃ, brāhmaṇa, evametaṃ, brāhmaṇa! Ye te, brāhmaṇa, kulaputtā mamaṃ uddissa saddhā agārasmā anagāriyaṃ pabbajitā, ahaṃ tesaṃ pubbaṅgamo, ahaṃ tesaṃ bahukāro, ahaṃ tesaṃ samādapetā; mama ca pana sā janatā diṭṭhānugatiṃ āpajjatī’’ti. ‘‘Durabhisambhavāni hi kho, bho gotama, araññavanapatthāni pantāni senāsanāni, dukkaraṃ pavivekaṃ, durabhiramaṃ ekatte, haranti maññe mano vanāni samādhiṃ alabhamānassa bhikkhuno’’ti . ‘‘Evametaṃ, brāhmaṇa, evametaṃ, brāhmaṇa! Durabhisambhavāni hi kho, brāhmaṇa, araññavanapatthāni pantāni senāsanāni, dukkaraṃ pavivekaṃ, durabhiramaṃ ekatte, haranti maññe mano vanāni samādhiṃ alabhamānassa bhikkhuno’’ti.

\paragraph{2.} ‘‘Mayhampi kho, brāhmaṇa, pubbeva sambodhā anabhisambuddhassa bodhisattasseva sato etadahosi – ‘durabhisambhavāni hi kho araññavanapatthāni pantāni senāsanāni, dukkaraṃ pavivekaṃ, durabhiramaṃ ekatte, haranti maññe mano vanāni samādhiṃ alabhamānassa bhikkhuno’ti. Tassa mayhaṃ brāhmaṇa, etadahosi – ‘ye kho keci samaṇā vā brāhmaṇā vā aparisuddhakāyakammantā araññavanapatthāni pantāni senāsanāni paṭisevanti, aparisuddhakāyakammantasandosahetu have te bhonto samaṇabrāhmaṇā akusalaṃ bhayabheravaṃ avhāyanti. Na kho panāhaṃ aparisuddhakāyakammanto araññavanapatthāni pantāni senāsanāni paṭisevāmi; parisuddhakāyakammantohamasmi. Ye hi vo ariyā parisuddhakāyakammantā araññavanapatthāni pantāni senāsanāni paṭisevanti tesamahaṃ aññataro’ti. Etamahaṃ, brāhmaṇa, parisuddhakāyakammataṃ attani sampassamāno bhiyyo pallomamāpādiṃ araññe vihārāya.

\paragraph{3.} ‘‘Tassa mayhaṃ, brāhmaṇa, etadahosi – ‘ye kho keci samaṇā vā brāhmaṇā vā aparisuddhavacīkammantā…pe… aparisuddhamanokammantā …pe… aparisuddhājīvā araññavanapatthāni pantāni senāsanāni paṭisevanti, aparisuddhājīvasandosahetu have te bhonto samaṇabrāhmaṇā akusalaṃ bhayabheravaṃ avhāyanti. Na kho panāhaṃ aparisuddhājīvo araññavanapatthāni pantāni senāsanāni paṭisevāmi; parisuddhājīvohamasmi. Ye hi vo ariyā parisuddhājīvā araññavanapatthāni pantāni senāsanāni paṭisevanti tesamahaṃ aññataro’ti. Etamahaṃ, brāhmaṇa, parisuddhājīvataṃ attani sampassamāno bhiyyo pallomamāpādiṃ araññe vihārāya.

\paragraph{4.} ‘‘Tassa mayhaṃ, brāhmaṇa, etadahosi – ‘ye kho keci samaṇā vā brāhmaṇā vā abhijjhālū kāmesu tibbasārāgā araññavanapatthāni pantāni senāsanāni paṭisevanti, abhijjhālukāmesutibbasārāgasandosahetu have te bhonto samaṇabrāhmaṇā akusalaṃ bhayabheravaṃ avhāyanti. Na kho panāhaṃ abhijjhālu kāmesu tibbasārāgo araññavanapatthāni pantāni senāsanāni paṭisevāmi; anabhijjhālūhamasmi. Ye hi vo ariyā anabhijjhālū araññavanapatthāni pantāni senāsanāni paṭisevanti , tesamahaṃ aññataro’ti. Etamahaṃ, brāhmaṇa, anabhijjhālutaṃ attani sampassamāno bhiyyo pallomamāpādiṃ araññe vihārāya.

\paragraph{5.} ‘‘Tassa mayhaṃ, brāhmaṇa, etadahosi – ‘ye kho keci samaṇā vā brāhmaṇā vā byāpannacittā paduṭṭhamanasaṅkappā araññavanapatthāni pantāni senāsanāni paṭisevanti, byāpannacittapaduṭṭhamanasaṅkappasandosahetu have te bhonto samaṇabrāhmaṇā akusalaṃ bhayabheravaṃ avhāyanti. Na kho panāhaṃ byāpannacitto paduṭṭhamanasaṅkappo araññavanapatthāni pantāni senāsanāni paṭisevāmi; mettacittohamasmi. Ye hi vo ariyā mettacittā araññavanapatthāni pantāni senāsanāni paṭisevanti tesamahaṃ aññataro’ti. Etamahaṃ, brāhmaṇa, mettacittataṃ attani sampassamāno bhiyyo pallomamāpādiṃ araññe vihārāya.

\paragraph{6.} ‘‘Tassa mayhaṃ, brāhmaṇa, etadahosi – ‘ye kho keci samaṇā vā brāhmaṇā vā thīnamiddhapariyuṭṭhitā araññavanapatthāni pantāni senāsanāni paṭisevanti, thīnamiddhapariyuṭṭhānasandosahetu have te bhonto samaṇabrāhmaṇā akusalaṃ bhayabheravaṃ avhāyanti. Na kho panāhaṃ thīnamiddhapariyuṭṭhito araññavanapatthāni pantāni senāsanāni paṭisevāmi; vigatathīnamiddhohamasmi. Ye hi vo ariyā vigatathīnamiddhā araññavanapatthāni pantāni senāsanāni paṭisevanti tesamahaṃ aññataro’ti. Etamahaṃ, brāhmaṇa, vigatathīnamiddhataṃ attani sampassamāno bhiyyo pallomamāpādiṃ araññe vihārāya.

\paragraph{7.} ‘‘Tassa mayhaṃ, brāhmaṇa, etadahosi – ‘ye kho keci samaṇā vā brāhmaṇā vā uddhatā avūpasantacittā araññavanapatthāni pantāni senāsanāni paṭisevanti, uddhataavūpasantacittasandosahetu have te bhonto samaṇabrāhmaṇā akusalaṃ bhayabheravaṃ avhāyanti. Na kho panāhaṃ uddhato avūpasantacitto araññavanapatthāni pantāni senāsanāni paṭisevāmi; vūpasantacittohamasmi. Ye hi vo ariyā vūpasantacittā araññavanapatthāni pantāni senāsanāni paṭisevanti, tesamahaṃ aññataro’ti. Etamahaṃ, brāhmaṇa, vūpasantacittataṃ attani sampassamāno bhiyyo pallomamāpādiṃ araññe vihārāya.

\paragraph{8.} ‘‘Tassa mayhaṃ, brāhmaṇa, etadahosi – ‘ye kho keci samaṇā vā brāhmaṇā vā kaṅkhī vicikicchī araññavanapatthāni pantāni senāsanāni paṭisevanti, kaṅkhivicikicchisandosahetu have te bhonto samaṇabrāhmaṇā akusalaṃ bhayabheravaṃ avhāyanti. Na kho panāhaṃ kaṅkhī vicikicchī araññavanapatthāni pantāni senāsanāni paṭisevāmi; tiṇṇavicikicchohamasmi. Ye hi vo ariyā tiṇṇavicikicchā araññavanapatthāni pantāni senāsanāni paṭisevanti tesamahaṃ aññataro’ti. Etamahaṃ, brāhmaṇa, tiṇṇavicikicchataṃ attani sampassamāno bhiyyo pallomamāpādiṃ araññe vihārāya.

\paragraph{9.} ‘‘Tassa mayhaṃ, brāhmaṇa, etadahosi – ‘ye kho keci samaṇā vā brāhmaṇā vā attukkaṃsakā paravambhī araññavanapatthāni pantāni senāsanāni paṭisevanti, attukkaṃsanaparavambhanasandosahetu have te bhonto samaṇabrāhmaṇā akusalaṃ bhayabheravaṃ avhāyanti . Na kho panāhaṃ attukkaṃsako paravambhī araññavanapatthāni pantāni senāsanāni paṭisevāmi ; anattukkaṃsako aparavambhīhamasmi. Ye hi vo ariyā anattukkaṃsakā aparavambhī araññavanapatthāni pantāni senāsanāni paṭisevanti tesamahaṃ aññataro’ti. Etamahaṃ, brāhmaṇa, anattukkaṃsakataṃ aparavambhitaṃ attani sampassamāno bhiyyo pallomamāpādiṃ araññe vihārāya.

\paragraph{10.} ‘‘Tassa mayhaṃ, brāhmaṇa, etadahosi – ‘ye kho keci samaṇā vā brāhmaṇā vā chambhī bhīrukajātikā araññavanapatthāni pantāni senāsanāni paṭisevanti, chambhibhīrukajātikasandosahetu have te bhonto samaṇabrāhmaṇā akusalaṃ bhayabheravaṃ avhāyanti. Na kho panāhaṃ chambhī bhīrukajātiko araññavanapatthāni pantāni senāsanāni paṭisevāmi; vigatalomahaṃsohamasmi. Ye hi vo ariyā vigatalomahaṃsā araññavanapatthāni pantāni senāsanāni paṭisevanti tesamahaṃ aññataro’ti. Etamahaṃ, brāhmaṇa, vigatalomahaṃsataṃ attani sampassamāno bhiyyo pallomamāpādiṃ araññe vihārāya.

\paragraph{11.} ‘‘Tassa mayhaṃ, brāhmaṇa, etadahosi – ‘ye kho keci samaṇā vā brāhmaṇā vā lābhasakkārasilokaṃ nikāmayamānā araññavanapatthāni pantāni senāsanāni paṭisevanti, lābhasakkārasilokanikāmana\footnote{nikāmayamāna (sī. syā.)} sandosahetu have te bhonto samaṇabrāhmaṇā akusalaṃ bhayabheravaṃ avhāyanti. Na kho panāhaṃ lābhasakkārasilokaṃ nikāmayamāno araññavanapatthāni pantāni senāsanāni paṭisevāmi; appicchohamasmi. Ye hi vo ariyā appicchā araññavanapatthāni pantāni senāsanāni paṭisevanti tesamahaṃ aññataro’ti. Etamahaṃ, brāhmaṇa, appicchataṃ attani sampassamāno bhiyyo pallomamāpādiṃ araññe vihārāya.

\paragraph{12.} ‘‘Tassa mayhaṃ, brāhmaṇa, etadahosi – ‘ye kho keci samaṇā vā brāhmaṇā vā kusītā hīnavīriyā araññavanapatthāni pantāni senāsanāni paṭisevanti , kusītahīnavīriyasandosahetu have te bhonto samaṇabrāhmaṇā akusalaṃ bhayabheravaṃ avhāyanti. Na kho panāhaṃ kusīto hīnavīriyo araññavanapatthāni pantāni senāsanāni paṭisevāmi; āraddhavīriyohamasmi. Ye hi vo ariyā āraddhavīriyā araññavanapatthāni pantāni senāsanāni paṭisevanti tesamahaṃ aññataro’ti. Etamahaṃ, brāhmaṇa, āraddhavīriyataṃ attani sampassamāno bhiyyo pallomamāpādiṃ araññe vihārāya.

\paragraph{13.} ‘‘Tassa mayhaṃ, brāhmaṇa, etadahosi – ‘ye kho keci samaṇā vā brāhmaṇā vā muṭṭhassatī asampajānā araññavanapatthāni pantāni senāsanāni paṭisevanti, muṭṭhassatiasampajānasandosahetu have te bhonto samaṇabrāhmaṇā akusalaṃ bhayabheravaṃ avhāyanti. Na kho panāhaṃ muṭṭhassati asampajāno araññavanapatthāni pantāni senāsanāni paṭisevāmi; upaṭṭhitassatihamasmi. Ye hi vo ariyā upaṭṭhitassatī araññavanapatthāni pantāni senāsanāni paṭisevanti tesamahaṃ aññataro’ti. Etamahaṃ, brāhmaṇa, upaṭṭhitassatitaṃ attani sampassamāno bhiyyo pallomamāpādiṃ araññe vihārāya.

\paragraph{14.} ‘‘Tassa mayhaṃ, brāhmaṇa, etadahosi – ‘ye kho keci samaṇā vā brāhmaṇā vā asamāhitā vibbhantacittā araññavanapatthāni pantāni senāsanāni paṭisevanti, asamāhitavibbhantacittasandosahetu have te bhonto samaṇabrāhmaṇā akusalaṃ bhayabheravaṃ avhāyanti. Na kho panāhaṃ asamāhito vibbhantacitto araññavanapatthāni pantāni senāsanāni paṭisevāmi; samādhisampannohamasmi. Ye hi vo ariyā samādhisampannā araññavanapatthāni pantāni senāsanāni paṭisevanti tesamahaṃ aññataro’ti. Etamahaṃ, brāhmaṇa, samādhisampadaṃ attani sampassamāno bhiyyo pallomamāpādiṃ araññe vihārāya.

\paragraph{15.} ‘‘Tassa mayhaṃ, brāhmaṇa, etadahosi – ‘ye kho keci samaṇā vā brāhmaṇā vā duppaññā eḷamūgā araññavanapatthāni pantāni senāsanāni paṭisevanti, duppaññaeḷamūgasandosahetu have te bhonto samaṇabrāhmaṇā akusalaṃ bhayabheravaṃ avhāyanti. Na kho panāhaṃ duppañño eḷamūgo araññavanapatthāni pantāni senāsanāni paṭisevāmi; paññāsampannohamasmi. Ye hi vo ariyā paññāsampannā araññavanapatthāni pantāni senāsanāni paṭisevanti tesamahaṃ aññataro’ti. Etamahaṃ, brāhmaṇa, paññāsampadaṃ attani sampassamāno bhiyyo pallomamāpādiṃ araññe vihārāya.

\xsubsubsectionEnd{Soḷasapariyāyaṃ niṭṭhitaṃ.}

\paragraph{16.} ‘‘Tassa mayhaṃ, brāhmaṇa, etadahosi – ‘yaṃnūnāhaṃ yā tā rattiyo abhiññātā abhilakkhitā – cātuddasī pañcadasī aṭṭhamī ca pakkhassa – tathārūpāsu rattīsu yāni tāni ārāmacetiyāni vanacetiyāni rukkhacetiyāni bhiṃsanakāni salomahaṃsāni tathārūpesu senāsanesu vihareyyaṃ appeva nāmāhaṃ bhayabheravaṃ passeyya’nti. So kho ahaṃ, brāhmaṇa, aparena samayena yā tā rattiyo abhiññātā abhilakkhitā – cātuddasī pañcadasī aṭṭhamī ca pakkhassa – tathārūpāsu rattīsu yāni tāni ārāmacetiyāni vanacetiyāni rukkhacetiyāni bhiṃsanakāni salomahaṃsāni tathārūpesu senāsanesu viharāmi. Tattha ca me, brāhmaṇa, viharato mago vā āgacchati, moro vā kaṭṭhaṃ pāteti, vāto vā paṇṇakasaṭaṃ\footnote{paṇṇasaṭaṃ (sī. pī.)} ereti; tassa mayhaṃ brāhmaṇa etadahosi\footnote{tassa mayhaṃ evaṃ hoti (sī. syā.)} – ‘etaṃ nūna taṃ bhayabheravaṃ āgacchatī’ti. Tassa mayhaṃ, brāhmaṇa, etadahosi – ‘kiṃ nu kho ahaṃ aññadatthu bhayapaṭikaṅkhī\footnote{bhayapāṭikaṅkhī (sī.)} viharāmi? Yaṃnūnāhaṃ yathābhūtaṃ yathābhūtassa\footnote{yathābhūtassa yathābhūtassa (sī. syā.)} me taṃ bhayabheravaṃ āgacchati, tathābhūtaṃ tathābhūtova\footnote{yathābhūto yathābhūtova (sī. syā.)} taṃ bhayabheravaṃ paṭivineyya’nti. Tassa mayhaṃ, brāhmaṇa, caṅkamantassa taṃ bhayabheravaṃ āgacchati. So kho ahaṃ, brāhmaṇa, neva tāva tiṭṭhāmi na nisīdāmi na nipajjāmi, yāva caṅkamantova taṃ bhayabheravaṃ paṭivinemi. Tassa mayhaṃ, brāhmaṇa, ṭhitassa taṃ bhayabheravaṃ āgacchati. So kho ahaṃ, brāhmaṇa, neva tāva caṅkamāmi na nisīdāmi na nipajjāmi. Yāva ṭhitova taṃ bhayabheravaṃ paṭivinemi. Tassa mayhaṃ, brāhmaṇa, nisinnassa taṃ bhayabheravaṃ āgacchati. So kho ahaṃ, brāhmaṇa, neva tāva nipajjāmi na tiṭṭhāmi na caṅkamāmi, yāva nisinnova taṃ bhayabheravaṃ paṭivinemi. Tassa mayhaṃ, brāhmaṇa, nipannassa taṃ bhayabheravaṃ āgacchati. So kho ahaṃ, brāhmaṇa, neva tāva nisīdāmi na tiṭṭhāmi na caṅkamāmi, yāva nipannova taṃ bhayabheravaṃ paṭivinemi.

\paragraph{17.} ‘‘Santi kho pana, brāhmaṇa, eke samaṇabrāhmaṇā rattiṃyeva samānaṃ divāti sañjānanti, divāyeva samānaṃ rattīti sañjānanti. Idamahaṃ tesaṃ samaṇabrāhmaṇānaṃ sammohavihārasmiṃ vadāmi. Ahaṃ kho pana, brāhmaṇa, rattiṃyeva samānaṃ rattīti sañjānāmi, divāyeva samānaṃ divāti sañjānāmi. Yaṃ kho taṃ, brāhmaṇa, sammā vadamāno vadeyya – ‘asammohadhammo satto loke uppanno bahujanahitāya bahujanasukhāya lokānukampāya atthāya hitāya sukhāya devamanussāna’nti, mameva taṃ sammā vadamāno vadeyya – ‘asammohadhammo satto loke uppanno bahujanahitāya bahujanasukhāya lokānukampāya atthāya hitāya sukhāya devamanussāna’nti.

\paragraph{18.} ‘‘Āraddhaṃ kho pana me, brāhmaṇa, vīriyaṃ ahosi asallīnaṃ, upaṭṭhitā sati asammuṭṭhā\footnote{appammuṭṭhā (syā.)}, passaddho kāyo asāraddho, samāhitaṃ cittaṃ ekaggaṃ. So kho ahaṃ, brāhmaṇa, vivicceva kāmehi vivicca akusalehi dhammehi savitakkaṃ savicāraṃ vivekajaṃ pītisukhaṃ paṭhamaṃ jhānaṃ upasampajja vihāsiṃ. Vitakkavicārānaṃ vūpasamā ajjhattaṃ sampasādanaṃ cetaso ekodibhāvaṃ avitakkaṃ avicāraṃ samādhijaṃ pītisukhaṃ dutiyaṃ jhānaṃ upasampajja vihāsiṃ. Pītiyā ca virāgā upekkhako ca vihāsiṃ, sato ca sampajāno sukhañca kāyena paṭisaṃvedesiṃ; yaṃ taṃ ariyā ācikkhanti – ‘upekkhako satimā sukhavihārī’ti tatiyaṃ jhānaṃ upasampajja vihāsiṃ. Sukhassa ca pahānā dukkhassa ca pahānā pubbeva somanassadomanassānaṃ atthaṅgamā adukkhamasukhaṃ upekkhāsatipārisuddhiṃ catutthaṃ jhānaṃ upasampajja vihāsiṃ.

\paragraph{19.} ‘‘So evaṃ samāhite citte parisuddhe pariyodāte anaṅgaṇe vigatūpakkilese mudubhūte kammaniye ṭhite āneñjappatte pubbenivāsānussatiñāṇāya cittaṃ abhininnāmesiṃ. So anekavihitaṃ pubbenivāsaṃ anussarāmi, seyyathidaṃ – ekampi jātiṃ dvepi jātiyo tissopi jātiyo catassopi jātiyo pañcapi jātiyo dasapi jātiyo vīsampi jātiyo tiṃsampi jātiyo cattālīsampi jātiyo paññāsampi jātiyo jātisatampi jātisahassampi jātisatasahassampi anekepi saṃvaṭṭakappe anekepi vivaṭṭakappe anekepi saṃvaṭṭavivaṭṭakappe – ‘amutrāsiṃ evaṃnāmo evaṃgotto evaṃvaṇṇo evamāhāro evaṃsukhadukkhappaṭisaṃvedī evamāyupariyanto, so tato cuto amutra udapādiṃ; tatrāpāsiṃ evaṃnāmo evaṃgotto evaṃvaṇṇo evamāhāro evaṃsukhadukkhappaṭisaṃvedī evamāyupariyanto, so tato cuto idhūpapanno’ti. Iti sākāraṃ sauddesaṃ anekavihitaṃ pubbenivāsaṃ anussarāmi. Ayaṃ kho me, brāhmaṇa, rattiyā paṭhame yāme paṭhamā vijjā adhigatā, avijjā vihatā vijjā uppannā, tamo vihato āloko uppanno, yathā taṃ appamattassa ātāpino pahitattassa viharato.

\paragraph{20.} ‘‘So evaṃ samāhite citte parisuddhe pariyodāte anaṅgaṇe vigatūpakkilese mudubhūte kammaniye ṭhite āneñjappatte sattānaṃ cutūpapātañāṇāya cittaṃ abhininnāmesiṃ. So dibbena cakkhunā visuddhena atikkantamānusakena satte passāmi cavamāne upapajjamāne hīne paṇīte suvaṇṇe dubbaṇṇe sugate duggate yathākammūpage satte pajānāmi – ‘ime vata bhonto sattā kāyaduccaritena samannāgatā vacīduccaritena samannāgatā manoduccaritena samannāgatā ariyānaṃ upavādakā micchādiṭṭhikā micchādiṭṭhikammasamādānā; te kāyassa bhedā paraṃ maraṇā apāyaṃ duggatiṃ vinipātaṃ nirayaṃ upapannā. Ime vā pana bhonto sattā kāyasucaritena samannāgatā vacīsucaritena samannāgatā manosucaritena samannāgatā ariyānaṃ anupavādakā sammādiṭṭhikā sammādiṭṭhikammasamādānā; te kāyassa bhedā paraṃ maraṇā sugatiṃ saggaṃ lokaṃ upapannā’ti. Iti dibbena cakkhunā visuddhena atikkantamānusakena satte passāmi cavamāne upapajjamāne hīne paṇīte suvaṇṇe dubbaṇṇe sugate duggate yathākammūpage satte pajānāmi. Ayaṃ kho me, brāhmaṇa, rattiyā majjhime yāme dutiyā vijjā adhigatā, avijjā vihatā vijjā uppannā, tamo vihato āloko uppanno, yathā taṃ appamattassa ātāpino pahitattassa viharato.

\paragraph{21.} ‘‘So evaṃ samāhite citte parisuddhe pariyodāte anaṅgaṇe vigatūpakkilese mudubhūte kammaniye ṭhite āneñjappatte āsavānaṃ khayañāṇāya cittaṃ abhininnāmesiṃ. So ‘idaṃ dukkha’nti yathābhūtaṃ abbhaññāsiṃ, ‘ayaṃ dukkhasamudayo’ti yathābhūtaṃ abbhaññāsiṃ, ‘ayaṃ dukkhanirodho’ti yathābhūtaṃ abbhaññāsiṃ, ‘ayaṃ dukkhanirodhagāminī paṭipadā’ti yathābhūtaṃ abbhaññāsiṃ. ‘Ime āsavā’ti yathābhūtaṃ abbhaññāsiṃ, ‘ayaṃ āsavasamudayo’ti yathābhūtaṃ abbhaññāsiṃ, ‘ayaṃ āsavanirodho’ti yathābhūtaṃ abbhaññāsiṃ, ‘ayaṃ āsavanirodhagāminī paṭipadā’ti yathābhūtaṃ abbhaññāsiṃ. Tassa me evaṃ jānato evaṃ passato kāmāsavāpi cittaṃ vimuccittha, bhavāsavāpi cittaṃ vimuccittha, avijjāsavāpi cittaṃ vimuccittha. Vimuttasmiṃ vimuttamiti ñāṇaṃ ahosi. ‘Khīṇā jāti, vusitaṃ brahmacariyaṃ, kataṃ karaṇīyaṃ, nāparaṃ itthattāyā’ti abbhaññāsiṃ. Ayaṃ kho me, brāhmaṇa, rattiyā pacchime yāme tatiyā vijjā adhigatā, avijjā vihatā vijjā uppannā, tamo vihato āloko uppanno, yathā taṃ appamattassa ātāpino pahitattassa viharato.

\paragraph{22.} ‘‘Siyā kho pana te, brāhmaṇa, evamassa – ‘ajjāpi nūna samaṇo gotamo avītarāgo avītadoso avītamoho, tasmā araññavanapatthāni pantāni senāsanāni paṭisevatī’ti. Na kho panetaṃ, brāhmaṇa, evaṃ daṭṭhabbaṃ. Dve kho ahaṃ, brāhmaṇa, atthavase sampassamāno araññavanapatthāni pantāni senāsanāni paṭisevāmi – attano ca diṭṭhadhammasukhavihāraṃ sampassamāno, pacchimañca janataṃ anukampamāno’’ti.

\paragraph{23.} ‘‘Anukampitarūpā vatāyaṃ bhotā gotamena pacchimā janatā , yathā taṃ arahatā sammāsambuddhena. Abhikkantaṃ, bho gotama! Abhikkantaṃ, bho gotama! Seyyathāpi, bho gotama, nikkujjitaṃ vā ukkujjeyya, paṭicchannaṃ vā vivareyya, mūḷhassa vā maggaṃ ācikkheyya, andhakāre vā telapajjotaṃ dhāreyya – ‘cakkhumanto rūpāni dakkhantī’ti; evamevaṃ bhotā gotamena anekapariyāyena dhammo pakāsito. Esāhaṃ bhavantaṃ gotamaṃ saraṇaṃ gacchāmi dhammañca bhikkhusaṅghañca. Upāsakaṃ maṃ bhavaṃ gotamo dhāretu ajjatagge pāṇupetaṃ saraṇaṃ gata’’nti.

\xsectionEnd{Bhayabheravasuttaṃ niṭṭhitaṃ catutthaṃ.}
