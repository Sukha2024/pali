\section{Mūlapariyāyasuttaṃ}

\paragraph{1.} Evaṃ me sutaṃ – ekaṃ samayaṃ bhagavā ukkaṭṭhāyaṃ viharati subhagavane sālarājamūle. Tatra kho bhagavā bhikkhū āmantesi – ‘‘bhikkhavo’’ti. ‘‘Bhadante’’ti te bhikkhū bhagavato paccassosuṃ. Bhagavā etadavoca – ‘‘sabbadhammamūlapariyāyaṃ vo, bhikkhave, desessāmi. Taṃ suṇātha, sādhukaṃ manasi karotha, bhāsissāmī’’ti. ‘‘Evaṃ, bhante’’ti kho te bhikkhū bhagavato paccassosuṃ. Bhagavā etadavoca –

\paragraph{2.} ‘‘Idha, bhikkhave, assutavā puthujjano ariyānaṃ adassāvī ariyadhammassa akovido ariyadhamme avinīto, sappurisānaṃ adassāvī sappurisadhammassa akovido sappurisadhamme avinīto – pathaviṃ\footnote{paṭhaviṃ (sī. syā. kaṃ. pī.)} pathavito sañjānāti; pathaviṃ pathavito saññatvā pathaviṃ maññati, pathaviyā maññati, pathavito maññati, pathaviṃ meti maññati , pathaviṃ abhinandati. Taṃ kissa hetu? ‘Apariññātaṃ tassā’ti vadāmi.

\paragraph{3.} ‘‘Āpaṃ āpato sañjānāti; āpaṃ āpato saññatvā āpaṃ maññati, āpasmiṃ maññati, āpato maññati, āpaṃ meti maññati, āpaṃ abhinandati. Taṃ kissa hetu? ‘Apariññātaṃ tassā’ti vadāmi.

\paragraph{4.} ‘‘Tejaṃ tejato sañjānāti; tejaṃ tejato saññatvā tejaṃ maññati, tejasmiṃ maññati, tejato maññati, tejaṃ meti maññati, tejaṃ abhinandati. Taṃ kissa hetu? ‘Apariññātaṃ tassā’ti vadāmi.

\paragraph{5.} ‘‘Vāyaṃ vāyato sañjānāti; vāyaṃ vāyato saññatvā vāyaṃ maññati, vāyasmiṃ maññati, vāyato maññati, vāyaṃ meti maññati, vāyaṃ abhinandati. Taṃ kissa hetu? ‘Apariññātaṃ tassā’ti vadāmi.

\paragraph{6.} . ‘‘Bhūte bhūtato sañjānāti; bhūte bhūtato saññatvā bhūte maññati, bhūtesu maññati, bhūtato maññati, bhūte meti maññati, bhūte abhinandati. Taṃ kissa hetu? ‘Apariññātaṃ tassā’ti vadāmi.

\paragraph{7.} ‘‘Deve devato sañjānāti; deve devato saññatvā deve maññati, devesu maññati, devato maññati, deve meti maññati, deve abhinandati. Taṃ kissa hetu? ‘Apariññātaṃ tassā’ti vadāmi.

\paragraph{8.} ‘‘Pajāpatiṃ pajāpatito sañjānāti; pajāpatiṃ pajāpatito saññatvā pajāpatiṃ maññati, pajāpatismiṃ maññati, pajāpatito maññati, pajāpatiṃ meti maññati, pajāpatiṃ abhinandati. Taṃ kissa hetu? ‘Apariññātaṃ tassā’ti vadāmi.

\paragraph{9.} ‘‘Brahmaṃ brahmato sañjānāti; brahmaṃ brahmato saññatvā brahmaṃ maññati , brahmasmiṃ maññati, brahmato maññati, brahmaṃ meti maññati, brahmaṃ abhinandati. Taṃ kissa hetu? ‘Apariññātaṃ tassā’ti vadāmi.

\paragraph{10.} ‘‘Ābhassare ābhassarato sañjānāti; ābhassare ābhassarato saññatvā ābhassare maññati, ābhassaresu maññati, ābhassarato maññati, ābhassare meti maññati, ābhassare abhinandati. Taṃ kissa hetu? ‘Apariññātaṃ tassā’ti vadāmi.

\paragraph{11.} ‘‘Subhakiṇhe subhakiṇhato sañjānāti; subhakiṇhe subhakiṇhato saññatvā subhakiṇhe maññati, subhakiṇhesu maññati, subhakiṇhato maññati, subhakiṇhe meti maññati, subhakiṇhe abhinandati. Taṃ kissa hetu? ‘Apariññātaṃ tassā’ti vadāmi.

\paragraph{12.} ‘‘Vehapphale vehapphalato sañjānāti; vehapphale vehapphalato saññatvā vehapphale maññati, vehapphalesu maññati, vehapphalato maññati, vehapphale meti maññati, vehapphale abhinandati. Taṃ kissa hetu? ‘Apariññātaṃ tassā’ti vadāmi.

\paragraph{13.} ‘‘Abhibhuṃ abhibhūto sañjānāti; abhibhuṃ abhibhūto saññatvā abhibhuṃ maññati, abhibhusmiṃ maññati, abhibhūto maññati, abhibhuṃ meti maññati, abhibhuṃ abhinandati. Taṃ kissa hetu? ‘Apariññātaṃ tassā’ti vadāmi.

\paragraph{14.} . ‘‘Ākāsānañcāyatanaṃ ākāsānañcāyatanato sañjānāti; ākāsānañcāyatanaṃ ākāsānañcāyatanato saññatvā ākāsānañcāyatanaṃ maññati, ākāsānañcāyatanasmiṃ maññati, ākāsānañcāyatanato maññati, ākāsānañcāyatanaṃ meti maññati, ākāsānañcāyatanaṃ abhinandati. Taṃ kissa hetu? ‘Apariññātaṃ tassā’ti vadāmi.

\paragraph{15.} ‘‘Viññāṇañcāyatanaṃ viññāṇañcāyatanato sañjānāti; viññāṇañcāyatanaṃ viññāṇañcāyatanato saññatvā viññāṇañcāyatanaṃ maññati, viññāṇañcāyatanasmiṃ maññati, viññāṇañcāyatanato maññati, viññāṇañcāyatanaṃ meti maññati, viññāṇañcāyatanaṃ abhinandati. Taṃ kissa hetu? ‘Apariññātaṃ tassā’ti vadāmi.

\paragraph{16.} ‘‘Ākiñcaññāyatanaṃ ākiñcaññāyatanato sañjānāti; ākiñcaññāyatanaṃ ākiñcaññāyatanato saññatvā ākiñcaññāyatanaṃ maññati, ākiñcaññāyatanasmiṃ maññati, ākiñcaññāyatanato maññati, ākiñcaññāyatanaṃ meti maññati, ākiñcaññāyatanaṃ abhinandati. Taṃ kissa hetu? ‘Apariññātaṃ tassā’ti vadāmi.

\paragraph{17.} ‘‘Nevasaññānāsaññāyatanaṃ nevasaññānāsaññāyatanato sañjānāti; nevasaññānāsaññāyatanaṃ nevasaññānāsaññāyatanato saññatvā nevasaññānāsaññāyatanaṃ maññati, nevasaññānāsaññāyatanasmiṃ maññati, nevasaññānāsaññāyatanato maññati, nevasaññānāsaññāyatanaṃ meti maññati, nevasaññānāsaññāyatanaṃ abhinandati. Taṃ kissa hetu? ‘Apariññātaṃ tassā’ti vadāmi.

\paragraph{18.} . ‘‘Diṭṭhaṃ diṭṭhato sañjānāti; diṭṭhaṃ diṭṭhato saññatvā diṭṭhaṃ maññati, diṭṭhasmiṃ maññati, diṭṭhato maññati, diṭṭhaṃ meti maññati, diṭṭhaṃ abhinandati. Taṃ kissa hetu? ‘Apariññātaṃ tassā’ti vadāmi.

\paragraph{19.} ‘‘Sutaṃ sutato sañjānāti; sutaṃ sutato saññatvā sutaṃ maññati, sutasmiṃ maññati, sutato maññati, sutaṃ meti maññati, sutaṃ abhinandati. Taṃ kissa hetu? ‘Apariññātaṃ tassā’ti vadāmi.

\paragraph{20.} ‘‘Mutaṃ mutato sañjānāti; mutaṃ mutato saññatvā mutaṃ maññati, mutasmiṃ maññati, mutato maññati, mutaṃ meti maññati, mutaṃ abhinandati. Taṃ kissa hetu? ‘Apariññātaṃ tassā’ti vadāmi.

\paragraph{21.} ‘‘Viññātaṃ viññātato sañjānāti; viññātaṃ viññātato saññatvā viññātaṃ maññati, viññātasmiṃ maññati, viññātato maññati, viññātaṃ meti maññati, viññātaṃ abhinandati. Taṃ kissa hetu? ‘Apariññātaṃ tassā’ti vadāmi.

\paragraph{22.} . ‘‘Ekattaṃ ekattato sañjānāti; ekattaṃ ekattato saññatvā ekattaṃ maññati, ekattasmiṃ maññati, ekattato maññati, ekattaṃ meti maññati, ekattaṃ abhinandati. Taṃ kissa hetu? ‘Apariññātaṃ tassā’ti vadāmi.

\paragraph{23.} ‘‘Nānattaṃ nānattato sañjānāti; nānattaṃ nānattato saññatvā nānattaṃ maññati, nānattasmiṃ maññati, nānattato maññati, nānattaṃ meti maññati, nānattaṃ abhinandati. Taṃ kissa hetu? ‘Apariññātaṃ tassā’ti vadāmi.

\paragraph{24.} ‘‘Sabbaṃ sabbato sañjānāti; sabbaṃ sabbato saññatvā sabbaṃ maññati, sabbasmiṃ maññati, sabbato maññati, sabbaṃ meti maññati, sabbaṃ abhinandati. Taṃ kissa hetu? ‘Apariññātaṃ tassā’ti vadāmi.

\paragraph{25.} ‘‘Nibbānaṃ nibbānato sañjānāti; nibbānaṃ nibbānato saññatvā nibbānaṃ maññati, nibbānasmiṃ maññati , nibbānato maññati, nibbānaṃ meti maññati, nibbānaṃ abhinandati. Taṃ kissa hetu? ‘Apariññātaṃ tassā’ti vadāmi.

\xsubsubsectionEnd{Puthujjanavasena paṭhamanayabhūmiparicchedo niṭṭhito.}

\paragraph{26.} . ‘‘Yopi so, bhikkhave, bhikkhu sekkho\footnote{sekho (sī. syā. kaṃ. pī.)} appattamānaso anuttaraṃ yogakkhemaṃ patthayamāno viharati, sopi pathaviṃ pathavito abhijānāti; pathaviṃ pathavito abhiññāya\footnote{abhiññatvā (ka.)} pathaviṃ mā maññi\footnote{vā maññati}, pathaviyā mā maññi, pathavito mā maññi, pathaviṃ meti mā maññi, pathaviṃ mābhinandi\footnote{vā abhinandati (sī.) ṭīkā oloketabbā}. Taṃ kissa hetu? ‘Pariññeyyaṃ tassā’ti vadāmi.

\paragraph{27.} ‘‘Āpaṃ…pe… tejaṃ… vāyaṃ… bhūte… deve… pajāpatiṃ… brahmaṃ… ābhassare… subhakiṇhe… vehapphale… abhibhuṃ… ākāsānañcāyatanaṃ… viññāṇañcāyatanaṃ… ākiñcaññāyatanaṃ… nevasaññānāsaññāyatanaṃ… diṭṭhaṃ… sutaṃ… mutaṃ… viññātaṃ… ekattaṃ… nānattaṃ… sabbaṃ… nibbānaṃ nibbānato abhijānāti; nibbānaṃ nibbānato abhiññāya nibbānaṃ mā maññi, nibbānasmiṃ mā maññi, nibbānato mā maññi, nibbānaṃ meti mā maññi, nibbānaṃ mābhinandi. Taṃ kissa hetu? ‘Pariññeyyaṃ tassā’ti vadāmi.

\xsubsubsectionEnd{Sekkhavasena\protect\footnote{satthāravasena (sī.), satthuvasena (syā. ka.)} dutiyanayabhūmiparicchedo niṭṭhito.}

\paragraph{28.} . ‘‘Yopi so, bhikkhave, bhikkhu arahaṃ khīṇāsavo vusitavā katakaraṇīyo ohitabhāro anuppattasadattho parikkhīṇabhavasaṃyojano sammadaññā vimutto, sopi pathaviṃ pathavito abhijānāti; pathaviṃ pathavito abhiññāya pathaviṃ na maññati, pathaviyā na maññati, pathavito na maññati, pathaviṃ meti na maññati, pathaviṃ nābhinandati. Taṃ kissa hetu? ‘Pariññātaṃ tassā’ti vadāmi.

\paragraph{29.} ‘‘Āpaṃ…pe… tejaṃ… vāyaṃ… bhūte… deve… pajāpatiṃ… brahmaṃ… ābhassare… subhakiṇhe… vehapphale… abhibhuṃ… ākāsānañcāyatanaṃ… viññāṇañcāyatanaṃ… ākiñcaññāyatanaṃ… nevasaññānāsaññāyatanaṃ… diṭṭhaṃ… sutaṃ… mutaṃ… viññātaṃ… ekattaṃ… nānattaṃ… sabbaṃ… nibbānaṃ nibbānato abhijānāti; nibbānaṃ nibbānato abhiññāya nibbānaṃ na maññati, nibbānasmiṃ na maññati, nibbānato na maññati, nibbānaṃ meti na maññati, nibbānaṃ nābhinandati. Taṃ kissa hetu? ‘Pariññātaṃ tassā’ti vadāmi.

\xsubsubsectionEnd{Khīṇāsavavasena tatiyanayabhūmiparicchedo niṭṭhito.}

\paragraph{30.} . ‘‘Yopi so, bhikkhave, bhikkhu arahaṃ khīṇāsavo vusitavā katakaraṇīyo ohitabhāro anuppattasadattho parikkhīṇabhavasaṃyojano sammadaññā vimutto, sopi pathaviṃ pathavito abhijānāti; pathaviṃ pathavito abhiññāya pathaviṃ na maññati, pathaviyā na maññati, pathavito na maññati, pathaviṃ meti na maññati, pathaviṃ nābhinandati. Taṃ kissa hetu? Khayā rāgassa, vītarāgattā.

\paragraph{31.} ‘‘Āpaṃ…pe… tejaṃ… vāyaṃ… bhūte… deve… pajāpatiṃ… brahmaṃ… ābhassare… subhakiṇhe… vehapphale… abhibhuṃ… ākāsānañcāyatanaṃ… viññāṇañcāyatanaṃ… ākiñcaññāyatanaṃ … nevasaññānāsaññāyatanaṃ … diṭṭhaṃ… sutaṃ… mutaṃ… viññātaṃ… ekattaṃ… nānattaṃ… sabbaṃ… nibbānaṃ nibbānato abhijānāti; nibbānaṃ nibbānato abhiññāya nibbānaṃ na maññati, nibbānasmiṃ na maññati, nibbānato na maññati, nibbānaṃ meti na maññati, nibbānaṃ nābhinandati. Taṃ kissa hetu? Khayā rāgassa, vītarāgattā.

\xsubsubsectionEnd{Khīṇāsavavasena catutthanayabhūmiparicchedo niṭṭhito.}

\paragraph{32.} ‘‘Yopi so, bhikkhave, bhikkhu arahaṃ khīṇāsavo vusitavā katakaraṇīyo ohitabhāro anuppattasadattho parikkhīṇabhavasaṃyojano sammadaññā vimutto, sopi pathaviṃ pathavito abhijānāti; pathaviṃ pathavito abhiññāya pathaviṃ na maññati, pathaviyā na maññati, pathavito na maññati, pathaviṃ meti na maññati, pathaviṃ nābhinandati. Taṃ kissa hetu? Khayā dosassa, vītadosattā.

\paragraph{33.} ‘‘Āpaṃ…pe… tejaṃ… vāyaṃ… bhūte… deve… pajāpatiṃ… brahmaṃ… ābhassare… subhakiṇhe… vehapphale… abhibhuṃ… ākāsānañcāyatanaṃ… viññāṇañcāyatanaṃ… ākiñcaññāyatanaṃ… nevasaññānāsaññāyatanaṃ… diṭṭhaṃ… sutaṃ… mutaṃ… viññātaṃ… ekattaṃ… nānattaṃ… sabbaṃ… nibbānaṃ nibbānato abhijānāti; nibbānaṃ nibbānato abhiññāya nibbānaṃ na maññati, nibbānasmiṃ na maññati, nibbānato na maññati, nibbānaṃ meti na maññati, nibbānaṃ nābhinandati. Taṃ kissa hetu? Khayā dosassa, vītadosattā.

\xsubsubsectionEnd{Khīṇāsavavasena pañcamanayabhūmiparicchedo niṭṭhito.}

\paragraph{34.} ‘‘Yopi so, bhikkhave, bhikkhu arahaṃ khīṇāsavo vusitavā katakaraṇīyo ohitabhāro anuppattasadattho parikkhīṇabhavasaṃyojano sammadaññā vimutto, sopi pathaviṃ pathavito abhijānāti; pathaviṃ pathavito abhiññāya pathaviṃ na maññati, pathaviyā na maññati, pathavito na maññati, pathaviṃ meti na maññati, pathaviṃ nābhinandati. Taṃ kissa hetu? Khayā mohassa, vītamohattā.

\paragraph{35.} ‘‘Āpaṃ…pe… tejaṃ… vāyaṃ… bhūte… deve… pajāpatiṃ… brahmaṃ… ābhassare… subhakiṇhe… vehapphale… abhibhuṃ… ākāsānañcāyatanaṃ… viññāṇañcāyatanaṃ… ākiñcaññāyatanaṃ … nevasaññānāsaññāyatanaṃ… diṭṭhaṃ… sutaṃ… mutaṃ… viññātaṃ… ekattaṃ… nānattaṃ… sabbaṃ… nibbānaṃ nibbānato abhijānāti; nibbānaṃ nibbānato abhiññāya nibbānaṃ na maññati, nibbānasmiṃ na maññati, nibbānato na maññati, nibbānaṃ meti na maññati, nibbānaṃ nābhinandati. Taṃ kissa hetu? Khayā mohassa, vītamohattā.

\xsubsubsectionEnd{Khīṇāsavavasena chaṭṭhanayabhūmiparicchedo niṭṭhito.}

\paragraph{36.} ‘‘Tathāgatopi, bhikkhave, arahaṃ sammāsambuddho pathaviṃ pathavito abhijānāti; pathaviṃ pathavito abhiññāya pathaviṃ na maññati, pathaviyā na maññati, pathavito na maññati, pathaviṃ meti na maññati, pathaviṃ nābhinandati . Taṃ kissa hetu? ‘Pariññātantaṃ tathāgatassā’ti vadāmi.

\paragraph{37.} ‘‘Āpaṃ…pe… tejaṃ… vāyaṃ… bhūte… deve… pajāpatiṃ… brahmaṃ… ābhassare… subhakiṇhe… vehapphale… abhibhuṃ… ākāsānañcāyatanaṃ… viññāṇañcāyatanaṃ … ākiñcaññāyatanaṃ… nevasaññānāsaññāyatanaṃ… diṭṭhaṃ… sutaṃ… mutaṃ… viññātaṃ… ekattaṃ… nānattaṃ… sabbaṃ… nibbānaṃ nibbānato abhijānāti; nibbānaṃ nibbānato abhiññāya nibbānaṃ na maññati, nibbānasmiṃ na maññati, nibbānato na maññati, nibbānaṃ meti na maññati, nibbānaṃ nābhinandati. Taṃ kissa hetu? ‘Pariññātantaṃ tathāgatassā’ti vadāmi.

\xsubsubsectionEnd{Tathāgatavasena sattamanayabhūmiparicchedo niṭṭhito.}

\paragraph{38.} ‘‘Tathāgatopi , bhikkhave, arahaṃ sammāsambuddho pathaviṃ pathavito abhijānāti; pathaviṃ pathavito abhiññāya pathaviṃ na maññati, pathaviyā na maññati, pathavito na maññati, pathaviṃ meti na maññati, pathaviṃ nābhinandati. Taṃ kissa hetu? ‘Nandī\footnote{nandi (sī. syā.)} dukkhassa mūla’nti – iti viditvā ‘bhavā jāti bhūtassa jarāmaraṇa’nti. Tasmātiha, bhikkhave, ‘tathāgato sabbaso taṇhānaṃ khayā virāgā nirodhā cāgā paṭinissaggā anuttaraṃ sammāsambodhiṃ abhisambuddho’ti vadāmi.

\paragraph{39.} ‘‘Āpaṃ …pe… tejaṃ… vāyaṃ… bhūte… deve… pajāpatiṃ… brahmaṃ… ābhassare… subhakiṇhe… vehapphale… abhibhuṃ… ākāsānañcāyatanaṃ… viññāṇañcāyatanaṃ… ākiñcaññāyatanaṃ… nevasaññānāsaññāyatanaṃ… diṭṭhaṃ… sutaṃ… mutaṃ… viññātaṃ… ekattaṃ… nānattaṃ… sabbaṃ… nibbānaṃ nibbānato abhijānāti; nibbānaṃ nibbānato abhiññāya nibbānaṃ na maññati, nibbānasmiṃ na maññati, nibbānato na maññati, nibbānaṃ meti na maññati, nibbānaṃ nābhinandati. Taṃ kissa hetu? ‘Nandī dukkhassa mūla’nti – iti viditvā ‘bhavā jāti bhūtassa jarāmaraṇa’nti. Tasmātiha, bhikkhave, ‘tathāgato sabbaso taṇhānaṃ khayā virāgā nirodhā cāgā paṭinissaggā anuttaraṃ sammāsambodhiṃ abhisambuddho’ti vadāmī’’ti.

\xsubsubsectionEnd{Tathāgatavasena aṭṭhamanayabhūmiparicchedo niṭṭhito.}

\paragraph{40.} Idamavoca bhagavā. Na te bhikkhū\footnote{na attamanā tebhikkhū (syā.), te bhikkhū (pī. ka.)} bhagavato bhāsitaṃ abhinandunti.

\xsectionEnd{Mūlapariyāyasuttaṃ niṭṭhitaṃ paṭhamaṃ.}
