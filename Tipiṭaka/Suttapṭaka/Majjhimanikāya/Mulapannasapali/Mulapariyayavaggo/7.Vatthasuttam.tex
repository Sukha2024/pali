\section{Vatthasuttaṃ}

\paragraph{1.}  Evaṃ me sutaṃ – ekaṃ samayaṃ bhagavā sāvatthiyaṃ viharati jetavane anāthapiṇḍikassa ārāme. Tatra kho bhagavā bhikkhū āmantesi – ‘‘bhikkhavo’’ti. ‘‘Bhadante’’ti te bhikkhū bhagavato paccassosuṃ. Bhagavā etadavoca –

\paragraph{2.} ‘‘Seyyathāpi, bhikkhave, vatthaṃ saṃkiliṭṭhaṃ malaggahitaṃ; tamenaṃ rajako yasmiṃ yasmiṃ raṅgajāte upasaṃhareyya – yadi nīlakāya yadi pītakāya yadi lohitakāya yadi mañjiṭṭhakāya\footnote{mañjeṭṭhakāya (sī. pī.), mañjeṭṭhikāya (syā.)} durattavaṇṇamevassa aparisuddhavaṇṇamevassa. Taṃ kissa hetu? Aparisuddhattā, bhikkhave, vatthassa. Evameva kho, bhikkhave, citte saṃkiliṭṭhe, duggati pāṭikaṅkhā. Seyyathāpi, bhikkhave, vatthaṃ parisuddhaṃ pariyodātaṃ; tamenaṃ rajako yasmiṃ yasmiṃ raṅgajāte upasaṃhareyya – yadi nīlakāya yadi pītakāya yadi lohitakāya yadi mañjiṭṭhakāya – surattavaṇṇamevassa parisuddhavaṇṇamevassa. Taṃ kissa hetu? Parisuddhattā, bhikkhave, vatthassa. Evameva kho, bhikkhave, citte asaṃkiliṭṭhe, sugati pāṭikaṅkhā.

\paragraph{3.} ‘‘Katame ca, bhikkhave, cittassa upakkilesā? Abhijjhāvisamalobho cittassa upakkileso, byāpādo cittassa upakkileso, kodho cittassa upakkileso, upanāho cittassa upakkileso, makkho cittassa upakkileso, paḷāso cittassa upakkileso, issā cittassa upakkileso, macchariyaṃ cittassa upakkileso, māyā cittassa upakkileso, sāṭheyyaṃ cittassa upakkileso, thambho cittassa upakkileso, sārambho cittassa upakkileso, māno cittassa upakkileso, atimāno cittassa upakkileso, mado cittassa upakkileso, pamādo cittassa upakkileso.

\paragraph{4.} ‘‘Sa kho so, bhikkhave, bhikkhu ‘abhijjhāvisamalobho cittassa upakkileso’ti – iti viditvā abhijjhāvisamalobhaṃ cittassa upakkilesaṃ pajahati; ‘byāpādo cittassa upakkileso’ti – iti viditvā byāpādaṃ cittassa upakkilesaṃ pajahati ; ‘kodho cittassa upakkileso’ti – iti viditvā kodhaṃ cittassa upakkilesaṃ pajahati; ‘upanāho cittassa upakkileso’ti – iti viditvā upanāhaṃ cittassa upakkilesaṃ pajahati; ‘makkho cittassa upakkileso’ti – iti viditvā makkhaṃ cittassa upakkilesaṃ pajahati; ‘paḷāso cittassa upakkileso’ti – iti viditvā paḷāsaṃ cittassa upakkilesaṃ pajahati; ‘issā cittassa upakkileso’ti – iti viditvā issaṃ cittassa upakkilesaṃ pajahati; ‘macchariyaṃ cittassa upakkileso’ti – iti viditvā macchariyaṃ cittassa upakkilesaṃ pajahati; ‘māyā cittassa upakkileso’ti – iti viditvā māyaṃ cittassa upakkilesaṃ pajahati; ‘sāṭheyyaṃ cittassa upakkileso’ti – iti viditvā sāṭheyyaṃ cittassa upakkilesaṃ pajahati; ‘thambho cittassa upakkileso’ti – iti viditvā thambhaṃ cittassa upakkilesaṃ pajahati; ‘sārambho cittassa upakkileso’ti – iti viditvā sārambhaṃ cittassa upakkilesaṃ pajahati; ‘māno cittassa upakkileso’ti – iti viditvā mānaṃ cittassa upakkilesaṃ pajahati; ‘atimāno cittassa upakkileso’ti – iti viditvā atimānaṃ cittassa upakkilesaṃ pajahati; ‘mado cittassa upakkileso’ti – iti viditvā madaṃ cittassa upakkilesaṃ pajahati; ‘pamādo cittassa upakkileso’ti – iti viditvā pamādaṃ cittassa upakkilesaṃ pajahati.

\paragraph{5.} ‘‘Yato kho\footnote{yato ca kho (sī. syā.)}, bhikkhave, bhikkhuno ‘abhijjhāvisamalobho cittassa upakkileso’ti – iti viditvā abhijjhāvisamalobho cittassa upakkileso pahīno hoti, ‘byāpādo cittassa upakkileso’ti – iti viditvā byāpādo cittassa upakkileso pahīno hoti; ‘kodho cittassa upakkileso’ti – iti viditvā kodho cittassa upakkileso pahīno hoti; ‘upanāho cittassa upakkileso’ti – iti viditvā upanāho cittassa upakkileso pahīno hoti; ‘makkho cittassa upakkileso’ti – iti viditvā makkho cittassa upakkileso pahīno hoti; ‘paḷāso cittassa upakkileso’ti – iti viditvā paḷāso cittassa upakkileso pahīno hoti; ‘issā cittassa upakkileso’ti – iti viditvā issā cittassa upakkileso pahīno hoti; ‘macchariyaṃ cittassa upakkileso’ti – iti viditvā macchariyaṃ cittassa upakkileso pahīno hoti; ‘māyā cittassa upakkileso’ti – iti viditvā māyā cittassa upakkileso pahīno hoti; ‘sāṭheyyaṃ cittassa upakkileso’ti – iti viditvā sāṭheyyaṃ cittassa upakkileso pahīno hoti; ‘thambho cittassa upakkileso’ti – iti viditvā thambho cittassa upakkileso pahīno hoti; ‘sārambho cittassa upakkileso’ti – iti viditvā sārambho cittassa upakkileso pahīno hoti; ‘māno cittassa upakkileso’ti – iti viditvā māno cittassa upakkileso pahīno hoti; ‘atimāno cittassa upakkileso’ti – iti viditvā atimāno cittassa upakkileso pahīno hoti; ‘mado cittassa upakkileso’ti – iti viditvā mado cittassa upakkileso pahīno hoti; ‘pamādo cittassa upakkileso’ti – iti viditvā pamādo cittassa upakkileso pahīno hoti.

\paragraph{6.} ‘‘So buddhe aveccappasādena samannāgato hoti – ‘itipi so bhagavā arahaṃ sammāsambuddho vijjācaraṇasampanno sugato lokavidū anuttaro purisadammasārathi satthā devamanussānaṃ buddho bhagavā’ti; dhamme aveccappasādena samannāgato hoti – ‘svākkhāto bhagavatā dhammo sandiṭṭhiko akāliko ehipassiko opaneyyiko paccattaṃ veditabbo viññūhī’ti; saṅghe aveccappasādena samannāgato hoti – ‘suppaṭipanno bhagavato sāvakasaṅgho, ujuppaṭipanno bhagavato sāvakasaṅgho, ñāyappaṭipanno bhagavato sāvakasaṅgho, sāmīcippaṭipanno bhagavato sāvakasaṅgho, yadidaṃ cattāri purisayugāni, aṭṭha purisapuggalā. Esa bhagavato sāvakasaṅgho āhuneyyo pāhuneyyo dakkhiṇeyyo añjalikaraṇīyo , anuttaraṃ puññakkhettaṃ lokassā’ti.

\paragraph{7.} ‘‘Yathodhi\footnote{yatodhi (aṭṭhakathāyaṃ pāṭhantaraṃ)} kho panassa cattaṃ hoti vantaṃ muttaṃ pahīnaṃ paṭinissaṭṭhaṃ, so ‘buddhe aveccappasādena samannāgatomhī’ti labhati atthavedaṃ, labhati dhammavedaṃ, labhati dhammūpasaṃhitaṃ pāmojjaṃ. Pamuditassa pīti jāyati, pītimanassa kāyo passambhati, passaddhakāyo sukhaṃ vedeti, sukhino cittaṃ samādhiyati; ‘dhamme…pe… saṅghe aveccappasādena samannāgatomhī’ti labhati atthavedaṃ, labhati dhammavedaṃ, labhati dhammūpasaṃhitaṃ pāmojjaṃ; pamuditassa pīti jāyati, pītimanassa kāyo passambhati, passaddhakāyo sukhaṃ vedeti, sukhino cittaṃ samādhiyati. ‘Yathodhi kho pana me cattaṃ vantaṃ muttaṃ pahīnaṃ paṭinissaṭṭha’nti labhati atthavedaṃ, labhati dhammavedaṃ, labhati dhammūpasaṃhitaṃ pāmojjaṃ; pamuditassa pīti jāyati, pītimanassa kāyo passambhati, passaddhakāyo sukhaṃ vedeti, sukhino cittaṃ samādhiyati.

\paragraph{8.} ‘‘Sa kho so, bhikkhave, bhikkhu evaṃsīlo evaṃdhammo evaṃpañño sālīnaṃ cepi piṇḍapātaṃ bhuñjati vicitakāḷakaṃ anekasūpaṃ anekabyañjanaṃ, nevassa taṃ hoti antarāyāya. Seyyathāpi, bhikkhave, vatthaṃ saṃkiliṭṭhaṃ malaggahitaṃ acchodakaṃ āgamma parisuddhaṃ hoti pariyodātaṃ , ukkāmukhaṃ vā panāgamma jātarūpaṃ parisuddhaṃ hoti pariyodātaṃ, evameva kho, bhikkhave, bhikkhu evaṃsīlo evaṃdhammo evaṃpañño sālīnaṃ cepi piṇḍapātaṃ bhuñjati vicitakāḷakaṃ anekasūpaṃ anekabyañjanaṃ , nevassa taṃ hoti antarāyāya.

\paragraph{9.} ‘‘So mettāsahagatena cetasā ekaṃ disaṃ pharitvā viharati, tathā dutiyaṃ, tathā tatiyaṃ, tathā catutthaṃ\footnote{catutthiṃ (sī. pī.)}. Iti uddhamadho tiriyaṃ sabbadhi sabbattatāya sabbāvantaṃ lokaṃ mettāsahagatena cetasā vipulena mahaggatena appamāṇena averena abyāpajjena pharitvā viharati; karuṇāsahagatena cetasā…pe… muditāsahagatena cetasā…pe… upekkhāsahagatena cetasā ekaṃ disaṃ pharitvā viharati, tathā dutiyaṃ, tathā tatiyaṃ, tathā catutthaṃ. Iti uddhamadho tiriyaṃ sabbadhi sabbattatāya sabbāvantaṃ lokaṃ upekkhāsahagatena cetasā vipulena mahaggatena appamāṇena averena abyāpajjena pharitvā viharati.

\paragraph{10.} ‘‘So ‘atthi idaṃ, atthi hīnaṃ, atthi paṇītaṃ, atthi imassa saññāgatassa uttariṃ nissaraṇa’nti pajānāti. Tassa evaṃ jānato evaṃ passato kāmāsavāpi cittaṃ vimuccati, bhavāsavāpi cittaṃ vimuccati, avijjāsavāpi cittaṃ vimuccati. Vimuttasmiṃ vimuttamiti ñāṇaṃ hoti. ‘Khīṇā jāti, vusitaṃ brahmacariyaṃ, kataṃ karaṇīyaṃ, nāparaṃ itthattāyā’ti pajānāti . Ayaṃ vuccati, bhikkhave – ‘bhikkhu sināto antarena sinānenā’’’ti.

\paragraph{11.} Tena kho pana samayena sundarikabhāradvājo brāhmaṇo bhagavato avidūre nisinno hoti. Atha kho sundarikabhāradvājo brāhmaṇo bhagavantaṃ etadavoca – ‘‘gacchati pana bhavaṃ gotamo bāhukaṃ nadiṃ sināyitu’’nti? ‘‘Kiṃ, brāhmaṇa, bāhukāya nadiyā? Kiṃ bāhukā nadī karissatī’’ti? ‘‘Lokkhasammatā\footnote{lokhyasammatā (sī.), mokkhasammatā (pī.)} hi, bho gotama, bāhukā nadī bahujanassa, puññasammatā hi, bho gotama, bāhukā nadī bahujanassa, bāhukāya pana nadiyā bahujano pāpakammaṃ kataṃ pavāhetī’’ti. Atha kho bhagavā sundarikabhāradvājaṃ brāhmaṇaṃ gāthāhi ajjhabhāsi –

\paragraph{12.}\begin{verse}
  ‘‘Bāhukaṃ adhikakkañca, \\gayaṃ sundarikaṃ mapi\footnote{sundarikāmapi (sī. syā. pī.), sundarikaṃ mahiṃ (itipi)};\\
  Sarassatiṃ payāgañca, \\atho bāhumatiṃ nadiṃ;\\
  Niccampi bālo pakkhando\footnote{pakkhanno (sī. syā. pī.)}, \\kaṇhakammo na sujjhati.
\end{verse}

\paragraph{13.}\begin{verse}
  ‘‘Kiṃ sundarikā karissati, \\kiṃ payāgā\footnote{payāgo (sī. syā. pī.)} kiṃ bāhukā nadī;\\
  Veriṃ katakibbisaṃ naraṃ, \\na hi naṃ sodhaye pāpakamminaṃ.
\end{verse}

\paragraph{14.}\begin{verse}
  ‘‘Suddhassa ve sadā phaggu, \\suddhassuposatho sadā;\\
  Suddhassa sucikammassa, \\sadā sampajjate vataṃ;\\
  Idheva sināhi brāhmaṇa, \\sabbabhūtesu karohi khemataṃ.
\end{verse}

\paragraph{15.}\begin{verse}
  ‘‘Sace musā na bhaṇasi, \\sace pāṇaṃ na hiṃsasi;\\
  Sace adinnaṃ nādiyasi, \\saddahāno amaccharī;\\
  Kiṃ kāhasi gayaṃ gantvā, \\udapānopi te gayā’’ti.
\end{verse}

\paragraph{16.} Evaṃ vutte, sundarikabhāradvājo brāhmaṇo bhagavantaṃ etadavoca – ‘‘abhikkantaṃ, bho gotama, abhikkantaṃ, bho gotama! Seyyathāpi, bho gotama, nikkujjitaṃ vā ukkujjeyya, paṭicchannaṃ vā vivareyya, mūḷhassa vā maggaṃ ācikkheyya, andhakāre vā telapajjotaṃ dhāreyya – cakkhumanto rūpāni dakkhantīti; evamevaṃ bhotā gotamena anekapariyāyena dhammo pakāsito. Esāhaṃ bhavantaṃ gotamaṃ saraṇaṃ gacchāmi dhammañca bhikkhusaṅghañca. Labheyyāhaṃ bhoto gotamassa santike pabbajjaṃ, labheyyaṃ upasampada’’nti. Alattha kho sundarikabhāradvājo brāhmaṇo bhagavato santike pabbajjaṃ, alattha upasampadaṃ. Acirūpasampanno kho panāyasmā bhāradvājo eko vūpakaṭṭho appamatto ātāpī pahitatto viharanto nacirasseva – yassatthāya kulaputtā sammadeva agārasmā anagāriyaṃ pabbajanti tadanuttaraṃ – brahmacariyapariyosānaṃ diṭṭhevadhamme sayaṃ abhiññā sacchikatvā upasampajja vihāsi. ‘‘Khīṇā jāti, vusitaṃ brahmacariyaṃ, kataṃ karaṇīyaṃ, nāparaṃ itthattāyā’’ti abbhaññāsi. Aññataro kho panāyasmā bhāradvājo arahataṃ ahosīti.

\xsectionEnd{Vatthasuttaṃ niṭṭhitaṃ sattamaṃ.}
