\section{Sammādiṭṭhisuttaṃ}

\paragraph{1.} Evaṃ me sutaṃ – ekaṃ samayaṃ bhagavā sāvatthiyaṃ viharati jetavane anāthapiṇḍikassa ārāme. Tatra kho āyasmā sāriputto bhikkhū āmantesi – ‘‘āvuso bhikkhave’’ti. ‘‘Āvuso’’ti kho te bhikkhū āyasmato sāriputtassa paccassosuṃ. Āyasmā sāriputto etadavoca –

\paragraph{2.} ‘‘‘Sammādiṭṭhi\footnote{sammādiṭṭhī (sī. syā.)} sammādiṭṭhī’ti, āvuso, vuccati. Kittāvatā nu kho, āvuso, ariyasāvako sammādiṭṭhi hoti, ujugatāssa diṭṭhi, dhamme aveccappasādena samannāgato, āgato imaṃ saddhamma’’nti?

\paragraph{3.} ‘‘Dūratopi kho mayaṃ, āvuso, āgaccheyyāma āyasmato sāriputtassa santike etassa bhāsitassa atthamaññātuṃ. Sādhu vatāyasmantaṃyeva sāriputtaṃ paṭibhātu etassa bhāsitassa attho. Āyasmato sāriputtassa sutvā bhikkhū dhāressantī’’ti. ‘‘Tena hi, āvuso, suṇātha, sādhukaṃ manasi karotha, bhāsissāmī’’ti. ‘‘Evamāvuso’’ti kho te bhikkhū āyasmato sāriputtassa paccassosuṃ. Āyasmā sāriputto etadavoca –

\paragraph{4.} ‘‘Yato kho, āvuso, ariyasāvako akusalañca pajānāti, akusalamūlañca pajānāti, kusalañca pajānāti, kusalamūlañca pajānāti – ettāvatāpi kho, āvuso, ariyasāvako sammādiṭṭhi hoti, ujugatāssa diṭṭhi, dhamme aveccappasādena samannāgato, āgato imaṃ saddhammaṃ. Katamaṃ panāvuso, akusalaṃ, katamaṃ akusalamūlaṃ, katamaṃ kusalaṃ, katamaṃ kusalamūlaṃ? Pāṇātipāto kho, āvuso, akusalaṃ, adinnādānaṃ akusalaṃ, kāmesumicchācāro akusalaṃ, musāvādo akusalaṃ, pisuṇā vācā\footnote{pisuṇavācā (ka.)} akusalaṃ, pharusā vācā\footnote{pharusavācā (ka.)} akusalaṃ, samphappalāpo akusalaṃ, abhijjhā akusalaṃ, byāpādo akusalaṃ, micchādiṭṭhi akusalaṃ – idaṃ vuccatāvuso akusalaṃ. Katamañcāvuso, akusalamūlaṃ? Lobho akusalamūlaṃ, doso akusalamūlaṃ, moho akusalamūlaṃ – idaṃ vuccatāvuso, akusalamūlaṃ.

\paragraph{5.} ‘‘Katamañcāvuso , kusalaṃ? Pāṇātipātā veramaṇī kusalaṃ, adinnādānā veramaṇī kusalaṃ, kāmesumicchācārā veramaṇī kusalaṃ, musāvādā veramaṇī kusalaṃ, pisuṇāya vācāya veramaṇī kusalaṃ, pharusāya vācāya veramaṇī kusalaṃ, samphappalāpā veramaṇī kusalaṃ, anabhijjhā kusalaṃ, abyāpādo kusalaṃ, sammādiṭṭhi kusalaṃ – idaṃ vuccatāvuso, kusalaṃ. Katamañcāvuso, kusalamūlaṃ? Alobho kusalamūlaṃ, adoso kusalamūlaṃ, amoho kusalamūlaṃ – idaṃ vuccatāvuso, kusalamūlaṃ.

\paragraph{6.} ‘‘Yato kho, āvuso, ariyasāvako evaṃ akusalaṃ pajānāti, evaṃ akusalamūlaṃ pajānāti, evaṃ kusalaṃ pajānāti, evaṃ kusalamūlaṃ pajānāti, so sabbaso rāgānusayaṃ pahāya, paṭighānusayaṃ paṭivinodetvā, ‘asmī’ti diṭṭhimānānusayaṃ samūhanitvā, avijjaṃ pahāya vijjaṃ uppādetvā, diṭṭhevadhamme dukkhassantakaro hoti – ettāvatāpi kho, āvuso, ariyasāvako sammādiṭṭhi hoti, ujugatāssa diṭṭhi, dhamme aveccappasādena samannāgato, āgato imaṃ saddhamma’’nti.

\paragraph{7.} ‘‘Sādhāvuso’’ti kho te bhikkhū āyasmato sāriputtassa bhāsitaṃ abhinanditvā anumoditvā āyasmantaṃ sāriputtaṃ uttari\footnote{uttariṃ (sī. syā. pī.)} pañhaṃ apucchuṃ\footnote{apucchiṃsu (syā.)} – ‘‘siyā panāvuso, aññopi pariyāyo yathā ariyasāvako sammādiṭṭhi hoti, ujugatāssa diṭṭhi, dhamme aveccappasādena samannāgato, āgato imaṃ saddhamma’’nti?

\paragraph{8.} ‘‘Siyā, āvuso. Yato kho, āvuso, ariyasāvako āhārañca pajānāti, āhārasamudayañca pajānāti, āhāranirodhañca pajānāti, āhāranirodhagāminiṃ paṭipadañca pajānāti – ettāvatāpi kho, āvuso, ariyasāvako sammādiṭṭhi hoti, ujugatāssa diṭṭhi, dhamme aveccappasādena samannāgato, āgato imaṃ saddhammaṃ. Katamo panāvuso, āhāro, katamo āhārasamudayo, katamo āhāranirodho, katamā āhāranirodhagāminī paṭipadā? Cattārome, āvuso, āhārā bhūtānaṃ vā sattānaṃ ṭhitiyā, sambhavesīnaṃ vā anuggahāya. Katame cattāro? Kabaḷīkāro āhāro oḷāriko vā sukhumo vā, phasso dutiyo, manosañcetanā tatiyā, viññāṇaṃ catutthaṃ. Taṇhāsamudayā āhārasamudayo, taṇhānirodhā āhāranirodho, ayameva ariyo aṭṭhaṅgiko maggo āhāranirodhagāminī paṭipadā, seyyathidaṃ – sammādiṭṭhi sammāsaṅkappo sammāvācā sammākammanto , sammāājīvo sammāvāyāmo sammāsati sammāsamādhi’.

\paragraph{9.} ‘‘Yato kho, āvuso, ariyasāvako evaṃ āhāraṃ pajānāti, evaṃ āhārasamudayaṃ pajānāti, evaṃ āhāranirodhaṃ pajānāti, evaṃ āhāranirodhagāminiṃ paṭipadaṃ pajānāti, so sabbaso rāgānusayaṃ pahāya, paṭighānusayaṃ paṭivinodetvā, ‘asmī’ti diṭṭhimānānusayaṃ samūhanitvā, avijjaṃ pahāya vijjaṃ uppādetvā, diṭṭhevadhamme dukkhassantakaro hoti – ettāvatāpi kho, āvuso, ariyasāvako sammādiṭṭhi hoti, ujugatāssa diṭṭhi, dhamme aveccappasādena samannāgato, āgato imaṃ saddhamma’’nti.

\paragraph{10.} ‘‘Sādhāvuso’’ti kho te bhikkhū āyasmato sāriputtassa bhāsitaṃ abhinanditvā anumoditvā āyasmantaṃ sāriputtaṃ uttari pañhaṃ apucchuṃ – ‘‘siyā panāvuso, aññopi pariyāyo yathā ariyasāvako sammādiṭṭhi hoti, ujugatāssa diṭṭhi, dhamme aveccappasādena samannāgato, āgato imaṃ saddhamma’’nti?

\paragraph{11.} ‘‘Siyā, āvuso. Yato kho, āvuso, ariyasāvako dukkhañca pajānāti, dukkhasamudayañca pajānāti, dukkhanirodhañca pajānāti, dukkhanirodhagāminiṃ paṭipadañca pajānāti – ettāvatāpi kho, āvuso, ariyasāvako sammādiṭṭhi hoti, ujugatāssa diṭṭhi, dhamme aveccappasādena samannāgato, āgato imaṃ saddhammaṃ. Katamaṃ panāvuso, dukkhaṃ, katamo dukkhasamudayo, katamo dukkhanirodho, katamā dukkhanirodhagāminī paṭipadā? Jātipi dukkhā, jarāpi dukkhā, maraṇampi dukkhaṃ, sokaparidevadukkhadomanassupāyāsāpi dukkhā, appiyehi sampayogopi dukkho, piyehi vippayogopi dukkho, yampicchaṃ na labhati tampi dukkhaṃ, saṃkhittena pañcupādānakkhandhā\footnote{pañcupādānakkhandhāpi (ka.)} dukkhā – idaṃ vuccatāvuso, dukkhaṃ. Katamo cāvuso, dukkhasamudayo? Yāyaṃ taṇhā ponobbhavikā nandīrāgasahagatā\footnote{ponobhavikā (sī. pī.)} tatratatrābhinandinī\footnote{nandirāgasahagatā (sī. pī.)}, seyyathidaṃ, kāmataṇhā bhavataṇhā vibhavataṇhā – ayaṃ vuccatāvuso, dukkhasamudayo. Katamo cāvuso, dukkhanirodho? Yo tassāyeva taṇhāya asesavirāganirodho cāgo paṭinissaggo mutti anālayo – ayaṃ vuccatāvuso, dukkhanirodho. Katamā cāvuso, dukkhanirodhagāminī paṭipadā? Ayameva ariyo aṭṭhaṅgiko maggo, seyyathidaṃ, sammādiṭṭhi…pe… sammāsamādhi – ayaṃ vuccatāvuso, dukkhanirodhagāminī paṭipadā.

\paragraph{12.} ‘‘Yato kho, āvuso, ariyasāvako evaṃ dukkhaṃ pajānāti, evaṃ dukkhasamudayaṃ pajānāti, evaṃ dukkhanirodhaṃ pajānāti, evaṃ dukkhanirodhagāminiṃ paṭipadaṃ pajānāti, so sabbaso rāgānusayaṃ pahāya, paṭighānusayaṃ paṭivinodetvā, ‘asmī’ti diṭṭhimānānusayaṃ samūhanitvā, avijjaṃ pahāya vijjaṃ uppādetvā, diṭṭhevadhamme dukkhassantakaro hoti – ettāvatāpi kho, āvuso, ariyasāvako sammādiṭṭhi hoti, ujugatāssa diṭṭhi, dhamme aveccappasādena samannāgato, āgato imaṃ saddhamma’’nti.

\paragraph{13.} ‘‘Sādhāvuso’’ti kho te bhikkhū āyasmato sāriputtassa bhāsitaṃ abhinanditvā anumoditvā āyasmantaṃ sāriputtaṃ uttari pañhaṃ apucchuṃ – ‘‘siyā panāvuso, aññopi pariyāyo yathā ariyasāvako sammādiṭṭhi hoti, ujugatāssa diṭṭhi, dhamme aveccappasādena samannāgato, āgato imaṃ saddhamma’’nti?

\paragraph{14.} ‘‘Siyā, āvuso. Yato kho, āvuso, ariyasāvako jarāmaraṇañca pajānāti, jarāmaraṇasamudayañca pajānāti, jarāmaraṇanirodhañca pajānāti, jarāmaraṇanirodhagāminiṃ paṭipadañca pajānāti – ettāvatāpi kho, āvuso, ariyasāvako sammādiṭṭhi hoti, ujugatāssa diṭṭhi, dhamme aveccappasādena samannāgato, āgato imaṃ saddhammaṃ. Katamaṃ panāvuso, jarāmaraṇaṃ, katamo jarāmaraṇasamudayo, katamo jarāmaraṇanirodho, katamā jarāmaraṇanirodhagāminī paṭipadā? Yā tesaṃ tesaṃ sattānaṃ tamhi tamhi sattanikāye jarā jīraṇatā khaṇḍiccaṃ pāliccaṃ valittacatā āyuno saṃhāni indriyānaṃ paripāko – ayaṃ vuccatāvuso, jarā. Katamañcāvuso, maraṇaṃ? Yā\footnote{yaṃ (pī. ka.), satipaṭṭhānasuttepi} tesaṃ tesaṃ sattānaṃ tamhā tamhā sattanikāyā cuti cavanatā bhedo antaradhānaṃ maccu maraṇaṃ kālaṃkiriyā khandhānaṃ bhedo, kaḷevarassa nikkhepo, jīvitindriyassupacchedo – idaṃ vuccatāvuso, maraṇaṃ. Iti ayañca jarā idañca maraṇaṃ – idaṃ vuccatāvuso, jarāmaraṇaṃ. Jātisamudayā jarāmaraṇasamudayo, jātinirodhā jarāmaraṇanirodho, ayameva ariyo aṭṭhaṅgiko maggo jarāmaraṇanirodhagāminī paṭipadā, seyyathidaṃ – sammādiṭṭhi…pe… sammāsamādhi.

\paragraph{15.} ‘‘Yato kho, āvuso, ariyasāvako evaṃ jarāmaraṇaṃ pajānāti, evaṃ jarāmaraṇasamudaṃ pajānāti, evaṃ jarāmaraṇanirodhaṃ pajānāti, evaṃ jarāmaraṇanirodhagāminiṃ paṭipadaṃ pajānāti, so sabbaso rāgānusayaṃ pahāya…pe… dukkhassantakaro hoti – ettāvatāpi kho, āvuso, ariyasāvako sammādiṭṭhi hoti, ujugatāssa diṭṭhi, dhamme aveccappasādena samannāgato, āgato imaṃ saddhamma’’nti.

\paragraph{16.} ‘‘Sādhāvuso’’ti kho…pe… apucchuṃ – siyā panāvuso…pe… ‘‘siyā, āvuso. Yato kho, āvuso, ariyasāvako jātiñca pajānāti, jātisamudayañca pajānāti, jātinirodhañca pajānāti, jātinirodhagāminiṃ paṭipadañca pajānāti – ettāvatāpi kho, āvuso, ariyasāvako sammādiṭṭhi hoti, ujugatāssa diṭṭhi, dhamme aveccappasādena samannāgato, āgato imaṃ saddhammaṃ. Katamā panāvuso, jāti, katamo jātisamudayo, katamo jātinirodho, katamā jātinirodhagāminī paṭipadā? Yā tesaṃ tesaṃ sattānaṃ tamhi tamhi sattanikāye jāti sañjāti okkanti abhinibbatti khandhānaṃ pātubhāvo, āyatanānaṃ paṭilābho – ayaṃ vuccatāvuso, jāti. Bhavasamudayā jātisamudayo, bhavanirodhā jātinirodho, ayameva ariyo aṭṭhaṅgiko maggo jātinirodhagāminī paṭipadā, seyyathidaṃ – sammādiṭṭhi…pe… sammāsamādhi.

\paragraph{17.} ‘‘Yato kho, āvuso, ariyasāvako evaṃ jātiṃ pajānāti, evaṃ jātisamudayaṃ pajānāti, evaṃ jātinirodhaṃ pajānāti, evaṃ jātinirodhagāminiṃ paṭipadaṃ pajānāti, so sabbaso rāgānusayaṃ pahāya…pe… dukkhassantakaro hoti – ettāvatāpi kho, āvuso, ariyasāvako sammādiṭṭhi hoti, ujugatāssa diṭṭhi, dhamme aveccappasādena samannāgato, āgato imaṃ saddhamma’’nti.

\paragraph{18.} ‘‘Sādhāvuso’’ti kho…pe… apucchuṃ – siyā panāvuso…pe… ‘‘siyā, āvuso. Yato kho, āvuso, ariyasāvako bhavañca pajānāti, bhavasamudayañca pajānāti, bhavanirodhañca pajānāti, bhavanirodhagāminiṃ paṭipadañca pajānāti – ettāvatāpi kho, āvuso, ariyasāvako sammādiṭṭhi hoti, ujugatāssa diṭṭhi, dhamme aveccappasādena samannāgato, āgato imaṃ saddhammaṃ. Katamo panāvuso, bhavo, katamo bhavasamudayo, katamo bhavanirodho, katamā bhavanirodhagāminī paṭipadā? Tayome, āvuso, bhavā – kāmabhavo, rūpabhavo, arūpabhavo. Upādānasamudayā bhavasamudayo, upādānanirodhā bhavanirodho, ayameva ariyo aṭṭhaṅgiko maggo bhavanirodhagāminī paṭipadā, seyyathidaṃ – sammādiṭṭhi…pe… sammāsamādhi.

\paragraph{19.} ‘‘Yato kho, āvuso, ariyasāvako evaṃ bhavaṃ pajānāti, evaṃ bhavasamudayaṃ pajānāti, evaṃ bhavanirodhaṃ pajānāti, evaṃ bhavanirodhagāminiṃ paṭipadaṃ pajānāti, so sabbaso rāgānusayaṃ pahāya…pe… dukkhassantakaro hoti. Ettāvatāpi kho, āvuso, ariyasāvako sammādiṭṭhi hoti, ujugatāssa diṭṭhi, dhamme aveccappasādena samannāgato, āgato imaṃ saddhamma’’nti.

\paragraph{20.} ‘‘Sādhāvuso’’ti kho…pe… apucchuṃ – siyā panāvuso…pe… ‘‘siyā, āvuso. Yato kho, āvuso, ariyasāvako upādānañca pajānāti, upādānasamudayañca pajānāti, upādānanirodhañca pajānāti, upādānanirodhagāminiṃ paṭipadañca pajānāti – ettāvatāpi kho, āvuso, ariyasāvako sammādiṭṭhi hoti, ujugatāssa diṭṭhi, dhamme aveccappasādena samannāgato, āgato imaṃ saddhammaṃ. Katamaṃ panāvuso, upādānaṃ, katamo upādānasamudayo, katamo upādānanirodho, katamā upādānanirodhagāminī paṭipadā? Cattārimāni, āvuso, upādānāni – kāmupādānaṃ, diṭṭhupādānaṃ, sīlabbatupādānaṃ, attavādupādānaṃ. Taṇhāsamudayā upādānasamudayo, taṇhānirodhā upādānanirodho, ayameva ariyo aṭṭhaṅgiko maggo upādānanirodhagāminī paṭipadā, seyyathidaṃ – sammādiṭṭhi…pe… sammāsamādhi.

\paragraph{21.} ‘‘Yato kho, āvuso, ariyasāvako evaṃ upādānaṃ pajānāti, evaṃ upādānasamudayaṃ pajānāti, evaṃ upādānanirodhaṃ pajānāti, evaṃ upādānanirodhagāminiṃ paṭipadaṃ pajānāti, so sabbaso rāgānusayaṃ pahāya…pe… dukkhassantakaro hoti – ettāvatāpi kho, āvuso, ariyasāvako sammādiṭṭhi hoti, ujugatāssa diṭṭhi, dhamme aveccappasādena samannāgato, āgato imaṃ saddhamma’’nti.

\paragraph{22.} ‘‘Sādhāvuso’’ti kho…pe… apucchuṃ – siyā panāvuso…pe… ‘‘siyā, āvuso. Yato kho, āvuso, ariyasāvako taṇhañca pajānāti, taṇhāsamudayañca pajānāti, taṇhānirodhañca pajānāti, taṇhānirodhagāminiṃ paṭipadañca pajānāti – ettāvatāpi kho, āvuso, ariyasāvako sammādiṭṭhi hoti, ujugatāssa diṭṭhi, dhamme aveccappasādena samannāgato, āgato imaṃ saddhammaṃ. Katamā panāvuso, taṇhā, katamo taṇhāsamudayo, katamo taṇhānirodho, katamā taṇhānirodhagāminī paṭipadā? Chayime, āvuso, taṇhākāyā – rūpataṇhā, saddataṇhā, gandhataṇhā, rasataṇhā, phoṭṭhabbataṇhā, dhammataṇhā. Vedanāsamudayā taṇhāsamudayo, vedanānirodhā taṇhānirodho, ayameva ariyo aṭṭhaṅgiko maggo taṇhānirodhagāminī paṭipadā, seyyathidaṃ – sammādiṭṭhi…pe… sammāsamādhi.

\paragraph{23.} ‘‘Yato kho, āvuso, ariyasāvako evaṃ taṇhaṃ pajānāti, evaṃ taṇhāsamudayaṃ pajānāti, evaṃ taṇhānirodhaṃ pajānāti, evaṃ taṇhānirodhagāminiṃ paṭipadaṃ pajānāti, so sabbaso rāgānusayaṃ pahāya…pe… dukkhassantakaro hoti – ettāvatāpi kho, āvuso, ariyasāvako sammādiṭṭhi hoti, ujugatāssa diṭṭhi, dhamme aveccappasādena samannāgato, āgato imaṃ saddhamma’’nti.

\paragraph{24.} ‘‘Sādhāvuso’’ti kho…pe… apucchuṃ – siyā panāvuso…pe… ‘‘siyā, āvuso. Yato kho, āvuso, ariyasāvako vedanañca pajānāti, vedanāsamudayañca pajānāti, vedanānirodhañca pajānāti, vedanānirodhagāminiṃ paṭipadañca pajānāti – ettāvatāpi kho, āvuso, ariyasāvako sammādiṭṭhi hoti, ujugatāssa diṭṭhi, dhamme aveccappasādena samannāgato, āgato imaṃ saddhammaṃ. Katamā panāvuso, vedanā, katamo vedanāsamudayo, katamo vedanānirodho, katamā vedanānirodhagāminī paṭipadā? Chayime, āvuso, vedanākāyā – cakkhusamphassajā vedanā, sotasamphassajā vedanā, ghānasamphassajā vedanā, jivhāsamphassajā vedanā, kāyasamphassajā vedanā, manosamphassajā vedanā. Phassasamudayā vedanāsamudayo, phassanirodhā vedanānirodho, ayameva ariyo aṭṭhaṅgiko maggo vedanānirodhagāminī paṭipadā, seyyathidaṃ – sammādiṭṭhi…pe… sammāsamādhi.

\paragraph{25.} ‘‘Yato kho, āvuso, ariyasāvako evaṃ vedanaṃ pajānāti, evaṃ vedanāsamudayaṃ pajānāti, evaṃ vedanānirodhaṃ pajānāti, evaṃ vedanānirodhagāminiṃ paṭipadaṃ pajānāti, so sabbaso rāgānusayaṃ pahāya…pe… dukkhassantakaro hoti – ettāvatāpi kho, āvuso, ariyasāvako sammādiṭṭhi hoti, ujugatāssa diṭṭhi, dhamme aveccappasādena samannāgato, āgato imaṃ saddhamma’’nti.

\paragraph{26.} ‘‘Sādhāvuso’’ti kho…pe… apucchuṃ – siyā panāvuso…pe… ‘‘siyā, āvuso. Yato kho, āvuso, ariyasāvako phassañca pajānāti, phassasamudayañca pajānāti, phassanirodhañca pajānāti, phassanirodhagāminiṃ paṭipadañca pajānāti – ettāvatāpi kho, āvuso, ariyasāvako sammādiṭṭhi hoti, ujugatāssa diṭṭhi, dhamme aveccappasādena samannāgato, āgato imaṃ saddhammaṃ. Katamo panāvuso, phasso, katamo phassasamudayo, katamo phassanirodho, katamā phassanirodhagāminī paṭipadā? Chayime, āvuso, phassakāyā – cakkhusamphasso, sotasamphasso , ghānasamphasso, jivhāsamphasso, kāyasamphasso, manosamphasso. Saḷāyatanasamudayā phassasamudayo, saḷāyatananirodhā phassanirodho, ayameva ariyo aṭṭhaṅgiko maggo phassanirodhagāminī paṭipadā, seyyathidaṃ – sammādiṭṭhi…pe… sammāsamādhi.

\paragraph{27.} ‘‘Yato kho, āvuso, ariyasāvako evaṃ phassaṃ pajānāti, evaṃ phassasamudayaṃ pajānāti, evaṃ phassanirodhaṃ pajānāti, evaṃ phassanirodhagāminiṃ paṭipadaṃ pajānāti, so sabbaso rāgānusayaṃ pahāya…pe… dukkhassantakaro hoti – ettāvatāpi kho, āvuso, ariyasāvako sammādiṭṭhi hoti, ujugatāssa diṭṭhi, dhamme aveccappasādena samannāgato, āgato imaṃ saddhamma’’nti.

\paragraph{28.} ‘‘Sādhāvuso’’ti kho…pe… apucchuṃ – siyā panāvuso…pe… ‘‘siyā, āvuso. Yato kho, āvuso, ariyasāvako saḷāyatanañca pajānāti, saḷāyatanasamudayañca pajānāti, saḷāyatananirodhañca pajānāti, saḷāyatananirodhagāminiṃ paṭipadañca pajānāti – ettāvatāpi kho, āvuso, ariyasāvako sammādiṭṭhi hoti, ujugatāssa diṭṭhi, dhamme aveccappasādena samannāgato, āgato imaṃ saddhammaṃ. Katamaṃ panāvuso, saḷāyatanaṃ, katamo saḷāyatanasamudayo, katamo saḷāyatananirodho, katamā saḷāyatananirodhagāminī paṭipadā ? Chayimāni, āvuso, āyatanāni – cakkhāyatanaṃ, sotāyatanaṃ, ghānāyatanaṃ, jivhāyatanaṃ, kāyāyatanaṃ, manāyatanaṃ. Nāmarūpasamudayā saḷāyatanasamudayo, nāmarūpanirodhā saḷāyatananirodho, ayameva ariyo aṭṭhaṅgiko maggo saḷāyatananirodhagāminī paṭipadā, seyyathidaṃ – sammādiṭṭhi…pe… sammāsamādhi.

\paragraph{29.} ‘‘Yato kho, āvuso, ariyasāvako evaṃ saḷāyatanaṃ pajānāti, evaṃ saḷāyatanasamudayaṃ pajānāti, evaṃ saḷāyatananirodhaṃ pajānāti , evaṃ saḷāyatananirodhagāminiṃ paṭipadaṃ pajānāti, so sabbaso rāgānusayaṃ pahāya…pe… dukkhassantakaro hoti – ettāvatāpi kho, āvuso, ariyasāvako sammādiṭṭhi hoti, ujugatāssa diṭṭhi, dhamme aveccappasādena samannāgato, āgato imaṃ saddhamma’’nti.

\paragraph{30.} ‘‘Sādhāvuso’’ti kho…pe… apucchuṃ – siyā panāvuso…pe… ‘‘siyā, āvuso. Yato kho, āvuso, ariyasāvako nāmarūpañca pajānāti, nāmarūpasamudayañca pajānāti, nāmarūpanirodhañca pajānāti, nāmarūpanirodhagāminiṃ paṭipadañca pajānāti – ettāvatāpi kho, āvuso, ariyasāvako sammādiṭṭhi hoti, ujugatāssa diṭṭhi, dhamme aveccappasādena samannāgato, āgato imaṃ saddhammaṃ. Katamaṃ panāvuso, nāmarūpaṃ, katamo nāmarūpasamudayo, katamo nāmarūpanirodho, katamā nāmarūpanirodhagāminī paṭipadā? Vedanā, saññā, cetanā, phasso, manasikāro – idaṃ vuccatāvuso, nāmaṃ; cattāri ca mahābhūtāni, catunnañca mahābhūtānaṃ upādāyarūpaṃ – idaṃ vuccatāvuso, rūpaṃ. Iti idañca nāmaṃ idañca rūpaṃ – idaṃ vuccatāvuso, nāmarūpaṃ. Viññāṇasamudayā nāmarūpasamudayo, viññāṇanirodhā nāmarūpanirodho, ayameva ariyo aṭṭhaṅgiko maggo nāmarūpanirodhagāminī paṭipadā, seyyathidaṃ – sammādiṭṭhi…pe… sammāsamādhi.

\paragraph{31.} ‘‘Yato kho, āvuso, ariyasāvako evaṃ nāmarūpaṃ pajānāti, evaṃ nāmarūpasamudayaṃ pajānāti, evaṃ nāmarūpanirodhaṃ pajānāti, evaṃ nāmarūpanirodhagāminiṃ paṭipadaṃ pajānāti, so sabbaso rāgānusayaṃ pahāya…pe… dukkhassantakaro hoti – ettāvatāpi kho, āvuso, ariyasāvako sammādiṭṭhi hoti, ujugatāssa diṭṭhi, dhamme aveccappasādena samannāgato, āgato imaṃ saddhamma’’nti.

\paragraph{32.} ‘‘Sādhāvuso’’ti kho…pe… apucchuṃ – siyā panāvuso…pe… ‘‘siyā, āvuso. Yato kho, āvuso, ariyasāvako viññāṇañca pajānāti, viññāṇasamudayañca pajānāti, viññāṇanirodhañca pajānāti, viññāṇanirodhagāminiṃ paṭipadañca pajānāti – ettāvatāpi kho, āvuso, ariyasāvako sammādiṭṭhi hoti, ujugatāssa diṭṭhi, dhamme aveccappasādena samannāgato, āgato imaṃ saddhammaṃ. Katamaṃ panāvuso, viññāṇaṃ, katamo viññāṇasamudayo, katamo viññāṇanirodho, katamā viññāṇanirodhagāminī paṭipadā? Chayime, āvuso, viññāṇakāyā – cakkhuviññāṇaṃ, sotaviññāṇaṃ, ghānaviññāṇaṃ, jivhāviññāṇaṃ, kāyaviññāṇaṃ, manoviññāṇaṃ. Saṅkhārasamudayā viññāṇasamudayo, saṅkhāranirodhā viññāṇanirodho, ayameva ariyo aṭṭhaṅgiko maggo viññāṇanirodhagāminī paṭipadā, seyyathidaṃ – sammādiṭṭhi…pe… sammāsamādhi.

\paragraph{33.} ‘‘Yato kho, āvuso, ariyasāvako evaṃ viññāṇaṃ pajānāti, evaṃ viññāṇasamudayaṃ pajānāti, evaṃ viññāṇanirodhaṃ pajānāti, evaṃ viññāṇanirodhagāminiṃ paṭipadaṃ pajānāti , so sabbaso rāgānusayaṃ pahāya…pe… dukkhassantakaro hoti – ettāvatāpi kho, āvuso, ariyasāvako sammādiṭṭhi hoti, ujugatāssa diṭṭhi, dhamme aveccappasādena samannāgato, āgato imaṃ saddhamma’’nti.

\paragraph{34.} ‘‘Sādhāvuso’’ti kho…pe… apucchuṃ – siyā panāvuso…pe… ‘‘siyā, āvuso. Yato kho, āvuso, ariyasāvako saṅkhāre ca pajānāti, saṅkhārasamudayañca pajānāti, saṅkhāranirodhañca pajānāti, saṅkhāranirodhagāminiṃ paṭipadañca pajānāti – ettāvatāpi kho, āvuso, ariyasāvako sammādiṭṭhi hoti, ujugatāssa diṭṭhi, dhamme aveccappasādena samannāgato, āgato imaṃ saddhammaṃ. Katame panāvuso, saṅkhārā, katamo saṅkhārasamudayo, katamo saṅkhāranirodho, katamā saṅkhāranirodhagāminī paṭipadā? Tayome, āvuso, saṅkhārā – kāyasaṅkhāro, vacīsaṅkhāro, cittasaṅkhāro. Avijjāsamudayā saṅkhārasamudayo, avijjānirodhā saṅkhāranirodho, ayameva ariyo aṭṭhaṅgiko maggo saṅkhāranirodhagāminī paṭipadā, seyyathidaṃ – sammādiṭṭhi…pe… sammāsamādhi.

\paragraph{35.} ‘‘Yato kho, āvuso, ariyasāvako evaṃ saṅkhāre pajānāti, evaṃ saṅkhārasamudayaṃ pajānāti, evaṃ saṅkhāranirodhaṃ pajānāti, evaṃ saṅkhāranirodhagāminiṃ paṭipadaṃ pajānāti, so sabbaso rāgānusayaṃ pahāya, paṭighānusayaṃ paṭivinodetvā, ‘asmī’ti diṭṭhimānānusayaṃ samūhanitvā, avijjaṃ pahāya vijjaṃ uppādetvā, diṭṭheva dhamme dukkhassantakaro hoti – ettāvatāpi kho, āvuso, ariyasāvako sammādiṭṭhi hoti, ujugatāssa diṭṭhi, dhamme aveccappasādena samannāgato, āgato imaṃ saddhamma’’nti.

\paragraph{36.} ‘‘Sādhāvuso’’ti kho…pe… apucchuṃ – siyā panāvuso…pe… ‘‘siyā, āvuso. Yato kho, āvuso, ariyasāvako avijjañca pajānāti, avijjāsamudayañca pajānāti, avijjānirodhañca pajānāti, avijjānirodhagāminiṃ paṭipadañca pajānāti – ettāvatāpi kho, āvuso, ariyasāvako sammādiṭṭhi hoti, ujugatāssa diṭṭhi, dhamme aveccappasādena samannāgato, āgato imaṃ saddhammaṃ. Katamā panāvuso, avijjā, katamo avijjāsamudayo, katamo avijjānirodho, katamā avijjānirodhagāminī paṭipadā? Yaṃ kho, āvuso, dukkhe aññāṇaṃ, dukkhasamudaye aññāṇaṃ, dukkhanirodhe aññāṇaṃ, dukkhanirodhagāminiyā paṭipadāya aññāṇaṃ – ayaṃ vuccatāvuso, avijjā. Āsavasamudayā avijjāsamudayo, āsavanirodhā avijjānirodho, ayameva ariyo aṭṭhaṅgiko maggo avijjānirodhagāminī paṭipadā, seyyathidaṃ – sammādiṭṭhi…pe… sammāsamādhi.

\paragraph{37.} ‘‘Yato kho, āvuso, ariyasāvako evaṃ avijjaṃ pajānāti, evaṃ avijjāsamudayaṃ pajānāti, evaṃ avijjānirodhaṃ pajānāti, evaṃ avijjānirodhagāminiṃ paṭipadaṃ pajānāti, so sabbaso rāgānusayaṃ pahāya, paṭighānusayaṃ paṭivinodetvā, ‘asmī’ti diṭṭhimānānusayaṃ samūhanitvā, avijjaṃ pahāya vijjaṃ uppādetvā, diṭṭheva dhamme dukkhassantakaro hoti – ettāvatāpi kho, āvuso, ariyasāvako sammādiṭṭhi hoti, ujugatāssa diṭṭhi, dhamme aveccappasādena samannāgato, āgato imaṃ saddhamma’’nti.

\paragraph{38.} ‘‘Sādhāvuso’’ti kho te bhikkhū āyasmato sāriputtassa bhāsitaṃ abhinanditvā anumoditvā āyasmantaṃ sāriputtaṃ uttari pañhaṃ apucchuṃ – ‘‘siyā panāvuso, aññopi pariyāyo yathā ariyasāvako sammādiṭṭhi hoti, ujugatāssa diṭṭhi, dhamme aveccappasādena samannāgato, āgato imaṃ saddhamma’’nti?

\paragraph{39.} ‘‘Siyā, āvuso. Yato kho, āvuso, ariyasāvako āsavañca pajānāti, āsavasamudayañca pajānāti, āsavanirodhañca pajānāti, āsavanirodhagāminiṃ paṭipadañca pajānāti – ettāvatāpi kho, āvuso, ariyasāvako sammādiṭṭhi hoti, ujugatāssa diṭṭhi, dhamme aveccappasādena samannāgato, āgato imaṃ saddhammaṃ. Katamo panāvuso, āsavo, katamo āsavasamudayo, katamo āsavanirodho, katamā āsavanirodhagāminī paṭipadāti? Tayome, āvuso, āsavā – kāmāsavo, bhavāsavo, avijjāsavo. Avijjāsamudayā āsavasamudayo, avijjānirodhā āsavanirodho, ayameva ariyo aṭṭhaṅgiko maggo āsavanirodhagāminī paṭipadā, seyyathidaṃ – sammādiṭṭhi…pe… sammāsamādhi.

\paragraph{40.} ‘‘Yato kho, āvuso, ariyasāvako evaṃ āsavaṃ pajānāti, evaṃ āsavasamudayaṃ pajānāti, evaṃ āsavanirodhaṃ pajānāti, evaṃ āsavanirodhagāminiṃ paṭipadaṃ pajānāti, so sabbaso rāgānusayaṃ pahāya, paṭighānusayaṃ paṭivinodetvā, ‘asmī’ti diṭṭhimānānusayaṃ samūhanitvā, avijjaṃ pahāya vijjaṃ uppādetvā, diṭṭheva dhamme dukkhassantakaro hoti – ettāvatāpi kho, āvuso, ariyasāvako sammādiṭṭhi hoti , ujugatāssa diṭṭhi, dhamme aveccappasādena samannāgato, āgato imaṃ saddhamma’’nti.

\paragraph{41.} Idamavocāyasmā sāriputto. Attamanā te bhikkhū āyasmato sāriputtassa bhāsitaṃ abhinandunti.

\xsectionEnd{Sammādiṭṭhisuttaṃ niṭṭhitaṃ navamaṃ\footnote{ito paraṃ kesuci potthakesu imāpi gāthāyo§evaṃ dissanti –§dukkhaṃ jarāmaraṇaṃ upādānaṃ, saḷāyatanaṃ nāmarūpaṃ. viññāṇaṃ yā sā pare, katamā panāvuso padānaṃ§kiṃ jāti taṇhā ca vedanā, avijjāya catukkanayo. cattāri pare katamā, panāvuso padānaṃ kevalaṃ§āhāro ca bhavo phasso, saṅkhāro āsavapañcamo. yāva pañca pare katamā, panāvuso padānaṃ kiṃ§katamanti chabbidhā vuttaṃ, katamāni catubbidhāni. katamo pañcavidho vutto, sabbesaṃ ekasaṅkhānaṃ pañcanayapadāni cāti }.}
