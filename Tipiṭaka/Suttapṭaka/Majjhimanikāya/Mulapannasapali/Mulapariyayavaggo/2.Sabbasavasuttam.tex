\section{Sabbāsavasuttaṃ}

\paragraph{1.} Evaṃ me sutaṃ – ekaṃ samayaṃ bhagavā sāvatthiyaṃ viharati jetavane anāthapiṇḍikassa ārāme. Tatra kho bhagavā bhikkhū āmantesi – ‘‘bhikkhavo’’ti. ‘‘Bhadante’’ti te bhikkhū bhagavato paccassosuṃ. Bhagavā etadavoca – ‘‘sabbāsavasaṃvarapariyāyaṃ vo, bhikkhave, desessāmi. Taṃ suṇātha , sādhukaṃ manasi karotha, bhāsissāmī’’ti. ‘‘Evaṃ, bhante’’ti kho te bhikkhū bhagavato paccassosuṃ. Bhagavā etadavoca –

\paragraph{2.} ‘‘Jānato ahaṃ, bhikkhave, passato āsavānaṃ khayaṃ vadāmi, no ajānato no apassato. Kiñca, bhikkhave, jānato kiñca passato āsavānaṃ khayaṃ vadāmi? Yoniso ca manasikāraṃ ayoniso ca manasikāraṃ. Ayoniso, bhikkhave, manasikaroto anuppannā ceva āsavā uppajjanti, uppannā ca āsavā pavaḍḍhanti; yoniso ca kho, bhikkhave, manasikaroto anuppannā ceva āsavā na uppajjanti, uppannā ca āsavā pahīyanti.

\paragraph{3.} ‘‘Atthi, bhikkhave, āsavā dassanā pahātabbā, atthi āsavā saṃvarā pahātabbā, atthi āsavā paṭisevanā pahātabbā, atthi āsavā adhivāsanā pahātabbā, atthi āsavā parivajjanā pahātabbā, atthi āsavā vinodanā pahātabbā, atthi āsavā bhāvanā pahātabbā.

\subsubsection{Dassanā pahātabbāsavā}

\paragraph{4.} ‘‘Katame ca, bhikkhave, āsavā dassanā pahātabbā? Idha, bhikkhave , assutavā puthujjano – ariyānaṃ adassāvī ariyadhammassa akovido ariyadhamme avinīto, sappurisānaṃ adassāvī sappurisadhammassa akovido sappurisadhamme avinīto – manasikaraṇīye dhamme nappajānāti, amanasikaraṇīye dhamme nappajānāti. So manasikaraṇīye dhamme appajānanto amanasikaraṇīye dhamme appajānanto, ye dhammā na manasikaraṇīyā, te dhamme manasi karoti, ye dhammā manasikaraṇīyā te dhamme na manasi karoti.

\paragraph{5.} ‘‘Katame ca, bhikkhave, dhammā na manasikaraṇīyā ye dhamme manasi karoti? Yassa, bhikkhave, dhamme manasikaroto anuppanno vā kāmāsavo uppajjati, uppanno vā kāmāsavo pavaḍḍhati; anuppanno vā bhavāsavo uppajjati, uppanno vā bhavāsavo pavaḍḍhati; anuppanno vā avijjāsavo uppajjati, uppanno vā avijjāsavo pavaḍḍhati – ime dhammā na manasikaraṇīyā ye dhamme manasi karoti.

\paragraph{6.} ‘‘Katame ca, bhikkhave, dhammā manasikaraṇīyā ye dhamme na manasi karoti? Yassa, bhikkhave, dhamme manasikaroto anuppanno vā kāmāsavo na uppajjati, uppanno vā kāmāsavo pahīyati; anuppanno vā bhavāsavo na uppajjati, uppanno vā bhavāsavo pahīyati; anuppanno vā avijjāsavo na uppajjati, uppanno vā avijjāsavo pahīyati – ime dhammā manasikaraṇīyā ye dhamme na manasi karoti.

\paragraph{7.} ‘‘Tassa amanasikaraṇīyānaṃ dhammānaṃ manasikārā manasikaraṇīyānaṃ dhammānaṃ amanasikārā anuppannā ceva āsavā uppajjanti uppannā ca āsavā pavaḍḍhanti.

\paragraph{8.} ‘‘So evaṃ ayoniso manasi karoti – ‘ahosiṃ nu kho ahaṃ atītamaddhānaṃ? Na nu kho ahosiṃ atītamaddhānaṃ? Kiṃ nu kho ahosiṃ atītamaddhānaṃ? Kathaṃ nu kho ahosiṃ atītamaddhānaṃ? Kiṃ hutvā kiṃ ahosiṃ nu kho ahaṃ atītamaddhānaṃ? Bhavissāmi nu kho ahaṃ anāgatamaddhānaṃ? Na nu kho bhavissāmi anāgatamaddhānaṃ? Kiṃ nu kho bhavissāmi anāgatamaddhānaṃ? Kathaṃ nu kho bhavissāmi anāgatamaddhānaṃ? Kiṃ hutvā kiṃ bhavissāmi nu kho ahaṃ anāgatamaddhāna’nti? Etarahi vā paccuppannamaddhānaṃ\footnote{paccuppannamaddhānaṃ ārabbha (syā.)} ajjhattaṃ kathaṃkathī hoti – ‘ahaṃ nu khosmi? No nu khosmi? Kiṃ nu khosmi? Kathaṃ nu khosmi? Ayaṃ nu kho satto kuto āgato? So kuhiṃ gāmī bhavissatī’ti?

\paragraph{9.} ‘‘Tassa evaṃ ayoniso manasikaroto channaṃ diṭṭhīnaṃ aññatarā diṭṭhi uppajjati. ‘Atthi me attā’ti vā assa\footnote{vāssa (sī. syā. pī.)} saccato thetato diṭṭhi uppajjati; ‘natthi me attā’ti vā assa saccato thetato diṭṭhi uppajjati; ‘attanāva attānaṃ sañjānāmī’ti vā assa saccato thetato diṭṭhi uppajjati; ‘attanāva anattānaṃ sañjānāmī’ti vā assa saccato thetato diṭṭhi uppajjati; ‘anattanāva attānaṃ sañjānāmī’ti vā assa saccato thetato diṭṭhi uppajjati; atha vā panassa evaṃ diṭṭhi hoti – ‘yo me ayaṃ attā vado vedeyyo tatra tatra kalyāṇapāpakānaṃ kammānaṃ vipākaṃ paṭisaṃvedeti so kho pana me ayaṃ attā nicco dhuvo sassato avipariṇāmadhammo sassatisamaṃ tatheva ṭhassatī’ti. Idaṃ vuccati, bhikkhave , diṭṭhigataṃ diṭṭhigahanaṃ diṭṭhikantāraṃ diṭṭhivisūkaṃ diṭṭhivipphanditaṃ diṭṭhisaṃyojanaṃ. Diṭṭhisaṃyojanasaṃyutto, bhikkhave, assutavā puthujjano na parimuccati jātiyā jarāya maraṇena sokehi paridevehi dukkhehi domanassehi upāyāsehi; ‘na parimuccati dukkhasmā’ti vadāmi.

\paragraph{10.} ‘‘Sutavā ca kho, bhikkhave, ariyasāvako – ariyānaṃ dassāvī ariyadhammassa kovido ariyadhamme suvinīto, sappurisānaṃ dassāvī sappurisadhammassa kovido sappurisadhamme suvinīto – manasikaraṇīye dhamme pajānāti amanasikaraṇīye dhamme pajānāti. So manasikaraṇīye dhamme pajānanto amanasikaraṇīye dhamme pajānanto ye dhammā na manasikaraṇīyā te dhamme na manasi karoti, ye dhammā manasikaraṇīyā te dhamme manasi karoti.

\paragraph{11.} ‘‘Katame ca, bhikkhave, dhammā na manasikaraṇīyā ye dhamme na manasi karoti? Yassa, bhikkhave, dhamme manasikaroto anuppanno vā kāmāsavo uppajjati, uppanno vā kāmāsavo pavaḍḍhati; anuppanno vā bhavāsavo uppajjati, uppanno vā bhavāsavo pavaḍḍhati; anuppanno vā avijjāsavo uppajjati, uppanno vā avijjāsavo pavaḍḍhati – ime dhammā na manasikaraṇīyā, ye dhamme na manasi karoti.

\paragraph{12.} ‘‘Katame ca, bhikkhave, dhammā manasikaraṇīyā ye dhamme manasi karoti? Yassa, bhikkhave, dhamme manasikaroto anuppanno vā kāmāsavo na uppajjati, uppanno vā kāmāsavo pahīyati; anuppanno vā bhavāsavo na uppajjati , uppanno vā bhavāsavo pahīyati; anuppanno vā avijjāsavo na uppajjati, uppanno vā avijjāsavo pahīyati – ime dhammā manasikaraṇīyā ye dhamme manasi karoti.

\paragraph{13.} ‘‘Tassa amanasikaraṇīyānaṃ dhammānaṃ amanasikārā manasikaraṇīyānaṃ dhammānaṃ manasikārā anuppannā ceva āsavā na uppajjanti, uppannā ca āsavā pahīyanti.

\paragraph{14.} ‘‘So ‘idaṃ dukkha’nti yoniso manasi karoti, ‘ayaṃ dukkhasamudayo’ti yoniso manasi karoti, ‘ayaṃ dukkhanirodho’ti yoniso manasi karoti, ‘ayaṃ dukkhanirodhagāminī paṭipadā’ti yoniso manasi karoti. Tassa evaṃ yoniso manasikaroto tīṇi saṃyojanāni pahīyanti – sakkāyadiṭṭhi, vicikicchā, sīlabbataparāmāso. Ime vuccanti, bhikkhave, āsavā dassanā pahātabbā.

\subsubsection{Saṃvarā pahātabbāsavā}

\paragraph{15.} ‘‘Katame ca, bhikkhave, āsavā saṃvarā pahātabbā? Idha, bhikkhave, bhikkhu paṭisaṅkhā yoniso cakkhundriyasaṃvarasaṃvuto viharati. Yañhissa, bhikkhave, cakkhundriyasaṃvaraṃ asaṃvutassa viharato uppajjeyyuṃ āsavā vighātapariḷāhā, cakkhundriyasaṃvaraṃ saṃvutassa viharato evaṃsa te āsavā vighātapariḷāhā na honti. Paṭisaṅkhā yoniso sotindriyasaṃvarasaṃvuto viharati…pe… ghānindriyasaṃvarasaṃvuto viharati…pe… jivhindriyasaṃvarasaṃvuto viharati…pe… kāyindriyasaṃvarasaṃvuto viharati…pe… manindriyasaṃvarasaṃvuto viharati. Yañhissa, bhikkhave , manindriyasaṃvaraṃ asaṃvutassa viharato uppajjeyyuṃ āsavā vighātapariḷāhā, manindriyasaṃvaraṃ saṃvutassa viharato evaṃsa te āsavā vighātapariḷāhā na honti.

\paragraph{16.} ‘‘Yañhissa, bhikkhave, saṃvaraṃ asaṃvutassa viharato uppajjeyyuṃ āsavā vighātapariḷāhā , saṃvaraṃ saṃvutassa viharato evaṃsa te āsavā vighātapariḷāhā na honti. Ime vuccanti, bhikkhave, āsavā saṃvarā pahātabbā.

\subsubsection{Paṭisevanā pahātabbāsavā}

\paragraph{17.} ‘‘Katame ca, bhikkhave, āsavā paṭisevanā pahātabbā? Idha, bhikkhave, bhikkhu paṭisaṅkhā yoniso cīvaraṃ paṭisevati – ‘yāvadeva sītassa paṭighātāya, uṇhassa paṭighātāya, ḍaṃsamakasavātātapasarīṃsapasamphassānaṃ\footnote{…siriṃsapa… (sī. syā. pī.)} paṭighātāya, yāvadeva hirikopīnappaṭicchādanatthaṃ’.

\paragraph{18.} ‘‘Paṭisaṅkhā yoniso piṇḍapātaṃ paṭisevati – ‘neva davāya, na madāya, na maṇḍanāya, na vibhūsanāya, yāvadeva imassa kāyassa ṭhitiyā yāpanāya, vihiṃsūparatiyā, brahmacariyānuggahāya, iti purāṇañca vedanaṃ paṭihaṅkhāmi navañca vedanaṃ na uppādessāmi, yātrā ca me bhavissati anavajjatā ca phāsuvihāro ca\footnote{cāti (sī.)}’.

\paragraph{19.} ‘‘Paṭisaṅkhā yoniso senāsanaṃ paṭisevati – ‘yāvadeva sītassa paṭighātāya, uṇhassa paṭighātāya, ḍaṃsamakasavātātapasarīṃsapasamphassānaṃ paṭighātāya, yāvadeva utuparissayavinodanapaṭisallānārāmatthaṃ’.

\paragraph{20.} ‘‘Paṭisaṅkhā yoniso gilānappaccayabhesajjaparikkhāraṃ paṭisevati – ‘yāvadeva uppannānaṃ veyyābādhikānaṃ vedanānaṃ paṭighātāya, abyābajjhaparamatāya\footnote{abyāpajjhaparamatāya (sī. syā. pī.), abyāpajjaparamatāya (ka.)}’.

\paragraph{21.} ‘‘Yañhissa, bhikkhave, appaṭisevato uppajjeyyuṃ āsavā vighātapariḷāhā, paṭisevato evaṃsa te āsavā vighātapariḷāhā na honti. Ime vuccanti, bhikkhave, āsavā paṭisevanā pahātabbā.

\subsubsection{Adhivāsanā pahātabbāsavā}

\paragraph{22.} ‘‘Katame ca, bhikkhave, āsavā adhivāsanā pahātabbā? Idha, bhikkhave, bhikkhu paṭisaṅkhā yoniso khamo hoti sītassa uṇhassa, jighacchāya pipāsāya. Ḍaṃsamakasavātātapasarīṃsapasamphassānaṃ, duruttānaṃ durāgatānaṃ vacanapathānaṃ, uppannānaṃ sārīrikānaṃ vedanānaṃ dukkhānaṃ tibbānaṃ\footnote{tippānaṃ (sī. syā. pī.)} kharānaṃ kaṭukānaṃ asātānaṃ amanāpānaṃ pāṇaharānaṃ adhivāsakajātiko hoti.

\paragraph{23.} ‘‘Yañhissa, bhikkhave, anadhivāsayato uppajjeyyuṃ āsavā vighātapariḷāhā, adhivāsayato evaṃsa te āsavā vighātapariḷāhā na honti. Ime vuccanti, bhikkhave, āsavā adhivāsanā pahātabbā.

\subsubsection{Parivajjanā pahātabbāsavā}

\paragraph{24.} ‘‘Katame ca, bhikkhave, āsavā parivajjanā pahātabbā? Idha, bhikkhave, bhikkhu paṭisaṅkhā yoniso caṇḍaṃ hatthiṃ parivajjeti, caṇḍaṃ assaṃ parivajjeti, caṇḍaṃ goṇaṃ parivajjeti, caṇḍaṃ kukkuraṃ parivajjeti, ahiṃ khāṇuṃ kaṇṭakaṭṭhānaṃ sobbhaṃ papātaṃ candanikaṃ oḷigallaṃ. Yathārūpe anāsane nisinnaṃ yathārūpe agocare carantaṃ yathārūpe pāpake mitte bhajantaṃ viññū sabrahmacārī pāpakesu ṭhānesu okappeyyuṃ, so tañca anāsanaṃ tañca agocaraṃ te ca pāpake mitte paṭisaṅkhā yoniso parivajjeti.

\paragraph{25.} ‘‘Yañhissa, bhikkhave, aparivajjayato uppajjeyyuṃ āsavā vighātapariḷāhā, parivajjayato evaṃsa te āsavā vighātapariḷāhā na honti. Ime vuccanti, bhikkhave, āsavā parivajjanā pahātabbā.

\subsubsection{Vinodanā pahātabbāsavā}

\paragraph{26.} ‘‘Katame ca, bhikkhave, āsavā vinodanā pahātabbā? Idha, bhikkhave, bhikkhu paṭisaṅkhā yoniso uppannaṃ kāmavitakkaṃ nādhivāseti pajahati vinodeti byantīkaroti anabhāvaṃ gameti, uppannaṃ byāpādavitakkaṃ…pe… uppannaṃ vihiṃsāvitakkaṃ…pe… uppannuppanne pāpake akusale dhamme nādhivāseti pajahati vinodeti byantīkaroti anabhāvaṃ gameti.

\paragraph{27.} ‘‘Yañhissa, bhikkhave, avinodayato uppajjeyyuṃ āsavā vighātapariḷāhā, vinodayato evaṃsa te āsavā vighātapariḷāhā na honti. Ime vuccanti, bhikkhave, āsavā vinodanā pahātabbā.

\subsubsection{Bhāvanā pahātabbāsavā}

\paragraph{28.} ‘‘Katame ca, bhikkhave, āsavā bhāvanā pahātabbā? Idha, bhikkhave, bhikkhu paṭisaṅkhā yoniso satisambojjhaṅgaṃ bhāveti vivekanissitaṃ virāganissitaṃ nirodhanissitaṃ vossaggapariṇāmiṃ; paṭisaṅkhā yoniso dhammavicayasambojjhaṅgaṃ bhāveti…pe… vīriyasambojjhaṅgaṃ bhāveti… pītisambojjhaṅgaṃ bhāveti… passaddhisambojjhaṅgaṃ bhāveti… samādhisambojjhaṅgaṃ bhāveti… upekkhāsambojjhaṅgaṃ bhāveti vivekanissitaṃ virāganissitaṃ nirodhanissitaṃ vossaggapariṇāmiṃ.

\paragraph{29.} ‘‘Yañhissa, bhikkhave , abhāvayato uppajjeyyuṃ āsavā vighātapariḷāhā, bhāvayato evaṃsa te āsavā vighātapariḷāhā na honti. Ime vuccanti, bhikkhave, āsavā bhāvanā pahātabbā.

\paragraph{30.} ‘‘Yato kho, bhikkhave, bhikkhuno ye āsavā dassanā pahātabbā te dassanā pahīnā honti, ye āsavā saṃvarā pahātabbā te saṃvarā pahīnā honti, ye āsavā paṭisevanā pahātabbā te paṭisevanā pahīnā honti, ye āsavā adhivāsanā pahātabbā te adhivāsanā pahīnā honti, ye āsavā parivajjanā pahātabbā te parivajjanā pahīnā honti, ye āsavā vinodanā pahātabbā te vinodanā pahīnā honti, ye āsavā bhāvanā pahātabbā te bhāvanā pahīnā honti; ayaṃ vuccati, bhikkhave – ‘bhikkhu sabbāsavasaṃvarasaṃvuto viharati, acchecchi\footnote{acchejji (ka.)} taṇhaṃ, vivattayi\footnote{vāvattayi (sī. pī.)} saṃyojanaṃ, sammā mānābhisamayā antamakāsi dukkhassā’’’ti.

\paragraph{31.} Idamavoca bhagavā. Attamanā te bhikkhū bhagavato bhāsitaṃ abhinandunti.

\xsectionEnd{Sabbāsavasuttaṃ niṭṭhitaṃ dutiyaṃ.}
